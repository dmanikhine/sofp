%% LyX 2.3.6.2 created this file.  For more info, see http://www.lyx.org/.
%% Do not edit unless you really know what you are doing.
\documentclass[10pt,russian,english,open=any,numbers=noenddot,index=totoc,bibliography=totoc,listof=totoc,fontsize=10pt]{scrbook}
\usepackage{amsmath}
\usepackage{mathpazo}
\usepackage{helvet}
\renewcommand{\ttdefault}{cmtt}
\usepackage{newtxmath}
\usepackage[T2A,T1]{fontenc}
\usepackage[utf8]{inputenc}
\usepackage[paperwidth=7.444in,paperheight=9.68in]{geometry}
\geometry{verbose,tmargin=1.175cm,bmargin=1.275cm,lmargin=2.2cm,rmargin=1.3cm,headsep=0.4cm,footskip=0.72cm}
\setcounter{secnumdepth}{3}
\usepackage{xcolor}
\usepackage{babel}
\usepackage{array}
\usepackage{verbatim}
\usepackage{refstyle}
\usepackage{wrapfig}
\usepackage{calc}
\usepackage{textcomp}
\usepackage{tipa}
\usepackage{tipx}
\usepackage{mathdots}
\usepackage{stmaryrd}
\usepackage{makeidx}
\makeindex
\usepackage{graphicx}
\usepackage{wasysym}
\usepackage[all]{xy}
\PassOptionsToPackage{normalem}{ulem}
\usepackage{ulem}
\usepackage[unicode=true,
 bookmarks=true,bookmarksnumbered=true,bookmarksopen=true,bookmarksopenlevel=2,
 breaklinks=true,pdfborder={0 0 0},pdfborderstyle={},backref=page,colorlinks=true]
 {hyperref}
\hypersetup{pdftitle={The Science of Functional Programming: A Tutorial, with Examples in Scala},
 pdfauthor={Sergei Winitzki},
 pdfsubject={Functional programming},
 pdfkeywords={Functional programming, Scala, Type theory, Category theory, Formal logic, Programming languages},
 linkcolor=hlink}

\makeatletter

%%%%%%%%%%%%%%%%%%%%%%%%%%%%%% LyX specific LaTeX commands.

\AtBeginDocument{\providecommand\subsecref[1]{\ref{subsec:#1}}}
\newcommand{\noun}[1]{\textsc{#1}}
\DeclareRobustCommand{\cyrtext}{%
  \fontencoding{T2A}\selectfont\def\encodingdefault{T2A}}
\DeclareRobustCommand{\textcyr}[1]{\leavevmode{\cyrtext #1}}

%% Because html converters don't know tabularnewline
\providecommand{\tabularnewline}{\\}
\RS@ifundefined{subsecref}
  {\newref{subsec}{name = \RSsectxt}}
  {}
\RS@ifundefined{thmref}
  {\def\RSthmtxt{theorem~}\newref{thm}{name = \RSthmtxt}}
  {}
\RS@ifundefined{lemref}
  {\def\RSlemtxt{lemma~}\newref{lem}{name = \RSlemtxt}}
  {}


%%%%%%%%%%%%%%%%%%%%%%%%%%%%%% Textclass specific LaTeX commands.
\newenvironment{lyxcode}
	{\par\begin{list}{}{
		\setlength{\rightmargin}{\leftmargin}
		\setlength{\listparindent}{0pt}% needed for AMS classes
		\raggedright
		\setlength{\itemsep}{0pt}
		\setlength{\parsep}{0pt}
		\normalfont\ttfamily}%
	 \item[]}
	{\end{list}}

%%%%%%%%%%%%%%%%%%%%%%%%%%%%%% User specified LaTeX commands.
\usepackage[all]{xy} % xypic

% pstricks with support for pdflatex
%\usepackage{pdftricks}
%\begin{psinputs}
%   \usepackage{pstricks}
%   \usepackage{multido}
%\end{psinputs}
\usepackage{pstricks}

% Cover picture on first page.
\usepackage{wallpaper}
% Custom commands for cover page.
\usepackage[absolute,overlay]{textpos}

% No page numbers on "Part" pages.
\renewcommand*{\partpagestyle}{empty}

% Use a special "equal by definition" symbol.
\renewcommand*{\triangleq}{\overset{\lower1mm\hbox{\texttt{\tiny def}}} {=}}

% Running head: works, but results are not satisfactory.
%\usepackage{scrlayer-scrpage}
%\automark[subsection]{chapter}


% "Better text justification"? Actually, this causes a fatal error "auto expansion is only possible with scalable fonts".
%\usepackage{microtype}

% Fix the numbering of exercises: subsubsections appear as paragraphs but are numbered.
%\usepackage{titlesec} % Incompatible with komascript\textsf{'}s later versions.
% See https://tex.stackexchange.com/questions/7627/how-to-reference-paragraph
% See the `titlesec` package documentation at http://www.ctex.org/documents/packages/layout/titlesec.pdf
%\titleformat{\subsubsection}[runin]{\normalfont\normalsize\bfseries}{}{0pt}{}
%\titlespacing{\subsubsection}{0pt}{5pt}{3\wordsep}
%\titleformat{\subparagraph}[runin]{\normalfont\normalsize\bfseries}{}{0pt}{}
%\titlespacing{\subparagraph}{\parindent}{\parskip}{3\wordsep}
%\titlespacing{\paragraph}{0pt}{3pt}{2\wordsep}

\renewcommand*{\subsubsectionformat}{}
\RedeclareSectionCommand[ % Statement 1.2.3.4
  runin=true,
  afterskip=2ex,
  beforeskip=2.5pt plus 0.3pt minus 0.05pt,
  afterindent=false,
  font={\normalfont\normalsize\bfseries}
]{subsubsection}
\RedeclareSectionCommand[ % Proof
  runin=true,
  font={\normalfont\normalsize\bfseries},
  afterindent=false,
  afterskip=2ex,
  beforeskip=0pt
]{subparagraph}
\RedeclareSectionCommand[
  runin=true,
  font={\normalfont\normalsize\bfseries},
  afterskip=1.3ex,
  beforeskip=0pt
]{paragraph}

% Make page headers and page numbers smaller
\addtokomafont{pagehead}{\small}
\addtokomafont{pagenumber}{\small}

% Double-stroked fonts to replace the non-working \mathbb{1}.
\usepackage{bbold}
\DeclareMathAlphabet{\bbnumcustom}{U}{BOONDOX-ds}{m}{n} % Use BOONDOX-ds or bbold.
\newcommand{\custombb}[1]{\bbnumcustom{#1}}
% The LyX document will define a macro \bbnum{#1} that calls \custombb{#1}.

% Scala syntax highlighting. See https://tex.stackexchange.com/questions/202479/unable-to-define-scala-language-with-listings
%\usepackage[T1]{fontenc}
%\usepackage[utf8]{inputenc}
%\usepackage{beramono}
%\usepackage{listings}
% The listing settings are now supported by LyX in a separate section "Listings".
\usepackage{xcolor}

\definecolor{scalakeyword}{rgb}{0.16,0.07,0.5}
\definecolor{dkgreen}{rgb}{0,0.6,0}
\definecolor{gray}{rgb}{0.5,0.5,0.5}
\definecolor{mauve}{rgb}{0.58,0,0.82}
\definecolor{aqua}{rgb}{0.9,0.96,0.999}
\definecolor{scalatype}{rgb}{0.2,0.3,0.2}
\definecolor{teal}{rgb}{0,0.6,0}

% These settings are now in the Listings tab in LyX.
%\lstdefinestyle{myScalastyle}{
%  language=scala, % This should be defined first!!! Otherwise it overrides all customization via morekeywords / otherkeywords.
%  otherkeywords={{=,=>,<-,<\%,<:,>:,\#,@,*,+,-,/,::,:,[,]}},
%  frame=tb,
%  aboveskip=2mm,
%  belowskip=2mm,
%  showstringspaces=false,
%  columns=flexible,
%  basicstyle={\small\ttfamily},
%  extendedchars=true,
%  %numbers=none,
%  numberstyle=\tiny\color{gray},
%  keywordstyle=\color{blue},
%  commentstyle=\color{dkgreen},
%  stringstyle=\color{mauve},
%  frame=single,
%  framerule=0.01mm,
%  breaklines=true,
%  breakatwhitespace=true,
%  tabsize=3,
%  framexleftmargin=4mm, framexrightmargin=4mm,
%  xleftmargin=4mm, xrightmargin=4mm, % Making these margins the same has a good effect.
%  framextopmargin=0.5mm, framexbottommargin=.5mm,
%  fillcolor=\color{aqua},
%  rulecolor=\color{aqua},
%  rulesepcolor=\color{aqua},
%  backgroundcolor=\color{aqua},
%  mathescape=true,
%}

% Example usage: \begin{lstlisting}[style=myScalastyle]  object blah \end{lstlisting}
%\newenvironment{scala}{\begin{lstlisting}[style=myScalastyle]}{\end{lstlisting}}
%\lstnewenvironment{scala}{\lstset{style=myScalastyle}}{}

\usepackage[nocenter]{qtree} % simple tree drawing
\usepackage{relsize} % make math symbols larger or smaller; supports \smaller etc.
\usepackage{stmaryrd} % some extra symbols such as \fatsemi
% Note: using \forwardcompose inside a \text{} will cause a LaTeX error!
\newcommand{\forwardcompose}{\hspace{1.2pt}\ensuremath\mathsmaller{\fatsemi}\hspace{1.5pt}}
% this is ugly, I used this before I found \fatsemi:
%\newcommand{\bef}{\hspace{1.0pt}\ensuremath\raisebox{2pt}{$\mathsmaller{\mathsmaller{\circ}}$}\hspace{-2.9pt},}
%\makeatletter
% Macros to assist LyX with XYpic when using scaling.
\newcommand{\xyScaleX}[1]{%
\makeatletter
\xydef@\xymatrixcolsep@{#1}
\makeatother
} % end of \xyScaleX
\makeatletter
\newcommand{\xyScaleY}[1]{%
\makeatletter
\xydef@\xymatrixrowsep@{#1}
\makeatother
} % end of \xyScaleY

% Increase the default vertical space inside table cells.
\renewcommand\arraystretch{1.4}

% Color for PDF hyperlinks.
\definecolor{hlink}{rgb}{0.06, 0.14, 0.48}

% Make underline green.
\definecolor{greenunder}{rgb}{0.1,0.6,0.2}
%\newcommand{\munderline}[1]{{\color{greenunder}\underline{{\color{black}#1}}\color{black}}}
\def\mathunderline#1#2{\color{#1}\underline{{\color{black}#2}}\color{black}}
% The LyX document will define a macro \gunderline{#1} that will use \mathunderline with the color `greenunder`.
%\def\gunderline#1{\mathunderline{greenunder}{#1}} % This is now defined by LyX itself with GUI support.

\newcommand{\shui}{\begin{CJK}{UTF8}{gbsn}水\end{CJK}}
\usepackage{CJKutf8} % For occasional Chinese characters. Also, add "russian" to documentclass.

% Prepare settings for imposing a color background for all displayed math. This will be done by a script later.
\usepackage{empheq} % Background on all displayed equations.
\definecolor{mathbg}{rgb}{1.0, .98, .87}
\newcommand*\mymathbgbox[1]{%
\setlength{\fboxsep}{0pt}%
\colorbox{mathbg}{\hspace{0.5mm}#1\hspace{0.5mm}}}
%\renewenvironment{align}{%
%\begin{empheq}[box=\mymathbgbox]{align}}{%
%\endalign\end{empheq}}
% Run a command such as LC_ALL=C sed -i bak -e \textsf{'}s|\\begin{align}|\\begin{empheq}[box=\\mymathbgbox]{align}|; s|\\end{align}|\\end{empheq}|' sofp-filterable.tex
% This is not used now because the results are not great.

% Better text quotes.
\renewcommand\textquotedblleft{\textsf{``}}
\renewcommand\textquotedblright{\textsf{''}}

% Better symbol for the pair mapper instead of \ogreaterthan and \varogreaterthan.
\newcommand{\boxrightarrow}{\mathbin{\ensuremath{%
\mathchoice%
  {\displaystyle{\boxminus}\kern-5.35pt\raisebox{0.75pt}{$\scriptstyle{\succ}$}}%
  {\boxminus\kern-5.35pt\raisebox{0.75pt}{$\scriptstyle{\succ}$}}%
  {\textstyle{\boxminus}\kern-5.35pt\raisebox{0.75pt}{$\scriptstyle{\succ}$}}%
  {\scriptstyle{\boxminus}\kern-3.7pt\raisebox{0.49pt}{$\scriptscriptstyle{\succ}$}}%
}% end of mathchoice with raisebox
\hspace{1.0pt}}}
\renewcommand{\ogreaterthan}{\boxrightarrow}
\renewcommand{\varogreaterthan}{\boxrightarrow}

\makeatother

\usepackage{listings}
\lstset{language=Scala,
morekeywords={{scala}},
otherkeywords={=,=>,<-,<\%,<:,>:,\#,@,:,[,],.,???},
keywordstyle={\color{scalakeyword}},
morekeywords={[2]{String,Short,Int,Long,Char,Boolean,Double,Float,BigDecimal,Seq,Map,Set,Option,Either,Future,Successful,LazyList,Vector,Range,IndexedSeq,true,false,None,List,Nil,Try,Success,Failure,Some,Left,Right,Nothing,Any,Array,Unit,Iterator,Stream,Throwable,Integer,Object}},
keywordstyle={[2]{\color{scalatype}}},
frame=tb,
aboveskip={1.5mm},
belowskip={0.5mm},
showstringspaces=false,
columns=fullflexible,
keepspaces=true,
basicstyle={\smaller\ttfamily},
extendedchars=true,
numbers=none,
numberstyle={\tiny\color{gray}},
commentstyle={\color{dkgreen}},
stringstyle={\color{mauve}},
frame=single,
framerule={0.0mm},
breaklines=true,
breakatwhitespace=true,
tabsize=3,
framexleftmargin={0.5mm},
framexrightmargin={0.5mm},
xleftmargin={1.5mm},
xrightmargin={1.5mm},
framextopmargin={0.5mm},
framexbottommargin={0.5mm},
fillcolor={\color{aqua}},
rulecolor={\color{aqua}},
rulesepcolor={\color{aqua}},
backgroundcolor={\color{aqua}},
mathescape=false,
extendedchars=true}
\addto\captionsenglish{\renewcommand{\lstlistingname}{\inputencoding{latin9}Listing}}
\addto\captionsrussian{\renewcommand{\lstlistingname}{\inputencoding{koi8-r}�������}}
\renewcommand{\lstlistingname}{\inputencoding{latin9}Listing}

\begin{document}
\extratitle{\input{sofp-cover-page}\hfill{}\thispagestyle{empty}The Science
of Functional Programming\hspace{1in}}
\title{The Science of Functional Programming}
\subtitle{A tutorial, with examples in Scala}
\author{by Sergei Winitzki, Ph.D.}
\date{Draft version of \today}
\publishers{Published by\textbf{ \href{http://lulu.com}{lulu.com}} in 2022}
\uppertitleback{Copyright \copyright\  2018-2022 by Sergei Winitzki\\
~\\
Printed copies may be ordered at \texttt{\href{http://www.lulu.com/content/paperback-book/24915714}{http://www.lulu.com/content/paperback-book/24915714}}\\
~\\
ISBN: 978-0-359-76877-6\\
\\
{\scriptsize{}Source hash (sha256): 5d8b4742ceaf99a95662743a261a6724e133b818089e7d0e4b2401550bb68034}\\
{\scriptsize{}Git commit: 2849b6123cc708d01f32e0a93c1aa660140670ec}\\
{\scriptsize{}PDF file built by pdfTeX 3.14159265-2.6-1.40.20 (TeX Live 2019) on Wed, 24 May 2023 19:58:25 +0200 by Darwin 21.6.0}\\
~\\
{\scriptsize{}Permission is granted to copy, distribute and/or modify
this document under the terms of the GNU Free Documentation License,
Version 1.2 or any later version published by the Free Software Foundation;
with no Invariant Sections, no Front-Cover Texts, and no Back-Cover
Texts. A copy of the license is included in the appendix entitled
\textsf{``}GNU Free Documentation License\textsf{''} (Appendix~\ref{sec:GFDL}).}\\
{\scriptsize{}~}\\
{\scriptsize{}A }\emph{\scriptsize{}Transparent}{\scriptsize{} copy
of the source code for the book is available at }\texttt{\scriptsize{}\href{https://github.com/winitzki/sofp}{https://github.com/winitzki/sofp}}{\scriptsize{}
and includes LyX, LaTeX, graphics source files, and build scripts.
A full-color hyperlinked PDF file is available at }\texttt{\scriptsize{}\href{https://github.com/winitzki/sofp/releases}{https://github.com/winitzki/sofp/releases}}{\scriptsize{}
under \textsf{``}Assets\textsf{''} as }\texttt{\scriptsize{}sofp.pdf}{\scriptsize{}
or }\texttt{\scriptsize{}sofp-draft.pdf}{\scriptsize{}. The source
code may be also included as a \textsf{``}file attachment\textsf{''} named }\texttt{\scriptsize{}sofp-src.tar.bz2}{\scriptsize{}
within a PDF file. To extract, run the command }\texttt{\scriptsize{}`pdftk
sofp.pdf unpack\_files output .`}{\scriptsize{} and then }\texttt{\scriptsize{}`tar
jxvf sofp-src.tar.bz2`}{\scriptsize{}. See the file }\texttt{\scriptsize{}README.md}{\scriptsize{}
for compilation instructions.}}
\lowertitleback{{\small{}This book is a pedagogical in-depth tutorial and reference
on the theory of functional programming (FP) as practiced in the early
21$^{\text{st}}$ century. Starting from issues found in practical
coding, the book builds up the theoretical intuition, knowledge, and
techniques that programmers need for rigorous reasoning about types
and code. Examples are given in Scala, but most of the material applies
equally to other FP languages.}\\
{\small{}}\\
{\small{}The book\textsf{'}s topics include working with FP-style collections;
reasoning about recursive functions and types; the Curry-Howard correspondence;
laws, structural analysis, and code for functors, monads, and other
typeclasses built upon exponential-polynomial data types; techniques
of symbolic derivation and proof; free typeclass constructions; and
parametricity theorems.}\\
{\small{}}\\
{\small{}Long and difficult, yet boring explanations are logically
developed in excruciating detail through 1860 Scala
code snippets, 187 statements with step-by-step derivations,
104 diagrams, 216 solved examples with tested
Scala code, and 291 exercises. Discussions further build
upon each chapter\textsf{'}s material.}\\
{\small{}}\\
{\small{}Beginners in FP will find tutorials about the }\inputencoding{latin9}\lstinline!map!\inputencoding{utf8}{\small{}/}\inputencoding{latin9}\lstinline!reduce!\inputencoding{utf8}{\small{}
programming style, type parameters, disjunctive types, and higher-order
functions. For more advanced readers, the book shows  the practical
uses of the Curry-Howard correspondence and the parametricity theorems
without unnecessary jargon; proves that all the standard monads (e.g.,
}\inputencoding{latin9}\lstinline!List!\inputencoding{utf8}{\small{}
or }\inputencoding{latin9}\lstinline!State!\inputencoding{utf8}{\small{})
satisfy the monad laws; derives lawful instances of }\inputencoding{latin9}\lstinline!Functor!\inputencoding{utf8}{\small{}
and other typeclasses from types; shows that monad transformers need
18 laws; and explains the use of parametricity for reasoning about
the Church encoding and the free typeclasses.}\\
{\small{}}\\
{\small{}Readers should have a working knowledge of programming; e.g.,
be able to write code that prints the number of distinct words in
a sentence. The difficulty of this book\textsf{'}s mathematical derivations
is at the level of undergraduate calculus, similar to that of multiplying
matrices or simplifying the expressions:
\[
\frac{1}{x-2}-\frac{1}{x+2}\quad\text{ and }\quad\frac{d}{dx}\left((x+1)f(x)e^{-x}\right)\quad.
\]
}\\
{\small{}Sergei Winitzki received a Ph.D.~in theoretical physics.
After a career in academic research, he currently works as a software
engineer.}}

\maketitle
\frontmatter\pagenumbering{roman}

\tableofcontents{}

\mainmatter\pagenumbering{arabic}
\global\long\def\gunderline#1{\mathunderline{greenunder}{#1}}%
\global\long\def\bef{\forwardcompose}%
\global\long\def\bbnum#1{\custombb{#1}}%
\global\long\def\pplus{{\displaystyle }{+\negmedspace+}}%

\begin{comment}
+ Each chapter is now a separate file. The LyX macro definitions should
be included at the top of each chapter file; then each chapter can
be previewed correctly by LyX, separately from other chapters, which
is faster. It is not necessary (but harmless) to copy the LaTeX preamble
from sofp.lyx to all chapter files.

+ No separate chapter for recursive types unless this is about \textsf{``}open\textsf{''}
co-products and \textsf{``}open\textsf{''} products. Recursive types are already covered
sufficiently. Perhaps this is too much and should be left for a future
edition.

- Write a summary of all results as a separate chapter with no proofs.

+ Cut the scope: no co-monads, no recursive open unions, no lenses
(?). Lenses can be added later. 

- Need to talk about monad recursion (? or cut the scope).

- Add code to the monad chapter 10 that shows defining a monad typeclass
instance for various constructions. I have this example code already,
but it was not included in the chapter. 
\end{comment}
\include{sofp-preface}

\part{Beginner level}

\include{sofp-nameless-functions}\include{sofp-induction}\include{sofp-disjunctions}

\part{Intermediate level}

\include{sofp-higher-order-functions}\include{sofp-curry-howard}

\include{sofp-functors}\include{sofp-reasoning}\include{sofp-typeclasses}\include{sofp-filterable}
\chapter{Computations in functor blocks. II. Semimonads and monads\label{chap:Semimonads-and-monads}}

\global\long\def\gunderline#1{\mathunderline{greenunder}{#1}}%
\global\long\def\bef{\forwardcompose}%
\global\long\def\bbnum#1{\custombb{#1}}%
\global\long\def\pplus{{\displaystyle }{+\negmedspace+}}%

In a Scala programmer\textsf{'}s view, functors are type constructors with
a \lstinline!map! method, filterable type constructors have a \lstinline!filter!
method, and \textbf{semimonads} are\index{semimonads} functors that
have a \lstinline!flatMap! method. This chapter begins by developing
an intuition for the behavior of \lstinline!flatMap!.

\section{Practical use of monads}

\subsection{Motivation for semimonads: Nested iteration}

How can we translate into code a computation that contains nested
iterations, such as:
\begin{equation}
\sum_{i=1}^{n}\sum_{j=1}^{n}\sum_{k=1}^{n}\frac{1}{1+i+j+k}=\,?\label{eq:semimonads-numerical-example-1}
\end{equation}
Recall that a \textsf{``}flat\textsf{''} (non-nested) iteration is translated into
the \lstinline!map! method applied to a sequence:

\begin{wrapfigure}{l}{0.4\columnwidth}%
\vspace{-0.45\baselineskip}
\begin{lstlisting}
(1 to n).map { i => 1.0 / (1 + i) }.sum
\end{lstlisting}
\vspace{-0.5\baselineskip}
\end{wrapfigure}%

~\vspace{-0.6\baselineskip}
\[
\sum_{i=1}^{n}\frac{1}{1+i}\quad.
\]
\vspace{-0.65\baselineskip}

The mathematical notation combines summation and iteration in one
symbol, but the code separates those two steps: first, a sequence
is computed as \lstinline!(1 to n).map { i => 1.0 / (1 + i) }!, holding
the values $\frac{1}{1+i}$ for $i=1,...,n$, and only then the \lstinline!sum!
function is applied to the sequence. This separation is useful because
it gives us full flexibility to transform or aggregate the sequence.

So, we will treat nested iterations in a similar way: first, compute
a sequence of values that result from nested iterations, and then
apply transformations or aggregations to that sequence.

If we use nested \lstinline!map! operations, we will obtain a nested
data structure, e.g., a vector of vectors:
\begin{lstlisting}
scala> (1 to 5).map(i => (1 to i).map(j => i * j))
res0: IndexedSeq[IndexedSeq[Int]] = Vector(Vector(1), Vector(2, 4), Vector(3, 6, 9), Vector(4, 8, 12, 16), Vector(5, 10, 15, 20, 25))
\end{lstlisting}
We need to \textsf{``}flatten\textsf{''} this nested structure into a simple, non-nested
sequence. The standard method for that is \lstinline!flatten!, and
its combination with \lstinline!map! can be replaced by \lstinline!flatMap!:
\begin{lstlisting}
scala> (1 to 4).map(i => (1 to i).map(j => i * j)).flatten
res1: IndexedSeq[Int] = Vector(1, 2, 4, 3, 6, 9, 4, 8, 12, 16)

scala> (1 to 4).flatMap(i => (1 to i).map(j => i * j))     // Same result as above.
res2: IndexedSeq[Int] = Vector(1, 2, 4, 3, 6, 9, 4, 8, 12, 16)
\end{lstlisting}
To represent more nesting, we use more \lstinline!flatMap! operations.
For example, to implement Eq.~(\ref{eq:semimonads-numerical-example-1}):
\begin{lstlisting}
def example(n: Int): Double = (1 to n).flatMap { i =>
  (1 to n).flatMap { j =>
    (1 to n).map { k => 
      1.0 / (1.0 + i + j + k) }
  }
}.sum

scala> example(10)
res3: Double = 63.20950497687006
\end{lstlisting}
These examples show that converting nested iterations into a simple
iteration means replacing all \lstinline!map! functions by \lstinline!flatMap!
except for the last \lstinline!map! call (which by itself produces
a non-nested sequence).

The \lstinline!for!/\lstinline!yield! syntax (or \textsf{``}functor block\textsf{''})
is an easier way of flattening nested iterations: just use a new source
line for each level of nesting. Compare the two following code fragments
line by line:

\noindent \texttt{\textcolor{blue}{\footnotesize{}}}%
\begin{minipage}[c]{0.475\columnwidth}%
\texttt{\textcolor{blue}{\footnotesize{}}}
\begin{lstlisting}
(for { i <- 1 to n
    j <- 1 to n
    k <- 1 to n
  } yield 1.0 / (1.0 + i + j + k)
).sum
\end{lstlisting}
%
\end{minipage}\texttt{\textcolor{blue}{\footnotesize{}\hspace*{\fill}}}%
\begin{minipage}[c]{0.475\columnwidth}%
\texttt{\textcolor{blue}{\footnotesize{}}}
\begin{lstlisting}
(1 to n).flatMap { i =>
   (1 to n).flatMap { j =>
     (1 to n).map { k =>
       1.0 / (1.0 + i + j + k)
  }}}.sum
\end{lstlisting}
%
\end{minipage}{\footnotesize\par}

\vspace{0.2\baselineskip}
The left arrows visually suggest that the variables \lstinline!i!,
\lstinline!j!, \lstinline!k! will iterate over the given sequences.
All left arrows except the last one are replaced by \lstinline!flatMap!s;
the last left arrow is replaced by a \lstinline!map!. These replacements
are performed automatically by the Scala compiler.

A functor block with source lines and conditionals corresponds to
the mathematical notation for creating sets of values. An example
of using that notation is the formula:

\begin{wrapfigure}{l}{0.22\columnwidth}%
\vspace{-0.6\baselineskip}
\begin{lstlisting}
val t = for {
  x <- p
  y <- q
  z <- r
  if f(x, y, z) == 0
} yield x + y + z
\end{lstlisting}
\vspace{0.6\baselineskip}
\end{wrapfigure}%

~\vspace{-0.3\baselineskip}
\[
T=\left\{ \left.x+y+z~\right|~x\in P,\,y\in Q,\,z\in R,\,f(x,y,z)=0\,\right\} \quad.
\]
Here, $P$, $Q$, $R$ are given sets of numbers, and the result is
a set $T$ of numbers obtained by adding some $x$ from $P$, some
$y$ from $Q$, and some $z$ from $R$ such that $f(x,y,z)=0$. A
direct implementation of this formula is the code shown at left. Here,
\lstinline!p!, \lstinline!q!, \lstinline!r! are given collections
(say, arrays) and the result \lstinline!t! is again an array. Just
like the mathematical formula\textsf{'}s result is a collection of some $x+y+z$
values, the functor block\textsf{'}s result is a collection of values computed
after the \lstinline!yield! keyword.

To develop more intuition about using functor blocks with multiple
left arrows, look at this code:

\noindent \texttt{\textcolor{blue}{\footnotesize{}}}%
\begin{minipage}[c]{0.475\columnwidth}%
\texttt{\textcolor{blue}{\footnotesize{}}}
\begin{lstlisting}
val result = for {
  i <- 1 to m
  j <- 1 to n
  x = f(i, j)
  k <- 1 to p
  y = g(i, j, k)
} yield h(x, y)
\end{lstlisting}
%
\end{minipage}\texttt{\textcolor{blue}{\footnotesize{}\hspace*{\fill}}}%
\begin{minipage}[c]{0.475\columnwidth}%
\texttt{\textcolor{blue}{\footnotesize{}}}
\begin{lstlisting}
val result = {
  (1 to m).flatMap { i =>
    (1 to n).flatMap { j =>
      val x = f(i, j)
      (1 to p).map { k =>
        val y = g(i, j, k)
        h(x, y)  } } } }
\end{lstlisting}
%
\end{minipage}{\footnotesize\par}

\vspace{0.2\baselineskip}
One can imagine that each line (which we can read as \textsf{``}for all $i$
in $\left[1,...,m\right]$\textsf{''}, \textsf{``}for all $j$ in $\left[1,...,n\right]$\textsf{''},
etc.) will produce an intermediate sequence of the same type. Each
next line continues the calculation from the previous intermediate
sequence. 

If this intuition is correct, we should be able to refactor the code
by cutting the calculation at any place and continuing in another
functor block, without changing the result value:

\noindent \texttt{\textcolor{blue}{\footnotesize{}}}%
\begin{minipage}[c]{0.475\columnwidth}%
\texttt{\textcolor{blue}{\footnotesize{}}}
\begin{lstlisting}
val result = for {
  i <- 1 to m
  j <- 1 to n
// We will cut the block here, making i and j
// available for further computations.
  x = f(i, j)
  k <- 1 to p
  y = g(i, j, k)
} yield h(x, y)
// The `result` is equal to `res2` at right.
\end{lstlisting}
%
\end{minipage}\texttt{\textcolor{blue}{\footnotesize{}\hspace*{\fill}}}%
\begin{minipage}[c]{0.475\columnwidth}%
\texttt{\textcolor{blue}{\footnotesize{}}}
\begin{lstlisting}
val res1 = for {
  i <- 1 to m
  j <- 1 to n
} yield (i, j) // Intermediate sequence `res1`.
val res2 = for {
  (i, j) <- res1   // Continue from `res1`.
  x = f(i, j)
  k <- 1 to p
  y = g(i, j, k)
} yield h(x, y)
\end{lstlisting}
%
\end{minipage}{\footnotesize\par}

\vspace{0.2\baselineskip}
This example illustrates the two features of functor blocks that often
cause confusion:
\begin{itemize}
\item Each \textsf{``}source line\textsf{''} computes an intermediate collection of the
same type, so all values to the right of \lstinline!<-! must use
\emph{the same} type constructor (or its subtypes).
\item The entire functor block\textsf{'}s result is again a collection using the
same type constructor. The result is \emph{not} the expression under
\lstinline!yield!; instead, it is a collection of those expressions.
\end{itemize}
So far, we have been using sequences as the main type constructor.
However, functor blocks with several left arrows will work with any
other type constructor that has \lstinline!map! and \lstinline!flatMap!
methods. In the next sections, we will see how to use functor blocks
with different type constructors. 

Functors having \lstinline!flatMap! methods are called \textbf{semimonads}
in\index{semimonads} this book.\footnote{There is no single accepted name. The libraries \texttt{scalaz} and
\texttt{cats} call the relevant typeclasses \lstinline!Bind! and
\lstinline!FlatMap! respectively.} In practice, most semimonads also have a \lstinline!pure! method
(i.e., belong to the \lstinline!Pointed! typeclass, see Section~\ref{subsec:Pointed-functors-motivation-equivalence}).
Semimonads with a \lstinline!pure! method (and obeying the appropriate
laws) are called \textbf{monads}.\index{monads} This chapter will
study semimonads and monads in detail. For now, we note that the functor
block syntax does not require functors to have a \lstinline!pure!
method; it works just as well with semimonads.

If a functor has a \lstinline!withFilter! method, Scala\textsf{'}s functor
block will also support the \lstinline!if! operation (see Section~\ref{sec:Practical-uses-of-filterable-functors}).
So, the full functionality of functor blocks can be used with \emph{filterable
semimonads}.

\subsection{List-like monads}

List-like monads are types that model nested loops over a collection
of data values. Examples of list-like monads are \lstinline!Seq!
and its subtypes, \lstinline!Stream!, \lstinline!Array!, non-empty
lists, sets, and dictionaries. These data types make different choices
of lazy or eager evaluation and memory allocation, but their \lstinline!flatMap!
methods work similarly: they \textsf{``}flatten\textsf{''} nested collections. 

Most list-like monads are also filterable. A common task for list-like
monads is to obtain a set of all possible solutions of a combinatorial
problem. One can then filter out unwanted combinations.

\subsubsection{Example \label{subsec:Example-list-monads-1}\ref{subsec:Example-list-monads-1}\index{solved examples}}

Compute all permutations of the three letters \lstinline!"a", "b", "c"!. 

\subparagraph{Solution}

We will compute a \emph{sequence} of all permutations by nested iteration.
First attempt:
\begin{lstlisting}
scala> for {
         x <- Seq("a", "b", "c")
         y <- Seq("a", "b", "c")
         z <- Seq("a", "b", "c")
       } yield x + y + z
res0: Seq[String] = List(aaa, aab, aac, aba, abb, abc, aca, acb, acc, baa, bab, bac, bba, bbb, bbc, bca, bcb, bcc, caa, cab, cac, cba, cbb, cbc, cca, ccb, ccc)
\end{lstlisting}
To obtain all permutations and nothing else, we need to exclude all
repeated subsequences such as \lstinline!"aab"!. To achieve that,
we must make \lstinline!y! iterate over letters that do not include
the current value of \lstinline!x!:
\begin{lstlisting}
val xs = Seq("a", "b", "c")

scala> for {
      x <- xs
      xsWithoutX = xs.filter(_ != x)
      y <- xsWithoutX
      xsWithoutXY = xsWithoutX.filter(_ != y)
      z <- xsWithoutXY
} yield x + y + z
res1: Seq[String] = List(abc, acb, bac, bca, cab, cba) 
\end{lstlisting}


\subsubsection{Example \label{subsec:Example-list-monads-2}\ref{subsec:Example-list-monads-2}}

Compute the set of all subsets of \lstinline!xs = Set("a", "b", "c")!. 

\subparagraph{Solution}

We aim to write the code as a nested iteration. Begin by choosing
one element, say \lstinline!"a"!. Some subsets of \lstinline!xs!
will contain \lstinline!"a"! and other subsets will not. So, let
\lstinline!xa! iterate over two possibilities: either an empty set,
\lstinline!Set()!, or a set containing just \lstinline!"a"!. Then
each subset of \lstinline!xs! is the union of \lstinline!xa! and
some subset that does not contain \lstinline!"a"!. We repeat the
same logic for \lstinline!"b"! and \lstinline!"c"!. The code is:
\begin{lstlisting}
val empty = Set[String]()

scala> for {
         xa <- Set(empty, Set("a"))
         xb <- Set(empty, Set("b"))
         xc <- Set(empty, Set("c"))
       } yield xa ++ xb ++ xc        // p ++ q is the union of the sets p and q.
res0: Set[Set[String]] = Set(Set(), Set(a, b), Set(b, c), Set(a, c), Set(a, b, c), Set(b), Set(c), Set(a))
\end{lstlisting}


\subsubsection{Example \label{subsec:Example-list-monads-3}\ref{subsec:Example-list-monads-3}}

Compute all sub-arrays of length $3$ in a given array. Type signature
and a test:
\begin{lstlisting}
def subarrays3[A](input: IndexedSeq[A]): IndexedSeq[IndexedSeq[A]] = ???

scala> subarrays3(IndexedSeq("a", "b", "c", "d"))
res3: IndexedSeq[IndexedSeq[String]] = Vector(Vector(a, b, c), Vector(a, b, d), Vector(a, c, d), Vector(b, c, d))
\end{lstlisting}


\subparagraph{Solution}

What are the indices of the elements of the required sub-arrays? Suppose
$n$ is the length of the given array. In any sub-array of length
$3$, the first element must have an index $i$ such that $0\leq i<n$.
The second element must have an index $j$ such that $i<j<n$, and
the third element\textsf{'}s index $k$ must satisfy $j<k<n$. So we can iterate
over the indices like this:
\begin{lstlisting}[mathescape=true]
def subarrays3[A](input: IndexedSeq[A]): IndexedSeq[IndexedSeq[A]] = {
  val n = input.length
  for {
    i <- 0 until n                   // Iterate over $\color{dkgreen} 0, 1, ..., n-1$.
    j <- i + 1 until n
    k <- j + 1 until n
  } yield IndexedSeq(input(i), input(j), input(k))
}
\end{lstlisting}


\subsubsection{Example \label{subsec:Example-list-monads-4}\ref{subsec:Example-list-monads-4}}

Generalize examples~\ref{subsec:Example-list-monads-1}\textendash \ref{subsec:Example-list-monads-3}
to support an arbitrary size $n$ instead of $3$.

\subparagraph{Solution}

\textbf{(a)} The task is to compute the set of all permutations of
$n$ letters. We note that the solution in Example~\ref{subsec:Example-list-monads-1}
used three source lines (such as \lstinline!x <- xs!), one for each
letter. To generalize that code to $n$ letters, we would need to
write a functor block with $n$ source lines. However, we cannot do
that if $n$ is a run-time parameter not fixed in advance. So, the
functor block must use recursion in $n$. Begin implementing \lstinline!permutations(xs)!
as a recursive function:
\begin{lstlisting}
def permutations(xs: Seq[String]): Seq[String] = for {
  x <- xs
  xsWithoutX = xs.filter(_ != x)
  ??? permutations(xsWithoutX) ??? // We need to use a recursive call somehow.
\end{lstlisting}
It is promising to use a recursive call of \lstinline!permutations!
with the sub-sequence \lstinline!xsWithoutX! that does not contain
a chosen letter \lstinline!x!. It remains to formulate the code as
nested iteration. Let us visualize the computation for \lstinline!xs == Seq("a","b","c")!.
While iterating over \lstinline!xs!, we start with \lstinline!x == "a"!,
which gives \lstinline!xsWithoutX == Seq("b","c")!. Iterating over
\lstinline!permutations(xsWithoutX)!, we obtain the permutations
\lstinline!"bc"! and \lstinline!"cb"!. These permutations need to
be concatenated with \lstinline!x == "a"!, yielding \lstinline!"abc"!
and \lstinline!"acb"!, which is the correct part of the final answer.
So, we write a nested iteration and concatenate the results:
\begin{lstlisting}
def permutations(xs: Seq[String]): Seq[String] = for {
  x <- xs
  xsWithoutX = xs.filter(_ != x)
  rest <- permutations(xsWithoutX)
} yield x + rest

scala> permutations(Seq("a", "b", "c", "d"))
res0: Seq[String] = List()
\end{lstlisting}
The code is wrong: it always returns an empty list! The reason is
that we provided no base case for the recursion. Eventually the intermediate
value \lstinline!xsWithoutX! becomes empty. A nested iteration with
an empty list always makes the final result also an empty list. To
fix this, add a base case:
\begin{lstlisting}
def permutations(xs: Seq[String]): Seq[String] = if (xs.length == 1) xs else for {
  x <- xs
  xsWithoutX = xs.filter(_ != x)
  rest <- permutations(xsWithoutX)
} yield x + rest

scala> permutations(Seq("a", "b", "c", "d"))
res1: Seq[String] = List(abcd, abdc, acbd, acdb, adbc, adcb, bacd, badc, bcad, bcda, bdac, bdca, cabd, cadb, cbad, cbda, cdab, cdba, dabc, dacb, dbac, dbca, dcab, dcba) 
\end{lstlisting}

\textbf{(b)} To find all subsets of a set via nested iteration, we
cannot directly extend the code from Example~\ref{subsec:Example-list-monads-2}
because we cannot write an unknown number of source lines in a functor
block. As in part \textbf{(a)}, we need to refactor the code and write
a functor block that uses recursion:
\begin{lstlisting}
def subsets[A](xs: Set[A]): Set[Set[A]] = for {
  x <- xs
  xsWithoutX = xs - x           // Use the difference operation for sets.
  ??? subsets(xsWithoutX) ???   // We need to use a recursive call somehow.
\end{lstlisting}
If \lstinline!xs == Set("a", "b", "c")! and \lstinline!x == "a"!
during an iteration, we get \lstinline!xsWithoutX == Set("b", "c")!.
Once we compute \lstinline!subsets(xsWithoutX)!, we need to use all
those subsets together with \lstinline!x! and also without adding
\lstinline!x!. We also should not forget to write the base case (an
empty set \lstinline!xs!). So, the code becomes:
\begin{lstlisting}
def subsets[A](xs: Set[A]): Set[Set[A]] = if (xs.isEmpty) Set(xs) else for {
  x <- xs
  xsWithoutX = xs - x
  rest <- subsets(xsWithoutX)   // Recursive call.
  maybeX <- Set(Set(x), Set())
} yield maybeX ++ rest

scala> subsets(Set("a", "b", "c", "d"))
res0: Set[Set[String]] = Set(Set(), Set(a, c, d), Set(a, b), Set(b, c), Set(a, d), Set(a, b, d), Set(b, c, d), Set(a, c), Set(c, d), Set(a, b, c), Set(d), Set(b), Set(b, d), Set(a, b, c, d), Set(c), Set(a)) 
\end{lstlisting}

\textbf{(c)} To compute all sub-arrays of length $n$ from a given
array, we rewrite the solution in Example~\ref{subsec:Example-list-monads-3}
via a recursive function that computes the sequence of indices:
\begin{lstlisting}
def subindices(begin: Int, end: Int, n: Int): IndexedSeq[IndexedSeq[Int]] =
  if (n == 0) IndexedSeq(IndexedSeq()) else for {
    i <- begin until end
    rest <- subindices(i + 1, end, n - 1)   // Recursive call.
  } yield IndexedSeq(i) ++ rest

scala> subindices(0, 4, 2)
res0: IndexedSeq[IndexedSeq[Int]] = Vector(Vector(0, 1), Vector(0, 2), Vector(0, 3), Vector(0, 4), Vector(1, 2), Vector(1, 3), Vector(1, 4), Vector(2, 3), Vector(2, 4), Vector(3, 4))
\end{lstlisting}
The sequence of subarrays is easy to compute once the sequence of
subarray indices is known: 
\begin{lstlisting}
def subarrays[A](n: Int, input: IndexedSeq[A]): IndexedSeq[IndexedSeq[A]] =
  subindices(0, input.length, n).map(_.map(i => input(i)))

scala> subarrays(4, IndexedSeq("a", "b", "c", "d", "e"))
res1: IndexedSeq[IndexedSeq[String]] = Vector(Vector(a, b, c, d), Vector(a, b, c, e), Vector(a, b, d, e), Vector(a, c, d, e), Vector(b, c, d, e))
\end{lstlisting}

The solutions \textbf{(a)}\textendash \textbf{(c)} are not tail recursive
since recursive calls within a source line in a functor block are
translated into recursive calls \emph{inside} \lstinline!map! or
\lstinline!flatMap! methods; that is, not in tail positions. Achieving
tail recursion in functor blocks requires techniques beyond the scope
of this chapter.

\subsubsection{Example \label{subsec:Example-list-monads-5}\ref{subsec:Example-list-monads-5}}

Find all solutions of the \textsf{``}$8$ queens\textsf{''} problem.\index{8@8 queens problem}

\subparagraph{Solution}

The $8$ queens must be placed on an $8\times8$ chess board so that
no queen threatens any other queen. To make our work easier, we note
that each queen must be placed in a different row. So, it is sufficient
to find the column index for each queen. A solution is a sequence
of $8$ indices.

Begin by iterating over all possible combinations of column indices:
\begin{lstlisting}
val solutions = for {
  x0 <- 0 until 8              // Queen 0 has coordinates (x0, 0).
  x1 <- 0 until 8              // Queen 1 has coordinates (x1, 1).
  x2 <- 0 until 8              // Queen 2 has coordinates (x2, 2).
\end{lstlisting}
It remains to filter out invalid positions. We should start filtering
as early as possible, since the total number of combinations grows
quickly during nested iterations:
\begin{lstlisting}
val solutions = for {
  x0 <- 0 until 8              // Queen 0 has coordinates (x0, 0).
  x1 <- 0 until 8              // Queen 1 has coordinates (x1, 1).
  if noThreat(x1, Seq(x0))     // Queen 1 does not threaten queen 0.
  x2 <- 0 until 8              // Queen 2 has coordinates (x2, 2).
  if noThreat(x2, Seq(x0, x1)) // Queen 2 does not threaten queens 0 and 1.
  ...
} yield Seq(x0, x1, x2, ...)
\end{lstlisting}
Here, \lstinline!noThreat! is a helper function that decides whether
a new queen threatens previous ones:
\begin{lstlisting}
def noThreat(otherX: Int, prev: Seq[Int]): Boolean = {
  val otherY = prev.length
  prev.zipWithIndex.forall { case (x, y) =>   // Check the vertical and the two diagonals.
    x != otherX && x - y != otherX - otherY && x + y != otherX + otherY
  }
}
\end{lstlisting}
We used a feature of Scala allowing us to pass a sequence of arguments
via the syntax \lstinline!Int*!, which means a variable number of
arguments of type \lstinline!Int!. We can now complete the code and
test it:
\begin{lstlisting}
val column = 0 until 8
val solutions = for {
  x0 <- column                 // Queen 0 has coordinates (x0, 0).
  x1 <- column                 // Queen 1 has coordinates (x1, 1).
  if noThreat(x1, Seq(x0))     // Queen 1 does not threaten queen 0.
  x2 <- column                 // Queen 2 has coordinates (x2, 2).
  if noThreat(x2, Seq(x0, x1)) // Queen 2 does not threaten queens 0 and 1.
  x3 <- column
  if noThreat(x3, Seq(x0, x1, x2))
  x4 <- column
  if noThreat(x4, Seq(x0, x1, x2, x3))
  x5 <- column
  if noThreat(x5, Seq(x0, x1, x2, x3, x4))
  x6 <- column
  if noThreat(x6, Seq(x0, x1, x2, x3, x4, x5))
  x7 <- column
  if noThreat(x7, Seq(x0, x1, x2, x3, x4, x5, x6))
} yield Seq(x0, x1, x2, x3, x4, x5, x6, x7)

scala> solutions.take(5)    // First 5 solutions.
res0: IndexedSeq[Seq[Int]] = Vector(List(0, 4, 7, 5, 2, 6, 1, 3), List(0, 5, 7, 2, 6, 3, 1, 4), List(0, 6, 3, 5, 7, 1, 4, 2), List(0, 6, 4, 7, 1, 3, 5, 2), List(1, 3, 5, 7, 2, 0, 6, 4))
\end{lstlisting}


\subsubsection{Example \label{subsec:Example-list-monads-6}\ref{subsec:Example-list-monads-6}}

Generalize Example~\ref{subsec:Example-list-monads-5} to solve the
\textsf{``}$n$ queens\textsf{''} problem.\index{$n$ queens problem}

\subparagraph{Solution}

In this problem, $n$ queens must be placed on an $n\times n$ chess
board. We need to find and count all solutions.\footnote{There is no known general formula for the number of solutions of the
$n$-queens problem. See a discussion here, \texttt{\href{https://math.stackexchange.com/questions/1872444/}{https://math.stackexchange.com/questions/1872444/}}} As in the $8$-queens problem, each queen must be placed in a different
row, and so we represent a solution by a sequence of $n$ column indices,
each index between $0$ and $n-1$.

Begin by writing code that expects to use itself recursively:
\begin{lstlisting}
def nQueens(n: Int): Seq[Seq[Int]] = for {
  x0 <- 0 until n
  ??? nQueens(n - 1) ???
\end{lstlisting}
Possible positions of new queens depend on the chosen positions for
all previous queens. So, the recursive function must receive that
information. Write an auxiliary recursive function \lstinline!nQueensPartial!
that computes all remaining positions given a sequence of (less than
$n$) previously found queens:
\begin{lstlisting}
def nQueensPartial(n: Int, prev: Seq[Int]): Seq[Seq[Int]] = for {
  x <- 0 until n
  if noThreat(x, prev)         // The new queen does not threaten any previous queens.
  rest <- nQueensPartial(n - 1, prev +: x)    // Find positions with n - 1 new queens.
} yield x +: rest               // Prepend the new queen to the other queen positions.
\end{lstlisting}

This code still has two problems: first, the base case ($n=0$) is
not covered; second, the recursive function must be initially called
with correct arguments. The complete code is:
\begin{lstlisting}
def nQueens(n: Int): Seq[Seq[Int]] = {
    def nQueensPartial(m: Int, prev: Seq[Int]): Seq[Seq[Int]] = if (m == 0) Seq(Seq()) else for {
        x <- 0 until n
        if noThreat(x, prev)
        rest <- nQueensPartial(m - 1, prev :+ x)
      } yield x +: rest
    nQueensPartial(n, Seq())
}

scala> (nQueens(8).length, nQueens(9).length, nQueens(10).length, nQueens(11).length)
res0: (Int, Int, Int, Int) = (92,352,724,2680)
\end{lstlisting}


\subsubsection{Example \label{subsec:Example-list-monads-7}\ref{subsec:Example-list-monads-7}{*}}

Formulas of Boolean logic may be transformed by \textsf{``}expanding brackets\textsf{''},
\begin{align*}
{\color{greenunder}\text{expand brackets with a conjunction inside}:}\quad & (a\wedge b)\vee c=\left(a\vee c\right)\wedge\left(b\vee c\right)\quad,\\
{\color{greenunder}\text{expand brackets with a disjunction inside}:}\quad & (a\vee b)\wedge c=\left(a\wedge c\right)\vee\left(b\wedge c\right)\quad.
\end{align*}
Implication is transformed as $(a\Rightarrow b)=((\neg a)\vee b)$.
These transformations (expressed in Scala at left)

\begin{wrapfigure}{l}{0.4\columnwidth}%
\vspace{-0.6\baselineskip}
\begin{lstlisting}
(a && b) || c  ==  (a || c) && (b || c)
(a || b) && c  ==  (a && c) || (b && c)
(a implies b)  ==  (!a) || b
\end{lstlisting}
\vspace{-0.8\baselineskip}
\end{wrapfigure}%

\noindent allow us to rewrite any Boolean formula as a conjunction
of disjunctions with no more nested conjunctions inside, e.g., \lstinline*(a || b) && ((!c) || d || e)*.
This form is called the \textbf{conjunctive normal form}\index{conjunctive normal form}\footnote{\texttt{\href{https://en.wikipedia.org/wiki/Conjunctive_normal_form}{https://en.wikipedia.org/wiki/Conjunctive\_normal\_form}}}
(CNF) of a Boolean formula. We can also rewrite Boolean formula into
a \emph{disjunction} of conjunctions with no more nesting inside,
e.g., \lstinline*(p && q && !r) || ((!x) && y)*. This is called the
\textbf{disjunctive normal form}\index{disjunctive normal form}\footnote{\texttt{\href{https://en.wikipedia.org/wiki/Disjunctive_normal_form}{https://en.wikipedia.org/wiki/Disjunctive\_normal\_form}}}
(DNF) of a Boolean formula. The task is to implement functions that
convert formulas from CNF to DNF and back. 

\subparagraph{Solution}

We begin by designing a data type to represent CNFs. Let the type
parameter $A$ stand for the type of elementary Boolean formulas that
we denoted by \lstinline!a!, \lstinline!b!, \lstinline*!c*, ...,
i.e., Boolean formulas that contain no conjunctions or no disjunctions.
Then we may represent a conjunction as a \lstinline!Set[A]!, and
a disjunction of conjunctions as a \lstinline!Set[Set[A]]!. For instance,
\lstinline!(a || b) && (c || d || e)! is represented by \lstinline!Set(Set(a, b), Set(c, d, e))!.
Because we are using \lstinline!Set!, disjunction of two disjunctions
is easily implemented as a union of sets. Since \lstinline!Set! eliminates
repeated elements, the representation based on \lstinline!Set! will
also automatically perform simplification of formulas such as \lstinline!(a || b || a) == (a || b)!.

Conjunction of two conjunctions may be implemented as a union of sets
for the same reason.

We also need to figure out how to represent the Boolean constants
\lstinline!true! and \lstinline!false! in the CNF. When we compute
a conjunction such as \lstinline!(a || b) && c && true!, the result
must be just \lstinline!(a || b) && c!. Since \lstinline!(a || b) && c!
is represented as \lstinline!Set(Set(a, b), Set(c))!, and since we
compute conjunctions via unions of sets, we expect to have the equation:
\begin{lstlisting}
Set(Set(a, b), Set(c)) union Set(???) == Set(Set(a, b), Set(c))
\end{lstlisting}
This must hold for any formulas; so we must represent \lstinline!true!
as an \emph{empty} set of disjunctions, \lstinline!Set()!.

\begin{wrapfigure}{l}{0.4\columnwidth}%
\vspace{-0.6\baselineskip}
\begin{lstlisting}
final case class CNF[A](s: Set[Set[A]])
def trueCNF[A]    = CNF[A](Set())
def falseCNF[A]   = CNF[A](Set(Set()))
\end{lstlisting}
\vspace{-0.8\baselineskip}
\end{wrapfigure}%

The value \lstinline!false! should have the property that for any
group $g$ of disjunctions, e.g., \lstinline!g == (x || y)!, we must
have \lstinline!g || false == g!. So, \lstinline!false! corresponds
to an empty disjunction. However, CNF represents disjunctions by nested
sets contained within conjunctions. Thus, \lstinline!Set(Set())!
must stand for \lstinline!false!.

Similar arguments hold for representing DNFs. We again use nested
sets, but this time the conjunctions and disjunctions are swapped,
so \lstinline!Set(Set(a, b), Set(c, d, e))! stands for the  DNF formula
\lstinline!(a && b) || (c && d && e)!. The value \lstinline!Set()!
represents an empty disjunction and stands for the constant \lstinline!false!,
while the constant \lstinline!true! is \lstinline!Set(Set())!, a
disjunction containing an empty conjunction.

\begin{wrapfigure}{l}{0.4\columnwidth}%
\vspace{-0.6\baselineskip}
\begin{lstlisting}
final case class DNF[A](s: Set[Set[A]])
def trueDNF[A]    = DNF[A](Set(Set()))
def falseDNF[A]   = DNF[A](Set())
\end{lstlisting}
\vspace{-1\baselineskip}
\end{wrapfigure}%

\noindent It is easy to make mistakes when reasoning about the constants
\lstinline!true! and \lstinline!false! in these representations.
We need to test the code to make sure our implementations of \lstinline!true!
and \lstinline!false! are correct.

Before writing the full code, let us implement the DNF-to-CNF conversion
for a short example: 
\begin{lstlisting}
     (a && b) || (c && d)         ---->          (a || c) && (a || d) && (b || c) && (b || d)
                                 // The same formulas are written in the set-based representation as:
DNF(Set(Set(a, b), Set(c, d)))    ---->      CNF(Set(Set(a, c), Set(a, d), Set(b, c), Set(b, d)))
\end{lstlisting}
A simple expression implementing this transformation for nested sets
could look like this: 
\begin{lstlisting}
for {
  x <- Set("a", "b")
  y <- Set("c", "d")
} yield Set(x) ++ Set(y)   // The result is Set(Set(a, c), Set(a, d), Set(b, c), Set(b, d)).
\end{lstlisting}
Generalizing this code to an arbitrary number of nested sets, we get
a first draft of a solution:
\begin{lstlisting}
def toCNF[A](dnf: Set[Set[A]]): Set[Set[A]] = for {
  x  <- dnf.head           // The `head` of a set; corresponds to Set("a", "b") in the example above.
  ys <- toCNF(dnf.tail)    // The `tail` of a set; corresponds to Set(Set("c", "d")) in the example.
                           // Converted to CNF, this corresponds to Set(Set("c"), Set("d")).
} yield Set(x) ++ ys  
\end{lstlisting}
This code is not yet fully correct: we have no base case for the recursion,
and the \lstinline!head! method will crash for an empty set. To make
the pattern matching on a \lstinline!Set! easier, let us implement
a function:
\begin{lstlisting}
def headTailOption[A](xs: Set[A]): Option[(A, Set[A])] =     // `List`-like pattern matching for Set.
  if (xs.isEmpty) None else Some((xs.head, xs.tail))
\end{lstlisting}
With that function, we can write the final code as:
\begin{lstlisting}
def toCNF[A](dnf: DNF[A]): CNF[A] = headTailOption(dnf.s) match {
  case None => falseCNF[A]     // Base case: The empty set Set() means `false` in DNF.
  case Some((head, tail)) => CNF(for {
    x  <- head                 // The first group of conjunctions.
    ys <- toCNF(DNF(tail)).s   // The remaining groups of conjunctions, recursively converted to CNF.
  } yield ys + x)              // For sets, `ys + x` is the same as `Set(x) ++ ys`.
}                              // Some tests for this code:
val dnf1 = DNF(Set(Set("a", "b"), Set("c", "d", "e")))
val cnf1 = CNF(Set(Set("a", "c"), Set("a", "d"), Set("a", "e"), Set("b", "c"), Set("b", "d"), Set("b", "e")))

scala> toCNF(dnf1) == cnf1
res0: Boolean = true 

scala> (toCNF(trueDNF) == trueCNF, toCNF(falseDNF) == falseCNF)
res1: (Boolean, Boolean) = (true,true)
\end{lstlisting}
Because the rules of Boolean logic are completely symmetric with respect
to swapping the operations $\vee$ and $\wedge$, the function \lstinline!cnf2dnf!
can be implemented via \lstinline!dnf2cnf! like this:
\begin{lstlisting}
def toDNF[A](cnf: CNF[A]): DNF[A] = DNF(toCNF(DNF(cnf.s)).s)

scala> (toDNF(trueCNF) == trueDNF, toDNF(falseCNF) == falseDNF)
res2: (Boolean, Boolean) = (true,true)
\end{lstlisting}

If we test some more, we will find that the functions \lstinline!toCNF!
and \lstinline!toDNF! are \emph{not} inverses:
\begin{lstlisting}
scala> toDNF(cnf1)     // Expected this to equal DNF(Set(Set("a", "b"), Set("c", "d", "e"))) == dnf1.
res3: DNF[String] = DNF(Set(Set(b, a), Set(b, a, c, e), Set(d, a, b), Set(b, e, a), Set(b, a, c), Set(d, e, c, b), Set(d, e, c), Set(d, a, c, b), Set(e, a, b, c, d), Set(d, a, e, c), Set(d, a, e, b)))
\end{lstlisting}
However, we already verified that \lstinline!cnf1 == toCNF(dnf1)!.
We expect the formula \lstinline!dnf1! to remain the same when we
convert it to CNF and back to DNF. Why the discrepancy? The reason
is that the transformations \lstinline!toDNF! and \lstinline!toCNF!
fail to perform certain simplifications allowed in Boolean logic.
For example, both \lstinline!(a) || (a && b)! and \lstinline!(a) && (a || b)!
are equivalent to just \lstinline!(a)!. In the set-based representations,
we must be able to simplify \lstinline!Set(Set("a"), Set("a", "b"))!
to just \lstinline!Set(Set("a"))!. Generally, we may discard any
conjunction group that happens to be a superset of another conjunction
group, and similarly for disjunction groups. After these simplifications,
the formula \lstinline!toDNF(cnf1)! becomes equal to \lstinline!dnf1!
because, as we can see, all inner sets in \lstinline!toDNF(cnf1)!
are supersets of either \lstinline!Set("a", "b")! or \lstinline!Set("c", "d", "e")))!
or both. We can implement the simplifications as a function \lstinline!simplifyCNF!:
\begin{lstlisting}
def simplifyCNF[A](cnf: CNF[A]): CNF[A] = CNF(
    cnf.s.toIndexedSeq.sortBy(_.size).foldLeft(Set[Set[A]]()) { case (prevGroups, group) =>
        if (prevGroups.exists(_ subsetOf group)) prevGroups else prevGroups + group
      }    // Omit `group` if it happens to be a superset of some previous group within `prevGroups`.
)    // Since `prevGroups` is sorted by group size, we only need to check later groups for supersets.

scala> simplifyCNF(CNF(Set(Set("a", "b"), Set("a", "b", "c", "d"))))
res3: CNF[String] = CNF(Set(Set(a, b)))
\end{lstlisting}
With this function, we may define more practically useful transformations
\lstinline!dnf2cnf! and \lstinline!cnf2dnf! as:
\begin{lstlisting}
def dnf2cnf[A](dnf: DNF[A]): CNF[A] = simplifyCNF(toCNF(dnf))
def cnf2dnf[A](cnf: CNF[A]): DNF[A] = DNF(dnf2cnf(DNF(cnf.s)).s)
                                                     // Verify that dnf2cnf and cnf2dnf are inverses:
scala> (dnf2cnf(cnf2dnf(cnf1)) == cnf1, cnf2dnf(dnf2cnf(dnf1)) == dnf1)
res4: (Boolean, Boolean) = (true,true)
\end{lstlisting}
The new conversion functions \lstinline!dnf2cnf! and \lstinline!cnf2dnf!
are inverses of each other.

\subsubsection{Example \label{subsec:Example-matrix-products}\ref{subsec:Example-matrix-products}{*}
(matrix operations)}

If matrices are represented by nested sequences of numbers, matrix
products can be calculated via nested functor blocks. The present
task is to implement: \textbf{(a)} matrix transposition, \textbf{(b)}
the vector-matrix dot product, and \textbf{(c)} the matrix-matrix
dot product.

\subparagraph{Solution}

\textbf{(a)} If a matrix has type \lstinline!Seq[Seq[A]]!, a transposed
matrix\index{matrix transposition} has the same type. An example:

\begin{wrapfigure}{l}{0.64\columnwidth}%
\vspace{-0.85\baselineskip}
\begin{minipage}[t]{0.45\linewidth}%
\begin{lstlisting}
val matrix: Seq[Seq[Int]] =
  Seq(
    Seq(1, 2, 3),
    Seq(10, 20, 30)
  )
\end{lstlisting}
%
\end{minipage}\hfill{}%
\begin{minipage}[t]{0.48\linewidth}%
\begin{lstlisting}
val matrix_T: Seq[Seq[Int]] =
  Seq(
    Seq(1, 10),
    Seq(2, 20),
    Seq(3, 30)
)
\end{lstlisting}
%
\end{minipage} ~ \vspace{-0.95\baselineskip}
\end{wrapfigure}%

\noindent To compute the transposed matrix, we have to iterate over
the initial matrix. A functor block of type \lstinline!Seq! will
return a value of type \lstinline!Seq[Seq[A]]! only if the \lstinline!yield!
expression itself returns a value of type \lstinline!Seq[A]!. The
example above shows that the first iteration should return \lstinline!Seq(1, 10)!,
which contains the first elements of all inner sequences from \lstinline!matrix!.
The second iteration should return all second elements, and so on.
We see that we need to iterate over the indices of the matrix columns.
Those indices are returned by the library method \lstinline!indices!:
\begin{lstlisting}
scala> Seq("Wilhelm Roentgen", "Henri Beckerel", "Marie Curie").indices
res0: scala.collection.immutable.Range = Range(0, 1, 2)
\end{lstlisting}
The value \lstinline!Seq(1, 10)! can be computed by the functor block:
\begin{lstlisting}
for { row <- matrix } yield row(0)
\end{lstlisting}
This functor block needs to be written \emph{inside} the \lstinline!yield!
value of the iterations of the indices:
\begin{lstlisting}
scala> val matrix_T = for { index <- matrix.head.indices }    // [0, 1, 2] are the column indices.
                      yield { for { row <- matrix }        // Iterate over the rows of the matrix.
                              yield row(index)
                            }
matrix_T: IndexedSeq[Seq[Int]] = Vector(List(1, 10), List(2, 20), List(3, 30))
\end{lstlisting}

The two nested \lstinline!for!/\lstinline!yield! blocks represent
two nested loops whose result value is again a nested sequence. We
have implemented nested loops over column and row indices by nested
functor blocks that use the column indices but directly iterate over
rows.

\textbf{(b)} To see how a vector-matrix dot product\index{dot product!of vector and matrix}
works, consider this example:
\[
\left|\begin{array}{ccc}
a_{0} & a_{1} & a_{2}\end{array}\right|\,\cdot\,\left|\begin{array}{cc}
b_{00} & b_{01}\\
b_{10} & b_{11}\\
b_{20} & b_{21}
\end{array}\right|\,=\,\left|\begin{array}{cc}
a_{0}\cdot b_{00}+a_{1}\cdot b_{10}+a_{2}\cdot b_{20} & a_{0}\cdot b_{01}+a_{1}\cdot b_{11}+a_{2}\cdot b_{21}\end{array}\right|\quad.
\]
The $2\times3$ matrix is represented as a sequence containing three
nested sequences. We need to iterate over the first elements of the
nested sequences ($b_{00}$, $b_{10}$, $b_{20}$) , multiply them
with the corresponding elements ($a_{0}$, $a_{1}$, $a_{2}$) of
the vector, and compute the sum of all products. The code is:
\begin{lstlisting}
Seq(b00, b10, b20).zip(Seq(a0, a1, a2)).map { case (x, y) => x * y }.sum
\end{lstlisting}
Then we need to do the same with the second elements ($b_{01}$, $b_{11}$,
$b_{21}$). To obtain the full code, we iterate over the indices of
the nested sequences, as we did in part \textbf{(a)} for the transposition:
\begin{lstlisting}
import scala.math.Numeric.Implicits.infixNumericOps
def vectorMatrixProduct[N: Numeric](vector: Seq[N], matrix: Seq[Seq[N]]): Seq[N] =
  for { index <- matrix.head.indices } yield {
    val b = for { row <- matrix } yield row(index)
    val pairs = for { (x, y) <- b.zip(vector) } yield x * y 
    pairs.sum
  }

scala> vectorMatrixProduct(Seq(3,4,5), matrix_T)
res1: Seq[Int] = Vector(26, 260)
\end{lstlisting}

\textbf{(c)} The matrix-matrix\index{dot product!of two matrices}
dot product is defined as a matrix containing the results of the vector-matrix
dot products, for all row vectors of the first matrix. For example:
\[
\left|\begin{array}{ccc}
a_{00} & a_{01} & a_{02}\\
a_{10} & a_{11} & a_{12}
\end{array}\right|\,\cdot\,\left|\begin{array}{cc}
b_{00} & b_{01}\\
b_{10} & b_{11}\\
b_{20} & b_{21}
\end{array}\right|\,=\,\left|\begin{array}{cc}
a_{00}\cdot b_{00}+a_{01}\cdot b_{10}+a_{02}\cdot b_{20} & a_{00}\cdot b_{01}+a_{01}\cdot b_{11}+a_{02}\cdot b_{21}\\
a_{10}\cdot b_{00}+a_{11}\cdot b_{10}+a_{12}\cdot b_{20} & a_{10}\cdot b_{01}+a_{11}\cdot b_{11}+a_{12}\cdot b_{21}
\end{array}\right|\quad.
\]
We reuse the solution of part \textbf{(b)} and write the code as:
\begin{lstlisting}
def matrixProduct[N: Numeric](matrix1: Seq[Seq[N]], matrix2: Seq[Seq[N]]): Seq[Seq[N]] =
  for { row1 <- matrix1 } yield vectorMatrixProduct(row1, matrix2)
\end{lstlisting}

The code shown in this example is for illustration only; for higher
performance, matrix operations must be implemented through flat arrays
rather than nested sequences.

\subsubsection{Exercise \label{subsec:Exercise-m-matrix-vector-dot-product}\ref{subsec:Exercise-m-matrix-vector-dot-product}\index{exercises}}

Implement the matrix-vector dot product as a function:
\begin{lstlisting}
def matrixVectorProduct[N: Numeric](matrix: Seq[Seq[N]], vector: Seq[N]): Seq[N] = ???
\end{lstlisting}
The result of computing the matrix-vector product is a column vector,
for example:
\[
\left|\begin{array}{ccc}
a_{00} & a_{01} & a_{02}\\
a_{10} & a_{11} & a_{12}
\end{array}\right|\,\cdot\,\left|\begin{array}{c}
b_{00}\\
b_{10}\\
b_{20}
\end{array}\right|\,=\,\left|\begin{array}{c}
a_{00}\cdot b_{00}+a_{01}\cdot b_{10}+a_{02}\cdot b_{20}\\
a_{10}\cdot b_{00}+a_{11}\cdot b_{10}+a_{12}\cdot b_{20}
\end{array}\right|\quad.
\]


\subsubsection{Exercise \label{subsec:Exercise-m-matrix-trace}\ref{subsec:Exercise-m-matrix-trace}}

The \textbf{trace} of a square matrix\index{trace of a matrix} is
the sum of its diagonal elements, e.g.:
\[
\text{Tr}\,\left|\begin{array}{ccc}
a_{00} & a_{01} & a_{02}\\
a_{10} & a_{11} & a_{12}\\
a_{20} & a_{21} & a_{22}
\end{array}\right|\,\triangleq a_{00}+a_{11}+a_{22}\quad.
\]
Implement the trace of a matrix (assuming a square matrix) as a function:
\begin{lstlisting}
def trace[N: Numeric](matrix: Seq[Seq[N]]): N = ???
\end{lstlisting}


\subsection{Pass/fail monads\label{subsec:Pass/fail-monads}}

The type \lstinline!Option[A]! can be viewed as a collection that
can either empty or hold a single value of type \lstinline!A!. An
\textsf{``}iteration\textsf{''} over such a collection will perform a computation
at most once:
\begin{lstlisting}
scala> for { x <- Some(123) } yield x * 2     // The computation is performed once.
res0: Option[Int] = Some(246) 
\end{lstlisting}
When an \lstinline!Option! value is empty, the computation is not
performed at all:
\begin{lstlisting}
scala> for { x <- None: Option[Int] } yield x * 2     // The computation is not performed at all.
res1: Option[Int] = None
\end{lstlisting}
What would a \emph{nested} \textsf{``}iteration\textsf{''} over several \lstinline!Option!
values do? When all of the \lstinline!Option! values are non-empty,
the \textsf{``}iteration\textsf{''} will perform some computations using the wrapped
values. However, if even one of the \lstinline!Option! values happens
to be empty, the computed result will be an empty value:
\begin{lstlisting}
scala> for {
         x <- Some(123)
         y <- None
         z <- Some(-1)
       } yield x + y + z
res2: Option[String] = None
\end{lstlisting}

Computations with \lstinline!Either! and \lstinline!Try! values
follow the same logic: nested \textsf{``}iteration\textsf{''} will perform no computations
unless all values are non-empty. This logic is useful for implementing
a series of computations that could produce failures, where any failure
should stop all further processing. For this reason (and since they
all support the \lstinline!pure! method and are lawful monads, as
this chapter will show), we call the type constructors \lstinline!Option!,
\lstinline!Either!, and \lstinline!Try! the \textbf{pass/fail monads}. 

The following schematic example illustrates this logic:
\begin{lstlisting}
val result: Try[A] = for { // Computations in the `Try` monad.
  x <- Try(k())            // First computation `k()`, may fail.
  y = f(x)                 // No possibility of failure in this line.
  if p(y)                  // The entire expression will fail if `p(y) == false`.
  z <- Try(g(x, y))        // The computation may also fail here.
  r <- Try(h(x, y, z))     // May fail here as well.
} yield r                  // If `r` has type `A` then `result` has type `Try[A]`.
\end{lstlisting}
The function \lstinline!Try()! catches exceptions thrown by its argument.
If one of \lstinline!k()!, \lstinline!g(x, y)!, or \lstinline!h(x, y, z)!
throws an exception, the corresponding \lstinline!Try(...)! call
will evaluate to a \lstinline!Failure(...)! case class, and further
computations will not be performed. The value \lstinline!result!
will indicate the \emph{first} encountered failure. Only if all \lstinline!Try(...)!
calls evaluate to a \lstinline!Success(...)! case class, the entire
expression evaluates to a \lstinline!result! of type \lstinline!Success!
that wraps a value of type \lstinline!A!.

Whenever this pattern of computation is found, a functor block gives
concise and readable code that replaces a series of nested \lstinline!if!/\lstinline!else!
or \lstinline!match!/\lstinline!case! expressions. Such a situation
was shown in Example~\ref{subsec:disj-Example-resultA} (Chapter~\ref{chap:Disjunctive-types}),
where a \textsf{``}safe integer\textsf{''} computation continues only as long as every
result is a success; the chain of operations stops at the first failure.
The code of Example~\ref{subsec:disj-Example-resultA} introduced
custom data type with hand-coded methods such as \lstinline!add!,
\lstinline!mul!, and \lstinline!div!. We can now implement equivalent
functionality using functor blocks and a standard type \lstinline!Either[String, Int]!:
\begin{lstlisting}
type Result = Either[String, Int]
def div(x: Int, y: Int): Result = if (y == 0) Left(s"error: $x / $y") else Right(x / y)
def sqrt(x: Int): Result = if (x < 0) Left(s"error: sqrt($x)") else Right(math.sqrt(x).toInt)
val initial: Result = Right(20)       // Start with a given `initial` value of type `Result`.

scala> val result: Result = for {     // Safe computation: `sqrt(1000 / initial - 100) + 20`.
  x <- initial
  y <- div(1000, x)
  z <- sqrt(y - 100)
} yield z + 1
result: Result = Left("error: sqrt(-50)")
\end{lstlisting}
The concise and readable code of \lstinline!val result! replaces
more verbose implementations such as:
\begin{lstlisting}
val result: Result = previous match {
  case Left(error)   => Left(error)
  case Right(x)      => div(1000, x) match {
    case Left(error)   => Left(error)
    case Right(y)      => sqrt(y - 100) match {
      case Left(error)   => Left(error)
      case Right(z)      => ...  // Keep writing repetitive, deeply nested code of this kind.
\end{lstlisting}

The following examples illustrate the typical tasks where pass/fail
monads are used. These tasks perform a linear sequence of computations
that may fail; the first failure is then returned as a value.

\subsubsection{Example \label{subsec:Example-:chain-with-option}\ref{subsec:Example-:chain-with-option}:
chaining computations with \texttt{Option}\index{solved examples}}

Some clients have placed some orders with some companies. The information
is made available via Java system properties, for example:
\begin{lstlisting}
System.setProperty("client 0", "company 2")
System.setProperty("client 1", "company 3")
System.setProperty("company 2", "order 4")
System.setProperty("company 3", "order 5")
System.setProperty("order 4", "123")
System.setProperty("order 5", "456")
\end{lstlisting}
Given a client\textsf{'}s name, obtain the corresponding order quantity if
it exists.

\subparagraph{Solution}

The Java method \lstinline!System.getProperty! returns the property
value as \lstinline!String! if the property is present, and otherwise
returns \lstinline!null!. Wrapping that return value into an \lstinline!Option()!
call, we replace null values by empty \lstinline!Option! values (i.e.,
\lstinline!None!). This makes the result values safe: using a \lstinline!null!
value may throw an exception, which will not happen when using \lstinline!map!
and \lstinline!flatMap! methods on \lstinline!Option! values. It
remains to chain the computations:

\begin{lstlisting}
def getOrderAmount(client: String): Option[Int] =   for {
      company     <- Option(System.getProperty(client))
      order       <- Option(System.getProperty(company))
      stringValue <- Option(System.getProperty(orders))
      intValue    <- Try(stringValue.toInt).toOption   // Non-integer string values are invalid.
    } yield intValue

scala> getOrderAmount("client 1")
res0: Option[Int] = Some(123)

scala> getOrderAmount("client 2")
res1: Option[Int] = Some(456)

scala> getOrderAmount("client 3")
res2: Option[Int] = None
\end{lstlisting}

In the example just shown, \lstinline!Option! values are sufficient
since the absence of a property is not an error situation. Now we
consider a task where we need to keep track of error information.

\subsubsection{Example \label{subsec:Example-:chain-with-option-1}\ref{subsec:Example-:chain-with-option-1}:
chaining computations with \texttt{Try}}

Three given functions $f$, $g$, $h$ all have Scala type \lstinline!Int => Int!
but may throw exceptions. Given an integer $x$, compute $f(g(h(x)))$
safely, reporting the first encountered error.

\subparagraph{Solution}

Wrap each function into a \lstinline!Try()! and chain the resulting
computations:
\begin{lstlisting}
def f(x: Int): Int = 1 / x
def g(x: Int): Int = 2 - x
def h(x: Int): Int = 2 / x

import scala.util.Try
def result(x: Int): Try[Int] = for {
      p <- Try(h(x))
      q <- Try(g(p))
      r <- Try(f(q))
    } yield r

scala> result(1)
res0: Try[Int] = Failure(java.lang.ArithmeticException: / by zero)
\end{lstlisting}
The result value shows information about the failure generated in
this computation.

\subsubsection{Example \label{subsec:Example-chaining-future}\ref{subsec:Example-chaining-future}:
chaining with \texttt{Future}}

Scala library\textsf{'}s \lstinline!Future! class can be seen as a pass/fail
monad because \lstinline!Future(x)! will encapsulate any exception
thrown while computing \lstinline!x!. However, in addition to the
pass/fail features, a \lstinline!Future! value has a concurrency
effect: the encapsulated computation \lstinline!x! is scheduled to
be run in parallel on another CPU thread. For this reason, \lstinline!Future!\textsf{'}s
methods (such as \lstinline!map! and \lstinline!flatMap!) require
an implicit \lstinline!ExecutionContext! argument, which provides
access to a JVM thread pool where computations will be scheduled.

As soon as a \lstinline!Future! value is created, its computation
is scheduled immediately. So, several \lstinline!Future! values may
run their computations in parallel. Nevertheless, computations chained
via \lstinline!flatMap! (or in a functor block) will run sequentially
if new values need to wait for previous values:
\begin{lstlisting}
import scala.concurrent.ExecutionContext.Implicits.global
def longComputation(x: Double): Future[Double] = Future { ... } // A long computation.

val result1: Future[Double] = for {
  p <- longComputation(10.0)
  q <- longComputation(p + 20.0)
  r <- longComputation(q - 20.0)
} yield p + q + r  // Three `longComputation` calls are running sequentially.
\end{lstlisting}
This code waits for the first \lstinline!Future! that computes \lstinline!p!,
then creates a \lstinline!Future! value that will eventually compute
\lstinline!q!, and finally creates a \lstinline!Future! that will
eventually compute \lstinline!r!; only then the sum \lstinline!p + q + r!
may be obtained (wrapped in a \lstinline!Future! constructor). This
computation cannot run the three \lstinline!longComputation(...)!
calls in parallel, since each call depends on the result of the previous
one.

Another possibility is that each \lstinline!longComputation(...)!
is independent of the results of the other computations. Then the
three \lstinline!Future! values may be created up front, and the
functor block code represents three \textsf{``}long computations\textsf{''} running
in parallel:
\begin{lstlisting}
val long1 = longComputation(10.0)
val long2 = longComputation(50.0)
val long3 = longComputation(100.0)

val result2: Future[Double] = for {
  p <- long1
  q <- long2
  r <- long3
} yield p + q + r  // Three `longComputation()` calls are running in parallel.
\end{lstlisting}


\subsection{Tree-like semimonads and monads\label{subsec:Tree-like-semimonads-and-monads}}

Tree-like type constructors are recursive types such as those described
in Section~\ref{sec:Lists-and-trees:recursive-disjunctive-types}.
A typical example of a tree-like type constructor is the binary tree
defined by the type equation:
\[
\text{BT}^{A}\triangleq A+\text{BT}^{A}\times\text{BT}^{A}.
\]

To show that $\text{BT}^{A}$ is a functor, Statement~\ref{subsec:functor-Statement-functor-composition-1}
replaces the right-hand side $A+\text{BT}^{A}\times\text{BT}^{A}$
by an arbitrary \textsf{``}recursion scheme\textsf{''}\index{recursion scheme} $S^{A,\text{BT}^{A}}$
and then shows that the recursive type constructor $L^{\bullet}$
defined by $L^{A}\triangleq S^{A,L^{A}}$ is a functor. (The type
$\text{BT}^{A}$ is obtained with $S^{A,R}\triangleq A+R\times R$.)
As we will see, the type constructor $L^{A}$ will be a semimonad
or a monad with certain choices of $S^{\bullet,\bullet}$.

For lists, nested iteration goes over inner lists contained in an
outer list. How does nested iteration work for a tree-shaped collection?
An iteration over a tree enumerates the values at the \emph{leaves}
of a tree. So, a tree analog of nested iteration implies that each
leaf of an outer tree contains an inner tree. A \lstinline!flatMap!
function must concatenate all nested trees into a single \textsf{``}flattened\textsf{''}
tree. 

Let us implement the \lstinline!flatMap! method for the binary tree
$BT^{\bullet}$ in that way. It is convenient to define an equivalent
curried function (denoted by \textsf{``}$\text{flm}$\textsf{''}) with type signature:
\[
\text{flm}^{A,B}:(A\rightarrow\text{BT}^{B})\rightarrow\text{BT}^{A}\rightarrow\text{BT}^{B}\quad.
\]
\begin{lstlisting}[numbers=left]
sealed trait BTree[A]
final case class Leaf[A](x: A) extends BTree[A]
final case class Branch[A](b1: BTree[A], b2: BTree[A]]) extends BTree[A]

def flm[A, B](f: A => BTree[B]): BTree[A] => BTree[B] = {             // t.flatMap(f) == flm(f)(t)
  case Leaf(x)          => f(x) // Here f(x) has type BTree[B], which could be a Leaf or a Branch.
  case Branch(b1, b2)   => Branch(flm(f)(b1), flm(f)(b2))             // Recursive calls of `flm`.
}
\end{lstlisting}
The same implementation is written in the code notation as:
\[
\text{flm}^{A,B}(f^{:A\rightarrow\text{BT}^{B}})\triangleq\,\begin{array}{|c||c|}
 & B+\text{BT}^{B}\times\text{BT}^{B}\\
\hline A & f\\
\text{BT}^{A}\times\text{BT}^{A} & b_{1}\times b_{2}\rightarrow\bbnum 0^{:B}+\overline{\text{flm}}(f)(b_{1})\times\overline{\text{flm}}(f)(b_{2})
\end{array}\quad.
\]

To visualize how the \lstinline!flatMap! method operates on binary
trees,  let us compute \lstinline!tree1.flatMap(f)!, where we take
\lstinline!tree1 = !{\scriptsize{} \Tree[  [ $a_1$ ] [ [ $a_2$ ] [ $a_3$ ] ] ] }
and a function $f:A\rightarrow\text{BT}^{B}$ that has $f(a_{1})=${\scriptsize{} \Tree[  [ $b_0$ ] [ $b_1$ ] ] },
$f(a_{2})=b_{2}$, and $f(a_{3})=${\scriptsize{} \Tree[  [ $b_3$ ] [ $b_4$ ] ]\relax}.
(Here $a_{i}$ for $i=1,2,3$ are some values of type $A$ and $b_{i}$
for $i=0,...,4$ are some values of type $B$.) Evaluating the code
of \lstinline!flatMap!, we find \lstinline!tree1.flatMap(f)! ={\scriptsize{} \Tree[  [ [ $b_0$ ] [ $b_1$ ] ] [ [ $b_2$ ] [ [ $b_3$ ] [ $b_4$ ] ] ] ]\relax}. 

So, we see that \lstinline!flatMap! works by grafting a subtree into
every \lstinline!Leaf! of a given tree: A leaf is replaced by a new
tree in line~6 in the code of \lstinline!flatMap!. That code can
be generalized to the recursive type $\text{PT}^{A}$ (representing
a \textsf{``}tree with $P$-shaped branches\textsf{''}) defined by:
\[
\text{PT}^{A}\triangleq A+P^{\text{PT}^{A}}\quad,
\]
for any given functor $P$. The disjunctive part $A+\bbnum 0$ is
replaced by a new tree:
\begin{lstlisting}
sealed abstract class PT[P[_] : Functor, A]    // Need an `abstract class` due to implicits.
final case class Leaf[P[_] : Functor, A](x: A)             extends PT[P, A]
final case class Branch[P[_] : Functor, A](p: P[PT[P, A]]) extends PT[P, A]

def flm[P[_]: Functor, A, B](f: A => PT[P, B]): PT[P, A] => PT[P, B] = {
  case Leaf(x)      => f(x)     // Here f(x) has type PT, which could be a Leaf or a Branch.
  case Branch(p)    => Branch(p.map(t => flm(f)(t))  // Conceptually, Branch(p.map(flm(f))). 
}
\end{lstlisting}
The same function is written in the code notation as:
\[
\text{flm}^{A,B}\big(f^{:A\rightarrow P^{\text{PT}^{B}}}\big)\triangleq\,\begin{array}{|c||c|}
 & B+P^{\text{PT}^{B}}\\
\hline A & f\\
P^{\text{PT}^{A}} & \big(\overline{\text{flm}}\,(f)\big)^{\uparrow P}
\end{array}\quad.
\]

We can also implement \lstinline!flatMap! for more general type constructors
$L$ defined by $L^{A}\triangleq P^{A}+P^{L^{A}}$ for some functor
$P$. Such $L^{A}$ can be visualized as trees with $P$-shaped branches
and $P$-shaped leaves.
\begin{lstlisting}
sealed abstract class PLPT[P[_] : Functor, A]
final case class Leaf[P[_] : Functor, A](px: P[A]) extends PLPT[P, A]
final case class Branch[P[_] : Functor, A](pb: P[PLPT[P, A]]) extends PLPT[P, A]

def flm[P[_]: Functor, A, B](f: A => PLPT[P, B]): PLPT[P, A] => PLPT[P, B] = {
  case Leaf(px)      => Branch(px.map(f))   // Here px.map(f) has type P[PLPT[P, B]].
  case Branch(pb)    => Branch(pb.map(t => flm(f)(t))
}
\end{lstlisting}
\[
\text{flm}^{A,B}\big(f^{:A\rightarrow P^{\text{PLPT}^{B}}}\big)\triangleq\,\begin{array}{|c||cc|}
 & P^{B} & P^{\text{PLPT}^{B}}\\
\hline P^{A} & \bbnum 0 & f^{\uparrow P}\\
P^{\text{PLPT}^{A}} & \bbnum 0 & \big(\overline{\text{flm}}\,(f)\big)^{\uparrow P}
\end{array}\quad.
\]
The code of \lstinline!flatMap! for the \lstinline!PLPT! tree never
creates any leaves, only branches. We will see later how this prevents
\lstinline!PLPT! from being a full monad (it is only a semimonad;
no \lstinline!pure! method can satisfy the required laws). Nevertheless,
having a lawful \lstinline!flatMap! is sufficient for using \lstinline!PLPT!
in functor blocks.

The following examples show some use cases for tree-like monads.

\subsubsection{Example \label{subsec:Example-monad-branching-properties}\ref{subsec:Example-monad-branching-properties}\index{solved examples}}

Implement the \lstinline!flatMap! operation for a tree of configuration
properties of the form:
\begin{lstlisting}
url: http://server:8000
users:
  user:
    name: abcde
    pass: 12345
  user2:
    name: fghij
    pass: 67890
\end{lstlisting}
Write a function to convert trees into strings in this format.

\subparagraph{Solution}

The code for this data structure must support any number of simple
properties or branches with \lstinline!String! labels. A suitable
structure is a tree with $P$-shaped leaves and $P$-shaped branches,
where the functor $P$ is defined as $P^{A}\triangleq\text{List}^{\text{String}\times A}$.
Implement the tree type:
\begin{lstlisting}
sealed trait PropTree[A]  // Introduce the type parameter A for the values of properties.
final case class Simple[A](props: List[(String, A)])                  extends PropTree[A]
final case class Branches[A](branches: List[(String, PropTree[A])])   extends PropTree[A]
\end{lstlisting}
To pretty-print trees of this type, we need to keep track of the current
level of indentation:
\begin{lstlisting}
def prettyPrint[A](pt: PropTree[A], indent: String = "")(toString: A => String): String = (pt match {
  case Simple(props) => props.map { case (name, a) => indent + name + ": " + toString(a) }
  case Branches(brs) => brs.map { case (name, ps) => indent + name + ":\n" +
    prettyPrint(ps, indent + "  ")(toString) }
}).mkString("\n") + "\n"
\end{lstlisting}
For convenience, the methods \lstinline!flatMap! and \lstinline!toString!
can be defined directly on the trait:
\begin{lstlisting}
sealed trait PropTree[A] {
  def flatMap[B](f: A => PropTree[B]): PropTree[B] = this match {
    case Simple(props) => Branches(props.map { case (name, a) => (name, f(a)) })
    case Branches(brs) => Branches(brs.map { case (name, ps) => (name, ps.flatMap(f)) })
  }
  override def toString: String = prettyPrint(this)(_.toString)
}
\end{lstlisting}

The following code illustrates the \lstinline!flatMap! operation
by replacing all integer leaf values below \lstinline!100! by a simple
property and other values by a branch of properties:
\begin{lstlisting}
val pt1: PropTree[Int] = Simple(List("quantity1" -> 50, "quantity2" -> 250))

scala> println(pt1.toString)
quantity1: 50
quantity2: 250

val pt2 = pt1.flatMap { x => if (x < 100) Simple(List("value" -> x)) else Branches(List(
               "large" -> Simple(List("value" -> x / 100, "factor" -> 100)),
               "small" -> Simple(List("value" -> x % 100, "factor" -> 1))
          )) }

scala> println(pt2.toString)
quantity1:
  value: 50
quantity2:
  large:
    value: 2
    factor: 100
  small:
    value: 50
    factor: 1
\end{lstlisting}


\subsubsection{Example \label{subsec:Example-monad-substitution-language}\ref{subsec:Example-monad-substitution-language}}

Implement variable substitution for a simple arithmetic language.
Example use:
\begin{lstlisting}
val expr1: Term[String] = Const(123) * Var("a") + Const(456) * Var("b")
val expr2: Term[String] = Const(20) * Var("c")

scala> expr1.flatMap { x => if (x == "a") expr2 else Var(x) } // Substitute "a" = expr2 in expr1.
res0: Term[String] = Plus(Mult(Const(123), Mult(Const(20), Var("c"))), Mult(Const(456), Var("b"))) 
\end{lstlisting}


\subparagraph{Solution}

Begin by implementing the basic functionality of the language: constants,
variables, addition, and multiplication. The type parameter \lstinline!A!
in \lstinline!Term[A]! is the type of labels for variables. 
\begin{lstlisting}
sealed trait Term[A] {
  def +(other: Term[A]): Term[A] = Plus(this, other)
  def *(other: Term[A]): Term[A] = Mult(this, other)
  def map[B](f: A => B): Term[B] = ???
  def flatMap[B](f: A => Term[B]): Term[B] = ???
}
final case class Const[A](value: Int)                extends Term[A]
final case class Var[A](name: A)                     extends Term[A]
final case class Plus[A](t1: Term[A], t2: Term[A])   extends Term[A]
final case class Mult[A](t1: Term[A], t2: Term[A])   extends Term[A]
\end{lstlisting}
The type constructor \lstinline!Term! is an example of an abstract
syntax tree\index{abstract syntax tree} that is equivalently defined
as:
\[
\text{Term}^{A}\triangleq L^{A}+S^{\text{Term}^{A}}\quad,\quad\quad L^{A}\triangleq\text{Int}+A\quad,\quad\quad S^{B}\triangleq B\times B+B\times B\quad,
\]
where the functors $L$ and $S$ describe the structure of the leaves
and the branches of the tree.

The code of the \lstinline!map! method is:
\begin{lstlisting}
def map[B](f: A => B): Term[B] = this match {  // This code must be within `trait Term[A]`.
  case Const(value)   => Const(value)
  case Var(name)      => Var(f(name))
  case Plus(t1, t2)   => Plus(t1 map f, t2 map f)
  case Mult(t1, t2)   => Mult(t1 map f, t2 map f)
}
\end{lstlisting}
The code of \lstinline!flatMap! replaces variables by new trees,
leaving everything else unchanged:
\begin{lstlisting}
def flatMap[B](f: A => Term[B]): Term[B] = this match {  // This code must be within `trait Term[A]`.
  case Const(value)   => Const(value)
  case Var(name)      => f(name)
  case Plus(t1, t2)   => Plus(t1 flatMap f, t2 flatMap f)
  case Mult(t1, t2)   => Mult(t1 flatMap f, t2 flatMap f)
}
\end{lstlisting}

Note that the \lstinline!flatMap! functions for lists, pass/fail
monads, and tree-like monads are information-preserving: no data is
discarded from the original tree or from any computed results.

\subsubsection{Example \label{subsec:Example-perfect-shaped-tree-not-monad}\ref{subsec:Example-perfect-shaped-tree-not-monad}}

Show that a perfect-shaped tree \emph{cannot} have an information-preserving
\lstinline!flatMap!.\index{perfect-shaped tree!is not a monad}

\subparagraph{Solution}

The type constructor $P^{A}$ for a perfect-shaped binary tree is
defined by $P^{A}\triangleq A+P^{A\times A}$. We can (non-rigorously)
view the type $P^{A}$ as a disjunction with \textsf{``}infinitely many\textsf{''}
parts:
\[
P^{A}\cong A+A\times A+A\times A\times A\times A+...\quad,
\]
where the $n^{\text{th}}$ part ($n=1,2,...$) is a product of $2^{n}$
copies of $A$. The \lstinline!flatMap! function must have the type:
\[
\text{flm}:(A\rightarrow P^{B})\rightarrow P^{A}\rightarrow P^{B}\quad.
\]
How could \lstinline!flatMap! work? For any function $f^{:A\rightarrow P^{B}}$,
we must compute a result value $\text{flm}\,(f)(p)$ of type $P^{B}$.
A possible value $p$ of type $P^{A}$ is, say, a product of $4$
copies of $A$. The only way of obtaining a value of type $P^{B}$
is to apply $f$ to some value of type $A$. But $p$ contains $4$
such values. If we apply $f$ to each one of them, we will obtain
$4$ different values of type $P^{B}$. Some of these values may correspond
to the part $B\times B$ in the disjunction, others to $B\times B\times B\times B$,
etc. The total number of values of type $B$ depends on the result
of that computation and is not necessarily equal to a power of $2$.
So, it will be impossible to accommodate all those values of type
$B$ within a \emph{single} perfect-shaped tree of type $P^{B}$.
If we discard some values of type $B$ to fit the rest into a perfect-shaped
tree, we will lose information. So, it is not possible to implement
an information-preserving \lstinline!flatMap! for a perfect-shaped
binary tree. (One can prove that $P^{A}$ does not have a monad implementation.)

\subsection{The \texttt{Reader} monad\label{subsec:The-Reader-monad}}

\index{monads!Reader monad@\texttt{Reader} monad}This chapter started
with the list-like monads whose \lstinline!flatMap! method is motivated
by the requirements of nested iteration. We then looked at tree-like
monads, which generalize nested list iterations to tree grafting.
It turns out that the \lstinline!flatMap! method can be generalized
to many other type constructors that are useful for various programming
tasks not limited to nested iteration.

A general (semi)monad type constructor $L^{A}$ no longer represents
a collection of data items of type $A$. Instead, we regard $L^{A}$
informally as a value of type $A$ wrapped in a special \textsf{``}computational
effect\textsf{''}. We view \textsf{``}computations with an $L$-effect\textsf{''} as functions
of type $A\rightarrow L^{B}$ (as in \lstinline!flatMap!\textsf{'}s argument
type). In this view, different monads \textemdash{} such as list-like,
pass/fail, or tree-like \textemdash{} implement different kinds of
effects. An ordinary function of type $A\rightarrow B$ is a computation
with a \textsf{``}trivial effect\textsf{''}.

In this sense, monadic effects are \emph{not} side effects.\index{side effect}
Functions of type $A\rightarrow L^{B}$ can be referentially transparent\index{referential transparency}
and behave as values.\index{value-like behavior} Informally, an \textsf{``}$L$-effect\textsf{''}
describes the information computed by a function of type $A\rightarrow L^{B}$
in addition to a value of type $B$. To make the vague idea of \textsf{``}effect\textsf{''}
concrete, we write the required type of the computation in the form
$A\rightarrow L^{B}$ with a specific type constructor $L$. In this
and the next subsections, we will look at some monads that can be
derived in that approach: the \lstinline!Reader!, \lstinline!Writer!,
\lstinline!Eval!, \lstinline!State!, and \lstinline!Cont! (\textsf{``}continuation\textsf{''})
monads.

The first of those, the \lstinline!Reader!, corresponds to a function
that computes a result while using some additional data of a fixed
type $Z$. A function consuming (\textsf{``}reading\textsf{''}) that additional data
will have type $A\times Z\rightarrow B$ instead of $A\rightarrow B$.
It remains to rewrite the type $A\times Z\rightarrow B$ in the form
$A\rightarrow L^{B}$ with a suitable choice of a type constructor
$L^{\bullet}$. By currying, we obtain an equivalent type:
\[
(A\times Z\rightarrow B)\cong(A\rightarrow Z\rightarrow B)\quad,
\]
which has the form $A\rightarrow L^{B}$ if we define $L^{A}\triangleq Z\rightarrow A$.
This type constructor is called the \lstinline!Reader! monad\index{monads!Reader monad@\texttt{Reader} monad}
and is denoted by $\text{Read}^{Z,A}\triangleq Z\rightarrow A$. The
Scala definition is \lstinline!type Reader[Z, A] = Z => A!.

Fully parametric implementations of \lstinline!map! and \lstinline!flatMap!
directly follow from their type signatures:
\begin{lstlisting}[mathescape=true]
def map[A, B](r: Z => A)(f: A => B): (Z => B) = r andThen f  // Example $\color{dkgreen} \ref{subsec:ch-solvedExample-5}$.
def flatMap[A, B](r: Z => A)(f: A => Z => B): (Z => B) = { z => f(r(z))(z) } // Exercise $\color{dkgreen} \ref{subsec:ch-Exercise-5}$(c).
\end{lstlisting}

What are the use cases for \lstinline!Reader!? A functor block with
a \textsf{``}source\textsf{''} value of type \lstinline!Reader[Z, A]! needs to have
other source values of type \lstinline!Reader[Z, B]!, \lstinline!Reader[Z, C]!,
etc. The type $Z$ must be fixed for the entire functor block. So,
a value of type $Z$ plays the role of a common dependency or common
\textsf{``}environment data\textsf{''} that may be used by all computations within
the functor block.

As an example, imagine a program built up by composing several \textsf{``}procedures\textsf{''}.
Each \textsf{``}procedure\textsf{''} takes an argument and returns a value, like an
ordinary function, but also may need to run a Unix shell command.
For simplicity, assume that the interface to shell commands is a function
that runs a new command with given input, capturing the command\textsf{'}s
entire output. The type of that function may be $\text{String}\times\text{String}\rightarrow\text{Int}\times\text{String}$,
taking the command string and the input for the command, and returning
the command\textsf{'}s exit code and the output.
\begin{lstlisting}
type RunSh = (String, String) => (Int, String)
\end{lstlisting}
A simple implementation (that does not handle any run-time exceptions)
is:
\begin{lstlisting}
import sys.process._
val runSh: RunSh = { (command, input) =>
  var result: Array[Char] = Array()
  val p: Process = command.run(new ProcessIO(
    { os => os.write(input.getBytes); os.close() },
    { is => result = scala.io.Source.fromInputStream(is).toArray; is.close() },
    _.close())
  )
  val exitCode = p.exitValue()
  (exitCode, new String(result))
}
\end{lstlisting}
We can now use this \textsf{``}shell runner\textsf{''} to execute some standard Unix
commands:
\begin{lstlisting}
scala> runSh("echo -n abcd", "") // Use `-n` to avoid trailing newlines.
res0: (Int, String) = (0, "abcd")

scala> runSh("cat", "xyz")._2   // Equivalent to `cat < $(echo -n xyz)`.
res1: String = "xyz"
\end{lstlisting}

Consider a program that determines the total line count of all files
under a given directory, but only including files whose names match
given patterns. We may split the computation into three \textsf{``}procedures\textsf{''}:
1) list all files under the given directory, 2) filter out files whose
names do not match the given patterns, 3) determine the line count
for each of the remaining files. For the purposes of this example,
we will implement each stage as a function that uses \lstinline!runSh!
to run a shell command.
\begin{lstlisting}
def listFiles(runSh: RunSh, dir: String): String = runSh(s"find $dir -type f", "")._2
def filterFiles(runSh: RunSh, files: String, patterns: String): String =
  runSh(s"grep -f $patterns", files)._2
def lineCounts(runSh: RunSh, files: String): Array[Int] = files.split("\n")   // Array of file names.
  .map { file => runSh("wc -l $file", "")._2.replaceAll("^ +", "").split(" ")(0).toInt }
\end{lstlisting}
This code assumes that file names do not contain the newline character
\lstinline!"\n"!. Use this code only as an illustration of a use
case for the \lstinline!Reader! monad.

We can now write the program like this:
\begin{lstlisting}
def getLineCount(runSh: RunSh, dir: String, patterns: String): Int = {
  val fileList = listFiles(runSh, dir)
  val filtered = filterFiles(runSh, fileList, patterns)
  val counts = lineCounts(runSh, filtered)
  counts.sum
}
\end{lstlisting}
The value \lstinline!runSh! is a common dependency of all the \textsf{``}procedures\textsf{''}
and is repeated throughout the code. This repetition becomes a problem
if we have many \textsf{``}procedures\textsf{''} within different code modules; or
when different \textsf{``}runners\textsf{''} of type \lstinline!RunSh! are used within
the program. For instance, one \textsf{``}runner\textsf{''} executes commands on a
remote machine, while another \textsf{``}runner\textsf{''} is used for testing and
returns fixed results without running any shell commands. How can
we avoid writing repetitive code and at the same time assure that
the correct \textsf{``}runners\textsf{''} are passed to all the \textsf{``}procedures\textsf{''}?

The \lstinline!Reader! monad offers a solution: it allows us to combine
smaller \textsf{``}procedures\textsf{''} into larger ones while passing the \textsf{``}runner\textsf{''}
values automatically. We first need to convert all \textsf{``}procedures\textsf{''}
into functions of type \lstinline!A => Reader[RunSh, B]! with suitable
choices of type parameters \lstinline!A!, \lstinline!B!:
\begin{lstlisting}
type Reader[Z, A] = Z => A
def listFilesR(dir: String): Reader[RunSh, String] = runSh => runSh(s"find $dir -type f", "")._2
def filterFilesR(patterns: String): String => Reader[RunSh, String] = files => runSh =>
  runSh(s"grep -f $patterns", files)._2
def lineCountsR(files: String): Reader[RunSh, Array[Int]] = runSh => files.split("\n")
  .map { file => runSh("wc -l $file", "")._2.replaceAll("^ +", "").split(" ")(0).toInt }
\end{lstlisting}
 This allows us to express \lstinline!getLineCount! as a combination
of the three \textsf{``}procedures\textsf{''} by using the \lstinline!Reader! monad\textsf{'}s
\lstinline!flatMap! function (for convenience, assume that we defined
an extension method \lstinline!flatMap!):
\begin{lstlisting}
def getLineCount(dir: String, patterns: String): Reader[RunSh, Int] = listFilesR(dir)
  .flatMap(files => filterFilesR(patterns)(files))
  .flatMap(lineCountsR).map(_.sum) // Assuming an extension method `map` is defined for `Reader`.
\end{lstlisting}
For better readability, rewrite this code equivalently in the functor
block syntax:
\begin{lstlisting}
def getLineCountR(dir: String, patterns: String): Reader[RunSh, Int] = for {
  files       <-  listFilesR(dir)
  filtered    <-  filterFilesR(patterns)(files)
  lineCounts  <-  lineCountsR(filtered)
} yield lineCounts.sum

val program: Reader[RunSh, Int] = getLineCountR(".", "patterns.txt") 
\end{lstlisting}
We obtained a value \lstinline!program! of type \lstinline!Reader[RunSh, Int]!,
which is a function type. It is important to note that at this point
no shell commands have been run yet. We merely packaged all the necessary
actions into a function value (a \textbf{monadic program}\index{monadic program}).
We now need to apply that function to a \textsf{``}runner\index{monads!runner}\index{runner!for monads}\textsf{''}
value of type \lstinline!RunSh!. Only then we will obtain the actual
line count:
\begin{lstlisting}
val count: Int = program(runSh)
\end{lstlisting}

The \lstinline!Reader! monad allows us to split the code into two
stages: first, we build up a monadic program (a value of type \lstinline!Reader!)
from its parts (\textsf{``}procedures\textsf{''}). Monadic programs are ordinary values
that can be passed as arguments to functions, stored in arrays, etc.
We may compose smaller monadic programs into larger ones using \lstinline!map!,
\lstinline!flatMap!, or \lstinline!for!/\lstinline!yield! blocks.
We have the full flexibility of manipulating those monadic values
and combining them into larger monadic programs in any order, since
no shell commands are being run while we perform these manipulations.
When we are done building the full monadic program, we can \textsf{``}run\textsf{''}
it using a chosen runner.\index{monads!runner}

Since the runner is the common dependency of all \lstinline!Reader!-monadic
programs, running a monadic program means performing \textbf{dependency
injection}\index{dependency injection}. At the point of running a
\lstinline!Reader! program, we have the full flexibility of choosing
the value of the dependency. The code guarantees that the dependency
will be passed correctly to each individual part of the monadic program.

\paragraph{Implicit values for dependency injection}

Scala\textsf{'}s implicit argument feature allows us to solve the problem of
dependency injection in a different way. Instead of converting all
code to use the \lstinline!Reader! monad, we convert all code to
use an implicit argument for the common dependency:
\begin{lstlisting}
def listFilesIm(dir: String)(implicit runSh: RunSh): String = ...
def filterFilesIm(patterns: String)(files: String)(implicit runSh: RunSh): String = ...
def lineCountsIm(files: String)(implicit runSh: RunSh): Array[Int] = ...

def getLineCountIm(dir: String, patterns: String)(implicit runSh: RunSh): Int = {
  val fileList = listFilesIm(dir)
  val filtered = filterFilesIm(fileList, patterns)
  val counts = lineCountsIm(filtered)
  counts.sum
}
\end{lstlisting}
Compare this code and the code that uses the \lstinline!Reader! monad:
the type signatures of functions are the same up to the \lstinline!implicit!
keyword in the last argument. Scala\textsf{'}s implicit arguments reproduce
the code style of the \lstinline!Reader! monad (especially with Scala
3\textsf{'}s implicit function types\footnote{See \texttt{\href{https://www.scala-lang.org/blog/2016/12/07/implicit-function-types.html}{https://www.scala-lang.org/blog/2016/12/07/implicit-function-types.html}}}).

\subsection{The \texttt{Writer} monad}

\index{monads!Writer monad@\texttt{Writer} monad}The \lstinline!Writer!
monad represents a function of type $A\rightarrow B$ that returns
its result (of type $B$) and additionally outputs some information
(say, logging data) about the computation just performed. Let $W$
be the type of the logging data. To model this situation, we then
need a function $f:A\rightarrow B$ and additionally a function $g:A\rightarrow W$
that computes the logging output. So, we can define the type of a
\textsf{``}computation with the \lstinline!Writer! effect\textsf{''} via the product
of the two function types:
\[
\left(A\rightarrow B\right)\times\left(A\rightarrow W\right)\cong A\rightarrow B\times W\quad.
\]
Since this type has be of the form $A\rightarrow L^{B}$, we must
define the functor $L$ as $L^{A}\triangleq A\times W$. This is the
type constructor of the \lstinline!Writer! monad, denoted by $\text{Writer}^{A,W}\triangleq A\times W$.

If several computations are performed one after another in the \lstinline!Writer!
monad, the logging information should be \textsf{``}accumulated\textsf{''} in some
way. In the logging example, additional lines are simply appended
to the log file. It means that we must be able somehow to combine
several values of type $W$ into one. A general way of doing that
is to require $W$ to be a semigroup (see Example~\ref{subsec:tc-Example-Semigroups})
with a binary operation $\oplus$. We can then implement the \lstinline!flatMap!
method for \lstinline!Writer! like this:
\begin{lstlisting}
final case class Writer[A, W: Semigroup](a: A, log: W) {
  def flatMap[B](f: A => Writer[B, W]): Writer[B, W] = {
   val Writer(b, newLog) = f(a) // Pattern-match to destructure the value f(a).
   Writer(b, log |+| newLog)    // Use the semigroup operation |+|.
  }
}
\end{lstlisting}
\begin{comment}
\[
\text{flm}_{\text{Write}}(f^{:A\rightarrow B\times W})\triangleq a\times w\rightarrow\left(\pi_{1}f(a)\right)\times\left(w\oplus\pi_{2}f(a)\right)\quad.
\]
\end{comment}

The logging type $W$ is often a monoid (a semigroup with an \textsf{``}empty\textsf{''}
value). If so, \lstinline!Writer[A, W]! will be a full monad whose
\lstinline!pure! method is implemented as:
\begin{lstlisting}
def pure[A, W: Monoid]: A => (A, W) = a => (a, Monoid[W].empty)
\end{lstlisting}
When $W$ is a semigroup but not a monoid, \lstinline!Writer[A, W]!
will be a semimonad but not a monad. 

An example of using a \lstinline!Writer! semimonad\index{semimonads!example of usage}
is logging with timestamps where we need to keep track of the earliest
and the latest timestamp. Define the type $W\triangleq~$\lstinline!Logs!
and a semigroup operation \lstinline!|+|! by:
\begin{lstlisting}
final case class Logs(begin: LocalDateTime, end: LocalDateTime, message: String) {
  def |+|(other: Logs): Logs = Logs(begin, other.end, message + "\n" + other.message)
}      // For simplicity, we assume that timestamps will be monotonically increasing.
\end{lstlisting}
The type \lstinline!Logs! is not a monoid because its binary operation
discards some of the input data, so we cannot define an \textsf{``}empty\textsf{''}
value satisfying the identity laws (see Eq.~(\ref{eq:identity-laws-of-monoid})
in Example~\ref{subsec:tc-Example-Monoids}).

We can now use the semimonad \lstinline!Writer[A, Logs]!. Here are
some example computations:
\begin{lstlisting}
type Logged[A] = Writer[A, Logs]
def log[A](message: String)(x: A): Logged[A] = {        // Define this function for convenience.
  val timestamp = LocalDateTime.now
  new Logged(x, Logs(timestamp, timestamp, message))
}
def compute[A](x: => A): A = { Thread.sleep(100L); x }           // Simulate a long computation.

scala> val result: Logged[Double] = for {
  x <- log("begin with 3")(compute(3))              // The initial source type is `Logged[Int]`.
  y <- log("add 1")(compute(x + 1))
  z <- log("multiply by 2.0")(compute(y * 2.0))  // The type of result becomes `Logged[Double]`.
} yield z                                 // The computation should take between 300 and 400 ms.
res0: Logged[Double] = Writer(8.0,Logs(2020-02-15T22:02:42.313,2020-02-15T22:02:42.484,begin with 3
add 1
multiply by 2.0))
\end{lstlisting}

Unlike the \lstinline!Reader! monad, which delays all computations
until a runner is called, a monadic value of type \lstinline!Writer[A, W]!
already contains the final computed values of types \lstinline!A!
and \lstinline!W!.

\subsection{The \texttt{State} monad\label{subsec:The-State-monad}}

Heuristically, the \lstinline!Reader! monad $\text{Read}^{S,A}$
is able to \textsf{``}read\textsf{''} values of type $S$, while the \lstinline!Writer!
monad $\text{Writer}^{A,S}$ may \textsf{``}write\textsf{''} values of type $S$,
in addition to computing the result of type $A$. The \lstinline!State!
monad, denoted by $\text{State}^{S,A}$, combines the functionality
of \lstinline!Reader! and \lstinline!Writer! in a special way: an
extra value of type $S$ is updated and automatically passed from
one computation to the next. 

To derive the required type constructor, consider a computation of
type $A\rightarrow B$ that additionally needs to read and to write
a value of type $S$. Since the total input is a pair of $A$ and
$S$, and the total output is a pair of $B$ and $S$, this kind of
computation is represented by a function of type $A\times S\rightarrow B\times S$.
We now try to rewrite this type in the form $A\rightarrow L^{B}$
with a suitable type constructor $L$. It is clear that we need to
curry the argument $A$. The result is:
\[
\left(A\times S\rightarrow B\times S\right)\cong\left(A\rightarrow S\rightarrow B\times S\right)=A\rightarrow L^{B}\quad,\quad\text{where}\quad L^{B}\triangleq S\rightarrow B\times S\quad.
\]
So, the\index{monads!State monad@\texttt{State} monad} \lstinline!State!
monad must be defined by the type constructor $\text{State}^{S,A}\triangleq S\rightarrow A\times S$.
This is a function that computes a value of type $A$ while using
and possibly updating the \textsf{``}state value\textsf{''} of type $S$.

To code of the \lstinline!flatMap! method for this type constructor
was derived in Example~\ref{subsec:ch-solvedExample-9}(c). It does
indeed pass the updated state value to the next computation:
\begin{lstlisting}
type State[S, A] = S => (A, S)
def flatMap[S, A, B](prev: State[S, A])(f: A => State[S, B]): State[S, B] = { s =>
  val (a, newState) = prev(s)   // Compute result of type `A`, updating the state.
  f(a)(newState)                // Pass the updated state to the next computation.
}
\end{lstlisting}

An example of using the \lstinline!State! monad is the task of implementing
a random number generator. A simple generator is the \textbf{Lehmer\textsf{'}s
algorithm}\index{Lehmer\textsf{'}s algorithm},\footnote{See \texttt{\href{https://en.wikipedia.org/wiki/Lehmer_random_number_generator}{https://en.wikipedia.org/wiki/Lehmer\_random\_number\_generator}}}
which generates integer sequences $x_{n}$ defined by:
\[
x_{n+1}\triangleq\left(48271*x_{n}\right)\%\,(2^{31}-1)\quad,\quad\quad1\leq x_{n}\leq2^{31}-2\quad,\quad\quad n=0,1,2,...
\]
The \textsf{``}updating\textsf{''} function for this sequence, $x_{n+1}=\text{lehmer}\,(x_{n})$,
can be implemented as:
\begin{lstlisting}
def lehmer(x: Long): Long = x * 48271L % ((1L << 31) - 1)
\end{lstlisting}
In many applications, one needs uniformly distributed floating-point
numbers in the interval $\left[0,1\right]$. To produce such numbers,
let us define a helper function:
\begin{lstlisting}
def uniform(x: Long): Double = (x - 1).toDouble / ((1L << 31) - 3)   // Enforce the interval [0, 1].
\end{lstlisting}

To use the uniform generator, we need to provide an initial value
$x_{0}$ (the \textsf{``}seed\textsf{''}) and then call the function \lstinline!lehmer!
repeatedly on successive values. The code would look like this:

\begin{wrapfigure}{l}{0.4\columnwidth}%
\vspace{-0.8\baselineskip}
\begin{lstlisting}
val s0 = 123456789L  // A "seed" value.
val s1 = lehmer(s0)
val r1 = uniform(s1)
... // Use pseudo-random value r1.
val s2 = lehmer(s1)
val r2 = uniform(s2)
... // Use pseudo-random value r2.
val s3 = lehmer(s2)       // And so on.
\end{lstlisting}

\vspace{-1\baselineskip}
\end{wrapfigure}%

\noindent We need to keep track of the generator\textsf{'}s state values \lstinline!s1!,
\lstinline!s2!, ..., that are not directly needed for other computations.
This \textsf{``}bookkeeping\textsf{''} is error-prone since we might reuse a previous
generator state by mistake. The \lstinline!State! monad keeps track
of the updated state values automatically and correctly. This comes
at a cost: we need to convert all computations into \lstinline!State!-typed
monadic programs.

As a simple example, consider the task of generating uniformly distributed
floating-point numbers in the interval $\left[0,1\right]$. We need
to maintain the generator state while computing the result. The floating-point
generator is implemented as a monadic value of type \lstinline!State[Long, Double]!:
\begin{lstlisting}
val rngUniform: State[Long, Double] = { oldState =>
  val result = uniform(oldState)   // Enforce the interval [0, 1].
  val newState = lehmer(oldState)
  (result, newState)
}
\end{lstlisting}
Code using \lstinline!rngUniform! will be of the monadic type \lstinline!State[Long, A]!
for some \lstinline!A!:
\begin{lstlisting}
val program: State[Long, String] = for {  // Assume flatMap and map methods are defined for State.
  r1 <- rngUniform
  ... // Use pseudo-random value r1. The internal state of rngUniform is maintained automatically.
  r2 <- rngUniform
  ... // Use pseudo-random value r2.
} yield s"Pair is $r1, $r2" // Compute result of type String.
\end{lstlisting}

Monadic programs of this type can be composed in arbitrary ways. The
\textsf{``}bookkeeping\textsf{''} of the state values is safely handled by the \lstinline!State!
monad and hidden from the programmer\textsf{'}s view. When the entire monadic
program has been composed, it needs to be \textsf{``}run\textsf{''} to extract its
result value. Since the type \lstinline!State[S, A]! is a function
with argument of type $S$, the runner\index{monads!runner} is just
an application of that function to an initial value of type $S$,
specifying the initial state:
\begin{lstlisting}
val seed = 123456789L     // Initial state of the generator.

scala> program(seed)      // Run this monadic program.
res0: (String, Long) = ("Pair is 0.028744523433557146, 0.5269012540999576", 216621204L)
\end{lstlisting}


\subsection{The eager/lazy evaluation monad\label{subsec:The-eager-lazy-evaluation-monad}}

The monads \lstinline!Reader!, \lstinline!Writer!, and \lstinline!State!
manage extra information about computations. Those monads\textsf{'}
effects can be viewed as working with an extra value of a certain
fixed type. We now turn to monads whose effects are not values but
special strategies of evaluation. 

The first of these monads is called \lstinline!Eval!, and its task
is to encapsulate lazy and eager evaluations into a single type. A
value of type \lstinline!Eval[A]! can be eager (available now) or
lazy (available later). Values of these sub-types can be combined
with correct logic: for instance, a combination of eager and lazy
values automatically becomes lazy.

To derive the type constructor, note that lazy values of type $A$
are equivalent to functions of type $\bbnum 1\rightarrow A$. So,
the disjunctive type $\text{Eval}^{A}\triangleq A+\left(\bbnum 1\rightarrow A\right)$
represents a value that is either lazy or eager:\index{monads!lazy/eager evaluation monad (Eval)@lazy/eager evaluation monad (\texttt{Eval})}
\begin{lstlisting}
sealed trait Eval[A]
final case class Eager[A](x: A)              extends Eval[A]
final case class Lazy[A](lazyX: Unit => A)   extends Eval[A]
\end{lstlisting}
It is useful to have functions converting between eager and lazy values
whenever needed:
\begin{lstlisting}
def get: Eval[A] => A = {
  case Eager(x)    => x
  case Lazy(lazyX) => lazyX(())
}

def now(x: A): Eval[A] = Eager(x)
def later(e: => A): Eval[A] = Lazy(_ => e)
\end{lstlisting}
Now we can implement a \lstinline!flatMap! method that correctly
keeps track of the evaluation strategy:
\begin{lstlisting}
def flatMap[A, B](f: A => Eval[B]): Eval[A] => Eval[B] = {
  case Eager(x)    => f(x)         // This value can be eager or lazy, according to f(x).
  case Lazy(lazyX) => Lazy(_ => get(f(lazyX(()))))  // Call `get` to avoid nested Lazy().
}
\end{lstlisting}
Assuming that \lstinline!map! and \lstinline!flatMap! are defined
as extension methods for \lstinline!Eval!, we may use functor blocks
to combine eager and lazy computations freely:

\begin{wrapfigure}{l}{0.631\columnwidth}%
\vspace{-0.8\baselineskip}
\begin{lstlisting}
val result: Eval[Int] = for {
  x <- later(longComputation1()) // Delay the long computation.
  y <- now(x + 2)       // Short computation, no need to delay.
  z <- later(longComputation2(y * 100))
} yield z
\end{lstlisting}

\vspace{-1\baselineskip}
\end{wrapfigure}%

\noindent The value of \lstinline!result! is a lazy computation because
it involves lazy steps. It is quick to compute \lstinline!result!
because the long computations are not yet started. To extract the
final value of type \lstinline!Int! out of \lstinline!result!, we
need to evaluate \lstinline!get(result)!, which will take a longer
time.

\subsection{The continuation monad\label{subsec:The-continuation-monad}}

The continuation monad is another monadic design pattern that involves
a special evaluation strategy called the \textbf{continuation-passing}
programming style.\index{continuation-passing|textit} In that style,
functions do not return their results directly but instead call an
auxiliary function that consumes the result. The auxiliary function
is called a \textsf{``}continuation\textsf{''} or a \textsf{``}callback\index{callback}\textsf{''}.

To compare the direct style with the continuation-passing style, consider
this calculation:
\begin{lstlisting}
def add3(x: Int): Int = x + 3
def mult4(x: Int): Int = x * 4
val result = add3(mul4(10))     // Will have result == 43 after this.
\end{lstlisting}
Now we add a callback argument to each function and rewrite the code
as:
\begin{lstlisting}
def add3(x: Int)(callback: Int => Unit): Unit = callback(x + 3)
def mult4(x: Int)(callback: Int => Unit): Unit = callback(x * 4)
def result(callback: Int => Unit): Unit = mult4(10)(r => add3(r)(callback))
\end{lstlisting}
To make the pattern more clear, replace the constant \lstinline!10!
by a function \lstinline!pure! with a callback argument:

\begin{wrapfigure}{l}{0.5\columnwidth}%
\vspace{-0.8\baselineskip}
\begin{lstlisting}
def pure(x: Int)(callback: Int => Unit): Unit =
  callback(x)

def result(callback: Int => Unit): Unit =
  pure(10) { x =>
    mult4(x) { y =>
      add3(y) { z =>
        callback(z)
      }
    }
  }
\end{lstlisting}

\vspace{-1\baselineskip}
\end{wrapfigure}%

\noindent This code is a typical pattern of continuation-passing style.
The final result of the calculation is only available as the bound
variable \lstinline!z! in a deeply nested function scope. This makes
working with this code style more difficult: We can continue the program
only by writing code either directly within that deeply nested scope
or within the given \lstinline!callback!. This is the heuristic reason
why callback programming is known as \textsf{``}continuation-passing\textsf{''} style.

Another feature of the continuation-passing style is that callbacks
could be called at a later time, concurrently with the main thread
of computation. To show an example, redefine \lstinline!add3! and
\lstinline!mult4! as:
\begin{lstlisting}
def add3(x: Int)(callback: Int => Unit): Unit = { Future(callback(x + 3)); () }
def mult4(x: Int)(callback: Int => Unit): Unit = { Future(callback(x * 4)); () }
\end{lstlisting}
The new code schedules the calls to \lstinline!callback! on separate
threads. But the type signatures of \lstinline!add3! and \lstinline!mult4!
hide this fact: they just return \lstinline!Unit!. So, the code of
\lstinline!def result(...)! remains unchanged.

The type signature of \lstinline!result! is $\left(\text{Int}\rightarrow\bbnum 1\right)\rightarrow\bbnum 1$,
which shows that it consumes a callback (of type $\text{Int}\rightarrow\bbnum 1$).
A function that consumes a callback is at liberty to call the callback
later, and to call it several times or not at all. The type $\left(\text{Int}\rightarrow\bbnum 1\right)\rightarrow\bbnum 1$
does not show whether the callback will be called; it merely \textsf{``}registers\textsf{''}
the callback for possible later use. This gives us flexibility in
the execution strategy, at the cost of making the code more complicated
to write and to understand.

One complication is that it is difficult to maintain code that contains
deeply nested function scopes. The continuation monad solves this
problem by converting nested function scopes into more easily composable
functor blocks. The code also becomes more readable.

To derive the required type constructor, consider a computation of
type $A\rightarrow B$ that needs to use the continuation-passing
style. Instead of returning a value of type $B$, it registers a callback
function for possible later use and returns a \lstinline!Unit! value.
If the callback has type $B\rightarrow\bbnum 1$, the total input
of our computation is $A\times\left(B\rightarrow\bbnum 1\right)$
while the output is simply $\bbnum 1$. So, the type of a continuation-passing
computation is $A\times\left(B\rightarrow\bbnum 1\right)\rightarrow\bbnum 1$.
Rewrite this type in the form $A\rightarrow L^{B}$ with a suitable
functor $L$:
\[
\left(A\times\left(B\rightarrow\bbnum 1\right)\rightarrow\bbnum 1\right)\cong\left(A\rightarrow\left(B\rightarrow\bbnum 1\right)\rightarrow\bbnum 1\right)=A\rightarrow L^{B}\quad,\quad\text{where}\quad L^{A}\triangleq\left(A\rightarrow\bbnum 1\right)\rightarrow\bbnum 1\quad.
\]
It is sometimes helpful if the callback returns a more informative
value than \lstinline!Unit!. For instance, that value could show
error information or give access to processes that were scheduled
concurrently.

So, we generalize the type constructor $L$ to $\left(A\rightarrow R\right)\rightarrow R$,
where $R$ is a fixed \textsf{``}result\textsf{''} type. This type constructor is
called the \textbf{continuation monad}\index{monads!continuation monad (Cont)@continuation monad (\texttt{Cont})}
and is denoted by $\text{Cont}^{R,A}\triangleq\left(A\rightarrow R\right)\rightarrow R$.

How does the continuation monad make callback-based code composable?
The answer is in the code of the \lstinline!flatMap! method. Its
implementation is (see Exercise~\ref{subsec:ch-Exercise-7}):
\begin{lstlisting}
type Cont[R, A] = (A => R) => R
def flatMap[R, A, B](ca: Cont[R, A])(f: A => Cont[R, B]): Cont[R, B] = { br => ca(a => f(a)(br)) }
\end{lstlisting}
The code of \lstinline!flatMap! substitutes a new callback, \lstinline!br: B => R!,
into the innermost scope of the computation \lstinline!f!. In this
way, we obtain easy access to the innermost callback scope. This trick
makes code with deeply nested callbacks composable.

After defining \lstinline!map! and \lstinline!flatMap! as extension
methods on \lstinline!Cont!, we can rewrite the code above as:
\begin{lstlisting}
def pure[A]: A => Cont[Unit, A] = a => ar => ar(a)
def add3(x: Int): Cont[Unit, Int] = callback => callback(x + 3)
def mult4(x: Int): Cont[Unit, Int] = callback => callback(x * 4)

val result: Cont[Unit, Int] = for {
  x <- pure(10)
  y <- mult4(x)
  z <- add3(y)
} yield z
\end{lstlisting}
This style of code is more readable and easier to modify.

The result of the computation has type $\text{Cont}^{\bbnum 1,\text{Int}}$
(so, it is a function). Like the \lstinline!State! and the \lstinline!Reader!
monads, the continuation monad delays all computations until we apply
a runner.\index{monads!runner} One way of extracting values from
\lstinline!Cont!-monadic programs is to use a runner producing a
\lstinline!Future! value that resolves when the callback is called:

\begin{wrapfigure}{l}{0.515\columnwidth}%
\vspace{-0.8\baselineskip}
\begin{lstlisting}
def runner[A](c: Cont[Unit, A]): Future[A] = {
  val pr = Promise[A]() // scala.concurrent.Promise
  c { a => pr.success(a) }  // Resolve the promise.
  pr.future        // Create a Future from Promise.
}
// Wait for the Future value.
Await.result(runner(result), Duration.Inf)
\end{lstlisting}

\vspace{-1\baselineskip}
\end{wrapfigure}%

\noindent This runner uses special low-level features of the \lstinline!Future!
class, such as a mutable value of type \lstinline!Promise!. If these
features are used in many places in the code, the programmer risks
creating concurrency bugs, such as race conditions or deadlocks, that
are difficult to fix. When using the continuation monad, typically
there will be only one place in the code where a monadic program is
being \textsf{``}run\textsf{''}. In this way, we isolate the high-level business logic
in the \lstinline!Cont! monad from the low-level code of the runners.

We conclude this subsection with some more examples of using the continuation
monad.

\subsubsection{Example \label{subsec:Example-continuation-monad-computation-cost}\ref{subsec:Example-continuation-monad-computation-cost}\index{solved examples}}

Each arithmetic computation, such as \lstinline!add3! or \lstinline!mult4!,
now has a certain cost, which is a value of a monoid type $W$. Use
the monad \lstinline!Cont[W, A]! to implement computations with a
specified cost. The total cost must add up automatically when computations
are chained using \lstinline!flatMap!.

\subparagraph{Solution}

The functions \lstinline!pure!, \lstinline!add3!, and \lstinline!mult4!
need to be redefined with new types:
\begin{lstlisting}
implicit val monoidW: Monoid[W] = ???  // Implement the "cost" monoid here.
def pure[A](a: A): Cont[W, A] = { ar => ar(a) }
def add3(x: Int, cost: W): Cont[W, Int] = { callback => callback(x + 3) |+| cost }
def mult4(x: Int, cost: W): Cont[W, Int] = { callback => callback(x * 4) |+| cost }
\end{lstlisting}
The computation of \lstinline!result! can be now written as:
\begin{lstlisting}
val result: Cont[W, Int] = for {
  x <- pure(10)
  y <- mult4(x, cost1) // Here, cost1 and cost2 are some values of type W.
  z <- add3(y, cost2)
} yield z
\end{lstlisting}
The implementation of \lstinline!runner! now needs to take an initial
cost value. We can generalize the previous code of the runner to an
arbitrary result type $R$:
\begin{lstlisting}
def runCont[R, A](c: Cont[R, A], init: R): (R, Future[A]) = {
  val promise = Promise[A]()
  val res = c {a => promise.success(a); init}    // Resolve the promise and return init.
  (res, promise.future)      // Return the new result value r together with a Future[A].
}
val (totalCost, futureResult) = runCont(result, Monoid[W].empty)
val resultInt = Await.result(futureResult, Duration.Inf)   // Wait for the Future value.
\end{lstlisting}

If the code contains many other monadic operations such as \lstinline!add3!
and \lstinline!mult4!, it is inconvenient to hard-code the cost each
time. Instead, we can easily implement a function that adds a given
cost to any given monadic operation:
\begin{lstlisting}
def addCost[A](c: Cont[W, A], cost: W): Cont[W, A] = { callback => c(callback) |+| cost }
\end{lstlisting}


\subsubsection{Example \label{subsec:Example-continuation-monad-java-api}\ref{subsec:Example-continuation-monad-java-api}}

Convert the callback-based API of \lstinline!java.nio! to the continuation
monad. Read a file into a string, write that string to another file,
and finally read the new file again to verify that the string was
written correctly. Do not implement any error handling.

\subparagraph{Solution}

The\lstinline! java.nio! package provides APIs for asynchronous input/output
using buffers. For instance, the code for reading a file into a string
works by creating a \lstinline!fileChannel! and then calling \lstinline!fileChannel.read!
with a callback (encapsulated by the \lstinline!CompletionHandler!
class). The result of reading the file is available only when the
callback is called, and only within the callback\textsf{'}s scope:
\begin{lstlisting}
import java.nio.ByteBuffer
import java.nio.channels.{AsynchronousFileChannel, CompletionHandler}
import java.nio.file.{Paths, StandardOpenOption => SOO}

val fileChannel = AsynchronousFileChannel.open(Paths.get("sample.txt"), SOO.READ)
val buffer = ByteBuffer.allocate(256)// In our simple example, the file is shorter than 256 bytes.

fileChannel.read(buffer, 0, null, new CompletionHandler[Integer, Object] {
  override def failed(e: Throwable, attachment: Object): Unit = println(s"Error reading file: $e")
  override def completed(byteCount: Integer, attachment: Object): Unit = {
    println(s"Read $byteCount bytes")
    fileChannel.close()
    buffer.rewind()
    buffer.limit(byteCount)
    val data = new String(buffer.array()) // Within this scope, we can work with the obtained data.
  }
}
\end{lstlisting}
Writing data to file is implemented similarly:
\begin{lstlisting}
val outputFileChannel = AsynchronousFileChannel.open(Paths.get("sample2.txt"), SOO.CREATE, SOO.WRITE)
outputFileChannel.write(buffer, 0, null, new CompletionHandler[Integer, Object] {
  override def failed(e: Throwable, attachment: Object): Unit = println(s"Error writing file: $e")
  override def completed(byteCount: Integer, attachment: Object): Unit = {
    println(s"Wrote $byteCount bytes")
    outputFileChannel.close()
    ... // Continue the program within the scope of this callback.
  }
}
\end{lstlisting}
This API forces us to write the business logic of the program in deeply
nested callbacks, since the results of input/output operations are
only available within the callback scopes. The continuation monad
solves this problem by converting all values into the function type
\lstinline!Cont[Unit, A]!. Begin by defining monadic-valued functions
that encapsulate the \lstinline!java.nio! APIs for reading and writing
files:
\begin{lstlisting}
type NioMonad[A] = Cont[Unit, A]
def nioRead(filename: String): NioMonad[ByteBuffer] = { callback =>
  val buffer = ByteBuffer.allocate(256)
  val channel = AsynchronousFileChannel.open(Paths.get(filename), SOO.READ)
  channel.read(buffer, 0, null, new CompletionHandler[Integer, Object] {
    override def failed(e: Throwable, attachment: Object): Unit = println(s"Error reading file: $e")
    override def completed(result: Integer, attachment: Object): Unit = {
      buffer.rewind()
      buffer.limit(result)
      channel.close()
      callback(buffer))
    }
  })
}
def nioWrite(buffer: ByteBuffer, filename: String): NioMonad[Int] = { callback =>
  val channel = AsynchronousFileChannel.open(Paths.get(filename), SOO.CREATE, SOO.WRITE) 
  channel.write(buffer, 0, null, new CompletionHandler[Integer, Object] {
    override def failed(e: Throwable, attachment: Object): Unit = println(s"Error writing file: $e")
    override def completed(result: Integer, attachment: Object): Unit = {
      channel.close()
      callback(result.intValue)
    }
  })
}
\end{lstlisting}
Using these functions, we can implement the required code using a
functor block in the \lstinline!Cont! monad:
\begin{lstlisting}
val filesHaveEqualContent: NioMonad[Boolean] = for {
  buffer1 <- nioRead("sample.txt")
  _       <- nioWrite("sample2.txt")
  buffer2 <- nioRead("sample2.txt")
} yield { new String(buffer1.array()) == new String(buffer2.array()) }
\end{lstlisting}
The code has become significantly easier to work with, as its high-level
logic is clearly displayed.

Since the input/output operations are run concurrently, the value
of type \lstinline!NioMonad[Boolean]! is a function that will compute
its \lstinline!Boolean! result at some time in the future. We can
use the runner shown above to wait for that value to become available:
\begin{lstlisting}
val result: Future[Boolean] = runner(filesHaveEqualContent)

scala> Await.result(result, Duration.Inf)
res0: Boolean = true
\end{lstlisting}


\subsection{Exercises\index{exercises}}

\subsubsection{Exercise \label{subsec:Exercise-monads-p1}\ref{subsec:Exercise-monads-p1}}

For a given set of type \lstinline!Set[Int]!, compute all subsets
$\left(w,x,y,z\right)$ of size 4 such that $w<x<y<z$ and $w+z=x+y$.
(The values $w$, $x$, $y$, $z$ must be all different.)

\subsubsection{Exercise \label{subsec:Exercise-monads-p1-1}\ref{subsec:Exercise-monads-p1-1}}

Given 3 sequences $xs$, $ys$, $zs$ of type \lstinline!Seq[Int]!,
compute all tuples $\left(x,y,z\right)$ such that $x\in xs$, $y\in ys$,
$z\in zs$ and $x<y<z$ and $x+y+z<10$.

\subsubsection{Exercise \label{subsec:Exercise-monads-p1-2}\ref{subsec:Exercise-monads-p1-2}{*}}

Solve the $n$-queens problem on an $3\times3\times3$ cube.

\subsubsection{Exercise \label{subsec:Exercise-monads-p1-4}\ref{subsec:Exercise-monads-p1-4}}

Read a file into a string and write it to another file using Java
\lstinline!Files! and \lstinline!Paths! API. Use \lstinline!Try!
and \lstinline!for!/\lstinline!yield! to make that API composable
and safe with respect to exceptions.

\subsubsection{Exercise \label{subsec:Exercise-monads-p1-3}\ref{subsec:Exercise-monads-p1-3}}

Write a tiny library for arithmetic using \lstinline!Future!s, implementing
the functions:
\begin{lstlisting}
def const(implicit ec: ExecutionContext): Int => Future[Int] = ???
def add(x: Int)(implicit ec: ExecutionContext): Int => Future[Int] = ???
def isEqual(x: Int)(implicit ec: ExecutionContext): Int => Future[Boolean] = ??? 
\end{lstlisting}
Use these functions to write a functor block (\lstinline!for!/\lstinline!yield!)
program that computes $1+2+...+100$ via a parallel computation and
verifies that the result is correct.

\subsubsection{Exercise \label{subsec:Exercise-monads-p1-5}\ref{subsec:Exercise-monads-p1-5}}

Given a semigroup $W$, make a semimonad out of the functor $F^{A}\triangleq E\rightarrow A\times W$.

\subsubsection{Exercise \label{subsec:Exercise-monads-p1-6}\ref{subsec:Exercise-monads-p1-6}}

Implement \lstinline!map! and \lstinline!flatMap! for the tree-like
functor $F^{A}\triangleq A+A\times A+F^{A}+F^{A}\times F^{A}$.

\subsubsection{Exercise \label{subsec:Exercise-monads-p1-7}\ref{subsec:Exercise-monads-p1-7}{*}}

Find the largest prime number below $1000$ via a simple sieve of
Eratosthenes.\footnote{See \texttt{\href{https://en.wikipedia.org/wiki/Sieve_of_Eratosthenes}{https://en.wikipedia.org/wiki/Sieve\_of\_Eratosthenes}}}
Use the \lstinline!State[S, Int]! monad with \lstinline!S = Array[Boolean]!.

\begin{comment}
in this part of the tutorial I will talk about Mona\textsf{'}s and semi Mona\textsf{'}s
this continues a consideration of how we can do computations in a
functor context or in Scala this is the functional block for the for
yield block in this part I will concentrate on practical issues the
first example we will use is this computation in this computation
we need nested iterations in the factor block nested iterations are
expressed with several left arrows or they're called generator arrows
so the program for this computation can look like this from the left
I show it in the filter block syntax on each line as you see the generator
line gives you an iteration I goes over this sequence from 1 to M
J also goes over the same sequence and K those over the same sequence
and they all go independently so I for each I and J goes over all
these and for each I and J K goes over all of this and in the yield
expression we compute this function f that we're supposed to compute
here and the result of this sub expression up to this parenthesis
is a large list of all values of F for each choice of I J and K and
then we take a sum of this large list so that computes the sum in
this expression if you replace the left arrows as the Scala compiler
does through map and flatmap and you will see that the map replaces
the last left arrow and flatmap replaces all other flap left arrows
or generator errors and so line for line translation of this code
could look like this so instead of this syntax we have this instead
of this we have this and so on and you see how this works so the last
expression is just a simple map because we just need to compute the
value F for each of these case but this result is a list and soldered
argument a flat map is a function that takes J and returns a list
the result of the flat map is again a list which is this one up to
this brace and this is the body of this function taking I as the argument
and returning this list so again the argument of flat map as a function
from a value to a list so this syntax is available for the sequence
factor because sequence has flat map defined in addition to map if
that function has a filter defined where the function method is called
with filter as we have seen in the previous tutorial then we can use
the if lines in the function block as well in this tutorial we will
occasionally see the if lines but we'll concentrate on what happens
when you have several left arrows it is named flat map because it
is actually equivalent to us first doing a map and then doing a flattened
how would that work well in this this code were the function in the
brackets or in the parentheses here takes J and returns a list the
result of that would be a list of Lists and when you flatten that
you get a simple list and that is the same as doing a flat map so
factors that have flat map or and flatten defining them I call them
platinum or semi monads this is my own terminology flattened able
and semi mu nuts is my terminology there is no accepted name for these
factors you must have heard of monent now monads are more than just
founders at her flat map or flattened monads also needed an additional
method called pure which has this type signature however this method
cannot be used in the function block directly so it does not correspond
to a specific construction of a functor block of course let me correct
myself it can be used in the function block as any other method but
it has to be used here on the right-hand side because here this an
arbitrary Scala code here so you can use any methods you want but
it is not special to the function block it does not have a special
meaning and also we will not need this method much in fact it is not
not very often used other methods are more important than pure and
so mathematically and adding the meta pure makes an interesting mathematical
structure but for the practical use that I'm going to talk about this
is not very important so full monitor monads are factors that have
flatmap and pure with appropriate laws which I will talk about in
another part of this tutorials semi monads are those that just have
flat map and they may or may not have pure in many cases they will
also have some natural definition of pure but in some cases they won't
so let\textsf{'}s concentrate therefore one semi Mona\textsf{'}s monads that don't have
pure necessarily let\textsf{'}s look at more a visual example of how a flat
map works with lists so consider this expression how we would compute
that expression so let\textsf{'}s assume that the function f takes a value
of x and returns a list of some values of type y now if you have specific
values x1 x2 x3 here then the result of applying F to them might be
a different list each time it could be even a list of different lengths
with different values inside so let\textsf{'}s imagine this is the result of
applying f2 x1 x2 and x3 so the flat map will put all of these values
together in one list we first do melt and then we do flatten and what
happens is that the map will replace each of these x ones with its
corresponding list x1 x2 x3 will be replaced by these three lists
so that would be a list of lists of Y and you flatten that and you
get a simple or flat list let\textsf{'}s develop some more intuition about
what happens with data in a factory or in a collection or in a container
when we use flatmap or when we use several generator errors in the
factor block so here is a schematic example of some code that tells
you that I goes over this collection or container or sequence in this
case J goes over here then we compute some X as a function of I J
now at this point we have computed a whole sequence of X\textsf{'}s for each
I and J different experts then we for each of those X\textsf{'}s we still have
another it nested iteration and then we compute some Y as a function
of I J and K and actually at this point we have computed a large list
of Y\textsf{'}s of perhaps different values of Y different for each I J and
K and then we compute another function H of x and y so this entire
result will be a long list of values of this H computed for each high
J and K a different value of H so this code is a translation into
non hunkler block syntax so you see line 4 line same thing 1 to M
1 to M I I 1 to N 1 to N J J so this is here x equals this here we
need to say well in the function block we don't say well that\textsf{'}s the
only syntactic difference so 1 2 P 1 2 P ok ok so here we having this
line by line translation this code computes exactly the same value
of the result which is going to be a list of so one thing we notice
is that every line that is a generative line that is every line it
has a left arrow must have the same type of container on the right
hand side so this is a list or sequence in general in this case this
is going to be subclasses of sequence some vector or something like
that each generator line needs to have the same type we could not
for example here use sequence and here use some completely different
containers such as let\textsf{'}s say tree or some point that we couldn't do
this must be the same type and this is so because flatmap is defined
like this flat map requires that this sequence type is the same as
the sequence type returned by this function which is all of that so
all of that should be a sequence of the same type as this and so for
this reason each container or sequencer or collection on the right
hand side of a generator arrow must have the same type same container
type another thing we know is that each generator line actually starts
an expression so if you look at this generator line for example DJ
the translation of this actually is this J goes to that so this entire
thing is a an expression that evaluates to to a list to a sequence
which is again a container of the same type so you can think about
this syntax as nested computation where each line starts in new computation
yielding finally the value of the same type let\textsf{'}s now look also at
the number of resulting data items now I goes from 1 to N J goes from
1 to N K goes from 1 to P so in this case clearly we have M times
n times P different elements in the resulting list here so this will
be M times n times P values in the result if we had some other code
here for example if this were not from 1 to n always but the length
of this sequence were somehow a function of this value 1 I let\textsf{'}s say
now we would probably have the different number of data items depending
on the data or we could have fewer data items because we could have
an if line which would filter out some of the data so we'll have less
than M times n times P but in any case we could have up to M times
n times P resulting data items in their container so the container
type that we need for this kind of computation is at least such that
it can hold m times n times P items if it can hold em items for n
items for p- so if we ask the question what kind of containers can
have flatmap in other words what kind of containers will fit this
style of computation the answer is at least this type of containers
must be such that if the container can hold em items of data and it
can hold end items of data it must also be able to hold at least m
times n items of data so the capacity of the container must be closed
under multiplication always the set of all possible capacities or
capacity counts of the container must be closed under multiplication
this is an interesting property with container types have this property
so for instance a sequence or a non-empty list these containers can
hold any number of items in the case of non-empty lists any number
of items that\textsf{'}s at least one so that set is a set of all integers
at at least some minimum that set is certainly closed under multiplication
and also well it\textsf{'}s not just multiplication must be closed under this
it must be if it has m elements and it has n elements it should be
able to hold all less than M times n elements any number of pretty
much elements less than that so obviously sequence and non-empty lists
are such containers with this property another important example of
such containers is the container that can hold only 0 or 1 elements
for example option is such a container option can hold one data port
it could be empty either is another container data item have some
error message thing try is another such container it could have a
date I could have an exception future is another such container it
holds the data item that is going to be available in the future or
it could fail also the computer so these I call the pass fail containers
these could hold at most one data item and clearly the set of capacities
for this container is a set of 2 L two numbers 0 and 1 and this set
is closed under multiplication another example of a container that
would have this property is a tree light container that must have
for instance three six nine twelve and so on elements that are always
multiples of three let\textsf{'}s say yeah so it\textsf{'}s bit branches in such a way
that it can always hold can only hold the number of items that\textsf{'}s multiplied
that smell a multiple of three and obviously this kind of set is closed
under multiplication finally there are several containers that I will
also talk about which are which I call non-standard examples of such
containers are these so these are functors because here a is a covariant
position here also a is in a covariant position because it\textsf{'}s behind
two arrows two functionaries so these are factors but they are not
really containers with data in an ordinary sense there\textsf{'}s no way in
which you can say they hold five or eleven items of data or something
like this they hold data in some non-standard way and we will talk
about the usage of these containers and give examples but these also
in certain cases certain function types like these also can have flat
map defined on them in a reasonable way so let\textsf{'}s look at examples
now the first set of examples are what I call a list like monads so
by the way all these examples here are monads are not just semi Munez
but we will just look at the flat map and so we will not use the fact
that there are full moon ads so for us it\textsf{'}s not really important to
make that distinction right now we only focus on flat map so what
are the typical tasks that list like monads before typical is make
a list of combinations or permutations to go over these filter out
what you want and get a list of results that\textsf{'}s a typical computation
of this kind another situation is that if you have some problem that
has many possible solutions you organize these solutions in a solution
tree and you traverse this tree with say a recursive depth first search
and then you again filter out solutions that are incorrect and you
get your resulting solutions now the containers that have this property
it may be eager sequence so eager means all elements of the sequence
are computed upfront before you can use the sequence for they can
be lazy which also iterator is one example is a sequence whose elements
are not all computed upfront you can already start using the iterator
and as you need it will compute new elements so that\textsf{'}s called lazy
computation stream is another the data type in the Scala standard
library that computes lazy value so it does not compute upfront all
the values in the stream computes them when you need to do so once
you have confused them they are stored in memory but until then they
are not so these are eager and lazy evaluation strategies but the
way your write code is very similar you just have flat map defined
on iterator you have flat map defined on stream and so you can just
use them usually list like containers have a lot of additional methods
defined in them it\textsf{'}s a very rich data structure so they're monads
they have pure metal defining them I just remind you that the pure
method has this type so it just takes one data item and creates a
container that has this one little item inside so clearly you can
put one item in the list or an iterator on the stream and so on comes
clear but many additional methods are used such as appending lists
pretending elements to list concatenating lists fill so you make a
list that has certain pre computed elements fold scan those are methods
are you are all defined on this list like Mona\textsf{'}s so actually program
code mostly uses methods like these and not pure so it is not very
important for programming for the practice of programming to have
the method viewer defined list like Mona and actually it\textsf{'}s the same
for most moments the method pure is quite secondary in its importance
for practical programming so let\textsf{'}s look at some example so I have
prepared working code in the repository so let\textsf{'}s take a look so here
is the first example how we compute things in the function block what
we want is to compute a {[}Music{]} multiplication table so the results
should be a sequence of strings such as these so these strings need
to be computed in this order so notice that the multiplication table
is only half of the matrix its we never compute for example three
times two we already have two times three and sufficient and so to
organize this kind of computation we write this so I goes from 1 to
5 but J goes from I to 5 so this is an example of the collection on
in the right hand side of the generator line and the collection is
a function of the value that we defined in the previous line so this
collection is a function of I so this is a example of having a different
list you know each time for each I will be a different list and that\textsf{'}s
fine this is just as easy to do that using the function lock with
mu naught so what do we yield in other words for each I and for each
J what do we compute or we compute the product of I and J and then
we print this line I times J equals product so that will produce a
string and the whole result will be a sequence of these strings so
we have a flat sequence so to speak remember this heuristic or a mnemonic
choice of word of the name flat map so we we have many nested iterations
here when the result is a flat sequence so this test verifies that
the result is this multiplication table here is how we can do the
same using a filter so here we do not make this second sequence depending
on their both same but then we filter out by the condition that J
must be not less than I so then we we can compute product here that\textsf{'}s
completely equivalent in terms of results and we yield this listener
perhaps slightly different and maybe easier to understand way of doing
the same thing so it still computes the same multiplication table
as we had here I already ran all these tests they all pass so you
I encourage you to download this and run as well and play with it
to save time I will not run tests but they all pass another important
thing here is to notice that if one of the generator arrows returns
you an empty sequence then remember this size of the result is a product
of sizes of these sequences and the sequences of length zero and so
the resulting sequence will be of length zero it will be empty so
just having one of these generators produce an empty sequence will
kill the entire computation it will make an empty sequence out of
the entire computation regardless of what you do here before or after
it will just completely collapse everything into an empty sequence
that is also an important property of the magnetic computation so
let\textsf{'}s now go through our worked examples in that slide the first example
is to compute all permutations of the sequence of three strings a
B and C so how do we compute that let\textsf{'}s just write code like this
so first we define the sequence excess or excess now obviously we
need to iterate over it in some way to get the permutation so let\textsf{'}s
do that so let\textsf{'}s say X goes over all this now once X let\textsf{'}s say is
a we need to go over the result over the remaining elements so let\textsf{'}s
compute the remaining elements here so diff is the library function
that computes the difference between two sequences or it removes elements
from the second elements that are in the second sequence are removed
from the first sequence when you do div and so the remain will be
BC for instance if X is a then we go over the remain and we find and
next remain so remain tube which is now all the rest and Z goes over
remain two so now we know that it\textsf{'}s a sequence of three elements so
we know that remain two is going to have just one element left so
that\textsf{'}s what we want and the result is we yield the sequence of XYZ
and so this is going to be computed for each choice of XY and Z so
the result will be a sequence of sequences so let\textsf{'}s check that control
shift right so that\textsf{'}s a sequence of sequence of string that\textsf{'}s the
result we check on this test that the permutations are correctly computed
the standard library in Scala already has the permutations function
just that the permutations function returns an iterator not a sequence
so we converted to see in order to run this test if we didn't convert
it to sequence and we couldn't compare our sequence of sequences against
an iterator the iterator doesn't have the values computed yet cannot
compare something that already has all the values computed and something
that doesn't we need to run that to them to compute all the values
so that\textsf{'}s what the two sequence does for me it\textsf{'}s very careful the
second example is to compute all subsets of this set now subsets of
the set is not the same at all as permutations of a sequence for instance
empty set would be a subset or a set of B alone would be a subset
or a set of B and C would be a subset but there is no difference between
the set of B and C and a set of C and B so all subsets is very different
from all permutations let\textsf{'}s see how we can do that so let\textsf{'}s think
about it so first the subsets could be empty so we need to allow empty
set the subsets can be also non empty so let\textsf{'}s allow that as well
so we we say X a would be of type set of string and so X a would go
over either an empty set or a set of a so in this generator line the
right-hand side is a collection or our in this case it is a set and
the set contains two elements an empty set of strings and a set of
single-a so exhale be  empty set or set of a xB would be similarly
either an empty set or set of B XC will be in either in the second
set of C and then we will concatenate all these sets now when we concatenate
sets equal sets would collapse into one and so the result would be
empty set will be present once then we'll be set of a together maybe
with empty or together a set of B so in this way we have eight combinations
that we need so either each of the X a xB XC goes over to possible
values and so the result is 8 X or two times two times two different
sets so indeed we have a standard library function subsets that returns
an iterator we convert that iterator to set and the result is correct
so it\textsf{'}s all subsets of this next example is to compute all sub sequences
of length 3 out of a given sequence so sub sequences are not necessarily
elements or they're next to each other in sequence but they must be
in the water so let\textsf{'}s look how we can do this so here\textsf{'}s an example
we have a sequence from 1 to 5 and the sub sequences of length 3 out
of this are listed here so it\textsf{'}s one two three one two four one two
five one three four and so on so how do we compute such a sub sequence
well we start reasoning by what kind of generator lines we should
write and what kind of filtering we should do that\textsf{'}s the way that
this kind of code is written we use the library function tails which
is a useful function saves us work what does it do it takes a sequence
and it computes a sequence of first initial one then the tail of the
sequence have a tail of the tail and the tail of that and so on until
we get an empty sequence so XS is going to be a sequence of these
ranges we filter out the new and then we know then it\textsf{'}s that it\textsf{'}s
not empty actually we filter out not the new we filter out the case
when this entire tails is empty because it could happen what if we're
given an empty sequence here then this entire tails sequence will
be empty so we filter out non-empty we take the head and that\textsf{'}s the
first sequence here and then we take the tail of that which so the
tail of excess is these and that\textsf{'}s our remain one so now we go over
the sequences in these and in the tail of XS so if let\textsf{'}s say first
of all would be DS again we filter out as its it must be non-empty
take the head and then take the tail of that and this what this does
is that in the next iteration we would have x1 here why being here
for example and so that\textsf{'}s how we find all the possible sub sequences
so we can take this this and this and taking heads of those would
give us 1 3 4 4 1 3 5 and so on so this is the way we induce the sub
sequences this is a bit manual there\textsf{'}s a lot of boilerplate code but
this is a code that kind of is obvious obviously correct we take details
then other tails and so on and it works as expected so you see we
have used if we have used non generator lines or the computation lines
which are map these are flat map this is a filter so we use all the
features of the list as we should we should always use all the features
if they are helpful so this code computes the sequences we expect
the next task is to generalize all these examples to support arbitrary
length instead of three so here we had hard-coded length in hard-coded
length 3 now we want to generalize and this is of course a little
more challenging but notice in all these code examples we had really
hard-coded the fact that we are looking at length 3 let\textsf{'}s see how
we can generalize this it\textsf{'}s not a lot of work it\textsf{'}s just needs to be
a little more clever so let\textsf{'}s look at the first example the first
example is permutations well obviously we still do the same thing
as before let\textsf{'}s take our code for permutations see what we do so we
we do X going through the sequence then we compute the remain the
remaining subsequence or part of the sequence by removing the X we
just chose so the result is a smaller subsequence and let\textsf{'}s just use
the same function to compute the permutations of that recursively
so that\textsf{'}s the idea here so we take X going to it over over all the
XS compute the remain and then the Y\textsf{'}s is going to go over all the
permutations of remain and that is a recursive call to the same function
so Y is going to be a sequence of elements because the function returns
a sequence of sequence so since on the right hand side of the generating
line we here have a sequence of sequences the left-hand side will
become a sequence now notice the generator light has to be the same
container type sequence doesn't have to be the same element type so
in the first generator line the container type is sequence the element
type is a and here the container type is sequence the element type
is sequence of a so that\textsf{'}s fine as long as the outer container type
stays the same now this variable is going to be sequence of a so we
need to append X which is the first element which shows and the permutations
of the other elements so the result is that we yield at this sequence
and the total result is going to be Kwan\textsf{'}s of these sequences for
all X\textsf{'}s and for all permutations on the remainder and that\textsf{'}s what
we need except you have to add this check at the very beginning because
we will eventually call the permutations on an empty sequence here
and we need to not break when we when we have that so that\textsf{'}s how it
works and now we can have any length and it works so let\textsf{'}s see how
example 2 is generalized we look at the code and example 2 we see
we need to basically repeat this n times we repeated this line three
times here but now we need to repeat this n times instead of three
times so how do we do that we use recursion of course so the first
line we can continue and as before then we do the remain which is
also the - here is the operation defined on set that removes elements
from a set well it\textsf{'}s not it doesn't really modify this set it\textsf{'}s just
compute a new set that has one element viewer possibly then we do
the recursive call of subsets on the remain and gives us a bunch of
sets so for each of those sets we have a wide age and we concatenate
the sets so it\textsf{'}s very similar thank very similar idea so we we do
one step that we did before and then we call the cursor away on the
remainder and that works for any length of the set example 3 is a
bit more ago but that\textsf{'}s exactly the same procedure in example three
we had this kind of code repeated three times ex going to seek on
something tails if non-empty get the head and compute the remained
take the tails if not everything at the head compute tail yet tails
of non empty so that is going to be repeated so we're going to write
this once and then do the recursive call so that\textsf{'}s how this works
in generalizing our example three so again we need to check for empty
sequences and that\textsf{'}s a little involved I won't go into details here
encourage you to look at it yourself the main computation is here
notice I have to put parentheses around the four because I needed
to seek on it and you cannot just do it to seek after this brace that
is not right syntax this brace is part of the yield expression and
so if you wanted to seek on the entire it\textsf{'}s for yield block you have
to put parentheses around it alright so this is our code that we have
to do go to tails non-empty get head and find remainder and then we
do a recursive call now notice recursive call is on n minus one because
now the task is the compute n element sub sequences of a given sequence
now we compute n minus 1 element sub sequences and sometimes this
n would be 0 sometimes the remain will be empty so we need to check
both of these cases separately so that\textsf{'}s how it works exactly the
same test passes next example is the well-known 8 Queens problem 8
Queens is a chess Queens on the chess board and you need to find all
locations for the 8 Queens so that they don't threaten each other
the Queens on the chess board threaten each other if they're in the
same row in the same column or on the same diagonal so let\textsf{'}s first
write a function that finds out whether Queens threaten each other
so on the chess board it is clear that each Queen must be in a different
room if any of the two queens are in the same row they they threaten
Charlie so we'll just take a shortcut and the Queen if every Queen
will be in the next row always and the only question is in which column
it is so the integer coordinates here is a column and these are columns
and we assume that they're all in consecutive rows so this function
will compute the condition that some Queen in this position in this
column is not threatened by any of the previous Queens given in the
previous rows with columns specified here so that is when X is not
the same as other X so they're not in the same column and when they're
not in the same diagonal so diagonals are computed by differences
row minus column row - come on your plus column so these are the two
diagonals alright so now how do we find all solutions of the hit Queens
problem so here\textsf{'}s a straightforward just to be quicker let\textsf{'}s say the
row is the set of indices 0 1 2 and so etcetera 7 now x1 is the column
of the first or maybe row of that says I want to say so x1 is the
column of the first clean x2 is a column of the second Queen now we
need to check that the second queen is not threatened or does not
threaten the first so then we iterate over the column for the third
queen and then we check the threat that the third queen is not threatened
by the first we already know that the first two don't threaten each
other so we all need to check that the additional Queen does not threaten
the previous columns and so on at each step we check that the additional
Queen does not threaten the previous Queens and the previous Queens
already find so this is the entire code and then we output the columns
for the Queens that we found the result is going to be a sequence
of sequence of integer because we yield sequence so each of our containers
is a sequence so all the types here are sequence and the result is
also a sequence but the type of element is different is sequence of
integer no and we just check that there are 92 solutions we know that
there are 92 solutions now let\textsf{'}s generalize this example to solve
any Queens problem that is n by n board with n Queens and we do the
same thing you notice here we had hard-coded their eight Queens and
all this code has to be now generalized which is done in the same
way as before by introducing recursion so how do we do that so again
let\textsf{'}s rename this to column because I prefer to think about this this
column actually so we define a function that is going to be the recursive
function that adds another queen we have a previous Queens and it
finds all the possible ways of adding another queen and so that\textsf{'}s
going to be our solution so n Queens is going to be {[}Music{]} :
n Queens partial on the required number of Queens and initially we
have none no Queens already selected so this function says I want
to add this many Queens and here are the initial here are the previously
selected kunas so we do the same as we did before so X is going to
be the column of the next queen then we check that the next queen
does not threaten previous Queens then we find the sequence of the
new Queens and call so these are the newly selected Queens after the
filter line so after this line we are sure that the X is an admissible
column for the new queen so this is going to be the sequence of the
Queen we found and we call the same function recursively so now we
need to add n minus 1 new Queens and here are the queens we found
so far so rest is going to be a sequence of integer and we append
that sequence to the X the X is being the queen we found and then
we're done and so to verify that this is correct I run the test for
eight nine ten and eleven Queens and check the links and it is known
how many solutions there must be look at this page here that I found
nobody really knows how to compute these numbers without numerating
all the Queen positions these numbers seem to be very hard to predict
otherwise so this is how we solve problems like permutations and traversing
a solution tree and filtering out solutions that are undesirable in
some way and finding the list of all solutions notice just a comment
here in all these examples functions that are in person are not actually
tail recursive because the recursive call occurs in the for yield
block and it\textsf{'}s in other words inside some deeply nested flat map somewhere
all these are translated into nested flat maps so if you look at this
also this is a recursive call you see this little symbol here so that
is the IntelliJ telling me that it\textsf{'}s recursive method but the recursion
is not tail recursion occurs on the right hand side here and so there
is some more computation that is being done after this call and so
it\textsf{'}s not a tail recursive call it\textsf{'}s not the last computation being
done the tail recursive call would have been if the result returned
by this function is the result that the entire call he returns but
that is not so after this whole there is more computation to be done
so this is just a short comment here that these non tail recursive
functions are certainly not great in terms of safety because if for
any reason you need a large number of recursive calls that will blow
up the stack give you a stack overflow exception there are ways of
making this stack safe but this is out of scope for this tutorial
right now we will talk about it in a later tutorial how to make monadic
recursion Starke safe it is slightly more involved than usual recursion
because the recursive calls a curve in a freon block under magnetic
flatmap context so that would be I will discuss in a different tutorial
so the last example for the list like Munez is a slightly different
problem that I found quite interesting I worked on it some time ago
when I was implementing and another open source project I found that
I have to transform boolean formulas from distributed normal form
and token sorry from disjunctive normal form into conjunctive normal
form so CNF is conjunctive normal form and I found that this transformation
is very simple if you formulate it in terms of the West model so let
me show you the code it\textsf{'}s really I'll explain now what it means to
transform between these normal forms in case you're not familiar with
the boolean logic it\textsf{'}s not difficult at all it\textsf{'}s just terminology
so what does it mean the conjunctive normal form so it\textsf{'}s all about
boolean formulas like this one so we have boolean operations or and
sorry and so these are boolean operations and the boolean formula
in general can have any combination of these boolean operations now
we say that the conjunctive normal form is when the formula has the
the shape has already it has some parentheses that are connected with
and and inside the parenthesis there all only or so there is nothing
no end is allowed inside the parenthesis no or is allowed outside
the parentheses so that is how we define the conjunctive normal form
and disjunctive normal form is the opposite inside the parentheses
the and only is allowed outside the parentheses the or only is the
left so these are these the normal forms and why are they important
at all the reason is any boolean formula can be transformed into one
of these forms into both actually whatever you want that the reason
is there are boolean identities for example the or and the end operations
are distributive so you can you can expand the brackets or the parentheses
so for instance if you imagine let me let me just make a little comment
here so that it is more illustrative so this imagine that you replace
or with a plus and you replace and with x so that becomes an algebraic
expression that you can transform by expanding the parentheses like
an ordinary algebra and then you can again replace the start of the
multiplication with end and the war with sorry the plus with an or
and you put parentheses around this and then you have a valid transformation
of william formulas so this is you see on the Left we have conjunctive
normal form because on the left hand side we have only our operations
inside parentheses which are disjunctions and conjunctions which are
and operations are outside so the simple expanding of brackets or
expanding of parentheses in the sense of ordinary algebra is what
transforms one of these normal forms into another so in this example
we will implement these transformations will implement the transformation
left responds to expanding brackets as a symbolic computation in order
to do this we need to represent the formulas in some way symbolically
so we will do this by using sets so let\textsf{'}s use a type parameter T as
a type of represents individual prepositions in the boolean formula
we won't do anything with that value it will be symbolic for this
reason just manipulate sets of these values of type T and so our in
our representation the normal form already forces us to have this
structure that there there is a one or more actually zero or more
parentheses and outside it\textsf{'}s always the boolean and and inside is
always boolean over so all we need to say is there is a there is this
set that has a and B in it and there is this set that has CD in it
the water of course is immaterial because any of these designs be
Jorge so sets are sufficient to represent this and then we have a
set of these two so the outer set will be impossible the boolean conjunction
and between these and the inner set will be implicitly the boolean
disjunction between lives so that is going to be our short representation
of the boolean formula so now let\textsf{'}s just briefly consider what would
be the true and false values in this representation the value true
is the empty conjunction which is a conjunction of no parentheses
and that\textsf{'}s empty set the value false is an empty disjunction which
means we do have one set of parentheses but there is nothing inside
it so that\textsf{'}s a set of a single empty set now the disjunctive normal
form has the same representation in terms of data type it still sets
of set of sets of T but it just that the other said now is the disjunction
and the inner set is conjunction so because of this the true and false
are represented in the opposite way so the true for this DN F is the
set of empty set its if you are wondering why is it that empty conjunction
is true while conjunctions are abundant so if you have some non empty
conjunction and you imagine that you have a conjunction of that with
an empty conjunction that shouldn't change anything and so that\textsf{'}s
why the the empty conjunction must be true because the conjunction
of true and X is X the same as with false false is the empty element
for the disjunction 108 so that\textsf{'}s why false is the empty disjunction
so now as I just showed you it is easy to convert one end to the other
you just need to expand brackets so let\textsf{'}s see how we can expand brackets
so let\textsf{'}s just define these types type constructors so that we can
distinguish them more easily the equality we just define this for
convenience to compare we want to run tests and want to convert one
to the other and compare results so these are our presentations of
true and false as discussed so let\textsf{'}s think about how we can expand
parentheses or brackets here for example so we have a set of sets
and we need to prepare a set of these sets of them so in order to
do this transformation using a functor block let\textsf{'}s go like this so
X goes over this set Y goes over that set and then we need a set of
all sets that has one X out of here and one way out of here right
okay see a deep B C B so that is easily accomplished we have X from
the first set a quiet from the second set we just make a set that
cascada means x and y so that\textsf{'}s going to be the result so if we just
had two sets of parenthesis then we would write code like this so
now we need to generalize this code so that it is applicable to any
number of parenthesis not just two and we do it the same way that
that we generalized before we write this code once and we use a recursive
call so let\textsf{'}s do that so the the trick I'm using is that I need to
check that the set is empty actually I I can check that I can do in
the previous code examples check that it\textsf{'}s empty and if not take the
head a slightly more visual and clear way of writing the same code
is to use head option so head option is defined on the set remember
V is the name of our data element inside the case class so that\textsf{'}s
just our set of sets we take a head option the head option is going
to be a set or other option of a set and we match that if there is
nothing that means we have a false so we could have put this CNF false
here actually just to be more visual then we have the case when it\textsf{'}s
not empty so then we have the first Clause we need to let X go over
the first Clause and then we need to do a recursive call on the rest
so in this example over just two we need to go over the second one
but actually in a recursive call with will have more than one y in
here so let\textsf{'}s take all of these waters is going to be then set Y are
all other terms that are connect converted to cmf by the recursive
call and so now we have an X which was chosen from the first set of
parentheses and all the Y\textsf{'}s is the rest of the CNF that was computed
by the recursive call so now we just need to concatenate these sets
and that\textsf{'}s the result and that\textsf{'}s actually the entire code so that\textsf{'}s
very simple in order to run we need to actually simplify things because
it turns out that simply expanding brackets will not produce results
that are identical here\textsf{'}s an example if you have this kind of thing
this kind of boolean formula it\textsf{'}s actually the same as this boolean
formula because this entire formula is true only when this is true
when this is true then here we have true we don't need to compute
anything else here I already have true in this set of parentheses
inside here so this can be just ignored this can be simplified away
it is unnecessary to compute any of this so this simplification can
be just made by by saying well is there any clause well these are
called closes is there any clause that is a subset of another Clause
so for example this Clause is a subset of this one if so then we can
ignore this larger clause and this is what this code does it\textsf{'}s sort
the closes by size and then it finds what are the closes that our
subsets of another and then if so we ignore the larger Clause which
is to the previous and if not we don't ignore it so this is the code
that simplifies using a fold and these are tests so for instance this
is the example and actually another interesting property is that this
function is its own inverse if we convert from DNF to CNF and then
we want to convert it back we can use the same function to convert
back this is so because expanding brackets is an operation doesn't
really depend on what operations are in and out of the brackets as
long as they are distributive and they are distributed in the boolean
logic in both directions and so it\textsf{'}s the same operation the enough
to CNF it\textsf{'}s the same as CNF to DNF it\textsf{'}s its own inverse and the test
verifies that go dnf2 CNF and then we first we convert this and then
we can write back and it\textsf{'}s the same thing I would like to add some
more examples of linear algebra manipulations and to illustrate certain
properties of the function block these are examples taken from the
standard Scala library documentation I rewrote them to be slightly
more functional clear so the first example is the computer transpose
of a matrix so in the matrix is represented as a sequence of sequences
and the result is again sequence of sequences so how do we transpose
a matrix it\textsf{'}s actually not so simple because for instance the first
sequence in the result must be the sequence of all first elements
of the sequences listed here the second sequence in this result is
the sequence of all second elements of these sequences so how do we
get all second elements of sequences we need to know the index 0 1
2 and take that by index so here\textsf{'}s what we do we define this index
as an iteration here going over all indices of the first sequence
in this sequence of sequences so indices and a standard library function
it returns a range such as zero to zero until something zero until
lengths so then once we go over these indices what do we healed we
yield another four expression it seems in other words we yield a sequence
that is computed in a different way how is it computed we need for
example here the first sequence we need to return is a sequence of
elements at index 0 so that\textsf{'}s what we return return I think the name
is back to high perhaps when I hear 0 that will return a sequence
because the for yield returns a sequence of whatever you yield here
a sequence of 0 of elements from X s where access goes over all of
these so in this way we'll return a sequence of 0 of elements of these
sequence of set first elements second elements and so on as I goes
over an indices so you see in order to do this we had to put a 4 inside
of a yield and the result is not easy to read this is a bit of a complication
so nested fours either inside of a yield or inside here you could
put a for else inside here it\textsf{'}s harder to read and probably easier
to refactor in some other way so for instance to make this function
depends on I and XS when they put that function here and make it more
clear what exactly is being computed but that is a core code that
and that can be refactored when necessary for clarity that\textsf{'}s how it
works transposes this into this the second example is to compute a
scalar product of two vectors so we use a zip function we have two
vectors two or two sequences generally and notice I'm using the numeric
typeclass in order to have the sum and I'll do this import so that
I now have syntax I can do star multiplication on the numeric type
you see X is a numeric type and but I have multiplication that I have
addition on it because it\textsf{'}s from the numeric typeclass so the zip
will create me a sequence of tuples of pairs so now for each tuple
I I can write syntax like this I can put a tuple right here I yield
the product so that means the result this entire four expression inside
the parentheses is a sequence of these products now I take the sum
of all these products that\textsf{'}s the scalar product of two vectors this
is a test that it works directly and finally I write matrix product
so again matrix product is kind of difficult because it\textsf{'}s it it has
to be a yield that has a nested four but this is how it works so first
of all we transpose the second Matrix and then we take the scalar
product of one vector the first matrix and the one better from the
second Matrix and we put that into the resulting matrix so that is
I'm not going to go into mathematical details here how to compute
matrix product but this is the way that you can use the for yield
block in order to iterate over these data structures not particularly
visual maybe but at least there\textsf{'}s a no way to make an error in terms
of indices or anything else so that\textsf{'}s an advantage the next type of
monads we are going to consider is the pass/fail units examples are
optional either trying future so as I already mentioned before pass/fail
monads are containers that can hold 1 or 0 values of some type usually
that is interpreted as success or failure so if you do hold a value
of type a and you have success you have successfully completed it
and if you don't then you have failed to compute it that\textsf{'}s how it\textsf{'}s
usually interpreted and these containers usually have special methods
in order to create pass and fail values and an example of this is
a is a try a good example try has methods to catch exceptions if exception
occurred then it will be a fail and if it didn't occur then you would
have a value so that\textsf{'}s a typical pass/fail monad here is a skeleton
or schematic example of a functor block program that uses try we have
some compute Asians that might throw exception so we put them in to
try then we say well X is from this and what does it mean well this
container can have at most one value inside so actually this is not
an iteration at all unlike what we had in the list known as this is
not an iteration this is we're binding X as a new name in case that
this is a success if it\textsf{'}s a failure then there\textsf{'}s nothing to buy in
so this entire filter block will collapsed to the failure remember
that when the collection or the container on the right hand side has
zero elements when the entire for yield or founder block computation
collapses to zero length container this is exactly what happens if
this computation here were to fail if it doesn't fail and we can continue
compute something else in this line nothing can well nothing can fail
provided that F does not throw an exception of course here we can
filter if this returns false then again the entire function block
will collapse to a zero length container so to speak if it does not
collapse then we continue with new computation again here this function
could throw an exception if so then the exception will be caught and
the entire thing collapses when this entire thing collapses the result
is still a try so the result will contain information about the failure
it will just contain no value of type a so there won't be any value
of type a in that case and so on so this is how we write code with
the pass/fail moments we keep assigning new variables here and hoping
for the best so the only way that we can yield the value here is when
all of these computations are successful so there\textsf{'}s no failure in
any of them if so then we owe the value of type a and the result is
their fourth of type try of it now the same rule holds that the right-hand
side of the generator lines must be the same type constructor but
the type of these values could be different in each line the type
of values held by the container may be different but the type of the
container itself the type constructor must be the same or at least
it must be some superclass in Scala we have inheritance so this could
be used to have a superclass and that is used for sequences you can
have different subclasses of sequence but that must be somehow convertible
to the same superclass so another important thing to notice about
pass/fail munoz is that the computation is really sequential so until
this is done until this computation is finished there\textsf{'}s no way to
continue we have to have this X to continue until this computation
is done and if it is successful of course there\textsf{'}s no going no way
to find Z and so we cannot have an X here until this and unless this
is successful so this is really a sequential computation we cannot
continue usually with the next line of the computation until the previous
line is done so this is true even if we have the future functor in
the future factor scheduled computations on different threads and
these computations could be proceeding in parallel but still the computations
are sequential we'll see an example about this {[}Music{]} so once
the any computation fails the entire functor block fails so remember
the number of elements in the entire collection is going to be the
product of the number of elements in the generator lines and sorry
for in if any one of them is zero then the entire thing collapses
to zero so there are going to be zero elements in the result in other
words it\textsf{'}s going to be a failure and if all the computations succeed
if there\textsf{'}s no failure Miniver so for option fader means none for either
failure means a left for try there is a failure constructor for future
there is a failure constructor only when all of them are successful
then this entire fungal block will have a value as a result just one
value also filtering can make it fail so the benefit of using the
for yield syntax with a type constructor such as option either try
and future is that you don't need to write code with nested if-else
or match case expressions you could have written this code with just
match case first you compute this drive then you match the result
if it\textsf{'}s successful then you get an X then you compute this and you
if this is successful then you compute this and so on then you match
on the result so it will be a bunch of nested if-else or a bunch of
nested match case that kind of code is hard to read hard to modify
this is kind of a flat flat looking our code that is easier to read
so you see this has to be done this has to be done the code is logically
flat that is the advantage it\textsf{'}s easier to understand easier to read
of course they have to be get used to the same that the type constructors
I'm just writing out again that these must be familiar to you by now
these are disjunctions so the pass/fail Monitor typically disjunctions
now the tri is equivalent to an either where the type Z is throwable
which captures exceptions and then there is this data type a so the
typical reason we use these moments is we need to perform a sequence
of computations each of these computations might fail and we cannot
continue if one of them fails we have to report the error somehow
so we just make this explicit with return an error value which is
not to crash it is not sometimes exception or error situation it\textsf{'}s
a value that captures the information we want to give about what happened
and why things failed and so that\textsf{'}s the typical use of a sphere moniz
let\textsf{'}s look at some work examples the first example is to read Java
properties so Java properties are strings which are held in the system
dictionary of key value pairs so you have a key which is a string
and a value which is a string so here\textsf{'}s an example we have some properties
and we want to compute something for example for the client we want
to find out which corporation the client is working for for that corporation
what are the orders as were posted and then for the order we want
to have the amount or something like this now the Java API is such
that you can say get property only when you do get property that returns
a string but actually it could return null if this property does not
exist and so in Java this is the usual reason for null pointer exceptions
so some function returns null but you didn't expect it you start following
methods on that you get a crash so in order to avoid this and make
it safe let\textsf{'}s put this inside an option so the option type has a function
that you put an option of something and if that something is null
then you will get an empty option otherwise you would get a non empty
option so that\textsf{'}s a very convenient function whenever you have some
Java API that can return null in order to signal that something wasn't
there or something was incorrect wrap it into an option like this
it becomes safe and now the code we write looks like this first let\textsf{'}s
say somebody gives us the client name we want to return the amount
of the order now it might be that this client is not found where the
corporation is not found or ordered is not found in this case we cannot
return the amount renamed this function for clarity I cannot return
the amount in this case so I will return an empty option otherwise
I will return a non empty option with the requested amount so this
is the idea are always return a well defined value just that sometimes
this value will be empty or it will not contain the data that I was
supposed to return because I can't return love data and so I write
a for yield with these values being strings but because this is a
monadic block whenever one of these is now this entire block would
collapse to an empty option so I can write the code as if everything
is good so this is the so-called good path or happy path but I know
that whenever something is null here either the corporation\textsf{'}s property
does not exist or orders property does not exist then this entire
for your blog will collapse into an empty option so I'm safe to write
code here last line I'm trying to convert the string value to integer
now the string value is guaranteed to be not novel but it could be
incorrectly formatted so it\textsf{'}s conversion to end might fail with an
exception no matter I'll grab that into a try so after that exception
is going to be caught and then I convert that to option so there was
this helpful method to convert to try values to option so if try had
I would have an empty ocean here and then everything will collapse
I'll return an empty option so you see this code has no if else it
looks very linear and yet it\textsf{'}s completely safe all the errors are
handled invalid non integer values either our handle they're going
to be ignored so the test is that if I want the order mount for client
3 there is no client 3 and the second example is to obtain values
from computations using future so this is an interesting example in
order to work with futures I make this import just for simplicity
so imagine I have a long computation now I'm going to compute something
not particularly useful maybe some kind of long Sun with cosine functions
and whatever it\textsf{'}s not particularly important I just want to have a
computation that takes time in order to save time this computation
I want to make it in parallel so I put this computation into a future
and I'm going to try to compute them in parallel now here\textsf{'}s the code
I write I have this auxiliary function time which will return me the
result and also the time it took to compute the result so how do i
compute the result well I have a 4 for yield which uses the future
so that the right-hand side of these computes the future of double
and therefore the left hand side is double so the type of the container
in this for yield block is future it must be the same type throughout
the container so that is future and the result type is going to be
a future double as well and so this looks like I am performing first
along computation which is going to be in the future when that computation
is done I perform another long computation using that value as input
which will call this function and create a new future and what future
will depend on this X because the computation uses the X C and so
I cannot really start this computation until this one is finished
putting future\textsf{'}s in a for yield block will sequence them in other
words first it will wait until this future has done its computation
to return its value and then will put this value in here call this
function and start a second future wait until that is done put this
here and wait another feature I should I should return I should return
Z probably to be slightly less useless for this block in any case
the result is that these futures are sequenced one of them waits until
the other is done now imagine that these computations were not depending
on each other well here they do this computation depends upon respects
imagine that that weren't the case if that were in the case let\textsf{'}s
say we have three computations and these parameters are known in advance
then we could have started started all of them up front so look at
this code so we create three futures of all of these part of type
future we create three futures and in Scala future is a funny type
because once you create a future you already schedule is to run there\textsf{'}s
no separate operation to started or to run it once you make a future
it means the system will attempt to run it already there will be probably
already running process so this means these three computations are
probably already running by the time we are here what do we do here
we make a four-year block which is superficially similar to this one
we just put the computations in variables and put them here so the
result however is going to be faster because all these futures are
started at the same time yes here you wait for them but they already
started all in parallel so you would just wait less you would first
wait until this is done and while you wait maybe this is already done
so this second wait will take no time at all and so the result I expect
to be faster so in this test I'm fringing how long it takes and the
typical output I got in my tests was like this so the first sequential
futures it took six seconds and a second took 2.7 seconds so this
is not exactly three times faster but it is more than twice as fast
in any case we were able to do this only because the three computations
don't depend on each other if they do is there is no no way to speed
it up it\textsf{'}s just there\textsf{'}s no advantage in putting that computation in
the future if you cannot do it in parallel for this at least for this
test let\textsf{'}s look at the other example example 3 which is we want to
make a rithmetic safe and we use the either type to return error messages
so that\textsf{'}s a very common use for disjunction type so let\textsf{'}s make this
type that has either string or double in it and string will be an
error message of some kind meaning that we cannot compute the double
value strictly speaking this is not a factor because it doesn't have
a type parameter so I'm just abusing well I'm the either is a function
I'm using the inner function and I'm just specifying the type to double
because this is going to be in my example but actually this is a functor
with double as a type parameter so we could think about this as a
function and that\textsf{'}s what it is how we are going to use it so the idea
is that if there is a operation that could fail then we return the
left with some error message and if there isn't success in returning
right with that value so if we already have a double and we do an
implicit conversion of that to the safe double by just putting a right
around it and in this example the only operation we're concerned is
division so we can divide by zero we don't want to try to do it do
that we want to return an error message and so that\textsf{'}s what this function
does is a safe divide it would tell tell me what I tried to divide
by zero so hopefully it will help me in my debugging so then that\textsf{'}s
the point of this example so here\textsf{'}s the code that I use so instead
of writing well x equals one and I've put everything in a functor
block and you see quite unlike the things we did with sequences these
are not loops at all these are not iterations in any sense for yield
is not a loop that is a important point that I would like to make
there is no iteration going on necessarily it\textsf{'}s a monadic sequencing
operation rather so we are sequencing computations that might fail
if they fail we want to fail the entire computation if any any of
the steps fail so here is what we do X is assigned to be one that\textsf{'}s
not going to fail because there\textsf{'}s no division here is a division let\textsf{'}s
say no it\textsf{'}s not going to fail but we still use a safe divide and then
we do this so the result is going to be always safe double so here
the right hand side of the generator arrows is either of the type
either already is a safe double or it\textsf{'}s automatically converted to
safe double so safe double is a type of the right hand side of this
look and here\textsf{'}s an example where it fails so we divide by zero we
do exactly the same things but because this step divided by zero the
entire thing collapses we never get here and the result is a left
of that a last example for the pass fail is to sequence amputations
that may throw an exception so here we sequenced computations that
could divide by zero and here we just have arbitrary exception so
here is what we do dividing by zero throws an exception for integers
it doesn't for doubles it gives you a not a number but for integers
it throws an exception so imagine we have some functions that might
throw an exception we wrap all of this into a try and then we have
code like this and so this is completely safe the result is of type
try of int and we can in examine the result to see that it\textsf{'}s a success
or a failure so we do match expression and then in this test I know
it\textsf{'}s going to be a failure because I first what what did I do if one
of one so I divide two by one the result is two then I subtract 2
minus 2 the result is 0 then I divide one by zero so I know I know
it\textsf{'}s going to be a failure but in principle here you could have a
case of success and that\textsf{'}s how you use the pass/fail chain so you
put everything in to try the result is also going to be a track so
I can just add the type here for clarity you don't have to write this
but for clarity I want to the next type of monads I'd like to talk
about are tree like monads so what are the tree like moments here
are some examples these are type constructors defined recursively
and I use the short type notation so it\textsf{'}s easier to understand what\textsf{'}s
happening so the binary tree is either a leaf or a pair of two binary
trees so that\textsf{'}s a familiar type perhaps and because it\textsf{'}s recursive
then this can be again either a leaf or a pair of two binary trees
and this also can be maybe a pair and so this is a pair and this is
another pair that just keeps splitting until it ends with a leaf another
stability of generalizing this kind of construction is to say well
actually this is a functor this is a pair of F and F let\textsf{'}s say that
the pair of F and F is a function y 2f let\textsf{'}s call this one term s
shape and so then it will be s of F of ready let\textsf{'}s take arbitrary
function s not necessarily just a pair so it could be triple then
the tree would branch in three branches at each each point instead
of two branch so we can just parameterize the shape of the tree by
an arbitrary function S this factor could be actually arbitrary could
be list when you would have a rosetree what\textsf{'}s called where the branching
can be arbitrarily large in ten you point could be any factor so that\textsf{'}s
what I call an s-shaped tree where s is a funder functor shape dream
another interesting example of a tree is when both the leaves and
the branches have the same shape so the leaf must have a pear and
the branches must be too and analogous generalization is when you
have functor shaped leaf and a functor shaped branching so these are
perhaps more rarely used but I found them interesting to Q consider
most examples for all of these you can implement flat map and we'll
look at examples of how to do that so here\textsf{'}s here\textsf{'}s how you implement
a flat map for a binary tree with binary leaves this is our first
type constructor here I'm sorry this is a third type constructor here
the binary tree with binary leaves so how do you implement flat map
well let\textsf{'}s first define the type so the type is this it\textsf{'}s a disjunction
the first element of the disjunction is a pair of a a so just follow
the type here the first element of the disjunction has a pair of a
a the second element the second part of the disjunction is a pair
of two three factors themselves so I call this B X and B Y now we
need to implement a functor instance for this and I have my own typeclass
here called semi monad just for convenience where I can define flatmap
{[}Music{]} so to define functor is pretty easy obviously if you have
a pair of a a you just map both of them with a function f and if you
have a branch then you recursively map each part of the branch in
the same way so flat map works actually quite similarly except for
the leaf if you have a leaf the leaf has two elements and so you have
a function that takes the leaf and returns a tree so now you have
two trees so you can't have a leaf if two trees in it but we can have
a branch and so we put these two new trees into a branch here is how
we can implement flat map for a functor shaped tree and here I make
a more abstract formulation where I have actually a tree that\textsf{'}s permit
rised by an arbitrary function so that\textsf{'}s a functor shaped tree so
the leaf is just one element of type a one data item and branch is
a functor s applied to the tree so it\textsf{'}s a functor shaped branch how
do we implement the functor instance and how to implement the semi
Monod instance for this tree this is a little involved but the sin
it\textsf{'}s just a syntax that\textsf{'}s a little complicated here because we need
to parameterize by an arbitrary function and so the type constructor
is this so the factor s is fixed the type parameter is free the map
works by matching so if it\textsf{'}s a leaf and we map the leaf value with
the function f if it\textsf{'}s a branch then we we use the map function only
the factor s which we should have because we assume that s as a function
and the function we use to map is the recursive instance of mapping
the same tree here is what we do where it were in this case so if
we want to map a to be we map a to be over here recursively and then
we map over the function s because we know it\textsf{'}s a function and flat
map works similarly if we have a function from A to F B then this
is just mapped to FB by itself and we haven't have been and this Maps
afraid to have be recursively and then the out the outer layer of
the Thunder s is mapped using its own map function so here\textsf{'}s how it
works so we have a function f from a to the tree if we have a leaf
and we just put that entire tree instead of the leaf or as one can
say we graft the subtree at this point and if we have a branch then
we map over the branch which it because we have the outlaw outer layer
on the family yes so we've mapped that and underneath we used the
recursive call to the flat map here\textsf{'}s an example of an entree like
type constructor it is a disjunction of this kind so it can have one
a can have two is it can have four A\textsf{'}s and so on all the powers of
two now this type constructor is actually not recursive so it is not
a tree like type constructor it is not one of these cannot be represented
as one of these and it is not a ma not in the usual sense or not a
tree like ma not in any case so a little bit more intuition about
how flat map map works for a binary tree so imagine that we have a
tree that looks like this and we need to flat map it with a function
that takes any so all these are leaves of type a and so a function
f takes a and returns a tree of type B imagine that when we apply
the function f to these three values of type a we get three different
trees of type B so suppose that these are the trees we get so the
tree that has these two leaves a tree that just has one leaf and a
tree that has these two so then if what the flat map is supposed to
do its supposed to replace a one with this subtree so instead of anyone
will have this subtree a tree with b2 so that\textsf{'}s this replacement a3
is going to be replaced with this subtree so a3 used to be here now
it\textsf{'}s this subtree so this is the result of applying flatmap to tree
like walnuts its grafts sub trees in places of leaves that plays the
role of flattening so actually nothing is being flattened here in
some in a sense of trees remain trees trees are not flat in the organ
Airy sense and they do not become flat in any sense however what becomes
flat is that we had a tree in the shui for each leaf we replaced that
with a tree but we don't get tree of trees tree of trees is just equivalent
to a tree that is what it means to be flat note that the tree becomes
somehow meant dreamlike but a tree of tree of B can be seen as simply
a tree of D there is no need to say that we have a tree of trees the
trees already branching enough so that\textsf{'}s what flattening does for
trees typical tasks that dream tree like monads perform are traversing
a tree and replacing leaves by sub trees are grafting leaves grafting
sub trees at leaves one example of doing this is a user is transferring
a syntax tree representing some expression where you substitute sub
expressions in it so if if this is some expression tree then you want
to substitute sub expressions instead of leaves and that\textsf{'}s it kind
of typical tasks that tree-like Mona will do so let\textsf{'}s look at worked
examples the first example is to implement a tree of strength properties
so let\textsf{'}s take a look at that so probe tree is going to be our factor
so just for simplicity I'm going to put a map and flatmap into the
street right there we'll implement them in case classes that implement
the trait this is not necessarily how you want to do it but if this
is your own type it\textsf{'}s easy to do that if it\textsf{'}s not your own type so
that your you cannot change the source code for some somebody else\textsf{'}s
type constructor then you need to do typeclasses and add map and flatmap
using type 1 system I just want to make it short okay so what is this
type so he has a leaf which has a value of a and also it has a fork
which has a name and a sequence of other trees so it\textsf{'}s a it\textsf{'}s a slightly
different shape so the forks are named so there\textsf{'}s a name and it can
have zero or more sub trees here is a sequence so it\textsf{'}s kind of a rosetree
with named branches so how do we implement map well if it\textsf{'}s a leaf
we just map the value if it\textsf{'}s a fork then name stays the same but
each of the trees is mapped with its its own function this is a recursive
call Raney however the recursion is hidden it\textsf{'}s not the same function
this function on a different object what\textsf{'}s the same it\textsf{'}s still recursive
flatmap so if we are in the leaf we need to replace value any with
a different proper tree we just graphed that probe tree in place of
the leaf that\textsf{'}s a standard thing to do and flatmap for a fork is just
mapping so basically this is our example of a functor shaped tree
this was this example where the founder s has a specific type it is
a product of a string and the sequence of a or a sequence of of trees
so that is just a special case of the construction of a functor shape
trip right so here is an example we have a fork with named a1 and
it has a leaf one a leaf - and another fork named a - with leaf 3
now we can map on the stree with the function that adds 10 so then
each leaf will get 10 added to it the structure of the tree remains
the same after mint because map doesn't graft anything it just changes
the values and leaves now let\textsf{'}s look at this code we want to look
at the leaves and somehow transform leaves that are small and small
is less or equal to 1 so here\textsf{'}s the code for this so we we said X
goes over tree which means that actually the value of X goes over
leaf values because that\textsf{'}s the values in the tree that are being being
used and then if the leaf is larger we keep it as a leaf otherwise
we make a fork in it and call it small and put a leaf inside this
work and so we yield that so this code transforms a tree the tree
that we had here has fork a 1 and so on it transforms into this so
the leaf one is a small and instead of leaf 1 we have now a different
tree this one everything else stays this sparkly one but me maybe
format this so that it\textsf{'}s easier to see easier to compare the two different
trees before it after the transformation so the only difference is
that instead of leaf 1 we now have this fork with extra information
so that\textsf{'}s the result of transforming the tree so you see this for
yield is a kind of a loop but it\textsf{'}s a loop over a tree and it can transform
a tree into a new one so just to repeat for yield is not really iteration
it\textsf{'}s more like a general kind of operation that goes over your container
and depending on what the container is like it will it can do many
things so for the tree like monads it\textsf{'}s usually tree traversal exactly
the same computation written by flatmap syntax to study for yield
is this the second example is to implement variable substitutions
for an arithmetic language so what do I mean by an arithmetic language
so imagine we have a symbolic manipulation program wants to manipulate
expressions it can be a compiler or it can be some kind of calculator
program or anything else like that so that kind of program needs to
work with symbolic language that does some kind of reputation and
let\textsf{'}s say every filter so for us we're going to have a very simple
language language is going to have variables and multiplication and
it will have constants integer constants and that\textsf{'}s it so to define
this language we define a disjunction so basically this is the term
let me write down maybe the short notation for this term is it was
a so int is one part of the disjunction which is called Const the
second part of the disjunction is the variable which is called the
var and the third part of the disjunction is malt which is two terms
so this is the definition of the type this is recursive and you see
this tree like except that we have this extra thing here now that\textsf{'}s
fine that doesn't carry modify and structure so much against that
belief now can have an extract information so let\textsf{'}s define map and
flatmap oh this is trivial and I'm going to go over it this is the
same structure we found before and now let\textsf{'}s implement variable substitution
so what does it mean so suppose we're so expressions in this language
will look like this as a constant times VAR times constant times more
and so on there\textsf{'}s nothing else in this language except constants variables
on multiplication so we can do that in any order we want and that\textsf{'}s
all now there are two operations we can do use a map and flatmap using
map I can modify variable names so we can for example append X to
every variable name so here\textsf{'}s what the result will be depending X
2 variable mass second thing we can do is substitute variables so
for example instead of very well a we can substitute this expression
so we can substitute new expressions of the same language instead
of variables and that\textsf{'}s usually what mathematical expressions do variables
in mathematics are substituted with new expressions of the same kind
so that\textsf{'}s what we are going to do here so let\textsf{'}s substitute there I
will a with this expression and the variable B with that expression
and we do it like this so X is going to range over there Able\textsf{'}s wine
is going to replace X so we'll do I instead of X if we just said for
excellent expression yield X that\textsf{'}s going to be the same unmodified
expression as before so we modify we say Y is is that and well we
just threw an exception here just so that I have some simple example
but basically if the name of the variable is a then we substitute
this otherwise we substitute that so the result is correct finally
we come to the non-standard containers for single value monads they
are not really single valued in the sense that they are not equivalent
their container with the single value container with a single value
is identity unit or identity functor it just a continuous has a value
agent that\textsf{'}s not what they actually do but they are similar to that
they can be imagined or reasoned about to some extent thinking that
the only and always hold one value actually the meaning of the single
value units is that they hold the value and also they have some kind
of context together with that value and when you do computations then
contexts play a role they can be combined and they can also be used
to do something and usually for these moments we have methods that
insert a value one container hold one value so that\textsf{'}s necessary to
be able to insert that value in there somehow and also usually there
are methods to work with this context and we will now look at five
examples of single value moments and we'll appreciate the variety
of what this idea can do holding a value in a context so the typical
tasks that single value Mullins do are to manage extra information
about computations as you do the computations along the way and also
to perform a chain of can rotations that have some kind of non-standard
evaluation strategy so in the first case where we are managing extra
information the context is this extra information in the second case
when we are changing computations with some non-standard evaluation
strategy for example in synchronous computation or lazy computation
then the context is that what makes the evaluation strategy non-standard
so it\textsf{'}s not a value this context cannot be just seen as a value in
this case it is kind of the way that we do computations or extra effects
or maybe side effects that occurred during the computation for something
like this so we'll see examples of both of these kinds so let\textsf{'}s look
at the first example it\textsf{'}s the writer monad the writer monad is defined
as this type constructor very simple is just a pair of a data item
of type a and a value of type W where W must be a mono it or a semigroup
suffer for this to be a full mode add W must be a monoid and for this
to be a semi mode add W can be a semigroup to remind you the difference
between 100 and semigroup a mono it has a unit element or identity
or empty element which is a selected element so that you can combine
it with other elements without changing those other elements a semigroup
does not have that element it\textsf{'}s just a set with the binary operation
that is associative so here in this tutorial we can consider a semi
munna des writer as well because we're not necessarily so interested
in the mona fully form on that right now and so we will assume that
w is the same so this w will hold some kind of logging information
about our computations and logging information should be able we should
be able to attend one piece of logging information to another and
that\textsf{'}s the semigroup operation so let\textsf{'}s look at the code so in this
example we have so I defined the writer as semi moonlit where a is
our data type and s is a semigroup and so that is just a pair of a
and s why I wrote s instead of W let me just make W out of this just
so that I'm consistent with my slides so we can easily find the instance
of functor and instance of semilunar now instance of sending one is
just flat map so semi-modern is typeclass that just requires flat
map with a standard type signature so how does that work well obviously
functor works by mapping mapping the value but not changing the log
so log is this information that we logged about the computation now
the flat map has this type signature so when we do the flat map we
need to take the previous value which is already a pair of a and W
then we map a to another pair of BMW now we have to double use so
we have FA dot log and FB deplored FB was computed knowing this and
so we combine the two log values by using the semi group operation
so this is how we could use this in practice imagine that I want to
record in my log some comments about the operation and the time let
the computations began and the time at computation is finished for
the entire chain of computations not just for each operation separately
but for the entire chain of computations I want to find begin time
and end here\textsf{'}s how I can implement this I make a semigroup that has
three values and the semigroup type is a triple of local date time
local date time and string how do i define a semigroup instance I
need to combine two logs of this type so the messages I combine by
just appending them with some new line separator but the beginning
end I combine in a specific way further from the begin I take the
first begin but for the end I take the second and second why today
because it I'm combining two operation so the first log in the second
blog and I assume that the first log occurred first and so the beginning
of this entire operation is the beginning of the first log while the
end of this entire operation is going to be the end of the second
log so that\textsf{'}s why I'm putting it like this now this is actually a
semigroup it is associative but it is not a monoid it cannot be read
into mono it so my writer here will be a semi Munna but that\textsf{'}s the
use case so that\textsf{'}s that\textsf{'}s what I want it for so now I define a convenience
type constructor which is the writer with this logs as a type and
i also define a help of helper function that will do the logging so
i have a value i have some message and then I insert the timestamp
into the log and the message that\textsf{'}s very easy mean I just insert begin
and / convenience as the same timestamp and things will automatically
be adjusted as necessary once I combine this with computations made
later so this import is necessary in order to use this machinery with
the same Amanat typeclass and here is the code so these compute things
are fake computations that pretend to take time so they actually wait
for this time but that\textsf{'}s fine for this example so you see I'm logging
so I log in everything with the message and the times will be automatically
looked so I first start with the log int so the integer value so X
so the integer then I add 1 to that X that becomes my life so that\textsf{'}s
going to be 4 and then I multiply by the 2.0 which is a double so
the result will be log double and I'll yield Z in the log function
block so the result is of type log the devil so as usual the right-hand
side of the generator lines is always of the same container type which
is logged this is the container type for this entire function block
this is the function so these are computations in the context of a
factor this is what I call computations in the context of a functor
we write as if these are values we compute but actually these are
inside the function so everything is so as if we return a double but
actually when we compute any factor parameterize by double so logged
double that\textsf{'}s the type of the result let me put it in here for clarity
and the result is 8 so it\textsf{'}s 3 plus 1 4 times 2 now we checked what
the log message is so the log message is actually beginner is 3 then
there is a newline head wound there\textsf{'}s a new line x 2.0 neither our
global messages and also I can check what is the interval between
beginning and end and test checks that this is about right so it\textsf{'}s
20 milliseconds 50 milliseconds at 100 milliseconds together it\textsf{'}s
going to be slightly bigger but anyway not much bigger so this is
the example of using a writer movement in our case it was a semi mu
not actually not a moaner use a writer monad we can easily make automatically
logging computations with keeping track of wall clock time the second
example is the reader movement so the reader monad is also to be thought
of as a container that holds exactly one value in the context and
the context is some value representing the environment some type II
which is given it is read-only you can read it and it\textsf{'}s always available
for you to read so this can be used for dependency injection or for
passing some common parameters that you don't want to pass with explicitly
with every function and dependency injection means that the computation
within the reader functor or with the context of the reader frontier
take this as a parameter automatically they don't depend on what it
is and then you you inject it later so that\textsf{'}s how it works let\textsf{'}s taken
let\textsf{'}s look at an example the example will be that we have logged computations
and the logger is the injected dependency so let\textsf{'}s say performs some
kind of side effects a function of this type and we want to make computations
that use this function but we don't we want to depend on this function
as read-only context so somebody will give this this to us later but
we want to write code now so we use the reader monad which is just
a cat\textsf{'}s reader now cats reader is equivalent to this type which is
just a function from Italy so for convenience we define this type
as a reader of log duration hey so we will understand this container
hence a container that has a single value of a but that value depends
somehow our computations can depend on this logger function in some
way so let\textsf{'}s make a constructor for convenience that will log a message
so this constructor now has a lazy argument X so this syntax in Scala
means that the function argument is lazy and unevaluated until the
function evaluates it so that\textsf{'}s kind of automatic lazy parameter passing
and here\textsf{'}s the code so the log returns a reader with the log duration
in it so it\textsf{'}s a function from log duration to a first we compute this
X here and we time it so we compute the time to actually run this
XC as I just said a result equals x actually because x is a lazy parameter
it is not yet evaluated until I do this so this is when X will be
evaluated and this might take time and so I compute that duration
and then I log it using blogger function and I return the result so
that\textsf{'}s my convenience function so once I coded this I can write code
like that like I did before very similar with now I had I can do add
another set with reading which is usually called tell now tell is
just a function that returns the injected dependency or the environment
or the context and this is just a reader of identity which is a function
from e to e that\textsf{'}s an identity function so using that you can always
extract that dependency explicitly and use it in if you want so here\textsf{'}s
how we do this we say log duration is tell and now we can use that
that\textsf{'}s our logger function we can just use a hand hook if we want
to we don't have to do this because our computations are automatically
logged what we might need to use it for some in some example so the
tail is the usual function that people define for the video moona
all right so how is it going to be tested I'm going to have logger
will just print some message how much it took nearly seconds and then
I'm going to do a run so important I already computed this result
reader but nothing as jim has run yet this result reader is a function
from log duration to double this function hasn't been called yet I
have written the code as if the computation is already done but actually
it isn't yet done it needs to be run still so units of this kind usually
have a method called run and so this method takes a log duration and
it gives me a double and so that\textsf{'}s what I have to do at the end and
so the there is always this printout as we expected it\textsf{'}s about 20
seconds about 50 milliseconds about hundred milliseconds and then
there is this extra ad hoc computation now just one note here so the
usefulness of this technique is that you can separate this code from
this code you can combine these readers with each other you can put
result reader here on the right hand side a generator line because
it has the right type the whatever is on the right of the generator
line must be of type reader of something as long as it\textsf{'}s the right
type a reader of log duration and something you can put it on the
right hand side so in this way you can combine several values of the
reader type income in computation to make a longer computation and
at the very end once you have done all you need you actually run it
at the very end when you actually know what logger you need or what
context you need what dependencies you inject you run it so this is
the this is the event that you can separate injecting your dependencies
from writing code it uses the dependencies and this code looks like
this dependency is invisible here we made it visible because we wanted
to illustrate that you can but you don't have to if I remove this
it\textsf{'}s invisible if you have this dependency your code doesn't get cluttered
with it so that\textsf{'}s this message a third example is an evil unit which
is something that cats library implements we will implement a very
simple version of it ourselves for purposes of illustration the idea
here is to perform lazy computations in other words computations that
are not immediately executed and so in order to do that let\textsf{'}s define
this type so this type is a disjunction is either has a value that\textsf{'}s
already computed or it has a value that will be computed once we call
this function now this function has a unit argument so there is no
nothing we are waiting for but we just haven't haven't called it yet
and so there\textsf{'}s no data that we need to receive from someone in order
to run this function well it\textsf{'}s up to us when we want to run it and
so that\textsf{'}s the point of this monad it\textsf{'}s either a value that\textsf{'}s already
been done and then so-called it\textsf{'}s memoirist in other words if you
try to compute it again it\textsf{'}s already there or it hasn't yet been computed
and then it\textsf{'}s this is a explicit representation of lazy value Scala
has built-in lazy values but sometimes it\textsf{'}s useful to have a more
explicit representation so here\textsf{'}s how it works so let\textsf{'}s define a trait
which is evil like I said the cats library already defines this with
slightly different API but I don't want to use their evil because
I want to show how it works the main function that it has is get so
yet means get give me the value so evil is a same single you monitored
it holds inside it a single value of type it so get me that value
and if that value hasn't been computed yet compute it and give it
to me otherwise if it\textsf{'}s already computed don't give it to me right
away that\textsf{'}s the point of this get and then we have a map and flatmap
because it\textsf{'}s a semi wounded it\textsf{'}s actually a full modded of course
because it\textsf{'}s easy to put a into here so pure would be very easy to
implement it has a disjunction as we have indicated a and 1/2 wave
so let\textsf{'}s see is this one it\textsf{'}s called eval now so it\textsf{'}s already evaluated
now and the second one is a function from unit 2 episodes eval later
it will be evaluated later so to get what\textsf{'}s already valued it is trivial
to get what will be very linear we'll just call that X let\textsf{'}s call
this actually call us if call this ok I don't want to be nameless
we just call that function on a unit argument and we get what we want
implementing the map and flatmap is trivial except well except for
one thing so for flat map we evaluator but what we evolve here we
get in other words we don't postpone that evaluation anymore we already
postponed once so we don't want the postponed twice that\textsf{'}s the point
of flat map so a non flared structure would be that we postpone the
postponed computation so that\textsf{'}s kind of nested postponement we've
want to flatten it so we want to postpone exactly once and so we postpone
wait for the unit but once the unit is given that\textsf{'}s evaluated it was
not postponed anymore that\textsf{'}s how it works now we can define convenience
constructor so this is a pillar which just gives you an eval now and
you can have a constructor for later which gives you an evaluator
and it\textsf{'}s just for convenience see I'm using lazy evaluated argument
in the function just one little comment here in programming a software
engineering community these arguments are called by named arguments
can't be more wrong than calling this I'm not by name they're undervalued
and or lazily evaluated it\textsf{'}s not the name that you pass you pass an
unavailing it'ld expression that needs to be evaluated you don't have
to have a name for it and often you don't so they are not by name
argument they're an evaluate it or lazy evaluated arguments anyway
so what do we do let\textsf{'}s again use our compute helper function that
will introduce delays into computations and here however we do we
can write a for yield block with all sorts of things in it and on
the right hand side we can either put later of something or not off
of something so that\textsf{'}s how we will use this and this way we'll chain
computations that are postponed and computations that are not postponed
so we can combine them very easily with very clear organization of
the code and the result is again eval in so this result is probably
postponed and we can get it and when we get it it will print things
so this is the output not you yet so when you create a value of this
type it doesn't necessarily compute anything and you see this because
this message is printed first so this message is printed first and
only then these two messages are printed that we have here so this
might be counterintuitive but actually when you do this none of this
is run yet again this is a kind of Munna that has non-standard evaluation
strategy it doesn't run immediately when you do a for yield it will
make these computations postponed you have to run it so the run for
this model is called get and so that\textsf{'}s when these two lines will be
printed so this could be a little counterintuitive when you work with
these single single value mu nuts because many of them encapsulate
non-trivial evaluation strategies reader monitor was at the same kind
once you do this nothing is that done yet the function is constructed
the function still has to be called so that\textsf{'}s the REM call here it\textsf{'}s
a gate call and that actually runs the moment the first example is
continuation unit this is not necessarily very clear what it means
I prefer to call it a callback monad but continuation monad is a standard
name for it so I will go into detail now but how it works the purpose
of this moment is to chain asynchronous operations operations that
register callbacks so it\textsf{'}s managing callbacks that\textsf{'}s the main use
case for this moment there are other use cases one of the main use
cases is this one so let\textsf{'}s look at the code an example I will show
is using the Java input/output operations on files and using asynchronous
file operations so these are called asynchronous file channels and
there are asynchronous in the sense that you start reading the file
and you need to pass a callback that will be called when it\textsf{'}s finished
reading it so this is asynchronous code in this example is to read
a file so I created a sample file which is this one which is very
short I'm going to read it into a buffer I'm going to write the same
thing that I just read into a different file into this one then I'm
going to read that file again and compare the results so that I verify
that whatever I have in this second file is the same as what I had
in the first file so to read it into a second buffer and assert that
the buffers have the same strings in them now you see how much code
was necessary to get that done all I did is to read one file write
the contents to another read that contents and compare with the first
look at how much code I have to write using the Java look at look
at this nested thing it\textsf{'}s a deeply nested combination of functions
this is a typical problem with asynchronous API is that you use callbacks
callbacks get nested let\textsf{'}s see how he how that works the API is like
this you make a file channel then you call the read function on it
which takes a buffer and some parameters just in my case are just
0 and then it takes a callback now the callback is a new object of
type or value rather of type completion handler and the completion
henry has two functions inside one is called when there\textsf{'}s a failure
and the other is called when there is success the file has been read
so for the purposes of this example I'm not going to handle errors
I'm just going to print that errors occurred so I'm just going to
concentrate on the happy path on the success so the callback for success
is this not in my code what I have to do is I need to well I need
to wait until this callback is called there\textsf{'}s no way to wait for it
so the only way I can write my program is to write the rest of the
program inside this callback so that\textsf{'}s how I get into a deep nesting
so the rest of my program is inside the callback I'm closing the channel
rewinding the buffer opening another channel and then I call the right
function on this other channel the right function similarly takes
the buffer and a completion Handler a new completion Handler with
its own callbacks the failed callback had print a message a completed
callback will be called when the second file is written successful
there is no way for me to find out when that happens so right because
I already called the right function the right function has finished
very quickly and then the file has started to be written after the
right function has already returned there is no way for me to figure
out when is completed function will be called this callback therefore
my entire rest of the program must be inside the callback I closed
the alpha channel I make another input channel I read and with a third
callback print message when failed completed and the entire rest of
my program must be in here so for example I want to check that bytes
were correctly read and correctly and incorrectly written I cannot
really pass this information to somebody else outside of this callback
unless I use local variables global Global mutable variables where
I I write some information and a semaphore that somebody will have
to wait for on the thread this is horrible very hard to write such
programs or I need another callback that I will call here to pass
this information I could do this if somebody gave me a call back such
as report result I could put that result into the callback and call
it so somebody else will then have to continue writing this kind of
code that everything is inside the callback because of this problem
people call this people usually say this is callback hell all these
callbacks that forced you to write the rest of your program inside
a deeply nested structure wangus a continuation monad manages all
this in a much better way so let\textsf{'}s rewrite this code by using the
continuation or not now the type of the continuation monad is going
to be this it is kind of not clear why this type is useful but let\textsf{'}s
look at this and consider what it is doing this type is a function
that takes this as an argument now this function itself this is a
callback this is a typical callback that we had in the completion
handler it took it takes some result or some in input and returns
unit so a function that takes data and returns unit that\textsf{'}s a typical
callback signature so the continuation monad is something that consumes
a callback it\textsf{'}s a consumer of a callback that\textsf{'}s how we can interpret
this type what can I do how what does it mean to consume a callback
well you can use the combat the only way to use the callback is to
to run it to give it a value of type a and run this function and then
whatever it does you don't know but that\textsf{'}s what you can do you can
call it twice if you have several different values a you can call
it many times or you can omit omit calling it at all if you don't
get any values of type way so that\textsf{'}s what you can do with a callback
so when you create a value of this type it means you have created
code that will do something possibly obtain one or more values of
type a and run this callback on those values that\textsf{'}s what the continuation
what it is doing so creating a value of this type of this continuation
type means you have created your callback consumer or your callback
logic that\textsf{'}s why it is so useful for situations where you need a lot
of callbacks you should not use callbacks like I just showed you in
this complicated looking code you should use the continuation monad
here is calm so in our case let\textsf{'}s come let\textsf{'}s define this type constructor
the read and write functions for the Java and IO need to be adapted
to this type so let\textsf{'}s create values of this type and for convenience
we will allocate the buffer inside this function so the read will
take the channel and return a value of the moolaade type so you see
the mana type is this type constructor and I will moon out of a so
we're going to use and I am honored of the pair of bytebuffer an integer
we can use anything as a type parameter why because the idea is that
the continuation monad manages callbacks or consumes callbacks that
will be run on values of type a and so the callback that we had in
the previous code returned an integer or rather it was called on an
integer value and so the byte buffer actually is another useful value
that we want to pass on and so we by using this type we make the API
easier to use so the callbacks in the Java API only take the result
or parameter which is a number of bytes read you are supposed to have
your byte buffers somewhere as global variables it will be much better
if they actually took data here has arguments and that\textsf{'}s what we can
easily accomplish so ni or it will be a function that returns the
Monod type therefore we can use ni or read at the right hand side
of the generator here\textsf{'}s how we do we count as a constructor of the
continuation which takes remember continuation is this type so we
need to create a value of this type in other words we need to write
a function that takes this and returns this so we write a function
that takes this and it returns unit so this entire thing should return
unit Handler is going to be this function because we defined it like
this {[}Music{]} so what does it do well we allocate the buffer for
the purposes of this example I'm cutting all kinds of corners I'm
just allocating a fixed length buffer for its simplicity and I'm ignoring
errors I'm just planting that errors happened all of this can be done
much better with more work so in order to start understanding how
to use it we start with a simple example so we allocate the buffer
then we call the read with this buffer and we make a completion you
know there is no going around this API but we only are going to do
it once and it\textsf{'}s going to be much easier to understand nothing will
be invested callbacks so the failed way ignore the completed log print
something then we do all this rewinding the buffer closing the channel
and then we call the handler on the data that we obtained on the buffer
and the number of bytes read so that\textsf{'}s the call that will return unit
remember so the handler is a callback that returns unit we can actually
rename this for clarity make it call back and so we call this callback
at the end giving it the results we obtained that is what a value
of this type can do it can take this callback produce some values
of type a and call the callback on this value so in our case is just
one value we're not going to call it ever in case of error we do exactly
the same thing for the right now the right method needs a buffer and
the channel has two arguments and it returns them an IO monad of integer
so that\textsf{'}s very similar let\textsf{'}s also rename this handler to callback
consistency and it\textsf{'}s exactly the same thing so check close the channel
and call the results so now let\textsf{'}s see how this is going to be used
we open the channel to the file can then we start writing the for
yield or the function block we do read the result is a pair of buffer
and result then we'll make another channel we do a write with this
buffer and this channel that\textsf{'}s a result we make another channel we
do a read of that channel into a buffer to err is not three when we
compute this is identical if the number of bytes the same and the
string in the buffer is the same after copying and we return or yield
rather is identical in other words this is a boolean value our entire
for yield gives us an enablement of boolean so this is a non-standard
container so none of this actually has been run yet when we do this
when we make this value none of those operations have been done yet
we need to run the bonnet in order to run it but how do we run it
well the run means that we take a value of this type and provided
with an argument of this type and applied that function to that argument
so we need to provide a callback of type a to unit where a is boolean
because we returned it and I have one of the boolean so we need to
provide a callback from boolean to unit let\textsf{'}s provide it so the status
is boolean so it\textsf{'}s a function from Wooyoung to unit so whatever we
want we can put in here and then we run that\textsf{'}s what\textsf{'}s going to print
and it does print exactly what we want at this point now just one
little remark here you see declaration is never used indeed we never
used this value it is returned so to speak but it\textsf{'}s not used that\textsf{'}s
just put an underscore here that is the usual way that we indicate
that there is a value but we don't need it and we put underscore over
there it\textsf{'}s alright so that\textsf{'}s how now our code has become flat in a
sense there is no nested anything there is a bit of there wasn't a
here because of all these options and so on well you can always refactor
that to be some function somewhere else but basically you you can
combine callbacks very easily like this now nothing is so so hard
to understand except for the type of this thing continuation or not
the type of the continuation well that is hard to understand have
to go through the steps it\textsf{'}s a callback consumer that will call the
call back when the value is available and that\textsf{'}s what the ROM does
and you need to understand that flatmap does the same thing so yeah
by the way the implementation of flatmap for my continuation that\textsf{'}s
automated that\textsf{'}s done by the Curie Harvard library all right let me
go back to my slides there\textsf{'}s a last example which is a statement a
statement is defined as this type and it\textsf{'}s used for doing a sequence
of steps that update some state along the way so each of these steps
is a computation that returns some result but also it has a side effect
of updating some state so recall this idea that single value monads
are are containers that hold one value and have a context so the context
for the state model is this state which is a value of type s as a
fixed type the type a is a type parameter of the container but the
state the state value S is a fixed type and the context consists of
this state and of the fact that you can update it as you go so you
can change it and this is the type constructor again just like continuation
it\textsf{'}s not easy to understand why it must be like this I will talk about
this a little later at first let\textsf{'}s see how it works so one good example
of using the state monad is to implement a random number generator
so a random number generator were more stood more precisely a pseudo-random
number generator has an internal state that is updated every time
you ask it for a new random number so it gives you a result which
is a random number but at the same time it updates its internal state
so state monad allows you to do that in a pure functional way without
any mutability and also it has the same advantages as other moments
that you can combine things much easier so without using the state
moment how will we do this so imagine that all the functions are already
implemented so here\textsf{'}s how the PCG random is a very little package
that I wrote it\textsf{'}s has an initial state so the state is some values
initial default state or you can seed it with some other value you
want and then it has a function which I called in 32 that takes the
initial state and returns a pair of some value and a new state then
you would have to call this function again with this new state and
get some random value Y and a state as two now you have to call it
again with s2 and get Z and s 3 and so on so you would have to yourself
keep track of these s0 here as near here as 1 here as 1 here as 3
as 2 here is 2 here and so and if you make a mistake it\textsf{'}s very easy
to make a mistake the types are not going to help you avoid this mistake
if you here say as instead of this one the type is still the same
compiler would not the compiler will not notice so that is error-prone
and ugly the state monad hides this so it hides this as they say threading
of the state for putting you know keeping track of this new state
every time here\textsf{'}s how so in 32 is a function that takes internal state
and returns a tuple of internal state an empty out so actually I need
to interchange the order of these things just for clarity in 32 returns
a pair of {[}Music{]} internal state and a random integer so R\&D
is going to be the type that is a state monad so this is going to
be the factor in the factor block orangie and here\textsf{'}s how we can use
it now look at this code the state is magically handled all we do
is as if we just have a random number function that gets us a new
integer and this is defined here so it\textsf{'}s a constructor of the state
monad which has this type which was asked to escape and this is exactly
the type signature of this function s 2 s so this is the result the
result will be an already of string well you see orangie of in doesn't
mean that it\textsf{'}s a random integer this is a constructor this is a functor
the type of the value inside the furniture can be anything doesn't
have to be random the randomness is only in this piece that returns
random integers once you have those into you can make doubles out
of them course and whatever you want restraining assaulted them so
just to warn you what looked confused that somehow this type constructor
itself makes everything random inside it no it doesn't this is just
a statement it it just takes care of the state and that\textsf{'}s how you
use it now you actually have to run I'm using the cat\textsf{'}s State Mona
which has it\textsf{'}s only POA has a run so you have to run it on initial
state and then the result will be something of which you need to get
a value and then that will be a tuple of state and result so you take
a second part of the tuple and that\textsf{'}s your result so that\textsf{'}s the API
of cats state monad and you do this only once anyway you usually accumulate
a state value combine them together it\textsf{'}s very easy to combine because
monads are easy to combine you can put a state value on the right-hand
side of a generative area you have combines them just like we did
here so once you're ready to run them you perform the run and get
all of this done so just like the other non-standard containers computing
this value does not actually run any of these computations it creates
a function that will run the computations this function still needs
to be called and that call is usually called as usually run the run
method so now we have seen how these different non-standard containers
are used and in what sense they can be interpreted as containers that
hold one value together with a context or how they perform computations
in a context in this slide I will try to motivate why these types
must be like that you see for the writer maybe you understand why
it must be like this we need to log some information so let\textsf{'}s just
put this information into this type here but for these especially
these this is kind of clear you either have a value that\textsf{'}s already
computed or it\textsf{'}s postponed these types are far from clear why are
they like this why do these types embody what they do so here is how
I could kind of derive or motivate from first principles the choice
of these type constructors the main principle here is that we want
to use flatmap in order to chain computations together and so flatmap
is the function that\textsf{'}s going to be transforming some previous values
to some next values in this chain of computations so let\textsf{'}s apply this
reasoning and see what is the signal type signature of flatmap and
just by type reasoning we will derive the correct types start with
the writer monad so the this computation and there\textsf{'}s some information
about it the code we would write would be that first we have some
value of type a would say some X of type a and we transform it into
some f of X of type B so that\textsf{'}s our computation that goes from A to
B at the same time we compute some login information of type W using
the X so what we have here in other words is a pair of functions one
from A to B and one from a to W that is the kind of single step computation
that we want to chain together so these are the things that we want
to change our flat map should take this as its argument because our
principle is that we want to make it into a chain of flat Maps so
in other words this type must be the type of the argument of flat
map but we know that flat map has a type signature which looks like
this from a to writer B right so in other word more precisely the
argument of flat map has this type the flat map has an argument of
this type and it returns a writing B it has an argument of type writer
a an argument of this type and it returns Rho B so writer a yes we
already have writer a always as the first argument so in other words
this type must be the same as this if that is so we would be able
to use flat map to chain computations what is the type writer be so
that this type is equivalent to this one in order to decide type equivalence
we need to use the arithmetic very hard correspondence as I called
it in this correspondence the function types corresponds to miracle
powers so A to B corresponds to B to the power a a to W corresponds
to table to the power a and product is product now using school level
algebra we can simplify this and we say it\textsf{'}s like this it\textsf{'}s the identity
and by the arithmetic correspondence this is also the type equivalence
and so this is a type from a to product of BW and so from a to product
of BW hence writer B must be productive BW so in this way we derived
what the type of writer must be let\textsf{'}s apply the same reasoning to
the other units the reader for example we want to read only context
or environment of type B so what is an elementary step in the computation
we have some X of type a and we compute some value of type B but we
can use this R which is the read-only environment of type e in other
words our type of the elementary step on the computation is this we
repeat the same reason if we want to chain these computations using
flatmap it means that the argument of flatmap must be of this type
but the argument of flatmap is of this type in order to do that we
must have that these types are equivalent this can be done when reader
is the function it to be again we use the arithmetic very hard correspondence
this type is B to the power of product AE so this is just algebraic
notation in a usual arithmetic I shouldn't say algebraic arithmetic
notation this is just B to the power of a times e which represents
this type and what we want is that this is something to the power
a and so this is B to the e to the power a therefore we have the type
e B so again a function type e to B is represented as B to the power
E with reversing the order of this that is how the arithmetic Harvard
correspondence works for function types I discussed this in the third
chapter of the functional programming tutorial look at the continuation
what not now the continuation monad or a call back unit if you wish
it\textsf{'}s a computation that registers a call back that will be called
asynchronously asynchronously means it will be cold later at some
later time or maybe not at all or maybe once or more than once what
is the computation of this kind we take an X of type a and we call
some function on the call back so what does it mean to register a
callback we need to prepare some callback and pass it to somebody
now this is the somebody who will take our callback and they will
return a unit usually I mean they're not going to call the callback
right away anyway so that so this function of registering the callback
normally returns a unit well it could return a different type let\textsf{'}s
say indicating some error in registering the callback but let\textsf{'}s suppose
for simplicity it returns unit so the callback is of this type because
usually callbacks also return unit so what is the type of this elementary
step the type is this it\textsf{'}s from a to a function f that takes a callback
and returns unit so if this function head type eight account B we
would be able to use it in a flat map well this is clear can be must
be this and we can generalize this to return a non unit result type
in case that this should return some kind of result may be error messages
or something else a callback could return more information than just
unit so maybe there\textsf{'}s some error maybe some other other information
so usually the continuation mullet therefore is defined as this type
where ere R where R is a fixed result type finally let\textsf{'}s look at the
state monad state monad is a computation that can update the state
while producing a result so what does it mean such a computation looks
like this first we take some X of type a we take the previous state
of type s and we produce some new value of type B using those two
at the same time we produce a new value of type s from the previous
{[}Music{]} from the previous state so we at the same time change
the state and we compute a new value and while we are computing a
new value we can use the old state and while we're changing the state
we can use the value X that was given to us therefore the type of
the elementary computation step is actually this it\textsf{'}s a pair of two
functions from a and s to B and from a and s to s so the first function
computes a new result and the second function computes a new state
value we now we repeat the same argument this type must be the type
of the argument of flatmap in other words so what is state B if this
type must be equivalent to this let\textsf{'}s make a computation we again
use the Curie Howard correspondence this type corresponds to this
expression we wanted to be something to the power a in order to reduce
it to this form so we transform it like this something to the power
a and we can also simplify it like this something to the power in
translating back to types it means that this type is equivalent to
this in other words a going to this or more simply this so state B
is the type that must be this so we have derived the type constructors
for the monads from first principles just the principle was we have
some elementary step of computation that deals with some kind of context
together with a new value and we want this elementary step of the
computation to be usable under flatmap in other words the type of
this elementary step of the computation must be of the kind a to FB
where F is the function that is the type of the argument of want map
so systematically we demanded that the type was of that kind and that
allowed us straightforwardly to derive the types of these moments
you need to get used to these moments in order to be proficient with
them but I hope that this derivation kind of lifts the veil of mystery
from the question of why are the types of that\textsf{'}s chosen like this
and what is this context and they manipulate here are some exercises
for you to get more familiar with using walnuts and to implement simple
examples using set sequence future lists or try and state units as
well as implementing semi modded instances which means simply implementing
flatmap for certain type instructors that concludes part 1 of chapter
7
\end{comment}


\section{Laws of semimonads and monads}

\subsection{Motivation for the semimonad laws}

This chapter introduced semimonads to encode nested iteration as functor
block programs with multiple source lines. What properties do we intuitively
expect such programs to have?

When functor blocks describe iterations over data collections, a source
line \lstinline!x <- c! means that the value of \lstinline!x! iterates
over items in the collection \lstinline!c!. An assignment line \lstinline!y = f(x)!
means that we define a local variable \lstinline!y! to equal the
expression \lstinline!f(x)!. We expect to get the same result by
iterating over a collection \lstinline!c! whose values \lstinline!x!
were replaced by \lstinline!f(x)!, i.e., by iterating over \lstinline!c.map(f)!.
It means that the following two code fragments should always give
the same results:

\vspace{0.3\baselineskip}

\noindent \texttt{\textcolor{blue}{\footnotesize{}}}%
\begin{minipage}[c]{0.475\columnwidth}%
\begin{lstlisting}
val result1 = for {
  x <- c
  y = f(x)
  z <- g(y) // Same as z <- g(f(x)).
} yield z
 // Translating the functor block into methods:
val result1 = c.flatMap(x => g(f(x)))
\end{lstlisting}
%
\end{minipage}\texttt{\textcolor{blue}{\footnotesize{}\hspace*{\fill}}}%
\begin{minipage}[c]{0.475\columnwidth}%
\begin{lstlisting}
val result2 = for {
  y <- c.map(x =>
                 f(x))
  z <- g(y) // This code is unchanged.
 } yield z
 // Translating the functor block into methods:
val result2 = c.map(f).flatMap(y => g(y))
\end{lstlisting}
%
\end{minipage}{\footnotesize\par}

\vspace{0\baselineskip}

In the code just shown, an assignment line \lstinline!y = f(x)! occurs
before the last source line \lstinline!z <- p(y)!. The other possibility
is when the assignment line occurs \emph{after} the last source line.
A similar reasoning gives the requirement of equality between these
two code fragments:

\vspace{0.3\baselineskip}

\noindent \texttt{\textcolor{blue}{\footnotesize{}}}%
\begin{minipage}[c]{0.475\columnwidth}%
\begin{lstlisting}
val result1 = for {
  x <- c
  z <- f(x)
  y = g(z)
} yield y
 // Translating the functor block into methods:
val result1 = c.flatMap(f).map(g)
\end{lstlisting}
%
\end{minipage}\texttt{\textcolor{blue}{\footnotesize{}\hspace*{\fill}}}%
\begin{minipage}[c]{0.475\columnwidth}%
\begin{lstlisting}
val result2 = for {
  x <- c
  y <- f(x).map(z =>
                    g(z))
 } yield y
 // Translating the functor block into methods:
val result2 = c.flatMap { x => f(x).map(g) }
\end{lstlisting}
%
\end{minipage}\vspace{0\baselineskip}

Now consider the case when there are more than two levels of nested
iterations. Such programs are usually written as functor blocks with
three or more source lines. However, any subgroup of source lines
can be refactored into a separate functor block, and the result remains
the same:

\vspace{0.3\baselineskip}

\noindent \texttt{\textcolor{blue}{\footnotesize{}}}%
\begin{minipage}[c]{0.475\columnwidth}%
\begin{lstlisting}
val result1 = for {
  x <- c
  y <- f(x)
  z <- g(y)
} yield z
 // Translating the functor block into methods:
val result1 = c.flatMap{ x => f(x).flatMap(g) }
\end{lstlisting}
%
\end{minipage}\texttt{\textcolor{blue}{\footnotesize{}\hspace*{\fill}}}%
\begin{minipage}[c]{0.475\columnwidth}%
\begin{lstlisting}
val result2 = for {
  yy <- for { x <- c
              y <- f(x) } yield y
  z  <- g(yy)
 } yield z
 // Translating the functor block into methods:
val result2 = c.flatMap(f).flatMap(yy => g(yy))
\end{lstlisting}
%
\end{minipage}\vspace{0\baselineskip}

We obtained three general requirements for the \lstinline!flatMap!
method. Let us now formulate these requirements as equations (or \textsf{``}laws\textsf{''}).

For brevity, we will denote the \lstinline!flatMap! method for a
semimonad \lstinline!S[_]! by \textsf{``}$\text{flm}$\textsf{''}. We will write
$\text{flm}_{S}$ when we want to indicate explicitly the type constructor
($S$) being used. The type signature is:

\begin{wrapfigure}{l}{0.475\columnwidth}%
\vspace{-0.9\baselineskip}

\begin{lstlisting}
def flm[A, B](f: A => S[B]): S[A] => S[B] 
\end{lstlisting}

\vspace{-0.5\baselineskip}
\end{wrapfigure}%

~\vspace{-0.5\baselineskip}
\[
\text{flm}^{A,B}:(A\rightarrow S^{B})\rightarrow S^{A}\rightarrow S^{B}\quad.
\]
The first law is written in Scala code as:

\begin{wrapfigure}{l}{0.37\columnwidth}%
\vspace{-0.75\baselineskip}

\begin{lstlisting}
c.flatMap(x => g(f(x))) ==
                 c.map(f).flatMap(g)
\end{lstlisting}
\vspace{0.2\baselineskip}
\[
\xymatrix{\xyScaleY{0.2pc}\xyScaleX{3pc} & S^{B}\ar[rd]\sp(0.5){~~\ \text{flm}\,(g^{:B\rightarrow S^{C}})}\\
S^{A}\ar[ru]\sp(0.5){f^{\uparrow S}}\ar[rr]\sb(0.5){\text{flm}\,(f^{:A\rightarrow B}\bef\,g^{:B\rightarrow S^{C}})\,} &  & S^{C}
}
\]

\vspace{-0.6\baselineskip}
\end{wrapfigure}%

\noindent In the code notation, we write this law in the point-free
style\index{point-free style} by omitting the argument \lstinline!c!:

\begin{equation}
\text{flm}\,(f^{:A\rightarrow B}\bef g^{:B\rightarrow S^{C}})=f^{\uparrow S}\bef\text{flm}\,(g)\quad.\label{eq:left-naturality-law-flatMap}
\end{equation}

\noindent This equation holds for arbitrary $f^{:A\rightarrow B}$
and $g^{:B\rightarrow S^{C}}$. This is a \textsf{``}\textbf{left naturality}
law\textsf{''}\index{naturality law!of flatMap@of \texttt{flatMap}} of \lstinline!flatMap!
since it exchanges the order of lifted functions to the \emph{left}
of \lstinline!flatMap!. More precisely, we may call this equation
the naturality law of \lstinline!flatMap[A, B]! \textsf{``}with respect to
\lstinline!A!\textsf{''} since $f^{\uparrow S}$ acts on the type parameter
\lstinline!A!.

The second law holds for arbitrary $f^{:A\rightarrow S^{B}}$ and
$g^{:B\rightarrow C}$:

\begin{wrapfigure}{l}{0.4\columnwidth}%
\vspace{-0.75\baselineskip}

\begin{lstlisting}
c.flatMap(f).map(g) ==
         c.flatMap { x => f(x).map(g) }
\end{lstlisting}
\vspace{0.2\baselineskip}
\[
\xymatrix{\xyScaleY{0.2pc}\xyScaleX{3pc} & S^{B}\ar[rd]\sp(0.5){\ \,(g^{:B\rightarrow C})^{\uparrow S}}\\
S^{A}\ar[ru]\sp(0.5){\text{flm}\,(f^{:A\rightarrow S^{B}})~~~}\ar[rr]\sb(0.5){\text{flm}\,(f^{:A\rightarrow S^{B}}\bef\,g^{\uparrow S})\,} &  & S^{C}
}
\]

\vspace{-0.2\baselineskip}
\end{wrapfigure}%

\noindent \vspace{-0.35\baselineskip}
\begin{equation}
\text{flm}\,(f^{:A\rightarrow S^{B}}\bef g^{\uparrow S})=\text{flm}\,(f)\bef g^{\uparrow S}\quad.\label{eq:right-naturality-law-flatMap}
\end{equation}

\noindent This is a \index{naturality law!of flatMap@of \texttt{flatMap}}\textbf{right
naturality} law or \textsf{``}naturality with respect to \lstinline!B!\textsf{''}
of \lstinline!flatMap[A, B]!. It manipulates a lifted function $g^{\uparrow S}$
to the right of \lstinline!flatMap!, acting on the type parameter
\lstinline!B!.

The third law relates nested \lstinline!flatMap! operations:

\begin{wrapfigure}{l}{0.4\columnwidth}%
\vspace{-0.6\baselineskip}

\begin{lstlisting}
c.flatMap { x => f(x).flatMap(g) } ==
                c.flatMap(f).flatMap(g)
\end{lstlisting}
\vspace{0\baselineskip}
\[
\xymatrix{\xyScaleY{0.2pc}\xyScaleX{3pc} & S^{B}\ar[rd]\sp(0.5){~\ \text{flm}\,(g^{:B\rightarrow S^{C}})}\\
S^{A}\ar[ru]\sp(0.5){\text{flm}\,(f^{:A\rightarrow S^{B}})~~\ }\ar[rr]\sb(0.5){\text{flm}\,(f^{:A\rightarrow S^{B}}\bef\,\text{flm}\,(g))\,} &  & S^{C}
}
\]

\vspace{0.6\baselineskip}
\end{wrapfigure}%
~\vspace{-1.6\baselineskip}

\begin{equation}
\text{flm}\,\big(f^{:A\rightarrow S^{B}}\bef\text{flm}\,(g^{:B\rightarrow S^{C}})\big)=\text{flm}\left(f\right)\bef\text{flm}\left(g\right)\quad.\label{eq:associativity-law-flatMap}
\end{equation}

\vspace{-0.2\baselineskip}

This equation is called the \textbf{associativity law}\index{associativity law!of flatMap@of \texttt{flatMap}}
of \lstinline!flatMap!, for reasons we will explain later.

The three laws of semimonads express a programmer\textsf{'}s intuitive reasoning
about code. If a monad\textsf{'}s implementation does not obey one of these
laws, programs written with that monad may give wrong results even
though the code looks correct. To avoid those hard-to-debug errors,
we must verify that the code for every monad\textsf{'}s \lstinline!flatMap!
method obeys the laws. 

At this point, the three laws of semimonads may appear complicated
and hard to understand and to verify. In the next subsections, we
will derive a shorter and clearer formulation of those laws. For now,
let us define a \lstinline!Semimonad! typeclass\index{typeclass!Semimonad@\texttt{Semimonad}}
and test the laws using the \lstinline!scalacheck! library:\index{scalacheck library@\texttt{scalacheck} library}\index{verifying laws with scalacheck@verifying laws with \texttt{scalacheck}}
\begin{lstlisting}
abstract class Semimonad[F[_]: Functor] {
  def flatMap[A, B](fa: F[A])(f: A => F[B]): F[B]
}
implicit class SemimonadOps[F[_]: Semimonad, A](fa: F[A]) { // Define flatMap as an extension method.
  def flatMap[B](f: A => F[B]): F[B] = implicitly[Semimonad[F]].flatMap(fa)(f)
}
def checkSemimonadLaws[F[_], A, B, C]()(implicit ff: Semimonad[F],   // Use the `Arbitrary` typeclass
   fa: Arbitrary[F[A]], ab: Arbitrary[A => F[B]], bc: Arbitrary[B => F[C]]) = { // from `scalacheck`.
    forAll { (f: A => F[B], g: B => F[C], fa: F[A]) =>               // Associativity law of flatMap.
      fa.flatMap(x => f(x).flatMap(g)) shouldEqual fa.flatMap(f).flatMap(g)
    }
} // Assuming that a Semimonad instance was defined for Seq[_], check the laws with specific A, B, C.
checkSemimonadLaws[Seq, Int, String, Double]()
\end{lstlisting}


\subsection{The laws of \texttt{flatten}}

In Section~\ref{subsec:Simplifying-the-filtering-laws-deflate} we
simplified the laws of the \lstinline!filter! operation by passing
to a simpler \lstinline!deflate! function. We then showed that these
two functions are equivalent if certain laws are assumed to hold for
\lstinline!filter!. We will now derive a similar relationship between
the methods \lstinline!flatMap! and \lstinline!flatten!. We will
see that \lstinline!flatten! has fewer laws, and that its laws are
simpler to verify.

\subsubsection{Statement \label{subsec:Statement-flatten-equivalent-to-flatMap}\ref{subsec:Statement-flatten-equivalent-to-flatMap}}

The type of the method \lstinline!flatten! (denoted by $\text{ftn}^{A}:S^{S^{A}}\rightarrow S^{A}$)
is equivalent\index{type equivalence!examples} to the type of \lstinline!flatMap!
as long as \lstinline!flatMap! satisfies its left naturality law~(\ref{eq:left-naturality-law-flatMap}).

\subparagraph{Proof}

By definition, \lstinline!flatMap! is expressed as a composition
of \lstinline!map! and \lstinline!flatten!:

\begin{wrapfigure}{l}{0.4\columnwidth}%
\vspace{-0.85\baselineskip}

\begin{lstlisting}
c.flatMap(f) == c.map(f).flatten
\end{lstlisting}
\vspace{0.2\baselineskip}
\[
\xymatrix{\xyScaleY{0.3pc}\xyScaleX{3.5pc} & S^{S^{B}}\ar[rd]\sp(0.5){\ \text{ftn}\ }\\
S^{A}\ar[ru]\sp(0.5){(f^{:A\rightarrow S^{B}})^{\uparrow S}\ }\ar[rr]\sp(0.5){\text{flm}\,(f^{:A\rightarrow S^{B}})\,} &  & S^{B}
}
\]

\vspace{1.3\baselineskip}
\end{wrapfigure}%

~\vspace{-0.5\baselineskip}
\[
\text{flm}_{S}(f)=f^{\uparrow S}\bef\text{ftn}_{S}\quad.
\]

\noindent Substituting $f\triangleq\text{id}^{:S^{A}\rightarrow S^{A}}$
into this equation, we get:
\[
\text{flm}\left(\text{id}\right)=\gunderline{\text{id}^{\uparrow S}}\bef\text{ftn}=\text{ftn}\quad.
\]
This expresses \lstinline!flatten! through \lstinline!flatMap!.
It remains to show that the relationship between \lstinline!flatten!
and \lstinline!flatMap! is an isomorphism. For that, we need to prove
two properties:

\textbf{(1)} Starting with a given function $\text{ftn}:S^{S^{A}}\rightarrow S^{A}$,
define $\text{flm}\left(f\right)\triangleq f^{\uparrow S}\bef\text{ftn}$
and then define a new function $\text{ftn}^{\prime}\triangleq\text{flm}\left(\text{id}\right)$.
Prove that $\text{ftn}^{\prime}=\text{ftn}$:
\[
\text{ftn}^{\prime}=\text{flm}\left(\text{id}\right)=\gunderline{\text{id}^{\uparrow S}}\bef\text{ftn}=\text{id}\bef\text{ftn}=\text{ftn}\quad.
\]
If \lstinline!flatMap! is defined via \lstinline!flatten!, the left
naturality law of \lstinline!flatMap! is automatically satisfied:
\begin{align*}
{\color{greenunder}\text{expect to equal }\text{flm}\left(f\bef g\right):}\quad & f^{\uparrow S}\bef\text{flm}\left(g\right)=\gunderline{f^{\uparrow S}\bef g^{\uparrow S}}\bef\text{ftn}\\
{\color{greenunder}\text{composition law of }S:}\quad & =(f\bef g)^{\uparrow S}\bef\text{ftn}=\text{flm}\left(f\bef g\right)\quad.
\end{align*}

\textbf{(2)} Starting with a given function $\text{flm}$ that satisfies
the left naturality law, define $\text{ftn}\triangleq\text{flm}\left(\text{id}\right)$
and then define a new function $\text{flm}^{\prime}\left(f\right)\triangleq f^{\uparrow S}\bef\text{ftn}$.
Prove that $\text{flm}^{\prime}=\text{flm}$:
\begin{align*}
{\color{greenunder}\text{expect to equal }\text{flm}\left(f\right):}\quad & \text{flm}^{\prime}\left(f\right)=f^{\uparrow S}\bef\text{ftn}=f^{\uparrow S}\bef\text{flm}\left(\text{id}\right)\\
{\color{greenunder}\text{left naturality law of }\text{flm}:}\quad & =\text{flm}\left(f\bef\text{id}\right)=\text{flm}\left(f\right)\quad.
\end{align*}


\subsubsection{Statement \label{subsec:Statement-flatten-has-2-laws}\ref{subsec:Statement-flatten-has-2-laws}}

If a \lstinline!flatMap! function satisfies the three laws~(\ref{eq:left-naturality-law-flatMap})\textendash (\ref{eq:associativity-law-flatMap}),
the corresponding \lstinline!flatten! function defined as $\text{ftn}\triangleq\text{flm}\left(\text{id}\right)$
satisfies its \emph{two} laws, with an arbitrary $f^{:A\rightarrow B}$:
\begin{align}
{\color{greenunder}\text{naturality law of }\text{ftn}:}\quad & f^{\uparrow S\uparrow S}\bef\text{ftn}=\text{ftn}\bef f^{\uparrow S}\quad,\label{eq:naturality-law-of-flatten}\\
{\color{greenunder}\text{associativity law of }\text{ftn}:}\quad & \text{ftn}^{\uparrow S}\bef\text{ftn}=\text{ftn}\bef\text{ftn}\quad.\label{eq:associativity-law-of-flatten}
\end{align}
The following type diagrams illustrate these laws:

\vspace{-1.5\baselineskip}
\begin{minipage}[t]{0.45\columnwidth}%
\[
\xymatrix{\xyScaleY{0.2pc}\xyScaleX{5pc} & S^{S^{B}}\ar[rd]\sp(0.5){\ \text{ftn}^{B}}\\
S^{S^{A}}\ar[ru]\sp(0.5){(f^{:A\rightarrow B})^{\uparrow S\uparrow S}\ \ }\ar[rd]\sb(0.5){\text{ftn}^{A}\,} & \hspace{-6em}{\color{teal}\text{(naturality)}}\hspace{-6em} & S^{B}\\
 & S^{A}\ar[ru]\sb(0.5){(f^{:A\rightarrow B})^{\uparrow S}}
}
\]
%
\end{minipage}\hfill{}%
\begin{minipage}[t]{0.45\columnwidth}%
\[
\xymatrix{\xyScaleY{0.2pc}\xyScaleX{5pc} & S^{S^{A}}\ar[rd]\sp(0.5){\ \text{ftn}^{A}}\\
S^{S^{S^{A}}}\ar[ru]\sp(0.5){(\text{ftn}^{A})^{\uparrow S}\ }\ar[rd]\sb(0.5){\text{ftn}^{S^{A}}\,} & \hspace{-6em}{\color{teal}\text{(associativity)}}\hspace{-6em} & S^{A}\\
 & S^{S^{A}}\ar[ru]\sb(0.5){\text{ftn}^{A}}
}
\]
%
\end{minipage}

Note that the associativity law involves two intermediate values of
type $S^{S^{A}}$ that are \emph{not} necessarily equal. The associativity
law requires only the final results (of type $S^{A}$) to be equal.

\subparagraph{Proof}

By Statement~\ref{subsec:Statement-flatten-equivalent-to-flatMap},
we have $\text{flm}\left(f\right)=f^{\uparrow S}\bef\text{ftn}$ since
it is given that the left naturality law of \lstinline!flatMap! holds.
Substituting that into the other two laws of \lstinline!flatMap!,
we get:
\begin{align*}
{\color{greenunder}\text{right naturality law of }\text{flm}:}\quad & \text{flm}\,\big(f\bef g^{\uparrow S}\big)=\text{flm}\left(f\right)\bef g^{\uparrow S}\quad,\\
{\color{greenunder}\text{substitute \ensuremath{\text{flm}} via }\text{ftn}:}\quad & \big(f\bef g^{\uparrow S}\big)^{\uparrow S}\bef\text{ftn}=f^{\uparrow S}\bef\text{ftn}\bef g^{\uparrow S}\quad,\\
{\color{greenunder}\text{set }f\triangleq\text{id}:}\quad & g^{\uparrow S\uparrow S}\bef\text{ftn}=\text{ftn}\bef g^{\uparrow S}\quad,
\end{align*}
which shows that Eq.~(\ref{eq:naturality-law-of-flatten}) holds;
and finally:
\begin{align*}
{\color{greenunder}\text{associativity law of }\text{flm}:}\quad & \text{flm}\,\big(f\bef\text{flm}\,(g)\big)=\text{flm}\left(f\right)\bef\text{flm}\left(g\right)\quad,\\
{\color{greenunder}\text{substitute \ensuremath{\text{flm}} via }\text{ftn}:}\quad & \big(f\bef g^{\uparrow S}\bef\text{flm}\,(g)\big)^{\uparrow S}\bef\text{ftn}=\big(\gunderline{f\bef g^{\uparrow S}}\bef\text{ftn}\big)^{\uparrow S}\bef\text{ftn}=\gunderline{f^{\uparrow S}}\bef\text{ftn}\bef\gunderline{g^{\uparrow S}}\bef\text{ftn}\quad,\\
{\color{greenunder}\text{set }f\triangleq\text{id}\text{ and }g\triangleq\text{id}:}\quad & \text{ftn}^{\uparrow S}\bef\text{ftn}=\text{ftn}\bef\text{ftn}\quad.
\end{align*}
This verifies Eq.~(\ref{eq:associativity-law-of-flatten}).

\subsubsection{Statement \label{subsec:Statement-flatten-laws-to-flatMap-laws}\ref{subsec:Statement-flatten-laws-to-flatMap-laws}}

If a \lstinline!flatten! function satisfies the laws~(\ref{eq:naturality-law-of-flatten})\textendash (\ref{eq:associativity-law-of-flatten})
then the corresponding \lstinline!flatMap! function defined by $\text{flm}\left(f\right)\triangleq f^{\uparrow S}\bef\text{ftn}$
will satisfy its three laws~(\ref{eq:left-naturality-law-flatMap})\textendash (\ref{eq:associativity-law-flatMap}).

\subparagraph{Proof}

By Statement~\ref{subsec:Statement-flatten-equivalent-to-flatMap},
the left naturality law holds. To check the right naturality law:
\begin{align*}
{\color{greenunder}\text{expect to equal }\text{flm}\left(f\right)\bef g^{\uparrow S}:}\quad & \text{flm}\,\big(f\bef g^{\uparrow S}\big)=\gunderline{\big(f\bef g^{\uparrow S}\big)^{\uparrow S}}\bef\text{ftn}=f^{\uparrow S}\bef\gunderline{g^{\uparrow S\uparrow S}\bef\text{ftn}}\\
{\color{greenunder}\text{naturality law of }\text{ftn}:}\quad & =\gunderline{f^{\uparrow S}\bef\text{ftn}}\bef g^{\uparrow S}=\text{flm}\left(f\right)\bef g^{\uparrow S}\quad.
\end{align*}
To check the associativity law:
\begin{align*}
{\color{greenunder}\text{expect to equal }\text{flm}\left(f\right)\bef\text{flm}\left(g\right):}\quad & \text{flm}\,\big(f\bef\text{flm}\,(g)\big)=\big(f\bef\text{flm}\,(g)\big)^{\uparrow S}\bef\text{ftn}=\big(f\bef g^{\uparrow S}\bef\gunderline{\text{ftn}\big)^{\uparrow S}\bef\text{ftn}}\\
{\color{greenunder}\text{associativity law of }\text{ftn}:}\quad & =f^{\uparrow S}\bef\gunderline{g^{\uparrow S\uparrow S}\bef\text{ftn}}\bef\text{ftn}\\
{\color{greenunder}\text{naturality law of }\text{ftn}:}\quad & =\gunderline{f^{\uparrow S}\bef\text{ftn}}\bef\gunderline{g^{\uparrow S}\bef\text{ftn}}=\text{flm}\left(f\right)\bef\text{flm}\left(g\right)\quad.
\end{align*}
 

We have proved that \lstinline!flatten! is equivalent to \lstinline!flatMap!
but has fewer laws.

By the \index{parametricity theorem}parametricity theorem (see Appendix~\ref{app:Proofs-of-naturality-parametricity}),
any purely functional code will obey naturality laws. All semimonads
and monads in this chapter (except \lstinline!Future!) are purely
functional, so we will not need to verify their naturality laws. However,
verifying the associativity law involves complicated derivations.
Using \lstinline!flatten! instead of \lstinline!flatMap! will often
make those derivations shorter.

\subsection{Verifying the associativity law via \texttt{flatten}}

\begin{comment}
The associativity law of \lstinline!flatMap! was demonstrated to
hold for the \lstinline!Option! monad in Chapter~\ref{chap:Filterable-functors}
(see Eq.~(\ref{eq:associativity-law-of-flatMap-for-Option}) in the
proof of Statement~\ref{subsec:Statement-filterable-coproduct-1}). 
\end{comment}
The following examples will verify the associativity law of \lstinline!flatten!
for some standard monads.

\subsubsection{Example \label{subsec:Example-flatten-verify-for-monad}\ref{subsec:Example-flatten-verify-for-monad}\index{solved examples}}

The standard Scala types \lstinline!Either! and \lstinline!Try!
are examples of the monad $F^{A}\triangleq Z+A$, where $Z$ is a
fixed type. Show that this monad satisfies the associativity law.

\subparagraph{Solution}

The type signature of \lstinline!flatten! is $\text{ftn}:Z+\left(Z+A\right)\rightarrow Z+A$,
and its code is:

\begin{wrapfigure}{l}{0.6\columnwidth}%
\vspace{-0.2\baselineskip}
\begin{lstlisting}
def flatten[A]: Either[Z, Either[Z, A]] => Either[Z, A] = {
  case Left(z)           => Left(z)
  case Right(Left(z))    => Left(z)
  case Right(Right(a))   => Right(a)
}
\end{lstlisting}

\vspace{-3.2\baselineskip}
\end{wrapfigure}%

~\vspace{-1.2\baselineskip}
\[
\text{ftn}^{:Z+Z+A\rightarrow Z+A}\triangleq\,\begin{array}{|c||cc|}
 & Z & A\\
\hline Z & \text{id} & \bbnum 0\\
Z & \text{id} & \bbnum 0\\
A & \bbnum 0 & \text{id}
\end{array}\quad.
\]
\vspace{-0.8\baselineskip}

Since \lstinline!flatten! is fully parametric, both sides of the
law are fully parametric functions with the type signature $Z+Z+Z+A\rightarrow Z+A$.
This type signature has \emph{only one} fully parametric implementation:
since it is not possible to produce values of unknown types $A$ and
$Z$ from scratch, an implementation of $Z+Z+Z+A\rightarrow Z+A$
must return $Z+\bbnum 0$ when the input contains a value of type
$Z$; otherwise it must return $\bbnum 0+A$. So, both sides of the
law must have the same code.

To prove rigorously that only one implementation exists, we must use
the Curry-Howard correspondence and a decision algorithm for constructive
logic. Instead, let us verify the associativity law by an explicit
derivation. First, we need to lift \lstinline!flatten! to the functor
$F$. The lifting code is:
\[
(f^{:A\rightarrow B})^{\uparrow F}\triangleq\,\begin{array}{|c||cc|}
 & Z & B\\
\hline Z & \text{id} & \bbnum 0\\
A & \bbnum 0 & f
\end{array}\quad,\quad\quad\text{ftn}^{\uparrow F}=\,\begin{array}{|c||ccc|}
 & Z & Z & A\\
\hline Z & \text{id} & \bbnum 0 & \bbnum 0\\
Z & \bbnum 0 & \text{id} & \bbnum 0\\
Z & \bbnum 0 & \text{id} & \bbnum 0\\
A & \bbnum 0 & \bbnum 0 & \text{id}
\end{array}\quad.
\]
For comparison, the Scala code for $\text{ftn}^{\uparrow F}$ (had
we needed to write it) would look like this:
\begin{lstlisting}
def fmapFlatten[A]: Either[Z, Either[Z, Either[Z, A]]] => Either[Z, Either[Z, A]] = {
  case Left(z)                  => Left(z)
  case Right(Left(z))           => Right(Left(z))
  case Right(Right(Left(z)))    => Right(Left(z))
  case Right(Right(Right(a)))   => Right(Right(a))
}
\end{lstlisting}

Now we can compute the two sides of the associativity law~(\ref{eq:associativity-law-of-flatten})
via matrix composition:
\begin{align*}
{\color{greenunder}\text{left-hand side}:}\quad & \text{ftn}^{\uparrow F}\bef\text{ftn}=\,\begin{array}{|c||ccc|}
 & Z & Z & A\\
\hline Z & \text{id} & \bbnum 0 & \bbnum 0\\
Z & \bbnum 0 & \text{id} & \bbnum 0\\
Z & \bbnum 0 & \text{id} & \bbnum 0\\
A & \bbnum 0 & \bbnum 0 & \text{id}
\end{array}\,\bef\,\begin{array}{|c||cc|}
 & Z & A\\
\hline Z & \text{id} & \bbnum 0\\
Z & \text{id} & \bbnum 0\\
A & \bbnum 0 & \text{id}
\end{array}\,=\,\begin{array}{|c||cc|}
 & Z & A\\
\hline Z & \text{id} & \bbnum 0\\
Z & \text{id} & \bbnum 0\\
Z & \text{id} & \bbnum 0\\
A & \bbnum 0 & \text{id}
\end{array}\quad,
\end{align*}
\begin{align*}
{\color{greenunder}\text{right-hand side}:}\quad & \text{ftn}^{:Z+Z+\left(Z+A\right)\rightarrow Z+\left(Z+A\right)}\bef\text{ftn}=\,\begin{array}{|c||cc|}
 & Z & \gunderline{Z+A}\\
\hline Z & \text{id} & \bbnum 0\\
Z & \text{id} & \bbnum 0\\
\gunderline{Z+A} & \bbnum 0 & \gunderline{\text{id}}
\end{array}\,\bef\,\begin{array}{|c||cc|}
 & Z & A\\
\hline Z & \text{id} & \bbnum 0\\
Z & \text{id} & \bbnum 0\\
A & \bbnum 0 & \text{id}
\end{array}\\
{\color{greenunder}\text{expand }\text{id}^{Z+A}:}\quad & \quad=\,\begin{array}{|c||ccc|}
 & Z & Z & A\\
\hline Z & \text{id} & \bbnum 0 & \bbnum 0\\
Z & \text{id} & \bbnum 0 & \bbnum 0\\
Z & \bbnum 0 & \text{id} & \bbnum 0\\
A & \bbnum 0 & \bbnum 0 & \text{id}
\end{array}\,\bef\,\begin{array}{|c||cc|}
 & Z & A\\
\hline Z & \text{id} & \bbnum 0\\
Z & \text{id} & \bbnum 0\\
A & \bbnum 0 & \text{id}
\end{array}=\,\begin{array}{|c||cc|}
 & Z & A\\
\hline Z & \text{id} & \bbnum 0\\
Z & \text{id} & \bbnum 0\\
Z & \text{id} & \bbnum 0\\
A & \bbnum 0 & \text{id}
\end{array}\quad.
\end{align*}
The two sides of the associativity law are equal.

When it works, the technique of Curry-Howard code inference\index{code inference}
gives much shorter proofs than explicit derivations:

\subsubsection{Example \label{subsec:Example-flatten-verify-for-monad-1}\ref{subsec:Example-flatten-verify-for-monad-1}}

Verify that the \lstinline!Reader! monad, $F^{A}\triangleq Z\rightarrow A$,
satisfies the associativity law.

\subparagraph{Solution}

The type signature of \lstinline!flatten! is $\left(Z\rightarrow Z\rightarrow A\right)\rightarrow Z\rightarrow A$.
Both sides of the law~(\ref{eq:associativity-law-of-flatten}) are
functions with the type signature $(Z\rightarrow Z\rightarrow Z\rightarrow A)\rightarrow Z\rightarrow A$.
By code inference with typed holes, we find that there is only one
fully parametric implementation of this type signature, namely:
\[
p^{:Z\rightarrow Z\rightarrow Z\rightarrow A}\rightarrow z^{:Z}\rightarrow p(z)(z)(z)\quad.
\]
So, both sides of the law must have the same code, and the law holds.

\subsubsection{Example \label{subsec:Example-flatten-verify-for-monad-1-1}\ref{subsec:Example-flatten-verify-for-monad-1-1}}

Show that the \lstinline!List! monad\index{monads!List monad@\texttt{List} monad}
($F^{A}\triangleq\text{List}^{A}$) satisfies the associativity law.

\subparagraph{Solution}

The \lstinline!flatten[A]! method has the type signature $\text{ftn}^{A}:\text{List}^{\text{List}^{A}}\rightarrow\text{List}^{A}$
and concatenates the nested lists in their order. Let us first show
a more visual (but less formal) proof of the associativity law. Both
sides of the law are functions of type $\text{List}^{\text{List}^{\text{List}^{A}}}\rightarrow\text{List}^{A}$.
We can visualize how both sides of the law are applied to a triple-nested
list value $p$ defined by:
\[
p\triangleq\left[\left[\left[x_{11},x_{12},...\right],\left[x_{21},x_{22},...\right],...\right],\left[\left[y_{11},y_{12},...\right],\left[y_{21},y_{22},...\right],...\right],...\right]\quad,
\]
where all $x_{ij}$, $y_{ij}$, $...$ have type $A$. Applying $\text{ftn}^{\uparrow\text{List}}$
flattens the inner lists and produces:
\[
p\triangleright\text{ftn}^{\uparrow\text{List}}=\left[\left[x_{11},x_{12},...,x_{21},x_{22},...\right],\left[y_{11},y_{12},...,y_{21},y_{22},...\right],...\right]\quad.
\]
Flattening that result gives a list of all values $x_{ij}$, $y_{ij}$,
..., in the order they appear in $p$:
\[
p\triangleright\text{ftn}^{\uparrow\text{List}}\triangleright\text{ftn}=\left[x_{11},x_{12},...,x_{21},x_{22},...,y_{11},y_{12},...,y_{21},y_{22},...,...\right]\quad.
\]
Applying $\text{ftn}^{\text{List}^{A}}$ to $p$ will flatten the
outer lists:
\[
p\triangleright\text{ftn}^{\text{List}^{A}}=\left[\left[x_{11},x_{12},...\right],\left[x_{21},x_{22},...\right],...,\left[y_{11},y_{12},...\right],\left[y_{21},y_{22},...\right],...\right]\quad.
\]
Flattening that value results in $p\triangleright\text{ftn}^{\text{List}^{A}}\triangleright\text{ftn}=\left[x_{11},x_{12},...,x_{21},x_{22},...,y_{11},y_{12},...,y_{21},y_{22},...,...\right]$.
This is exactly the same as $p\triangleright\text{ftn}^{\uparrow\text{List}}\triangleright\text{ftn}$:
namely, the list of all values in the order they appear in $p$.

Let us now prove the associativity law by an explicit derivation.
Using the recursive type definition $\text{List}^{A}\triangleq\bbnum 1+A\times\text{List}^{A}$,
we can define \lstinline!flatten! as a recursive function:
\[
\text{ftn}^{A}\triangleq\,\begin{array}{|c||c|}
 & \bbnum 1+\text{List}^{A}\times\text{List}^{\text{List}^{A}}\\
\hline \bbnum 1 & 1\rightarrow1+\bbnum 0\\
\text{List}^{A}\times\text{List}^{\text{List}^{A}} & h^{:\text{List}^{A}}\times t^{:\text{List}^{\text{List}^{A}}}\rightarrow h\pplus(t\triangleright\overline{\text{ftn}})
\end{array}\quad,
\]
where we have used the function \lstinline!concat! (denoted $\pplus$)
whose associativity property was derived in Statement~\ref{subsec:Statement-concat-array-associativity}.
The operation of lifting to the \lstinline!List! functor is defined
for arbitrary functions $f$ by:
\[
(f^{:A\rightarrow B})^{\uparrow\text{List}}\triangleq\,\begin{array}{|c||cc|}
 & \bbnum 1 & B\times\text{List}^{B}\\
\hline \bbnum 1 & \text{id} & \bbnum 0\\
A\times\text{List}^{A} & \bbnum 0 & h^{:A}\times t^{:\text{List}^{A}}\rightarrow f(h)\times\big(t\triangleright f^{\overline{\uparrow\text{List}}}\big)
\end{array}\quad.
\]
The proof is by induction; the overline denotes recursive function
calls, for which all laws hold by inductive assumption. We write the
code for $\text{ftn}^{\uparrow\text{List}}$ and $\text{ftn}^{\text{List}^{A}}$,
both of type $\text{List}^{\text{List}^{\text{List}^{A}}}\rightarrow\text{List}^{\text{List}^{A}}$:
\begin{align*}
\text{ftn}^{\uparrow\text{List}} & =\,\begin{array}{|c||cc|}
 & \bbnum 1 & \text{List}^{A}\times\text{List}^{\text{List}^{A}}\\
\hline \bbnum 1 & \text{id} & \bbnum 0\\
\text{List}^{\text{List}^{A}}\times\text{List}^{\text{List}^{\text{List}^{A}}} & \bbnum 0 & h^{:\text{List}^{\text{List}^{A}}}\times t^{:\text{List}^{\text{List}^{\text{List}^{A}}}}\rightarrow\text{ftn}\,(h)\times\big(t\triangleright\text{ftn}^{\overline{\uparrow\text{List}}}\big)
\end{array}\quad,\\
\text{ftn}^{\text{List}^{A}} & =\,\,\begin{array}{|c||c|}
 & \bbnum 1+\text{List}^{A}\times\text{List}^{\text{List}^{A}}\\
\hline \bbnum 1 & 1\rightarrow1+\bbnum 0\\
\text{List}^{\text{List}^{A}}\times\text{List}^{\text{List}^{\text{List}^{A}}} & h^{:\text{List}^{\text{List}^{A}}}\times t^{:\text{List}^{\text{List}^{\text{List}^{A}}}}\rightarrow h\pplus\big(t\triangleright\overline{\text{ftn}}^{\text{List}^{A}}\big)
\end{array}\quad.
\end{align*}
It remains to compute $\text{ftn}^{\uparrow\text{List}}\bef\text{ftn}^{A}$
and $\text{ftn}^{\text{List}^{A}}\bef\text{ftn}^{A}$ via matrix composition.
The first one is quick:
\begin{align*}
 & \text{ftn}^{\uparrow\text{List}}\bef\text{ftn}^{A}=\,\begin{array}{||cc|}
\text{id} & \bbnum 0\\
\bbnum 0 & h\times t\rightarrow\text{ftn}\,(h)\times\big(t\triangleright\text{ftn}^{\overline{\uparrow\text{List}}}\big)
\end{array}\,\bef\,\begin{array}{||c|}
1\rightarrow1+\bbnum 0\\
h\times t\rightarrow h\pplus(t\triangleright\overline{\text{ftn}})
\end{array}\\
 & \quad=\,\begin{array}{|c||c|}
 & \bbnum 1+A\times\text{List}^{A}\\
\hline \bbnum 1 & 1\rightarrow1+\bbnum 0\\
\text{List}^{\text{List}^{A}}\times\text{List}^{\text{List}^{\text{List}^{A}}} & h\times t\rightarrow\text{ftn}\,(h)\pplus\big(t\triangleright\text{ftn}^{\overline{\uparrow\text{List}}}\triangleright\overline{\text{ftn}}\big)
\end{array}\quad.
\end{align*}
The second calculation gets stuck because the code matrix for $\text{ftn}^{\text{List}^{A}}$
has an unsplit column:
\[
\text{ftn}^{\text{List}^{A}}\bef\text{ftn}^{A}=\,\begin{array}{|c||c|}
 & \bbnum 1+\text{List}^{A}\times\text{List}^{\text{List}^{A}}\\
\hline \bbnum 1 & 1\rightarrow1+\bbnum 0\\
\text{List}^{\text{List}^{A}}\times\text{List}^{\text{List}^{\text{List}^{A}}} & h\times t\rightarrow h\pplus\big(t\triangleright\overline{\text{ftn}}^{\text{List}^{A}}\big)
\end{array}\,\bef\,\begin{array}{||c|}
1\rightarrow1+\bbnum 0^{:A\times\text{List}^{A}}\\
h\times t\rightarrow h\pplus(t\triangleright\overline{\text{ftn}})
\end{array}\quad.
\]
We cannot split the column because the expression $h\pplus\big(t\triangleright\overline{\text{ftn}}^{\text{List}^{A}}\big)$
may evaluate to any part of the disjunction $\bbnum 1+\text{List}^{A}\times\text{List}^{\text{List}^{A}}$
depending on the values of $h$ and $t$. We are unable to compose
the result with the second matrix unless we identify that part of
the disjunction and then substitute into the first or the second column
of that matrix. A general recipe in such situations is to perform
additional pattern matching on the arguments of the function and to
substitute the results into the matrix. Consider the value of $h:\text{List}^{\text{List}^{A}}$,
which can be an empty list, $h=1+\bbnum 0$, or a product, $h=\bbnum 0+g^{:\text{List}^{A}}\times k^{:\text{List}^{\text{List}^{A}}}$.
Substitute these possibilities into the matrix expression for $\text{ftn}^{\text{List}^{A}}\bef\text{ftn}^{A}$:
\begin{align*}
{\color{greenunder}\text{with }h=1+\bbnum 0:}\quad & \left(\bbnum 0+\left(1+\bbnum 0\right)\times t\right)\triangleright\text{ftn}^{\text{List}^{A}}\bef\text{ftn}^{A}=\gunderline{\left(1+\bbnum 0\right)\pplus}\big(t\triangleright\overline{\text{ftn}^{\text{List}^{A}}}\bef\text{ftn}^{A}\big)\\
{\color{greenunder}\text{concatenate with empty list}:}\quad & =t\triangleright\overline{\text{ftn}^{\text{List}^{A}}}\triangleright\text{ftn}^{A}\quad.
\end{align*}
Substituting this value $h$ into $\text{ftn}^{\uparrow\text{List}}\bef\text{ftn}^{A}$,
we get:
\[
\left(\bbnum 0+\left(1+\bbnum 0\right)\times t\right)\triangleright\text{ftn}^{\uparrow\text{List}}\bef\text{ftn}^{A}=t\triangleright\overline{\text{ftn}^{\overline{\uparrow\text{List}}}}\triangleright\overline{\text{ftn}}\quad.
\]
Now we just need to show that:
\[
t\triangleright\overline{\text{ftn}^{\text{List}^{A}}}\triangleright\text{ftn}^{A}\overset{?}{=}t\triangleright\overline{\text{ftn}^{\overline{\uparrow\text{List}}}}\triangleright\overline{\text{ftn}}\quad.
\]
This holds by the inductive assumption.

It remains to examine the second possibility, $h=\bbnum 0+g\times k$:
\begin{align*}
{\color{greenunder}\text{with }h=\bbnum 0+g\times k:}\quad & \left(\bbnum 0+\left(\bbnum 0+g\times k\right)\times t\right)\triangleright\text{ftn}^{\text{List}^{A}}\bef\text{ftn}^{A}=\big(\left(\bbnum 0+g\times k\right)\pplus\big(t\triangleright\overline{\text{ftn}^{\text{List}^{A}}}\big)\big)\triangleright\text{ftn}^{A}\\
{\color{greenunder}\text{code of }\pplus:}\quad & =\big(\bbnum 0+g\times\big(k\pplus\big(t\triangleright\overline{\text{ftn}^{\text{List}^{A}}}\big)\big)\big)\triangleright\text{ftn}^{A}\\
{\color{greenunder}\text{code of }\text{ftn}^{A}:}\quad & =g\pplus\big(k\pplus\big(t\triangleright\overline{\text{ftn}^{\text{List}^{A}}}\big)\big)\triangleright\overline{\text{ftn}}\\
{\color{greenunder}\text{Exercise~\ref{subsec:Exercise-flatten-concat-distributive-law}}:}\quad & =\gunderline{g\pplus\big(k\triangleright\overline{\text{ftn}}\big)}\pplus\big(t\triangleright\overline{\text{ftn}^{\text{List}^{A}}}\triangleright\overline{\text{ftn}}\big)=\left(\bbnum 0+g\times k\right)\triangleright\overline{\text{ftn}}\pplus\big(\gunderline{t\triangleright\overline{\text{ftn}^{\text{List}^{A}}}\triangleright\overline{\text{ftn}}}\big)\\
{\color{greenunder}\text{inductive assumption}:}\quad & =\text{ftn}\,(h)\pplus\big(t\triangleright\overline{\text{ftn}^{\uparrow\text{List}}}\triangleright\overline{\text{ftn}}\big)\quad.
\end{align*}
This is the same as the result of substituting $\bbnum 0+h\times t$
into $\text{ftn}^{\uparrow\text{List}}\bef\text{ftn}^{A}$.

\subsubsection{Example \label{subsec:Example-flatten-verify-for-monad-5}\ref{subsec:Example-flatten-verify-for-monad-5}}

Consider the \lstinline!List! type constructor with a \lstinline!flatten!
method that concatenates the nested lists in \emph{reverse} order.
Show that this implementation violates the associativity law of \lstinline!flatten!.

\subparagraph{Solution}

Apply both sides of the law to the nested list $p\triangleq\left[\left[\left[a,b\right],\left[c,d\right]\right],\left[\left[e,f\right],\left[g,h\right]\right]\right]$:
\begin{align*}
 & p\triangleright\text{ftn}^{\uparrow\text{List}}=\left[\left[c,d,a,b\right],\left[g,h,e,f\right]\right]\quad,\\
{\color{greenunder}\text{left-hand side}:}\quad & p\triangleright\text{ftn}^{\uparrow\text{List}}\triangleright\text{ftn}=\left[g,h,e,f,c,d,a,b\right]\quad,\\
 & p\triangleright\text{ftn}^{\text{List}^{A}}=\left[\left[e,f\right],\left[g,h\right],\left[a,b\right],\left[c,d\right]\right]\quad,\\
{\color{greenunder}\text{right-hand side}:}\quad & p\triangleright\text{ftn}^{\text{List}^{A}}\triangleright\text{ftn}=\left[c,d,a,b,g,h,e,f\right]\quad.
\end{align*}
The results are not the same. So, $p$ is a counterexample that refutes
the associativity law.

\subsubsection{Example \label{subsec:Example-flatten-verify-for-monad-2}\ref{subsec:Example-flatten-verify-for-monad-2}}

Show that the \lstinline!Writer! semimonad, $F^{A}\triangleq A\times W$,
is lawful if $W$ is a semigroup.

\subparagraph{Solution}

The type signature of \lstinline!flatten! is $\text{ftn}:\left(A\times W\right)\times W\rightarrow A\times W$,
and the code is:
\[
\text{ftn}:\left(a\times w_{1}\right)\times w_{2}\rightarrow a\times\left(w_{1}\oplus w_{2}\right)\quad,
\]
where $\oplus$ is the binary operation of the semigroup $W$. The
lifting to $F$ is $f^{\uparrow F}=a\times w\rightarrow\left(a\triangleright f\right)\times w$,
so:
\[
\text{ftn}^{\uparrow F}=\left(\left(a\times w_{1}\right)\times w_{2}\right)\times w_{3}\rightarrow\left(\left(\left(a\times w_{1}\right)\times w_{2}\right)\triangleright\text{ftn}\right)\times w_{3}=\left(a\times\left(w_{1}\oplus w_{2}\right)\right)\times w_{3}\quad.
\]
To verify the associativity law, it is convenient to substitute a
value $\left(\left(a\times w_{1}\right)\times w_{2}\right)\times w_{3}$
of type $F^{F^{F^{A}}}$ into both sides of the law:
\begin{align*}
 & \left(\left(\left(a\times w_{1}\right)\times w_{2}\right)\times w_{3}\right)\triangleright\text{ftn}^{\uparrow F}\triangleright\text{ftn}=\left(\left(a\times\left(w_{1}\oplus w_{2}\right)\right)\times w_{3}\right)\triangleright\text{ftn}=a\times\left(\left(w_{1}\oplus w_{2}\right)\oplus w_{3}\right)\quad,\\
 & \left(\left(\left(a\times w_{1}\right)\times w_{2}\right)\times w_{3}\right)\triangleright\text{ftn}^{F^{A}}\triangleright\text{ftn}=\left(\left(a\times w_{1}\right)\times\left(w_{2}\oplus w_{3}\right)\right)\triangleright\text{ftn}=a\times\left(w_{1}\oplus\left(w_{2}\oplus w_{3}\right)\right)\quad.
\end{align*}
The operation $\oplus$ is associative since $W$ is a semigroup.
So, both sides of the law are equal.

\subsubsection{Example \label{subsec:Example-flatten-verify-for-monad-4}\ref{subsec:Example-flatten-verify-for-monad-4}}

Consider the functor $F^{A}\triangleq A\times V\times V$ with the
\lstinline!flatten! method defined by:
\[
\text{ftn}\triangleq\big(a^{:A}\times u_{1}^{:V}\times u_{2}^{:V}\big)\times v_{1}^{:V}\times v_{2}^{:V}\rightarrow a\times v_{2}\times v_{1}\quad.
\]
Show that this definition violates the associativity law of \lstinline!flatten!.

\subparagraph{Solution}

Substitute a value of type $F^{F^{F^{A}}}$ into both sides of the
law and get unequal results:
\begin{align*}
 & \left(\left(\left(a\times u_{1}\times u_{2}\right)\times v_{1}\times v_{2}\right)\times w_{1}\times w_{2}\right)\triangleright\text{ftn}^{\uparrow F}\triangleright\text{ftn}=\left(\left(a\times v_{2}\times v_{1}\right)\times w_{1}\times w_{2}\right)\triangleright\text{ftn}=a\times w_{2}\times w_{1}\quad,\\
 & \left(\left(\left(a\times u_{1}\times u_{2}\right)\times v_{1}\times v_{2}\right)\times w_{1}\times w_{2}\right)\triangleright\text{ftn}^{F^{A}}\triangleright\text{ftn}=\left(\left(a\times u_{1}\times u_{2}\right)\times w_{2}\times w_{1}\right)\triangleright\text{ftn}=a\times w_{1}\times w_{2}\quad.
\end{align*}
Had the code not exchanged the order of $w_{1}$ and $w_{2}$, the
law would have held.

\subsection{From semimonads to monads: Motivation for the identity laws}

Semimonads are heuristically viewed as values with a special \textsf{``}computational
effect\textsf{''}. Semimonad-valued computations can be composed using the
\lstinline!flatMap! method, which will \textsf{``}merge\textsf{''} the effects associatively.
It is generally useful to be able to create values with an \textsf{``}empty
effect\textsf{''}, such that merging the empty effect leaves other effects
unchanged. A monad $M$ is a semimonad that has a method for creating
values with \textsf{``}empty effect\textsf{''}. That method is called \lstinline!pure!
(notation $\text{pu}_{M}$):

\begin{wrapfigure}{l}{0.475\columnwidth}%
\vspace{-0.8\baselineskip}
\begin{lstlisting}
def pure[A](a: A): M[A]
\end{lstlisting}
\vspace{-0.6\baselineskip}
\end{wrapfigure}%

~\vspace{-0.5\baselineskip}
\[
\text{pu}_{M}^{A}:A\rightarrow M^{A}\quad.
\]

To get intuition about the properties of a vaguely defined \textsf{``}empty
effect\textsf{''}, again consider nested iteration over arrays. The \textsf{``}empty
effect\textsf{''} is an array containing \emph{one} element, because an iteration
of such an array goes over a single value, which is equivalent to
no iteration. In a functor block, this intuition says that a source
line with an \textsf{``}empty effect\textsf{''}, \lstinline!y <- pure(x)!, should
be equivalent to just \lstinline!y = x!. This line must occur either
before or after another source line, for instance:

\begin{comment}
So, we need to examine two possibilities: first, an empty effect comes
before another source line,
\end{comment}

\noindent \texttt{\textcolor{blue}{\footnotesize{}}}%
\begin{minipage}[c]{0.475\columnwidth}%
\texttt{\textcolor{blue}{\footnotesize{}}}
\begin{lstlisting}
result1 = for {
    ... // Some code, then:
    y <- pure(x)   // "Empty effect" with x: A.
    z <- someArray(y)   // someArray: A => M[B]
     // Same as z <- pure(x).flatMap(someArray)
\end{lstlisting}
%
\end{minipage}\texttt{\textcolor{blue}{\footnotesize{}\hspace*{\fill}}}%
\begin{minipage}[c]{0.475\columnwidth}%
\texttt{\textcolor{blue}{\footnotesize{}}}
\begin{lstlisting}
result2 = for {
    ... // Some code, then:
    y = x               // x: A
    z <- someArray(y)   // someArray: A => M[B]
    // Same as z <- someArray(x)
\end{lstlisting}
%
\end{minipage}{\footnotesize\par}

\noindent \vspace{0.1\baselineskip}
The equality of \lstinline!result1! and \lstinline!result2! gives
the following law: for all $g^{:A\rightarrow M^{B}}$,

\begin{wrapfigure}{l}{0.3\columnwidth}%
\vspace{-0.9\baselineskip}
\begin{lstlisting}
pure(x).flatMap(g) == g(x)
\end{lstlisting}
\vspace{-0.6\baselineskip}
\end{wrapfigure}%

~\vspace{-1.4\baselineskip}
\begin{align}
{\color{greenunder}\text{left identity law of }M:}\quad & \text{pu}_{M}\bef\text{flm}_{M}(g^{:A\rightarrow M^{B}})=g\quad.\label{eq:monad-left-identity-law-for-flatMap}
\end{align}
\vspace{-1.2\baselineskip}

The second possibility is that an empty effect comes \emph{after}
a source line:

\noindent \texttt{\textcolor{blue}{\footnotesize{}}}%
\begin{minipage}[c]{0.475\columnwidth}%
\texttt{\textcolor{blue}{\footnotesize{}}}
\begin{lstlisting}
result1 = for {
    x <- someArray   // someArray: M[A]
    y <- pure(x)     // Empty effect with x: A.
// Same as y <- someArray.flatMap(x => pure(x))
\end{lstlisting}
%
\end{minipage}\texttt{\textcolor{blue}{\footnotesize{}\hspace*{\fill}}}%
\begin{minipage}[c]{0.475\columnwidth}%
\texttt{\textcolor{blue}{\footnotesize{}}}
\begin{lstlisting}
result2 = for {
    x <- someArray   // someArray: M[A]
    y = x
    // Same as y <- someArray
\end{lstlisting}
%
\end{minipage}{\footnotesize\par}

\noindent \vspace{0.1\baselineskip}
Then the equality of \lstinline!result1! and \lstinline!result2!
gives the law:

\begin{wrapfigure}{l}{0.3\columnwidth}%
\vspace{-0.85\baselineskip}
\begin{lstlisting}
g.flatMap(pure) == g
\end{lstlisting}
\vspace{-0.6\baselineskip}
\end{wrapfigure}%

~\vspace{-1.4\baselineskip}
\begin{align}
{\color{greenunder}\text{right identity law of }M:}\quad & \text{flm}_{M}(\text{pu}_{M})=\text{id}^{:M^{A}\rightarrow M^{A}}\quad.\label{eq:monad-right-identity-law-for-flatMap}
\end{align}

A typeclass\index{typeclass!Monad@\texttt{Monad}} for monads and
a law-checking test function can be defined by:
\begin{lstlisting}
abstract class Monad[F[_]: Functor : Semimonad] {
  def pure[A](a: A): F[A]
}
def checkMonadIdentityLaws[F[_], A, B]()(implicit mf: Monad[F], sf: Semimonad[F],
      aa: Arbitrary[A], af: Arbitrary[F[A]], ab: Arbitrary[A => F[B]]) = {
    forAll { (x: A, g: A => F[B]) =>               
      mf.pure(x).flatMap(g) shouldEqual g(x)   // Left identity law.
    }
    forAll { (fa: F[A]) =>               
      fa.flatMap(mf.pure[A]) shouldEqual fa   // Right identity law.
    }
}
\end{lstlisting}

Note that \lstinline!pure! is the same method as in the \lstinline!Pointed!
typeclass (Sections~\ref{subsec:Pointed-functors-motivation-equivalence}\textendash \ref{subsec:Pointed-functors:-structural-analysis}).
So, we could say that a monad is a pointed semimonad whose \lstinline!pure!
method obeys the two identity laws~(\ref{eq:monad-left-identity-law-for-flatMap})\textendash (\ref{eq:monad-right-identity-law-for-flatMap}).
Although the \lstinline!pure! method can be replaced by a simpler
\textsf{``}wrapped unit\textsf{''} value ($\text{wu}_{M}$), derivations turn out
to be easier when using \lstinline!pure!.

The \lstinline!Pointed! typeclass requires the \lstinline!pure!
method to satisfy the naturality law~(\ref{eq:naturality-law-of-pure}).
A full monad\textsf{'}s \lstinline!pure! method must also satisfy that naturality
law, in addition to the two identity laws.

Just as some useful semigroups are not monoids, there exist some useful
semimonads that are not full monads. A simple example is the \lstinline!Writer!
semimonad $F^{A}\triangleq A\times W$ whose type $W$ is a semigroup
but not a monoid (see Exercise~\ref{subsec:Exercise-semimonad-not-monad}).

\subsection{The monad identity laws in terms of \texttt{pure} and \texttt{flatten}}

Since the laws of semimonads are simpler when formulated via the \lstinline!flatten!
method, let us convert the identity laws to that form. We use the
code for \lstinline!flatMap! in terms of \lstinline!flatten!:
\[
\text{flm}_{M}(f^{:A\rightarrow M^{B}})=f^{\uparrow M}\bef\text{ftn}_{M}\quad.
\]
Begin with the left identity law of \lstinline!flatMap!, written
as:

\begin{wrapfigure}{l}{0.32\columnwidth}%
\vspace{-0\baselineskip}
\[
\xymatrix{\xyScaleY{1.0pc}\xyScaleX{3pc} & M^{M^{A}}\ar[rd]\sp(0.5){\ \text{ftn}^{A}}\\
M^{A}\ar[ru]\sp(0.5){\text{pu}^{M^{A}}}\ar[rr]\sb(0.5){\text{id}} &  & M^{A}
}
\]
\vspace{-0.2\baselineskip}
\end{wrapfigure}%

~\vspace{-0.8\baselineskip}
\[
\text{pu}_{M}\bef\text{flm}_{M}(f)=f\quad.
\]

Since this law holds for arbitrary $f$, we can set $f\triangleq\text{id}$
and get:
\begin{equation}
\text{pu}_{M}\bef\text{ftn}_{M}=\text{id}^{:M^{A}\rightarrow M^{A}}\quad.\label{eq:left-identity-law-for-flatten}
\end{equation}
\vspace{-0.5\baselineskip}

\noindent This is the \textbf{left identity law} of \lstinline!flatten!.
Conversely, if Eq.~(\ref{eq:left-identity-law-for-flatten}) holds,
we can compose both sides with an arbitrary function $f^{:A\rightarrow M^{B}}$
and recover the left identity law of \lstinline!flatMap! (Exercise~\ref{subsec:Exercise-1-monads-2}).

The \index{identity laws!of pure and flatten@of \texttt{pure} and \texttt{flatten}}\textbf{right
identity law} of \lstinline!flatten! is written as:

\begin{wrapfigure}{l}{0.32\columnwidth}%
\vspace{-1.6\baselineskip}
\[
\xymatrix{\xyScaleY{0.2pc}\xyScaleX{3pc} & M^{M^{A}}\ar[rd]\sp(0.5){\ \text{ftn}^{A}\ }\\
M^{A}\ar[ru]\sp(0.5){(\text{pu}^{A})^{\uparrow M}\quad}\ar[rr]\sb(0.5){\text{id}} &  & M^{A}
}
\]
\vspace{-1\baselineskip}
\end{wrapfigure}%

~\vspace{-0.6\baselineskip}
\begin{align}
 & \text{flm}_{M}(\text{pu}_{M})=\text{pu}_{M}^{\uparrow M}\bef\text{ftn}_{M}\overset{!}{=}\text{id}\quad.\label{eq:right-identity-law-for-flatten}
\end{align}
\vspace{-0.1\baselineskip}

In the next section, we will see a reason why these laws have their
names.

\subsection{Monad laws in terms of Kleisli functions}

A \textbf{Kleisli function}\index{Kleisli!functions|textit} is a
function with type signature $A\rightarrow M^{B}$ where $M$ is a
monad. We first encountered Kleisli functions in Section~\ref{subsec:Motivation-and-laws-for-liftopt-and-equivalence}
when deriving the laws of filterable functors using the \lstinline!liftOpt!
method. At that point, $M$ was the simple \lstinline!Option! monad.
We found that functions of type $A\rightarrow\bbnum 1+B$ can be composed
using the Kleisli composition denoted by $\diamond_{_{\text{Opt}}}$
(see page~\pageref{kleisli-composition}). Later, Section~\ref{subsec:Generalizing-the-laws-of-liftings-kleisli-functions}
stated the general properties of Kleisli composition. We will now
show that the Kleisli composition gives a useful way of formulating
the laws of a monad.

The Kleisli composition\index{Kleisli composition} operation for
a monad $M$, denoted $\diamond_{_{M}}$, is a function with type
signature:
\[
\diamond_{_{M}}:(A\rightarrow M^{B})\rightarrow(B\rightarrow M^{C})\rightarrow A\rightarrow M^{C}\quad.
\]
This resembles the forward composition of ordinary functions, $\left(\bef\right):\left(A\rightarrow B\right)\rightarrow\left(B\rightarrow C\right)\rightarrow A\rightarrow C$,
except for different types of functions. If $M$ is a monad, the implementation
of $\diamond_{_{M}}$ is:

\begin{wrapfigure}{l}{0.67\columnwidth}%
\vspace{-0.8\baselineskip}
\begin{lstlisting}
def <>[M[_]: Monad, A,B,C](f: A => M[B], g: B => M[C]): A => M[C] =
  { x => f(x).flatMap(g) }
\end{lstlisting}
\vspace{-0.9\baselineskip}
\end{wrapfigure}%

~\vspace{-0.8\baselineskip}
\begin{equation}
f\diamond_{_{M}}g\triangleq f\bef\text{flm}_{M}(g)\quad.\label{eq:def-of-kleisli-composition-for-monad-via-flatMap}
\end{equation}
\vspace{-0.5\baselineskip}

The Kleisli composition can be equivalently expressed by a functor
block code as:

\begin{lstlisting}[mathescape=true]
def <>[M[_]: Monad, A,B,C](f: A => M[B], g: B => M[C]): A => M[C] = { x =>
  for {
    y <- f(x)
    z <- g(y)
  } yield z
}
\end{lstlisting}

This example shows that Kleisli composition is a basic part of functor
block code: it expresses the chaining of two consecutive \textsf{``}source\textsf{''}
lines.

Let us now derive the laws of Kleisli composition $\diamond_{_{M}}$,
assuming that the monad laws hold for $M$. 

\subsubsection{Statement \label{subsec:Statement-identity-laws-for-kleisli}\ref{subsec:Statement-identity-laws-for-kleisli}}

For a lawful monad $M$, the Kleisli composition $\diamond_{_{M}}$
satisfies the identity laws:
\begin{align}
{\color{greenunder}\text{left identity law of }\diamond_{_{M}}:}\quad & \text{pu}_{M}\diamond_{_{M}}f=f\quad,\quad\forall f^{:A\rightarrow M^{B}}\quad,\label{eq:kleisli-left-identity-law}\\
{\color{greenunder}\text{right identity law of }\diamond_{_{M}}:}\quad & f\diamond_{_{M}}\text{pu}_{M}=f\quad,\quad\forall f^{:A\rightarrow M^{B}}\quad.\label{eq:kleisli-right-identity-law}
\end{align}


\subparagraph{Proof}

We may assume that Eqs.~(\ref{eq:monad-left-identity-law-for-flatMap})\textendash (\ref{eq:monad-right-identity-law-for-flatMap})
hold. Using the definition~(\ref{eq:def-of-kleisli-composition-for-monad-via-flatMap}),
we find:
\begin{align*}
{\color{greenunder}\text{left identity law of }\diamond_{_{M}},\text{ should equal }f:}\quad & \text{pu}_{M}\diamond_{_{M}}f=\gunderline{\text{pu}_{M}\bef\text{flm}_{M}}(f)\\
{\color{greenunder}\text{use Eq.~(\ref{eq:monad-left-identity-law-for-flatMap})}:}\quad & =f\quad,\\
{\color{greenunder}\text{right identity law of }\diamond_{_{M}},\text{ should equal }f:}\quad & f\diamond_{_{M}}\text{pu}_{M}=f\bef\gunderline{\text{flm}_{M}(\text{pu}_{M})}\\
{\color{greenunder}\text{use Eq.~(\ref{eq:monad-right-identity-law-for-flatMap})}:}\quad & \quad=f\bef\text{id}=f\quad.
\end{align*}

The following statement and the identity law~(\ref{eq:monad-right-identity-law-for-flatMap})
show that \lstinline!flatMap! can be viewed as a \textsf{``}lifting\textsf{''},
\[
\text{flm}_{M}:(A\rightarrow M^{B})\rightarrow(M^{A}\rightarrow M^{B})\quad,
\]
from Kleisli functions $A\rightarrow M^{B}$ to $M$-lifted functions
$M^{A}\rightarrow M^{B}$, except that Kleisli functions must be composed
using $\diamond_{_{M}}$, while $\text{pu}_{M}$ plays the role of
the Kleisli-identity function.

\subsubsection{Statement \label{subsec:Statement-flatMap-lifting-composition-law-for-kleisli}\ref{subsec:Statement-flatMap-lifting-composition-law-for-kleisli}}

For a lawful monad $M$, the \lstinline!flatMap! method satisfies
the composition law\index{composition law!of flatMap@of \texttt{flatMap}}:

\begin{wrapfigure}{l}{0.32\columnwidth}%
\vspace{-1.5\baselineskip}
\[
\xymatrix{\xyScaleY{0.8pc}\xyScaleX{3pc} & M^{B}\ar[rd]\sp(0.5){\ \text{flm}_{M}(g)\ }\\
M^{A}\ar[ru]\sp(0.5){\text{flm}_{M}(f)\quad}\ar[rr]\sb(0.5){\text{flm}_{M}(f\diamond_{_{_{M}}}g)} &  & M^{C}
}
\]
\vspace{0.1\baselineskip}
\end{wrapfigure}%

~\vspace{-0.2\baselineskip}
\[
\text{flm}_{M}(f\diamond_{_{_{M}}}g)=\text{flm}_{M}(f)\bef\text{flm}_{M}(g)\quad.
\]
\vspace{-0.8\baselineskip}


\subparagraph{Proof}

We may use Eq.~(\ref{eq:associativity-law-flatMap}) since $M$ is
a lawful monad. A direct calculation verifies the law:
\begin{align*}
{\color{greenunder}\text{expect to equal }\text{flm}_{M}(f)\bef\text{flm}_{M}(g):}\quad & \text{flm}_{M}(f\diamond_{_{_{M}}}g)=\text{flm}_{M}(f\bef\text{flm}_{M}(g))\\
{\color{greenunder}\text{use Eq.~(\ref{eq:associativity-law-flatMap})}:}\quad & =\text{flm}_{M}(f)\bef\text{flm}_{M}(g)\quad.
\end{align*}

The following statement motivates calling Eq.~(\ref{eq:associativity-law-flatMap})
an \textsf{``}associativity\textsf{''} law.

\subsubsection{Statement \label{subsec:Statement-associativity-law-for-kleisli}\ref{subsec:Statement-associativity-law-for-kleisli}}

For a lawful monad $M$, the Kleisli composition $\diamond_{_{M}}$
satisfies the \textbf{associativity law}\index{associativity law!of Kleisli composition}
\begin{align}
 & \left(f\diamond_{_{M}}g\right)\diamond_{_{M}}h=f\diamond_{_{M}}\left(g\diamond_{_{M}}h\right)\quad,\quad\quad\forall f^{:A\rightarrow M^{B}},\,g^{:B\rightarrow M^{C}},\,h^{:C\rightarrow M^{D}}\quad.\label{eq:kleisli-associativity-law}
\end{align}
So, we may write $f\diamond_{_{M}}g\diamond_{_{M}}h$ unambiguously
with no parentheses.

\subparagraph{Proof}

Substitute Eq.~(\ref{eq:def-of-kleisli-composition-for-monad-via-flatMap})
into both sides of the law:
\begin{align*}
{\color{greenunder}\text{left-hand side}:}\quad & (\gunderline{f\diamond_{_{M}}g})\diamond_{_{M}}h=\left(f\bef\text{flm}_{M}(g)\right)\gunderline{\diamond_{_{M}}h}=f\bef\text{flm}_{M}(g)\bef\text{flm}_{M}(h)\quad,\\
{\color{greenunder}\text{right-hand side}:}\quad & \gunderline{f\diamond_{_{M}}}(g\diamond_{_{M}}h)=f\bef\gunderline{\text{flm}_{M}(g\diamond_{_{M}}h)}\\
{\color{greenunder}\text{use Statement~\ref{subsec:Statement-flatMap-lifting-composition-law-for-kleisli}}:}\quad & \quad=f\bef\text{flm}_{M}(g)\bef\text{flm}_{M}(h)\quad.
\end{align*}
Both sides of the law are now equal.

We find that the properties of the operation $\diamond_{_{M}}$ are
similar to the identity and associativity properties of the function
composition $f\bef g$ except for using $\text{pu}_{M}$ instead of
the identity function.\footnote{It means that Kleisli functions satisfy the properties of morphisms
of a category; see Section~\ref{subsec:Motivation-for-using-category-theory}.}

Since the Kleisli composition describes the chaining of consecutive
lines in functor blocks, its associativity means that multiple lines
are chained unambiguously. For example, this code:

\begin{wrapfigure}{l}{0.14\columnwidth}%
\vspace{0.1\baselineskip}

\begin{lstlisting}[numbers=left]
x => for {
  y <- f(x)
  z <- g(y)
  t <- h(z)
} yield t
\end{lstlisting}
\vspace{0\baselineskip}
\end{wrapfigure}%

\noindent corresponds to the Kleisli composition:\vspace{-0.2\baselineskip}
\[
(x\rightarrow f(x))\diamond_{_{_{M}}}(y\rightarrow g(y))\diamond_{_{_{M}}}(z\rightarrow h(z))
\]
and does not need to specify whether lines 2 and 3 are chained before
appending line 4, or lines 3 and 4 are chained before prepending line
2.

We will now prove that the Kleisli composition with its laws is equivalent
to \lstinline!flatMap! with \emph{its} laws. In other words, we may
equally well use the Kleisli composition when formulating the requirements
for a functor $M$ to be a monad.

\subsubsection{Statement \label{subsec:Statement-equivalence-kleisli-composition-and-flatMap}\ref{subsec:Statement-equivalence-kleisli-composition-and-flatMap}}

The types of Kleisli composition $\diamond_{_{M}}$ and of $M$\textsf{'}s
\lstinline!flatMap! are equivalent:
\begin{equation}
f^{:A\rightarrow M^{B}}\diamond_{_{M}}g^{:B\rightarrow M^{C}}=f\bef\text{flm}_{M}(g)\quad,\quad\quad\text{flm}_{M}(f^{:A\rightarrow M^{B}})=\text{id}^{:M^{A}\rightarrow M^{A}}\diamond_{_{M}}f\quad,\label{eq:express-kleisli-composition-via-flatMap-and-back}
\end{equation}
provided that Eqs.~(\ref{eq:kleisli-left-identity-law})\textendash (\ref{eq:kleisli-associativity-law})
and the following additional law hold for $\diamond_{_{M}}$:
\begin{align}
{\color{greenunder}\text{left naturality of }\diamond_{_{M}}:}\quad & (f^{:A\rightarrow B}\bef g^{:B\rightarrow M^{C}}\big)\diamond_{_{M}}h^{:C\rightarrow M^{D}}=f\bef\big(g\diamond_{_{M}}h\big)\quad.\label{eq:left-naturality-of-kleisli-composition}
\end{align}
Note that this law makes parentheses unnecessary in the expression
$f\bef g\diamond_{_{M}}h$.\index{Kleisli composition!with function composition}

\subparagraph{Proof}

Equations~(\ref{eq:express-kleisli-composition-via-flatMap-and-back})
map $\diamond_{_{M}}$ to $\text{flm}_{M}$ and back. We have to show
that these mappings are isomorphisms when the given laws hold. We
proceed in two steps:

(1) Given an operation $\diamond_{_{M}}$, we define $\text{flm}_{M}$
and then a new operation $\diamond_{_{M}}^{\prime}$ using Eq.~(\ref{eq:express-kleisli-composition-via-flatMap-and-back}).
We then need to prove that $\diamond_{_{M}}^{\prime}=\diamond_{_{M}}$.
Calculate using arbitrary functions $f^{:A\rightarrow M^{B}}$ and
$g^{:B\rightarrow M^{C}}$:
\begin{align*}
{\color{greenunder}\text{use Eq.~(\ref{eq:express-kleisli-composition-via-flatMap-and-back})}:}\quad & f\diamond_{_{M}}^{\prime}g=f\bef\text{flm}_{M}(g)=f\bef\big(\text{id}^{M^{B}}\diamond_{_{M}}g\big)\\
{\color{greenunder}\text{left naturality~(\ref{eq:left-naturality-of-kleisli-composition}) of }\diamond_{_{M}}:}\quad & =(\gunderline{f\bef\text{id}})\diamond_{_{M}}g=f\diamond_{_{M}}g\quad.
\end{align*}

When $\diamond_{_{M}}$ is defined via $\text{flm}_{M}$, the left
naturality law~(\ref{eq:left-naturality-of-kleisli-composition})
will hold because \textsf{``}$\bef$\textsf{''} is associative,
\begin{align*}
 & (f\bef g)\diamond_{_{M}}h=(f\bef g)\bef\text{flm}_{M}(h)=f\bef(g\bef\text{flm}_{M}(h))=f\bef(g\diamond_{_{M}}h)\quad.
\end{align*}

(2) Given a function $\text{flm}_{M}$, we define $\diamond_{_{M}}$
and then a new function $\text{flm}_{M}^{\prime}$ using Eq.~(\ref{eq:express-kleisli-composition-via-flatMap-and-back}).
We then need to prove that $\text{flm}_{M}^{\prime}=\text{flm}_{M}$.
Calculate using an arbitrary function $f^{:A\rightarrow M^{B}}$:
\begin{align*}
{\color{greenunder}\text{use Eq.~(\ref{eq:express-kleisli-composition-via-flatMap-and-back})}:}\quad & \text{flm}_{M}^{\prime}(f)=\text{id}^{M^{A}}\diamond_{_{M}}f=\gunderline{\text{id}}\bef\text{flm}_{M}(f)=\text{flm}_{M}(f)\quad.
\end{align*}

We have already derived the laws of Kleisli composition from the laws
of \lstinline!flatMap!. We will now derive the converse statement.
In this way, we will show that the Kleisli composition laws and the
\lstinline!flatMap! laws are fully equivalent.

\subsubsection{Statement \label{subsec:Statement-equivalence-kleisli-laws-and-flatMap-laws}\ref{subsec:Statement-equivalence-kleisli-laws-and-flatMap-laws}}

If the Kleisli composition $\diamond_{_{M}}$ obeys the laws~(\ref{eq:kleisli-left-identity-law})\textendash (\ref{eq:kleisli-associativity-law}),
the corresponding \lstinline!flatMap! method defined by Eq.~(\ref{eq:express-kleisli-composition-via-flatMap-and-back})
will satisfy the laws~(\ref{eq:left-naturality-law-flatMap})\textendash (\ref{eq:associativity-law-flatMap}).

\subparagraph{Proof}

To derive the identity laws of \lstinline!flatMap!:
\begin{align*}
{\color{greenunder}\text{left identity law}:}\quad & \text{pu}_{M}\bef\gunderline{\text{flm}_{M}(f)}=\gunderline{\text{pu}_{M}\bef\text{id}}\diamond_{_{M}}f=\gunderline{\text{pu}_{M}\diamond_{_{M}}}f=f\quad,\\
{\color{greenunder}\text{right identity law}:}\quad & \text{flm}_{M}(\text{pu}_{M})=\text{id}\,\gunderline{\diamond_{_{M}}\text{pu}_{M}}=\text{id}\quad.
\end{align*}

To derive the associativity law~(\ref{eq:associativity-law-flatMap})
of \lstinline!flatMap!, substitute the definition of $\text{flm}_{M}$
into both sides:
\begin{align*}
{\color{greenunder}\text{left-hand side}:}\quad & \text{flm}_{M}(f\bef\text{flm}_{M}(g))=\text{id}\diamond_{_{M}}(\gunderline{f\bef\text{id}}\diamond_{_{M}}g)=\text{id}\diamond_{_{M}}(f\diamond_{_{M}}g)\\
{\color{greenunder}\text{associativity law~(\ref{eq:kleisli-associativity-law})}:}\quad & \quad=(\text{id}\diamond_{_{M}}f)\diamond_{_{M}}g\quad,\\
{\color{greenunder}\text{right-hand side}:}\quad & \text{flm}_{M}(f)\bef\text{flm}_{M}(g)=(\text{id}\diamond_{_{M}}f)\,\gunderline{\bef(\text{id}}\diamond_{_{M}}g)\\
{\color{greenunder}\text{left naturality~(\ref{eq:left-naturality-of-kleisli-composition}) of }\diamond_{_{M}}:}\quad & \quad=(\text{id}\diamond_{_{M}}f)\,\gunderline{\bef\text{id}}\diamond_{_{M}}g=(\text{id}\diamond_{_{M}}f)\diamond_{_{M}}g\quad.
\end{align*}
Both sides of the law are now equal.

The two naturality laws of \lstinline!flatMap! are equivalent to
the three naturality laws of $\diamond_{_{M}}$, but we omit those
derivations. 

\subsection{Verifying the monad laws using Kleisli functions}

When the monad laws are formulated via Kleisli composition, the intuition
behind the laws becomes clearer: they are analogous to the identity
and associativity laws of the function composition ($\bef$). The
price is that the type signatures become complicated. For instance,
the associativity law~(\ref{eq:kleisli-associativity-law}) has four
type parameters, while the corresponding law~(\ref{eq:associativity-law-of-flatten})
for \lstinline!flatten! has only one. For certain monads, however,
a trick called \index{flipped@\textsf{``}flipped Kleisli\textsf{''} technique|textit}\textbf{flipped
Kleisli} makes direct proofs of laws much shorter. That trick applies
to monads of a function type, such as the continuation and the state
monads.

\subsubsection{Statement \label{subsec:Statement-continuation-monad-is-lawful}\ref{subsec:Statement-continuation-monad-is-lawful}}

The continuation monad, $\text{Cont}^{R,A}\triangleq\left(A\rightarrow R\right)\rightarrow R$,
satisfies all monad laws.

\subparagraph{Proof}

Begin by writing the type of a Kleisli function corresponding to this
monad:
\[
A\rightarrow\text{Cont}^{R,B}=A\rightarrow\left(B\rightarrow R\right)\rightarrow R\quad.
\]
This function type has two curried arguments. The first step of the
flipped Kleisli technique is to change the types of the Kleisli functions
by flipping their two curried arguments. We obtain:
\[
\left(B\rightarrow R\right)\rightarrow A\rightarrow R\quad.
\]
This type looks like a function of the form $K^{B}\rightarrow K^{A}$,
where we temporarily defined $K^{A}\triangleq A\rightarrow R$. The
remaining steps are to flip the arguments of $\text{pu}_{\text{Cont}}$
and get a modified $\tilde{\text{pu}}_{\text{Cont}}$. Then, we modify
the Kleisli composition $\diamond_{_{\text{Cont}}}$ into $\tilde{\diamond}_{_{\text{Cont}}}$.
The flipped Kleisli functions are composed using $\tilde{\diamond}_{_{\text{Cont}}}$.
We will then prove the laws:
\[
\tilde{\text{pu}}_{\text{Cont}}\tilde{\diamond}_{_{\text{Cont}}}f=f\quad,\quad\quad f\,\tilde{\diamond}_{_{\text{Cont}}}\tilde{\text{pu}}_{\text{Cont}}=f\quad,\quad\quad(f\,\tilde{\diamond}_{_{\text{Cont}}}g)\,\tilde{\diamond}_{_{\text{Cont}}}h=f\,\tilde{\diamond}_{_{\text{Cont}}}(g\,\tilde{\diamond}_{_{\text{Cont}}}h)\quad.
\]
Flipping the arguments is an operation that maps functions to equivalent
functions. So, if we prove the laws of identity and composition for
flipped Kleisli functions, it will follow that the same laws hold
for the original functions.

The original \lstinline!pure! method is:
\[
\text{pu}_{\text{Cont}}\triangleq a^{:A}\rightarrow f^{:A\rightarrow R}\rightarrow f(a)\quad.
\]
We find that the flipped \lstinline!pure! method is just an identity
function:
\[
\tilde{\text{pu}}_{\text{Cont}}\triangleq f^{:A\rightarrow R}\rightarrow a^{:A}\rightarrow f(a)=f^{:A\rightarrow R}\rightarrow f=\text{id}^{:\left(A\rightarrow R\right)\rightarrow A\rightarrow R}=\text{id}^{:K^{A}\rightarrow K^{A}}\quad.
\]
This is a significant simplification. The flipped Kleisli composition
$\tilde{\diamond}_{_{\text{Cont}}}$ has the type signature:
\[
f^{:K^{B}\rightarrow K^{A}}\tilde{\diamond}_{_{\text{Cont}}}g^{:K^{C}\rightarrow K^{B}}=\text{???}^{:K^{C}\rightarrow K^{A}}\quad.
\]
There is only one implementation of that type signature, namely the
backward composition:
\[
f^{:K^{B}\rightarrow K^{A}}\tilde{\diamond}_{_{\text{Cont}}}g^{:K^{C}\rightarrow K^{B}}\triangleq g\bef f=f\circ g\quad.
\]
So, this must be the code of the flipped Kleisli composition. It is
now quick to verify the laws:
\begin{align*}
{\color{greenunder}\text{left identity law}:}\quad & \tilde{\text{pu}}_{\text{Cont}}\tilde{\diamond}_{_{\text{Cont}}}f=\text{id}\circ f=f\quad,\\
{\color{greenunder}\text{right identity law}:}\quad & f\,\tilde{\diamond}_{_{\text{Cont}}}\tilde{\text{pu}}_{\text{Cont}}=f\circ\text{id}=f\quad,\\
{\color{greenunder}\text{associativity law}:}\quad & \big(f\,\tilde{\diamond}_{_{\text{Cont}}}g\big)\,\tilde{\diamond}_{_{\text{Cont}}}h=(f\circ g)\circ h=f\circ(g\circ h)=f\,\tilde{\diamond}_{_{\text{Cont}}}\big(g\,\tilde{\diamond}_{_{\text{Cont}}}h\big)\quad.
\end{align*}
We could have avoided writing the last three lines by noticing that
functions of types $K^{B}\rightarrow K^{A}$ automatically satisfy
the laws of (backward) function composition (Section~\ref{subsec:Laws-of-function-composition}).

\subsubsection{Statement \label{subsec:Statement-state-monad-is-lawful}\ref{subsec:Statement-state-monad-is-lawful}}

The state monad, $\text{State}^{S,A}\triangleq S\rightarrow A\times S$,
satisfies all monad laws.

\subparagraph{Proof}

Begin by writing the type of a Kleisli function corresponding to this
monad:
\[
A\rightarrow\text{State}^{S,B}=A\rightarrow S\rightarrow B\times S\quad.
\]
An equivalent type is obtained by uncurrying the two arguments:
\[
\left(A\rightarrow S\rightarrow B\times S\right)\cong\left(A\times S\rightarrow B\times S\right)=K^{A}\rightarrow K^{B}\quad,
\]
where we temporarily defined $K^{A}\triangleq A\times S$. This type
looks like an ordinary function, which promises to simplify the proof.
Uncurrying is an equivalence transformation. So, let us uncurry the
arguments in all Kleisli functions and prove the laws in the \textsf{``}uncurried
Kleisli\textsf{''} formulation. We need to uncurry the arguments in $\text{pu}_{\text{State}}$
and $\diamond_{_{\text{State}}}$ as well. Denote the resulting functions
$\tilde{\text{pu}}_{\text{State}}$ and $\tilde{\diamond}_{_{\text{State}}}$:
\[
\tilde{\text{pu}}_{\text{State}}\triangleq a^{:A}\times s^{:S}\rightarrow a\times s=\text{id}^{:K^{A}\rightarrow K^{A}}\quad,\quad\quad f^{:K^{A}\rightarrow K^{B}}\tilde{\diamond}_{_{\text{State}}}g^{:K^{B}\rightarrow K^{C}}\triangleq f\bef g\quad.
\]
We can see that the composition $f\bef g$ implements the correct
logic of the state monad: an initial state value $s^{:S}$ is updated
by $f$ and then passed to $g$.

We found that the uncurried Kleisli functions have simple types $K^{A}\rightarrow K^{B}$,
and the operation $\tilde{\diamond}_{_{\text{State}}}$ is just the
ordinary composition of those functions. Since we already know that
the laws of identity and associativity hold for ordinary functions
(Section~\ref{subsec:Laws-of-function-composition}), the proof is
finished.

For comparison, look at the type signatures of \lstinline!flatten!
for the state and continuation monads:
\begin{align*}
\text{ftn}_{\text{State}^{S,\bullet}} & :\left(S\rightarrow\left(S\rightarrow A\times S\right)\times S\right)\rightarrow S\rightarrow A\times S\quad,\\
\text{ftn}_{\text{Cont}^{R,\bullet}} & :\left(\left(\left(\left(A\rightarrow R\right)\rightarrow R\right)\rightarrow R\right)\rightarrow R\right)\rightarrow\left(A\rightarrow R\right)\rightarrow R\quad.
\end{align*}
These type signatures are complicated and confusing to read. Direct
proofs of the monad laws for these functions are much longer than
the proofs of Statements~\ref{subsec:Statement-state-monad-is-lawful}\textendash \ref{subsec:Statement-continuation-monad-is-lawful}.
When a monad $M$ has a function type, the Kleisli function $A\rightarrow M^{B}$
has two curried arguments. Flipping or uncurrying these arguments
often produces an equivalent function that is easier to work with. 

\subsection{Constructions of semimonads and monads\label{subsec:Structural-analysis-of-monads}}

We have seen different examples of well-known monads that were discovered
by programmers working on specific tasks. Hoping to find systematically
all possible monads, we will now apply structural analysis to semimonads
and monads. For each type construction, we will prove rigorously that
the monad laws hold.

\paragraph{Type parameters}

Three type constructions are based on using just type parameters:
the constant functor ($\text{Const}^{Z,A}\triangleq Z$), the identity
functor ($\text{Id}^{A}\triangleq A$), and the functor composition
($L^{A}\triangleq F^{G^{A}}$).

A constant functor $F^{A}\triangleq Z$ is a lawful semimonad because
we can implement:
\[
\text{ftn}_{F}=\text{id}^{:Z\rightarrow Z}\quad.
\]
An identity function will always satisfy the laws. To obtain a full
monad, we need to implement:
\[
\text{pu}_{F}:A\rightarrow Z\quad.
\]
This is possible only if we have a default value $z_{0}$ of type
$Z$. Assuming that, we set $\text{pu}_{F}\triangleq\_\rightarrow z_{0}$
and check the identity laws:
\begin{align*}
{\color{greenunder}\text{left identity law of }F:}\quad & \text{pu}_{F}^{:F^{A}\rightarrow F^{F^{A}}}\bef\text{ftn}_{F}=\text{pu}_{F}\bef\text{id}=\text{pu}_{F}\overset{?}{=}\text{id}\quad.
\end{align*}
The function $\text{pu}_{F}$ is a constant function that always returns
$z_{0}$; and yet it must be equal to the identity function. This
law can be satisfied only if the identity function of type $Z\rightarrow Z$
always returns the same value $z_{0}$. It follows that $z_{0}$ is
the only available value of the type $Z$, which means $Z\cong\bbnum 1$.
In that case, all functions become constants returning $1$, and the
laws are trivially satisfied. We conclude that the only case when
a constant functor is a lawful monad is when $L^{A}\triangleq\bbnum 1$
(the constant \lstinline!Unit! type).

The identity functor $\text{Id}^{A}\triangleq A$ is a monad: its
\lstinline!pure! and \lstinline!flatten! methods are identity functions.

Functor composition $F\circ G$ is not guaranteed to produce monads
even if $F$ and $G$ are both monads. A counterexample can be found
by taking $F^{A}\triangleq Z+A$ and $G^{A}\triangleq R\rightarrow A$;
the composition $L^{A}\triangleq Z+\left(R\rightarrow A\right)$ is
not even a semimonad.

\subsubsection{Statement \label{subsec:Statement-not-semimonad-1+r-a}\ref{subsec:Statement-not-semimonad-1+r-a}}

The functor $L^{A}\triangleq Z+\left(R\rightarrow A\right)$, where
$R$ and $Z$ are fixed but arbitrary types, cannot have a \lstinline!flatten!
method.

\subparagraph{Proof}

The type signature of \lstinline!flatten! is:
\[
\text{ftn}_{L}:Z+\left(R\rightarrow Z+\left(R\rightarrow A\right)\right)\rightarrow Z+\left(R\rightarrow A\right)\quad.
\]
A fully parametric implementation is impossible: we would need to
compute either a value of type $Z$ or of type $R\rightarrow A$ from
an argument that may have either type $Z$ or type $R\rightarrow Z+\left(R\rightarrow A\right)$.
When the argument has the latter type, it is impossible to compute
either a value of type $Z$ or a value of type $R\rightarrow A$,
because different values may be returned for different values of $R$,
and we do not have any known values of type $R$ available. 

\paragraph{Products}

The product construction works for semimonads as well as for monads.

\subsubsection{Statement \label{subsec:Statement-monad-semimonad-product}\ref{subsec:Statement-monad-semimonad-product}}

Given two semimonads $F^{A}$ and $G^{A}$, the functor $L^{A}\triangleq F^{A}\times G^{A}$
is a semimonad. If both $F^{A}$ and $G^{A}$ are monads then $L^{A}$
is also a monad.

\subparagraph{Proof}

Begin by defining the \lstinline!flatten! method for the semimonad
$L$ via the \lstinline!flatten! methods of $F$ and $G$. The \lstinline!flatten!
method needs to transform a value of type $L^{L^{A}}=F^{F^{A}\times G^{A}}\times G^{F^{A}\times G^{A}}$
into a value of type $F^{A}\times G^{A}$. Since $F$ and $G$ are
functors, we can extract $F^{F^{A}}$ out of $F^{F^{A}\times G^{A}}$
by lifting the standard projection function $\pi_{1}$:
\[
\pi_{1}^{\uparrow F}:F^{F^{A}\times G^{A}}\rightarrow F^{F^{A}}\quad.
\]
Then we use $F$\textsf{'}s \lstinline!flatten! method to obtain a value of
type $F^{A}$. In a similar way, we transform a value of type $G^{F^{A}\times G^{A}}$
into $G^{A}$. The resulting code is:
\begin{lstlisting}
def flatten_L[A]: (F[(F[A], G[A])], G[(F[A], G[A])]) => (F[A], G[A]) = { case (fla, gla) =>
  val ffa: F[F[A]] = fla.map(_._1)
  val gga: G[G[A]] = gla.map(_._2)
  (flatten_F(ffa), flatten_G(gga))
}
\end{lstlisting}
\begin{equation}
\text{ftn}_{L}\triangleq f^{:F^{F^{A}\times G^{A}}}\times g^{:G^{F^{A}\times G^{A}}}\rightarrow\big(f\triangleright\pi_{1}^{\uparrow F}\triangleright\text{ftn}_{F}\big)\times\big(g\triangleright\pi_{2}^{\uparrow G}\triangleright\text{ftn}_{G}\big)=(\pi_{1}^{\uparrow F}\bef\text{ftn}_{F})\boxtimes(\pi_{2}^{\uparrow G}\bef\text{ftn}_{G})\quad.\label{eq:monad-product-flatten-def-ftn}
\end{equation}
To verify the associativity law, we need to write the code of $\text{ftn}_{L}^{\uparrow L}$:
\[
(h^{:A\rightarrow B})^{\uparrow L}=h^{\uparrow F}\boxtimes h^{\uparrow G}\quad,\quad\quad\text{ftn}_{L}^{\uparrow L}=\text{ftn}_{L}^{\uparrow F}\boxtimes\text{ftn}_{L}^{\uparrow G}\quad.
\]
Substitute these definitions into the associativity law:
\begin{align*}
{\color{greenunder}\text{left-hand side}:}\quad & \text{ftn}_{L}^{\uparrow L}\bef\text{ftn}_{L}=\big(\text{ftn}_{L}^{\uparrow F}\boxtimes\text{ftn}_{L}^{\uparrow G}\big)\bef\big((\pi_{1}^{\uparrow F}\bef\text{ftn}_{F})\boxtimes(\pi_{2}^{\uparrow G}\bef\text{ftn}_{G})\big)\\
{\color{greenunder}\text{use Eq.~(\ref{eq:pair-product-composition-law})}:}\quad & \quad=\big(\text{ftn}_{L}^{\uparrow F}\bef\pi_{1}^{\uparrow F}\bef\text{ftn}_{F}\big)\boxtimes\big(\text{ftn}_{L}^{\uparrow G}\bef\pi_{2}^{\uparrow G}\bef\text{ftn}_{G}\big)\quad,\\
{\color{greenunder}\text{right-hand side}:}\quad & \text{ftn}_{L}\bef\text{ftn}_{L}=\big((\pi_{1}^{\uparrow F}\bef\text{ftn}_{F})\boxtimes(\pi_{2}^{\uparrow G}\bef\text{ftn}_{G})\big)\bef\big((\pi_{1}^{\uparrow F}\bef\text{ftn}_{F})\boxtimes(\pi_{2}^{\uparrow G}\bef\text{ftn}_{G})\big)\\
{\color{greenunder}\text{use Eq.~(\ref{eq:pair-product-composition-law})}:}\quad & \quad=\big(\pi_{1}^{\uparrow F}\bef\text{ftn}_{F}\bef\pi_{1}^{\uparrow F}\bef\text{ftn}_{F}\big)\boxtimes\big(\pi_{2}^{\uparrow G}\bef\text{ftn}_{G}\bef\pi_{2}^{\uparrow G}\bef\text{ftn}_{G}\big)\quad.
\end{align*}
The only given information about $\text{ftn}_{F}$ and $\text{ftn}_{G}$
is that they obey their associativity laws; e.g.,
\[
\text{ftn}_{F}^{\uparrow F}\bef\text{ftn}_{F}=\text{ftn}_{F}\bef\text{ftn}_{F}\quad.
\]
In order to use this law, we need to move the two functions $\text{ftn}_{F}$
next to each other in the expressions 
\[
\big(\text{ftn}_{L}^{\uparrow F}\bef\pi_{1}^{\uparrow F}\bef\text{ftn}_{F}\big)\quad\text{ and }\quad\big(\pi_{1}^{\uparrow F}\bef\text{ftn}_{F}\bef\pi_{1}^{\uparrow F}\bef\text{ftn}_{F}\big)\quad,
\]
hoping to show that these expressions are equal. We begin with the
first of those expressions:
\begin{align*}
 & \gunderline{\text{ftn}_{L}^{\uparrow F}\bef\pi_{1}^{\uparrow F}}\bef\text{ftn}_{F}=\big(\gunderline{\text{ftn}_{L}\bef\pi_{1}}\big)^{\uparrow F}\bef\text{ftn}_{F}\\
{\color{greenunder}\text{left projection law~(\ref{eq:pair-product-projection-laws})}:}\quad & =\big(\pi_{1}\bef\pi_{1}^{\uparrow F}\bef\text{ftn}_{F}\big)^{\uparrow F}\bef\text{ftn}_{F}=\pi_{1}^{\uparrow F}\bef\pi_{1}^{\uparrow F\uparrow F}\bef\gunderline{\text{ftn}_{F}^{\uparrow F}\bef\text{ftn}_{F}}\\
{\color{greenunder}\text{associativity law of }F:}\quad & =\pi_{1}^{\uparrow F}\bef\gunderline{\pi_{1}^{\uparrow F\uparrow F}\bef\text{ftn}_{F}}\bef\text{ftn}_{F}\\
{\color{greenunder}\text{naturality law of }\text{ftn}_{F}:}\quad & =\pi_{1}^{\uparrow F}\bef\gunderline{\text{ftn}_{F}\bef\pi_{1}^{\uparrow F}}\bef\text{ftn}_{F}\quad.
\end{align*}
Both expressions are now the same. An analogous derivation shows that:
\[
\text{ftn}_{L}^{\uparrow G}\bef\pi_{2}^{\uparrow G}\bef\text{ftn}_{G}=\pi_{2}^{\uparrow G}\bef\text{ftn}_{G}\bef\pi_{2}^{\uparrow G}\bef\text{ftn}_{G}\quad.
\]
So, both sides of the associativity law are equal.

Now we assume that $F$ and $G$ are monads with given \lstinline!pure!
methods $\text{pu}_{F}$ and $\text{pu}_{G}$. We define:
\[
\text{pu}_{L}\triangleq a^{:A}\rightarrow\text{pu}_{F}(a)\times\text{pu}_{G}(a)=\Delta\bef(\text{pu}_{F}\boxtimes\text{pu}_{G})\quad.
\]
Assuming that identity laws hold for $F$ and $G$, we can now verify
the identity laws for $L$:
\begin{align*}
{\color{greenunder}\text{left identity law of }L:}\quad & \text{pu}_{L}\bef\text{ftn}_{L}=\Delta\bef(\text{pu}_{F}\boxtimes\text{pu}_{G})\bef\big((\pi_{1}^{\uparrow F}\bef\text{ftn}_{F})\boxtimes(\pi_{2}^{\uparrow G}\bef\text{ftn}_{G})\big)\\
{\color{greenunder}\text{use Eq.~(\ref{eq:pair-product-composition-law})}:}\quad & \quad=\Delta\bef\big((\gunderline{\text{pu}_{F}\bef\pi_{1}^{\uparrow F}}\bef\text{ftn}_{F})\boxtimes(\gunderline{\text{pu}_{G}\bef\pi_{2}^{\uparrow G}}\bef\text{ftn}_{G})\big)\\
{\color{greenunder}\text{naturality of }F,G:}\quad & \quad=\Delta\bef\big((\pi_{1}\bef\gunderline{\text{pu}_{F}\bef\text{ftn}_{F}})\boxtimes(\pi_{2}\bef\gunderline{\text{pu}_{G}\bef\text{ftn}_{G}})\big)\\
{\color{greenunder}\text{identity laws of }F,G:}\quad & \quad=\Delta\bef(\pi_{1}\times\pi_{2})=\text{id}\quad,\\
{\color{greenunder}\text{right identity law of }L:}\quad & \text{pu}_{L}^{\uparrow L}\bef\text{ftn}_{L}=\big(\big(\Delta\bef(\text{pu}_{F}\boxtimes\text{pu}_{G})\big)^{\uparrow F}\boxtimes\big(\Delta\bef(\text{pu}_{F}\boxtimes\text{pu}_{G})\big)^{\uparrow G}\big)\bef\text{ftn}_{L}\\
{\color{greenunder}\text{use Eq.~(\ref{eq:pair-product-composition-law})}:}\quad & \quad=\big(\big(\Delta\bef\gunderline{(\text{pu}_{F}\boxtimes\text{pu}_{G})\big)^{\uparrow F}\bef\pi_{1}^{\uparrow F}}\bef\text{ftn}_{F}\big)\boxtimes\big(\big(\Delta\bef\gunderline{(\text{pu}_{F}\boxtimes\text{pu}_{G})\big)^{\uparrow G}\bef\pi_{2}^{\uparrow G}}\bef\text{ftn}_{G}\big)\\
{\color{greenunder}\text{projection laws~(\ref{eq:pair-product-projection-laws})}:}\quad & \quad=\big((\gunderline{\Delta\bef\pi_{1}}\bef\text{pu}_{F})^{\uparrow F}\bef\text{ftn}_{F}\big)\boxtimes\big(\big(\gunderline{\Delta\bef\pi_{2}}\bef\text{pu}_{G}\big)^{\uparrow G}\bef\text{ftn}_{G}\big)\\
{\color{greenunder}\text{identity laws~(\ref{eq:pair-identity-law-left})}:}\quad & \quad=\big(\gunderline{\text{pu}_{F}^{\uparrow F}\bef\text{ftn}_{F}}\big)\boxtimes\big(\gunderline{\text{pu}_{G}^{\uparrow G}\bef\text{ftn}_{G}}\big)=\text{id}\boxtimes\text{id}=\text{id}\quad.
\end{align*}

Let us build some intuition about how the product of two monads works
in practice. A simple example is the product of two identity monads,
$L^{A}\triangleq A\times A$. This type constructor is a monad whose
\lstinline!flatten! function is defined by
\begin{lstlisting}
type Pair[A] = (A, A)
def flatten[A]: Pair[Pair[A]] => Pair[A] = { case ((a, b), (c, d)) => (a, d) }
\end{lstlisting}
A sample calculation shows that \textsf{``}nested iterations\textsf{''} apply functions
element by element:
\begin{lstlisting}
final case class P[A](x: A, y: A) {
  def map[B](f: A => B): P[B] = P(f(x), f(y))
  def flatMap[B](f: A => P[B]): P[B] = P(f(x).x, f(y).y)
}

scala> for {
  x <- P(1, 10)
  y <- P(2, 20)
  z <- P(3, 30)
} yield x + y + z       // The result is P(1 + 2 + 3, 10 + 20 + 30).
res0: P[Int] = P(6, 60)
\end{lstlisting}

Note that the pair type $A\times A$ is equivalent to the function
type $\bbnum 2\rightarrow A$ (in Scala, \lstinline!Boolean => A!).
We know that $\bbnum 2\rightarrow A$ is a \lstinline!Reader! monad.
One can check that the implementation of \lstinline!flatten! for
the \lstinline!Reader! monad $\bbnum 2\rightarrow A$ is equivalent
to the code of \lstinline!flatten! for $L^{A}\triangleq A\times A$
as shown above.

Now consider the product $F^{A}\times G^{A}$ with arbitrary monads
$F$ and $G$ that represent some effects. Then a Kleisli function
of type $A\rightarrow F^{B}\times G^{B}$ contains both effects. How
does the monad $L$ combine two effects when we compose two such Kleisli
functions? We see from the code of $L$\textsf{'}s \lstinline!flatten! that
the first effect in $F$ is combined with the second effect in $F$,
and the first effect in $G$ is combined with the second effect in
$G$. The part $F^{G^{A}}$ from $F^{F^{A}\times G^{A}}$ is discarded
by \lstinline!flatten!, leaving only $F^{F^{A}}$, i.e., a combination
of two $F$-effects. Also, $G^{F^{A}}$ is discarded from $G^{F^{A}\times G^{A}}$,
leaving only $G^{G^{A}}$. As an example, consider this code:
\begin{lstlisting}
val result: (F[C], G[C]) = for {
  x <- (fa, ga)          // Assume fa: F[A], ga: G[A]
  y <- (fb, gb)          // Assume fb: F[B], gb: G[B]
} yield h(x, y)          // Assume h: (A, B) => C
\end{lstlisting}
The expression \lstinline!result! is equivalent to the following
code that works with $F$ and $G$ separately:\vspace{-0.95\baselineskip}

\begin{center}
\begin{minipage}[t]{0.3\columnwidth}%
\begin{lstlisting}
val result1: F[C] = for {
  x <- fa
  y <- fb
} yield h(x, y)
\end{lstlisting}
%
\end{minipage}\hfill{} %
\begin{minipage}[t]{0.3\columnwidth}%
\begin{lstlisting}
val result2: G[C] = for {
  x <- ga
  y <- gb
} yield h(x, y)
\end{lstlisting}
%
\end{minipage}\hfill{} %
\begin{minipage}[t]{0.3\columnwidth}%
\begin{lstlisting}
val result: (F[C], G[C]) =
  (result1, result2)
\end{lstlisting}
%
\end{minipage}\vspace{-0.35\baselineskip}
\par\end{center}

A composition of two $L$-effects is a pair consisting of $F$-effects
and $G$-effects composed separately. Because of that, the composition
of $L$-effects is associative (and so $L$ is a lawful semimonad)
as long as $F$- and $G$-effects are themselves composed associatively.

\subsubsection{Statement \label{subsec:Statement-semimonad-only-product-a-ga}\ref{subsec:Statement-semimonad-only-product-a-ga}}

Given any functor $F$, the functor $L^{A}\triangleq A\times F^{A}$
is a semimonad.

\subparagraph{Proof}

We begin by implementing the \lstinline!flatten! method for $L$,
with the type signature $A\times F^{A}\times F^{A\times F^{A}}\rightarrow A\times F^{A}$.
Since we know nothing about the functor $F$, we cannot extract values
of type $A$ from $F^{A}$. We also do not have a \lstinline!flatten!
method for $F$. How can we get values of type $A$ and $F^{A}$ out
of $A\times F^{A}\times F^{A\times F^{A}}$? One possibility is simply
to discard the part of type $F^{A\times F^{A}}$:
\begin{lstlisting}
def flatten1_L[A]: ((A, F[A]), F[(A, F[A])]) => (A, F[A]) = _._1
\end{lstlisting}
The other possibility is to transform $F^{A\times F^{A}}$ to $F^{A}$
within the functor $F$:
\begin{lstlisting}
def flatten2_L[A]: ((A, F[A]), F[(A, F[A])]) => (A, F[A]) = { case (afa, fafa) =>
  (afa._1, fafa.map(_._1))
}
\end{lstlisting}
In the code notation, these alternative implementations may be written
as:
\[
\text{ftn}_{1\,L}\triangleq\pi_{1}=p^{:A\times F^{A}}\times q^{:F^{A\times F^{A}}}\rightarrow p\quad,\quad\text{ftn}_{2\,L}\triangleq p^{:A\times F^{A}}\times q^{:F^{A\times F^{A}}}\rightarrow\big(p\triangleright\pi_{1}\big)\times\big(q\triangleright\pi_{1}^{\uparrow F}\big)=\pi_{1}\boxtimes\pi_{1}^{\uparrow F}\quad.
\]

To check the associativity laws, we need to prepare the code for lifting
to the functor $L$:
\[
f^{\uparrow L}=f\boxtimes f^{\uparrow F}\quad,\quad\quad\text{ftn}_{1\,L}^{\uparrow L}=\pi_{1}\boxtimes\pi_{1}^{\uparrow F}\quad,\quad\quad\text{ftn}_{2\,L}^{\uparrow L}=(\pi_{1}\boxtimes\pi_{1}^{\uparrow F})\boxtimes(\pi_{1}\boxtimes\pi_{1}^{\uparrow F})^{\uparrow F}\quad.
\]
Then we verify the associativity law for $\text{ftn}_{1\,L}$ using
the projection law~(\ref{eq:pair-product-projection-laws}): 
\begin{align*}
{\color{greenunder}\text{left-hand side}:}\quad & \text{ftn}_{1\,L}^{\uparrow L}\bef\text{ftn}_{1\,L}=(\pi_{1}\boxtimes\pi_{1}^{\uparrow F})\bef\pi_{1}=\pi_{1}\bef\pi_{1}\quad,\\
{\color{greenunder}\text{right-hand side}:}\quad & \text{ftn}_{1\,L}\bef\text{ftn}_{1\,L}=\pi_{1}\bef\pi_{1}\quad.
\end{align*}
The law holds. For $\text{ftn}_{2\,L}$, we also find that the two
sides of the law are equal:
\begin{align*}
 & \text{ftn}_{2\,L}^{\uparrow L}\bef\text{ftn}_{2\,L}=\big((\pi_{1}\boxtimes\pi_{1}^{\uparrow F})\boxtimes(\pi_{1}\boxtimes\pi_{1}^{\uparrow F})^{\uparrow F}\big)\bef(\pi_{1}\boxtimes\pi_{1}^{\uparrow F})\\
{\color{greenunder}\text{use Eq.~(\ref{eq:pair-product-composition-law})}:}\quad & \quad=\big((\pi_{1}\boxtimes\pi_{1}^{\uparrow F})\bef\pi_{1}\big)\boxtimes\big(\gunderline{(\pi_{1}\boxtimes\pi_{1}^{\uparrow F})^{\uparrow F}\bef\pi_{1}^{\uparrow F}}\big)=\big(\gunderline{(\pi_{1}\boxtimes\pi_{1}^{\uparrow F})\bef\pi_{1}}\big)\boxtimes\big(\gunderline{(\pi_{1}\boxtimes\pi_{1}^{\uparrow F})\bef\pi_{1}}\big)^{\uparrow F}\\
{\color{greenunder}\text{use Eq.~(\ref{eq:pair-product-projection-laws})}:}\quad & \quad=(\pi_{1}\bef\pi_{1})\boxtimes(\pi_{1}\bef\pi_{1})^{\uparrow F}=(\pi_{1}\bef\pi_{1})\boxtimes(\pi_{1}^{\uparrow F}\bef\pi_{1}^{\uparrow F})\quad,\\
 & \text{ftn}_{2\,L}\bef\text{ftn}_{2\,L}=(\pi_{1}\boxtimes\pi_{1}^{\uparrow F})\bef(\pi_{1}\boxtimes\pi_{1}^{\uparrow F})=(\pi_{1}\bef\pi_{1})\boxtimes(\pi_{1}^{\uparrow F}\bef\pi_{1}^{\uparrow F})\quad.
\end{align*}

So, both implementations ($\text{ftn}_{2\,L}$ and $\text{ftn}_{2\,L}$)
are associative. Those implementations of \lstinline!flatten! discard
either the first $F$-effect or the second one. Since all implementations
of \lstinline!flatten! discard effects, identity laws cannot hold,
and so the semimonads of the form $L^{A}\triangleq A\times F^{A}$
are not monads. A semimonad $L$ combines $F$-effects similarly to
a \textsf{``}trivial\textsf{''} semigroup (see Section~\ref{subsec:Semigroups-constructions})
whose binary operation simply discards the left or the right argument.
Such semigroups cannot be made into monoids since the identity laws
will not hold. This construction illustrates the connections between
monoids, semigroups, monads, and semimonads.

\paragraph{Co-products}

As a rule, the co-product of two monads ($F^{A}+G^{A}$) is not a
monad. For simple examples, see Exercise~\ref{subsec:Exercise-1-monads-6}
for $\bbnum 1+F^{A}$ (where $F^{A}\triangleq A\times A$) and Exercise~\ref{subsec:Exercise-1-monads-4}
for $M^{A}+M^{A}$ with an arbitrary monad $M$. An exception to that
rule is a co-product with the \emph{identity} monad:

\subsubsection{Statement \label{subsec:Statement-co-product-with-identity-monad}\ref{subsec:Statement-co-product-with-identity-monad}}

If $F$ is any monad, the functor $L^{A}\triangleq A+F^{A}$ is a
monad. (The functor $L$ is called the \textbf{free pointed}\index{free pointed monad}\index{monads!free pointed}\index{free pointed functor}
\textbf{functor on} $F$, for reasons explained in Chapter~\ref{chap:Free-type-constructions}.)

\subparagraph{Proof}

We need to define the monad methods for $L$, for which we may use
the \lstinline!pure! and \lstinline!flatten! methods of $F$. Begin
with the \lstinline!flatten! method, which needs to have the type
signature:
\[
\text{ftn}_{L}:L^{L^{A}}\rightarrow L^{A}=A+F^{A}+F^{A+F^{A}}\rightarrow A+F^{A}\quad.
\]
Since we know nothing about the specific monad $F$, we cannot extract
a value of type $A$ out of $F^{A}$. However, we can use $F$\textsf{'}s \lstinline!pure!
method to create a value of type $F^{A}$ out of $A$. This allows
us to convert $A+F^{A}$ into $F^{A}$ using the function we will
denote $\gamma$:

\begin{wrapfigure}{l}{0.55\columnwidth}%
\vspace{-0.8\baselineskip}
\begin{lstlisting}
type L[A] = Either[A, F[A]]
def gamma[A]: L[A] => F[A] = {
  case Left(a)   => F.pure(a)
  case Right(fa) => fa
}
\end{lstlisting}

\vspace{-2\baselineskip}
\end{wrapfigure}%

~\vspace{-0.2\baselineskip}
\[
\gamma^{A}\triangleq\,\begin{array}{|c||c|}
 & F^{A}\\
\hline A & \text{pu}_{F}\\
F^{A} & \text{id}
\end{array}\quad.
\]

\noindent Lifting this function to $F$, we can convert $F^{A+F^{A}}$
into $F^{F^{A}}$ and finally into $F^{A}$ via $F$\textsf{'}s \lstinline!flatten!
method:

\begin{wrapfigure}{l}{0.55\columnwidth}%
\vspace{-0.5\baselineskip}
\begin{lstlisting}
def flatten_L[A]: L[L[A]] => L[A] = {
  case Left(Left(a))   => Left(a)
  case Left(Right(fa)) => Right(fa)
  case Right(g)        => Right(g.map(gamma).flatten)
}  // The last line equals `Right(g.flatMap(gamma))`.
\end{lstlisting}

\vspace{-3\baselineskip}
\end{wrapfigure}%

~\vspace{-1.1\baselineskip}
\[
\text{ftn}_{L}\triangleq\,\begin{array}{|c||cc|}
 & A & F^{A}\\
\hline A & \text{id} & \bbnum 0\\
F^{A} & \bbnum 0 & \text{id}\\
F^{L^{A}} & \bbnum 0 & \gamma^{\uparrow F}\bef\text{ftn}_{F}
\end{array}\quad.
\]

Is there another implementation for $\text{ftn}_{L}$? We could have
replaced $A$ by $F^{A}$ using $\text{pu}_{F}$. However, that code
would never return a result of type $A+\bbnum 0$, which makes it
impossible to satisfy identity laws such as $\text{pu}_{F}\bef\text{ftn}_{F}=\text{id}$.

The \lstinline!pure! method for $L$ could be defined in two ways:
$\text{pu}_{L}\triangleq a^{:A}\rightarrow a+\bbnum 0$ or $\text{pu}_{L}\triangleq a\rightarrow\bbnum 0+\text{pu}_{F}(a)$.
It turns out that only the first definition satisfies the monad $L$\textsf{'}s
identity laws (Exercise~\ref{subsec:Exercise-1-monads-12}).

To verify the identity laws for the definition $\text{pu}_{L}\triangleq a^{:A}\rightarrow a+\bbnum 0$,
begin by writing the lifting code for the functor $L$ and a fully
split matrix for $\text{pu}_{L}^{\uparrow L}$:
\[
(f^{:A\rightarrow B})^{\uparrow L}=\,\begin{array}{|c||cc|}
 & B & F^{B}\\
\hline A & f & \bbnum 0\\
F^{A} & \bbnum 0 & f^{\uparrow F}
\end{array}\quad,\quad\quad\text{pu}_{L}^{\uparrow L}=\,\begin{array}{|c||cc|}
 & L^{A} & F^{L^{A}}\\
\hline A & \text{pu}_{L} & \bbnum 0\\
F^{A} & \bbnum 0 & \text{pu}_{L}^{\uparrow F}
\end{array}\,=\,\begin{array}{|c||ccc|}
 & A & F^{A} & F^{L^{A}}\\
\hline A & \text{id} & \bbnum 0 & \bbnum 0\\
F^{A} & \bbnum 0 & \bbnum 0 & \text{pu}_{L}^{\uparrow F}
\end{array}\quad.
\]
Then write the two identity laws and simplify using matrix compositions:
\[
\text{pu}_{L}^{L^{A}}\bef\text{ftn}_{L}=\,\begin{array}{|c||ccc|}
 & A & F^{A} & F^{L^{A}}\\
\hline A & \text{id} & \bbnum 0 & \bbnum 0\\
F^{A} & \bbnum 0 & \text{id} & \bbnum 0
\end{array}\,\bef\,\begin{array}{|c||cc|}
 & A & F^{A}\\
\hline A & \text{id} & \bbnum 0\\
F^{A} & \bbnum 0 & \text{id}\\
F^{L^{A}} & \bbnum 0 & \gamma^{\uparrow F}\bef\text{ftn}_{F}
\end{array}\,=\,\begin{array}{|c||cc|}
 & A & F^{A}\\
\hline A & \text{id} & \bbnum 0\\
F^{A} & \bbnum 0 & \text{id}
\end{array}\,=\text{id}\quad,
\]
\[
\text{pu}_{L}^{\uparrow L}\bef\text{ftn}_{L}=\,\begin{array}{|c||ccc|}
 & A & F^{A} & F^{L^{A}}\\
\hline A & \text{id} & \bbnum 0 & \bbnum 0\\
F^{A} & \bbnum 0 & \bbnum 0 & \text{pu}_{L}^{\uparrow F}
\end{array}\,\bef\,\begin{array}{|c||cc|}
 & A & F^{A}\\
\hline A & \text{id} & \bbnum 0\\
F^{A} & \bbnum 0 & \text{id}\\
F^{L^{A}} & \bbnum 0 & \gamma^{\uparrow F}\bef\text{ftn}_{F}
\end{array}\,=\,\begin{array}{|c||cc|}
 & A & F^{A}\\
\hline A & \text{id} & \bbnum 0\\
F^{A} & \bbnum 0 & \text{pu}_{L}^{\uparrow F}\bef\gamma^{\uparrow F}\bef\text{ftn}_{F}
\end{array}\quad.
\]
We will show that the last matrix equals identity by proving that:
\[
\text{id}\overset{?}{=}\text{pu}_{L}^{\uparrow F}\bef\gamma^{\uparrow F}\bef\text{ftn}_{F}=(\text{pu}_{L}\bef\gamma)^{\uparrow F}\bef\text{ftn}_{F}\quad.
\]
Let us simplify the function $\text{pu}_{L}\bef\gamma$ in a separate
calculation:
\[
\text{pu}_{L}\bef\gamma=\,\begin{array}{|c||cc|}
 & A & F^{A}\\
\hline A & \text{id} & \bbnum 0
\end{array}\,\bef\,\begin{array}{|c||c|}
 & F^{A}\\
\hline A & \text{pu}_{F}\\
F^{A} & \text{id}
\end{array}\,=\text{pu}_{F}\quad.
\]
It remains to show that:
\[
\text{id}\overset{?}{=}(\text{pu}_{L}\bef\gamma)^{\uparrow F}\bef\text{ftn}_{F}=\text{pu}_{F}^{\uparrow F}\bef\text{ftn}_{F}\quad.
\]
But this is the right identity law of the monad $F$, which (by assumption)
is a lawful monad. 

The next step is to verify the associativity law for $L$. Write the
code for $\text{ftn}_{L}^{\uparrow L}$ and $\text{ftn}_{L}^{L^{A}}$:
\begin{align*}
 & \text{ftn}_{L}^{\uparrow L}=\,\begin{array}{|c||cc|}
 & L^{A} & F^{L^{A}}\\
\hline L^{L^{A}} & \text{ftn}_{L} & \bbnum 0\\
F^{L^{L^{A}}} & \bbnum 0 & \text{ftn}_{L}^{\uparrow F}
\end{array}\,=\,\begin{array}{|c||ccc|}
 & A & F^{A} & F^{L^{A}}\\
\hline A & \text{id} & \bbnum 0 & \bbnum 0\\
F^{A} & \bbnum 0 & \text{id} & \bbnum 0\\
F^{L^{A}} & \bbnum 0 & \gamma^{\uparrow F}\bef\text{ftn}_{F} & \bbnum 0\\
F^{L^{L^{A}}} & \bbnum 0 & \bbnum 0 & \text{ftn}_{L}^{\uparrow F}
\end{array}\quad,\\
 & \text{ftn}_{L}^{L^{A}}=\,\begin{array}{|c||cc|}
 & L^{A} & F^{L^{A}}\\
\hline L^{A} & \text{id} & \bbnum 0\\
F^{L^{a}} & \bbnum 0 & \text{id}\\
F^{L^{L^{A}}} & \bbnum 0 & \gamma^{\uparrow F}\bef\text{ftn}_{F}
\end{array}\,=\,\begin{array}{|c||ccc|}
 & A & F^{A} & F^{L^{A}}\\
\hline A & \text{id} & \bbnum 0 & \bbnum 0\\
F^{A} & \bbnum 0 & \text{id} & \bbnum 0\\
F^{L^{A}} & \bbnum 0 & \bbnum 0 & \text{id}\\
F^{L^{L^{A}}} & \bbnum 0 & \bbnum 0 & \gamma^{\uparrow F}\bef\text{ftn}_{F}
\end{array}\quad.
\end{align*}
We are ready to verify the associativity law. Simplify both sides
using matrix compositions:
\begin{align*}
 & \negthickspace\negthickspace\negthickspace\negthickspace\text{ftn}_{L}^{\uparrow L}\bef\text{ftn}_{L}=\,\begin{array}{|c||ccc|}
 & A & F^{A} & F^{L^{A}}\\
\hline A & \text{id} & \bbnum 0 & \bbnum 0\\
F^{A} & \bbnum 0 & \text{id} & \bbnum 0\\
F^{L^{A}} & \bbnum 0 & \gamma^{\uparrow F}\bef\text{ftn}_{F} & \bbnum 0\\
F^{L^{L^{A}}} & \bbnum 0 & \bbnum 0 & \text{ftn}_{L}^{\uparrow F}
\end{array}\,\bef\,\begin{array}{|c||cc|}
 & A & F^{A}\\
\hline A & \text{id} & \bbnum 0\\
F^{A} & \bbnum 0 & \text{id}\\
F^{L^{A}} & \bbnum 0 & \gamma^{\uparrow F}\bef\text{ftn}_{F}
\end{array}\,=\,\begin{array}{|c||cc|}
 & A & F^{A}\\
\hline A & \text{id} & \bbnum 0\\
F^{A} & \bbnum 0 & \text{id}\\
F^{L^{A}} & \bbnum 0 & \gamma^{\uparrow F}\bef\text{ftn}_{F}\\
F^{L^{L^{A}}} & \bbnum 0 & \text{ftn}_{L}^{\uparrow F}\bef\gamma^{\uparrow F}\bef\text{ftn}_{F}
\end{array}\quad,\\
 & \negthickspace\negthickspace\negthickspace\negthickspace\text{ftn}_{L}^{L^{A}}\bef\text{ftn}_{L}=\,\begin{array}{|c||ccc|}
 & A & F^{A} & F^{L^{A}}\\
\hline A & \text{id} & \bbnum 0 & \bbnum 0\\
F^{A} & \bbnum 0 & \text{id} & \bbnum 0\\
F^{L^{A}} & \bbnum 0 & \bbnum 0 & \text{id}\\
F^{L^{L^{A}}} & \bbnum 0 & \bbnum 0 & \gamma^{\uparrow F}\bef\text{ftn}_{F}
\end{array}\,\bef\,\begin{array}{|c||cc|}
 & A & F^{A}\\
\hline A & \text{id} & \bbnum 0\\
F^{A} & \bbnum 0 & \text{id}\\
F^{L^{A}} & \bbnum 0 & \gamma^{\uparrow F}\bef\text{ftn}_{F}
\end{array}\,=\,\begin{array}{|c||cc|}
 & A & F^{A}\\
\hline A & \text{id} & \bbnum 0\\
F^{A} & \bbnum 0 & \text{id}\\
F^{L^{A}} & \bbnum 0 & \gamma^{\uparrow F}\bef\text{ftn}_{F}\\
F^{L^{L^{A}}} & \bbnum 0 & \gamma^{\uparrow F}\bef\text{ftn}_{F}\bef\gamma^{\uparrow F}\bef\text{ftn}_{F}
\end{array}\quad.
\end{align*}
We will show that the last matrices are equal if we prove the equality
of the expressions:
\[
\text{ftn}_{L}^{\uparrow F}\bef\gamma^{\uparrow F}\bef\text{ftn}_{F}\overset{?}{=}\gamma^{\uparrow F}\bef\text{ftn}_{F}\bef\gamma^{\uparrow F}\bef\text{ftn}_{F}\quad.
\]
These expressions are functions of type $F^{L^{L^{A}}}\rightarrow F^{A}$.
Simplify them separately:
\begin{align*}
{\color{greenunder}\text{left-hand side}:}\quad & \text{ftn}_{L}^{\uparrow F}\bef\gamma^{\uparrow F}\bef\text{ftn}_{F}=\big(\text{ftn}_{L}\bef\gamma\big)^{\uparrow F}\bef\text{ftn}_{F}\quad,\\
{\color{greenunder}\text{right-hand side}:}\quad & \gamma^{\uparrow F}\bef\gunderline{\text{ftn}_{F}\bef\gamma^{\uparrow F}}\bef\text{ftn}_{F}\\
{\color{greenunder}\text{naturality of }\text{ftn}_{F}:}\quad & \quad=\gamma^{\uparrow F}\bef\gamma^{\uparrow F\uparrow F}\bef\gunderline{\text{ftn}_{F}\bef\text{ftn}_{F}}\\
{\color{greenunder}\text{associativity law of }F:}\quad & \quad=\gunderline{\gamma^{\uparrow F}\bef\gamma^{\uparrow F\uparrow F}\bef\text{ftn}_{F}^{\uparrow F}}\bef\text{ftn}_{F}=\big(\gamma\bef\gamma^{\uparrow F}\bef\text{ftn}_{F}\big)^{\uparrow F}\bef\text{ftn}_{F}\quad.
\end{align*}
It remains to show the equality of the functions under $\big(...\big)^{\uparrow F}$:
\begin{equation}
\text{ftn}_{L}\bef\gamma\overset{?}{=}\gunderline{\gamma\bef\gamma^{\uparrow F}}\bef\text{ftn}_{F}=\gamma^{\uparrow L}\bef\gamma\bef\text{ftn}_{F}\quad,\label{eq:monad-construction-a+f-a-derivation1}
\end{equation}
where in the last step we used the naturality law of $\gamma^{:L^{^{A}}\rightarrow F^{A}}$,
which is a natural transformation:
\[
\gamma\bef f^{\uparrow F}=f^{\uparrow L}\bef\gamma\quad,\quad\text{for all }f^{:A\rightarrow B}\quad.
\]
Finally, the functions in Eq.~(\ref{eq:monad-construction-a+f-a-derivation1})
are equal due to $F$\textsf{'}s identity law ($\text{pu}_{F}\bef\text{ftn}_{F}=\text{id}$):
\begin{align*}
 & \text{ftn}_{L}\bef\gamma=\,\begin{array}{|c||cc|}
 & A & F^{A}\\
\hline A & \text{id} & \bbnum 0\\
F^{A} & \bbnum 0 & \text{id}\\
F^{L^{A}} & \bbnum 0 & \gamma^{\uparrow F}\bef\text{ftn}_{F}
\end{array}\,\bef\,\begin{array}{|c||c|}
 & F^{A}\\
\hline A & \text{pu}_{F}\\
F^{A} & \text{id}
\end{array}\,=\,\begin{array}{|c||c|}
 & F^{A}\\
\hline A & \text{pu}_{F}\\
F^{A} & \text{id}\\
F^{L^{A}} & \gamma^{\uparrow F}\bef\text{ftn}_{F}
\end{array}\,=\,\begin{array}{|c||c|}
 & F^{A}\\
\hline L^{A} & \gamma\\
F^{L^{A}} & \gamma^{\uparrow F}\bef\text{ftn}_{F}
\end{array}\quad,\\
 & \negthickspace\negthickspace\negthickspace\negthickspace\gamma^{\uparrow L}\bef\gamma\bef\text{ftn}_{F}=\,\begin{array}{|c||cc|}
 & F^{A} & F^{F^{A}}\\
\hline L^{A} & \gamma & \bbnum 0\\
F^{L^{A}} & \bbnum 0 & \gamma^{\uparrow F}
\end{array}\,\bef\,\begin{array}{|c||c|}
 & F^{F^{A}}\\
\hline F^{A} & \text{pu}_{F}\\
F^{F^{A}} & \text{id}
\end{array}\,\bef\text{ftn}_{F}=\,\begin{array}{|c||c|}
 & F^{A}\\
\hline L^{A} & \gamma\bef\gunderline{\text{pu}_{F}\bef\text{ftn}_{F}}\\
F^{L^{A}} & \gamma^{\uparrow F}\bef\text{id}\bef\text{ftn}_{F}
\end{array}\,=\,\begin{array}{|c||c|}
 & F^{A}\\
\hline L^{A} & \gamma\\
F^{L^{A}} & \gamma^{\uparrow F}\bef\text{ftn}_{F}
\end{array}\quad.
\end{align*}
This concludes the proof.

\paragraph{Function types}

If $F$ is a monad, can we find a type constructor $H$ such that
$L^{A}\triangleq H^{A}\rightarrow F^{A}$ is a monad? In order for
$L$ to be a functor, $H$ must be a contrafunctor. We have already
seen that $Z\rightarrow A$ is a monad (a \lstinline!Reader!) when
$Z$ is a fixed type. The type expression $Z\rightarrow A$ is indeed
of the form $H^{A}\rightarrow F^{A}$ if the type constructors are
set to $H^{A}\triangleq Z$ and $F^{A}\triangleq A$. The following
statements show two generalizations of the \lstinline!Reader! monad
to other $H$ and $F$.

\subsubsection{Statement \label{subsec:Statement-monad-construction-1}\ref{subsec:Statement-monad-construction-1}}

For a fixed type $Z$, if a functor $F$ is a (semi)monad then so
is $L^{A}\triangleq Z\rightarrow F^{A}$.

\subparagraph{Proof}

As with other monads of function type, derivations are made shorter
by using the flipped Kleisli trick. We assume that $F$\textsf{'}s Kleisli
composition $\diamond_{_{F}}$ is known and obeys the associativity
law. When $F$ is a monad, we also assume that its pure method $\text{pu}_{F}$
is known and obeys the identity laws. For the monad $L$, we will
now define the \index{flipped@\textsf{``}flipped Kleisli\textsf{''} technique}flipped
Kleisli composition $\tilde{\diamond}_{_{L}}$ and prove its laws.

The ordinary Kleisli functions have types $A\rightarrow Z\rightarrow F^{B}$,
so the flipped Kleisli functions have types $Z\rightarrow A\rightarrow F^{B}$.
We begin by defining the flipped Kleisli composition $\tilde{\diamond}_{_{L}}$
and the flipped method $\tilde{\text{pu}}_{L}$ that has type $Z\rightarrow A\rightarrow F^{A}$
(note that we use the \emph{non-flipped} methods $\diamond_{_{F}}$
and $\text{pu}_{F}$ of the monad $F$):
\[
f^{:Z\rightarrow A\rightarrow F^{B}}\tilde{\diamond}_{_{L}}g^{:Z\rightarrow B\rightarrow F^{C}}\triangleq z^{:Z}\rightarrow f(z)\diamond_{_{F}}g(z)\quad,\quad\quad\tilde{\text{pu}}_{L}\triangleq\_^{:Z}\rightarrow\text{pu}_{F}\quad.
\]

To verify the laws, it is convenient to substitute an arbitrary $z^{:Z}$
and define $\tilde{\diamond}_{_{L}}$ and $\tilde{\text{pu}}_{L}$
by:
\[
z\triangleright(f\tilde{\diamond}_{_{L}}g)\triangleq(z\triangleright f)\diamond_{_{F}}(z\triangleright g)\quad,\quad\quad z\triangleright\tilde{\text{pu}}_{L}\triangleq\text{pu}_{F}\quad.
\]
For the left-hand side of the associativity law, we write:
\[
z\triangleright\big((f\,\tilde{\diamond}_{_{L}}g)\,\tilde{\diamond}_{_{L}}h\big)=\big(z\triangleright(f\,\tilde{\diamond}_{_{L}}g)\big)\diamond_{_{F}}(z\triangleright h)=\big(z\triangleright f)\diamond_{_{F}}(z\triangleright g)\diamond_{_{F}}(z\triangleright h)\quad,
\]
where we omitted parentheses around $\diamond_{_{F}}$ since its associativity
is given. The right-hand side:
\[
z\triangleright\big(f\,\tilde{\diamond}_{_{L}}(g\,\tilde{\diamond}_{_{L}}h\big)=(z\triangleright f)\diamond_{_{F}}\big(z\triangleright(g\,\tilde{\diamond}_{_{L}}h)\big)=\big(z\triangleright f)\diamond_{_{F}}(z\triangleright g)\diamond_{_{F}}(z\triangleright h)\quad.
\]
We obtained the same expression for both sides of the law.

Assuming now that $F$ is a monad with a lawful \lstinline!pure!
method, we verify $L$\textsf{'}s identity laws:
\begin{align*}
{\color{greenunder}\text{left identity law of }L:}\quad & z\triangleright\big(\tilde{\text{pu}}_{L}\tilde{\diamond}_{_{L}}f\big)=(z\triangleright\tilde{\text{pu}}_{L})\diamond_{_{F}}(z\triangleright f)=\gunderline{\text{pu}_{F}\diamond_{_{F}}}(z\triangleright f)=z\triangleright f\quad,\\
{\color{greenunder}\text{right identity law of }L:}\quad & z\triangleright\big(f\,\tilde{\diamond}_{_{L}}\tilde{\text{pu}}_{L}\big)=(z\triangleright f)\diamond_{_{F}}(z\triangleright\tilde{\text{pu}}_{L})=(z\triangleright f)\,\gunderline{\diamond_{_{F}}\text{pu}_{F}}=z\triangleright f\quad.
\end{align*}

It remains to convert the flipped Kleisli methods $\tilde{\diamond}_{_{L}}$
and $\tilde{\text{pu}}_{L}$ to the code of $L$\textsf{'}s \lstinline!flatMap!
and \lstinline!pure!:
\begin{lstlisting}
type L[A] = Z => F[A]           // The type Z and a semimonad F must be already defined.
def flatMap_L[A, B](la: L[A])(f: A => L[B]): L[B] = { z => la(z).flatMap(a => f(a)(z)) }
def pure_L[A](a: A): L[A] = { _ => implicitly[Monad[F]].pure(a) }
\end{lstlisting}


\subsubsection{Statement \label{subsec:Statement-monad-construction-2}\ref{subsec:Statement-monad-construction-2}}

For any contrafunctor $H^{A}$, the functor $L^{A}\triangleq H^{A}\rightarrow A$
is a monad.

\subparagraph{Proof}

We use the flipped Kleisli formulation for $L$. The flipped $L$-Kleisli
functions have types $H^{B}\rightarrow A\rightarrow B$ instead of
$A\rightarrow H^{B}\rightarrow B$. The flipped Kleisli composition
$\tilde{\diamond}_{_{L}}$ of an $f^{:H^{B}\rightarrow A\rightarrow B}$
and a $g^{:H^{C}\rightarrow B\rightarrow C}$ must have type $H^{C}\rightarrow A\rightarrow C$.
To infer this function\textsf{'}s code, begin with a typed hole:
\[
f^{:H^{B}\rightarrow A\rightarrow B}\tilde{\diamond}_{_{L}}g^{:H^{C}\rightarrow B\rightarrow C}=k^{:H^{C}}\rightarrow\text{???}^{:A\rightarrow C}\quad.
\]
Looking at the available data, we notice that a value of type $A\rightarrow C$
will be found if we apply $f$ and $g$ to some arguments and then
compose the resulting functions of types $A\rightarrow B$ and $B\rightarrow C$:
\[
\text{???}^{:A\rightarrow C}=f(\text{???}^{:H^{B}})\bef g(\text{???}^{:H^{C}})\quad.
\]
We already have a value $h$ of type $H^{C}$, so we can apply $g(k)$
and get a function of type $B\rightarrow C$. The remaining typed
hole requires a value of type $H^{B}$; but we only have the value
$k^{:H^{C}}$. We cannot create values of type $H^{B}$ from scratch.
The only way of filling that type hole is to transform $k$ into the
type $H^{B}$. This is possible by lifting a function of type $B\rightarrow C$
to the contrafunctor $H$. We already have such a function, namely
$g(k)$. So, we write:
\[
\text{???}^{:H^{B}}=k^{:H^{C}}\triangleright(\text{???}^{:B\rightarrow C})^{\downarrow H}=k\triangleright(g(k))^{\downarrow H}\quad.
\]
Putting the entire code together and substituting an arbitrary value
$k^{:H^{C}}$, we get:
\begin{equation}
k^{:H^{C}}\triangleright\big(f^{:H^{B}\rightarrow A\rightarrow B}\tilde{\diamond}_{_{L}}g^{:H^{C}\rightarrow B\rightarrow C}\big)\triangleq f\big(k\triangleright(g(k))^{\downarrow H}\big)\bef g(k)\quad.\label{eq:def-of-Kleisli-for-exp-construction-h-a-a}
\end{equation}

The flipped \lstinline!pure! method ($\tilde{\text{pu}}_{L}$) is
defined by:
\begin{equation}
\tilde{\text{pu}}_{L}^{:H^{A}\rightarrow A\rightarrow A}\triangleq\_^{:H^{A}}\rightarrow\text{id}^{:A\rightarrow A}\quad,\quad\quad\tilde{\text{pu}}_{L}(k)=\text{id}\quad.\label{eq:def-of-pure-for-exp-construction-h-a-a}
\end{equation}
These formulas yield the following Scala definitions of \lstinline!pure!
and \lstinline!flatMap! (see Statement~\ref{subsec:Statement-choice-monad-definition-of-flm}):
\begin{lstlisting}[mathescape=true]
type L[A] = H[A] => A                 // The contrafunctor H must be already defined.
def pure_L[A](a: A): L[A] = { _ => a }
def flatMap_L[B, C](lb: L[B])(g: B => L[C]): L[C] = { (k: H[C]) =>
  val bc: (B => C) = { b => g(b)(k) } // Corresponds to $\color{dkgreen}g(k):B\rightarrow C$.
  val hb: H[B] = k.contramap(bc)     // Fill the typed hole $\color{dkgreen}\textrm{???}:\,\scriptstyle H^B$.
  g(lb(hb))(k)          // Corresponds to the composition $\color{dkgreen} f(...) \bef g(k)$.
}
\end{lstlisting}

Let us now verify the monad laws of $L$ in the flipped Kleisli form.
The identity laws:
\begin{align*}
{\color{greenunder}\text{expect }g(k):}\quad & k\triangleright\big(\tilde{\text{pu}}_{L}\tilde{\diamond}_{_{L}}g\big)=\gunderline{\tilde{\text{pu}}_{L}(k\triangleright(g(k))^{\downarrow H})}\bef g(k)=\gunderline{\text{id}}\bef g(k)=g(k)\quad,\\
{\color{greenunder}\text{expect }f(k):}\quad & k\triangleright\big(f\,\tilde{\diamond}_{_{L}}\tilde{\text{pu}}_{L}\big)=f(k\triangleright(\gunderline{\tilde{\text{pu}}_{L}(k)})^{\downarrow H})\bef\gunderline{\tilde{\text{pu}}_{L}(k)}=f(k\triangleright\gunderline{\text{id}^{\downarrow H}})\bef\gunderline{\text{id}}=f(k)\quad.
\end{align*}

To verify the associativity law, write the two sides separately and
use Eq.~(\ref{eq:def-of-Kleisli-for-exp-construction-h-a-a}) repeatedly:
\begin{align*}
 & k^{:H^{C}}\triangleright\big((f\,\tilde{\diamond}_{_{L}}g)\,\tilde{\diamond}_{_{L}}h\big)=\big(\gunderline{(f\,\tilde{\diamond}_{_{L}}g)(k\triangleright(h(k))^{\downarrow H})}\big)\bef h(k)=\big(\gunderline{k\triangleright(h(k))^{\downarrow H}\triangleright(f\,\tilde{\diamond}_{_{L}}g)}\big)\bef h(k)\\
 & \quad=f\big(k\triangleright(h(k))^{\downarrow H}\triangleright(g(k\triangleright(h(k))^{\downarrow H}))^{\downarrow H}\big)\bef g(k\triangleright(h(k))^{\downarrow H})\bef h(k)\quad,\\
 & k\triangleright\big(f\,\tilde{\diamond}_{_{L}}(g\,\tilde{\diamond}_{_{L}}h)\big)=f\big(k\triangleright(\gunderline{(g\,\tilde{\diamond}_{_{L}}h)(k)})^{\downarrow H}\big)\bef\gunderline{(g\,\tilde{\diamond}_{_{L}}h)(k)}=f\big(k\triangleright(\gunderline{k\triangleright(g\,\tilde{\diamond}_{_{L}}h)})^{\downarrow H}\big)\bef\gunderline{k\triangleright(g\,\tilde{\diamond}_{_{L}}h)}\\
 & \quad=f\big(k\triangleright(\gunderline{g(k\triangleright(h(k))^{\downarrow H})\bef h(k)})^{\downarrow H}\big)\bef g(k\triangleright(h(k))^{\downarrow H})\bef h(k)\\
{\color{greenunder}\text{lift to }H:}\quad & \quad=f\big(k\triangleright(h(k))^{\downarrow H}\triangleright(g(k\triangleright(h(k))^{\downarrow H}))^{\downarrow H}\big)\bef g(k\triangleright(h(k))^{\downarrow H})\bef h(k)\quad.
\end{align*}
Now both sides are rewritten into identical expressions. This confirms
the associativity law for $L$. $\square$

Statement~\ref{subsec:Statement-monad-construction-2} applies to
type constructors of the form $H^{A}\rightarrow A$ but not to $H^{A}\rightarrow G^{A}$
with an arbitrary monad $G$. It turns out that $H^{A}\rightarrow G^{A}$
is a monad when the contrafunctor $H$ is suitably adapted to the
monad $G$. The necessary properties of $H$ are formulated by the
notion of \index{$M$-filterable contrafunctor} $G$-filterable contrafunctor,
which we briefly introduced in Section~\ref{subsec:Motivation-for-using-category-theory}.
We repeat the definition here:

\subsubsection{Definition \label{subsec:Definition-M-filterable-contrafunctor}\ref{subsec:Definition-M-filterable-contrafunctor}}

For a given monad $G$, a contrafunctor $H$ is $G$\textbf{-filterable}
if it has a \lstinline!lift! method,
\[
\text{lift}_{G,H}:(A\rightarrow G^{B})\rightarrow H^{B}\rightarrow H^{A}\quad,
\]
satisfying the identity and (contravariant) composition laws\index{identity laws!of $M$-filterable contrafunctors}\index{composition law!of $M$-filterable contrafunctors}
\[
\text{lift}_{G,H}(\text{pu}_{G})=\text{id}^{:H^{A}\rightarrow H^{A}}\quad,\quad\quad\text{lift}_{G,H}(f\diamond_{_{G}}g)=\text{lift}_{G,H}(g)\bef\text{lift}_{G,H}(f)\quad.
\]

Let us look at some simple examples of $G$-filterable contrafunctors.

\subsubsection{Statement \label{subsec:Statement-examples-of-filterable-contrafunctors}\ref{subsec:Statement-examples-of-filterable-contrafunctors}}

For any given (semi)monad $G$ and any fixed type $Z$, the following
contrafunctors are $G$-filterable: \textbf{(a)} $H^{A}\triangleq Z$,
\textbf{(b)} $H^{A}\triangleq G^{A}\rightarrow Z$, \textbf{(c)} $H^{A}\triangleq A\rightarrow G^{Z}$.

\subparagraph{Proof}

In each case, we need to define the function $\text{lift}_{G,H}$
and verify its laws. If $G$ is only a semimonad, we will derive the
composition law of $\text{lift}_{G,H}$ assuming only the associativity
law of $G$ (but not assuming any identity laws). For a full monad
$G$, we will derive the identity law of $\text{lift}_{G,H}$ by using
the identity laws of $G$. 

\textbf{(a)} Since $H^{A}\triangleq Z$ is a constant contrafunctor,
we define $\text{lift}_{G,H}(f)\triangleq\text{id}^{:Z\rightarrow Z}$.
The identity function will always satisfy the identity and composition
laws.

\textbf{(b)} Define the function $\text{lift}_{G,H}$ by:
\begin{lstlisting}
type H[A] = G[A] => Z       // The type Z and the monad G must be already defined.
def lift_G[A, B](k: H[B])(f: A => G[B]): H[A] = { (ga: G[A]) => k(ga.flatMap(f)) }
\end{lstlisting}
\[
\text{lift}_{G,H}(f^{:A\rightarrow G^{B}})\triangleq k^{:G^{B}\rightarrow Z}\rightarrow\text{???}^{:G^{A}\rightarrow Z}=k^{:G^{B}\rightarrow Z}\rightarrow\text{flm}_{G}(f)\bef k\quad,\quad\quad k\triangleright\text{lift}_{G,H}(f)\triangleq\text{flm}_{G}(f)\bef k\quad.
\]
Verify the laws of $\text{lift}_{G,H}$ using the right identity and
the associativity law of $G$\textsf{'}s \lstinline!flatMap!:
\begin{align*}
{\color{greenunder}\text{identity law; expect to equal }k:}\quad & k\triangleright\text{lift}_{G,H}(\text{pu}_{G})=\gunderline{\text{flm}_{G}(\text{pu}_{G})}\bef k=\text{id}\bef k=k\quad,\\
{\color{greenunder}\text{left-hand side of composition law}:}\quad & k\triangleright\text{lift}_{G,H}(f\diamond_{_{G}}g)=\text{flm}_{G}(f\diamond_{_{G}}g)\bef k=\gunderline{\text{flm}_{G}(f\bef\text{flm}_{G}(g))}\bef k\\
{\color{greenunder}\text{associativity law of }\text{flm}_{G}:}\quad & \quad=\text{flm}_{G}(f)\bef\text{flm}_{G}(g)\bef k\quad,\\
{\color{greenunder}\text{right-hand side of composition law}:}\quad & k\triangleright\text{lift}_{G,H}(g)\bef\text{lift}_{G,H}(f)=k\triangleright\text{lift}_{G,H}(g)\,\gunderline{\triangleright\,\text{lift}_{G,H}(f)}\\
 & \quad=\text{flm}_{G}(f)\bef\big(\gunderline{k\triangleright\text{lift}_{G,H}(g)}\big)=\text{flm}_{G}(f)\bef\text{flm}_{G}(g)\bef k\quad.
\end{align*}
Both sides of the composition law of $\text{lift}_{G,H}$ are now
equal.

\textbf{(c)} Define the function $\text{lift}_{G,H}$ by:
\begin{lstlisting}
type H[A] = A => G[Z]  // The type Z and the monad G must be already defined.
def lift_G[A, B](k: H[B])(f: A => G[B]): H[A] = { (a: A) => f(a).flatMap(k) }
\end{lstlisting}
\[
\text{lift}_{G,H}(f^{:A\rightarrow G^{B}})\triangleq k^{:B\rightarrow G^{Z}}\rightarrow\text{???}^{:A\rightarrow G^{Z}}=k^{:G^{B}\rightarrow Z}\rightarrow f\diamond_{_{G}}k\quad,\quad\quad k\triangleright\text{lift}_{G,H}(f)\triangleq f\diamond_{_{G}}k\quad.
\]
The laws of $\text{lift}_{G,H}$ are satisfied because the operation
$\diamond_{_{G}}$ obeys its identity and associativity laws:
\begin{align*}
{\color{greenunder}\text{identity law; expect to equal }k:}\quad & k\triangleright\text{lift}_{G,H}(\text{pu}_{G})=\text{pu}_{G}\diamond_{_{G}}k=k\quad,\\
{\color{greenunder}\text{left-hand side of composition law}:}\quad & k\triangleright\text{lift}_{G,H}(f\diamond_{_{G}}g)=(f\diamond_{_{G}}g)\,\diamond_{_{G}}k\quad,\\
{\color{greenunder}\text{right-hand side of composition law}:}\quad & k\triangleright\text{lift}_{G,H}(g)\bef\text{lift}_{G,H}(f)=k\triangleright\text{lift}_{G,H}(g)\,\gunderline{\triangleright\,\text{lift}_{G,H}(f)}\\
 & \quad=f\diamond_{_{G}}\big(\gunderline{k\triangleright\text{lift}_{G,H}(g)}\big)=f\diamond_{_{G}}(g\diamond_{_{G}}k)\quad.
\end{align*}
Both sides of the composition law of $\text{lift}_{G,H}$ are equal
due to the associativity law of $\diamond_{_{G}}$. $\quad$$\square$

We will now use filterable contrafunctors to generalize Statements~\ref{subsec:Statement-monad-construction-1}
and~\ref{subsec:Statement-monad-construction-2}.

\subsubsection{Statement \label{subsec:Statement-monad-construction-3}\ref{subsec:Statement-monad-construction-3}}

For any (semi)monad $G$ and any $G$-filterable contrafunctor $H$,
the functor $L$ defined by $L^{A}\triangleq H^{A}\rightarrow G^{A}$
is a (semi)monad.

\subparagraph{Proof}

We use the flipped Kleisli formulation for $L$ and the ordinary Kleisli
formulation for $G$. The flipped $L$-Kleisli functions have types
$H^{B}\rightarrow A\rightarrow G^{B}$ instead of $A\rightarrow H^{B}\rightarrow G^{B}$.
The flipped Kleisli composition $\tilde{\diamond}_{_{L}}$ of an $f^{:H^{B}\rightarrow A\rightarrow G^{B}}$
and a $g^{:H^{C}\rightarrow B\rightarrow G^{C}}$ must have type $H^{C}\rightarrow A\rightarrow G^{C}$.
Similarly to what we did in the proof of Statement~\ref{subsec:Statement-monad-construction-2},
we will define $\tilde{\diamond}_{_{L}}$ by applying $f$ and $g$
to typed holes, getting values of types $A\rightarrow G^{B}$ and
$B\rightarrow G^{C}$, and using the given operation $\diamond_{_{G}}$:
\[
f^{:H^{B}\rightarrow A\rightarrow G^{B}}\tilde{\diamond}_{_{L}}g^{:H^{C}\rightarrow B\rightarrow G^{C}}=k^{:H^{C}}\rightarrow\text{???}^{:A\rightarrow G^{C}}=k^{:H^{C}}\rightarrow f(\text{???}^{:H^{B}})\diamond_{_{G}}g(\text{???}^{:H^{C}})\quad.
\]
The typed hole $\text{???}^{:H^{C}}$ is filled by $k$, while $\text{???}^{:H^{B}}$
is computed from $k$ using $\text{lift}_{G,H}$:
\[
\text{???}^{:H^{B}}=k\triangleright\text{lift}_{G,H}(\text{???}^{:B\rightarrow G^{C}})=k\triangleright\text{lift}_{G,H}(g(k))\quad.
\]
Putting the code together, we obtain
\begin{align*}
 & k^{:H^{C}}\triangleright\big(f^{:H^{B}\rightarrow A\rightarrow G^{B}}\tilde{\diamond}_{_{L}}g^{:H^{C}\rightarrow B\rightarrow G^{C}}\big)\triangleq f\big(k\triangleright\text{lift}_{G,H}(g(k))\big)\diamond_{_{G}}g(k)\quad,\\
 & k^{:H^{C}}\triangleright\tilde{\text{pu}}_{L}\triangleq\text{pu}_{G}\quad.
\end{align*}
These definitions reduce to Eqs.~(\ref{eq:def-of-Kleisli-for-exp-construction-h-a-a})\textendash (\ref{eq:def-of-pure-for-exp-construction-h-a-a})
if we set $G$ to the identity monad, $G^{A}=\text{Id}^{A}\triangleq A$
. Any contrafunctor $H$ is $\text{Id}$-filterable because we can
define the required method $\text{lift}_{\text{Id},H}$ by:
\[
\text{lift}_{\text{Id},H}(f^{:B\rightarrow C})\triangleq f^{\downarrow H}\quad.
\]
The laws of $\text{lift}_{\text{Id},H}$ follow from the identity
and composition laws of the contrafunctor $H$.

Translating the monad $L$\textsf{'}s Kleisli composition into Scala code for
\lstinline!flatMap!, we obtain:
\begin{lstlisting}[mathescape=true]
type L[A] = H[A] => G[A] // A contrafunctor H and a monad G must be already defined.
def flatMap_L[A, B](la: L[A])(g: A => L[B]): L[B] = { (k: H[B]) =>
  val agb: (A => G[B]) = { a => g(a)(k) }    // Corresponds to flipped $\color{dkgreen}g(k)$.
  val ha: H[A] = lift_G(k)(agb)    // Corresponds to $\color{dkgreen}k\triangleright  \textrm{lift}_{G,H}(g(k))$.
  la(ha).flatMap(agb)    // Corresponds to the Kleisli composition $\color{dkgreen} f(...) \diamond_{_G}g(k)$.
}
\end{lstlisting}

We now verify the laws via derivations similar to those in the proof
of Statement~\ref{subsec:Statement-monad-construction-2}.

We will need to use the laws of $\text{lift}_{G,H}$ (we will write
simply \textsf{``}$\text{lift}$\textsf{''} for brevity). The identity laws of $L$
follow from the identity laws of $\text{lift}_{G,H}$ and $G$:
\begin{align*}
{\color{greenunder}\text{expect }g(k):}\quad & k\triangleright\big(\tilde{\text{pu}}_{L}\tilde{\diamond}_{_{L}}g\big)=\gunderline{\tilde{\text{pu}}_{L}(k\triangleright\text{lift}\,(g(k)))}\diamond_{_{G}}g(k)=\text{pu}_{G}\diamond_{_{G}}g(k)=g(k)\quad,\\
{\color{greenunder}\text{expect }f(k):}\quad & k\triangleright\big(f\,\tilde{\diamond}_{_{L}}\tilde{\text{pu}}_{L}\big)=f\big(k\triangleright\gunderline{\text{lift}\,(\tilde{\text{pu}}_{L}(k))}\big)\bef\gunderline{\tilde{\text{pu}}_{L}(k)}=f(k\triangleright\gunderline{\text{id}})\diamond_{_{G}}\text{pu}_{G}=f(k)\quad.
\end{align*}

The associativity law of $L$ follows from the associativity of $\diamond_{_{G}}$
and the composition law of $\text{lift}_{G,H}$:
\begin{align*}
 & k^{:H^{C}}\triangleright\big((f\,\tilde{\diamond}_{_{L}}g)\,\tilde{\diamond}_{_{L}}h\big)=\big(\gunderline{(f\tilde{\diamond}_{_{L}}g)(k\triangleright\text{lift}\,(h(k)))}\big)\diamond_{_{G}}h(k)=\big(\gunderline{k\triangleright\text{lift}\,(h(k))\triangleright(f\,\tilde{\diamond}_{_{L}}g)}\big)\diamond_{_{G}}h(k)\\
 & \quad=f\big(k\triangleright\text{lift}\,(h(k))\triangleright\text{lift}\,(g(k\triangleright\text{lift}\,(h(k))))\big)\diamond_{_{G}}g\big(k\triangleright\text{lift}\,(h(k))\big)\diamond_{_{G}}h(k)\quad,\\
 & k\triangleright\big(f\,\tilde{\diamond}_{_{L}}(g\,\tilde{\diamond}_{_{L}}h)\big)=f\big(k\triangleright\text{lift}\,(\gunderline{(g\,\tilde{\diamond}_{_{L}}h)(k)})\big)\diamond_{_{G}}\gunderline{(g\,\tilde{\diamond}_{_{L}}h)(k)}=f\big(k\triangleright\text{lift}\,(\gunderline{k\triangleright(g\,\tilde{\diamond}_{_{L}}h)})\big)\diamond_{_{G}}\gunderline{k\triangleright(g\,\tilde{\diamond}_{_{L}}h)}\\
 & \quad=f\big(k\triangleright\text{lift}\,(\gunderline{g(k\triangleright\text{lift}\,(h(k)))\diamond_{_{G}}h(k)})\big)\diamond_{_{G}}g(k\triangleright\text{lift}\,(h(k)))\diamond_{_{G}}h(k)\\
 & \quad=f\big(k\triangleright\text{lift}\,(h(k))\bef\text{lift}\,(g(k\triangleright\text{lift}\,(h(k))))\big)\diamond_{_{G}}g\big(k\triangleright\text{lift}\,(h(k))\big)\diamond_{_{G}}h(k)\quad.
\end{align*}
Both sides of the associativity law are now converted into identical
expressions. $\square$

Some examples of monads obtained with these constructions for $G^{A}\triangleq\bbnum 1+A$
are:
\[
\text{Sel}^{Z,A}\triangleq\left(A\rightarrow Z\right)\rightarrow A\quad,\quad\quad\text{Search}^{Z,A}\triangleq\left(A\rightarrow\bbnum 1+Z\right)\rightarrow\bbnum 1+A\quad,\quad\quad L^{A}\triangleq\left(\bbnum 1+A\rightarrow Z\right)\rightarrow\bbnum 1+A\quad.
\]
The first two are known as the selector monad\index{monads!Sel (selector) monad@\texttt{Sel} (selector) monad}
and the \index{monads!Search monad@\texttt{Search} monad}search monad.\footnote{See \texttt{\href{http://math.andrej.com/2008/11/21/a-haskell-monad-for-infinite-search-in-finite-time/}{http://math.andrej.com/2008/11/21/a-haskell-monad-for-infinite-search-in-finite-time/}}}
Statement~\ref{subsec:Statement-monad-construction-3} shows how
to implement the monad methods for these monads and guarantees that
the laws hold.

\paragraph{Recursive types}

The tree-like monads can be generalized to arbitrary shape of leaves
and branches. This gives two constructions that create a monad and
a semimonad from an arbitrary functor.

\subsubsection{Statement \label{subsec:Statement-monad-construction-4-free-monad}\ref{subsec:Statement-monad-construction-4-free-monad}}

For any given functor $F$, the recursively defined functor $L^{A}\triangleq A+F^{L^{A}}$
is a (tree-like) monad. This monad is called the \textbf{free monad}\index{free monad}\textbf{
on} $F$, for reasons explained in Chapter~\ref{chap:Free-type-constructions}.

\subparagraph{Proof}

Begin by implementing the monad methods for $L$ as usual for a tree-like
monad:
\begin{lstlisting}
type L[A] = Either[A, F[L[A]]] // The functor F must be already defined.
def pure_L[A](a: A): L[A] = Left(a)
def flatMap_L[A, B](la: L[A])(f: A => L[B]): L[B] = la match {
  case Left(a)    => f(a)
  case Right(fla) => Right(fla.map { (x: L[A]) => flatMap_L(x)(f) })  // Recursive call of flatMap_L. 
}
def flatten_L[A]: L[L[A]] => L[A] = {   // Match on L[L[A]] = Either[Either[A, F[L[A]]], F[L[L[A]]]].
  case Left(a)              => Left(a)
  case Left(Right(fla))     => Right(fla)
  case Right(Right(flla))   => Right(flla.map(x => flatten_L(x)))     // Recursive call of flatten_L.
}
\end{lstlisting}
\[
\text{pu}_{L}\triangleq\,\begin{array}{|c||cc|}
 & A & F^{L^{A}}\\
\hline A & \text{id} & \bbnum 0
\end{array}\quad,\quad\quad\text{ftn}_{L}^{A}:L^{L^{A}}\rightarrow L^{A}\triangleq\,\begin{array}{|c||cc|}
 & A & F^{L^{A}}\\
\hline A & \text{id} & \bbnum 0\\
F^{L^{A}} & \bbnum 0 & \text{id}\\
F^{L^{L^{A}}} & \bbnum 0 & \overline{\text{ftn}}_{L}^{\uparrow F}
\end{array}\quad.
\]

To verify the monad laws for $L$, we need to lift $\text{pu}_{L}$
and $\text{ftn}_{L}$ to the functor $L$. The lifting code is
\[
(f^{:A\rightarrow B})^{\uparrow L}=\,\begin{array}{|c||cc|}
 & B & F^{L^{B}}\\
\hline A & f & \bbnum 0\\
F^{L^{A}} & \bbnum 0 & \big(f^{\overline{\uparrow L}}\big)^{\uparrow F}
\end{array}\quad.
\]
In order to compute function compositions such as $\text{pu}_{L}^{\uparrow L}\bef\text{ftn}_{L}$
and $\text{ftn}_{L}^{\uparrow L}\bef\text{ftn}$ in the matrix notation,
we need to expand the matrices of $\text{pu}_{L}$ and $\text{ftn}_{L}$,
so that their output types are fully split:
\[
\text{pu}_{L}^{\uparrow L}=\,\begin{array}{|c||cc|}
 & L^{A} & F^{L^{L^{A}}}\\
\hline A & \text{pu}_{L} & \bbnum 0\\
F^{L^{A}} & \bbnum 0 & \big(\text{pu}_{L}^{\overline{\uparrow L}}\big)^{\uparrow F}
\end{array}\,=\,\begin{array}{|c||ccc|}
 & A & F^{L^{A}} & F^{L^{L^{A}}}\\
\hline A & \text{id} & \bbnum 0 & \bbnum 0\\
F^{L^{A}} & \bbnum 0 & \bbnum 0 & \big(\text{pu}_{L}^{\overline{\uparrow L}}\big)^{\uparrow F}
\end{array}\quad,
\]
\[
\text{ftn}_{L}^{\uparrow L}=\,\begin{array}{|c||cc|}
 & L^{A} & F^{L^{L^{A}}}\\
\hline L^{L^{A}} & \text{ftn}_{L} & \bbnum 0\\
F^{L^{L^{L^{A}}}} & \bbnum 0 & \big(\text{ftn}_{L}^{\overline{\uparrow L}}\big)^{\uparrow F}
\end{array}\,=\,\begin{array}{|c||ccc|}
 & A & F^{L^{A}} & F^{L^{L^{A}}}\\
\hline A & \text{id} & \bbnum 0 & \bbnum 0\\
F^{L^{A}} & \bbnum 0 & \text{id} & \bbnum 0\\
F^{L^{L^{A}}} & \bbnum 0 & \overline{\text{ftn}}_{L}^{\uparrow F} & \bbnum 0\\
F^{L^{L^{L^{A}}}} & \bbnum 0 & \bbnum 0 & \big(\text{ftn}_{L}^{\overline{\uparrow L}}\big)^{\uparrow F}
\end{array}\quad.
\]
We can now verify the identity and associativity laws of $\text{pu}_{L}$
and $\text{ftn}_{L}$. Identity laws:
\begin{align*}
 & \text{pu}_{L}^{L^{A}}\bef\text{ftn}_{L}=\,\begin{array}{|c||ccc|}
 & A & F^{L^{A}} & F^{L^{L^{A}}}\\
\hline A & \text{id} & \bbnum 0 & \bbnum 0\\
F^{L^{A}} & \bbnum 0 & \text{id} & \bbnum 0
\end{array}\,\bef\,\begin{array}{|c||cc|}
 & A & F^{L^{A}}\\
\hline A & \text{id} & \bbnum 0\\
F^{L^{A}} & \bbnum 0 & \text{id}\\
F^{L^{L^{A}}} & \bbnum 0 & \overline{\text{ftn}}_{L}^{\uparrow F}
\end{array}\,=\,\begin{array}{|c||cc|}
 & A & F^{L^{A}}\\
\hline A & \text{id} & \bbnum 0\\
F^{L^{A}} & \bbnum 0 & \text{id}
\end{array}\,=\text{id}\quad,\\
 & \text{pu}_{L}^{\uparrow L}\bef\text{ftn}_{L}=\,\begin{array}{|c||ccc|}
 & A & F^{L^{A}} & F^{L^{L^{A}}}\\
\hline A & \text{id} & \bbnum 0 & \bbnum 0\\
F^{L^{A}} & \bbnum 0 & \bbnum 0 & \big(\text{pu}_{L}^{\overline{\uparrow L}}\big)^{\uparrow F}
\end{array}\,\bef\,\begin{array}{|c||cc|}
 & A & F^{L^{A}}\\
\hline A & \text{id} & \bbnum 0\\
F^{L^{A}} & \bbnum 0 & \text{id}\\
F^{L^{L^{A}}} & \bbnum 0 & \overline{\text{ftn}}_{L}^{\uparrow F}
\end{array}\\
 & \quad=\,\begin{array}{|c||cc|}
 & A & F^{L^{A}}\\
\hline A & \text{id} & \bbnum 0\\
F^{L^{A}} & \bbnum 0 & \gunderline{\big(\text{pu}_{L}^{\overline{\uparrow L}}\big)^{\uparrow F}\bef\overline{\text{ftn}}_{L}^{\uparrow F}}
\end{array}\,=\,\begin{array}{|c||cc|}
 & A & F^{L^{A}}\\
\hline A & \text{id} & \bbnum 0\\
F^{L^{A}} & \bbnum 0 & \text{id}
\end{array}\,=\text{id}\quad.
\end{align*}
In the last line, we used the inductive assumption: $\big(\text{pu}_{L}^{\overline{\uparrow L}}\big)^{\uparrow F}\bef\overline{\text{ftn}}_{L}^{\uparrow F}=\text{id}$.

To verify the associativity law, simplify its both sides separately
and use fully split matrices:
\begin{align*}
 & \text{ftn}_{L}^{L^{A}}\bef\text{ftn}_{L}=\,\begin{array}{|c||ccc|}
 & A & F^{L^{A}} & F^{L^{L^{A}}}\\
\hline A & \text{id} & \bbnum 0 & \bbnum 0\\
F^{L^{A}} & \bbnum 0 & \text{id} & \bbnum 0\\
F^{L^{L^{A}}} & \bbnum 0 & \bbnum 0 & \text{id}\\
F^{L^{L^{L^{A}}}} & \bbnum 0 & \bbnum 0 & \overline{\text{ftn}}_{L}^{\uparrow F}
\end{array}\,\bef\,\begin{array}{|c||cc|}
 & A & F^{L^{A}}\\
\hline A & \text{id} & \bbnum 0\\
F^{L^{A}} & \bbnum 0 & \text{id}\\
F^{L^{L^{A}}} & \bbnum 0 & \overline{\text{ftn}}_{L}^{\uparrow F}
\end{array}\,=\,\begin{array}{|c||cc|}
 & A & F^{L^{A}}\\
\hline A & \text{id} & \bbnum 0\\
F^{L^{A}} & \bbnum 0 & \text{id}\\
F^{L^{L^{A}}} & \bbnum 0 & \overline{\text{ftn}}_{L}^{\uparrow F}\\
F^{L^{L^{L^{A}}}} & \bbnum 0 & \overline{\text{ftn}}_{L}^{\uparrow F}\bef\overline{\text{ftn}}_{L}^{\uparrow F}
\end{array}\quad,\\
 & \negthickspace\negthickspace\negthickspace\negthickspace\text{ftn}_{L}^{\uparrow L}\bef\text{ftn}_{L}=\,\begin{array}{|c||ccc|}
 & A & F^{L^{A}} & F^{L^{L^{A}}}\\
\hline A & \text{id} & \bbnum 0 & \bbnum 0\\
F^{L^{A}} & \bbnum 0 & \text{id} & \bbnum 0\\
F^{L^{L^{A}}} & \bbnum 0 & \overline{\text{ftn}}_{L}^{\uparrow F} & \bbnum 0\\
F^{L^{L^{L^{A}}}} & \bbnum 0 & \bbnum 0 & \big(\text{ftn}_{L}^{\overline{\uparrow L}}\big)^{\uparrow F}
\end{array}\,\bef\,\begin{array}{|c||cc|}
 & A & F^{L^{A}}\\
\hline A & \text{id} & \bbnum 0\\
F^{L^{A}} & \bbnum 0 & \text{id}\\
F^{L^{L^{A}}} & \bbnum 0 & \overline{\text{ftn}}_{L}^{\uparrow F}
\end{array}\,=\,\begin{array}{|c||cc|}
 & A & F^{L^{A}}\\
\hline A & \text{id} & \bbnum 0\\
F^{L^{A}} & \bbnum 0 & \text{id}\\
F^{L^{L^{A}}} & \bbnum 0 & \overline{\text{ftn}}_{L}^{\uparrow F}\\
F^{L^{L^{L^{A}}}} & \bbnum 0 & \big(\text{ftn}_{L}^{\overline{\uparrow L}}\big)^{\uparrow F}\bef\overline{\text{ftn}}_{L}^{\uparrow F}
\end{array}\quad.
\end{align*}
The two resulting matrices are equal due to the inductive assumption:
$\overline{\text{ftn}}_{L}\bef\overline{\text{ftn}}_{L}=\text{ftn}_{L}^{\overline{\uparrow L}}\bef\overline{\text{ftn}}_{L}$.

\subsubsection{Statement \label{subsec:Statement-monad-construction-5}\ref{subsec:Statement-monad-construction-5}}

For any functor $F$, the recursive functor $L^{A}\triangleq F^{A}+F^{L^{A}}$
(representing a tree with $F$-shaped leaves and branches) is a semimonad
but not a monad.

\subparagraph{Proof}

We have already implemented the \lstinline!flatMap! method for an
equivalent functor in Section~\ref{subsec:Tree-like-semimonads-and-monads}.
The corresponding Scala code for the functor \lstinline!L! and for
its \lstinline!flatten! method is:
\begin{lstlisting}
type L[A] = Either[F[A], F[L[A]]]     // The functor F must be already defined.
def flatten_L[A]: L[L[A]] => L[A] = { // Match on L[L[A]] = Either[F[L[A]], F[L[L[A]]]].
  case Left(fla)     => Right(fla)
  case Right(flla)   => Right(flla.map(x => flatten_L(x)))
}
\end{lstlisting}
Write the matrix notation for the code of \lstinline!flatten! and
for the lifting of \lstinline!flatten! to $L$:
\[
\text{ftn}_{L}^{A}=\,\begin{array}{|c||cc|}
 & F^{A} & F^{L^{A}}\\
\hline F^{L^{A}} & \bbnum 0 & \text{id}\\
F^{L^{L^{A}}} & \bbnum 0 & \overline{\text{ftn}_{L}}^{\uparrow F}
\end{array}\quad,\quad\quad\text{ftn}_{L}^{L^{A}}=\,\begin{array}{|c||cc|}
 & F^{L^{A}} & F^{L^{L^{A}}}\\
\hline F^{L^{L^{A}}} & \bbnum 0 & \text{id}\\
F^{L^{L^{L^{A}}}} & \bbnum 0 & \overline{\text{ftn}_{L}}^{\uparrow F}
\end{array}\quad,
\]
\[
(f^{:A\rightarrow B})^{\uparrow L}=\,\begin{array}{|c||cc|}
 & F^{B} & F^{L^{B}}\\
\hline F^{A} & f^{\uparrow F} & \bbnum 0\\
F^{L^{A}} & \bbnum 0 & \big(f^{\overline{\uparrow L}}\big)^{\uparrow F}
\end{array}\quad,\quad\quad\text{ftn}_{L}^{\uparrow L}=\,\begin{array}{|c||cc|}
 & F^{L^{A}} & F^{L^{L^{A}}}\\
\hline F^{L^{L^{A}}} & \overline{\text{ftn}_{L}}^{\uparrow F} & \bbnum 0\\
F^{L^{L^{L^{A}}}} & \bbnum 0 & \big(\text{ftn}_{L}^{\overline{\uparrow L}}\big)^{\uparrow F}
\end{array}\quad.
\]
To verify the associativity law, write both its parts separately:
\[
\text{ftn}_{L}^{L^{A}}\bef\text{ftn}_{L}=\,\begin{array}{|c||cc|}
 & F^{L^{A}} & F^{L^{L^{A}}}\\
\hline F^{L^{L^{A}}} & \bbnum 0 & \text{id}\\
F^{L^{L^{L^{A}}}} & \bbnum 0 & \overline{\text{ftn}_{L}}^{\uparrow F}
\end{array}\,\bef\,\begin{array}{|c||cc|}
 & F^{A} & F^{L^{A}}\\
\hline F^{L^{A}} & \bbnum 0 & \text{id}\\
F^{L^{L^{A}}} & \bbnum 0 & \overline{\text{ftn}_{L}}^{\uparrow F}
\end{array}\,=\,\begin{array}{|c||cc|}
 & F^{A} & F^{L^{A}}\\
\hline F^{L^{L^{A}}} & \bbnum 0 & \overline{\text{ftn}}^{\uparrow F}\\
F^{L^{L^{L^{A}}}} & \bbnum 0 & \overline{\text{ftn}_{L}}^{\uparrow F}\bef\overline{\text{ftn}_{L}}^{\uparrow F}
\end{array}\quad,
\]
\[
\negthickspace\negthickspace\negthickspace\negthickspace\text{ftn}_{L}^{\uparrow L}\bef\text{ftn}_{L}=\,\begin{array}{|c||cc|}
 & F^{L^{A}} & F^{L^{L^{A}}}\\
\hline F^{L^{L^{A}}} & \overline{\text{ftn}_{L}}^{\uparrow F} & \bbnum 0\\
F^{L^{L^{L^{A}}}} & \bbnum 0 & \big(\text{ftn}_{L}^{\overline{\uparrow L}}\big)^{\uparrow F}
\end{array}\,\bef\,\begin{array}{|c||cc|}
 & F^{A} & F^{L^{A}}\\
\hline F^{L^{A}} & \bbnum 0 & \text{id}\\
F^{L^{L^{A}}} & \bbnum 0 & \overline{\text{ftn}_{L}}^{\uparrow F}
\end{array}\,=\,\begin{array}{|c||cc|}
 & F^{A} & F^{L^{A}}\\
\hline F^{L^{L^{A}}} & \bbnum 0 & \overline{\text{ftn}_{L}}^{\uparrow F}\\
F^{L^{L^{L^{A}}}} & \bbnum 0 & \big(\text{ftn}_{L}^{\overline{\uparrow L}}\big)^{\uparrow F}\bef\overline{\text{ftn}_{L}}^{\uparrow F}
\end{array}\quad.
\]
The two resulting matrices are equal due to the inductive assumption:
$\overline{\text{ftn}}_{L}^{\uparrow F}\bef\overline{\text{ftn}}_{L}^{\uparrow F}=\big(\text{ftn}_{L}^{\overline{\uparrow L}}\big)^{\uparrow F}\bef\overline{\text{ftn}}_{L}^{\uparrow F}$.

Could $L$ be a full monad? Suppose we were given some code for the
\lstinline!pure! method ($\text{pu}_{L}$). If we substitute an arbitrary
value $k^{:L^{A}}$ into the left identity law, we expect to obtain
again $k$:
\[
k^{:L^{A}}\triangleright\text{pu}_{L}\bef\text{ftn}_{L}=k\triangleright\text{pu}_{L}\triangleright\,\begin{array}{|c||cc|}
 & F^{A} & F^{L^{A}}\\
\hline F^{L^{A}} & \bbnum 0 & \text{id}\\
F^{L^{L^{A}}} & \bbnum 0 & \overline{\text{ftn}_{L}}^{\uparrow F}
\end{array}\,\overset{?}{=}k\quad.
\]
We notice that the matrix for $\text{ftn}_{L}$ has no elements in
the $F^{A}$ column. This suggests information loss: the value $k\triangleright\text{pu}_{L}\bef\text{ftn}_{L}$
will be always of the form $\bbnum 0+q^{:F^{L^{A}}}$ with some $q$,
regardless of $k$ and of the implementation of $\text{pu}_{L}$.
So, if we set $k\triangleq p^{:F^{A}}+\bbnum 0$ with some $p$, the
law cannot  hold. The same problem occurs with the right identity
law. We conclude that $L$ cannot be a full monad.

\subsection{Exercises: laws and structure of monads\index{exercises}}

\subsubsection{Exercise \label{subsec:Exercise-semimonad-not-monad}\ref{subsec:Exercise-semimonad-not-monad}}

For fixed types $U$ and $V$, define \lstinline!flatten! for the
semimonad $F^{A}\triangleq A\times U\times V$ as:
\[
\text{ftn}_{F}\triangleq(a^{:A}\times u_{1}^{:U}\times v_{1}^{:V})\times u_{2}^{:U}\times v_{2}^{:V}\rightarrow a\times u_{1}\times v_{2}\quad.
\]
A \lstinline!pure! method could be defined for $F$ if some fixed
values $u_{0}^{:U}$ and $v_{0}^{:V}$ were given:
\[
\text{pu}_{F}\triangleq a^{:A}\rightarrow a\times u_{0}\times v_{0}\quad.
\]
Show that $F$ would still not be a full monad because identity laws
fail to hold.

\subsubsection{Exercise \label{subsec:Exercise-1-monads-2}\ref{subsec:Exercise-1-monads-2}}

Suppose a monad $M$ is defined using the \lstinline!pure! and \lstinline!flatten!
methods that satisfy all laws, in particular, the left identity law
$\text{pu}_{M}\bef\text{ftn}_{M}=\text{id}$. The \lstinline!flatMap!
function is then defined through \lstinline!flatten! by $\text{flm}_{M}(f)\triangleq f^{\uparrow M}\bef\text{ftn}_{M}$.
Show that this \lstinline!flatMap! function will automatically satisfy
the left identity law: $\text{pu}_{M}\bef\text{flm}_{M}(f)=f$ for
any $f^{:A\rightarrow M^{B}}$.

\subsubsection{Exercise \label{subsec:Exercise-1-monads-3}\ref{subsec:Exercise-1-monads-3}}

Verify the associativity law for the \lstinline!Reader! monad by
an explicit derivation.

\subsubsection{Exercise \label{subsec:Exercise-1-monads-3-1}\ref{subsec:Exercise-1-monads-3-1}}

Show that the laws hold for a non-standard \lstinline!List! monad
whose \lstinline!flatten! method is:\index{monads!List monad with empty sub-lists@\texttt{List} monad with empty sub-lists}
\begin{lstlisting}
def flatten[A](p: List[List[A]]): List[A] = if (p.exists(_.isEmpty)) Nil else p.flatten
\end{lstlisting}


\subsubsection{Exercise \label{subsec:Exercise-1-monads-8-1}\ref{subsec:Exercise-1-monads-8-1}{*}}

Define \lstinline!ET2[A] = Option[Tree2[A]]! using \lstinline!Tree2!
from Section~\ref{subsec:Binary-trees}. Implement \emph{two} inequivalent
versions of \lstinline!flatten! for \lstinline!ET2!. Show that the
monad laws hold for both versions. (This is similar to having the
standard and non-standard \lstinline!flatten! methods for the \lstinline!List!
monad.)

\subsubsection{Exercise \label{subsec:Exercise-1-monads-4}\ref{subsec:Exercise-1-monads-4}}

Given an arbitrary monad $M$, show that the functor $F^{A}\triangleq\bbnum 2\times M^{A}$
(which is equivalent to $M^{A}+M^{A}$) is a semimonad but not a full
monad.

\subsubsection{Exercise \label{subsec:Exercise-1-monads-9}\ref{subsec:Exercise-1-monads-9}}

Show that $P^{A}\triangleq Z+W\times A$ is a (full) monad if $W$
is a monoid.

\subsubsection{Exercise \label{subsec:Exercise-1-monads-5}\ref{subsec:Exercise-1-monads-5}}

If $W$ and $R$ are arbitrary fixed types, is any of the functors
$F^{A}\triangleq W\times\left(R\rightarrow A\right)$ and $G^{A}\triangleq R\rightarrow\left(W\times A\right)$
a lawful semimonad? Hint: try to define \lstinline!flatten! for these
functors.

\subsubsection{Exercise \label{subsec:Exercise-1-monads-6}\ref{subsec:Exercise-1-monads-6}}

Show that $F^{A}\triangleq\left(P\rightarrow A\right)+\left(Q\rightarrow A\right)$
is not a semimonad (cannot define \lstinline!flatMap!) when $P$
and $Q$ are fixed types that are different and not related by subtyping.

\subsubsection{Exercise \label{subsec:Exercise-1-monads-7-not-a-monad}\ref{subsec:Exercise-1-monads-7-not-a-monad}{*}}

Consider the functor $D^{A}\triangleq\bbnum 1+A\times A$ (in Scala,
\lstinline!type D[A] = Option[(A, A)]!). Implement the \lstinline!flatten!
and \lstinline!pure! methods for $D$ in at least two different ways.
Show that some of the monad laws fail to hold for those implementations.\footnote{One can prove that $D^{A}\triangleq\bbnum 1+A\times A$ cannot be
a lawful monad; see \texttt{\href{https://stackoverflow.com/questions/49742377/}{https://stackoverflow.com/questions/49742377/}}}

\subsubsection{Exercise \label{subsec:Exercise-1-monads-8}\ref{subsec:Exercise-1-monads-8}}

For the type constructor $F^{A}\triangleq Z\times\left(A\rightarrow Z\right)$,
a programmer defined \lstinline!flatMap! by:
\begin{lstlisting}
type F[A] = (Z, A => Z) // The fixed type Z must be already defined.
def flatMap_F[A, B](fa: F[A])(f: A => F[B]): F[B] = fa match { case (z, g) => (z, _ => z) }
\end{lstlisting}
\[
\text{flm}_{F}(f^{:A\rightarrow Z\times\left(B\rightarrow Z\right)})\triangleq z^{:Z}\times g^{:A\rightarrow Z}\rightarrow z\times(\_^{:B}\rightarrow z)\quad.
\]
(Here $Z$ is a fixed type.) Show that this implementation fails to
satisfy the semimonad\textsf{'}s law. 

\subsubsection{Exercise \label{subsec:Exercise-flatten-concat-distributive-law}\ref{subsec:Exercise-flatten-concat-distributive-law}{*}}

Prove (by induction on $p$) that \lstinline!flatten! and \lstinline!concat!
satisfy the distributive law\index{distributive law!of concat@of \texttt{concat}}:
\[
\left(p\pplus q\right)\triangleright\text{ftn}=\left(p\triangleright\text{ftn}\right)\pplus\left(q\triangleright\text{ftn}\right)\quad\quad\text{for all }p:\text{List}^{\text{List}^{A}},\,q:\text{List}^{\text{List}^{A}}\quad.
\]


\subsubsection{Exercise \label{subsec:Exercise-1-monads-9-1-1}\ref{subsec:Exercise-1-monads-9-1-1}}

Given a monad $M$, consider a function \lstinline!toUnit! (denoted
$\text{tu}_{M}$) with type signature:

\begin{wrapfigure}{l}{0.61\columnwidth}%
\vspace{-0.5\baselineskip}
\begin{lstlisting}
def toUnit[M[_]: Monad, A]: M[A] => M[Unit] = _.map(_ => ())
\end{lstlisting}

\vspace{-0.5\baselineskip}
\end{wrapfigure}%

~\vspace{-0.9\baselineskip}
\[
\text{tu}_{M}:M^{A}\rightarrow M^{\bbnum 1}\triangleq(\_^{:A}\rightarrow1)^{\uparrow M}\quad.
\]
Show that this function satisfies the equation:
\[
\text{pu}_{M}^{:A\rightarrow M^{A}}\bef\text{tu}_{M}^{:M^{A}\rightarrow M^{1}}=\_^{:A}\rightarrow\text{pu}_{M}(1)\quad.
\]


\subsubsection{Exercise \label{subsec:Exercise-monad-of-monoid-is-monoid}\ref{subsec:Exercise-monad-of-monoid-is-monoid}}

Show that \lstinline!M[W]! is a monoid if \lstinline!M[_]! is a
monad and \lstinline!W! is a monoid.

\subsubsection{Exercise \label{subsec:Exercise-1-monads-9-1}\ref{subsec:Exercise-1-monads-9-1}}

A monad is called \textbf{commutative}\index{monads!commutative}
if swapping the order of its effects leaves the result unchanged.
In terms of functor block code, this means that \lstinline!result1!
and \lstinline!result2! are equal:

\noindent \texttt{\textcolor{blue}{\footnotesize{}}}%
\begin{minipage}[c]{0.475\columnwidth}%
\texttt{\textcolor{blue}{\footnotesize{}}}
\begin{lstlisting}
result1 = for { // Assume p: M[A],  q: M[B],
    x <- p    // and q does not depend on x.
    y <- q
} yield f(x, y)   // For any f: (A, B) => C.
\end{lstlisting}
%
\end{minipage}\texttt{\textcolor{blue}{\footnotesize{}\hspace*{\fill}}}%
\begin{minipage}[c]{0.475\columnwidth}%
\texttt{\textcolor{blue}{\footnotesize{}}}
\begin{lstlisting}
result2 = for { // Assume p: M[A],  q: M[B],
    y <- q    // and p does not depend on y.
    x <- p
} yield f(x, y)   // For any f: (A, B) => C.
\end{lstlisting}
%
\end{minipage}{\footnotesize\par}

\noindent \vspace{0\baselineskip}

\textbf{(a)} Find out which of the monads \lstinline!Option!, \lstinline!List!,
\lstinline!Reader!, \lstinline!Writer!, and \lstinline!State! are
commutative.

\textbf{(b)} Consider the type $M^{\bbnum 1}$ (in Scala, this is
\lstinline!M[Unit]!). Since the \lstinline!Unit! type is a (trivial)
monoid, Exercise~\ref{subsec:Exercise-monad-of-monoid-is-monoid}
shows that $M^{\bbnum 1}$ is also a monoid. Prove that if the monad
$M$ is commutative then the binary operation $\oplus_{M}$ of the
monoid $M^{\bbnum 1}$ is also commutative.

\subsubsection{Exercise \label{subsec:Exercise-1-monads-11}\ref{subsec:Exercise-1-monads-11}}

Use monad constructions (so that law checking is unnecessary) to implement
\lstinline!flatten! and \lstinline!pure! for $F^{A}\triangleq A+\left(R\rightarrow A\right)$,
where $R$ is a fixed type.

\subsubsection{Exercise \label{subsec:Exercise-1-monads-12}\ref{subsec:Exercise-1-monads-12}}

Given a monad $G$, consider an alternative way of defining the \lstinline!pure!
method for the functor $L^{A}\triangleq A+G^{A}$ (see Statement~\ref{subsec:Statement-co-product-with-identity-monad}).
Show that the identity laws will \emph{not} hold for $L$ if \lstinline!pure!
is defined as \lstinline!a => Right(Monad[G].pure(a))! (in the code
notation, $\text{pu}_{L}\triangleq a\rightarrow\bbnum 0+a\triangleright\text{pu}_{G}$).

\subsubsection{Exercise \label{subsec:Exercise-1-monads-13}\ref{subsec:Exercise-1-monads-13}}

Implement the functor and monad methods for $F^{A}\triangleq\left(Z\rightarrow\bbnum 1+A\right)\times\text{List}^{A}$
using the known constructions (so that law checking is unnecessary).

\subsubsection{Exercise \label{subsec:Exercise-1-monads-16}\ref{subsec:Exercise-1-monads-16}}

Use monad constructions to show that the functors $F$ defined below
are monads:

\textbf{(a)} $F^{A}\triangleq A+A+...+A$ and $F^{A}\triangleq A\times A\times...\times A$
(with a fixed number of $A$\textsf{'}s).

\textbf{(b)} A polynomial functor $F^{A}\triangleq p(A)$ when $p(x)$
is a polynomial of the form $p(x)=x^{n_{1}}+x^{n_{2}}+...+x^{n_{k}}$
with some positive integers $n_{1}$, ..., $n_{k}$. For example,
$F^{A}\triangleq A\times A+A\times A\times A+A\times A\times A\times A\times A$. 

\textbf{(c)} A polynomial functor $F^{A}\triangleq p(A)$ when $p(x)$
is a polynomial of the form $p(x)=c_{1}x^{n_{1}}+c_{2}x^{n_{2}}+...+c_{k-1}x^{n_{k-1}}+x^{n_{k}}$
with some positive integers $n_{1}$, ..., $n_{k}$, $c_{1}$, ...,
$c_{k-1}$, where the coefficient at the highest power ($x^{n_{k}}$)
\emph{must} be $1$. For example, $F^{A}\triangleq\bbnum 5\times A+\bbnum 2\times A\times A\times A+A\times A\times A\times A$.

\subsubsection{Exercise \label{subsec:Exercise-monad-composition-mm}\ref{subsec:Exercise-monad-composition-mm}{*}}

For a monad $M$, show that $L^{A}\triangleq M^{M^{A}}$ is a semimonad
but not always a monad.

\section{Further developments}

\subsection{Why monads must be covariant functors}

We defined monads as functors with extra properties. It turns out
that monads cannot be contrafunctors, and that the functor laws of
a monad can be derived from the laws of \lstinline!flatMap!.

A monad\textsf{'}s code can be specified in three equivalent ways: via \lstinline!flatMap!
and \lstinline!pure!, via \lstinline!flatten! and \lstinline!pure!,
or via the Kleisli composition and \lstinline!pure!. These methods
have different laws and different uses. The \lstinline!flatMap! method
is used most frequently, while the Kleisli composition is rarely used.
Since:
\[
\text{flm}_{M}(f)=f^{\uparrow M}\bef\text{ftn}_{M}\quad,
\]
we could imagine that \lstinline!flatMap! already has the capability
of lifting a function ($f^{\uparrow M}$). Let us extract that capability
from the code of \lstinline!flatMap!: compose $f^{\uparrow M}$ with
the right identity law:
\[
\text{pu}_{M}^{\uparrow M}\bef\text{ftn}_{M}=\text{id}\quad,\quad\text{ so }\quad f^{\uparrow M}\bef\gunderline{\text{pu}_{M}^{\uparrow M}\bef\text{ftn}_{M}}=f^{\uparrow M}\quad.
\]
The left-hand side equals $\text{flm}_{M}(f\bef\text{pu}_{M})$. We
found a formula that expresses \lstinline!map! through \lstinline!flatMap!:

\begin{wrapfigure}{l}{0.5\columnwidth}%
\vspace{-0.5\baselineskip}
\begin{lstlisting}
p.map(f) == p.flatMap(f andThen M.pure)
\end{lstlisting}

\vspace{-0.5\baselineskip}
\end{wrapfigure}%

~\vspace{-0.1\baselineskip}
\begin{equation}
f^{\uparrow M}=\text{flm}_{M}(f\bef\text{pu}_{M})\quad.\label{eq:express-map-through-flatMap}
\end{equation}

The laws of \lstinline!map! will then follow if we assume that \lstinline!flatMap!
is lawful. The identity law of \lstinline!map!:
\begin{align*}
 & \text{id}^{\uparrow M}=\text{flm}_{M}(\text{id}\bef\text{pu}_{M})=\gunderline{\text{flm}_{M}(\text{pu}_{M})}\\
{\color{greenunder}\text{right identity law of }\text{flm}_{M}:}\quad & =\text{id}\quad.
\end{align*}
The composition law of \lstinline!map!:
\begin{align*}
{\color{greenunder}\text{expect to equal }(f\bef g)^{\uparrow M}:}\quad & f^{\uparrow M}\bef g^{\uparrow M}=\text{flm}_{M}(f\bef\text{pu}_{M})\bef\text{flm}_{M}(g\bef\text{pu}_{M})\\
{\color{greenunder}\text{associativity law of }\text{flm}_{M}:}\quad & =\text{flm}_{M}(f\bef\gunderline{\text{pu}_{M}\bef\text{flm}_{M}}(g\bef\text{pu}_{M}))\\
{\color{greenunder}\text{left identity law of }\text{flm}_{M}:}\quad & =\text{flm}_{M}(f\bef g\bef\text{pu}_{M})=(f\bef g)^{\uparrow M}\quad.
\end{align*}

If a monad $M$ is specified via its Kleisli composition $\diamond_{_{M}}$,
we can first define \lstinline!flatMap! by:
\[
\text{flm}_{M}(f^{:A\rightarrow M^{B}})\triangleq\text{id}^{:M^{A}\rightarrow M^{A}}\diamond_{_{M}}f\quad,
\]
and then derive \lstinline!map! from \lstinline!flatMap! as before.

If a monad\textsf{'}s code is defined via \lstinline!pure! and \lstinline!flatten!,
we cannot derive the \lstinline!map! method. So, we could indeed
say heuristically that \lstinline!flatten! has the functionality
of \lstinline!flatMap! without \lstinline!map!.

In the previous chapter, we have seen that the laws of filterable
functors are straightforwardly adapted to filterable contrafunctors.
Could we do that with monads to obtain \textsf{``}contramonads\textsf{''}? It turns
out that nontrivial \textsf{``}contramonads\index{contramonads}\textsf{''} do not
exist. 

The first problem is defining a method analogous to \lstinline!flatMap!
for contrafunctors. We turn to category theory (Section~\ref{subsec:Motivation-for-using-category-theory})
for guidance. The type signature of a monad\textsf{'}s \lstinline!flatMap!
is that of a lifting,
\[
\text{flm}_{M}:(A\rightarrow M^{B})\rightarrow M^{A}\rightarrow M^{B}\quad,
\]
corresponding to a functor from the $M$-Kleisli category to the $M$-lifted
category. To adapt this construction to a contrafunctor $H^{A}$,
we need to define a lifting called, say, \textsf{``}\lstinline!contraFlatMap!\textsf{''},
with the type:
\[
\text{cflm}_{H}:(A\rightarrow H^{B})\rightarrow H^{B}\rightarrow H^{A}\quad.
\]
Note a curious feature of this type signature: it is covariant in
$B$ (since it contains the contrafunctor $H^{B}$ in contravariant
positions). So, $\text{cflm}_{H}$ does not have the type of a natural
transformation with respect to $B$, unlike all other liftings we
have seen, such as \lstinline!map!, \lstinline!liftOpt!, and \lstinline!flatMap!.
If the code of $\text{cflm}_{H}$ is fully parametric then $\text{cflm}_{H}$
must satisfy a corresponding naturality law. It turns out that the
naturality law gives a constraint so strong that it forces $H$ to
be a \emph{constant} contrafunctor.

To see that, consider the left naturality law of $\text{cflm}_{H}$
(with respect to the type parameter $A$):
\begin{equation}
\text{cflm}_{H}(f^{:A\rightarrow B}\bef g^{:B\rightarrow H^{C}})=\text{cflm}_{H}(g)\bef f^{\downarrow H}\quad.\label{eq:contramonad-left-naturality-law}
\end{equation}
If $H$ were a \textsf{``}contramonad\textsf{''}, it would have a method analogous
to \lstinline!pure! with type signature $A\rightarrow H^{A}$ satisfying
the right identity law:
\[
\text{cflm}_{H}(\text{pu}_{H})=\text{id}\quad.
\]
Substitute $g=\text{pu}_{H}$ into Eq.~(\ref{eq:contramonad-left-naturality-law})
and obtain:
\begin{equation}
\text{cflm}_{H}(f\bef\text{pu}_{H})=\text{id}\bef f^{\downarrow H}=f^{\downarrow H}\quad.\label{eq:contramonad-naturality-derivation1}
\end{equation}
The method \lstinline!pure! for a contrafunctor $H$ is equivalent
to a value $\text{wu}_{H}$ of type $H^{\bbnum 1}$ (Section~\ref{subsec:Pointed-contrafunctors}).
In other words, the naturality law for $\text{pu}_{H}:A\rightarrow H^{A}$
forces it to be a function that ignores its argument and always returns
a fixed value of type $H^{A}$, denoted by $\text{cpu}_{H}$ in Sections~\ref{subsec:Co-pointed-functors}\textendash \ref{subsec:Pointed-contrafunctors}:
\[
\text{pu}_{H}=\_^{:A}\rightarrow\text{cpu}_{H}^{A}=\_^{:A}\rightarrow\text{wu}_{H}\triangleright(\_^{:A}\rightarrow1)^{\downarrow H}\quad.
\]
Since $\text{pu}_{H}$ ignores its argument, we have $f\bef\text{pu}_{H}=\text{pu}_{H}$
for any function $f$. Then Eq.~(\ref{eq:contramonad-naturality-derivation1})
gives:
\[
f^{\downarrow H}=\text{cflm}_{H}(f\bef\text{pu}_{H})=\text{cflm}_{H}(\text{pu}_{H})=\text{id}\quad.
\]
The equation $f^{\downarrow H}=\text{id}$ holds only for constant
contrafunctors $H^{A}\triangleq Z$ (with some fixed type $Z$). Constant
contrafunctors are functors at the same time. We know from Section~\ref{subsec:Structural-analysis-of-monads}
that constant functors $F^{A}=Z$ are monads only when $Z=\bbnum 1$.
So, nontrivial \textsf{``}contramonads\textsf{''} are not possible. 

\subsection{Equivalence of a natural transformation and a \textquotedblleft lifting\textquotedblright\label{subsec:Equivalence-of-a-natural-transformation-and-lifting}}

In this and the previous chapters, we have derived the laws for three
typeclasses: filterable functors, filterable contrafunctors, and monads.
In each case, the laws were formulated in terms of lifting-like functions
(\lstinline!liftOpt!, \lstinline!flatMap!), with type signatures:
\begin{align*}
{\color{greenunder}\text{for a filterable functor }F:}\quad & \text{liftOpt}_{F}:(A\rightarrow\text{Opt}^{B})\rightarrow F^{A}\rightarrow F^{B}\quad,\\
{\color{greenunder}\text{for a filterable contrafunctor }F:}\quad & \text{liftOpt}_{F}:(A\rightarrow\text{Opt}^{B})\rightarrow F^{B}\rightarrow F^{A}\quad,\\
{\color{greenunder}\text{for a monad }F:}\quad & \text{flm}_{F}:(A\rightarrow F^{B})\rightarrow F^{A}\rightarrow F^{B}\quad.
\end{align*}
The laws were equivalently written in terms of natural transformations
(\lstinline!deflate!, \lstinline!inflate!, \lstinline!flatten!):
\begin{align*}
{\color{greenunder}\text{for a filterable functor }F:}\quad & \text{deflate}_{F}:F^{\text{Opt}^{A}}\rightarrow F^{A}\quad,\\
{\color{greenunder}\text{for a filterable contrafunctor }F:}\quad & \text{inflate}_{F}:F^{A}\rightarrow F^{\text{Opt}^{A}}\quad,\\
{\color{greenunder}\text{for a monad }F:}\quad & \text{ftn}_{F}:F^{F^{A}}\rightarrow F^{A}\quad.
\end{align*}
How can the liftings be \emph{equivalent} to their corresponding natural
transformations, which have fewer type parameters and simpler types?
The equations relating these functions have a clear pattern:
\begin{align*}
 & \text{deflate}_{F}=\text{liftOpt}_{F}(\text{id})\quad,\quad\quad\text{liftOpt}_{F}(f)=f^{\uparrow F}\bef\text{deflate}\quad,\\
 & \text{inflate}_{F}=\text{liftOpt}_{F}(\text{id})\quad,\quad\quad\text{liftOpt}_{F}(f)=\text{inflate}\bef f^{\downarrow F}\quad,\\
 & \text{ftn}_{F}=\text{flm}_{F}(\text{id})\quad,\quad\quad\text{flm}_{F}(f)=f^{\uparrow F}\bef\text{ftn}_{F}\quad.
\end{align*}
The pattern is that the code of the lifting-like function contains
both the functionality of \lstinline!map! (lifting the function $f$
to $f^{\uparrow F}$ or $f^{\downarrow F}$) and the functionality
of the corresponding natural transformation. Setting $f$ to an identity
function will cancel the lifting (since $\text{id}^{\uparrow F}=\text{id}$
and $\text{id}^{\downarrow F}=\text{id}$), so that only the functionality
of the natural transformation remains.

We have proved the equivalence of lifting-like functions and their
corresponding natural transformations for filterable functors in Statement~\ref{subsec:Statement-liftOpt-equivalent-to-deflate}
and for monads in Statement~\ref{subsec:Statement-flatten-equivalent-to-flatMap}.
It turns out that these equivalence properties are special cases of
a more general construction. To generalize the type signatures of
\lstinline!deflate! and \lstinline!flatten!, consider arbitrary
functors $F$, $G$, $H$, and a natural transformation \lstinline!tr!
with the type signature:
\[
\text{tr}:F^{G^{A}}\rightarrow H^{A}\quad.
\]
A lifting-like function \lstinline!ftr! corresponding to the natural
transformation \lstinline!tr! is defined by:
\begin{equation}
\text{ftr}:(A\rightarrow G^{B})\rightarrow F^{A}\rightarrow H^{B}\quad,\quad\quad\text{ftr}\,(f^{:A\rightarrow G^{B}})\triangleq f^{\uparrow F}\bef\text{tr}\quad.\label{eq:define-ftr-via-tr}
\end{equation}
For filterable functors, we set $G=\text{Opt}$ and $F=H$ and for
monads, we set additionally $G=F$.

By setting $f=\text{id}$ in Eq.~(\ref{eq:define-ftr-via-tr}), we
find that $\text{tr}=\text{ftr}\,(\text{id})$. When are \lstinline!tr!
and \lstinline!ftr! equivalent?

\subsubsection{Statement \label{subsec:Statement-tr-equivalent-to-ftr}\ref{subsec:Statement-tr-equivalent-to-ftr}}

The natural transformation \lstinline!tr! is equivalent to the \textsf{``}lifting-like\textsf{''}
function \lstinline!ftr! via:

\begin{wrapfigure}{l}{0.3\columnwidth}%
\vspace{-1.9\baselineskip}
\[
\xymatrix{\xyScaleY{0.5pc}\xyScaleX{2.8pc} & F^{G^{B}}\ar[rd]\sp(0.5){\ \text{tr}\ }\\
F^{A}\ar[ru]\sp(0.5){(f^{:A\rightarrow G^{B}})^{\uparrow F}~\ }\ar[rr]\sb(0.5){\text{ftr}\,(f^{:A\rightarrow G^{B}})\,} &  & H^{B}
}
\]
\vspace{-0.6\baselineskip}
\end{wrapfigure}%

~\vspace{-0.6\baselineskip}
\begin{equation}
\text{ftr}\,(f)=f^{\uparrow F}\bef\text{tr}\quad,\quad\quad\text{tr}=\text{ftr}\,(\text{id})\quad,\label{eq:define-tr-via-ftr}
\end{equation}
provided that \lstinline!ftr! obeys the naturality law with respect
to the type parameter $A$:
\begin{equation}
g^{\uparrow F}\bef\text{ftr}\left(f\right)=\text{ftr}\left(g\bef f\right)\quad.\label{eq:ftr-left-naturality-law}
\end{equation}


\subparagraph{Proof}

We need to derive the equivalence between \lstinline!tr! and \lstinline!ftr!
in both directions:

\textbf{(a)} Given a function \lstinline!tr!, we first define \lstinline!ftr!
via Eq.~(\ref{eq:define-ftr-via-tr}) and then define a new function
\lstinline!tr!$^{\prime}$ via Eq.~(\ref{eq:define-tr-via-ftr}).
We then show that \lstinline!tr!$^{\prime}$ equals \lstinline!tr!:
\[
\text{tr}^{\prime}=\text{ftr}\left(\text{id}\right)=\gunderline{\text{id}^{\uparrow F}}\bef\text{tr}=\text{tr}\quad.
\]

If the function \lstinline!ftr! is defined via \lstinline!tr!, the
naturality law~(\ref{eq:ftr-left-naturality-law}) holds automatically:
\[
g^{\uparrow F}\bef\text{ftr}\left(f\right)=\gunderline{g^{\uparrow F}\bef f^{\uparrow F}}\bef\text{tr}=(g\bef f)^{\uparrow F}\bef\text{tr}=\text{ftr}\left(g\bef f\right)\quad.
\]

\textbf{(b)} Given a function \lstinline!ftr! that satisfies the
law~(\ref{eq:ftr-left-naturality-law}), we define \lstinline!tr!
via Eq.~(\ref{eq:define-tr-via-ftr}) and then define a new function
\lstinline!ftr!$^{\prime}$ via Eq.~(\ref{eq:define-ftr-via-tr}).
We then show that \lstinline!ftr!$^{\prime}$ equals \lstinline!ftr!.
For an arbitrary $f^{:A\rightarrow G^{B}}$, write:
\begin{align*}
{\color{greenunder}\text{expect to equal }\text{ftr}\left(f\right):}\quad & \text{ftr}^{\prime}(f)=f^{\uparrow F}\bef\text{tr}=f^{\uparrow F}\bef\text{ftr}\left(\text{id}\right)\\
{\color{greenunder}\text{naturality law~(\ref{eq:ftr-left-naturality-law})}:}\quad & =\text{ftr}\left(f\bef\text{id}\right)=\text{ftr}(f)\quad.
\end{align*}

The full equivalence between \lstinline!tr! and \lstinline!ftr!
would not hold without the naturality law~(\ref{eq:ftr-left-naturality-law}).
To build intuition for this, consider that \lstinline!ftr! has a
more complicated type signature with one more type parameter than
\lstinline!tr!. So, heuristically, there are \textsf{``}many more\textsf{''} possible
functions with the type of \lstinline!ftr!. The imposed naturality
law~(\ref{eq:ftr-left-naturality-law}) constrains the code of \lstinline!ftr!
so much that all possible functions \lstinline!ftr! satisfying Eq.~(\ref{eq:ftr-left-naturality-law})
are in a $1$-to-$1$ correspondence with all possible functions \lstinline!tr!.

$\square$

A similar statement can be proved for contrafunctors $F$, by considering
the type signatures:
\[
\text{tr}:H^{B}\rightarrow F^{G^{B}}\quad,\quad\quad\text{ftr}:(A\rightarrow G^{B})\rightarrow H^{B}\rightarrow F^{A}\quad.
\]

An equivalence between \lstinline!ftr! and \lstinline!tr! also holds
with functors $F$ and the type signatures:
\[
\text{tr}:H^{A}\rightarrow F^{G^{A}}\quad,\quad\quad\text{ftr}:(G^{A}\rightarrow B)\rightarrow H^{A}\rightarrow F^{B}\quad.
\]

By referring to these constructions, we can omit detailed proofs of
equivalence in all such cases.

\subsection{Monads, effects, and runners}

Monads allow us to compose \textsf{``}monadic programs\index{monadic program}\textsf{''}
(i.e., values of type $M^{A}$ for some monad $M$). For list-like
and tree-like monads $M$, a value of type $M^{A}$ represents a collection
of values of type $A$ and is a result of nested iterations. For a
pass/fail monad $M$, a value of type $M^{A}$ holds either a result
of type $A$ or information about a failure. However, for many other
monads $M$, a monadic program $m:M^{A}$ represents a single value
of type $A$ wrapped by an \textsf{``}effectful wrapper\textsf{''} of some sort. In
an application, we usually need to extract that value of type $A$
from the \textsf{``}wrapper\textsf{''} $m$.

To see how this works in general, consider monads of function type
(such as \lstinline!Reader!, \lstinline!State!, and \lstinline!Cont!).
These monads delay their computations until a runner is applied. A
monadic program $m:M^{A}$ is a function whose body somehow wraps
a value of type $A$, and a runner extracts that value. The function-type
monads are often defined as case classes with a single part called
\lstinline!run!, for example:
\begin{lstlisting}
final case class Reader[Z, A](run: Z => A)
final case class State[S, A](run: S => (A, S))
final case class Cont[R, A](run: (A => R) => R)
\end{lstlisting}
In Scala syntax, this makes monadic programs appear to have a method
called \lstinline!run!:
\begin{lstlisting}
val s: State[S, A] = for { ... } yield { ... }   // A monadic program in the State monad.
val init: S = ???                                // An initial state.
val result: A = s.run(init)._1                   // Run the monadic program and extract the result.
\end{lstlisting}
We can package this code into a \lstinline!runner! function:
\begin{lstlisting}
def runner[S, A](init: S): State[S, A] => A = _.run(init)._1
\end{lstlisting}
The runner for the \lstinline!State! monad computes a final result
value of type $A$ by performing all the state updates and other computations
encapsulated by a monadic program (\lstinline!s: State[S, A]!). We
say that a runner \textsf{``}runs the monad\textsf{'}s effects\textsf{''} in order to extract
the result value.

These examples motivate a general definition: A \textbf{runner}\index{monads!runner|textit}
for a monad $M$ is a function of type $M^{A}\rightarrow A$, which
we will denote by $\theta^{A}:M^{A}\rightarrow A$.\index{runner!for monads|textit}

To be useful in practice, a runner $\theta$ must satisfy certain
laws. To motivate those laws, consider how we would use runners with
a monadic program $m:M^{B}$ that is composed from two parts: the
first part is a monadic program $m_{1}:M^{A}$, and the second part
is a function $m_{2}:A\rightarrow M^{B}$ that depends on the result
(of type $A$) of the first monadic program.

\begin{wrapfigure}{l}{0.4\columnwidth}%
\vspace{-0.85\baselineskip}

\begin{lstlisting}
val m = for {  // m == m1.flatMap(m2)
  x <- m1
  y <- m2(x)
} yield y
\end{lstlisting}
\vspace{-0.6\baselineskip}
\end{wrapfigure}%

\noindent The composition of $m_{1}$ and $m_{2}$ can be written
as a functor block or as an application of the \lstinline!flatMap!
method,
\begin{equation}
m=m_{1}\triangleright\text{flm}_{M}(m_{2})\quad.\label{eq:monad-runners-derivation1}
\end{equation}
\vspace{-1.2\baselineskip}

We may imagine that $\theta\left(m\right)$ first runs the effects
of $m_{1}$ obtaining a value $x$, and then runs the effects of $m_{2}(x)$
obtaining a value $y$. So, it is natural to require that a runner
$\theta$ applied to $m$ should give the same results as applying
$\theta$ to $m_{1}$, which extracts a value $x$, and then applying
the runner to $m_{2}(x)$. We can formulate this requirement as a
law called the runner\textsf{'}s \index{composition law!of monad runners}\textbf{composition
law}:

\begin{wrapfigure}{l}{0.4\columnwidth}%
\vspace{-0.85\baselineskip}

\begin{lstlisting}
runner(m) == runner(m2(runner(m1)))
\end{lstlisting}
\vspace{-0.7\baselineskip}
\end{wrapfigure}%

~\vspace{-0.4\baselineskip}
\[
m\triangleright\theta=m_{1}\triangleright\theta\triangleright m_{2}\triangleright\theta\quad.
\]
Substituting $m$ from Eq.~(\ref{eq:monad-runners-derivation1}),
we get $m_{1}\triangleright\text{flm}_{M}(m_{2})\bef\theta=m_{1}\triangleright\theta\bef m_{2}\bef\theta$.
Since this equation must hold for any $m_{1}$, we can simplify it
to:
\[
\text{flm}_{M}(m_{2})\bef\theta=\theta\bef m_{2}\bef\theta\quad.
\]

Let us reformulate the composition law in terms of the monad $M$\textsf{'}s
\lstinline!flatten! method:
\[
m_{2}^{\uparrow M}\bef\text{ftn}_{M}\bef\theta=\theta\bef m_{2}\bef\theta\quad.
\]
We would like to move $m_{2}$ to the left of $\theta$ in the right-hand
side, so that $m_{2}$ could drop out of the equation. To interchange
the order of composition $\theta\bef m_{2}$, we assume a naturality
law\index{naturality law!of monad runners} for $\theta$:
\begin{equation}
f^{\uparrow M}\bef\theta=\theta\bef f\quad.\label{eq:runner-naturality-law}
\end{equation}
It follows that $\theta\bef m_{2}\bef\theta=m_{2}^{\uparrow M}\bef\theta\bef\theta$
and that $\theta\bef\theta=\theta^{\uparrow M}\bef\theta$. The runner\textsf{'}s
composition law becomes:

\begin{wrapfigure}{l}{0.2\columnwidth}%
\vspace{-1.9\baselineskip}
\[
\xymatrix{\xyScaleY{2.0pc}\xyScaleX{2.3pc}M^{M^{A}}\ar[d]\sb(0.5){\text{ftn}_{M}}\ar[r]\sp(0.5){\theta^{\uparrow M}}\sb(0.5){\theta^{M^{A}}} & M^{A}\ar[d]\sb(0.45){\theta}\\
M^{A}\ar[r]\sp(0.5){\theta} & A
}
\]
\vspace{-0\baselineskip}
\end{wrapfigure}%

~\vspace{-0.4\baselineskip}
\begin{equation}
\text{ftn}_{M}\bef\theta=\theta\bef\theta=\theta^{\uparrow M}\bef\theta\quad.\label{eq:runner-composition-law}
\end{equation}

\noindent Another reasonable requirement for runners is that a monadic
value with an \textsf{``}empty effect\textsf{''}, such as \lstinline!pure(x)!, should
be mapped by the runner to just \lstinline!x!:
\[
x\triangleright\text{pu}_{M}\triangleright\theta=x\quad,\quad\text{or equivalently:}\quad\text{pu}_{M}\bef\theta=\text{id}\quad.
\]
This is the \index{identity laws!of monad runners}\textbf{identity
law} of monad runners.

If a runner has fully parametric code (such as the runner for the
\lstinline!State! monad shown above), the naturality law~(\ref{eq:runner-naturality-law})
will hold automatically due to the parametricity theorem. However,
monad runners are not always fully parametric. For example, the runner
for the continuation monad shown in Section~\ref{subsec:The-continuation-monad}
uses Scala\textsf{'}s \lstinline!Future! and \lstinline!Promise! classes,
which involve mutable values. (Nevertheless, the naturality, identity,
and composition laws hold for that runner.)

We have argued that list-like, tree-like, and pass/fail monads do
not need runners; but those monads \emph{cannot} have fully parametric
runners. For instance, a runner for the \lstinline!Option! monad
should be a function with type $\theta^{A}:\bbnum 1+A\rightarrow A$
that must produce a value of type $A$ even if the \lstinline!Option!
value is empty. However, it is impossible to produce a value of an
arbitrary type $A$ from scratch using fully parametric code. For
the \lstinline!Option! monad and other pass/fail monads, we could
only use runners $\theta^{A}$ that work for a specific type $A$,
for example, for $A=$ \lstinline!Int!:
\begin{lstlisting}
def runner: Option[Int] => Int = _.getOrElse(0) // For empty Option values, return a default.
\end{lstlisting}
Even if we restrict all types to \lstinline!Int!, this runner will
fail to obey the composition law:
\begin{lstlisting}
val m1: Option[Int] = None
val m2: Int => Option[Int] = { x => Some(x + 1) }
val m: Option[Int] = for { x <- m1; y <- m2(x) } yield y

scala> runner(m)   // Composition law: runner(m) == runner(m2(runner(m1))).
res0: Int = 0

scala> runner(m2(runner(m1)))
res1: Int = 1
\end{lstlisting}
Similar arguments apply to list-like and tree-like monads. These monads
may have an empty value, which cannot be correctly handled by a runner
$\theta^{A}$ whose result must be of parametric type $A$. 

A solution is to define runners not as functions of type $M^{A}\rightarrow A$
but as functions of type $M^{A}\rightarrow N^{A}$, where $N$ is
another monad (the \textsf{``}target\textsf{''} monad of the runner). For example,
if $M$ is a pass/fail monad, we may choose the target monad as $N^{A}=E+A$,
where a fixed type $E$ represents error information. We can then
define a runner for the \lstinline!Option! monad like this:

\begin{wrapfigure}{l}{0.43\columnwidth}%
\vspace{-0.85\baselineskip}

\begin{lstlisting}
val error: E = ... // Describe the error.
def run[A]: Option[A] => Either[E, A] = {
  case None    => Left(error)
  case Some(a) => Right(a)
}
\end{lstlisting}
\vspace{-2.2\baselineskip}
\end{wrapfigure}%

~\vspace{-0.4\baselineskip}
\[
\theta^{:\bbnum 1+A\rightarrow E+A}\triangleq\begin{array}{|c||cc|}
 & E & A\\
\hline \bbnum 1 & \_\rightarrow\text{error} & \bbnum 0\\
A & \bbnum 0 & \text{id}
\end{array}\quad.
\]
In the next section, we will formulate the laws for runners of type
$M^{A}\rightarrow N^{A}$.

\subsection{Monads in category theory. Monad morphisms\label{subsec:Monads-in-category-theory-monad-morphisms}}

For any monad $M$, one defines a category, called the $M$-\index{Kleisli!category}Kleisli
category where objects are all types (\lstinline!Int!, \lstinline!String!,
etc.) and morphisms between types $A$ and $B$ are Kleisli functions
of type $A\rightarrow M^{B}$. 

One axiom of a category requires us to have an identity morphism $A\rightarrow M^{A}$
for every object $A$; this is the monad $M$\textsf{'}s \lstinline!pure!
method, $\text{pu}_{M}:A\rightarrow M^{A}$. Another axiom is the
associativity of morphism composition operation, which must combine
functions of types $A\rightarrow M^{B}$ and $B\rightarrow M^{C}$
into a function of type $A\rightarrow M^{C}$. The Kleisli composition
$\diamond_{_{M}}$ is precisely that operation, and its associativity
law, $f\diamond_{_{M}}(g\diamond_{_{M}}h)=(f\diamond_{_{M}}g)\,\diamond_{_{M}}h$,
is equivalent to a law of \lstinline!flatMap! (Statements~\ref{subsec:Statement-associativity-law-for-kleisli}\textendash \ref{subsec:Statement-equivalence-kleisli-laws-and-flatMap-laws}).

So, a functor $M$ is a monad if and only if the corresponding $M$-Kleisli
category is lawful. This is an concise way of formulating the monad
laws.

We have seen that, for some monads, proofs of the laws are easier
when written in terms of Kleisli morphisms. It turns out that the
properties of monad runners also have a concise formulation in the
language of categories.

A monad $M$\textsf{'}s \lstinline!pure! method has type $A\rightarrow M^{A}$,
while a runner $\theta_{M}$ has type $M^{A}\rightarrow A$. Since
the type parameter $A$ itself can be viewed as the identity monad
$\text{Id}^{A}\triangleq A$, we can write the types as:
\[
\text{pu}_{M}:\text{Id}^{A}\rightarrow M^{A}\quad,\quad\quad\theta_{M}:M^{A}\rightarrow\text{Id}^{A}\quad.
\]
 These two types can be generalized to a transformation between two
monads $M$ and $N$:
\[
\phi:M^{A}\rightarrow N^{A}\quad.
\]
As we have seen in the previous section, some monads $M$ require
runners of this more general type, rather than of type $M^{A}\rightarrow A$.

Another use case for the runner type $M^{A}\rightarrow N^{A}$ comes
from our code for the continuation monad\textsf{'}s runner. The code first
transforms a value of type \lstinline!Cont[R, A]! into a \lstinline!Future[A]!
and then waits for the \lstinline!Future! to complete. So, \lstinline!Cont!\textsf{'}s
runner can be seen as a composition of two transformations:
\[
\xymatrix{\xyScaleY{0.8pc}\xyScaleX{4.5pc}\text{Cont}^{R,A}\ar[r]\sp(0.5){\theta_{\text{Cont-Future}}} & \text{Future}^{A}\ar[r]\sp(0.6){\theta_{\text{Future}}} & A & \theta_{\text{Cont}}=\theta_{\text{Cont-Future}}\bef\theta_{\text{Future}}\quad.}
\]
The intermediate runner $\theta_{\text{Cont-Future}}$ converts continuation-based
monadic programs into \lstinline!Future!-based ones. That conversion
will be compatible with the way we write and refactor functor blocks
only if we impose laws similar to the laws of runners shown in the
previous section. Lawful runners of type $M^{A}\rightarrow N^{A}$
are called \textsf{``}monad morphisms\textsf{''}. The following definition states
their laws:

\subsubsection{Definition \label{subsec:Definition-monad-morphism}\ref{subsec:Definition-monad-morphism}}

A natural transformation $\phi^{A}:M^{A}\rightarrow N^{A}$, also
denoted by $\phi:M\leadsto N$, is called a \textbf{monad morphism}\index{monad morphism|textit}
between monads $M$ and $N$ if the following two laws hold:\index{identity laws!of monad morphisms}
\begin{align}
{\color{greenunder}\text{identity law of }\phi:}\quad & \text{pu}_{M}\bef\phi=\text{pu}_{N}\quad,\label{eq:monad-morphism-identity-law}\\
{\color{greenunder}\text{composition law of }\phi:}\quad & \text{ftn}_{M}\bef\phi=\phi^{\uparrow M}\bef\phi\bef\text{ftn}_{N}\quad.\label{eq:monad-morphism-composition-law-using-ftn}
\end{align}
\vspace{-1.2\baselineskip}
\[
\xymatrix{\xyScaleY{1.2pc}\xyScaleX{4.8pc}A\ar[d]\sb(0.5){\text{pu}_{M}}\ar[rd]\sp(0.5){\text{pu}_{N}} &  & M^{M^{A}}\ar[d]\sb(0.5){\phi^{\uparrow M}}\ar[r]\sp(0.5){\text{ftn}_{M}} & M^{A}\ar[rd]\sp(0.5){\phi}\\
M^{A}\ar[r]\sp(0.42){\phi} & N^{A} & M^{N^{A}}\ar[r]\sp(0.5){\phi^{N^{A}}} & N^{N^{A}}\ar[r]\sp(0.5){\text{ftn}_{N}} & N^{A}
}
\]

The composition law\index{composition law!of monad morphisms} can
be equivalently expressed using the \lstinline!flatMap! method:
\begin{equation}
\text{flm}_{M}(f^{:A\rightarrow M^{B}})\bef\phi^{:M^{B}\rightarrow N^{B}}=\phi\bef\text{flm}_{N}(f\bef\phi)\quad.\label{eq:monad-morphism-composition-law-using-flatMap}
\end{equation}

In terms of the Kleisli composition operations $\diamond_{_{M}}$
and $\diamond_{_{N}}$, the composition law is:
\[
(f^{:A\rightarrow M^{B}}\bef\phi^{:M^{B}\rightarrow N^{B}})\diamond_{_{N}}(g^{:B\rightarrow M^{C}}\bef\phi^{:M^{C}\rightarrow N^{C}})=(f\diamond_{_{M}}g)\bef\phi^{:M^{C}\rightarrow N^{C}}\quad.
\]
This formulation shows more visually that a monad morphism $\phi:M\leadsto N$
replaces $M$-effects by $N$-effects while preserving the composition
of effectful computations.

The name \textsf{``}monad morphism\textsf{''} is motivated by considering the \emph{category
of monads}.\index{monads!category of}\index{category of monads}
The objects of that category are all the possible monads (type constructors
such as \lstinline!Option!, \lstinline!Try!, etc.). The morphisms
between objects $M$ and $N$ of that category are monad morphisms
$M\leadsto N$ as defined above: natural transformations that preserve
the structure of the monadic operations.

At the same time, a monad morphism $\phi$ can be viewed as a correspondence
$g=f\bef\phi$ between Kleisli functions $f:A\rightarrow M^{B}$ and
$g:A\rightarrow N^{B}$. The laws of monad morphisms guarantee that
the mapping $\phi$ is compatible with the Kleisli composition operations
$\diamond_{_{M}}$ and $\diamond_{_{N}}$. So, it can be seen also
as expressing a (categorical) \emph{functor}\index{functor!in category theory}\index{category theory!functor}
between the $M$-Kleisli and the $N$-Kleisli categories.

Monad morphisms between arbitrary monads will be used in Chapters~\ref{chap:Free-type-constructions}
and~\ref{chap:monad-transformers}. To build up more intuition, let
us look at some examples of monad morphisms.

\subsubsection{Example \label{subsec:Example-monad-morphism-either-option}\ref{subsec:Example-monad-morphism-either-option}\index{solved examples}}

Show that the function $\phi^{A}:Z+A\rightarrow\bbnum 1+A$ defined
below is a monad morphism between the \lstinline!Either! and \lstinline!Option!
monads. The implementation of $\phi$ is:

\begin{wrapfigure}{l}{0.5\columnwidth}%
\vspace{-0.75\baselineskip}

\begin{lstlisting}
def toOption[Z, A]: Either[Z, A] => Option[A] = {
  case Left(z)    => None
  case Right(a)   => Some(a)
}
\end{lstlisting}
\vspace{-1.7\baselineskip}
\end{wrapfigure}%

~\vspace{-0.7\baselineskip}
\[
\phi\triangleq\,\begin{array}{|c||cc|}
 & \bbnum 1 & A\\
\hline Z & \_\rightarrow1 & \bbnum 0\\
A & \bbnum 0 & \text{id}
\end{array}\quad.
\]


\subparagraph{Solution}

To verify the identity law~(\ref{eq:monad-morphism-identity-law}):
\[
\text{pu}_{\text{Either}}\bef\phi=\,\begin{array}{|c||cc|}
 & Z & A\\
\hline A & \bbnum 0 & \text{id}
\end{array}\,\bef\,\begin{array}{|c||cc|}
 & \bbnum 1 & A\\
\hline Z & \_\rightarrow1 & \bbnum 0\\
A & \bbnum 0 & \text{id}
\end{array}\,=\,\begin{array}{|c||cc|}
 & \bbnum 1 & A\\
\hline A & \bbnum 0 & \text{id}
\end{array}\,=\text{pu}_{\text{Opt}}\quad.
\]

To verify the composition law~(\ref{eq:monad-morphism-composition-law-using-ftn}),
show that both sides are equal:
\begin{align*}
{\color{greenunder}\text{left-hand side}:}\quad & \text{ftn}_{\text{Either}}\bef\phi=\,\begin{array}{|c||cc|}
 & Z & A\\
\hline Z & \text{id} & \bbnum 0\\
Z & \text{id} & \bbnum 0\\
A & \bbnum 0 & \text{id}
\end{array}\,\bef\,\begin{array}{|c||cc|}
 & \bbnum 1 & A\\
\hline Z & \_\rightarrow1 & \bbnum 0\\
A & \bbnum 0 & \text{id}
\end{array}\,=\,\begin{array}{|c||cc|}
 & \bbnum 1 & A\\
\hline Z & \_\rightarrow1 & \bbnum 0\\
Z & \_\rightarrow1 & \bbnum 0\\
A & \bbnum 0 & \text{id}
\end{array}\quad,\\
{\color{greenunder}\text{right-hand side}:}\quad & \phi^{\uparrow\text{Either}}\bef\phi\bef\text{ftn}_{\text{Opt}}=\,\begin{array}{|c||ccc|}
 & Z & \bbnum 1 & A\\
\hline Z & \text{id} & \bbnum 0 & \bbnum 0\\
Z & \bbnum 0 & \_\rightarrow1 & \bbnum 0\\
A & \bbnum 0 & \bbnum 0 & \text{id}
\end{array}\,\bef\,\begin{array}{|c||ccc|}
 & \bbnum 1 & \bbnum 1 & A\\
\hline Z & \_\rightarrow1 & \bbnum 0 & \bbnum 0\\
\bbnum 1 & \bbnum 0 & \text{id} & \bbnum 0\\
A & \bbnum 0 & \bbnum 0 & \text{id}
\end{array}\,\bef\,\begin{array}{|c||cc|}
 & \bbnum 1 & A\\
\hline \bbnum 1 & \text{id} & \bbnum 0\\
\bbnum 1 & \text{id} & \bbnum 0\\
A & \bbnum 0 & \text{id}
\end{array}
\end{align*}
\begin{align*}
 & \quad=\,\begin{array}{|c||ccc|}
 & \bbnum 1 & \bbnum 1 & A\\
\hline Z & \_\rightarrow1 & \bbnum 0 & \bbnum 0\\
Z & \bbnum 0 & \_\rightarrow1 & \bbnum 0\\
A & \bbnum 0 & \bbnum 0 & \text{id}
\end{array}\,\bef\,\begin{array}{|c||cc|}
 & \bbnum 1 & A\\
\hline \bbnum 1 & \text{id} & \bbnum 0\\
\bbnum 1 & \text{id} & \bbnum 0\\
A & \bbnum 0 & \text{id}
\end{array}\,=\,\begin{array}{|c||cc|}
 & \bbnum 1 & A\\
\hline Z & \_\rightarrow1 & \bbnum 0\\
Z & \_\rightarrow1 & \bbnum 0\\
A & \bbnum 0 & \text{id}
\end{array}\quad.
\end{align*}

The monad morphism $\phi$ maps the \lstinline!Either!-effect (error
information) to the \lstinline!Option!-effect (absence of value)
by discarding the error information. The laws of the monad morphism
guarantee that composition of \lstinline!Either!-effects is mapped
into composition of \lstinline!Option!-effects. For instance, if
some computations caused an error encapsulated by \lstinline!Either!,
the corresponding value after $\phi$ will be an empty \lstinline!Option!
(i.e., \lstinline!None!). 

The function $\phi$ is so simple that the preservation of effects
appears to be automatic. The next example shows a monad morphism that
translates effects in a nontrivial way.

\subsubsection{Example \label{subsec:monad-morphism-writer-state}\ref{subsec:monad-morphism-writer-state}}

Show that \lstinline!toCont! is a monad morphism between the \lstinline!Writer!
and the \lstinline!Cont! monads:
\begin{lstlisting}
def toCont[W: Monoid, A]: ((A, W)) => (A => W) => W = { case (a, w) => k => k(a) |+| w }
\end{lstlisting}
\[
\phi:A\times W\rightarrow\left(A\rightarrow W\right)\rightarrow W\quad,\quad\quad\phi\triangleq a^{:A}\times w^{:W}\rightarrow k^{:A\rightarrow W}\rightarrow k\left(a\right)\oplus w\quad,
\]
where the fixed type $W$ is a monoid with a binary operation $\oplus$
and empty value $e$. The continuation monad\textsf{'}s callback returns a
value of type $W$ that depends on the output $w$ from the \lstinline!Writer!
monad.

\subparagraph{Solution}

To verify the identity law, use the definitions of \lstinline!pure!
for the \lstinline!Writer! and \lstinline!Cont! monads:
\begin{align*}
{\color{greenunder}\text{expect to equal }\text{pu}_{\text{Cont}}:}\quad & \text{pu}_{\text{Writer}}\bef\phi=(a\rightarrow a\times e)\bef(a\times w\rightarrow k\rightarrow k\left(a\right)\oplus w)\\
{\color{greenunder}\text{compute composition}:}\quad & =a\rightarrow k\rightarrow k\left(a\right)\,\gunderline{\oplus\,e}=a\rightarrow k\rightarrow k\left(a\right)=\text{pu}_{\text{Cont}}\quad.
\end{align*}

To verify the composition law, we need an implementation of \lstinline!flatten!
for the \lstinline!Cont! monad:
\begin{lstlisting}
def flatten[W, A]: ((((A => W) => W) => W) => W) => (A => W) => W = { g => k => g(c => c(k)) }
\end{lstlisting}
\[
\text{ftn}_{\text{Cont}}=g^{:\left(\left(\left(A\rightarrow W\right)\rightarrow W\right)\rightarrow W\right)\rightarrow W}\rightarrow k^{:A\rightarrow W}\rightarrow g\,(c^{:\left(A\rightarrow W\right)\rightarrow W}\rightarrow c\left(k\right))\quad.
\]
Now we show that the two sides of the law~(\ref{eq:monad-morphism-composition-law-using-ftn})
reduce to the same function:
\begin{align*}
{\color{greenunder}\text{left-hand side}:}\quad & \text{ftn}_{\text{Writer}}\bef\phi=(a\times u\times w\rightarrow a\times(u\oplus w))\bef(a\times w\rightarrow k\rightarrow k\left(a\right)\oplus w)\\
 & =a\times u\times w\rightarrow k\rightarrow k\left(a\right)\oplus u\oplus w\quad,\\
{\color{greenunder}\text{right-hand side}:}\quad & \phi^{\uparrow\text{Writer}}\bef\phi^{\text{Cont}^{W,A}}\bef\text{ftn}_{\text{Cont}}\\
 & =\gunderline{(a\times u\times w\rightarrow(k\rightarrow k\left(a\right)\oplus u)\times w)\bef(c\times w\rightarrow h\rightarrow h\left(c\right)\oplus w)}\bef\text{ftn}_{\text{Cont}}\\
{\color{greenunder}\text{compute composition}:}\quad & =(a\times u\times w\rightarrow h\rightarrow h\left(k\rightarrow k\left(a\right)\oplus u\right)\oplus w)\bef\gunderline{\text{ftn}_{\text{Cont}}}\\
 & =(a\times u\times w\rightarrow h\rightarrow h\left(k\rightarrow k\left(a\right)\oplus u\right)\oplus w)\bef\left(g\rightarrow p\rightarrow g(c\rightarrow c\left(p\right))\right)\\
{\color{greenunder}\text{compute composition}:}\quad & =a\times u\times w\rightarrow p\rightarrow\gunderline{(c\rightarrow c\left(p\right))(k\rightarrow k\left(a\right)\oplus u)}\oplus w\\
 & =a\times u\times w\rightarrow p\rightarrow p\left(a\right)\oplus u\oplus w\quad.
\end{align*}
$\square$

We conclude this section by proving some properties of monad morphisms.

\subsubsection{Statement \label{subsec:Statement-flatMap-formulation-of-monad-morphism}\ref{subsec:Statement-flatMap-formulation-of-monad-morphism}}

If a function $\phi^{A}:M^{A}\rightarrow N^{A}$ satisfies Eqs.~(\ref{eq:monad-morphism-identity-law})
and~(\ref{eq:monad-morphism-composition-law-using-flatMap}) then
\textbf{(a)} the function $\phi$ is a natural transformation, and
\textbf{(b)} the function $\phi$ also satisfies Eq.~(\ref{eq:monad-morphism-composition-law-using-ftn}).

\subparagraph{Proof}

\textbf{(a)} The naturality law of $\phi$,
\begin{equation}
(g^{:A\rightarrow B})^{\uparrow M}\bef\phi=\phi\bef g^{\uparrow N}\quad,\label{eq:monad-morphism-naturality-law}
\end{equation}
is derived from Eq.~(\ref{eq:monad-morphism-composition-law-using-flatMap})
by setting $f^{:A\rightarrow M^{B}}\triangleq g^{:A\rightarrow B}\bef\text{pu}_{M}^{:B\rightarrow M^{B}}$:
\begin{align*}
{\color{greenunder}\text{left-hand side of Eq.~(\ref{eq:monad-morphism-composition-law-using-flatMap})}:}\quad & \text{flm}_{M}(f)\bef\phi=\gunderline{\text{flm}_{M}(f}\bef\text{pu}_{M})\bef\phi\\
{\color{greenunder}\text{left naturality of }\text{flm}_{M}:}\quad & \quad=f^{\uparrow M}\bef\gunderline{\text{flm}_{M}(\text{pu}_{M})}\bef\phi\\
{\color{greenunder}\text{right identity law of }\text{flm}_{M}:}\quad & \quad=f^{\uparrow M}\bef\phi\quad,\\
{\color{greenunder}\text{right-hand side of Eq.~(\ref{eq:monad-morphism-composition-law-using-flatMap})}:}\quad & \phi\bef\text{flm}_{N}(f\bef\phi)=\phi\bef\text{flm}_{N}(g\bef\gunderline{\text{pu}_{M}\bef\phi})\\
{\color{greenunder}\text{identity law~(\ref{eq:monad-morphism-identity-law})}:}\quad & \quad=\phi\bef\text{flm}_{N}(g\bef\text{pu}_{N})\\
{\color{greenunder}\text{left naturality of }\text{flm}_{N}:}\quad & \quad=\phi\bef g^{\uparrow N}\bef\gunderline{\text{flm}_{N}(\text{pu}_{N})}\\
{\color{greenunder}\text{right identity law of }\text{flm}_{N}:}\quad & \quad=\phi\bef g^{\uparrow N}\quad.
\end{align*}
The two sides of Eq.~(\ref{eq:monad-morphism-composition-law-using-flatMap})
are equal to the two sides of Eq.~(\ref{eq:monad-morphism-naturality-law}).

\textbf{(b)} Substitute the definitions $\text{flm}_{M}(f)=f^{\uparrow M}\bef\text{ftn}_{M}$
and $\text{flm}_{N}(f)=f^{\uparrow N}\bef\text{ftn}_{N}$ into Eq.~(\ref{eq:monad-morphism-composition-law-using-flatMap}):
\begin{align*}
 & f^{\uparrow M}\bef\text{ftn}_{M}\bef\phi=\gunderline{\phi\bef(f\bef\phi)^{\uparrow N}}\bef\text{ftn}_{N}\quad.\\
{\color{greenunder}\text{use Eq.~(\ref{eq:monad-morphism-naturality-law})}:}\quad & =\gunderline{(f\bef\phi)^{\uparrow M}}\bef\phi\bef\text{ftn}_{N}=f^{\uparrow M}\bef\phi^{\uparrow M}\bef\phi\bef\text{ftn}_{M}\quad.
\end{align*}
This equality holds for any $f$, in particular with $f=\text{id}$,
which directly gives Eq.~(\ref{eq:monad-morphism-composition-law-using-ftn}).

\subsubsection{Statement \label{subsec:Statement-monadic-morphism-composition}\ref{subsec:Statement-monadic-morphism-composition}}

If $L$, $M$, $N$ are monads and $\phi:L\leadsto M$ and $\chi:M\leadsto N$
are monad morphisms then the composition $\phi\bef\chi:L\leadsto N$
is also a monad morphism.

\subparagraph{Proof}

The identity law for $\phi\bef\chi$ is verified by:
\begin{align*}
{\color{greenunder}\text{expect to equal }\text{pu}_{N}:}\quad & \gunderline{\text{pu}_{L}\bef(\phi}\bef\chi)\\
{\color{greenunder}\text{identity law for }\phi:}\quad & =\text{pu}_{M}\bef\chi\\
{\color{greenunder}\text{identity law for }\chi:}\quad & =\text{pu}_{N}\quad.
\end{align*}
The composition law for $\phi\bef\chi$ is verified by:
\begin{align*}
{\color{greenunder}\text{expect to equal }\text{ftn}_{L}\bef\phi\bef\chi:}\quad & (\phi\bef\gunderline{\chi)^{\uparrow L}\bef(\phi}\bef\chi)\bef\text{ftn}_{N}\\
{\color{greenunder}\text{naturality of }\phi:}\quad & =\phi^{\uparrow L}\bef\phi\bef\gunderline{\chi^{\uparrow M}\bef\chi\bef\text{ftn}_{N}}\\
{\color{greenunder}\text{composition law for }\chi:}\quad & =\gunderline{\phi^{\uparrow L}\bef\phi\bef\text{ftn}_{M}}\bef\chi\\
{\color{greenunder}\text{composition law for }\phi:}\quad & =\text{ftn}_{L}\bef\phi\bef\chi\quad.
\end{align*}


\subsubsection{Statement \label{subsec:Statement-pure-M-is-monad-morphism}\ref{subsec:Statement-pure-M-is-monad-morphism}}

For any monad $M$, the method $\text{pu}_{M}:A\rightarrow M^{A}$
is a monad morphism $\text{pu}_{M}:\text{Id}\leadsto M$ between the
identity monad and $M$.

\subparagraph{Proof}

The identity law requires $\text{pu}_{\text{Id}}\bef\text{pu}_{M}=\text{pu}_{M}$.
This holds because $\text{pu}_{\text{Id}}=\text{id}$. The composition
law requires $\text{ftn}_{\text{Id}}\bef\text{pu}_{M}=\text{pu}_{M}^{\uparrow\text{Id}}\bef\text{pu}_{M}\bef\text{ftn}_{M}$.
Since $\text{ftn}_{\text{Id}}=\text{id}$, the left-hand side of the
composition law simplifies to $\text{pu}_{M}$. Transform the right-hand
side:
\begin{align*}
{\color{greenunder}\text{expect to equal }\text{pu}_{M}:}\quad & \gunderline{\text{pu}_{M}^{\uparrow\text{Id}}}\bef\text{pu}_{M}\bef\text{ftn}_{M}\\
{\color{greenunder}\text{lifting to the identity functor}:}\quad & =\text{pu}_{M}\bef\gunderline{\text{pu}_{M}\bef\text{ftn}_{M}}\\
{\color{greenunder}\text{left identity law for }M:}\quad & =\text{pu}_{M}\quad.
\end{align*}


\subsubsection{Exercise \label{subsec:Exercise-fmap-is-not-monadic-morphism}\ref{subsec:Exercise-fmap-is-not-monadic-morphism}\index{exercises}}

Suppose $M$ is a given monad, $Z$ is a fixed type, and a \emph{fixed}
value $m_{0}:M^{Z}$ is given.

\textbf{(a)} Consider the function $f$ defined as:
\[
f:\left(Z\rightarrow A\right)\rightarrow M^{A}\quad,\quad\quad f\,(q^{:Z\rightarrow A})\triangleq m_{0}\triangleright q^{\uparrow M}\quad.
\]
Prove that $f$ is \emph{not} a monad morphism from the \lstinline!Reader!
monad $R^{A}\triangleq Z\rightarrow A$ to the monad $M^{A}$, despite
having the correct type signature.

\textbf{(b)} Under the same assumptions, consider the function $\phi$
defined as:
\[
\phi:(Z\rightarrow M^{A})\rightarrow M^{A}\quad,\quad\quad\phi\,(q^{:Z\rightarrow M^{A}})\triangleq m_{0}\triangleright\text{flm}_{M}(q)\quad.
\]
Show that $\phi$ is \emph{not} a monad morphism from the monad $Q^{A}\triangleq Z\rightarrow M^{A}$
to $M^{A}$.

\subsubsection{Exercise \label{subsec:Exercise-fmap-is-not-monadic-morphism-1}\ref{subsec:Exercise-fmap-is-not-monadic-morphism-1}}

Show that \lstinline!List!\textsf{'}s \lstinline!headOption! method viewed
as a function of type $\forall A.\:\text{List}^{A}\rightarrow\bbnum 1+A$
is a natural transformation but \emph{not} a monad morphism between
the monads \lstinline!List! and \lstinline!Option!.

\subsection{Constructions of polynomial monads\label{subsec:Constructions-of-polynomial-monads}}

\textsf{``}Polynomial monads\textsf{''} are polynomial functors that have lawful monad
methods. The product and co-product constructions allow us to create
polynomial monads via the following operations:
\begin{enumerate}
\item Start with $F^{A}\triangleq Z+W\times A$, which is a monad (semimonad)
when $W$ is a monoid (semigroup).
\item Given a polynomial monad $F^{A}$, create the monad $L^{A}\triangleq A+F^{A}$.
\item Given two polynomial monads $F^{A}$ and $G^{A}$, create the monad
$L^{A}\triangleq F^{A}\times G^{A}$.
\item Given a polynomial monad $F^{A}$, create the monad $L^{A}\triangleq F^{Z+W\times A}$
(see Section~\ref{sec:transformers-linear-monads}).
\end{enumerate}
It is an open question (see Problem~\ref{par:Problem-monads}) that
these are the only constructions available for polynomial monads.
If the conjecture is true, we can create an algorithm that recognizes
whether a given polynomial functor can be made into a monad by suitable
definitions of \lstinline!flatten! and \lstinline!pure!. 

As an example, consider the fact that the polynomial functor $F^{A}\triangleq\bbnum 1+A\times A$
cannot be made into a monad (Exercise~\ref{subsec:Exercise-1-monads-6}).
One can also show that $F^{A}$ cannot be obtained through the monad
constructions listed above. Indeed, the corresponding polynomial $f(x)=1+x^{2}$
does not contain any first powers of $x$. However, all constructions
either start with a polynomial containing $x$, or add $x$, or take
a product of two such polynomials. None of these operations could
cancel first powers of $x$ since all coefficients are types and cannot
be subtracted to zero.

By the same logic, we can conclude that $F^{A}\triangleq\bbnum 1+A\times A\times A$
cannot be obtained through monad constructions. It is likely (although
this book does not have a proof) that $F^{A}\triangleq\bbnum 1+A\times A\times A$,
$F^{A}\triangleq\bbnum 1+A\times A\times A\times A$, and all other
similarly constructed functors are \emph{not} monads.

On the other hand, some polynomial functors can be obtained via the
monad constructions in more than one way. A simple example is:
\[
F^{A}\triangleq A+A\times A\cong A\times(\bbnum 1+A)\quad.
\]
The result is that the functor $F$ has two \emph{inequivalent} monad
instances. The first instance is the free pointed monad (construction
2) on the pair monad $G^{A}\triangleq A\times A$ (construction 3).
The second is the product (construction 3) of the identity monad ($\text{Id}^{A}\triangleq A$)
and the \lstinline!Option! monad ($\text{Opt}^{A}\triangleq\bbnum 1+A$,
construction 1). The two monad instances are inequivalent because,
for instance, they define their \lstinline!pure! method in different
ways:
\[
\text{pu}_{1}\triangleq a^{:A}\rightarrow a+\bbnum 0^{:A\times A}\quad,\quad\quad\text{pu}_{2}\triangleq a^{:A}\rightarrow a\times(\bbnum 0^{:\bbnum 1}+a)\quad.
\]
Both instances are lawful since the constructions preserve the monad
laws.

When we say \textsf{``}a functor $F$ is a monad\textsf{''}, we mean that there exists\emph{
at least one} lawful monad instance for $F$. When working with a
custom type constructor such as $F$, the programmer can first check
whether $F$ is a monad. When $F$ has more than one lawful monad
instance, the programmer will then need to choose the instance suitable
for a given application.

\subsection{Constructions of $M$-filterable functors and contrafunctors\label{subsec:Constructions-of-M-filterables}}

In Chapter~\ref{chap:Filterable-functors}, we have used the \lstinline!Opt!-Kleisli
category (that is, the $M$-Kleisli category with $M$ set to the
\lstinline!Option! monad) to formulate the laws of filterable (contra)functors.
We found that the laws of a filterable (contra)functor $F$ are equivalent
to the requirement that the function \lstinline!liftOpt! be a lawful
(categorical) functor from the \lstinline!Opt!-Kleisli category to
an $F$-lifted category:
\begin{align*}
{\color{greenunder}\text{filterable functor }F:}\quad & \text{liftOpt}_{F}:(A\rightarrow\text{Opt}^{B})\rightarrow F^{A}\rightarrow F^{B}\quad,\\
{\color{greenunder}\text{filterable contrafunctor }F:}\quad & \text{liftOpt}_{F}:(A\rightarrow\text{Opt}^{B})\rightarrow F^{B}\rightarrow F^{A}\quad.
\end{align*}
It is natural to generalize this formulation from the \lstinline!Option!
monad to an arbitrary monad $M$:
\begin{align*}
{\color{greenunder}M\text{-filterable functor }F:}\quad & \text{lift}_{M,F}:(A\rightarrow M^{B})\rightarrow F^{A}\rightarrow F^{B}\quad,\\
{\color{greenunder}M\text{-filterable contrafunctor }F:}\quad & \text{lift}_{M,F}:(A\rightarrow M^{B})\rightarrow F^{B}\rightarrow F^{A}\quad.
\end{align*}
 This gave us the definitions of $M$-filterable functors and contrafunctors.\index{$M$-filterable contrafunctor}\index{$M$-filterable functor}

We have shown some simple examples of $M$-filterable functors in
Statement~\ref{subsec:Statement-examples-of-filterable-contrafunctors}.
Structural analysis can discover other examples of such functors systematically.
As in the case of ordinary filterable functors, it turns out that
we must at the same time analyze $M$-filterable contrafunctors.

In the following constructions, we always assume that $M$ is a fixed,
lawful monad.

We omit the proofs of all following statements because they are fully
analogous to the proofs of filterable functor and contrafunctor constructions
in Sections~\ref{subsec:Constructions-of-filterable-functors} and~\ref{subsec:Constructions-of-filterable-contrafunctors}.
In those proofs, we only used the properties of the functions $\text{liftOpt}_{F}$,
which are fully analogous to the properties of the function $\text{lift}_{M,F}$
except for replacing the \lstinline!Option! monad by the monad $M$
and the operation $\diamond_{_{\text{Opt}}}$ by $\diamond_{_{M}}$.

\paragraph{Type parameters}

A constant functor $F^{A}\triangleq Z$ is $M$-filterable with $\text{lift}_{M,F}(f)=\text{id}$
(Statement~\ref{subsec:Statement-examples-of-filterable-contrafunctors}).
The same statement shows that $F^{A}\triangleq M^{A}\rightarrow Z$
and $F^{A}\triangleq A\rightarrow M^{Z}$ are $M$-filterable contrafunctors.

The monad $M^{A}$ itself is $M$-filterable; $\text{lift}_{M,M}(f)\triangleq\text{flm}_{M}(f)$.

The identity functor is not $M$-filterable except when $M$ is the
identity monad, $M^{A}=\text{Id}^{A}\triangleq A$. (However, with
$M=\text{Id}$, the concept of $M$-filterable functor becomes trivial
because all functors and all contrafunctors are $\text{Id}$-filterable.
So, we will assume that $M\neq\text{Id}$.)

The (contra)functor $F^{A}\triangleq G^{H^{A}}$ is $M$-filterable
when $H$ is $M$-filterable and $G$ is any (contra)functor (Statements~\ref{subsec:Statement-filterable-composition-functors}
and~\ref{subsec:Statement-filterable-contrafunctor-composition}),
where $G$ and $H$ can be functors or contrafunctors independently.

\paragraph{Products}

If $F^{A}$ and $G^{A}$ are $M$-filterable then $L^{A}\triangleq F^{A}\times G^{A}$
is $M$-filterable (Statement~\ref{subsec:Statement-filterable-functor-product}).

\paragraph{Co-products}

If $F^{A}$ and $G^{A}$ are $M$-filterable then $L^{A}\triangleq F^{A}+G^{A}$
is $M$-filterable (Statement~\ref{subsec:Statement-filterable-coproduct}).

\paragraph{Function types}

If $F^{A}$ is an $M$-filterable functor and $G^{A}$ is an $M$-filterable
contrafunctor then $F^{A}\rightarrow G^{A}$ and $G^{A}\rightarrow F^{A}$
are $M$-filterable (contra)functors (Statements~\ref{subsec:Statement-filterable-function-type}
and~\ref{subsec:Statement-function-type-exponential-filterable-contrafunctor}).

The filterable contrafunctor $M^{A}\rightarrow Z$ from Statement~\ref{subsec:Statement-examples-of-filterable-contrafunctors}
is obtained from this construction if we set $F^{A}=M^{A}$ and $G^{A}=Z$;
both $F$ and $G$ are filterable as we have already seen.

\paragraph{Recursive types}

If $S^{A,R}$ is a bifunctor that is $M$-filterable with respect
to $A$, the recursive functor $F^{A}$ defined by the type equation
$F^{A}\triangleq S^{A,F^{A}}$ is $M$-filterable (Statement~\ref{subsec:Statement-filterable-recursive-type-1}).

If $S^{A,R}$ is a profunctor\index{profunctor} contravariant in
$A$ and covariant in $R$, and additionally $S^{\bullet,R}$ is $M$-filterable
(with the type parameter $R$ fixed), then the recursive contrafunctor
$F^{A}\triangleq S^{A,F^{A}}$ is $M$-filterable (see Statement~\ref{subsec:Statement-recursive-filterable-contrafunctor}).

In summary, any recursively exponential-polynomial type expression
$F^{A}$ will be an $M$-filterable (contra)functor if it depends
on a type parameter $A$ only through type expressions $M^{A}$ or
$A\rightarrow M^{Z}$ (where $Z$ is a fixed type). While there may
be other $M$-filterable (contra)functors, the structural analysis
covers a broad class of type expressions of the form:
\[
S^{M^{A},A\rightarrow M^{Z_{1}},...,A\rightarrow M^{Z_{n}}}\quad,
\]
where $S^{A,B_{1},...,B_{n}}$ is a type constructor covariant in
$A$ and contravariant in $B_{1}$, ..., $B_{n}$ (or vice versa),
and $Z_{1}$, ..., $Z_{n}$ are fixed types. 

\begin{comment}
this is part two of chapter seven in part one we have looked at several
examples of Mona\textsf{'}s and we found that generalizing the monad type signature
led to very different types there are different properties of containers
some of them expressed iteration other expressed failures recovery
from flavors evaluation strategies and soon in this part I will talk
in more detail about the laws and structure of these containers of
these types and we will see why is it that flatmap type signature
which is kind of a little strange and bizarre maybe at first sight
gives rise to such a generalization we'll see that the properties
of Mona\textsf{'}s are completely logically the derived from the properties
that the computations must have so let us think back to our examples
of Thunder block programs and let\textsf{'}s for simplicity consider that we
are talking about the container such as list where the functor block
let\textsf{'}s say of this kind expresses iteration over a list so we have
been here for example in nested iteration of some sort what will be
the properties of counter block programs that we expect to have the
main intuition is that when we write a line like this with the left
arrow which is in scala called a generator we expect that in the later
lines the value of x will go over items that are held in the container
see this is our main intuition so in particular we expect that if
we first say that X goes over items in container 1 and then we make
some transformation of that X let\textsf{'}s say using a function f then we
expect and then we continue with that in some other way with some
other generator we expect that the result will be the same as if we
first transformed container 1 and replaced all its items by the transformed
items by the f of X and then continued so so in other words we expect
that this code and whatever follows it should be equivalent to this
code and whatever follows it now if you remember the main intuition
behind how to interpret the generator lines each generator line together
with all the code that follows it defines a new container which would
be a result of some flat map call so let\textsf{'}s write down what that flat
map call is for the left it has count 1 flat map and then X goes to
cone 2 of f of X because Y is just a replacement of f of X on the
right hand side we first apply a map on the cont 1 and then we do
a flat map with y going to constitute of Y so if the code on the left
is to be equal to the code on the right in all situations it means
that we have this code should produce the same result as this code
so that is an equation that we expect flat map to satisfy so flat
map together with map must satisfy this equation the same situation
should happen if we first have some generator and then we perform
the same thing so in this example we first manipulated items and then
we did another generator here we first do some generator and then
we manipulate items it should be the same result that gives a rise
to this law which is that first container flat map of this should
be the same as the first container flat map of all this so that\textsf{'}s
the second law that we expect to hold it is necessary to do these
two laws because the way that the lines are translated into flat maps
is linear so it\textsf{'}s the first line and then the second line and the
third line and so if you have this construction replaced by this after
a generator and that\textsf{'}s a different code then if it were before January
this these things cannot be interchanged especially since this could
depend on X and so X is only available after this line so we could
not possibly put this line after these two and so most in most cases
but you cannot interchange lines in a functor block without changing
the results so that\textsf{'}s why we needed to have these two situations when
the replacement is preceded by a generator and when the replacement
is followed by a generator finally we expect another thing which is
that we expect to be able to refactor programs so a for yield blog
or a functor block as I call it returns a container value and this
container value could be put on the right-hand side of another for
yield blocks generator line we expect that this should not change
the meaning if we in line caught the contents of that four yield block
so here\textsf{'}s an example on the Left we have a free Oh block with three
generator lines on the right we put these two first lines into a four
yield of their own the result of that four yield is a another container
and we put that container in the right hand side of a generator line
and we continue like this so this is I just call this YY for simplicity
yes it\textsf{'}s exactly the same as this Y except the two lines here are
in line here they're hearing a separate for um block so we expect
this to always give the same result as that if that were not true
would be very hard to reason about such programs if you in line things
in programs and have gives you different results that\textsf{'}s a bug usually
it would be very hard to find languages where this happens are broken
and shouldn't be used if they have better choices so now we therefore
require that this law should hold so if you in line things then results
should be the same and if you express this in code then you see here
on the left you have called a flat map of P and then flat map of contour
on the right you have firstly have a con flat map of P which gives
you this and then you get flat map of come to so you see this second
flat map is inside the first flat map as it should be in the filter
block but the in this case it\textsf{'}s not inside as you first do the flat
map is a separate for you block and then we do another regenerating
life so in this way we have these three laws now if we write these
laws in this way that\textsf{'}s in principle sufficient to check those laws
for specific examples I just need to write code and transform code
we would like to be able to reason about these laws in a more concise
and elegant way we like to understand what these laws mean in a different
way so that we can conceptualize them because right now it\textsf{'}s just
some complicated chunk of code should be able to give the same result
as some another complicated chunk of code and it\textsf{'}s not I'm just not
clear what will it mean it\textsf{'}s not easy to understand these laws like
this so we're going to rewrite them in an equivalent way using a different
notation so first of all we introduce notation which is flm which
is kind of flat map with arguments reversed similar to what we did
with f map where we put in a function argument first and then the
monad type argument second whereas the flat map usually as this argument
first in this argument second so for convenience we do that and that
turns out to be much much similar to F map it\textsf{'}s a kind of lifting
so you lift this kind of function into this kind of function we will
exploit very much this property and therefore FLM is a more convenient
type signature for reasoning about the properties over semi monad
I remind you that a semi monad is a monad without the pure method
so semi monads just have flat map and you can define flattened in
terms of flat map but that\textsf{'}s it we do not have the pure method in
semi monads and a full monads additionally must have the pure method
so right now we start with semi monads so in other words we only talk
about flat map and its properties which are summarized here later
we will talk about pure and its properties so in what follows I will
fix the factor s the semi Morland and I will not explicitly put that
s as a type parameter anywhere so we F map will be with respect to
the factor s and F Alam will be with respect to the founder s so let\textsf{'}s
write down these three laws in this notation and if you look at this
I'm just going to translate this code into that notation you'll see
it becomes more concise and then we look at types and we will be able
to reason about it much easier so for example this is a composition
of two functions we first apply f when we applied comes to so this
is a flat map of a composition whereas here it\textsf{'}s a map followed by
flat map so it\textsf{'}s a composition of map and flatmap sir here\textsf{'}s this
law flat map of a composition is equal to a composition of map and
flatmap so this is now I wrote out the types of the functions F and
G and here\textsf{'}s a type diagram for this equation I remind you that the
type diagrams are just a fancy way of writing equations more verbally
with more detail and more visually so here the left-hand side is a
function on the right-hand side of the function and these functions
go from here to here so si is the initial type as C is the final type
and the first function is a composition of F map and f LM and the
second function is a nephilim over composition so if you go from here
to here whether you go under upper route or the lower route you get
the equal results that is the meaning of this diagram in mathematics
this is called a commutative diagram meaning that this path and these
paths can be commuted they can be in touch with no changes in results
I will not call them commutative diagram because it\textsf{'}s to me this is
confusing what is what are we committing all the time it\textsf{'}s just a
type diagram for us that shows us very clearly what the types are
and what the functions are between each pair of types and what are
the intermediate types when we do a composition of functions and there\textsf{'}s
some intermediate result and I remind you also that my notation is
such that this composition goes from left to right so we first apply
this function to some value and then to forget the result this is
this intermediate result of this type and when we applied that function
to that guzol we get the final result so in this way it\textsf{'}s easier to
read the type diagrams so they follow the same order first F map Jennifer
land from Steph map then f LM in many mathematics books this notation
is used in other books it\textsf{'}s used the opposite way where you first
apply the function on the right and let me apply the function at the
left of the composition now I write now I feel that this not convention
is more visual there of course completely equivalent in terms of what
you can compute with those notations and in terms of how easy it is
to compute but this is a little more visual first you do this then
we do that in Scala you have the operation called end then which is
exactly the symbol that I'm using and also you have the operation
called compose on functions which is in the opposite order so it\textsf{'}s
up to you what you want to use in star alright so now I have rewritten
all these three laws in terms of this short notation now this here
for example is flat map followed by map is equal to flat map of a
map sorry of constitute followed by map so it is flat map followed
by map is far or composition or something followed by a new app and
the last one is flat map of something followed by flat map is flat
map followed by flat map so that\textsf{'}s flat my about something called
by flat map is flat map followed by flat map and so these are the
types so all these types go basically from si to SB 2 s C where a
B and C are arbitrary types and then you get the equations by either
going the upper route or going a lower route I also wrote the names
of these laws which are just for illustration purposes and to kind
of give you a way of remembering these laws the first two laws are
naturality so what does it mean naturality well naturality means that
there\textsf{'}s a natural transformation going on somewhere between the two
factors and in terms of equation a naturality law means that you have
F map maybe on the left hand side or on the right hand side and you
pull it out of that side and put it on the other side for example
here F is f is under FL m and here F is before phones are pulled out
the F out of fom but now I have to use EFT map on it after I pull
it out so that\textsf{'}s a typical thing for nationality you have a function
that you pull out and then sometimes we use F map on it after you
pull down sometimes before so here\textsf{'}s another naturality so why is
it naturally an a well it\textsf{'}s because the function f so that pull out
transforms the type a into B and here the function G that I pull out
says forms a Type B into C and so flat map it goes always a to s of
B it has to type Traverse a and B and the naturality should be in
both of these type parameters so flatmap can be seen as a natural
transformation in two ways in both of these parameters and so that\textsf{'}s
why we have two naturality laws the third law is associative 'ti and
it\textsf{'}s not obvious why that is cold like that and we will see that much
easier in later Oregon this tutorial but basically if you just look
at this equation you see this is a kind of a law for composition of
flat maps what happens when you compose two flat maps you can put
one of the flat maps inside it\textsf{'}s the same result so now this is much
better than the previous formulation with code it is much shorter
and you see the types goats always from a satyr as beta SC but still
these laws are kind of complicated so there\textsf{'}s this F map here you
have to remember there\textsf{'}s no F map here this is a bit a bit complicated
so let\textsf{'}s find out if there is a better and shorter formulation of
these laws and remember what we did in the previous chapter when we
talked about the filter rules we found a better formulation in that
we factored out the flat that the F map out of some function we got
an easier function which we called deflate back then so let\textsf{'}s do the
same thing here this functionally called flatten which I denote in
the short notation as f TM and this function is standard in Scala
understand the library is called flatten which is basically flat map
on identity if you consider identity function of type as a to SMA
then you can imagine that SAS is some other type C and so basically
you have a function from C to sa you can do a flat map on it and you
get a function from s ce2 s a and C is s to a so I put that brand
on here and the result is a function from s of s of a to SMA and you
can also define flat map out of flatten by prepending it with with
a map with map so this is a diagram it shows their relationship so
if you have an essay you can s map it with a function it was B you
get an SS B then you flatten it to s B and that\textsf{'}s the same as a flat
map so that\textsf{'}s a well-known equivalence but flat map is basically a
map followed by flatten and that\textsf{'}s a scholar convention for naming
this function sir that flat map is basically flat a flatten that is
applied to a result of a map and this is the type diagram that shows
how that works so that the map from a to s be will replace this a
by s beat the result will be s of s would be and then you flatten
went back to SP so just like we found in the previous chapter on filterable
it turns out that this function flatten has fewer laws than flat map
it has only two laws its type signature is also simpler it has fewer
type parameters and it\textsf{'}s a shorter type signature so it turns out
that this is a easiest way to reason about semi monent laws that is
to consider flatten not to consider flat map to your flat instead
so what are the two laws of flat the first law turns out to be this
which is double F map of a function f and then flatten gives you a
flattened followed by an F map of function f so that\textsf{'}s naturally so
naturality here is much easier it\textsf{'}s just commuting flattened with
a function so here you have that function on the left hand side the
flattened here\textsf{'}s on the right hand side of flatten and they need an
extra F map on that it\textsf{'}s important to have two F maps here on the
one here you kind of just replace this with an arbitrary function
G for example this you cannot replace this with an arbitrary G and
have an F map of G in the red right on the left hand side here that
law does not hold it\textsf{'}s it\textsf{'}s mean it\textsf{'}s incorrect so the type diagram
for this law is like this so you start with s of SMA you do a double
flat map sorry you do a double map double F map of a function f which
goes a to b so then you get a survey survey into SMS of B after the
double map then you flatten that into s B or you directly flatten
first a survey survey into a survey and then you just have a single
AF map of A to B and you get a survey to assume D so those must be
identically equal now just one more comment about notation I'm using
here the short notation where I say for example F map F with a space
F map space F I don't right parenthesis here I do that for functions
of one argument and when when things are short here I don't I say
F map of F map of F because this is not short this is a longer expression
and be harder to read that\textsf{'}s my notation so it\textsf{'}s exactly equivalent
to putting parentheses around this F around this F here it\textsf{'}s shorter
to read this so this is similar to the mathematical notation where
you write cosine of X without parentheses you read cosine X cosine
2x sometimes without parentheses just shorter the same thing the second
law now looks like this F map of flattened followed by flatten is
flatten followed by flapping except that there is first flatten is
a different type parameter as it\textsf{'}s applied to a survey survey so let\textsf{'}s
look at the type diagram for this law both sides of this law applied
to a value of this type which is kind of ridiculous but that\textsf{'}s what
it is it\textsf{'}s a triple application of the factor s and you can flatten
it into a single application and you can flatten it in two ways first
the upper path in this diagram you f map of flatten which means that
you flatten this into si and you f map the result so that you get
flattened as a resume and then you flatten again the second way of
flattening is to pretend that this type is some B so this is just
a service of B you flatten that you get s of B now B is s of a but
you just apply the same code for flatten to a different type parameter
parameter s of a instead of parameter a and then you get again a service
of a and then they flatten it again so the result must be the same
of going up or going down now it\textsf{'}s important that all so that we flatten
twice these two are not going to be equal after the first step only
after the second step they're going to because we'll see that on an
example so why is this called associativity well this is a little
easier to understand now why so look at this triple-s implication
we can flatten it first by flattening the inner pair of s and then
flattening the result or we can flatten it by first fighting the outer
pair of s which is going this way and then flattening the result so
this is like a subjectivity first we do we have three things we can
first group two of them together and then group the result and the
other thing together and that\textsf{'}s two ways of doing that and so in mathematics
and social division law is usually of that kind you have three things
you can pair the first to combine them and you get the result and
you can prepare that with a third one or you pair the last two combine
them get the result and pair with the first one and that if if the
two results are the same regardless of would you pair first that\textsf{'}s
a social tippity law that\textsf{'}s the mathematical intuition so now it\textsf{'}s
a little easier to see why this is called associativity but the equation
for this law does not look like a social ticket it doesn't look like
there are three things that we're appearing together so that still
maybe not great we'll see a different formulation of the law where
it is completely obvious that that\textsf{'}s associative 'ti and it looks
like a socially routine but now we already see that it\textsf{'}s getting there
with this pairing of the functor layers now a little aside here we
found that the functions F alone flat map and flatten are equivalent
does it mean equivalent if you have one of them you can define the
other if you have the other you have can define the first one but
not only that but these definitions are equivalent if you take the
first if you somebody gives you a definition of the first you define
a second one and then you define again the first one through that
second one you should get again the same function that you were given
so that\textsf{'}s full equivalence and we have seen this kind of equivalence
like this in Chapter six when we looked at deflate and F map opt they
were equivalent in a similar way deflate was F map of identity F map
opt was F map followed by the flight it\textsf{'}s exactly the same thing here
with flatten and F and flat map so naturally I asked myself is there
some general pattern where this kind of situation happens in two functions
are equivalent yes there is it better it\textsf{'}s a little difficult to see
maybe right away but there is an obvious pattern in the end so here\textsf{'}s
the pattern suppose you have a natural transformation between two
functors F of G of a and F of a that\textsf{'}s how it must be sorry this is
a complexity here F of G of a goes to F of a that\textsf{'}s the entire complexity
that needs to be understood before you go through this this example
so you assume the two factors F and G and there\textsf{'}s a natural transformation
of this kind so {[}Music{]} TR is the transformation of this kind
now we define F TR which is this type signature some sounds familiar
right it\textsf{'}s not quite so it\textsf{'}s not quite it\textsf{'}s a different filter here
than here so it\textsf{'}s not the flat now but it\textsf{'}s quite similar that\textsf{'}s the
pattern so how do we define this f TR we first do an F map of F so
f is this when we do an F map we get an F so we start with F of a
we do have if map of F we get an F of G of B and then we apply the
transformation TR which goes from F of G of B to F of B and then that\textsf{'}s
how we get F of B now it follows obviously that this TR is f TR of
identity so if you put identity here instead of F then F map of identity
is again identity so it\textsf{'}s identity followed by TR that\textsf{'}s TR so that
kind of thing is immediate what is less obvious is that TR and f TR
are equivalent not just TR can be defined from FDR but FDR is defined
from TR and these two definitions are equivalent here\textsf{'}s the type diagram
we start from F a we do an F map with a function f from A to G B we
get an F G B and we transform that with TR into FB we assume that
this is given this this is a transformation that is available and
the other way is to do F G R of F and that should be the same so that\textsf{'}s
a definition you can see that as a definition of F G are given TR
or a definition of TR even f TR because you can put identity here
and there are two interesting things that follow in this construction
first interesting thing is that there is an automatic law for FDR
that follows from the definition of FDR through TR so the naturality
in a for FDR follows automatically and here\textsf{'}s how it falls with an
F map of G and FDR then you substitute the definition of FDR so then
we get this then you have the F map composition law so you get this
and then this is again a definition of fgr in terms of TR so you give
this so that is a natural T law that pulls out G out of f TR and puts
it in light left-hand side with an F map and this law automatically
follows from the definition of FDR\textsf{'}s root here and that\textsf{'}s why TR has
100 fewer than FDR that\textsf{'}s why we had flattened has two laws and flatmap
has three laws same thing was with deflate and f map opted deflate
has fewer laws one fewer laws then F my pooped for this reason because
one law automatically follows from the definition and the second funny
thing that follows is that they're always accruing these functions
they don't this proof we can do a proof of their equivalents and the
proof is for any F and G so this will be the same proof for deflate
and I've mapped as for F a lemon of T M I believe in Chapter six I
did not go through this proof I just told you that deflate can be
defined from a flap opt and asthma pooped can be defined from the
fly but I did not prove that these definitions are equivalent and
it could be that they are not equivalent without proof we don't know
that and the way that they couldn't be not equivalent is that somebody
gives you a t flight you define a left may opt out of it then you
define a deflate out of sorry sorry it\textsf{'}s here somebody gives you a
deflate you define define f map opt out of it and then you use that
F map opt define another D flight here and that second deflate could
be different from the first one and if that were so these definitions
are not equivalent would be not equal so this is not so these definitions
are always equal so how do you do that well the equivalents must be
demonstrated into both directions so in one direction is obvious because
it\textsf{'}s just identity you substitute identity and that gives you the
same function back in the other direction is less obvious you start
with an arbitrary FTR that already satisfies this law the naturality
in a look at the type signature and fti it has two type parameters
a and B so it has naturally low in a and that relative low and B so
what happens when you first transform a that\textsf{'}s not reality in a what
happens when you transform B that\textsf{'}s not relevant B so you have to
assume that you're given some FTR with this type signature that already
satisfies the naturality in a if that so you can define TR of it by
substituting an identity and then you define again another of TR by
using that TR you just defined so you want to verify that that FG
r is equal to your previous one that was given to you here how do
you fara Phi this well you take F map F followed by TR substitute
the definition of TR then you have your natural it in low right here
what you use you get f TR of G followed by evidence and that\textsf{'}s FD
R of F followed by identity identity disappears even FD R of F so
that\textsf{'}s why you're very that\textsf{'}s how you verify the squiggles so we have
shown at once with one proof we have shown equivalence of deflating
as my popped and equivalence of flatten and flat map because they're
just particular case of the same construction with different F and
G if you look at the type signatures then it\textsf{'}s clear clearly self
now let\textsf{'}s actually derive the laws for flat I have shown you the laws
I have not derived them showing you these two laws per flat and I
have not derived them yet so I will derive them now to make the derivation
quicker I will have this notation instead of F map I'll put an up
arrow now the up arrow reminds you that it\textsf{'}s lifted into the functor
so instead of Q a function of A to B you have a lifted Q which is
a function from s a to s B so using this notation I'm just going to
write shorter acquaintance other than that it\textsf{'}s just F map and same
properties flat map is defined like this let\textsf{'}s substitute that into
the three laws of flat map so the first law of flat map is like this
second was like mysteries like this now I'm not going to write any
types in these equations because we know that the types match and
everything we substitute has matching types so we don't need to check
that every time the types match so for example here I was writing
these equations I wrote types in certain places so f is it to be for
example I wrote types in full in these diagrams so once we have verified
that the types match we don't need to keep writing these types we
know they match so f is a to B let\textsf{'}s just not right a to be here anymore
f is a to B G cannot be just B to C because it\textsf{'}s under flat map so
G must be some b-2s C right so where is this law here G must be of
type B 2 SC otherwise flat map doesn't have the right type of its
argument so that is check to check this once we don't have to keep
writing these types and it will be just shorter if we don't we believe
now that types are correct initially and if they're correct initially
whatever we substitute the types are continuing going to continue
to be correct and so that\textsf{'}s just going to save us time reading equations
but in principle you should understand that these are specific types
of example F here must be a to B and G here must be of type B going
to SC otherwise it just doesn't work and similarly here so here this
is lifted G so this is some s B 2 SC already and because of that F
must be going to SB from something from a let\textsf{'}s say a to SB so all
these are implicitly the same as here and so I'm not going to repeat
the types ok first law we take this we substitute a definition of
F L M into both sides on the left it will be FG lifted followed by
flatten on the right will be F lifted G lifted followed by flatten
clearly this is always holding because of lifting is an F map and
that preserves function composition second law substitute the definition
of f LM and we have this so now if you think about the functional
composition here then the lifting which is F map will preserve function
compositions are all being F lifted followed by G double lifted and
there was enough lifted on the left here as well so we can get rid
of this F lifted because the SLO should hold for any F so we could
for example substitute F equals identity into both sides and I will
just F will just disappear I've lived in his disappearance the result
will be this G double lifted followed by flatten is flat and followed
by G lifted so that is the naturality law for flatten which we had
here F method of G f SS 'td : back flatten is flat and followed by
s waisted so that\textsf{'}s naturality so the first law was holding automatically
that\textsf{'}s the same thing that we found in general construction one fewer
laws for fun the third law now the associativity law again we substitute
the definition and we get this so flatten is this lifted followed
by flat so this would be this F lifted G double lifted flatten lifted
followed by slide on the right hand side will be F lifted flattened
G lifted flattened so again we we find we can use the neutrality here
so flatten followed by G lifted is here we replace it by this and
so we get F lifted G double lift it flatten flatten and flip the G
double if that is on the left it\textsf{'}s a common factor we can just omit
it or substitute both F and G identity for simplicity but it\textsf{'}s clear
why we can do this it\textsf{'}s just a common factor on the two sides of the
equation and the result will be flattened lifted followed by flattened
equals flatten followed by flatten so that\textsf{'}s the associativity law
so that\textsf{'}s how we can derive this law and because of this general construction
are explained here once you start with flatten in the define flat
map then the extra law will be holding automatically so in it\textsf{'}s very
similar way we're also going to we can also derive the laws backs
if you assume that somebody gives you a flatten that satisfies these
two laws then we can derive the laws for fom which is basically the
same calculation except you see here the F and G are arbitrating so
you can have to start from here and go back to this in the same way
these are all equations and they are equal in both both directions
we have been careful and we do not lose generality so in this way
I have shown that flatten laws are equivalent to flat map laws but
flatten has a simpler type signature and the fewest laws so when we
check laws for monads and semicolons I will use flatten laws rather
than flat map it\textsf{'}s quicker even though flatten has this complicated
Esteves of s of any type in its laws but even that complication is
offset by the simplicity in in other places and there are fewer laws
naturality is usually easy to check and the reason is that if the
code of the function is pure it has no side effects and it is fully
parametric so that it has no specific reference to a type other than
the tag parameter so their only arguments that are type parameters
and the only operations we use are those that are compatible with
arbitrary types as type parameters if so it said what I call it fully
metric code and then there is a periodicity theorem which says that
if you have a function of this type with a tag parameter a and F and
G being factors then this function code if this functions code is
fully parametric and pure then this function implements of natural
transformation what\textsf{'}s the theorem I'm not going to prove that here
but that\textsf{'}s something we will use for basically not checking any naturality
if it\textsf{'}s obvious that the functions code is fully parametric has no
side effects and does not refer to any specific type so for example
doesn't match on type rather a being integer and then does something
special none of that is permitted in fully parametric code checking
associativity means a lot more work for monads it\textsf{'}s a complicated
law and that\textsf{'}s not easy to check so I will show in detail how to do
that on a number of examples as a as a remark so I've been talking
about silly monads the catch library has a flat map typeclass which
has a flatten method defined by a flat map but that type was in in
the cache library also has another method called tail rec M which
is the recursive modown method and that method is out of place at
this point it\textsf{'}s it\textsf{'}s different more complicated method and not all
walnuts have that and I'm not going to use the flat map typeclass
from the cats library because of this but I can't define it without
defining this extra method that\textsf{'}s really out of place I believe that
the scholars new library has also type glass like this with no such
extra methods so good of you scholars the scholars in star classes
but actually I will just define my own standing water plant class
it\textsf{'}s just not hard and not a lot of work so now let\textsf{'}s go and check
the code to see how we verify of that laws hold for the standard walnut
so go through the list of standard units will implement the flatten
for each of these the code implementing flat o is going to be fully
parametric type parameters so there\textsf{'}s only one type parameter in the
flatten type signature its SOS so very going to isolate and so we're
not going to check naturality it\textsf{'}s it\textsf{'}s going to be automatic but
the social DVD has to be checked so after we check all this I will
show you why certain examples are not fully correct they're incorrect
implementations of flatten and that\textsf{'}s that would be useful for you
to understand that these laws actually are not arbitrate they express
what it means for Lunada to do the computation we wanted to do to
remind zero started all the way from what we want these programs to
be like and these programs need to have certain properties if they
don't have these properties which can happen if we don't implement
the functions correctly mr. in cases then the programs written using
those types will have very difficult to find bugs and that\textsf{'}s a very
bad situation that we can avoid so let\textsf{'}s go into the code now we start
with the option bow nod and the option monad has the flatten function
so I'm just going to be writing out Scala code for all of this this
is an obvious implementation of flat if the option is empty we have
to return empty there\textsf{'}s nothing else for us to return if it\textsf{'}s not
empty then there\textsf{'}s an option inside we return that optional as there
is not we also need a functor instance for this because we are going
to use F map to check the law so the functor instance of course Scala
library has flattened and map defined already on the option type but
I want to write out this code explicitly so that we can check the
law explicitly so there\textsf{'}s this code if it\textsf{'}s not then it\textsf{'}s not if it\textsf{'}s
something we substitute the function instead of the value and curl
F optional is an action of all witnesses of type a so I should not
remain this into a perhaps clarity alright so now that we have this
let\textsf{'}s start the verifying the law how do we verify the law the law
here\textsf{'}s well morality we don't need to verify we verified this law
associativity to verify this law we need to compute the left-hand
side and the right-hand side and we need to compute them symbolically
in other words we write code for the function that computes this we
write code for the punch it appears that compare these two pieces
of code and show that they are identical code so this is not right
running a test with numbers sorry you know numerical check or arbitrary
strings random strangers anything like this this is actual symbolic
proof that these are identical functions symbolically and in order
to go through that proof we need to compute for example this the F
map of flatten as symbolic code then we will compute the as map of
flattened followed by flatten again as symbolic codes what\textsf{'}s go and
see how that works so first let\textsf{'}s compute F map of flatten so we have
asked my up here we can flatten here let\textsf{'}s combine them compose these
functions so how do we do that well we say we first write the code
of F map which is this let me write that in a car in a comment perhaps
so that it\textsf{'}s easy to see why that is like that so first I start with
this code this is a code of ethnic now instead of a function f I needed
to put FTM now what does f TM f TM is this code is so f TN of a is
a match of this so that is the code that I'd see here so that\textsf{'}s how
it is that\textsf{'}s how it was the same code except I write X here and sort
of a so all right so now that\textsf{'}s less less less code so now let\textsf{'}s compute
this thing which is the right-hand side of the associativity law how
do we compute that we write the code of FGM applied to the type parameter
F of 8 well that is not going to change the code of the code the vestian
is generic it works for any age so we don't need to change the code
before we need to change is to change the type parameter so this is
going to be just instead of optional a this is going to be optional
optional a and so there\textsf{'}s going to be some different type the code
remains the same we can put X instead of oh it\textsf{'}s just the same code
okay so this is this is flattened now we need to apply another flattened
to the result of this now the result of this has two pieces there
is this result and there is this result so let\textsf{'}s apply this a flattened
to each of the cases so this is going to be flattened of none this
is going to be flattened of X so now we need to substitute the definition
of flatten into here see what we're doing here is we pretend that
we are the compiler and we symbolically write code that the program
are specified by in lining functions we're just substituting definitions
of functions where they are used so flatten is like this and so none
goes to none and some of away he goes - away therefore none goes to
none so this is none FTN of none is none an FTM of X is so let\textsf{'}s call
as X so this is going to be X match and then this code so that\textsf{'}s why
ft and of X is this so that\textsf{'}s why we write that code so now we have
FGM followed by left in the same way we compute ft/s map of FTM followed
by f teen where none is still none and then I still have this we have
this code so now we compare the code for this and the code for this
and we see it\textsf{'}s exactly the same code the types are the same but leave
the types must be the same because that\textsf{'}s one team but the code is
actually same if we were named variables you know I renamed some variables
X instead of Kawai whatever that doesn't change the code so after
in naming variables we have exactly the same code therefore the law
codes so that\textsf{'}s how we check the law for the option not well so far
we checked only semi mana so we check the associativity the next example
is either gonna which is defined like this is some type Z which is
fixed and the type A and the flatten has this type signature so in
a Scala syntax flatten would have this type signature either of Z
either of Z equal to either now we could do exactly the same thing
we will will write down flatten write down F map then compute symbolically
a flap of flatten by substituting in computing symbolically this substituting
in computing that and compare the code we got exactly the same code
but now you're free to look at this computation and follow it in an
example code but actually for either there is an easier way out you
can check associativity with very little work and the reason is that
the type signature of F map and F we have slather followed by flower
and of this the type signatures actually misses one mistake it\textsf{'}s miss
Z plus Z plus Z was going to zero said right so Z plus a is just my
short type notation for either of Z a now it turns out if you look
at the curly Howard correspondence and try to give derive the implementation
of a function of this type it turns out that this type signature only
has one implementation it has only one implementation which which
is because either you have a Z in one of these positions in one of
these parts of the disjunctions or you have a name have a Z there\textsf{'}s
only one thing you can return you can we must return the left part
of the disjunction with a Z until heavenly you must return the right
part of the disjunction with a knife there\textsf{'}s nothing you can do other
than that so there\textsf{'}s only one implementation of a function of this
type as long as you use peer functions that are fully parametric of
course and that\textsf{'}s where curry how it responds is valid only for those
functions so therefore there must be exactly the same code for this
function and for this function we don't have to prove that they are
equivalent there\textsf{'}s only one way to implement anything here so this
one that can be implemented completely automatically from the type
signature and no law needs to be checked in terms of associativity
law doesn't need to be checked because there\textsf{'}s only one implementation
so that\textsf{'}s a shortcut if you don't want to take this shortcut look
at this code is done exactly the same way as we did for the option
so I will not go into this huge detail the next example is the list
monad now the list monad has a kind of a more difficult definition
because it concatenates lists so we know in the Scala standard library
the list flatten method is defined which just in cabinets nested lists
into one nested list so it works like this you have you have a list
of lists like this so all the nested lists are just concatenated together
and one flat list is returned that\textsf{'}s how flatten works so let\textsf{'}s now
show symbolically that the flatten defined in this way satisfies associative
so we're just going to to do that so how did you probably show that
I'm not going to write code it\textsf{'}s possible certainly to prove this
using code but it\textsf{'}s much more cumbersome and it doesn't really give
us a lot of new insight gives us it\textsf{'}s obvious enough how it works
so here it is so f map of flattened would take a list of lists of
lists of it so here\textsf{'}s a list of lists of wisdom I have X 1 1 X 1 2
1 1 1 1 y 1 2 and so on so this is the first nested thing this is
the second mr.thang another more maybe of those and what it does is
that it flattens the inner ones because we're were lifting the flatten
which means that the outer layer remains the same but we're operating
on the inner layer so we're gluing together these and the result would
be a list like this and then when we flatten that we get the same
result as when we first flattened the outer layers and then flatten
the inner layers so that is obvious because flattening basically says
if you have a nested list of any depths you don't care about the depths
you just erase all the brackets in between and just erase all of this
and you get one big list with all the elements that you ever have
in this order together and of course this doesn't depend on the order
in which you erase brackets and that\textsf{'}s why associativity holds so
we can first flatten the inner two nested lists or we can flatten
first the other two nested lists are going to be the same thing so
flattened as applied to the type parameter list way means that the
inner list of a remains untouched we just flatten the outer layers
so we have as a result a list a flat list of all the inner lists the
result is going to be the same so here I have some numerical tests
to illustrate this so I made this list of lists of lists and if I
flattened it first like this I get this and I flatten flatten I get
this if I first a map flatten then I have this list which is first
concatenated in earlier and then I can catenate it again so here I
first and get immediately outer layers and the inner layers inner
list remained unchanged so this illustrates how the list monad works
the next example is the writer movement now the writer monad is this
type or W must be something good so let me say that explicitly that
W must be a semigroup and then flatten is defined like this so we
have to pull a double on a tuple W and either these are the two WS
and we just combine them using the semigroup operation so this is
the second group operation and I'm using Katz syntax with seven OOP
so let\textsf{'}s check that the laws of a social Ava\textsf{'}s associativity works
here we will not be able to use the correspondence here because this
does not follow from the types the second group operation can be arbitrary
it doesn't follow from the types must be given so F map is obvious
we just don't touch W we transform the first element of the tuple
so we compute them flattened of flatten symbolically and that is we
need to first flatten so that first flattening will give us combining
W 2 and W 3 and the inner one untouched and then we can bind this
and the result will be that so now it\textsf{'}s the other way around first
we do the F metal flatten which will combine the inner ones and flattening
that will combine the outer ones after one with this result so we
see the code is exactly the same except for the difference in the
order in which we apply the semi group operation and so if the semi
group operation is associative which it must be if it\textsf{'}s a lawful semi
group then the code is identical it will give you identical results
so assuming that the law of serogroup holds which is a subjectivity
of a single group operation we can see that the writer Malad is associative
and the next example is the reader moment for the reader model we
can use the curl our trick in fact did I check that there\textsf{'}s only one
implementation in the easier movement I did so I use the curly Howard
library here which has this function so this is a great hard library
and the function is any of type which gives you a list of all implementations
of a given type and then I check the links of that list so if you
look at the type of this function it\textsf{'}s a sequence of functions of
this type so this is a special API that allows you to check that actually
how many implementations exist or given type and the test assert so
there\textsf{'}s only one implementation so that\textsf{'}s what we can do with a curly
Howard library automatically check how many implementations there
exist and that\textsf{'}s the same thing here so the flatten signature is like
this for the reader wouldn't and the signature for the war for the
associativity law is read a really reader of a which is this going
to read their obeyed and again there\textsf{'}s only one implementation and
so therefore it\textsf{'}s not necessary to check my hand any laws but I show
nevertheless nevertheless how to check it\textsf{'}s a little instructive so
I'm an exercise in substituting functions until arguments or you have
higher order functions as arguments so this is a bit complicated but
let me just give you this if you want to go through it you can follow
this derivation is there anything is commented let me just show you
the beginning steps so the flattened function is defined in the obvious
way basically you have a result and you need to return this function
so then you're right this function which you return it takes an argument
R and it must return a result of type a and the result of type a you
can only get by substituting r twice into the function of this type
which i called r ra to make it more visual what that type is and similarly
the f map implementation is automatic there\textsf{'}s only one way to implementing
it you must take an R because that\textsf{'}s the function you have to return
and you must return a.b the only way to return a B is to apply F to
some a the only way to get an ace to apply r8 or some art that\textsf{'}s the
only are you have and so you apply our a to that are and you apply
F to that and then we compute for example this symbolically how do
we compute that while we take FTL which is this function are going
to this except you have this now as an argument as a different type
of the code is the same it\textsf{'}s generic code so the code doesn't change
I just change the name of the variable here for more visual reference
and then you apply FTM to that so when you apply FTM to that you take
the code of FTM and substitute that function instead of this and instead
of argument of ft n so when you do that you have are going to this
function applied to R and again to R so this function applied the
first one to R will give you this and then you again apply this to
heart so let me get 3 R\textsf{'}s and in the same way you compute so you substitute
so the only the only problem here is that you need to understand what
it means to substitute a function in as an argument and the function
is given like this so you want to compute FTM of this function which
is given by an expression so you need to substitute instead of RRA
here intersubjective this expression so you do that step by step the
first application ar-ar-ar the first part is what will give this argument
art here so then you substitute the body of this function which is
our RA of our of our into our RA of our so ok let me show you that
perhaps on an example so this is FTM now instead of rrn you need to
use this this is the argument of FGM now so first we apply art to
it so you apply it to R which will give you the first argument you
need and so that means this goes away and this goes away and that\textsf{'}s
it so that\textsf{'}s how it works not to get any result so in the same way
you follow the other derivations and I'm going to skip them in the
interest of time they're straightforward and you always get this AR
AR AR AR AR AR AR as a result the next example is the statement the
statement has some deep connections with category theory which are
beyond the scope of this tutorial because for all those elegant mathematical
connections I haven't seen much or at all that any code can be written
because of knowledge of those theoretical connections so for this
reason this lack of practical application I will not talk about this
very much and also I haven't studied it extremely deeply but nevertheless
it\textsf{'}s important to understand the statement is quite special and so
in particular it well this is the type of the statement it\textsf{'}s not obvious
that this type does what it should in the part one from this tutorial
I have given some intuition behind choosing this type but nevertheless
this type is not so easy to understand so unfortunately the curry
Howard method does not work for this because there are several implementations
of the type signature of flatten and so there\textsf{'}s no way to argue that
since there is only one implementation then the law must hold so let\textsf{'}s
check the law explicitly the associativity for the state model the
flatten that we defined for the state model has this type signature
and it\textsf{'}s defined by returning this value which is a function and so
your return function it takes the value of type s and then it should
this function should return a tuple of a and s and what tuple does
it return well the only way to return anything that contains a so
to use this function somehow so we call this function on this value
of s it doesn't seem to be much else we can do the result of colony\textsf{'}s
function is a tuple of two values one is this which I denoted here
in SAS and the other is another value of S which I call this one so
now we have SAS which is this and we have an S one which is of type
s we're supposed to produce a tuple of a yes now we could call this
function on an value of s to produce a tuple and that\textsf{'}s what we do
we call this function on s1 now this is quite important that we call
it on s1 not an S as we could do all kinds of things we could call
this function on s instead of this one or we could call this function
and take just the a out of it it returns a tuple so we just take the
first part of the tuple in the second part we could substitute again
either by the S or by this one so there are different implementations
possible I just outlined four different implementations of the function
flatten or the same type signature and there are probably more implementations
so for this reason the type signature alone is not enough to fix the
implementation and it\textsf{'}s far from obvious that we need to do it like
this that this s 1 must be here but this is the intuition that we
have is that the state monad should update a state value and so each
time you call this function it can give you a new value of the state
s and you should use a new value henceforth so the intuition behind
this implementation is that once you have used the old value you get
a new state well you shouldn't use the old one anymore so whatever
you do you should use a new one here after this step you get a value
of type PS so that\textsf{'}s again a new state you should use that you should
return the new state shouldn't return the old states that you have
have been used up so that\textsf{'}s the intuition behind this implementation
but of course this intuition is not sufficient to show that this is
the correct implementation of the state monad so that\textsf{'}s what we'll
have to do now so in order to demonstrate the associativity law we
need to implement F map so I simplement it it takes an essayist and
your turns in SBS so here is what we need to do again here we return
the new state and not the old state we could presumably here return
the old state but that would not be correct now let\textsf{'}s compute the
composition of flattened and flattened so here\textsf{'}s what we do well we
need to confront this triple layering of the state model I'm gonna
type a which I denote it like this so first we apply flat into it
so I'm just pasting the code for flatten which is this and substituting
this thing in it and then you have to apply flatten to the result
now how do you apply flatten to the result or you substitute the definition
of wanton flatten of something is equal to this and then instead of
this application we put the previous code the code that was here applied
to the value s so that is now how we get the code of flatten of slapping
it remains to simplify this code a little bit so we can pull this
Val outside of the block because it doesn't depend on anything so
you can pull it easily outside let\textsf{'}s pull it outside now we don't
need the blocks and then we have this more streamlined code so we
have the first date we get it updated get the second state this one
goes into here and gets us two and finally we use this tool and return
the new state and the new value so that\textsf{'}s kind of natural given that
the statement is supposed to update previous state into a new state
and return a new state so now let\textsf{'}s see if we got the same code by
looking up the other side of the associative a table so I remind you
what that law is is that flatten flatten is equal to lifted flatten
followed by flatten so we computed this part so for fighting fighting
now we need to compute this so first we can put the lifting of flatten
so how do we compute that we put the definition of flat inside of
F map so f map of this function we substitute so it\textsf{'}s the same way
we substitute and we get this code so this is the code of flattened
substituted into the code of F now now this is not easy to understand
what I'm going to look at this right now and try to simplify it or
just going to continue and simplify at the end so now we take this
code and we apply flatten to that so the result is going to be less
we we take the code of flatten which is this and instead of so let\textsf{'}s
look at called a flatten again just quickly it is this code so the
argument of Latin is dysfunction that is applied so now the argument
of flatten is this lifted flatten so first we applied the lifted flat
and then we apply flatten so the argument of flatten is this so therefore
it\textsf{'}s like this so we need to apply it like this so now let\textsf{'}s substitute
the definition so we substitute this code into here we get this I
encourage you to go through this yourself because it\textsf{'}s hard to show
exactly what happens but I'm basically just substituting a definition
of this function which is here into here and applying it to the arguments
so the first argument is this and the second argument is yes yes so
the result is the scope now we need to simplify so how do we simplify
I will pull out again we can pull out this and out of the block and
then we notice that we have this function essayist that is being applied
to this one and it\textsf{'}s defined like this so I renamed it the first three
but this is just the argument of this function I say yes so when we
apply a sinister s1 it means that here we get s 1 instead of s 3 so
that\textsf{'}s just replace as 3 by s 1 in this block and in line it so the
result will be this so now you see it\textsf{'}s exactly the same code as we
had when we did the first part is this good so after identical transformations
identical transformations are just inlining definitions of functions
into the code and substituting arguments into functions as if we are
evaluating but we're evaluating some bulletin so the result is another
piece of code that\textsf{'}s what what I mean by evaluating symbolically so
this shows that the subjectivity law holds for this implementation
of the statement and in fact if we had any other implementation for
example if we had here a second instead of s 1 then this law would
fail there\textsf{'}s only one implementation of the statement that satisfies
the laws using connections to category theory that I was talking about
you could show that laws are satisfied much easier but the price for
that is a huge amount of extra effort in understanding the so called
at junctions or adjoint factors and those are quite technical and
not easy to imagine what they mean so I rather not go there right
now and so we have actually valuable experience reasoning about code
and that\textsf{'}s good enough for now the next example is a continuation
when I which is this type flatten would have this intimidating time
signature well so actually again unfortunately we can't use the Curie
Harvard correspondence because lots of implementations the correct
hard very hard library returns ten implementations of fat map and
56 implementations of the type that is certainly flat not flattened
and 56 implementations are flattened of flat type services can't count
called triple layering of the Monad on top of the typing so we can't
use that argument unfortunately so let\textsf{'}s bite the bullet and unlike
we did in the reader model where we could now if we actually have
similar arguments with a bit of more complicated function types so
how do we define flatten for the continuation monad well it\textsf{'}s like
this so a continuation is this type and if you have that type signature
that was written over there then you need to return a function of
this type but you're given a function of much more complicated type
so let\textsf{'}s actually write I'll just type maybe for reference so now
CCA is continuation of continuation of a CCA is this and were given
that CC is as you see is a function which has an argument of this
type so we need to give that function so that CCA in order to get
an R so once we do that basically that\textsf{'}s what we need so we're given
a tour which is this argument we need to return R so the only way
to return or is to call CC on this function how do we get this function
I'll just write it saying that it takes this which is a see a continuation
of it and returns R so how do you return R if you are given this or
you call this on it you are and you have a tour so we call that on
a tour and that\textsf{'}s the function that we pass an argument to see see
that\textsf{'}s how we implement the flatten for the continuation Mona anything
bought clear actually once you look at this code it\textsf{'}s very convoluted
there all these functions that you create that are returning something
it\textsf{'}s unclear what this old us but that\textsf{'}s kind of important just now
F map for this mother is actually somewhat easier it\textsf{'}s just a factor
so we need to wrap this function under this it\textsf{'}s a factor because
the type parameter a is behind two layers of function arguments and
so it is in a covariant position so this this entire parentheses is
an in a contravariant position because it\textsf{'}s behind the mirror and
then a is behind another arrow within that so them a is covariant
when wanting a factor for that is always possible and that\textsf{'}s what
we do here so we have a function f going from A to B we have a continuation
from a winter eternal continuation from B so that\textsf{'}s a function taking
a be R which is a type B to R fucking right down from this to this
so how do we do that so we get a br we need to return or we're given
this which is ca we call CA on the function a to R so how do we get
a function a to R we take an A so we write those functions to ourselves
we take an am or returning our so how do we turn our we call BR on
a B when B is obtained by calling ethylene so that\textsf{'}s another exercise
in juggling around functions and their arguments with higher-order
functions so now we do the same kind of games we did with the readable
not so for example we compute this symbolically first so we need to
compute FTL of FTM now afternoon has this code so I just copied it
here and renamed it to and CA to instead of the yarns yeah because
we're going to have AR NCA all over the place you know and so I'll
be confusing so I remained up now I have FTN of this so what is that
last um is this function where instead of CCA I have to put in a set
this function so that\textsf{'}s what I do I copy this code and replace CCNE
with this to notice that now I substitute so any r2 is the argument
of this function which is now being applied to this so and so the
pr2 I write this so let me do not I get this they are going to see
CCA of c2 c2 and then instead of a until I wrote that so for I'm just
mechanically substituting I'm not trying to simplify much as long
as I don't have to I just substitute an argument into a function so
that\textsf{'}s the symbolic evaluation of this now let\textsf{'}s compute a sniper
vestian so that\textsf{'}s Earth Map I'm just copying from this method over
here and instead of F I put FTM so now I substitute FTM code and I
get this now I apply F TN to that so f TN of Earth map of T an old
CCC a so now at the end of this and just copy that over yet now I'll
substitute f GN of x equals this and X now is this function and so
since X is this entire thing I get to call X on this argument and
X has one argument which is BR so instead of BR I need to write this
total PR so let me do that I have that and I'm still here BR here
because I don't want to make it too complicated but BR is basically
this and now if I want to simplify this further then what I can do
is I can say BR of this yes well I can just put it in there CBR has
one argument instead of that argument I have this function the body
of BR is applying this function to a hard one so basically instead
of AR I must use AR one and then I like to write this as a result
so that\textsf{'}s the result with AR one instead of AR the final expression
is this now if you compare that which is here and that which is here
line three three five one three five five they are exactly the same
functions except you need to rename a to see a two and AR one to VR
so that\textsf{'}s the proof that they are the same code you can go with rename
and you get this income so now let\textsf{'}s look at these two examples so
this example is a useful semi do not what is not a full more that
I talked about this example in part 1 of this tutorial so this is
a reader one out with the type for the reader value sorry it\textsf{'}s a right
your motive not a reader mr. right your model the type of the writer
value is a product of v MW and that needs to be a semi-group so what
semigroup law do I use the same rope is that from the two V\textsf{'}s it takes
the left one but from the two w\textsf{'}s it takes the right one so this is
an interesting semigroup which is not trivial but it is another group
it\textsf{'}s not a monoid cannot make it into a monoid and so because of that
you cannot make this thing into a formula but this is nevertheless
a useful example now let me give you examples of incorrect implementations
so here\textsf{'}s an implementation where it is a writer monad and the writer
type consists of a product of WNW but the flattened function uses
this this computation so it takes it ignores V it takes W 1 and W
2 inputs the millander in the opposite order so the type is correct
but we'll see that the subjectivity law fails so alternatively we
can say this fails because the pannier w w is not a semigroup when
we define the binary operation like this when we ignore the first
two and we take a second to and reverse the order so let\textsf{'}s just verify
that this is not a semigroup but the social tivity fails for for the
same 804 this would be similar and that\textsf{'}s the here\textsf{'}s a numerical test
we implement this combine like this so we take ignore P 1 we take
P 2 and reverse the order of the parts on that tuple so then we have
the tests so combine 1 2 3 4 gives you 4 strip that\textsf{'}s the definition
and so then let\textsf{'}s take 1 2 3 4 5 6 if we first combine 3 4 and 5 6
and then we combine the result with 1 2 and the first combine will
reverse the order of 5 6 and the second will again reverse the order
5 6 so then the order will be unchanged but if we first combine 1
2 \& 3 4 and then 5 6 then we'll get 6 5 so associative 18 is obviously
failing and the second example where associativity is film is that
we take a list and we define flatten in non-standard way it concatenates
the nested lists in reverse order so in other words we just define
it as reverse and then flatten so instead of I would say reverse one
reverse one everywhere and then we do the computations that we did
before in a triple nested list and we see what happens so the first
flat it turns you this order but the second first you give a map and
reversed pile and one reverse button and gives you a list of elements
in a different order so they're not equal so once you start changing
the order of things you break a subjectivity and this is because if
you first flatten the two inner lists then you reverse the order a
new flat and the outer list you can reverse the oilers the order a
different way then if you first reverse the order within the outer
part so this numerical example shows you how that works so we have
done computations with semi moments so far we have been looking at
flat map only or flat and looks like the lowest 420 and that\textsf{'}s a semi
moon now film will not have additionally a method called pure here\textsf{'}s
motivation as to why that seems to be useful as I was describing the
front in the previous part with part 1 of this tutorial moon and represent
values that have some kind of special computational context either
they are evaluated non-standard way order many of these values extra
value attached to them or some forest and monads would describe methods
we would describe values that have different kinds of values of the
context so you could have you could imagine for example for a list
the context means you have several values so you could have a large
list or smallest so specific monads will have methods that trade there
is different contexts another example is the ether moment that has
an error value and a successful value and you could have different
error values for example so these are different contexts now when
you compose monads then the contacts are combined in some way and
as we just looked at the laws we find that the contacts need to be
combined in an associative way so context in some sense make make
up a semigroup for a similar that the contacts combine in a way that
is associative now generally useful would be to have an empty contacts
the contacts that you can combine with another context and that doesn't
change that other contexts so some kind of a neutral element so if
you think about contexts as values which is not always possible directly
but it is possible for example for the writing movements so the writer
more of this a good example where you have a value and another value
which represents the context and this value is explicitly combined
with other such values using a semigroup and so that\textsf{'}s exactly what
happens in general with monads except in general you cannot say that
the moolaade is a product of a and some value it\textsf{'}s not some so in
general but contexts combined associative way and if you have an empty
context and it will be like a neutral element or identity element
of the context set its and so combining empty context and another
context should be a no op and should not change another context so
that\textsf{'}s the motivation and in algebra we have a binary operation with
a neutral element analysis that\textsf{'}s a monoid associative binary operation
with a neutral element and so in the writer monad the type W is required
to be a mono in the writer semi Munna it\textsf{'}s required to be only a semi
group so what does it mean an empty context specifically for a muna
it means that you have a function called peer which has this type
so for any value a you can create a monadic value ma that contains
in some sense this value a with an empty context or neutral context
no effects if a monitor represents some kind of effect some kind of
side effect then this value has no side effect so when you combine
this value with another monadic value that has the side effect then
that one out side effect is not have not changed so that\textsf{'}s that\textsf{'}s
the idea let\textsf{'}s useful to have such a such an element for mathematics
and so we hope it will be useful also for programming in fact that
is not so useful for programming you don't often use this method you
use it sometimes but specific ones need to have many more different
methods to create various non empty contexts as well as empty contexts
if the only thing you could do is create empty contexts it will be
impossible to use a moment for anything useful so certainly any specific
one that needs more methods than just pure and flat nut but from the
mathematical point of view as we will see this requirement that there
should be a pyramid is a useful requirement it constrains the types
in a useful way it kills off quite a few implementations that cannot
admit this kind of function with correct laws and that\textsf{'}s a good thing
that you mean you know there\textsf{'}s as we have seen well see later part
of this tutorial many many more semigroups than mana with many many
more semi models the Mona\textsf{'}s so it\textsf{'}s useful to constrain the types
in some mathematically motivated way so what are the mathematical
properties that we want now the two properties that I just again written
up here in terms of code so if you make an empty context it means
you are in certain the given value le into the monad with no extra
computational effect or context or anything then this value should
act as a no hope so as if you did not actually use them on that so
for example this code or you iterate let\textsf{'}s again think about lists
as mu not so we iterate first over this container and then over this
container but this container is only one value which is this X that
you inserted and so this code should be equivalent to this and the
code that you want is let\textsf{'}s say you have a pure of X a flat map with
that and that should be the same as count of 1 which is X context
in the short notation it is a pure followed by flat map and it should
be the same as the function if I'm your thunder and the second is
that when first you have a generator and then you do a pure so just
like we did in the associative also you need two situations when your
construction is before a generator and your construction is a storage
generator and in the second case that pure will be inside the flat
map so that should be exactly the same as this should be should be
able to simplify your code if you have this into this code and that
is account flat map of X going to pure of X so that should be the
same as just cont so this should give you exactly the same container
as before and so that means flat map of pure is identity function
so these are two was the called left identity and right identity at
this point is not obvious why they're called like this and we will
see children so there\textsf{'}s an additional law in fact for pure it\textsf{'}s required
to be a natural transformation so pure is a map between a and M a
sub a is the identity function and when is the MM functor so it\textsf{'}s
a map between two factors required to be a natural transformation
and that chirality law looks like this difficult for materiality you
interchange the order of some arbitrary function with your thumb in
the natural transformation and that should work you need a knife map
on the right hand side and here are the types so you start with a
map to be and you insert that into the moolaade using pure ore you
first insert and then here F map the moment and that should be the
same result a left identity looks like this so let\textsf{'}s substitute the
definition of a flat map in terms of flatten which is f my path followed
by flatten and then we have the law that F pure flatten it is F for
any F which means that pure followed by flattened is identity but
here F must be given your type inside the Munna so the both side of
this identity must be applied to as a just with some weight so that\textsf{'}s
how it works some types of you start with some si the pure must be
applied to the data so it gives you an SOS a so it\textsf{'}s a pure that is
applied to this type gives you a source a and when you flatten not
give you back a say that should be identity so this should not introduce
any extra effects or anything the right identity is similarly you
substitute the definition of flat map in terms of flatten into the
flat map of pure again F map of pure followed by flat that should
be identity again both sides are applied to a type si so F map of
pure is inserting a into the unit under s so that sa ghost SMA and
it\textsf{'}s different from this where we just applied pure to si as if that
was some B so ever say this essay is just some B but here we don't
be with her F map so I indicate this by putting a type parameter of
pure explicitly just to be sure that it is clear what we're doing
so this is a different function this is a pure and this is a hash
map of pure although they work on the same types and then you flatten
and that should again be identity so in this formulation it is shorter
so you know then a formulation of Lois with like that but still it\textsf{'}s
not clear why these are left and right identity laws what is left
and what is right here exactly well you can say for fom pure was on
the right and here P Rose on the left but for flatten it is not for
flatten both times flatten is on the right so we'll see why does it
so but we know this either the laws in order to understand more deeply
why what\textsf{'}s happening here and why we're writing the laws and talking
about them as right and left identity let\textsf{'}s recall how we formulated
the laws of filterable factories so we used the F map hoped which
have the this type signature and then we found that we had to compose
quite often functions of this type a t1 plus B so a t1 plus B and
then beta 1 plus C we define an operation which we defined already
denoted like this to compose these functions so we have a very similar
type signature here the flood map except that this was the option
factor but this one is the same Thunder s as here so let\textsf{'}s try to
see if we can compose these functions so these are closely functions
this is just how they're called a class-d function is a function of
type it goes to s be where s is a certain factor so it\textsf{'}s kind of a
twisted type it\textsf{'}s a function but it\textsf{'}s life is a bit twisted and so
because of the twisted type you cannot directly compose them a to
s be beta SC kind of directly compose but using flatmap you can easily
compose of course notes f8 o has been G beta SC and then you just
take firstly apply F to some a you get an SD then you apply flatmap
G to that s be so flat map G I goes from s B to C so you can easily
compose that with us so you can post flat map G with F and that\textsf{'}s
the definition of what we call the closely composition which is denoted
by diamond now further filterable factors I have the super had the
subscript opt under the diamond just to remind us that the option
is the optional factor that is being used to twist the function type
in the classical function here it is the founder s that is being used
for the class Li function type so if I were to be completely pedantic
here I would have used diamond with subscript s but that would be
a lot of extra symbols and actually we only have diamond s everywhere
we don't have any other factors except s right now and so let me just
her gravity always diamondden I mean diamond s so diamond defined
like this where this flat map is for the same unit s or for one address
we defined the class Li identity which is a function of this type
and that\textsf{'}s just a pure the pure has the right type so now let\textsf{'}s see
what the laws are so the composition law actually {[}Music{]} can
be written like this because the composition law has flat map of F
followed by flat map G so it\textsf{'}s like this so basically the composition
law for flat map which is similar to that of f map opt from chapter
6 this composition law shows that flat nervous some kind of lifting
it takes functions of these types and it produces functions of these
types and such that composition of these functions corresponds to
composition of those functions after lifting so of course on the left
are slightly composition on the right there is the ordinary composition
but this is very similar to lifting and the laws are similar to function
if ting laws as we will see so what are the properties of this closely
operation so let\textsf{'}s reformulate the laws of flat map in terms of the
class the operation a class decomposition a diamond so the formulation
becomes a very elegant set of laws so that left and right identity
laws are like this so pure composed with F is if F composed with pure
is f now here F must be one of these functions in now it\textsf{'}s obvious
why they're called left and right identity loss pure is identity and
this is exactly like a binary operation in a mono ed which has left
identity right identity associative eighty law is written like this
which is very concise and it follows directly from flm law because
they're phalam law all you need to do is you write the FLN law which
is this the that equals that and you prepend it with some function
f arbitrary function f and then you rewrite this by definition F followed
by flatmap is the Dimond operation so that becomes directly the left-hand
side from here and the right-hand side from here now in written in
this way the laws are very suggestive of a monrad so these laws express
amyloid of functions where the binary operation is the diamond composition
or the classic composition the functions must be all Class C functions
so they must all have the as twisted type a to s be for some a and
B and pure must be a natural transformation that is of the type a
to s a and so this is why you hear that Vinod is a mullet in a category
of and the factors now I don't point explaining the details of this
because actually after studying it I found it\textsf{'}s not very useful as
a description of what a monad is what is somewhat useful however is
to look at the laws of the moon ad in this formulation it is actually
not very convenient to program with this operation the diamond is
also not very convenient to check laws for it because of the complexity
lose all these type parameters and arbitrary functions that you have
to keep but the formulation of the laws is certainly the most clear
and suggestive so so this is a monoid in certain sense so on the set
of functions of this type so if you consider just a set of functions
of this type for any a and B then they form a monolid together with
this as this empty element or neutral element or identity element
whatever you want to call that and the operation which is the diamond
so after this let me explain what is this category Theory stuff about
and why we want to use it until until now we have seen several kinds
of liftings that mapped functions from one kind of functions to another
and so let\textsf{'}s try to generalize all these different liftings that we
saw so far we have seen liftings of plane functions until these kind
of functions lifted into the Thunderer F this is an F map so f map
was lifting from here to here we have seen F map opt which listed
from a to option B into FA to FB so this was option and for some F
the filter of the functor and this is now we have seen a lifting of
this to this directly with the same factor nicely over half is how
we call these functions nicely functions with 1/2 F or earth closely
functions over factor f in each of these cases we saw an identity
function also being present in some way except that in the closely
functions the role of the identity function is played by the pure
the composition was given by ordinary function composition of these
first two cases and by the diamond operation in the third case however
the laws are the same left identity right identity and subjectivity
of composition so category theory generalizes the situation and says
that the difference between these situations is just in the type of
functions that are being used and the kind of composition that is
being used other than that situations are very similar each of these
are called a category and so to say I will present a very concrete
view right now which is that category is basically a certain class
of twisted functions which are denoted with this squiggly arrow and
in different categories these types of these functions could be different
so here are the three examples we have seen so far so twisted functions
are called morphisms in category theory and you have to specify which
category are working with so usually we work with playing functions
and sometimes we work with of Indies now category must have certain
properties these properties are that for any to morphisms there must
be a composition morphism and the type you must be like this a to
B B to C A to C now the squiggly arrow remains the same it could be
each time this or each time this for each category the second axiom
is that for each type any there must exist an identity morphism which
has this type and the other two axioms are identity loss and associativity
law so if you have found somehow the type of twisted functions or
morphisms like this and you can define identity morphism and you can
define the composition such that the laws hold then you have defined
a category that is the idea so category is kind of an a twisting of
the idea of functions and the value of this generalization is that
you can now define factors in a general way as a map from one category
to another so for example a map from this category to this is a functor
when it preserves identity and compositions these are the functor
laws identity in this category must be mapped into identity in that
category so if you lift this twist and you must lift this to that
and composition of function in this category or morphisms generally
composition of morphisms must be lifted into the composition of lifted
morphisms in this second category and this is the same for all these
lifts so for example lifting from here to here same properties must
be so it\textsf{'}s flat magnet lifts but the properties are the same it preserves
identity and compositions are the three laws that must hold now what
we called functors so far in category theory is called endo factor
so what we call functor is a lifting from here to here and this is
called endo factor but category theory has a lot of terminology that
is not particularly useful in programming just a terminology that
is useful is what I'm talking about here so I just tell you about
this in case you encounter this word basically this is just factor
and it just goes from plane functions so I would say in my terminology
the category of plane functions is just ordinary functions with ordinary
identity in ordinary composition but the category of functions lifted
to a founder F or type constructor F let\textsf{'}s say is a category of functions
of this type with identity like this and composition is still the
ordinary composition class by category over from a third constructor
F is this and then you can demand the properties of lifting and that
would make it a functor or if you lift from here to here that would
demand it to be a mu naught and so category theory in this way has
a very concise language it allows you to define things like functor
and monad just by saying this is a certain category that you you've
got to lift from this category to that category and that already tells
you what the rule must be what the laws must be everything the definitions
it\textsf{'}s a very concise way of talking things that a high level of abstraction
but at this level of abstraction not much code can be written directly
and so I I think this is kind of optional term to go this far and
so as a last kind of abstract slide here I will show that if {[}Music{]}
if these classic functions form a category for a specific type constructor
s then it is a monad so first thing to notice is that if you are given
just a class decomposition and the the pure operation and you can
define map and flatmap for your type constructor so here\textsf{'}s how so
flat map of F is class decomposition of identity and F now this identity
is an ordinary identity is not pure it\textsf{'}s just ordinary identity of
this type and it can be closely composed with F by pretending that
this is some type Z and so this is z2 as a and F is a toe has been
so the result would be z2 as but Z is si so that\textsf{'}s how it works so
we can use identity ordinary identity and we can use the viewer and
we can use the ordinary composition and we can use the class Li composition
using those we can define map and flatmap so here\textsf{'}s how well the help
is defined through class Li like this it\textsf{'}s just a flat map of a pure
function which we'll see here so actually it turns out that we need
to require two additional materiality Louis for peer which are {[}Music{]}
written like this and they connect ordinary composition closely composition
F map and peer so what I believe is that if somebody gives you just
the Kleiss Lee composition operation and the pure we still have to
verify that these laws hold it must be natural transformations so
these laws kind of say that F followed by pure is kind of similar
to closely functions if you have a non closely function have ordinary
type you just take on a pure at the end of it and you get a closely
function and then it behaves as a class Li function so ordinary composition
can be replaced by closely composition and if you have an F map of
an ordinary function then you can pretend this is a Class C function
by putting it into pure and then you can do this so in this way you
can also compute f map because you can put identity on the left here
and then you get a live map with identity of G and pure and so then
once you define in this way I can assume these laws the laws of pure
and flatmap follow from the category noxious indeed so left and right
identity laws or immediately discovered if you're right in this and
for example this is just identity law because you can take any function
nicely composed with pure and lets that function itself so if you
write this and you write down the class like composition {[}Music{]}
in terms of flatmap then you get left and right identity tools it\textsf{'}s
social do it if your flat map follows like this when you write this
which is true and write it out and substitute flat map like this so
identity diamond def is flat map F so this is flat map F and then
this is another flat map of F G which is a flat mapping in this guy\textsf{'}s
where this F followed by flat man G so you just write it out and you
get the Loess enough left naturality which we assumed here allows
you to compute this which is kind of interesting fpou G H is f G H
so there is a kind of a weird associated between going on here between
ordinary composition and classic composition which is interesting
and very useful for computations so then you get naturally for pure
for example by writing this this is our neutrality assumption and
you get the natural D for pure out of it out of that flatten has defined
them as identity diamond identity with these types and that rally
for flatten can be found by saying well what is this it\textsf{'}s flatten
his identity diamond identity f map his F pure using this definition
then you simplify you get identity where this thing can operate here
is identity in diamond sometime so identity diamond something is F
map of that something so then you can get rid of one of the identities
you get this and the other side of the flatten that trowel tool is
double F map which you have to write out so you write it out then
you get this identity it can be simplified the way then you have again
finally you know just this and these are the same so in these in this
way I can derive from the moon applause I can derive the closely packed
category lows and from plastic category rules together with these
extra nationality assumptions I can give the Monad laws so let\textsf{'}s go
back to a more concrete world where we combine contexts associative
way and in the semi monad that\textsf{'}s sufficient the context are combined
as an SME group but an effeminate the contexts are combined in as
in the monoid they have an empty context which we can insert and so
let\textsf{'}s see what mono is are what are the types that have the property
of being semi groups and one works there are quite a few examples
of semi groups and monoids and there are some specific examples that
have particular nature for example for integer type in many ways of
defining a monoid you can have a product some maximum minimum different
kinds of semi groups and monoids for string for example you can define
different models different semigroups concatenating strings with separators
for example in different ways for the set type you can define intersection
of subsets or union of subsets as monoid operation another interesting
example is the route type in the a collider in the HTTP which has
an empty route that always rejects everything and the route concatenation
operation that puts one round on top of another and combines them
and that\textsf{'}s a manorial operation but these are kind of special and
here I listed some generic constructions of Mao\textsf{'}s so that you could
appreciate how to build new monoliths from old ones and what kind
of properties are required for a monoid so let\textsf{'}s now go through raising
examples the first example is that you take any type and you make
it into a semi group and that\textsf{'}s very easy you just define a combined
operation that ignore is one of the values so for example you just
delete the right value ignore the right value you take the left value
and that\textsf{'}s your result that\textsf{'}s how you come back this is a kind of
a trivial operation that combines non-trivial perhaps not very interesting
way but it satisfies associative it let\textsf{'}s see why here\textsf{'}s an example
I define a 7 group for anyway and now any types are combined and ABCs
of strings integers everything is combined in this way now after I
define a simple associativity for example here\textsf{'}s ABC and I combined
them in two different ways and there\textsf{'}s always always a so why is the
social TVT correct that\textsf{'}s because in any combination of these the
operation will always between the leftmost value all the other values
will be simply even ordered so obviously this is a so associative
you are deleting all values except the leftmost value and it doesn't
matter in which order you do need them and similarly the right trivial
semigroup which ignores the left value and returns the right the next
example is already seen in the previous chapter where you have a selling
group and you add one to it so you have an option of a semigroup I'm
not going to go through this but the laws are satisfied as long as
that is a semigroup so identity laws are satisfied by construction
and the semigroup laws are satisfied that that is a semigroup another
example is the list list as a monoid for any type of sequence and
so on because you can concatenate lists and that\textsf{'}s a valid operation
and the empty element is the empty list so you can catenate empty
list with anything and that doesn't change that other list so obviously
it\textsf{'}s associative because you just concatenate the lists in the order
in which you have written them and so that doesn't depend on the order
in which you remove parentheses between the lists this construction
for is also generic so it\textsf{'}s for any type a it\textsf{'}s a function from a
to a that is a monoid the operation which I denote one this it could
be a composition of functions in any specific order so there are two
different one if you choose one order of competition or another so
let\textsf{'}s look at this left composition or you say X and then Y for the
combined operation and the empty element is identity function so obviously
the composition respects identity and subjectivity is clear because
you apply X then you apply Y then your plan Z so the order in which
you apply x y\&z is the same and it doesn't matter in which order
you put parentheses here that\textsf{'}s left composition as a right composition
when you write composed instead of and then and compose is X compose
Y of a is X of Y of a which is the same as Y and then X are we first
you say y avait and then X of that identity laws are again obvious
associative it is again easy because X compose why compose is basically
this X Y Z there\textsf{'}s no way and the order is the same and there\textsf{'}s no
way that this can change and so doesn't matter where you put parentheses
here parentheses in here are unimportant the next example is total
order type well this is also a generic kind of example of any total
order types in the enumeration or integers so the only thing from
on the way you need you need a maximum or minimum so if each have
maximum as your binary operation you need a neutral element for a
monoid and you might not have a neutral element for example for integers
there is no maximum integer if you if you use arbitrary precision
integers if you use finite integers then there is it max int so you
could use that associative 'ti is clear because you take a maximum
of several elements and doesn't matter which order you compute the
maximum it\textsf{'}s going to be the same maximum or minimum the next example
is the product so if these two are seven groups or monoids then the
product is also a semi group or a monoid let\textsf{'}s see how that is done
so I'm defining a one or a typeclass instance for the product given
that these two were monoids the empty is a pair of two empty elements
and the combine is a component wise combination so I have monoid operations
separately in this one against two and because they are performed
separately in each part of the tuple then each part of the tuple separately
will satisfy all the laws because you can just delete the other part
of the tuple temporarily and look only at what happens to one part
and then obviously you just have the first one oh it you assumed its
laws already hold or or Senegal and so obviously then the entire tuple
will also satisfy the laws the next example is the disjunction of
Tim seven groups are monuments which is kind of maybe less trivial
than a product so here\textsf{'}s what we do here it\textsf{'}s a right biased either
we have to choose if it\textsf{'}s left biased or right box up to two different
ways of doing this so there\textsf{'}s not there\textsf{'}s not a symmetric way of combining
two semigroups or two more nodes into one but even either the left
side or the left on the right side must be and chosen as the main
side in particular the north and it\textsf{'}s neutral overland and it must
be either on the left or on the right there\textsf{'}s no nowhere to do to
do it symmetrical unlike in the product case so let\textsf{'}s say we have
a right biased either what does it mean well here\textsf{'}s how it works we
need to combine some values of type either a B where a and B are both
mono it I'd say if all of them are left all of these others are left
then we combine their values they're all values of type a so we can
combine them into a left of some eight if they're not all of tied
left already then at least some of them are right a right of B then
we discard all the left and we combine all remaining right of the
operands into one right that\textsf{'}s that\textsf{'}s the idea now this would be associative
because we just formulated the rule that doesn't depend on the order
in which we apply that rule so discard all operands of type left a
dozen doesn't depend on the order in which we discard and then you
combine all of them into one using the monoi operation which again
my assumption is associative and so identity element must be the empty
element on the left and then clearly combining it with the left will
produce correct results combining it with the right will produce there
the right now you without change because you discard the left and
you just get there right so that\textsf{'}s why it will be respecting all the
laws so here\textsf{'}s an implementation so I'm writing down a monoid typed
class instance for either a B given that a and B are monoids the empty
is the left of empty a so we have chosen a right biased either so
then the mono of the empties on the left now how do we define combined
well we have two elements either could be a and B so we match there
for combinations left and left we combine right and right we combine
if we have a left on the right then we discard the left so in the
first case we we must do this because if if X is empty then the result
must be this and there\textsf{'}s no other way to implement that so once we
implemented in this way the laws will hold by construction here\textsf{'}s
an example how this works left one right two left three I'm just using
the integer bond modulus addition then all the left are deleted in
the right to remains right one right two left three the left are deleted
so right one right two remains are combined into right three so for
example u 2 and Q 3 deletes this keeps right - and another example
so again left is deleted right and right are combined the next example
is construction it if we have a monad can type constructor in and
Illinois s then M of s is a manured kind of an interesting example
because now you can construct a lot of types as more nodes if you
have some standard monads and your planet to one law it\textsf{'}s constructed
previously so here\textsf{'}s how it works I'm defining a typeclass instance
for semigroup so here I will only check the seven group not a full
monoid and actually exercise one will be to show that it is a full
moon would if if M is a moon odd then M is a monad s is a monoid an
M of s is a Mulla Mulla well I'm only going to use a semi monad and
the semi group so I only need to define the combined operation so
how am I going to define that well like this very interesting elegant
piece of code X is a monadic value with s y is another man Alec value
so I just combine them using them monadic combination and semigroup
combination so associativity follows because if you do this then you
would have called like that and if you first do the first two that\textsf{'}s
what would be if you first computed for yield with these two and then
inserted it into another for yield here which is exactly the first
the the associativity law for for the Monad which is that you can
inline parts of your for yield block into a larger for yield block
so that associativity guarantees associative 'ti at this level and
then the semigroup which is being used here must be assumed also to
be associative and that guarantees associative et in the last line
so it\textsf{'}s kind of very easy to assume that this thing is associative
without going through a lot of a lot of computation and coding as
we did before and an example of when it becomes a full mode let\textsf{'}s
take a reader model reader monad and apply it to a Molloy yes so the
empty element is a function that takes an argument and return the
empty value of the mundo it and the combined two readers is to take
the Z and we apply x and y 2z and you combine the results and so basically
all the neural operations are performed with the values of s that
you've obtained by applying functions to some Z so for each fixed
value of Z once we apply this function you get a monoid valued and
all the laws hold so they hold separately for each value of Z and
so they always hold for all Z and therefore a function from Z to a
moon or it is itself a monolid and Z doesn't have to be infinitely
it\textsf{'}s just fixed type doesn't have to be moderate adorned the final
construction is more complicated and it\textsf{'}s motivated by some mathematics
but actually it has applications in practice so the construction is
a product but only one part of the product is a semigroup and the
other is not necessary necessarily a semi-group so SS is similar but
P is not P is just some type however the semigroup pacts on P so it
has an action loopy and that is what makes makes it possible to define
a semigroup on this product what does it mean that s was an action
on the P and actually the function from s to a function from B to
B so for each s there is a transformation of on P and these transformations
must be such that their composition corresponds to the non-oil composition
of s so this is in mathematics a typical situation then let\textsf{'}s say
the group acts on a vector space there is a transformation and multiplication
of transformations corresponds to group multiplication and here we
just use a semi group and that\textsf{'}s sufficient so the result is a product
s and S\&P which is called the twisted product so it\textsf{'}s we wouldn't
be able to define a semi group we define a semi group using this action
without this action would be very very useful we could define a trivial
selling group with one of these ignoring operations but I wouldn't
be very interesting so here\textsf{'}s an example of such situation if s is
this which is something then it acts on a because for each function
you can transform this P alpha is identity and obviously it satisfies
this law because the composition of functions is the same as composition
of transformations another example is a product of boolean and an
option of a where you can act with boolean by filtering so filter
operation satisfies this law because as we know from the properties
of filter balls filter composition with another filter is a filter
with boolean boolean conjunction of two boolean values so if we define
this semi group as boolean conjunction then the filter would be good
so this could be any filter it\textsf{'}s not notice is really an option I
just put here an option as an example but this could be any filterable
so this is a generic example of a semi group with a non trivial structure
so we're going to show you some code for this twisted product so here\textsf{'}s
how we define it so I'm defining a semigroup instance for don't actually
doing this I have a single group s in a3 which is unknown type and
I have an action has to be 2p and then I can define a semi group of
SP with this combined operation so the first element of the tuple
is combined using the semi group actions a semi group operation but
the second one is an action that puts together the first X so the
first semigroup and the second P so the first one here packets on
this and that\textsf{'}s the result so that\textsf{'}s kind of a twist and here I can
verify symbolically that associativity works I consider this definition
so that is I just defined other things and then I consider three different
values and their combination in two different orders and the result
would be that the actions would be like that and they should be the
same and there will be they're going to be the same as long as the
action a satisfies the property that we assumed in other words a FS
1 over L of s 2 of P 3 is the same as a of s 1 plus s 2 of P 3 so
and that is so let me just give an example where we define a semigroup
using a boolean and an option int so boolean acts on any filter mode
using the filter function so optional suitable we're just using a
standard library for naught and we define this implicit value of semigroup
for Q for the type Q by calling that function we just defined and
I see that the values work and there is associative it now these are
just very simple examples I'm not sure this is extremely useful to
filter options with boolean but this could be useful in some application
perhaps in any case these are the generic constructions that you can
use to make new scenario humanoids out of old ones in the heart of
other parts with this as inspiration let us now look at what are the
constructions of possible semi models in Malad our intuition is that
the best for analysis is to consider the flattened functions as the
simplest type signatures simplest laws and the way if flattened works
this text data in this nested container and somehow fits that data
back into the original container and this should must be a natural
transformation so you don't actually perform any computation with
this data other than reshuffling it in some way now you have seen
examples of monads and you have probably asked yourself a question
of what are all the possible Muna\textsf{'}s are so different how do I know
that the given type is a unit or not in fact this is an open question
I believe it is not obvious that we have an algorithm to decide whether
any given type expression is a monitor or not and as you have seen
it\textsf{'}s a bit cumbersome to verify the laws of the moolaade but here
are some constructions that I found that always give you lawful monads
and so if you construct a monad using these then you don't need to
check laws you can prove this in advance that all these constructions
give you correct units and then you just use them you don't need to
go through checking the laws every time in your application whatever
you define new data type with a unit instance so all these constructions
use exponential play level types so they are either products or disjunctions
or functions function types so the simplest construction is that of
a constant factor and the constant factor is semi Malad for fixed
type disease but it is not a fulminant because the identity laws cannot
be satisfied unless the type G is unit so the only constant factor
that is am honored fully model is a unit type constant factor so let\textsf{'}s
see why that is so let\textsf{'}s define this type constructor with Z type
parameter which is that constant and he is a functor type parameter
that is not used in the actual type because it\textsf{'}s a constant factor
so the function is going to be written like this in the syntax for
the Skull kind projector plug-in so here\textsf{'}s the inst instance of a
semi mullet we just do nothing flatmap returns the initial value with
no changes ignoring this function that is given there\textsf{'}s no changes
because you map a to some something else but there\textsf{'}s no way for you
to there\textsf{'}s no a there\textsf{'}s no way for you to use this function if so
you can't call it you have to return this Z so your Z is just stays
the same flatten is identity flat map is identity F map is also identity
because there\textsf{'}s no a to total transform and so associativity is trivial
there are other entity functions composition of identity functions
is identity however this is not a fulminant the right identity law
fails here\textsf{'}s why the right identity law says that flatten of pure
of some X must be equal to X for all X now what can pure do pure goes
from A to Z it cannot do anything except give some fixed value of
Z suppose I give you some fixed value of Z for all a there\textsf{'}s no way
that you can use the value of a 2 let\textsf{'}s try that so pure actually
cannot depend on its argument and therefore pure of X does not have
any information about X it will be a fixed value of Z that\textsf{'}s independent
of X and flatten of that cannot possibly restore eggs the only exception
is that if if type Z itself is unit then there is only one value of
x you can always just put it in pure returns one flatten of that returns
one and that\textsf{'}s the same one that you had here everything is just always
equal to unit and so in that case identity loss hold so that\textsf{'}s the
only way in which a constant factor can be am honored and fully lawful
one world as well as a unit constant fuck factor otherwise it\textsf{'}s a
seven moment 

the next example is the construction of a semi mullet or a monad which
looks like this for any factor G we take the tuple of a and G of a
and that is the new type constructor F and we will now show that in
general you can define a semi mullet for F and if G is unit that is
there\textsf{'}s no G here just a that\textsf{'}s the identity function and that\textsf{'}s a
full moon that there\textsf{'}s no other way that you can get full mana out
of this construction so how do we show this first of all we need to
define the function instance for this and then we'll define a semi
Menard instance define the Phantom instance we need to produce an
implicit value of type factor of F but actually F is not defined as
a type constructor because we have this arbitrary function G so we
cannot define the type constructor F without first defining a G directly
so instead of defining it directly we use the kind projector and so
then this syntax will represent the type constructor that takes a
type X and produces the type of tuple X and G of X and so we will
have to keep writing this extra type expression instead of F which
is okay if you don't have to do it for T many times later we will
say we're looking a little calm down in this repetition so the factory
instance for this type constructor is straight forward we need to
implement the function map which takes an FA which is a engine way
it takes also a function f which is arbitrary an arbitrary function
of arbitrary type it to be and we need to produce a tuple of B and
G of B so to do that is relatively straightforward and unambiguous
we use the fact that G is already a functor so G already has a map
function of its own we use the type constraint here typeclass constraint
so say that G is a factor and we have imported the syntax for the
for functors and so now we can just say g g a dot map so that\textsf{'}s what
we do here using the factor map from the front row G so how do we
produce a tuple of being G of big well obviously we have a tuple of
a and G of a so this is the stupidest structure it using a match expression
obviously we need to use the function f to produce anything that has
been it because there\textsf{'}s no other way to get any any kind of be acceptable
using F so we call F on a to be get a value of type B here and we
do a G a dot map F and that gets us G B so that\textsf{'}s easy enough now
how do we define a cell or not this is a little more involved so again
we assume that this is a factor and we are going to produce a value
of type semi Malad of this type structor expression and that\textsf{'}s going
to be the implicit value but this value is characterized by G so we
need to make it a death so the function we need to define for the
semicolon is flat map so how do we define flat map well we have an
arbitrary function of type F sorry of type A to B and G B we have
a tuple a G a and we need to obtain B G B so let\textsf{'}s think about how
that function can be defined and to get intuition about it let\textsf{'}s remember
that mu not represent some kind of computational context or effect
that accompanies a value and so here obviously we have a value of
type a and also a value of type G of a which we could consider to
be the effect or the computational context of the value a so just
intuitively now we have the initial value that has its own effect
that\textsf{'}s the first one and also we have when we apply F to anything
we will get another G will be which is a second effect now clearly
these G of a and G G of B near arbitrary functors we can't combine
them so usually Mona would combine two effects into one so for instance
if this were just the reader Mona that\textsf{'}s that would be a tuple of
a and W and here will be a tuple of B and some other W we will just
combine this w and that W using the semigroup but here we don't have
a second group G is an arbitrary factor there\textsf{'}s no way for us to combine
two values of type G of BC we could try to map F over this but then
we'll have G of B and we can't combine that G of B together so we
have to discard one of the effects basically that\textsf{'}s the basic reason
why this cannot be a full moon usually when they discard effects that\textsf{'}s
red flag in terms of having a full monad film Allah doesn't discard
the two effects it combines them in a mono Idol fashion semigroup
combination can discard but 108 combination cannot discard if you
you cannot defy them annoyed when you define your binary operation
that discards one of the arguments that will be not a lot from annoyed
because it would not have an identity value because you could not
combine identity and X to get X when X is discarded so you're not
allowed to discard if you want to have a mono it\textsf{'}s similarly here
we shouldn't be discarding any effects or any information when we
would like to have a full owner all right so that intuition gives
us a guidance that probably we're going to have a semi monad here
and not a homo not because we have to discourage something there\textsf{'}s
no way we can take into account all this information because we cannot
combine G\textsf{'}s so let\textsf{'}s just arbitrarily decide to describe the second
effect and that means we will basically ignore the second part of
the tuple the function f produces so ignoring the first effect would
mean that we ignore this you know in a second effect means that we
ignore this so let\textsf{'}s just arbitrarily decide that we will want to
ignore the second effect so that means we define a function f1 so
the first part of F which is this so this will apply F and then apply
the extraction of the first element of the tuple which is then basically
means we discard this and then we get the function a to b and now
we just apply this map function that we have before so with a we get
this result this is the same code as was above here so this is one
way of defining a same unit which is really arbitrarily decided that
we want to describe the second effect who would have decided them
we could have decided to describe the first effect instead wouldn't
be also similar than to show this as one of the exercises so let\textsf{'}s
continue now let\textsf{'}s take care of the case when G equals unit and then
we just have the identity function as our identity factor now this
is also the identity monad because just define all the functions as
identity flatmap f map flatten is identity pure is identity everything
is identity and then obviously all Monad laws will trivially hold
because all these laws mean composition of some identity with identity
which should be equal to identity so it\textsf{'}s always going to hold all
those if you define all your functions as identity functions so for
this function you can define all your methods as identity functions
and that\textsf{'}s what we do that\textsf{'}s called the identity monad not a very
interesting will not per se but it\textsf{'}s interesting to note that it is
a unit so for instance this is a functor also not a very interesting
one but it\textsf{'}s important to have that founder as an example because
for instance in a construction you use an arbitrary factor or an arbitrary
unit and you can substitute identity founder or identity monad into
some constructions like one of these constructions that have arbitrary
functional units in them and you get an example of a new unit by that
so identity models and identity functions are not necessarily useless
not necessarily useful directly but they are useful for constructing
new moon and sometimes let\textsf{'}s now go on to verify the associativity
law for the semi moment that we found so for brevity will define type
aliases so f is this you see this is not that this F is not the not
a function by itself good it has two type construction I had to type
parameters so it is a type constructor that has two type parameters
a functor should be a type constructor with 1 type parameter so we
need to write things with the kind projector as we did here in order
to get the right syntax for a functor type constructor will not that
constructor but just for gravity we're going to introduce this notation
these are type aliases so this is fine this is f of GA which is there
but what would you know that there is FA here it is explicitly taking
G as a parameter now F F is just f of F this is just the same as if
we stop writing G and that\textsf{'}s what it won't be and and this is F of
F of F of ad writing it now it would be quite cumbersome so I just
want to introduce these type a leases now in order to check the laws
the easiest ways to look at the flatten function what F not the flat
map function so let\textsf{'}s derive the flatten function probably the definition
of the flat map that we were given above so we given this definition
of flat map and flatten is flat map of identity which and the identity
must be of this type so let\textsf{'}s copy this code and substitute identity
instead of F so that\textsf{'}s what we get now F 1 is a simple function that
just takes the first value of the tuple so we can substitute that
into the code so let\textsf{'}s see I'm going to rename this to FA and this
to G FA because the types of these things are FA and GF this FFA is
of type F F of GA which is this it\textsf{'}s a tuple of F and G of F so therefore
I'll choose my names here fa g fa these names were just copied from
above so I want to rename for clarity to remind myself what types
these variables are so now if when you say here for example f1 okay
but with one is just taking the first element so instead of this I
write this and here I write this so let\textsf{'}s further simplify now obviously
if a is FF a dot underscore one so then this is dot underscore one
dot on your square one and this is FF a dot underscore two map underscore
underscore alright so this is the code of the function f flatten now
this is what I call a symbolic derivation of the code so this code
was derived by substituting function definitions and simplifying but
simplifying is more or less like in algebra you have an expression
you substitute a function into it and for example here FA is FF a
dot underscore one and so you substitute here instead of F a so this
is quite similar to mathematical derivations in algebra so you just
substitute equal values for equal values anywhere and substitute definition
of functions to apply them and then simplify and and so on and so
this is a symbolic derivation of the code I could have just written
this as a definition of flatten this would not be a symbol in derivation
because I don't get the code of the function cannot reason about that
code as easily here is the code of the function that I can later reason
about I can again do the same substitute is simplified and that\textsf{'}s
what I will have to do in order to verify the laws so there\textsf{'}s a there\textsf{'}s
a symbolic clarification that I'm going to do rather than so to speak
a numerical verification if I just wrote this and then used Scala
check to call this function on hundred examples with integer types
or something like that now it will be you know certainly a bit helpful
perhaps to do that unfortunately in this example we're trying to prove
something for an arbitrary function G how are you going to do this
with Scala check I don't think you can do this let\textsf{'}s go check because
there\textsf{'}s no such thing as an arbitrary function in it I don't think
so I will be interesting to add that functionality but it will be
hard to do it just here in this tutorial and this functionality generating
an arbitrary function is this is kind of vicious so you probably need
a lot of code for that so anyway it would be hard to do and it would
only give you specific type parameters and specific functions checked
not arbitrary types and not an arbitrary function or as a symbolic
computation that I'm doing here is more like a mathematical proof
it is rigorous it is proving it correct for all type parameters and
for all functors now if you remember the associativity law for the
semilunar it involves f map of flatten so we need to implement ethnic
let\textsf{'}s do that F map is just the same as map what we did above except
it\textsf{'}s arguments are flipped first comes the function f and then comes
F of a other than that the code is identical now the associativity
law is that this expression is equal to this expression so this function
and this function are identical that\textsf{'}s the law so in order to check
that we need to symbolically derive the code of these two functions
and compare and somehow show by simplifying maybe renaming variables
and such that they have the same code these two functions so let\textsf{'}s
start by computer for F map of flattened because that\textsf{'}s the first
function we need to check is that is f map of flatten followed by
flatten so that\textsf{'}s I just want to remind you is the law of associativity
find that slide this is the law of flatten right here written in this
short notation this is lifting in the functor so this is f man so
basically f map of flatten followed by flatten must be equal to flatten
followed by flatten with types accordingly matching what me back to
this slide and go back to my code so first step therefore is to compute
the code for this so let\textsf{'}s do it how do we do it well with I I do
it by writing out this function as if that were a separate function
I'm defining and code for the body of this function will be symbolically
derived so I call this function as my up flatten just it doesn't matter
what I call it really I look at the COS function in the code so I
name this function just to be sure about what I'm computing so this
function f map of flatten takes an argument of type F F F of a and
returns F F of a because flatten takes F of F of a and returns F of
a and I'm lifting flatten once so I'm adding one more layer of F so
in order to derive the code let\textsf{'}s take the code of F map and substitute
the function flatten instead of F into that code so I take this and
instead of F I write flatten so that\textsf{'}s what is going to be so it\textsf{'}s
going to be flattened of a which is here called f f/a because it\textsf{'}s
that type and then it\textsf{'}s going to be this map F on this map flat so
that\textsf{'}s going to be the result of substituting the definition of F
map in here so that\textsf{'}s first step so now let\textsf{'}s substitute the definition
of flatten which was right here yeah that was flattened so let me
keep it on the screen after we simplified it so that was the code
for flatten so we need to apply this to F FA and we need to do the
so f fa is instead of yeah if FA is right here so we just repeat this
in here and the last part of the to boy is unchanged so I put this
all in comments because this is my preparation now FFA is the first
part of F F F a and G FFA is the second part so then I rewrite it
like this so instead of FFA I write F F F a wand and so on so instead
of G FFA I write F FFA too and so I get this expression which is kind
of a longer expression but this is a this is the code of this function
nested tuple I have a double nested tuple whatever and that\textsf{'}s what
I have to return and the Scala compiler compiles this and so that\textsf{'}s
how I check that I haven't made some trivial mistakes so having compiled
this let us compile as certain having having derived this part let
us now derive this entire thing symbolically so again I wrote a bunch
of things in comments which is how I derived it so first step is to
simply rewrite this notation in scope so rewriting this notation means
you take F FFA first you apply this to F of F a and then you apply
this to the result so in other words in scholar you would apply this
to the result of applying this to F F F a which is written over here
so I just copied it over here so this is now after inside of flatten
and after that we substitute the definition of flatten from definition
of flatten was up there let me look at it again it\textsf{'}s right here so
now I need to take that and apply that definition so this one one
for example it would be just this because the first this is the first
tuple and in the first tuple I take the first part so that is if F
a 1 1 so this whole thing 1 1 it is this and so so I just apply these
things so I extract two parts from here which is immediately possible
notice here I didn't do anything with this F GM inside the map I just
keep it why I can't do anything with it right now I can't simplify
anything in this part of the expression it\textsf{'}s under map so I've no
idea what it is acting on and so I don't know what that is I cannot
simplify any more but later I will be able to simplify maybe so I'll
just keep it like that all right now we got this so this look yeah
I still have this map a flattened with no simplification however now
we can simplify because we have dot map of something don't map or
something else we can combine the two maps and that will be like that
so I just keep the first things unchanged and here I combine first
I take flatten of some things I write out this function as FFA goes
to something if a faith first goes to flight another for a and then
I take the first element of that so however now we can simplify this
flattened followed by taking the first element is just this remember
the cone for flattening respects so the first element of flatten is
this so then I can simplify and this is my result I could simplify
this further I like that at the risk of getting less readable but
we're not going to read this much more hopefully we'll get the second
function now and have the same code and won't be done let\textsf{'}s see how
that goes all right so the second function we need to compute is this
this is the right-hand side of that of the associativity law let\textsf{'}s
do the same thing again so we apply this to some arbitrary fffe and
notice that the type of the return of this function is f so flatten
flatten takes a triple F and flatten this twice and returns a single
nothing here map flatten flatten did the same thing by first concatenate
in the inner layers and this first thing getting into the outer layers
so that\textsf{'}s such a DVD law that\textsf{'}s combining triple F into F doesn't
depend on the order in which you combine the layers of it all right
so let\textsf{'}s continue the first step is to substitute the definition of
flatten in here and so now we have flattened off this let\textsf{'}s again
substitute the definition of flattening now for the outer flat so
we take this 1 1 and so on so once we figure that out it\textsf{'}s going to
be this I'm copying 1 1 and then again the same to map 1 remember
them right there the code is take this FFA 1 1 and then take it to
map one that\textsf{'}s good I'm just being very careful to make no mistakes
so I'm going slowly copying twice now let\textsf{'}s simplify right this is
a tuple we can compute what it means to have one one of it it\textsf{'}s just
this so then we take not we obtaining this so now we have again a
situation with map map we can combine them into a single map which
is less and voila we have the same exact code of the functions we
just look at this code and we see they were the same expression so
this shows the associativity law for the same unit why can it not
be a full moment but mostly for the reason that you can't have a woman
away if you discard information but if you discard one of the arguments
in time but more formally I would say for a full minute we need to
define purer and purer must have this type how do you define it how
do you get a value of G of a for an arbitrary factor G you can't there\textsf{'}s
no way of doing that now if you now say well maybe G is not an arbitrary
factor and maybe there is a function from A to G of a well that is
already suspicious because then maybe G is already itself able nod
or something like that so for arbitrary function G certainly I won't
work we will have a construction three next which is similar work
where we have two minutes a tuple of two morons so let us now go to
example three and I will show that a tuple of two monads or semi womens
is again Amanat or same unit so here I'm preparing the type aliases
first now it\textsf{'}s going to be quite both because I have two arbitrary
Hunter\textsf{'}s G and H and so how can I do anything with them well let me
just do this for brevity and define this notation now notice we have
G of a tuple of G of H of a so this is getting quite complicated each
of a topology of a each way so we have G inside GG inside H and H
inside G and H inside H which is kind of complicated so how can we
define a munna instance products so I'm not going to define flatmap
because it\textsf{'}s much easier to define flat I did define flat map in the
previous example but then I have to define flatten and so why don't
I just start with flatten it\textsf{'}s easier now how do you define flatten
now we have to somehow convert this into this in imagine that we already
know how to convert G of G of a into G of a because G is a unit in
th a very into HIV because H is a moment or centric paalsamy mood
let\textsf{'}s say when we have more than this we have G of two po G of H of
a so there\textsf{'}s an H inside of G how could we flatten that into G H is
an arbitrary type constructor well it is a 7 1 AD or a Mona but it
there are so many different type constructors that fit that description
as we have already seen it seems there\textsf{'}s nothing else we could do
here except to discard the H inside of G and to discard the G inside
of H discarding it is easy because they're factors so we can f map
or map this value with a function that takes the second part of the
tuple or the first part of the tuple for this I can just map like
this and that will discard those parts that we don't want so it seems
this is the only way we can solve this problem and implement for it
let\textsf{'}s do it so that\textsf{'}s going to be the first part map one flattened
so this flatten is M G and the second part mapped to flatten that
flattened image now we have interestingly here a map followed by flattened
so here map fold by flatten this map is also in the factor H as this
flatten this map is a new function G as this flatten therefore we
can combine map and flatten into a flat map and we can write somewhat
shorter code for this flatten function notice this flat map is in
the first element of the tuple which is the function G so this is
the flat map of the semi 1fg whereas this is the flat map of the 7-1
at H there are two different flat Maps really being used here to define
this all right let\textsf{'}s go 125 pure that\textsf{'}s much easier we just take the
pure of the Monon G and the pure of the Monad age and combine them
in an inter tuple we're done with writing a code for the unit now
let\textsf{'}s prove the laws to prove the most women would need F map so what\textsf{'}s
the F map was just a tuple of two F knobs so here\textsf{'}s a little bit of
intuition about what the small knob will do so we have combined two
more hands into one what does it give us so here\textsf{'}s an example so actually
if you look at the flattening you see only g and g are combined only
H and H are combined so the cross terms so to speak H of G \& G of
age are just ignored so that suggests if we do if we write code like
this then there will be no interaction between the two parts of the
tuple so this will interact with this and this won't work with that
so the result would be exactly the same as if we just split it into
two and wrote two pieces of code separately like this and then combine
the results in a tuple like that so it\textsf{'}s just exactly the same so
the result would be for instance here if G is a monad H is a monad
then you would just perform the for yield block separately for G and
H and combine the results in a tuple but you don't have to write it
separately you can rank it when this is shorter than the writing news
so let\textsf{'}s go on to the laws the identity laws are easier to verify
because the code is simpler so let\textsf{'}s start with those two identity
laws the left identity and the right identity now these are at least
two laws so the first law means that a composition of these two functions
is identity so let\textsf{'}s define this composition as a function called
pure flatten so again I have to write all this parameter stuff the
type of this function is f - f so ya go back to the slides and look
at the clause just to see that we have done it right so pure followed
by flooding is identity services this so the type is si - si when
our notation is FA - Fe and the right identity is also half a tuna
Fame so that\textsf{'}s what we do okay so flatten of pure of FA this is what
it means first we apply pure to some arbitrary FA of this type and
then we apply flatten to the result let\textsf{'}s substitute the definition
of pure which is this now we have plot flatten as the substitute the
definition of flatten which is written here this is the definition
so we just substitute it in there and we take the first one flat map
one second one flat map cube now just not much we can do with this
expression anymore unless we remember that this is and this are from
one add H and this and this are from onaji and these monads must already
satisfy the same law which is that pure followed by flat map of some
F is the same as applying F to this X so let\textsf{'}s use that law and that
means we need to apply this function to this FA which means FA dot
on your square one and similarly here so the result is this now if
a is a tuple so if you have a tuple of which you take the first part
and the second part and then put them back together you get the same
tuples you started with so therefore this is just identity function
and in this way we've showed that the first law holds so let\textsf{'}s look
at the second law similarly with good definition of what it means
to have flat map viewer followed by flattened so you take some arbitrary
FFA and you first apply F map of pure of it and then you apply flatten
to the result so let\textsf{'}s substitute what it means so basically F map
works component by component on a tuple so you need to do FFA one
in a pure if you fail to not pure then you substitute the definition
of flatten which gives you a flat map one and flat map two on each
release alright so now we have this situation we have map flatmap
so this is in the factor G and this is in the function H so we can
use a natural T laws for these two factors which we assumed already
hold and if you look up what the naturality lawyers it basically can
interchange the order of map and flatmap that\textsf{'}s what we do so instead
of this this is a naturality law map f flat map g gives you a flat
map F and then G so if we do this then you get for example flat map
of pure and then this so we have such expressions but if you look
at the definition of pure it\textsf{'}s one a deep your coma will not H pure
and so the first part of this is 1 a G pure so then we can simplify
that the result is going to be this again we wouldn't be able to simplify
here but any further except that we notice this flat map is from the
moon LG and so we have a flat map acting on the pure from the same
unit we can use the identity law for that one ad which means that
this is identity function and this is for the moon and H the identity
function and so the result is going to be F FA 1 and then identity
function so we can just delete that get if FA 1 if you fatal and that\textsf{'}s
exactly identical to just F FA something I just something I just noticed
is that the type here is FF and it should have been F so why did everything
compile because this FF type actually is just more restrictive than
F so we should have just changed this to F and things still compile
now I can rename this for clarity so that it\textsf{'}s going to be consistent
so in this way we have symbolically verified the identity laws now
let\textsf{'}s verify dissociative eg law and this is going to be a similar
exercise and substituting functional definitions and simplifying now
notice that we have used the moolaade laws for G and H to derive the
identity laws for the construction F topology image but if G and H
are only semi moments and not for women then we cannot use that but
so we won't be able to derive pure either we won't be able to derive
the laws for the construction so if these are just semi monads then
all the proof of we have done so far does not apply and we can only
prove the same unit for the resulting construction on the other hand
in that proof we don't need have a full moon of the instances for
G and H we only need a semi moon had laws and semi one-eyed functions
where I'm going to use paper for G and H and so if G and H are semi
bonnets then the result will be similar and we are going to prove
that now so if we prove that associativity law that\textsf{'}s what we will
prove buts anymore let\textsf{'}s give us a 1 have you film or not give you
a full minute and and this is because in this proof now we're not
going to use the pure functions from G and H we're not going to use
anything but associativity laws for G H so let\textsf{'}s see how that goes
again we have to do two things we have to compare that these two functions
and show that they have the same code so let\textsf{'}s begin with this function
so flat and followed by flatten I have written out the types for flattened
type parameters just for clarity now let\textsf{'}s apply this function to
an arbitrary FFF a of this type so that means we first apply flatten
to this function and then we again apply flattened to the result so
flatten is this so we apply flatten to the result and there is an
final expression is this so applying flatten to something means we
have a tuple with underscore 1 5 I have a new square one go to five
manners go to this twice so that\textsf{'}s what you get and let\textsf{'}s just leave
it like this we could simplify this because there is a composition
of two flat maps but it\textsf{'}s not clear that this will help so let\textsf{'}s keep
it here the second function is the F map of flat and followed by flatten
so that means we have F map of flatten applying it to F F F a and
then applying flat into the result that\textsf{'}s how it is we again do the
same so applying to F F F a you get the first part of it if FA mapping
flatten so that\textsf{'}s I'm just substituting the definition of F map which
is up here which is you take the first element of the tuple apply
the map and you take the first second element of the tuple apply the
map now these are two different maps this is the map from the function
G and this is the map from the function H so that\textsf{'}s that\textsf{'}s what you
have to do so that\textsf{'}s what is here now first not pure I'm sorry I'm
looking in their own thing yeah first my flat and second one now let\textsf{'}s
substitute a definition of the Eldar flattened here which will add
a flat map one to each part of a tuple in flat magnitude to the second
part of the tuple now let\textsf{'}s look at this we can simplify actually
because this is a map in the Hunter G this is a flat map in the function
G and there\textsf{'}s a naturality law for flat map so that war holds even
for 7-1 ology because it\textsf{'}s a largest involving flat map we're not
using pure 4G anywhere so now we can exchange like this using a chirality
we do that and we get those things that we've done before so flattened
and then take first flatten and then take second so we do that simplify
to first flat map one so now we have this expression flat map of first
and flat map one flat map of second infinitum so now we have a flat
map of something with flat map inside so that can be you know if you
compare with what we have here we don't have your flat map inside
flat but there\textsf{'}s social DVT law for the moon ads if you look at the
law for flat map that\textsf{'}s exactly what it does it tells you that you
can put flatten up inside flat map or you can put it outside and the
result is the same so you can simplify it like this if you have flat
man inside flat map and you can just simply find like this and if
you do that again that so now this clearly one two oh eight nine two
two three or identical so since they're the same expression then you
social diva team or holes so the next construction is oh heavens haven't
shown this so this is generally not saying well not even if G and
H are semi magnets and the reason is remember how we discarded across
terms the G of H and H of G well you cannot do that with a disjunction
you can do that with a product but not with a son because if you want
to implement flatten for this then you would have to combine g of
g plus h plus h of g plus h into G Plus H G of G Plus H could just
be G of H all the way so there could be no G of G because it\textsf{'}s a disjunction
so it could be either one or the other so G of G of H sorry G of G
Plus H is possible to be just of type G of 0 plus h whatever the G
of H and you cannot flatten that in general so that means this construction
does not exist for disjunction of two monads to disjunction of two
more lines are generally not known are not even a semi none of them
easy the next construction is that you take a fixed type car and you
take a function from a fixed type R into a semi one on G or a monarchy
I will not give a proof for this I will leave this as an exercise
I will give proofs for most other constructions and especially from
the last construction you will see a similar kind of thing but as
I leave for you to prove so let us now consider a construction five
construction five is an interesting one not often seen basically it\textsf{'}s
making a new monad by disjunction with the type a itself I've seen
we've seen that this is generally not a monolith you have here G and
H but if one of them is just the identity movement then you can do
it it turns out you need however a fro moment for a for G does not
work with a semi mu naught so this construction is sometimes called
a free pointed factor over G it\textsf{'}s not important why it is called like
this at this point but also later when we later in the tutorial when
we talk about free constructions such as free factor or free will
nod this will be one of those free constructions for you construct
a new factor with another with a new property out of a factor that
doesn't have this property now as I said for this construction G must
actually be a full monad so it must be pointed when remind you would
point it means for a functor pointed me in the natural transformation
from identity in other words a natural transformation of type A to
G of a which is the type of pure so if a founder has a function with
the same signature as pure that what it means that the function is
pointed so what\textsf{'}s however these are theoretical considerations let\textsf{'}s
go to the code which shows how to implement the model instance for
this and to prove that it is a lawful minute so here I'll again prepare
my type our oasis for brevity the type F is just an either of a and
G of a and F F is just F of F and so now how do we define in the monad
or seven wounded instance all the easiest ways to define flatten in
order to define flatten we need to take this kind of value and return
this so this is an F of f of a written out info and this is an F of
a so how can we transform this into this given a disjunction means
where we could be given any part of it we could be given just this
or just this or this and inside could be just this or just this or
some combination because this G is an arbitrary monad so it could
be a function a container having several values of this type so this
could be one value up on the left one value on the right and so on
so this could be complicated so how do you return a plus G of any
out of this now if you are given this then you can just return the
same with no change that is easy but what if you're given this part
of the disjunction how do you extract a plus G away out of that now
what seems to be a little difficult and the trick is that actually
there is a function you can define which has this type which I call
marriage I I don't attach a lot of importance to the name of this
function marriage it\textsf{'}s just for convenience let\textsf{'}s call it marriage
and this function can be defined to define right here I'll look at
it in a second given that this function is defined we can map this
function over to the factor G and lift it into G the result would
be a function from G of this into G of G of a so we have this function
and that function is what we need to transform this and if we were
given this part of the disjunction into G of G of a now we can flatten
the G of G away into G of X and G is a mu naught and that\textsf{'}s what is
going to give us G of a and we can return the right part of the disjunction
and we're done how do we define the verge function now that\textsf{'}s a little
interesting if we're given the F of G available means were given this
we're either given a or not given G of a you need to return G of Allah
forgiving a we use pure from the G when we return the uvula and if
we're given G away we just return that so in defining this we use
the fact that pure is already defined for the modern G this wouldn't
be possible if G were a semi limit so the code for flatten follows
more or less straightforwardly then flatten on the left of this this
is FA so left of FA just returns that FA which is this type and the
right of some GF it again I'm using variable names that conform to
their type you see the type of G of F of V so if you have this then
we would have mapped with the merge and then flattened which is flat
map with lunch so we use that that\textsf{'}s how it works so this flat map
is giving you a G of B and you put that G of B in this case of GMA
you put it in into the right part of the disjunction here so this
disjunction is returned so we're done so this is the definition of
flatten for the construction definition of pure is very easy because
if you have an any just return the left of that a you don't actually
use the pure of G to define this but as I have written here in a comment
the Monad laws actually actually won't hold unless G is a full moon
ad and we have used pure to define flatten here so we have used people
of G already and that needs to satisfy the laws the F matrix standard
just a slab for disjunction if you have a left apply F right apply
F after now now just just a remark here we have been able to define
pure without using the pure function from G and this is why this construction
is called free pointed we were able to define a point for a pure function
which is where old point in some libraries without so suffer the and
constructed new frontier for this filter we're able to define the
point or pure function without using the point or pure function on
this and so that\textsf{'}s that\textsf{'}s why it\textsf{'}s called free point it\textsf{'}s a we wait
we have information enough to construct the point function without
already having given having been given this function before in the
G now let\textsf{'}s verify the laws i this is rather reasonably simple the
first law and the second law for pure these let\textsf{'}s begin with them
so first we have a pure and then apply flatten so take some arbitrary
FA applied pure to it and then apply flattened to the result now pure
over failure is just left over fade that\textsf{'}s the definition of pure
now we flatten the left when you flatten the left it just gives you
the value under the left therefore flatten of left of FA is just FA
and that\textsf{'}s the identity function as required now flat map of pure
sorry F map of pure followed by flatten that also has to be identity
now let\textsf{'}s do a F map of pure and applying it to some arbitrary FFA
I think I'm making again the same mistake as before let me check yes
this type must be F to F naught F F F F and this I also should ever
need so now this must be of type G yes good all right so let\textsf{'}s go
through this derivation first we substitute the definition of F map
which is this where instead of the function f we use pure so we write
this code but we put pure instead of F so that gives us this code
now we can substitute the definition of outer flattened which is if
you have a left than you so you see if flatten is applied to the result
of this which is either this or that so we take these things and apply
flatten to this which will be that and we apply flatten to this which
will be that because the flatten of the right is a flat map merge
all right so now we have this code now we need to simplify well what
can we simplify here not obvious well so they have this map and flatmap
and these are in the Mona G so Mona G must have its natural T law
for flat map so once you use that this law and we use that law with
F being equal to pure and the result is going to be flat map of F
and then G so pure and energy so pure and then merge is the same as
pure now this pure is our pure that we are defining for F is not the
pure virgin I always write explicitly this for the pure for G because
it is easy to get lost otherwise so pure is not for for G\textsf{'}s for F
now if we substitute what it does pure and then merge so pure gives
you any left of something emerge of the left is a monad G pure of
the content of the left so that means we have a monad G pure so pure
and then merge is the same as moment jean-pierre so let\textsf{'}s substitute
it in there so now we have this code we have a flat map of cannot
G pure and this flat map is in the mono G because I'm reminded by
the name of this variable and I can check this also with IntelliJ
was I'm sorry can you check it because it\textsf{'}s because it\textsf{'}s in the comment
so I can check it here so the this is of type when this I should have
said I should've called this da for less confusion but it\textsf{'}s just names
and variables doesn't even matter what they are but it\textsf{'}s just helping
to see what types they are so flat map is in the Mona G and therefore
we have a low flat map of pure his identity so therefore we have this
that is obviously identity so this verifies the identity laws for
a woman if you know let\textsf{'}s look at the associativity law oh yeah and
I never mentioned any naturality laws for monad if we don't check
them because the code already guarantee is not reality there is nothing
in the code that uses specific types such as a being int or string
or anything like this the code is a composition of functions it is
substituting functions into arguments that is completely natural in
the sense of natural transformation so any code like that is fully
parametric it uses functions such as swag map which are already natural
transformations by assumption as they are in the moment G so compositions
of natural transformations are any kind of use of natural transformations
will be natural so we don't need to check much rather many of this
code any of these constructions are automatically satisfying when
chirality laws but associative eg1 needs to be checked so go on to
check that law to check that law we need to compare this so flatten
off flatten and this flat and F map of flatten begin with flatten
of flatten so we apply flat on a flattened to cell FFF a so first
we substitute the definition of flatten which is this and then we
have the outer flatten on that substitute again the definition of
outer flatten which means that on the left hand sides stay but the
right hand sides get a flatten applied to them with me this is like
that so left FFA goes to flatten of the FFA and write G FFA goes into
this I can't simplify this anymore well I could like a sec before
get another flat map the flat map that could be simplified into a
flat map of large and flat map of merge I'm not sure that will help
any right now so I'll keep this code as it is and look at the other
one maybe we'll see what we need to simplify if we need to so the
other code first so I define this function which is I don't do it
now by steps this first do it right away so this function is going
to compute that so it\textsf{'}s just F map of flattened applied to an arbitrary
FFA and then we apply flatten to that so first we substitute the F
map of flatten of F F F F which is this this is a definition of F
map where we put flatten instead of the function f so now we have
this flatten and also this flatten inside go on substitute the outer
flat so that means we apply flatten to these right hand side parts
so this will apply flatten to this and we apply flatten to that so
flatten of the left something is just that something so that inner
flatten still remains here see I put type parameters explicitly so
let\textsf{'}s color compiles because I remove this it\textsf{'}s read interesting right
the types are correct and just that Scala cannot infer that your guess
would be Piper it must be and this is often the case also here I had
to put it in now notice this outer flap has been substituted already
this is the inner flap in that still remains and this is also the
inner flap them that still remains so now if we compare these to the
left case is already identical so we don't have to worry about it
anymore the right case is not yet identical it has this versus this
so let\textsf{'}s continue we need to show that these two are equal it\textsf{'}s easier
if we just stop writing all this code and start just reading one next
smaller expression at a time so in the last expression we have a map
and a flat map from the factor G so let\textsf{'}s combine them using the natural
to look for G and then we get this now let\textsf{'}s substitute the definition
of flatten and keep merge as it is we have merge applied to the result
of flatten so see this is just the code of flatten if you get rid
of merge here then it\textsf{'}s just going to be the destination of one so
now we can substitute the definition of merge on the left so merge
over Roger Bobb this weekend up simplify but we can simplify it on
on the on the right so merge of the right let\textsf{'}s look at again a definition
of marriage what it is mergers right is just the content of that right
good so it\textsf{'}s just going to give you this so that\textsf{'}s the simplification
we can do so now let\textsf{'}s compare so we were supposed to compare these
two functions we have simplified the second one and we got this expression
so let\textsf{'}s compare so now it seems to have a flat map on here of some
larger function and here we have two flat maps of merge in order to
compare them would be easy to merge these two flat maps together using
associative et law or combine them together here and if we do that
we will then have to compare GFF a flat map of blah with Jake if a
flat map of something else and that would be a direct comparison we
can drop jmf a flat map and compare those things inside so do that
basically we need to compare the two functions inside flat map inside
this construction and this is fine because G FFA is an arbitrary function
sorry an arbitrary value and so if we show that these are the same
and obviously we will have proved associativity long so then the result
so far is that we need to prove that these two functions are the same
these are the functions inside the flat map let\textsf{'}s take this function
substitute the definition of marriage and I put a flat map so if you
delete this flat map that would be just a definition of merge and
we have applied flat map on its result so far okay now we need to
compare this and this the right cases are identical the left cases
are not immediately identical but again we have here pure from model
G and the flat map of money on G how do we know it\textsf{'}s from one engine
because as a result of pure is a G of something so that\textsf{'}s going to
be a flat map also in G therefore we can use the identity law for
G and that\textsf{'}s just that function so that\textsf{'}s just going to be emerge
of a fade which is exactly this so when once you simplify this you
get this therefore both cases are identical so this finishes the proof
of the associativity law for the thunder F notice that we have used
both the identity law for G when we did this combination and we use
the associative law for G when we combine the two flat maps on the
G so G must be a full moon odd for associativity here to work he also
must be a full moon odd for identity lost to work but this shows that
unless G is a full moon that even associative et law would not call
for F so it wouldn't be an even a semi movement unless G is a full
moon that but if g is a film on that then f is also a full moon that
so therefore we have shown that this F is a full monad for G also
being a full moon and that\textsf{'}s the only construction we can show the
next construction is this one now it\textsf{'}s a G which is an arbitrary monad
applied to this type expression which is a kind of a linear function
of a type A with coefficients Z and W where Z is a fixed type arbitrary
type and W is M you know it so this is a kind of a straightforward
construction if you understand and will not transformers but I just
mentioned this for people who already know but if you don't know yet
then this is just a construction that can be shown to work so let\textsf{'}s
see why that construction works now I'm getting tired of writing up
all these type parameters all over the place like this so I'm going
to cut down on the boilerplate in the later part of this tutorial
and I'm going to just put these parameters up front so I'm going to
define all the code inside a function that already has these type
parameters and then inside I don't need to talk about these type parameters
this is just to calm down unavoidably all right so now for this construction
what we need is to apply some arbitrary Menagerie a type constructor
which is this either of Z and tuple of W and a and this allows us
to use a type alias now to define these things because g ZW and so
on are already defined as type parameters up here all right now actually
p is itself a monad and that will be an exercise line to show we have
seen parts of P so to speak we have seen this this is the writer muna
and we have seen that is either monad this is a combination of either
and writer and this is also honored mathematically speaking this is
Z plus W times a and so this is like a linear function of a which
is kind of a simple example of a moment all right now f map is defined
in the obvious way if you have a left of z you don't you don't change
it remains left of z if you have a right of tuple w a then w doesn't
change you transform a now we can write a cat\textsf{'}s monad instance if
we feel like it cats monad is my own typeclass that has flat map and
pure and that can automatically export your model into cats how did
you find Flattr Wallace is a little maybe come ok but it\textsf{'}s it\textsf{'}s very
straightforward for P you don't do anything with the left and then
you map on the right because on the right you have an a so you have
some W 1 a 1 you take f of that you get a P of B then you match that
if it\textsf{'}s a left I mean again you return the left if the right and you
now you have to double use but you combine using the semi group and
you have a a 2 which is actually a B so it should not be called a
2 but snow nameless you call this B to be more clear the pure function
is defined clearly us is a right of every team annoyed value of W
and they that you're given so that\textsf{'}s clear but just like the right
terminal and either o the flattened defined for P I'm writing this
code here because we will need it for reasoning so let\textsf{'}s look at how
it works so we have this type expression and we want to convert it
into this how do we do that well if we have a Z so see looking at
this type expression means that we can have a Z or we can have a W
times Z or we could have a W times W times a because this is just
like school algebra so you just expand parentheses and plus and times
symbols distribute so if you have a Z or if you have a W times Z you
cannot possibly return on a T so you must return you see here the
other case is that we have W times W times a so you can return an
A and you combine these two w\textsf{'}s so that\textsf{'}s what this chrome does were
consistent done so we have flattened and F map for P which I explicitly
defined as flatten P and F map BMI also you find a flat map for people
let\textsf{'}s not let\textsf{'}s see how we can define the flat map and flatten or
something for F now for G we already have flat map and everything
we just want names for them for convenience so I you find them here
again with these names I'm using already defined called from a functor
and the mall and typeclasses for G so that\textsf{'}s just for convenience
because I we need to reason about these F map and flatmap G and so
we want to have these functions but of course we can't reason about
these much because we don't know what these actually do this is an
arbitrary monad so we don't know what code is inside lease all we
know is the properties so we know that this is a lawful monad or semi
unit so it\textsf{'}s the same story again we will see that to prove associativity
we don't need the pure from G and we don't need the pure when we don't
need the laws of identity for G we only need the social tivity of
G and so if G is a semi wanna this will be Samuel F G is a film owner
than F will be also a full moon on the F map for F is defined by doing
a functor map on the function that takes a function P map because
that\textsf{'}s just a composition of two functions so f is defined as a composition
of G and P so it\textsf{'}s a composition of two maps map of that but let\textsf{'}s
write down the short notation which will be quite helpful so f map
f is f map G of F of F of F or we can write it like this which is
shorter and let\textsf{'}s see if we use that in reasoning in a certain way
so then we can define a function instance for g sr e 4f by composing
my factor instances and we can define pure for f so this is a pure
for F I should have probably called it a few F just so that we are
not confused let me do this so pure for F is defined in the usual
way we take the pure of G and we apply that to this which is a pure
of P right when you find pure of P in here so pure of F is just your
F equals purity followed by energy just like F map is like this curious
like this let\textsf{'}s see if this will help us reason about it so now the
interesting part comes I need to define flat for this functor so the
function f is G of P of a and so flatten needs to transform F of F
of a into F of a so f of F of a is this and we need to flatten it
into just one G and one P so we have two Gs and to peace and they
are interleaved so somehow we need to transform that and we could
transform this if we change the order here of p and g players if we
could transform that into this type then it would be easy you just
merge this flatten this into one G when you flatten that into one
P and you're done so what you need is to define this so-called sequencing
function which changes the order in the sequence of applying factors
so it takes P of G of whatever and returns G of P of that and this
function does not always exist for all kind of funk stars PMG you
won't be able to define in general such a function but this function
will exist for a specific factor T which is defined specifically by
this type expression so for this function key and front a few other
such factors this function will exist for the function changes the
order and let\textsf{'}s define it now how does it work so it\textsf{'}s supposed to
take a PGC and return this so well P is either the only thing we can
lose to match so what is the result of matching if we have a left
of Z then we should return a G of something well the only thing we
can do obviously is use the pure of G to return the left of Z because
there\textsf{'}s no no way for us to return any Ace or double use in this case
if we are in the right then we have a G actually and we have a W so
let\textsf{'}s see what we can do with this we can take this GC and map over
it and we can add the W to that see what\textsf{'}s inside it the result will
be a tuple of WC we could put it into the right and that will be of
type P of C and so now we have G of G of C exactly as we need so this
is a little involved but that is a very important function without
that kind of function was no hope to define this construction for
a composition of two funders being imminent so seek this function
seek is a transformation between functors P of G \& G of K and since
it\textsf{'}s it\textsf{'}s you only uses natural transformations and fully generic
code fully parametric type C is anything we don't use any information
about what C is then it\textsf{'}s a natural transformation we don't need to
verify the natural anymore and this is a naturality war the functor
on the left is PF Geneva function on the right as G of since of maps
must be in the first factor and F maps in the second factor that\textsf{'}s
a natural G law that exchanges the order of F map and your natural
transformation so I'm not going to verify this because of the permit
tricity theorem but it could be done there easily of course in the
same way as we verify all these other laws just write down the code
of F map PF map gene so one of you transform and substitute and simplify
and you get that these two functions have exactly the same source
code and so that\textsf{'}s just a waste of time to do that because the chromaticity
theorem guarantees naturality but you could do it so now let\textsf{'}s define
flatten for F we define it like this it was a monad G flat Maps it\textsf{'}s
a flat map from G which takes a function of complicated type so let\textsf{'}s
look carefully what it does we're supposed to do F of F of it now
F of F of a is actually G of P of G of P of a and if we flatmap with
something then something must be of type P of G of T of a going to
something so therefore the inner function must take P of G of P of
a and return something well what should it return well it should return
G of something because that\textsf{'}s what flat map does that map takes a
function from some X into some G of Y and the result will be G of
Y now the result must be this therefore this function this inner function
that we are not yet ready to write this function must take this type
and finally return that type so I'm showing how I got got this how
I derived it you see I'm completely looking at types and I'm not guessing
I'm just saying flat map must take that type and that\textsf{'}s what it should
be and it should return that type in order to return this now how
do we get from here to here obviously we use the seek which will interchange
p.m. G into G and P so that will be this and then we flatten P but
flattening of P must be done under a layer of G so therefore it is
F map G of flatten P and once we do that we're done so therefore the
type of this inner function must be this and G of flat map of that
type would be a function of this type exactly as we need take some
time to check this so that the types are correct a flat map the right
type parameters and if you once you have checked it it will see why
it must be like this this is flatmap must have that signature G of
X going to G of Y having a function X - G of Y as an argument and
so X is this why is this and then if everything works so finally this
is the code I'm just writing what I have seen what would I have done
here secret random f9g faculty so this little piece of syntax is necessary
for Scala {[}Music{]} for some reason I think it won't compile I had
trouble compiling it um without this but this is basically saying
that this is a function the seek is a function it\textsf{'}s unnecessary parenthesis
but let\textsf{'}s put them in for clarity all right now the short notation
for this function would be useful for reason is that I take flatmap
in the G mu not of this composition and you can use natural T for
flat map G to rewrite it like this so now actually we are in a very
good shape we don't need to write code because the short notation
works well enough now in the previous examples I didn't do this and
I wrote the code but let me try to avoid writing code I could do the
same as I did before just keep rewriting the code but let me see if
the short notation works maybe it will work on up to a point alright
so in order to verify the associativity of all we need to verify that
this is equal to this so let\textsf{'}s verify that F map F of flatten F followed
by flattened F so we just substitute this definition of F where F
which is this and then substitute the solution of flatten F which
is yes flat map G of seek followed by F map G of Kelantan P so I just
put it in here and put it in here and here I now look and try to simplify
these things so how do we actually simplify anything like this just
looks very complicated the way to do it is to try to understand what
parts of these things belong together to be simplified so for example
F map G and F a level G they can be manipulated because they're in
the same factor and there\textsf{'}s naturality law with interchanges flat
map and F map or flattened and F map it has a much reality law but
only in the same factors so flat map P does not have any natural ality
law with F map G so we have to pull these things together somehow
in order to in order to simplify anything so for example I want to
use natural 'ti of seek how do how can I use it the only way I can
use it is what I have F map P of F F of G of some F immediately before
seek and here I have what I have seek here and seek here nothing is
immediately before it so somehow I want to pull this thing together
so that it\textsf{'}s immediately before seek and then I can achieve maybe
some simplification using neutrality for seek so how do I pull that
together well I have this I have this thing F map of this so let\textsf{'}s
see if we can first use this naturality which is f map G of something
and FLL G of something else appears at higher level we have F map
G of all this and F LMG of this so how do we pull this together we
use this naturally G law and we apply it so we apply this law and
we get FLNG of a big thing so that\textsf{'}s the first step so inside is going
to be all this stuff without the second F of energy because the second
of all energy is going to be out so the result is that we have one
big FLNG so by combining this and this {[}Music{]} using parentheses
here somewhere just to be completely pedantic one two three two three
two one zero that\textsf{'}s right all right so now we have combined this F
map G and F LMG into one big ecologic the result is this FLNG of this
so now does this look like seek of and although to the left of seek
I have {[}Music{]} F map PF mataji of something does it look like
that to the left of seek I have F naught G of something but F map
P is over there it needs to be immediately here below to the left
of this so how do I get that well if map we can split so f map of
composition is f map of this followed by f map of that so we can split
that in the factor P the result is this F map p f mappy seek ethnology
great now this is what we wanted we wanted to pull together C and
F map P of F map G of whatever now we can pull this on the other side
of seek by using that reality of see so much relative seek is this
one and using this law we pull it and the other side of seek and also
it exchanges the order of F map J and P but that\textsf{'}s okay so the result
is this now that\textsf{'}s examine this we have F naught G F mataji let\textsf{'}s
pull them together maybe we get an F map G of this now this is a social
dignity law for B it\textsf{'}s equal to just flattened P followed by 20 people
so this is a social activity law for tipping the result is their fullness
now let\textsf{'}s pull it apart and we get F map G under F L M G so then we
can pull that out maybe it\textsf{'}s not clear that this will help but let\textsf{'}s
just see if that helps because we still haven't seen the other expression
so maybe the other expression will be similar to this all right so
at this point there is nothing else we could usefully simplify let\textsf{'}s
look at the other expression the expression is lists so we substitute
the definition of flatten F and we get this so now how can we compare
this code and this code well both of them end in the same way that\textsf{'}s
good but this part is one big flat map and this part is so I have
F map flatmap so we need to combine this into one big flat point to
do that by using naturality and associativity of G we can do it so
naturally allows me to pull this inside the flat map which gives me
that associativity allows me to pull the second flat map inside the
first part map which he gives me this so the result is one big flat
map with this followed by this piece which is the same as that so
don't want to compare you could say except for what\textsf{'}s inside the big
flat maps and both of these functions have the same type and look
at the types yeah quite complicated but I just want to write down
what we had here F may have seek holiness here after my PF a flat
map G of seek followed by seek followed by this I just write down
Scala code for this in order to make sure that everything is still
like checked and correct so this is that now a short remark here about
rotation all these computations I've done here I've never mentioned
any types as if the types will always match whatever I do why is this
why don't I need to check at every step that all these types are correct
no the reason is that these functions are polymorphic they will adapt
types to each other as long as types can match they will adapt as
necessary each of our laws is a fully polymorphic fully parametric
type function and so if the argument type ins to be changed it can
be adapted automatically if will never leave their type mismatch whatever
you substitute if a function is equal to another is substitute maybe
the type needs to be adapted but it can be always adapted because
there\textsf{'}s always some more general type for which the laws hold which
is the way we usually write them so for example for this law this
goes from F of F of F of a to f of a but a is completely arbitrary
so as long as you put that into any expression a might adapt to something
and because it\textsf{'}s an arbitrary type type but adapting won't ever break
any types so the laws hold for the more general types compatible with
these function signatures and that\textsf{'}s why we don't actually need to
check any types while we do this kind of manipulation so now let\textsf{'}s
see now these two functions is not obvious that they are the same
there are quite different so first of all it starts with seek with
a different type parameter and this is the seek under F met P FLNG
it\textsf{'}s completely unclear whether this is the same and it\textsf{'}s probably
not the same as this you follow it by flanton P under F map G and
here the flatten PF my G is at the end and so it\textsf{'}s not obviously these
functions are the same so let\textsf{'}s maybe write the code for these functions
actually we're maybe trying to evaluate them on some on some values
because it\textsf{'}s impossible to simplify these expressions further in order
to to show that they're equal simplify what I mean is that it\textsf{'}s impossible
to simplify by using laws of flat happens fmg symbolically like this
and not actually getting inside the code of seek for example so these
functions are equal but only for the specific code of seek that we
defined so that\textsf{'}s why we need to now go a little deeper into the code
so the arguments of these functions are P of something because of
that the argument can be either left of Z or a right of that so let\textsf{'}s
apply these two functions to a left of Z and then we apply to these
functions to a right of this and see what happens so if say X is left
of Z if we substitute the definition of seek then seek of left of
Z where it is assumed to be of of that type of this type so Z left
of Z is still of this type then the sick of it will be by definition
of sick if you substitute into the code it will be this and flatten
of it is just left of Z so you substitute the flatten P and again
that and so if we try to evaluate F 1 which is sick followed by F
map G or flatten P and followed by flat map G of Z so what happens
first we do seek so we get a pure then we apply flat map so we apply
a F map G of flatten so inside this pure applying F map means applying
the function to this so we apply flatten to this we get left and then
we do a flat map of seek on pure and F map of seek on pure is just
sick applied to this left of Z which is this so sick of left of Z
is ma not be pure of left of z f2 on the other hand gives me a trivial
result because this F map doesn't change anything because I have a
left of Z so f map P is identity on that so I have left of Z then
I have seek which leaves me sick of left of Z which is this and then
I have f map G again a flat of P which is the same as what we'll just
complete it and so it\textsf{'}s both f1 and f2 evaluated on left of Z will
give me the same result so that is the first case now let\textsf{'}s look at
the second case that\textsf{'}s going to be longer if X is the right of this
then we need to compute think of it which is going to be that after
seek we compute F map G and that means we are we're already inside
map in the G function so we just add another map in the G factor so
we do that the result is we can combine the two maps into one and
first we apply this to some PG and then we put a flat mu P on top
of that so that\textsf{'}s the function we get therefore {[}Music{]} finally
f1 so if one is seek than this and then flat map of seek so add that
we can put in a map and flatmap by that\textsf{'}s reality because they're
both in the g-factor and when we get seek applied to that that\textsf{'}s the
result now this kind of thing is the result of F 1 of X now for F
2 of X we first compute a flat flat map G of seek under F map P which
means that well for the functor key F map Maps the right and it leaves
W unchanged that map\textsf{'}s the value of x a so then we get this and then
we do a flat map of seek on that finally we apply a seek so let\textsf{'}s
go back to definition yeah so we have done this so far we apply seek
to that then we have to apply this so it\textsf{'}s saying that we apply over
here okay ply seek to that we and get this finally we apply F map
of level B to this then we just tack on document flooding because
this is a all these flat maps maps and maps they're all all analogy
functor so let\textsf{'}s pull out into one flat map using naturality begin
this so now the result is we have F 2 and F 1 of X computer F 1 of
X is this F 2 of X is is that both of them have the foreign GPG dot
flat map of dois so we need to compare just those functions this functional
body let\textsf{'}s compare them one is this and the other is that now it\textsf{'}s
kind of hard to compare this because PGA is arbitrary W is arbitrary
what can we do so what\textsf{'}s pattern match on PGA and well I should have
written code maybe but I always let\textsf{'}s just substitute values and see
if it\textsf{'}s working better so PGA is either left Z or is the right of
this with some W 2 and G if it\textsf{'}s a left z then let\textsf{'}s look at let\textsf{'}s
look at that so flatten of this would give you a left z but can't
give you much else you couldn't possibly give you a right of anything
because it doesn't for the right you need you know W so you have a
lefty so sequins LLC is pure G of Z and so we have pure J of MZ as
a result for f2 we have so we'll look at this function where PGA is
left z c curve left is pair G and then pure G of left GU map whatever
you want to map the Z is not going to go anywhere it\textsf{'}s no it\textsf{'}s not
going to change so the result is this so therefore for the left these
two expressions are the same so now it remains to compare these two
expressions if PGA is right of something so we compute that and the
result is that here we have the flattened p of the right of this and
here we have to combine W and W tube using a semigroup operation the
sick of that is going to be equal to this and for F 2 we perform similar
computation and we basically again have the same expression because
the flatten P works like that and that finishes our proof so I tried
to make it shorter so part of the way I was just doing symbolic computation
at the level of factors and units with no specific code so this would
be general but after this point I could not continue the fully general
computation I had to put in a specific code for seek and flatmap flatten
P and so that\textsf{'}s at this point maybe it will be sure to just start
writing code and compare the code for these two functions but I found
that if I first substituted left it was very quick and so onion it
this way the next construction is this one it\textsf{'}s a very important construction
because it\textsf{'}s recursive function f is defined recursively and it gives
you a monad for any functor G a and G does not have to be a melody
itself so this is the only construction here where get a moon ad out
of a arbitrary factor rather than out of an arbitrary another one
odd but because you get a moon ad out of an arbitrary factor so to
speak for free remember it\textsf{'}s called a free monad over G I mean mentioned
that there are constructions called free constructions that give you
properties for free well for free means you don't have these properties
in the data that you are given and you're just creating them in some
way out of the structure of new functor but in a later tutorial I
will explain the free constructions in a more detailed presentation
so for now it\textsf{'}s just a name for this construction so let\textsf{'}s see how
it works now we have seen this construction in the examples and this
was the G shaped tree functor so the G shaped tree in other words
a tree with G shaped branches that\textsf{'}s exactly this construction so
the tree has either a leaf with a single value of type A or it has
branches inside a functor G so the function G distribution describes
the shape of the branches and under it each branch there\textsf{'}s another
tree again of the same recursive shape so let\textsf{'}s see now how this how
this works we assume an arbitrary function G not Superman I said yes
so it\textsf{'}s any function G wasn't this functor doesn't have to be itself
Amana I define this function construction seven just so that I have
fewer things to type now we can't define the type alias for this because
it\textsf{'}s recursive so I have to define a case class with a type parameter
and that forces me to have a name for this inner part but that\textsf{'}s not
going to be used much so this is just a plus G of f of a how do we
define flatten and F map which thing as before it\textsf{'}s the tree the G
shaped tree so f map on the leaf it\textsf{'}s just a transformed leaf F map
on a branch is the same F map recursively applied to each value in
the branch container so G works as a container but may have one or
more values and that\textsf{'}s the branching if it has more than one value
then you have several branches the flattened works by keeping leaves
as they are so if you have a tree of trees and the leaf of the tree
means you have just a single tree in return that tree if you have
a branch then you need a map of flatten so this recursive case is
going to be the same for all these of these constructions this code
cannot be written otherwise they have to map over this functor with
a recursive call to the same function pure is defined by just returning
a tree having a single leaf that carries that given value let us verify
the laws this is the identity second identity law we just substitute
the definitions again so to verify this law we need to take some arbitrary
FA apply pure to it and then apply flatten to that result so let\textsf{'}s
flatten of pure of F a substitute the definition of pure will get
F of left of F a substitute definition of flatten what\textsf{'}s FA it\textsf{'}s against
identity and let\textsf{'}s compute F F map pure of flattened but actually
it will be may be helpful to compute flatten the F map of something
followed by flatten is flat map where definition so let\textsf{'}s compute
that function first flat map so how do you do that so it\textsf{'}s flattened
of F map of F of some arbitrary FC what\textsf{'}s the definition of F map
is this matching with recursive cursive match and when we substitute
the definition of flatten which means that we apply flatten to these
right hand sides of this match so this remains the same the right
hand side is flattened flattening this gives you f of C flattening
this gives you that and we can replace F map don't flatten because
this is map of this dot map of this which is equal to dot map of this
followed by this and we can just replace that again by a full line
because that\textsf{'}s the definition of you follow so that\textsf{'}s going to be
the code for a fella so again just your cursive keys is always the
same we just use different functions of your map that\textsf{'}s always the
same alright so what is now a full MP rifle unpure is this f LM Puran
must be in apply this to arbitrary F a substitute the definition of
a for Lam which is this put P you are instead of F and so then again
this now pure of C is f of left of C by definition so this first case
is just identity by definition of pure the right case is this now
if we have proved that it is identity on the leaf now we have a recursive
case now we cannot directly prove that it is identity because we don't
know what that is we haven't yet shown that the right case gives also
identity so we need to prove this by induction induction is on the
structure of the tree if we are in the leaf we have proved that it
is identity if we are in the branches and for each branch we've already
proved that this is an identity then we need to prove that it\textsf{'}s identity
here in other words we need to we can assume that the is identity
when we recursively call that function once we assume that then it\textsf{'}s
obvious just map identity so this is f of right of GFC and so that\textsf{'}s
clearly identity so assuming inductive assumption which is that this
function is equal to identity on any of the sub-trees we can prove
that it\textsf{'}s equal to identity on the whole tree that\textsf{'}s how all these
proofs are going to go for recursive functions that\textsf{'}s the only thing
you can do you assume that the recursive calls to your function already
satisfy the property that you want and then you prove that\textsf{'}s the inactive
assumption then you prove the step of the induction alright so this
proves the identity laws let\textsf{'}s now prove the associativity law so
she'll give it a law which is this you know let\textsf{'}s write down the code
for these functions apply this function to an arbitrary FFF a of this
type substitute the definitions we get this so the left case is that
the right case is kind of complicated so we can substitute this into
into the first map so that will be flat and followed by flat which
is the same as this function so now we can therefore write the code
like this so you see all of these functions are always going to have
the right case using exactly the same code so that\textsf{'}s kind of a boilerplate
isn't it there is a way of getting rid of us but it\textsf{'}s more advanced
these are called recursion skills and I will talk about this in a
different tutorial but for now we will just keep writing this boilerplate
each function will have a second case of this of this sort now flat
map flatten is the second function we need to compare with this one
we have substituted again flat map instead of this composition so
flat map of flatten applied to an arbitrary FFF a what\textsf{'}s substituted
definition a flat map which is this and then we get if we rename FFA
to instead of C then we get the code that\textsf{'}s exactly the same as this
except for the name of the function and the recursive call is of course
to the function name differently so if we rename the function rather
than we can't distinguish the two functions in the world so it means
that their code is identical so this shows the associativity we have
not used any properties of G other than map so all we have here is
this map from G we have not used anything else we have concatenated
the two maps into a single map that\textsf{'}s a property of a functor and
the composition law we have not used any other laws for G and so therefore
indeed we have produced a monad out of an arbitrary function G the
next two constructions are in general only semi winnette constructions
so they do not yield full o'Nuts the first such construction is the
G shaped leave and D shaped branch tree which we have seen before
so leaves are of shape G and the branches are also a shape G that\textsf{'}s
a tree like this but we'll see it cannot be made a film or not only
a Samana so what\textsf{'}s looking this construction will show that it is
not a Mona so how well let\textsf{'}s define first so we define again a case
class with type parameter it\textsf{'}s a recursive type because it uses F
inside itself so it\textsf{'}s on either leaf and so we have G for functor
G describing the leaf and the factor G describing the branches imagine
that G is a pair of a a then this will be a leaf consisting of two
values of n this will be two branches F of F of F consisting of two
new trees so that\textsf{'}s either two values or to introduce notice that
the shape of this tree so the flattened can be defined certainly and
by redistributing leaves in two branches so if you're on on the left
and you have these leaves then you just these are leaves so you have
G of F of a so this is of type G of F of a and that\textsf{'}s what you need
to return can return the right of that and you are in here you're
in the right part of the either and so you can just immediately return
that so essentially you're redistributing G leaves into G branches
you have this option because the tree has this shape the the right
case is just a recursive case doesn't do anything it just repeats
the same operation on the branches F map again same thing you map
over the leaves and you do the same operation on all the branches
now I will leave it to exercise 15 to show that this is associative
the proof is somewhat similar to associative 'ti of the ordinary train
but I will show that you cannot make this into a full minute here
if we wanted to do that and we defined pure well we need obviously
a methodology in which which makes you some valium well suppose we
had one maybe you can always have a pure from the free construction
if you have any any factor say H that doesn't help here and you take
an either of a H and that has a pure and just a pure rule on being
left away so that\textsf{'}s a free point to do for that construction you can
always so that\textsf{'}s it you can always do that so let\textsf{'}s imagine that G
has a pure method whatever it is then we can generate G leaves by
using that method and we can define pure we could also generate G
branches like this by doing recursive call of the pure but that\textsf{'}s
of course infinite recursion so that\textsf{'}s not that\textsf{'}s not great that\textsf{'}s
not it not great at all or we could terminate that recursion at some
point for example we can do a pure of a which is this one-step pure
generating leaves and we can put that into branches into another peer
so it\textsf{'}s pure of pure base because it\textsf{'}s kind of a two-step regenerate
any branches but each of these branches is a leaf itself so you could
do this you could imagine defining pure in a number of ways and none
of this works here is why we need to have this Lord some flatten of
pure needs to be identity on an arbitrary FA of type F of a so flatten
of pure if we substitute definition of pure is like this I'm substituting
this definition the first one the most straightforward one and the
result is that flatten will redistribute the leaves into branches
and so the result of this is always going to be a right of something
there\textsf{'}s no way this could be equal to FA for all FA what if I face
your left that\textsf{'}s it you can't do it it\textsf{'}s not going to be under the
identity function now no matter how you implement pure whatever pure
returns it to be turned left it could return right after flatten you
will get our F of right of something so there\textsf{'}s no hope that that
could be an identity function because it will never return an F of
left of something and it must do that for some FA like let\textsf{'}s say for
this kind of a thing and so already we see that the left identity
law cannot possibly hold for for this truth no matter how we implement
peel one of these implementations we'll have a look all right so I
keep you busy with the exercise that will show a social duty of this
some walnut and only one construction is left this construction is
unusual because it gives you a mu net out of an arbitrary contractor
or a semi Monad out of an arbitrary contra funder and a functor similarly
to this construction a full monad is only obtained when G is absent
so this is a construction that gives a full moon of the results are
different from those in the construction we have seen before those
in construction - or here you could have a moon out where G is a moment
not just an arbitrary factor here you cannot have a moment even if
G is a moment we will see how that works to define this construction
we start with semi Malad so assume when G is an arbitrary factor H
is arbitrary contra factor and we define the type F as a function
type going from H of a to the pair of a and G of a we will have to
define a filter instance for F for this it is convenient to find to
define a function instance for this type instructor which I'll denote
F map a G which is what we've seen before in the previous construction
in the construction - this is the same type constructor now F map
F is defined like this so did you find it we need to take a function
f going from A to B and arbitrary value F a of type F of a which is
this function so f a is this function and we need to produce F of
B which is again this kind of function will be instead of a so to
produce that we write a function expression starting with HB and then
we have to return here a tuple of B and G of B we do that by using
F map in G that returns us a tuple B of G of B and we apply that F
map F on some tuple of a G of a which we obtained by calling this
function on em h way how do we get an HIV when we have H of B we use
contra map on H of B using function f so contra map goes in the other
direction takes a function of a to b and transforms h of b to HIV
so in this way we obtain the correct type and there is no other way
of doing it now how do we can compute flatten well we need to define
a function that takes an arbitrary f of F of a and returns F of a
how do we define that well to return F of a means to return a function
of this type so we start the code by taking an argument of type H
of a and we need to return now a tuple of AGOA we have a function
of this type what we need is to return this tuple of a G of a so clearly
we have to call this function on something there\textsf{'}s nothing else we
can do we cannot just create a tuple of type a G of a out of no data
so we have to call this function but on what argument we need to get
an argument of type H of F of a in other words of this type and suppose
we can do that then we call F F of F FA on this we get a tuple of
this kind well then we could discard the second part of the tuple
and we get our F of a as this as as as required so it reminds just
to get somehow a value of this type so how do we get the value of
this type well we actually have a value of HIV so what we need is
to produce H of F of a using HIV while using HIV because they have
no other data at the moment so let\textsf{'}s think about this so we can use
contra map on H of a to get H of F of a and the control map would
need a function of this type fly control map there\textsf{'}s nothing else
we can do with HIV H of a is a value of an arbitrary control function
H we don't know anything about it and we the only thing we can do
is to use control map on that control factor and notice we are trying
to do nothing special with these types will not trying to match on
types or use reflection to see what that H is we just use information
that is available which is that each is a contra factor and G is a
factor and since we keep doing just that the result will be a natural
transformation so we're not doing everything special with specific
types always fully generic in our type parameters okay so how do we
get the function of this type of F of a going to a is a function that
can be written like this it takes F of a and produces an e so how
do we produce an a well we must apply F of a to something to produce
an e F of a is this type so we need to apply F of a to some a tree
and we have one this one so we apply F of a to HMA we get a to go
NGO a we take the first part of the tool and that\textsf{'}s an e so that\textsf{'}s
the function we need here on which we use contra map I write out the
argument the type of argument of this function for contra map to compile
because if I don't Scala will get confused we're done with this so
we now have defined a value of type H of F of a we can apply FFA to
it and get this and then we take the first part of the two ball and
that\textsf{'}s an F we apply that to H a and we get what we need so this is
a bit convoluted but this is the correct way were the only actually
way of doing this or implementing the type now let\textsf{'}s think about this
a little more and think about how we can simplify this flat map flatmap
or flat pack flatten well this thing is a function that takes HIV
and returns hm f of a and it turns out that this is a function we
will use repeatedly so let\textsf{'}s define this function separately with
the name I call it insert f because it inserts a layer of F inside
a layer of H which can be done with this specific type and it just
usually cannot be done like this you cannot insert something into
another type constructor arbitrarily but it is done as we have seen
with these types so I just copy this inside the code here and then
I can write the definition of flatten in a shorter way as this f f/a
instead of H if a I use this so then it\textsf{'}s clear this is a function
of H a so the argument of AJ is used twice because of this it\textsf{'}s hard
to rewrite it in a point free style point free style means we don't
write arguments of functions we just write function compositions so
if we didn't have this application then I would be able to write it
in a point free style as FTL equals 50 n takes FFA and the result
is a function that takes a chain applies insert F first to H a let
me let me write it in a nice way I'm looking for the functional composition
symbol this one copy it over there so first we apply insert if tha
then we apply ff8 TJ and then we apply the first element in the tuple
extraction but actually this still has to be is still a function that
needs to be applied to each e and there\textsf{'}s no nice way of writing that
down so can't really write fully point freestyle and also a fully
point freestyle would be the FT N equals something I'm not f TMF FFA
or something and we can't do that is that the phase used like this
inside it\textsf{'}s very hard probably you could invent a notation for this
will probably not be very useful for reasoning about it so that\textsf{'}s
I'm not sure if anyone has an invented notation for the point freestyle
for such things and if so it\textsf{'}s probably not very useful for reasoning
but in Haskell ecosystem there is a tool called 0.3 that transforms
functions into a point freestyle the result of that tool not always
eliminating so we tried we don't see how much much useless let\textsf{'}s not
do it so we'll keep it in the code let\textsf{'}s now verify the associative
a table so to verify a social deity we need to compare this with this
so let\textsf{'}s define the first function which is a composition of flatten
and flatten this function takes an arbitrary FFF a first it applies
flatten to that so which is a flattened with the type parameter F
of a I'm copying now the short definition of flatten and I rename
h8 of HF a because that\textsf{'}s the type of that other than that it\textsf{'}s the
same code now I apply flatten to this code which means I write again
this expression but instead of FFA I take this this function so this
function is applied to insert F of H a so that means instead of this
I have insert F of H a so instead of H F a I have insert F of H a
so I just substitute I write this and instead of HFA I write insert
F of H a so then the result is this I have now insert F of H a and
insert F of insert F of H a because I substituted h fa into here the
second function well there\textsf{'}s nothing here we can simplify so at this
point let\textsf{'}s keep in this way we'll see what we need to transform when
we look at the second function so the second function is flat map
of ATM of F T n followed by F T n so we take F F F a we first applied
flat map of F T onto it which is sorry not a flat map with F map so
F map of f TM is obtained by substituting the definition of F map
which is this so we copy this code in here and I just renamed it your
face so that it\textsf{'}s easier to be substitute later now we apply f TM
to it we get f TM of that I just copied here substitute the definition
of f TN which is the same as the substitute this into this expression
so we have h a goes to this this H a comes from the definition of
F T and I have TN and returns a function that starts with H a and
returns this and instead of F FA you insert the function to which
you apply flat map sorry flatten so result is this so this is instead
of FFA in death code if you do that you get this so now it looks like
can do very much except try to substitute the definition of F map
a G now look we have F map a G of something dot underscore one so
let\textsf{'}s see what earth map AG does when we do underscore one after it
so actually this can be simplified if you look at the definition of
F map AG it returns a tuple so dot one of it is just this in other
words applying F to the first part of Ag so that\textsf{'}s what it is applying
F to the first part of Ag now we can simplify F map a G of this function
instead of F and this instead of a G so this is a G dot one to which
we apply this function f teen mystery of F and then we still have
to apply the result to AJ now see I'm very careful I'm just copying
the code and inserting I'm not checking any type stuff actually I
could I just uncomment this and save scholar complains well it probably
complains what it complains about too many arguments oh yeah because
I have this this thing let me comment this out right this is not valid
scholar syntax right so this has a correct type and restore the comments
as they were before so by just putting this into the Scala intelligent
code I check the types that\textsf{'}s how I did it so each of these would
was type checked before I come in the default so now it remains to
substitute the definition of FTM so we have FTM here as a shorter
form with two arguments FFA and AJ so let\textsf{'}s just list so we put this
and substitute instead of F F F F F n this and AJ AJ so we just put
that there and we get this now let\textsf{'}s compare these two expressions
HHA sorry fi cafe of something dot one of this the one of that so
that dot one of blah is the common element here F F FA of something
is common element so the only difference is these two pieces the in
self of NSF and NSF come from AB flat so these two are different so
let\textsf{'}s see if they are transformed into each other if we put more effort
both of these expressions have type H of F of F of a because we insert
twice here and here we insert once and then we'll come to map so let\textsf{'}s
show that they're always equal and then if we can show that we're
done we show that these two expressions are always equal this one
in this one so it remains to compare these two so let me define just
functions that take AJ and return this and the other function will
take AJ and return that no no no I'll transform the code of these
functions I'm defining them just so that I can easier easily check
types without repeating all that so now the first function is this
let\textsf{'}s substitute the definition of insert F and we get F FA of this
underscore one of this underscore one remind you the definition of
constraint F is a shape control map of this we do a cheek contour
map of this contra map of that which can be contracted to a single
control map with composition of these two functions so that\textsf{'}s what
I did here combined to cause and the result is this we have this expression
let\textsf{'}s try to see what happens in the other one we have this let\textsf{'}s
substitute the definition of insert F which is this one now let\textsf{'}s
compose the two contra maps so we need to compose in the opposite
direction so we first apply this function and then we apply this function
in this case the first function is flatly and the second function
is is this one apply a chain not one so then we have this contra map
first we flatten take some arbitrary F of F FA flatten it first and
then apply this function to the result that results in this function
then we substitute the definition of flatten and that is this short
definition so now we can compare these two expressions and we see
that they are identical you see I can probably remove this yes let\textsf{'}s
type a notation I can remove and I can just rewrite this for syntax
so then they are syntactically identical so this shows that these
two expressions are identical so that concludes our proof of the associativity
law that\textsf{'}s to say it\textsf{'}s a semi mu not the only thing we have used is
a contra map property and the map property for G when we defined F
map a G so f map a G is used to define F map for our constructor F
and confirm map is used to define flat so that\textsf{'}s the only thing we
define we used to define flatten and F map for this type constructor
so then it becomes a semi moment and we have checked the social activity
so now let\textsf{'}s see if this can be defined into a full moment we'll show
that actually this cannot be done unless G of a his unit so in other
words we can only done it in that way oh how to show that well first
of all how can we create a pure for this F when you take an A and
we need to return a function it takes H of a and returns a tuple of
eight and G of a well how can we return a tuple of a and G of a we
have an eight we need a G of a also the control factor H of a is of
no help it cannot give us any values of type A or of type G of F so
we must ignore that contra factor argument of type H a we have to
ignore that argument and we return a tuple of le and then we have
to have some function like pure G of a that returns a G of a given
a name in other words we need that function in order to define pure
flotsam and imagine we have that function in other words the factor
G is pointed that\textsf{'}s what it means to have that function a natural
transformation with this type signature so imagine that the font of
G is pointed can we define a full monad for the type constructor F
let\textsf{'}s see we can define pure with the right type signature that\textsf{'}s
for sure let\textsf{'}s check the laws identity laws is that pure followed
by flatten is identity so let\textsf{'}s do that take an arbitrary FA apply
pura to it and then apply flat into the result but substitute the
definition of flattening acting on pure so that will be this now pure
of a PHA is something that returns a function that ignores its argument
so we can ignore this and the first part of pure of the fame of something
is fa so therefore this is a che going to FA of AJ so all this goes
away this returns a che so H sorry this returns FA so this becomes
FA of AJ so now the code is like this and the function taking page
a and returning FA of which is the same as the function FA so we don't
we could rewrite this as simply effect so that is the left identity
law so that holds let\textsf{'}s take the right identity law its F map pure
and followed by flatten so we take an arbitrary FA we apply F map
Puran to it and then we apply flatten to the result and let\textsf{'}s write
a tray right here for simplicity and so we'll apply this to AJ what\textsf{'}s
now transform this expression this expression we substitute of the
diff the definition of flattened FFA HJ where FFA is this so that
gives you this piece followed by application to NSF of LJ dot 1 over
J so that\textsf{'}s not now let\textsf{'}s substitute the definition of this in here
where F is puree FA is FA and HB is this so the result I'm just going
to take the code for F map which is up here and substitute F and H
B as I just described so the result is this so now we again have the
situation of F map a G of something dot under square one so that can
be simplified we have a fear of whatever it was here the applied to
this dot one so pure was the first argument of F map and G this was
the second argument of F naught a G so then we have this first argument
applied to the second argument dot underscore one now let\textsf{'}s see what
we can simply find here well we we can we need to substitute peer
and we need to substitute this we already know that in insert F of
H taken from map you really know we can simplify that so let\textsf{'}s want
to do that it\textsf{'}s an H taken from a pure and then this function which
is in SEF so we can combine them in into one function first applying
pure and then applying that to the result so that gives us this function
now this can be further simplified because that\textsf{'}s definition of pure
taking that one of this gives you just a so then basically this is
just an identity function and applying culture map of identity function
so this is just a dynasty function country map identity function is
identity so H a culture map of this is just H a so therefore we can
simplify our expression but using that this is just H a so then we
have pure very fa fa j dot one h a we can simplify this further by
inserting the definition of pure which is the tuple of this and pure
g of that so now we have the simplify this we can simplify anymore
because we don't know what pure gene does and we don't know what FA
does and what if a old age is so now let\textsf{'}s compare this has to be
equal to F a of H a because this entire thing must be an identity
function so identity function takes F a and returns again FA in other
words it must return a function at X H a and the plies FA to a chip
so that\textsf{'}s we expect to see just this instead we see this tuple what
is that - PO well this tuple has the first part from FHA now if a
of H a is itself a tuple let\textsf{'}s check that so FA of AJ is of type a
G of a so it\textsf{'}s a tuple and if this were equal to f AV J then this
should have been a favorite a dot underscore one comma F aah a dot
underscore two so instead of FA of H a dot underscore two we see this
how could this two things these two things be equal for arbitrary
a finucci it\textsf{'}s the same as to say that we have an arbitrary tuple
of this type and the second element of that tuple must be equal to
pure G of the first element of the tuple so the function pure G must
somehow be able to compute the second element of an arbitrary tuple
from the first element that\textsf{'}s impossible a tuple of two elements contains
information in both elements so this could be some arbitrary value
of type a this could be some arbitrary value of type G of a there\textsf{'}s
no possible way to compute a second value from the first in general
as long as the second value contains any non-trivial information as
long as it has more than one different value we couldn't possibly
guess what that value is given some other value unrelated to it it\textsf{'}s
all related because we don't know what there\textsf{'}s no constraint on the
function f a if a is an arbitrary function that takes H a and gives
a value of this type so the function f a gives an arbitrary value
of type a and then an arbitrary another value of times G of a they're
not relating these to the type of the type is like this so this is
G of a and that\textsf{'}s the same as here that\textsf{'}s just the type the values
are not related so it\textsf{'}s as long as the value of type G of a can be
more than one different now here we can posit we cannot possibly guess
or compute what it is so it\textsf{'}s impossible that this returns fa unless
there\textsf{'}s only one possible value in this type in other words G of a
contains no information there\textsf{'}s only one value of this type which
means it\textsf{'}s a unit type for away so G of a must be a trivial factor
constant factor that returns a unit type for all types a so in that
case pure G is just a function that returns unit and FA must have
been a function that returns a unit in the second point of the tuple
and we're done and it\textsf{'}s the identity law wouldn't hold just the first
part of the tuples is fine it\textsf{'}s the second part of the tuple that\textsf{'}s
broken so if G of a is unit then the second part of the tuple is always
unit and both identity laws hold so that conclude completes the proof
that this construction returns a full monad when G is unit in other
words when we don't have any gf and we can simply find a way product
with unit until just a so this completes the proof of all our constructions
let me give a brief overview of what we have found so first of all
these are not all possible constructions that give you a monad out
of something certain you can combine these constructions and you are
assured that you will get a monad as a result but there are other
there further constructions which I did not talk about which I know
and probably there are also constructions that I don't know there
isn't it seems a theory in the literature that explains to you what
are all possible constructions or what are all possible monads that
theory seems to be lacking in the literature at least I couldn't find
it I did a search online I made a question on Stack Overflow about
this but nobody seems to know so for example the question of how to
recognize a semi 1 ad or a monad from its type expression that seems
to be an open question in other words nobody knows the answer if I
give you some arbitrary type expression it\textsf{'}s not clear that you can
easily recognize that it\textsf{'}s a monad or not you can certainly try to
implement the methods pure and flatten there might be many implementations
of these of these that fit the types and then you you would have to
prove the flaws hold that\textsf{'}s a lot of work for any given type expression
it could be a huge amount of computation that is not easy and so it
doesn't seem to be an easy criterion however if you can build a monad
out of known constructions like these you're guaranteed that laws
hold and so there\textsf{'}s no need to check the laws so these constructions
give you examples of Vinod\textsf{'}s such as the constant function so again
for full model you need a unit but for semi wallet it can be just
any type a fixed type then you have a product of a and something which
is a semi model only a full model only when the identity monad is
considered so G is 1 so these are examples constant bonded identity
model and we have a product of two models but not the disjunction
or sometimes co-product as it is called you have a function from a
fixed type to a moment or same Amanat note that here you don't have
a function from a fixed type terminal if this is a moment that doesn't
help to make that a moment so recall we just proved that the identities
laws cannot hold unless G a is actually equal to one so if G a is
a moon at itself that doesn't help it needs to be actually equal to
one for this construction for this construction GA can be a moon odd
and then it\textsf{'}s a product of identity mullet and this moon up and the
product works so this is how it works and here you can have a function
from a fixed type to an arbitrary unit here not the fixed type is
a control factor so this construction contains a fixed type as an
option so if H a could be just some fixed type R but the right-hand
side must be of this form it must be a if you want to fool monad so
in this construction it\textsf{'}s a constant on the left but this can be an
arbitrary monad this construction is a free pointed so it gives you
another moment for a given what does this construction do for you
if you already have the moment why would you add a to it the reason
is this monad can be easily recognized as a value being pure or not
pure so in this mo not pure values are on the left and any non pure
values are on the right so any value that has an effect is on the
right and any value that doesn't have an effect is on the left so
I'm saying that a monadic value or a value of a monadic type has an
effect when it is not a result of applying pure to something so if
it if it\textsf{'}s equal to pure of something then that monadic value has
no effect as as empty effect in it that\textsf{'}s just an intuitive way of
talking about it it\textsf{'}s not really that we can recognize the presence
of an effect in a value except for this mode so in this model it\textsf{'}s
very easy to recognize the presence of an effect if it\textsf{'}s on the left
and there is no effect it wasn't a result of pure but on the right
then there is an effect so for example you start with pure you apply
some flat maps to it you could get this but if you just apply maps
to it then you would stay pure and in this mode it is easily recognizable
if a value is pure or not you can pattern match on the disjunction
and find out so maybe that is useful for certain applications to be
able to recognize by pattern matching whether a monadic value is pure
that this has an empty effect or it is not pure as some non-empty
effect and also note remember in the derivation of this we had a function
called merge which takes this and returns G of a so you can always
merge this a back into the monad G and that could be useful if Mona
Jie has some computational significance and this is just some extra
structure that you need temporarily for computation and then you can
get rid of it without losing any information or an emergent back so
that could be useful structure now this is a model where you substitute
the type parameter for another lunatic actually we have seen where
you will see in an exercise that this itself is the moon odd the combination
of writer monad and either moment so this is now a combination of
an arbitrary monad and this now this combination is interesting because
it\textsf{'}s a come it\textsf{'}s a functor composition of two minutes of G and of
this and so this is an interesting example where a functor composition
of two monads is again a monad this is by no means always the case
but there are some units such as this one such that a functor composition
of an arbitrary one at G and this mono is again a monad so that is
important to know that that such one else exists in a thesis a certain
class of Mona that have this property another class of Mona that have
similar property is this this is also a function composition of the
reader monad are two a and G of a but on in the opposite order so
G of a is inside the reader unit and here G of a is outside of this
moment so there this is a so reader belongs to a different class of
monads such that their functor composition with any other monad inside
is again a monad but this would not be the case if we use this moment
if you put G instead of here that would not work so this so the reader
madad and this linear type linear polynomial monad are of different
classes with respect to founder composition and that has relevance
for monad transformers which I will talk about in a later tutorial
now this construction is very important it\textsf{'}s a free madad over function
G and it gives you a mu nod out of an arbitrary function G no restrictions
on factor G so this is used a lot to obtain monads out of arbitrary
factors now the last two constructions are somewhat peculiar haven't
seen them in the literature much or at all I don't know if they have
names but so this construction is unfortunately only a semi mu not
it\textsf{'}s a tree with G shaped leaves and G shaped branches and this construction
is interesting because it gives you a monad out of an arbitrary control
function now control factors can be seen as type constructors that
have a function in them that consumes some values of type a let\textsf{'}s
say a general contractor can be a function from a functor to some
constant type let\textsf{'}s say my function from A to C or C is a constant
type or function from a functor G A to Z where Z is a constant type
so that\textsf{'}s a general kind of control factor of polynomial exponential
class of course I don't know if any other way of reasoning about types
except by talking about exponential polynomial types that is type
expressions that have function type disjunction and conjunction of
product so within this class of types all control factors can be seen
as functions from some G a let\textsf{'}s say to a constant type Z so an example
of this would be arbitrary factor G a going to Z all of that going
to a and that\textsf{'}s a moment for arbitrary control factor H a or if you
represent a chase through another factor then it will be a moon ad
for an arbitrary factor H now some G so this is another construction
that gives you a moon out of an arbitrary factor or an arbitrary control
factor I don't know yet what the use cases for those monads there
might be but at this point these are just the constructions that I
found and there are a few other constructions that come from one are
transformers but then you're almost general and they all require Mona
so there are no other constructions I know that take an arbitrary
factor which is itself not a unit or an arbitrary control function
and make a monad out of that these are these two constructions and
finally I don't think there is any such thing as a contaminant or
control function that is itself having a magnetic property I don't
think that is possible because a contractor such as H a consumes values
of a so the a inside the control factor are in the contravariant position
now monad structure would mean that you transform H of H of a into
H of a but if H is a control function then each of HIV is a functor
so because because a to contravariant positions cancel each other
it\textsf{'}s impossible to have a natural transformation between a contra
factor and D factor so there is no way that you could possibly have
a function such as flatten that takes F of F of a and returns F of
a where F is a contra factor just contravariance here and here is
different there is no way to have a natural transformation between
them between the founder and the control factor and so there\textsf{'}s no
way for you to implement flatten and so there is no analogue of contra
factors that are monads in any way moani\textsf{'}s must always be functors
so this concludes the theoretical part now the exercises let me make
some comments about the exercises so the first exercise is to complete
the proof that we started in the working examples we showed that this
is a semi group for a semi monad and semi group s but now if if in
a film or not you should show that this is a film on right even if
s is itself a mono it also for monoid the second exercise is a specific
example of one noted non-trivial monoid this exercise is to show explicitly
by symbolic code transformations this is a semi model you have to
implement F nap and flatten and peer notice here boo is a specific
type z is not but misses boo boo is a monoid so you can use any kind
of OneNote structure on boo the next exercise is to show that this
can be a semi woman but not a Mona no no yes this is actually equivalent
to a disjunction of MA and ma because boolean is just two values and
so you can expand the brackets and you get a disjunction so ma disjunction
of ma and ma is not a monad we know it\textsf{'}s a semi modem because we have
the construction that product instruction but it\textsf{'}s not a moaner so
sure that the laws don't hold next is again an exercise to showing
showing that one of them is emunah the other cannot be made into a
semi moment at this point is see this is a reader for that and we
compose readable not and the writer monad so we can pose it in one
order first the reader and then apply the writer to that or we first
take the writer and then apply the reader to them now this is construction
for so this can be made in the same way but I did not prove construction
for so you don't know that actually it will be exercise ten to verify
construction for so for this exercise just look at the types and see
which one can have the type of flatten that we require in other words
the type of flatten for this needs to be implemented can it be implemented
and can this be implemented and if so sure one of them expect this
one to be not sad because the type cannot be implemented so similarly
here show that you cannot implement the type necessary for a flat
map or or flatten whatever you choose now here P and Q are arbitrary
and different types you don't know anything about them so for specific
types maybe you could implement the same Emunah but not for the arbitrary
types in this exercise you have this type constructor which is just
this and you can implement Clanton and pure for it the types could
be implemented but the monado\textsf{'}s the knothole so the exercise for you
is to show that you can implement pure and you can implement flatten
in many ways so the exercise tells you to do it in at least two different
ways so these two different pure and at least two different flanking
but the laws never hold whatever whatever combination of those implementations
you take the Monad loads will not hold at least some of them will
fail in this exercise I don't expect you to enumerate all possible
implementations of pure and flatly and check that for all combinations
of these the laws fail but this is so there is a stack overflow question
about it which I initiated indeed not not I initiated somebody else
actually had a comment on something I said in some other question
and they asked this question so people have checked by explicit and
full calculation and I have checked it as well myself that no implementation
of pure and flatten for this type constructor will satisfy laws so
this is an example of a very simple type constructor which is a functor
but not a monad cannot be made into a monad for some obscure reasons
it\textsf{'}s not obvious why but something is missing on this type and you
cannot make a monad out of it next exercise is to check the laws now
for functor not a monad this the functor laws are what we discussed
in chapter 4 so this type constructor is actually not a function because
it is not covariant it it contains an alien a covariant position and
then a in a contravariant tradition so this is actually not a factor
not a country function either but for some obscure reason you can
implement a function f map that has the correct type signature this
is the code for it you should check that this has the right types
and so why is this possible well this is just some kind of accident
but imagine that the programmer didn't think about covariance here
and thought that this is this is my data type and I want F nap for
it and I can implement it great does that make this function you know
unless it satisfies the factor law so they the exercise is to show
it doesn't the next exercise is to check the Monad laws for this which
is similar to checking the Monad laws for for this except now instead
of boom you here you only verify the social Liberty and here you need
to verify everything and W must be a monoid then in this exercise
this is actually one of the constructions but the exercise is to write
down implementations for platinum pure explicitly and check the laws
now in this exercise it\textsf{'}s to show that pure so the construction 5
let me look at it within Islam so the construction 5 is this in this
construction when you find fewer for F as left of a now the exercise
here is to show that if we didn't and we defined it as right of Mona
G pure of a which we could have done conceivably because G is a monad
so we could put pure into the right part into here so show that if
we did that the one of the identity laws would fail so there\textsf{'}s no
way this would work in the next exercise the question is to take this
and find the constructions that you can use to construct this moment
and once you find it you already know that list is a monad for example
you don't need to check the laws so then you need to just implement
the monad methods for it and to do that we look at the constructions
and the constructions actually implement the Monad methods for each
step of the construction so you don't have to guess how to implement
a monad here because you have each step of the constructions showing
you how to implement flatten given previous monads implementation
but of course if you feel like guessing from scratch then you can
implement the Monad and methods for this for you from scratch in the
next exercise is to try the construction - which was this one in the
worked examples I showed I implemented this construction by discarding
the second effect in this exercise you should repeat that but they
start the first defense it would be a different implementation of
a semi Monnet and it\textsf{'}s still a semi Monat so you show it and it should
show that associativity is still satisfied for construction eight
in the next exercise I I did not show that associativity holds I just
showed that it cannot be a monad but I did not show that cannot be
a full Mona so it\textsf{'}s still a semi mode that only I did not show such
activity so you should show it in this exercise and finally not in
the last exercise you should revisit the standard known as the state
and a continuation moments from the first parts of the tutorial we
have verified the associativity laws for them but not the identity
laws so this exercise is to verify the identity loss for them good
luck with the exercises 
\end{comment}


\chapter{Applicative functors and contrafunctors\label{chap:8-Applicative-functors,-contrafunctors}}

\global\long\def\gunderline#1{\mathunderline{greenunder}{#1}}%
\global\long\def\bef{\forwardcompose}%
\global\long\def\bbnum#1{\custombb{#1}}%
\global\long\def\pplus{{\displaystyle }{+\negmedspace+}}%


\section{Motivation and first examples}

In previous chapters, we generalized the \lstinline!map!, \lstinline!filter!,
and \lstinline!flatMap! methods from sequences to other type constructors
that support such methods obeying suitable laws. Following the same
path, we now turn to the \lstinline!zip! method. Although \lstinline!zip!
is most often used with sequences, many other type constructors can
also have a suitable \lstinline!zip! method. Those type constructors
are known as\textbf{ applicative}.

\subsection{Generalizing the \texttt{zip} method from sequences to other types}

Chapter~\ref{chap:2-Mathematical-induction} showed the use of the
\lstinline!zip! operation for sequences and other collections. For
lists, the \lstinline!zip! method is equivalent to a function with
the following type signature:

\begin{wrapfigure}{l}{0.55\columnwidth}%
\vspace{-0.85\baselineskip}
\begin{lstlisting}
def zip[A, B](la: List[A], lb: List[B]): List[(A, B)]
\end{lstlisting}

\vspace{-0.25\baselineskip}
\end{wrapfigure}%

\noindent \vspace{-0.5\baselineskip}
\[
\text{zip}:\text{List}^{A}\times\text{List}^{B}\rightarrow\text{List}^{A\times B}\quad.
\]
It turns out that a broad class of type constructors $L$ can have
a similar \lstinline!zip! method:
\[
\text{zip}:L^{A}\times L^{B}\rightarrow L^{A\times B}\quad.
\]
Using this type signature, the \lstinline!zip! operation may be implemented
for many type constructors, not only for \lstinline!List!-like collections.
In order to ensure that the implementation of \lstinline!zip! is
useful and safe, we will establish and verify the laws of the \lstinline!zip!
operation later in this chapter. For now, let us look at some examples
of implementing a \lstinline!zip! operation.

\subsubsection{Example \label{subsec:Example-applicative-not-monad}\ref{subsec:Example-applicative-not-monad}}

A \lstinline!zip! method can be implemented for the type constructor
$L^{A}\triangleq\bbnum 1+A\times A$:
\begin{lstlisting}
type L[A] = Option[(A, A)]
def zip[A, B](la: L[A], lb: L[B]): L[(A, B)] = (la, lb) match {
  case (None, _) | (_, None)              => None
  case (Some((a1, a2)), Some((b1, b2)))   => Some(((a1, b1), (a2, b2)))
}

scala> zip(Some((123, 456)), Some(("abc", "def")))
res0: L[(Int, String)] = Some(((123, "abc"), (456, "def")))
\end{lstlisting}


\subsubsection{Example \label{subsec:Example-applicative-tree}\ref{subsec:Example-applicative-tree}}

A \lstinline!zip! method can be implemented for a binary tree. We
define the type constructor \lstinline!BTree[A]! by this code:
\begin{lstlisting}
sealed trait BTree[A]
final case class Leaf[A](a: A) extends BTree[A]
final case class Branch[A](left: BTree[A], right: BTree[A]) extends BTree[A]
\end{lstlisting}
The \lstinline!zip! method should have the following type signature:
\begin{lstlisting}
def zip[A, B](ta: BTree[A], tb: BTree[B]): BTree[(A, B)] = ???
\end{lstlisting}
We would like to preserve the tree structure of \lstinline!ta! and
\lstinline!tb! as much as possible, zipping together subtrees of
equal shape without adding any new branchings. For example, {\tiny{}}zip({\tiny{} \Tree[ [ [ $a$ ] [ $b$ ] ]  [ $c$ ] ] }, {\tiny{} \Tree[ [ [ $d$ ] [ $e$ ] ]  [ $f$ ] ] })
should evaluate to the tree {\tiny{}}{\tiny{} \Tree[ [ [ $a\times d$ ] [ $b\times e$ ] ]  [ $c\times f$ ] ] }.
If a subtree of \lstinline!ta! is a \lstinline!Leaf(x)! while the
corresponding subtree of \lstinline!tb! is a \lstinline!Branch!,
the value \lstinline!x! must be replicated to match the subtree of
\lstinline!tb!. So, the result of evaluating {\tiny{}}zip({\tiny{} \Tree[ [ $b$ ] [ $c$ ] ] }, {\tiny{} \Tree[ [ [ $a$ ] [ [ $d$ ] [ $e$ ] ] ]  [ $f$ ] ] })
should be {\tiny{}}{\tiny{} \Tree[ [ [ $b\times a$ ] [ [ $b\times d$ ] [ $b\times e$ ] ] ]  [ $c\times f$ ] ] }
with replicated $b$.

\subparagraph{Solution}

Begin writing the pattern-matching code: 
\begin{lstlisting}
def zip[A, B](ta: BTree[A], tb: BTree[B]): BTree[(A, B)] = (ta, tb) match {
  case (Leaf(x), Leaf(y))                 => Leaf((x, y))
  case (Branch(lx, rx), Leaf(y))          => ???
  case (Leaf(x), Branch(ly, ry))          => ???
  case (Branch(lx, rx), Branch(ly, ry))   => ???
}
\end{lstlisting}
The type constructor \lstinline!BTree! is a functor and has a \lstinline!map!
method:
\begin{lstlisting}
def map[A, B](ta: BTree[A])(f: A => B): BTree[B] = ta match {
  case Leaf(a)          => Leaf(f(a))
  case Branch(ta, tb)   => Branch(map(ta)(f), map(tb)(f))
}
\end{lstlisting}
When \textsf{``}zipping\textsf{''} a \lstinline!Leaf! with a \lstinline!Branch!,
we use the \lstinline!map! method to replicate the value from the
leaf:
\begin{lstlisting}
  case (Branch(lx, rx), Leaf(y))  => map(Branch(lx, rx))(x => (x, y))
  case (Leaf(x), Branch(ly, ry))  => map(Branch(ly, ry))(y => (x, y)) 
\end{lstlisting}
For the case of two \lstinline!Branch! values, we use two recursive
calls to \lstinline!zip!:
\begin{lstlisting}
  case (Branch(ax, ay), Branch(bx, by))   => Branch(zip(ax, bx), zip(ay, by))
\end{lstlisting}
The final code of the \lstinline!zip! method, after some simplifications,
becomes:

\begin{lstlisting}
def zip[A, B](ta: BTree[A], tb: BTree[B]): BTree[(A, B)] = (ta, tb) match {
  case (Leaf(x), Leaf(y))                 => Leaf((x, y))
  case (xa, Leaf(b))                      => map(xa)(a => (a, b))
  case (Leaf(a), xb)                      => map(xb)(b => (a, b))
  case (Branch(ax, ay), Branch(bx, by))   => Branch(zip(ax, bx), zip(ay, by))
}
\end{lstlisting}
To test our code, let us run the given examples and verify that we
get the required results:
\begin{lstlisting}
val ta: BTree[Int] = Branch(Branch(Leaf(1), Leaf(2)), Leaf(3))
val tb: BTree[String] = Branch(Leaf("b"), Leaf("c"))
val tc: BTree[String] = Branch(Branch(Leaf("a"), Branch(Leaf("d"), Leaf("e"))), Leaf("f")) 

scala> zip(ta, ta)
res0: BTree[(Int, Int)] = Branch(Branch(Leaf((1, 1)), Leaf((2, 2))), Leaf((3, 3)))

scala> zip(tb, tc)
res1: BTree[(String, String)] = Branch(Branch(Leaf(("b", "a")), Branch(Leaf(("b", "d")), Leaf(("b", "e")))), Leaf(("c", "f"))) 
\end{lstlisting}
$\square$

The \lstinline!zip! operation is sometimes defined even for type
constructors that are \emph{not} functors:

\subsubsection{Example \label{subsec:Example-applicative-profunctor}\ref{subsec:Example-applicative-profunctor}}

A pair of monoids is also a monoid (the \textsf{``}products\textsf{''} construction
of Section~\ref{subsec:Monoids-constructions}). A \lstinline!Monoid!
typeclass instance for a type \lstinline!A! can be represented by
a value of type $L^{A}$, where the type constructor $L$ is defined
by:

\begin{wrapfigure}{l}{0.5\columnwidth}%
\vspace{-0.85\baselineskip}
\begin{lstlisting}
type L[A] = (A, (A, A) => A)
\end{lstlisting}

\vspace{-0.25\baselineskip}
\end{wrapfigure}%

\noindent \vspace{-0.5\baselineskip}
\[
L^{A}\triangleq A\times\left(A\times A\rightarrow A\right)\quad.
\]

\noindent We can write a function that creates a monoid typeclass
instance for the type $A\times B$ when $A$ and $B$ are monoids.
That function has the type signature of a \lstinline!zip! method
for the type constructor $L$:
\begin{lstlisting}
def zip[A, B](la: L[A], lb: L[B]): L[(A, B)] = ( (la._1, lb._1),   // The result has type ((A, B),
  { case ((a1, b1), (a2, b2)) => (la._2(a1, a2), lb._2(b1, b2)) }  // ((A, B), (A, B)) => (A, B)).
)
\end{lstlisting}
Applying this function to any two monoid instances, we obtain an instance
value for the pair:
\begin{lstlisting}
def monoidPair[A, B](implicit ma: Monoid[A], mb: Monoid[B]): Monoid[(A, B)] = zip(ma, mb)
\end{lstlisting}

Several typeclasses (such as the \lstinline!Monoid!) have instances
whose type constructors are neither covariant nor contravariant yet
have a \lstinline!zip! method. If we implement a \lstinline!zip!
method for typeclass instances, we will obtain automatic typeclass
derivation for product types (e.g., case classes).

\subsection{Gathering all errors during computations\label{subsec:Programs-that-accumulate-errors}}

A monadic program using pass/fail monads must stop at the first failure:
the code \lstinline!flatMap(x => expr)! cannot start evaluating \lstinline!expr!
if a previous computation failed to produce a value for \lstinline!x!.
However, if some pass/fail computations are independent of each other\textsf{'}s
results, we may wish to run all those computations and gather all
errors.

As an example, consider the task of implementing \textsf{``}safe arithmetic\textsf{''}
where a division by zero or square root of a negative number will
give error messages (see Example~\ref{subsec:disj-Example-resultA}).
To be specific, let us perform the computation \lstinline!(1 / 0) + (2 / 0)!
in the \textsf{``}safe arithmetic\textsf{''}. A monadic implementation (see Section~\ref{subsec:Pass/fail-monads})
will stop the computation after the first error:
\begin{lstlisting}
type Result[A] = Either[String, A]
def add(x: Int, y: Int): Result[Int] = Right(x + y)
def div(x: Int, y: Int): Result[Int] = if (y == 0) Left(s"error: $x / $y") else Right(x / y)

scala> for {
         x <- div(1, 0)
         y <- div(2, 0)
         z <- add(x, y)
       } yield z
res0: Either[String, Int] = Left(error: 1 / 0)
\end{lstlisting}
We notice that the two \lstinline!div! operations do not depend on
each other and may be computed separately. To achieve this, we define
a \lstinline!map2! function for the type constructor \lstinline!Result!:
\begin{lstlisting}
def map2[A, B, C](ra: Result[A], rb: Result[B])(f: (A, B) => C): Result[C] = (ra, rb) match {
  case (Left(e1), Left(e2))     => Left(e1 + "\n" + e2)    // Messages are separated by a newline.
  case (Left(e1), _)            => Left(e1)
  case (_, Left(e2))            => Left(e2)
  case (Right(a), Right(b))     => Right(f(a, b))
}
\end{lstlisting}
 We can now use the \lstinline!map2! function to compute the two
\lstinline!div! operations and gather the errors:
\begin{lstlisting}
scala> for {
         p <- map2(div(1,0), div(2,0)) { (x, y) => (x, y) }    // Create a tuple (x, y).
         z <- add(p._1, p._2)
       } yield z
res1: Either[String, Int] = Left(error: 1 / 0
error: 2 / 0)
\end{lstlisting}
The result of \lstinline!map2! is used in further monadic computations.
In this way, we can combine code that gathers many errors with ordinary
pass/fail monadic code that stops at the first error.

This example can be generalized to the type \lstinline!Either[E, A]!,
where the type \lstinline!E! is a semigroup with a \lstinline!combine!
operation (denoted by \lstinline!|+|!). In other words, error messages
of type \lstinline!E! can be combined in some way. Let us implement
a \lstinline!zip! method that works similarly to \lstinline!map2!
by gathering all error messages:
\begin{lstlisting}
def zipE[A, B, E: Semigroup](x: Either[E, A], y: Either[E, B]): Either[E, (A, B)] = (x, y) match {
  case (Left(e1), Left(e2))   => Left(e1 |+| e2)
  case (Right(a), Left(e2))   => Left(e2)
  case (Left(e1), Right(b))   => Left(e1)
  case (Right(a), Right(b))   => Right((a, b))
}
\end{lstlisting}
In the code notation, the \lstinline!zipE! function is written like
this:
\[
\text{zip}_{E}:(E+A)\times(E+B)\rightarrow E+A\times B\quad,\quad\quad\text{zip}_{E}\triangleq\,\begin{array}{|c||cc|}
 & E & A\times B\\
\hline E\times E & e_{1}\times e_{2}\rightarrow e_{1}\oplus e_{2} & \bbnum 0\\
A\times E & a\times e_{2}\rightarrow e_{2} & \bbnum 0\\
E\times B & e_{1}\times b\rightarrow e_{2} & \bbnum 0\\
A\times B & \bbnum 0 & a\times b\rightarrow a\times b
\end{array}\quad.
\]

When both arguments of \lstinline!zipE! have error messages, the
code should not drop one of the errors and return \lstinline!Left(e1)!,
say (although that would still conform to the type signature of \lstinline!zipE!).
Instead, our code combines both error messages; this preserves more
information.

Comparing the code of \lstinline!map2! and \lstinline!zip!, we find
only one difference. Namely, the code of the \lstinline!zip! function
can be obtained from the code of \lstinline!map2! if we replace \lstinline!f!
by an identity function. We will see later that this correspondence
between \lstinline!zip! and \lstinline!map2! works for all applicative
functors.

\subsection{Monadic programs with independent effects\label{subsec:Monadic-programs-with-independent-effects-future-applicative}}

Another motivation for applicative functors comes from considering
monadic programs where some source lines do not depend on previous
results. As an example, consider the following monadic program involving
\lstinline!Future!:

\begin{wrapfigure}{l}{0.535\columnwidth}%
\vspace{-0.85\baselineskip}
\begin{lstlisting}
val result: Future[Int] = for {
  x <- Future { func1() }  // func1() returns an Int.
  y <- Future { func2() }  // Similarly for func2().
  z <- Future { func3() }  // Similarly for func3().
} yield x + y + z
\end{lstlisting}

\vspace{-1\baselineskip}
\end{wrapfigure}%

\noindent Recall that Scala\textsf{'}s \lstinline!Future! values are computations
that have been scheduled to run on other threads (and may be already
running). In this example, the computations in \lstinline!func1()!,
\lstinline!func2()!, and \lstinline!func3()! do not depend on the
previously computed values (\lstinline!x! and \lstinline!y!). So,
we would like to run those computations in parallel. However, the
monadic program will create a second \lstinline!Future! value and
schedule the computation \lstinline!func2()! only after \lstinline!func1()!
is done. Similarly, \lstinline!func3()! will not start until \lstinline!func2()!
is done. This is because the functor block shown above is translated
into \lstinline!flatMap! and \lstinline!map! methods like this:

\begin{wrapfigure}{l}{0.535\columnwidth}%
\vspace{-0.9\baselineskip}
\begin{lstlisting}
val result: Future[Int] =
  Future { func1() }.flatMap { x =>
    Future { func2() }.flatMap { y =>
      Future { func3() }.map { z => x + y + z } } }
\end{lstlisting}

\vspace{-1.2\baselineskip}
\end{wrapfigure}%

\noindent The implementations of \lstinline!flatMap! and \lstinline!map!
cannot detect whether the next computations actually depend on any
previously computed values. So, the code of the \lstinline!flatMap!
and \lstinline!map! methods \emph{must} wait until all previous values
are computed. We need a new function that can take advantage of the
independence of effects. A possible type signature for that function
is:
\begin{lstlisting}
def map3(f1: Future[Int], f2: Future[Int], f3: Future[Int], f: (Int, Int, Int) => Int): Future[Int] 
\end{lstlisting}
Generalizing this type signature to arbitrary types, we obtain the
following \lstinline!map3! method:
\begin{lstlisting}
def map3[F[_], A, B, C, D](f1: F[A], f2: F[B], f3: F[C])(f: (A, B, C) => D): F[D]
\end{lstlisting}
It is straightforward to implement this method for \lstinline!F = Future!:
\begin{lstlisting}
def map3[A, B, C, D](f1: Future[A], f2: Future[B], f3: Future[C])(f: (A, B, C) => D): Future[D] =
  for { x <- f1; y <- f2; z <- f3 } yield f(x, y, z)
\end{lstlisting}
This will run the three \lstinline!Future! computations in parallel
because the arguments of \lstinline!map3! must be evaluated before
\lstinline!map3! is called, and evaluating a \lstinline!Future!
will schedule its computation immediately. To support different number
of arguments, we can implement similar functions \lstinline!map2!,
\lstinline!map4!, etc.

It is not obvious that the function \lstinline!map3! is related to
\lstinline!zip!. However, once we study the properties of those functions
in more detail, we will find that \lstinline!map3! can be expressed
through \lstinline!map2!, and that defining \lstinline!map2! is
equivalent to defining \lstinline!map! and \lstinline!zip! (assuming
that certain naturality laws hold for these functions). So, applicative
functors can be viewed as functors having either a \lstinline!map2!
or a \lstinline!zip! method in addition to \lstinline!map!. We will
also see that \lstinline!map4!, \lstinline!map5!, etc., can be all
expressed through \lstinline!zip!.

\section{Practical use of applicative functors}

Applicative functors appear whenever it is useful to have a \lstinline!zip!
or a \lstinline!map2! operation. We will now look at a few specific
situations where applicative operations help write better code.

\subsection{Transposing a matrix via \texttt{map2}}

The standard sequence type (\lstinline!Seq!) already has a \lstinline!zip!
method, so implementing a \lstinline!map2! for \lstinline!Seq! is
simple:
\begin{lstlisting}
def map2[A, B, C](as: Seq[A], bs: Seq[B])(f: (A, B) => C): Seq[C] =
  (as zip bs).map { case (a, b) => f(a, b) }

scala> map2(List(1, 2), List(100, 200))(_ + _)
res0: Seq[Int] = List(101, 202)
\end{lstlisting}

\index{matrix transposition}In Example~\ref{subsec:Example-matrix-products}(a),
we implemented matrix transposition using \lstinline!flatMap! and
index-based access to sequences, which is not available for some sequence
types. The index-based access is avoided if we implement \lstinline!transpose!
via \lstinline!map2!. As before, a matrix is represented by a sequence
of sequences:
\begin{lstlisting}
def transpose[A](s: Seq[Seq[A]]): Seq[Seq[A]] = ???

scala> transpose(List(List(1, 2), List(3, 4), List(5, 6)))
res1: Seq[Seq[Int]] = List(List(1, 3, 5), List(2, 4, 6))
\end{lstlisting}
We would like to define \lstinline!transpose! by induction. The base
case is an \textsf{``}empty\textsf{''} matrix:
\begin{lstlisting}
def transpose[A](s: Seq[Seq[A]]): Seq[Seq[A]] = s.headOption match {
  case None          => Seq()
  case Some(heads)   => val tails = s.tail
                        ???
}
\end{lstlisting}
For the inductive step, we assume that the \lstinline!tails! sub-matrix
is already transposed. So, it remains to combine \lstinline!transpose(tails)!
with \lstinline!heads!. Consider the test example shown above:
\[
\left|\begin{array}{cc}
1 & 2\\
3 & 4\\
5 & 6
\end{array}\right|\,\triangleright\text{transpose}=\,\left|\begin{array}{ccc}
1 & 3 & 5\\
2 & 4 & 6
\end{array}\right|\quad,\,\,\quad\left|\begin{array}{cc}
1 & 2\\
3 & 4\\
5 & 6
\end{array}\right|\,\triangleright\text{head}=\,\left|\begin{array}{cc}
1 & 2\end{array}\right|\quad,\,\,\quad\left|\begin{array}{cc}
1 & 2\\
3 & 4\\
5 & 6
\end{array}\right|\,\triangleright\text{tail}=\,\left|\begin{array}{cc}
3 & 4\\
5 & 6
\end{array}\right|\quad.
\]
Splitting this matrix into the first row (\lstinline!heads!) and
the rest (\lstinline!tails!), we will obtain \lstinline!heads == List(1, 2)!
and \lstinline!tails == List(List(3, 4), List(5, 6))!. The inductive
assumption is that \lstinline!transpose(tails)! works correctly and
yields \lstinline!List(List(3, 5), List(4, 6))!. How to combine this
with \lstinline!heads == List(1, 2)! to obtain the desired result
\lstinline!List(List(1, 3, 5), List(2, 4, 6))!? We need to prepend
each element of \lstinline!heads! to the corresponding sub-list in
\lstinline!transpose(tails)!. This is implemented by using \lstinline!map2!:
\begin{lstlisting}
def transpose[A](s: Seq[Seq[A]]): Seq[Seq[A]] = s.headOption match {
  case None          => Seq()
  case Some(heads)   =>
    val tails = s.tail
    map2(heads, transpose(tails)) { (x, y) => x +: y }
}
\end{lstlisting}
This code is still incomplete: it returns an empty sequence for all
arguments. We need to add another base case when the matrix \lstinline!s!
has only one row (so \lstinline!tails! is empty). Here is the final
code:
\begin{lstlisting}
def transpose[A](s: Seq[Seq[A]]): Seq[Seq[A]] = s.headOption match {
  case None          => Seq()
  case Some(heads)   => s.tail match {
    case Seq()   => heads.map(a => Seq(a))
    case tails   => map2(heads, transpose(tails)) { (x, y) => x +: y }
  }
}
\end{lstlisting}


\subsection{Data validation with error reporting}

Suppose we need to create a data structure with \textsf{``}validated\textsf{''} parts
(the data type could be a simple tuple, a case class, a list, a tree,
etc.). While validating any part of the data structure, we may get
an error. This situation happens when validating Web input data, reading
a set of command-line options, or parsing formatted text (such as
JSON). We would like to report all errors to the user, rather than
stopping at the first error. 

To simplify the reasoning, assume that the required data structure
is a case class, such as:
\begin{lstlisting}
final case class MyData(userId: Long, userName: String, userEmails: List[String])
\end{lstlisting}
More generally, consider the type $A\times B\times C$ containing
values of some chosen types $A$, $B$, and $C$:
\begin{lstlisting}
final case class MyData[A, B, C](a: A, b: B, c: C)
\end{lstlisting}

We assume that the validation functions will return values of types
\lstinline!F[A]! defined by

\begin{wrapfigure}{l}{0.5\columnwidth}%
\vspace{-0.85\baselineskip}
\begin{lstlisting}
type F[A] = Either[E, A]
\end{lstlisting}

\vspace{-0.25\baselineskip}
\end{wrapfigure}%

\noindent \vspace{-0.5\baselineskip}
\[
F^{A}\triangleq E+A\quad.
\]

Here, \lstinline!A! is the type of a successfully validated value,
while the fixed type \lstinline!E! describes the error information.
We will assume that $E$ is a semigroup, so that different pieces
of error information can be combined into a larger value of the same
type $E$.

If some of the validations fail, the result should be a value of type
$E$. So, our task is to create a value of type $E+A\times B\times C$
given values of types $E+A$, $E+B$, and $E+C$. Using the type constructor
$F$ and the type \lstinline!MyData[A, B, C]!, we write the requirements
as the following type signature:
\begin{lstlisting}
def validated[A, B, C](fa: F[A], fb: F[B], fc: F[C]): F[MyData[A, B, C]] = ???
\end{lstlisting}
In the type notation, this is written as:
\[
\text{validated}:F^{A}\times F^{B}\times F^{C}\rightarrow F^{A\times B\times C}\quad.
\]
If we implement this type signature, we will be able to perform data
validation while collecting all errors. Note that the type signature
of \lstinline!validated! is quite similar to that of \lstinline!zip!
except for using \emph{three} different types rather than two. So,
let us define a suitable function \lstinline!zip3!. The code of \lstinline!zip3!
uses the ordinary \lstinline!zip! method twice:
\begin{lstlisting}
def zip3[A, B, C](fa: F[A], fb: F[B], fc: F[C]): F[(A, B, C)] =
  zip(fa, zip(fb, fc)).map { case (a, (b, c)) => (a, b, c) }
\end{lstlisting}
Also, \lstinline!validated! has to insert a \lstinline!MyData! type
constructor, returning a value of type \lstinline!F[MyData[A, B, C]]!
rather than \lstinline!F[(A, B, C)]!. So, we may implement \lstinline!validated!
as:
\begin{lstlisting}
def validated[A, B, C](fa: F[A], fb: F[B], fc: F[C]): F[MyData[A, B, C]] =
  zip(fa, zip(fb, fc)).map { case (a, (b, c)) => MyData(a, b, c) }
\end{lstlisting}

The difference between \lstinline!zip3! and \lstinline!validated!
is only in the function used in the last \lstinline!map! step. So,
we can make that last function a parameter and define the operation
\lstinline!map3! like this:
\begin{lstlisting}
def map3[A, B, C, D](fa: F[A], fb: F[B], fc: F[C])(f: (A, B, C) => D): F[D] =
  zip(fa, zip(fb, fc)).map { case (a, (b, c)) => f(a, b, c) }
\end{lstlisting}
Now the validation function can be implemented via \lstinline!map3!
more concisely:
\begin{lstlisting}
def validated[A, B, C](fa: F[A], fb: F[B], fc: F[C]): F[MyData[A, B, C]] =
  map3(fa, fb, fc)(MyData.apply)
\end{lstlisting}


\subsection{Implementing the functions \texttt{map2}, \texttt{map3}, etc. The
\texttt{ap} method}

Working with applicative functors involves using methods such as \lstinline!map2!,
\lstinline!map3!, and so on. The code of \lstinline!map2! for the
functor $L^{A}\triangleq\text{String}+A$ contains a pattern match
with four cases (see Section~\ref{subsec:Programs-that-accumulate-errors}).
Writing similar code for \lstinline!map3! requires \emph{eight} cases
(to match a triple of \lstinline!Either! values), and \lstinline!map4!
would require $16$ cases. How can we implement all these functions
without writing a lot of code?

One idea is to use the \lstinline!zip! method repeatedly, as we saw
in the previous section. We would then implement \lstinline!map2!
through \lstinline!zip!, \lstinline!map3! through \lstinline!zip3!,
and so on. For the purposes of illustration, let us choose the types
of all elements to be the same, so that the type signatures become:
\[
\text{zip}:L^{A}\times L^{A}\rightarrow L^{A\times A}\quad,\quad\quad\text{zip}_{3}:L^{A}\times L^{A}\times L^{A}\rightarrow L^{A\times A\times A}\quad.
\]
We can now implement a general function ($\text{zip}_{n}$) that uses
a list of values of type $L^{A}$:
\begin{lstlisting}
def zipN[A](xs: List[Either[String, A]]): Either[String, List[A]] = xs match {
  case Nil            => Right(Nil)
  case head :: tail   => zipE(head, zipN(tail)).map { case (h, t) => h :: t }
}
\end{lstlisting}
The corresponding function \lstinline!mapN! is then defined by
\begin{lstlisting}
def mapN[A, Z](xs: List[Either[String, A]])(f: List[A] => Z): Either[String, Z] = zipN(xs).map(f)
\end{lstlisting}

If we wanted to define general functions \lstinline!zipN! and \lstinline!mapN!
that could take $N$ arguments of arbitrary types (and not all of
type $A$), we would need to use techniques of dependent-type programming,
which is beyond the scope of this book. We will now describe a simpler
solution that implements \lstinline!mapN! via a helper function (\lstinline!ap!)
that performs a recursion step expressing $\text{map}_{N}$ through
$\text{map}_{N-1}$.

If the \lstinline!ap! method can do that, it can also express \lstinline!map2!
via \lstinline!map!. So, let us determine the functionality that
is present in \lstinline!map2! but missing from \lstinline!map!.
That missing functionality is what \lstinline!ap! must implement.

To compare \lstinline!map! and \lstinline!map2! functions more easily,
consider their curried versions \lstinline!fmap! and \lstinline!fmap2!:

\begin{wrapfigure}{l}{0.543\columnwidth}%
\vspace{-0.8\baselineskip}
\begin{lstlisting}
def fmap[A,B](f: A => B): L[A] => L[B]
def fmap2[A,B,C](f: A => B => C): L[A] => L[B] => L[C]
\end{lstlisting}

\vspace{-1.45\baselineskip}
\end{wrapfigure}%

~\vspace{-1.75\baselineskip}
\begin{align*}
 & \text{fmap}:\left(A\rightarrow B\right)\rightarrow L^{A}\rightarrow L^{B}\quad,\\
 & \text{fmap}_{2}:\left(A\rightarrow B\rightarrow C\right)\rightarrow L^{A}\rightarrow L^{B}\rightarrow L^{C}\quad.
\end{align*}
\vspace{-1.7\baselineskip}

If we try implementing \lstinline!fmap2! via \lstinline!map!, we
get stuck:
\begin{lstlisting}
def fmap2[A, B, C](f: A => B => C): L[A] => L[B] => L[C] = { la: L[A] => la.map(f) ??? }
\end{lstlisting}
The value \lstinline!la.map(f)! has type \lstinline!L[B => C]! instead
of the required type \lstinline!L[B] => L[C]!. So, we need a function
that converts between those types. The new function is called \lstinline!ap!
and has the type signature:

\begin{wrapfigure}{l}{0.543\columnwidth}%
\vspace{-0.6\baselineskip}
\begin{lstlisting}
def ap[B, C](lf: L[B => C]): L[B] => L[C]
\end{lstlisting}

\vspace{-0.5\baselineskip}
\end{wrapfigure}%

~\vspace{-0.5\baselineskip}
\[
\text{ap}_{L}:L^{A\rightarrow B}\rightarrow L^{A}\rightarrow L^{B}\quad.
\]

If we have an implementation of \lstinline!ap! for a functor $L$,
we can write the code for \lstinline!fmap2! as:
\begin{lstlisting}
def fmap2[A, B, C](f: A => B => C): L[A] => L[B] => L[C] = { la: L[A] => ap[B, C](la.map(f)) }
\end{lstlisting}
Written in the point-free style using the code notation, this definition
looks like this:
\[
\text{fmap}_{2}(f)\triangleq f^{\uparrow L}\bef\text{ap}_{L}\quad.
\]
Taking into account the curried type signature of \lstinline!fmap2!,
we may rewrite this as \lstinline!fmap2(f)(la)(lb) == ap(la.map(f))(lb)!.
This expression is made clearer if we implement infix syntax for \lstinline!fmap!
and \lstinline!ap!:
\begin{lstlisting}
implicit class FmapSyntax[A, B](f: A => B) {
  def <@>[F[_] : Functor](fa: F[A]): F[B] = fa.map(f)
} 
implicit class ApSyntax[A, B](lab: L[A => B]) {
  def <*>(la: L[A]): L[B] = ap(lab)(la)
}
\end{lstlisting}
Then we rewrite \lstinline!fmap2(f)(la)(lb)! as \lstinline!f <@> la <*> lb!.
This works because all infix operators group to the left, so \lstinline!f <@> la <*> lb!
means \lstinline!(f <@> la) <*> lb!, which is translated into \lstinline!ap(la.map(f))(lb)!.

The main advantage of this new syntax is in making it easier to implement
\lstinline!fmap3!, \lstinline!fmap4!, etc. Let us see how we can
use the \lstinline!ap! method to implement \lstinline!fmap3!:
\begin{lstlisting}
def fmap3[A, B, C, D](f: A => B => C => D): L[A] => L[B] => L[C] => L[D] = { la: L[A] =>
  ap[B, C => D](la.map(f)) andThen ap[C, D] }
\end{lstlisting}
In the point-free style, this definition is written as:
\[
\text{fmap}_{3}(f)\triangleq f^{\uparrow L}\bef\text{ap}_{L}\bef\left(g\rightarrow g\bef\text{ap}_{L}\right)\quad.
\]
It is harder to write and to understand the code of \lstinline!fmap3!
than that of \lstinline!fmap2!. An implementation of \lstinline!fmap4!
via \lstinline!ap! will be even more complicated. But with the infix
syntax shown above, we can write:
\begin{lstlisting}
def fmap3[A, B, C, D](f: A => B => C => D)(a: L[A])(b: L[B])(c: L[C]): L[D] = f <@> a <*> b <*> c
def fmap4[A, B, C, D, E](f: A => B => C => D => E)(a: L[A])(b: L[B])(c: L[C])(d: L[D]): L[E] =
  f <@> a <*> b <*> c <*> d
\end{lstlisting}
These examples show how to implement the function $\text{fmap}_{N}$
for any $N$ and any argument types.

For type constructors that are applicative but not covariant, a method
analogous to \lstinline!map2! must have a different type signature.
For instance, consider the type constructor $L$ defined in Example~\ref{subsec:Example-applicative-profunctor}.
We cannot define \lstinline!map2! as a composition of \lstinline!zip!
and \lstinline!map! because $L^{A}$ does not have a \lstinline!map!
method (as $L$ is not covariant in $A$). Instead, we note that $L$
is a profunctor\index{profunctor} and supports an \lstinline!xmap!
method:
\begin{lstlisting}
def xmap[A, B](la: L[A])(f: A => B, g: B => A): L[B] = (f(la._1), (b1, b2) => f(la._2(g(b1), g(b2))))
\end{lstlisting}
Composing \lstinline!zip! with \lstinline!xmap!, we obtain a new
method we may call \lstinline!xmap2!:
\begin{lstlisting}
def xmap2[A,B,C](la: L[A], lb: L[B])(f: ((A,B)) => C, g: C => (A,B)): L[C] = xmap(zip(la, lb))(f, g)
\end{lstlisting}
Methods such as \lstinline!xmap3!, \lstinline!xmap4!, etc., may
be defined similarly via \lstinline!zip3!, \lstinline!zip4!, etc.
However, a method analogous to \lstinline!ap! does not exist for
applicative profunctors.

\subsection{The applicative \texttt{Reader} functor\label{subsec:The-applicative-Reader-functor}}

The \lstinline!Reader! functor, $L^{A}\triangleq E\rightarrow A$,
is applicative and supports a \lstinline!zip! method as well:
\begin{lstlisting}
type Reader[A] = E => A    // The fixed type E must be already defined.
def zip[A, B](ra: Reader[A], rb: Reader[B]): Reader[(A, B)] = { e => (ra(e), rb(e)) }
\end{lstlisting}
The \lstinline!map2! method is implemented similarly:
\begin{lstlisting}
def map2[A, B, C](ra: Reader[A], rb: Reader[B])(f: A => B => C): Reader[C] = { e => f(ra(e))(rb(e)) }
\end{lstlisting}
These are the \emph{only} fully parametric implementations of the
type signatures of \lstinline!zip! and \lstinline!map2! for \lstinline!Reader!.

Since \lstinline!Reader! is also a monad (see Section~\ref{subsec:The-Reader-monad}),
we may implement the type signature of \lstinline!map2! as:
\begin{lstlisting}
def map2[A, B, C](ra: Reader[A], rb: Reader[B])(f: A => B => C): Reader[C] = for {
  x <- ra
  y <- rb
} yield f(x)(y)
\end{lstlisting}
This code is fully parametric. So, this \lstinline!map2! must be
equal to the direct implementation shown above.

It is noteworthy that the \lstinline!Reader!\textsf{'}s applicative and monad
instances agree on the implementation of \lstinline!map2!. We can
understand this intuitively if we consider that \lstinline!Reader!\textsf{'}s
effect is a dependency on a constant \textsf{``}environment\textsf{''} (a value of
type $E$). All \lstinline!Reader! computations in a given functor
block will read the same value of the \textsf{``}environment\textsf{''}. So, the \lstinline!Reader!
effects are always independent, and a \lstinline!map2! function does
not need to be implemented separately (it can be expressed via \lstinline!flatMap!).

\subsection{Single-traversal \texttt{fold} operations. I. Applicative \textquotedblleft fold
fusion\textquotedblright\label{subsec:Single-traversal-fold-operations-applicative-fold-fusion}}

We have seen various \textsf{``}\lstinline!fold!-like\textsf{''} operations (such
as \lstinline!foldLeft!, \lstinline!foldRight!, or \lstinline!reduce!)
that iterate over a sequence and accumulate a result value. An example
is the computation of the average of a list of numbers (see Example~\ref{subsec:ch1-aggr-Example-4}):
\begin{lstlisting}
def average(s: List[Double]): Double = s.sum / s.size
\end{lstlisting}
Both operations \lstinline!s.sum! and \lstinline!s.size! will iterate
over the list \lstinline!s!. So, \lstinline!average(s)! iterates
\emph{twice} over \lstinline!s!.

It is sometimes undesirable (or impossible) to iterate over a sequence
more than once. An example of that situation is a data stream whose
data elements arrive over a network at high speed. Usually, the stream
data cannot be stored because the data volume would grow too quickly.
So, the program needs to be implemented as an aggregation\index{aggregation}
operation that uses a single traversal of the stream. 

Another example is the computation of word distribution statistics
in a large text corpus.\index{corpus}\footnote{Such as the \textsf{``}Common Crawl\textsf{''} corpus, see \texttt{\href{https://commoncrawl.org/}{https://commoncrawl.org/}}}
A \textsf{``}corpus\textsf{''} is a sequence of large chunks of text. Each chunk takes
a long time to download, and storing the entire sequence in memory
is impossible. It is necessary to minimize the number of traversals
of the corpus. Ideally, all computations need to be implemented in
a single traversal.

Note that the \lstinline!sum! and \lstinline!size! methods are particular
cases of a \lstinline!foldLeft! operation. We could avoid a double
traversal in the code of \lstinline!average! if we implemented it
as a single \lstinline!foldLeft! call:
\begin{lstlisting}
def average(s: List[Double]): Double = {
  val (sum, size) = s.foldLeft((0.0, 0)) { case ((sum, size), x) => (sum + x, size + 1) }
  sum / size
}
\end{lstlisting}
This maintains a combined accumulated state of type \lstinline!(Double, Int)!
for the \lstinline!sum! and \lstinline!size! aggregations. 

We could always rewrite any number of aggregations as a single \lstinline!foldLeft!
that combines the aggregation codes into a more complicated updater
function operating on a combined state value. How to avoid writing
that code manually? The trick known as \textbf{fold fusion}\index{fold fusion}
allows us to merge any number of \lstinline!fold!-like operations
into a single operation. The code will automatically build and update
the accumulated state correctly, traversing the sequence only once.

In this section, we will implement fold fusion in two ways: as an
\textsf{``}applicative\textsf{''} fusion and as a \textsf{``}monadic\textsf{''} fusion. The contrast
between these approaches will illustrate the difference between applicative
and monadic functors.

The idea of fold fusion is to represent \lstinline!fold!-like operations
by values of a certain type, and to implement functions that combine
and \textsf{``}run\textsf{''} such operations.

We begin by defining a type constructor that represents a \lstinline!fold!-like
operation as a value. Let us look at the type signature of \lstinline!foldLeft!:
\begin{lstlisting}
def foldLeft[Z, R](zs: Seq[Z])(init: R)(update: (R, Z) => R): R
\end{lstlisting}
We should be able to apply a given \lstinline!fold!-like operation
to many different sequences \lstinline!zs!. So, we translate the
type signature of \lstinline!foldLeft! into a value by omitting the
argument \lstinline!zs!:
\begin{lstlisting}
final case class Fold[Z, R](init: R, update: (R, Z) => R)
\end{lstlisting}
To run the \lstinline!fold!-like operation represented by a value
of type \lstinline!Fold[Z, R]!, we implement a \textsf{``}runner\textsf{''}:\index{runner!for fold operations@for \texttt{fold} operations}
\begin{lstlisting}
def run[Z, R](zs: Seq[Z], fold: Fold[Z, R]): R = zs.foldLeft(fold.init)(fold.update)
\end{lstlisting}

Two \lstinline!fold!-like operations can be combined only if they
have the same type \lstinline!Z!. If we combine an \lstinline!op1: Fold[Z, R]!
with an \lstinline!op2: Fold[Z, S]!, we should obtain a new \lstinline!fold!-like
operation of type \lstinline!Fold[Z, (R, S)]!. So, the operation
of combining \lstinline!op1! and \lstinline!op2! is equivalent to
a \lstinline!zip! operation for the type constructor \lstinline!Fold[Z, R]!
with respect to the type parameter \lstinline!R!, while the type
\lstinline!Z! is kept fixed:
\begin{lstlisting}
def zipFold[Z, R, S](op1: Fold[Z, R], op2: Fold[Z, S]): Fold[Z, (R, S)] =
  Fold((op1.init, op2.init), (r, z) => (op1.update(r._1, z), op2.update(r._2, z)))
\end{lstlisting}

The types of \lstinline!Fold! and \lstinline!zipFold! are expressed
in the short type notation as:
\begin{align*}
 & \text{Fold}^{Z,R}\triangleq R\times\left(R\times Z\rightarrow R\right)\quad,\\
 & \text{zip}_{\text{Fold}}:\text{Fold}^{Z,R}\times\text{Fold}^{Z,S}\rightarrow\text{Fold}^{Z,R\times S}\quad,\quad\quad\text{or equivalently:}\\
 & \text{zip}_{\text{Fold}}:R\times\left(R\times Z\rightarrow R\right)\times S\times\left(S\times Z\rightarrow S\right)\rightarrow\left(R\times S\right)\times(R\times S\times Z\rightarrow R\times S)\quad.
\end{align*}
A fully parametric, information-preserving implementation of \lstinline!zipFold!
follows unambiguously from its type signature. So, let us use the
\index{curryhoward library@\texttt{curryhoward} library}\lstinline!curryhoward!
library\footnote{See \texttt{\href{https://github.com/Chymyst/curryhoward}{https://github.com/Chymyst/curryhoward}}}
to generate the code automatically:
\begin{lstlisting}
import io.chymyst.ch._       // Import some symbols from the `curryhoward` library.
def zipFold[Z, R, S](op1: Fold[Z, R], op2: Fold[Z, S]): Fold[Z, (R, S)] = implement
\end{lstlisting}

With these definitions, we can rewrite \lstinline!average! as a single-traversal
calculation:
\begin{lstlisting}
val sum = Fold[Double, Double](0, _ + _)
val length = Fold[Double, Int](0, (n, _) => n + 1)
val sumLength: Fold[Double, (Double, Int)] = zipFold(sum, length)
val res: (Double, Int) = run(Seq(1.0, 2.0, 3.0), sumLength)

scala> val average = res._1 / res._2
average: Double = 2.0
\end{lstlisting}
In this way, we may combine any number of \lstinline!fold!-like operations
into a single traversal. 

While this simple implementation of applicative fold fusion works,
it is not easy to use. The result of \textsf{``}running\textsf{''} a combined \lstinline!Fold!
operation will be a tuple structure that must be transformed in just
the right way (do we need \lstinline!res._1 / res._2! or \lstinline!res._2 / res._1!?)
to obtain the final result. A better approach is to implement a syntax
that combines some \lstinline!Fold! operations and at the same time
defines a final transformation to be applied to the results. We would
like to write code that directly manipulates \lstinline!fold!-like
operations and produces the final result:
\begin{lstlisting}
val sum = ...            // Define `sum` and `length` in a suitable way.
val length = ...
Seq(1.0, 2.0, 3.0).runFold(sum / length)    // Should return `2.0` here.
\end{lstlisting}

To enable this kind of code,\footnote{A more fully featured library using this approach is \texttt{scala-fold},
see \texttt{\href{https://github.com/amarpotghan/scala-fold}{https://github.com/amarpotghan/scala-fold}} } we need to add a final transformation to the \lstinline!Fold! type:
\begin{lstlisting}
final case class FoldOp[Z, R, A](init: R, update: (R, Z) => R, transform: R => A)
\end{lstlisting}
A value of type \lstinline!FoldOp[Z, R, A]! represents a \lstinline!fold!-like
operation for sequences of type \lstinline!Seq[Z]!. The operation
accumulates an intermediate state of type \lstinline!R! and computes
a final result of type \lstinline!A!.

The runner is implemented as an extension method on \lstinline!Seq!:
\begin{lstlisting}
implicit class FoldOpSyntax[Z](zs: Seq[Z]) {
  def runFold[R, A](op: FoldOp[Z, R, A]): A = op.transform(zs.foldLeft(op.init)(op.update))
}
\end{lstlisting}

How can we combine two \lstinline!fold!-like operations of types
$\text{FoldOp}^{Z,R,A}$ and $\text{FoldOp}^{Z,S,B}$? The combined
operation must maintain intermediate states of types $R$ and $S$.
Also, we have two result values of types $A$ and $B$. It seems that
the combined operation needs an intermediate state of type $R\times S$
and a final result of type $A\times B$. So, let us implement a \lstinline!zip!
extension method by using those types:
\begin{lstlisting}
implicit class FoldOpZip[Z, R, A](op: FoldOp[Z, R, A]) {
  def zip[S, B](other: FoldOp[Z, S, B]): FoldOp[Z, (R, S), (A, B)] = implement
  def map[B](f: A => B): FoldOp[Z, R, B] = implement
  def map2[S, B, C](other: FoldOp[Z, S, B])(f: (A, B) => C): FoldOp[Z, (R, S), C] = implement
} // The type signatures unambiguously determine the implementations.
\end{lstlisting}
The \lstinline!map! and \lstinline!map2! methods exist because \lstinline!FoldOp[Z, R, A]!
is covariant with respect to \lstinline!A!.

We can now implement extension methods allowing us to do arithmetic
on \lstinline!fold!-like operations:
\begin{lstlisting}
implicit class FoldOpMath[Z, R](op: FoldOp[Z, R, Double]) {
  def binaryOp[S](other: FoldOp[Z, S, Double])(f: (Double, Double) => Double): FoldOp[Z, (R, S), Double] = op.map2(other) { case (x, y) => f(x, y) }
  def +[S](other: FoldOp[Z, S, Double]): FoldOp[Z, (R, S), Double] = op.binaryOp(other)(_ + _) 
  def /[S](other: FoldOp[Z, S, Double]): FoldOp[Z, (R, S), Double] = op.binaryOp(other)(_ / _) 
} // May need to define more operations here.
\end{lstlisting}

After these definitions, the following code will work:
\begin{lstlisting}
val sum = FoldOp[Double, Double, Double](0, (s, i) => s + i, identity)
val length = FoldOp[Double, Double, Double](0, (s, _) => s + 1, identity)

scala> Seq(1.0, 2.0, 3.0).runFold(sum / length)
res0: Double = 2.0
\end{lstlisting}

We note that the type signatures of \lstinline!zip! and \lstinline!map2!
are different from the usual ones: the accumulated state\textsf{'}s type (\lstinline!R!)
is transformed into a pair type \lstinline!(R, S)!. However, the
programmer\textsf{'}s code does not need to annotate those types because they
are automatically inferred by the Scala compiler.

The standard \lstinline!scanLeft! method is similar to \lstinline!foldLeft!
except it outputs all intermediate accumulated states (while \lstinline!foldLeft!
outputs only the last one). Can we merge several \lstinline!scanLeft!
operations into a single traversal? Since the type signature of \lstinline!scanLeft!
is the same as that of \lstinline!foldLeft! except for the return
type, the \lstinline!FoldOp! data structure already stores the information
needed to run \lstinline!scanLeft!. The code for a corresponding
runner (called \lstinline!runScan!) is:
\begin{lstlisting}
implicit class FoldOpScan[Z](zs: Seq[Z]) {
 def runScan[R,A](op: FoldOp[Z,R,A]): Seq[A] = zs.scanLeft(op.init)(op.update).map(op.transform).tail
}
\end{lstlisting}

As an example of using this functionality, consider the task of computing
the running average of a sequence. A running average could be computed
from the beginning of the sequence:
\[
\text{ave}_{k}\triangleq\frac{1}{k}\sum_{i=0}^{k-1}s_{i}\quad.
\]
The code \lstinline!val average = sum / length! implements a \lstinline!fold!-like
operation computing this average.

Another task is to compute the average over a sliding window of a
fixed size $n$:
\[
\text{ave}_{n,k}\triangleq\frac{1}{n}\sum_{i=k-n}^{k-1}s_{i}\quad.
\]
To implement this, we first create a sliding window as a \lstinline!fold!-like
operation and then do averaging:
\begin{lstlisting}
def window[A](n: Int): FoldOp[A, IndexedSeq[A], IndexedSeq[A]] = FoldOp(init = Vector(),
  update = { (window, x) => (if (window.size < n) window else window.drop(1)) :+ x }, identity)

def window_average(n: Int) = window[Double](n).map(_.sum / n)
\end{lstlisting}
To test that the averaging code works as expected, we run some fold
and scan operations:
\begin{lstlisting}
scala> (0 to 10).map(_.toDouble).runFold(average)
res0: Double = 5.0

scala> (0 to 10).map(_.toDouble).runScan(average)
res1: Vector[Double] = Vector(0.0, 0.5, 1.0, 1.5, 2.0, 2.5, 3.0, 3.5, 4.0, 4.5, 5.0)

scala> (0 to 10).map(_.toDouble).runScan(window_average(3))
res2: Vector[Double] = Vector(0.0, 0.3333333333333333, 1.0, 2.0, 3.0, 4.0, 5.0, 6.0, 7.0, 8.0, 9.0)
\end{lstlisting}


\subsection{Single-traversal \texttt{fold} operations. II. Monadic \textquotedblleft fold
fusion\textquotedblright}

The applicative fold fusion as defined in the previous section works
by combining the results of independent \lstinline!fold!-like operations.
For instance, the definition \lstinline!average = sum / length! uses
\lstinline!sum! and \lstinline!length! as independent \lstinline!fold!-like
operations (the results computed by \lstinline!sum! do not depend
on the results of \lstinline!length!). The independence of effects
is a defining feature of applicative composition. In contrast, monadic
composition supports effects that may depend on previously computed
values. To illustrate this contrast, let us implement the monadic
version of fold fusion.

The type constructor \lstinline!FoldOp[Z, R, A]! can be a monad only
if it is covariant. So, a \lstinline!flatMap! method must operate
on the type parameter \lstinline!A!. We begin writing the type signature
as:
\begin{lstlisting}
def flatMap[Z, R, A, S, B](op: FoldOp[Z, R, A])(f: A => FoldOp[Z, S, B]): FoldOp[Z, ???]
\end{lstlisting}
The first \lstinline!fold!-like operation (\lstinline!op!) will
compute a value of type \lstinline!A! at each step. We will then
apply the given function \lstinline!f! to that value and obtain a
new \lstinline!fold!-like operation (of type \lstinline!FoldOp[Z, S, B]!)
that also needs to be evaluated. Since \lstinline!flatMap! will need
to run two \lstinline!fold!-like operations in one traversal, it
must maintain a combined accumulated state of type \lstinline!(R, S)!.
So, the type signature of \lstinline!flatMap! needs to accommodate
the change of that type (just as the type signature of \lstinline!map2!
did in the previous section). For convenience, we define \lstinline!flatMap!
as an extension method: 
\begin{lstlisting}
implicit class FoldFlatMap[Z, R, A](op: FoldOp[Z, R, A]) {
  def flatMap[S, B](f: A => FoldOp[Z, S, B]): FoldOp[Z, (R, S), B] = ???
}
\end{lstlisting}
The result of \lstinline!op1.flatMap(x => op2(x))! must be a \lstinline!fold!-like
operation that combines \lstinline!op1! and \lstinline!op2! and
supports arbitrary dependency in \lstinline!op2(x)! on the intermediate
result (\lstinline!x: A!), which is obtained by running \lstinline!op1!.
To achieve that, we implement \lstinline!flatMap! via the following
code:
\begin{lstlisting}[numbers=left]
implicit class FoldFlatMap[Z, R, A](op: FoldOp[Z, R, A]) {
   def flatMap[S, B](f: A => FoldOp[Z, S, B]): FoldOp[Z, (R, S), B] = {
      // To create a new `FoldOp()` value, we need `init`, `update`, and `transform`.
      val init: (R, S) = (op.init, f(op.transform(op.init)).init) // Use `init` from both operations.
      val update: ((R, S), Z) => (R, S) = { case ((r, s), z) =>   // Run both `update` functions:
          val newR = op.update(r, z)                              // First `update` function.
          val newOp: FoldOp[Z, S, B] = f(op.transform(newR))      // We may use `newR` or `r` here!
          val newS = newOp.update(s, z)                           // Second `update` function.
          (newR, newS)
      }
      val transform: ((R, S)) => B = { case (r, s) =>
          val newOp: FoldOp[Z, S, B] = f(op.transform(r))
          newOp.transform(s)
      }
      FoldOp(init, update, transform)
  }
}
\end{lstlisting}

In line 7 of the code above, we have a choice of applying \lstinline!op.transform(...)!
to the old accumulated value (\lstinline!r!) or to the updated value
(\lstinline!newR!). It appears to be better to use the updated value
(\lstinline!newR!). 

To motivate this choice, consider a running average operation applied
twice:
\begin{lstlisting}
scala> (0 to 10).toList.map(_.toDouble).runScan(average).runScan(average)
res3: List[Double] = List(1.0, 1.25, 1.5, 1.75, 2.0, 2.25, 2.5, 2.75, 3.0, 3.25)
\end{lstlisting}
Using \lstinline!flatMap!, we can express the same computation as
a single \lstinline!fold!-like operation:
\begin{lstlisting}
def add(x: Double) = FoldOp[Double, Double, Double](0, (a, _) => a + x, identity)

scala> (0 to 10).toList.map(_.toDouble).runScan(average.flatMap(x => add(x) / length))
res4: List[Double] = List(1.0, 1.25, 1.5, 1.75, 2.0, 2.25, 2.5, 2.75, 3.0, 3.25)
\end{lstlisting}
The code can be rewritten as a functor block, making it more visually
clear:
\begin{lstlisting}
val average2a = for {
      x <- average
      accum <- add(x)
      n <- length
} yield accum / n

scala> (0 to 10).toList.map(_.toDouble).runScan(average2a)
res5: List[Double] = List(1.0, 1.25, 1.5, 1.75, 2.0, 2.25, 2.5, 2.75, 3.0, 3.25)
\end{lstlisting}
We get the same result as when applying \lstinline!runScan(average)!
twice. The result would not be obtained if we used the old value (\lstinline!r!)
in line 7 in the definition of \lstinline!flatMap! above.

The \lstinline!flatMap! function implements a monadic fusion of \lstinline!fold!-like
operations.\footnote{The \textsf{``}\texttt{origami}\textsf{''} library (see \texttt{\href{https://github.com/atnos-org/origami}{https://github.com/atnos-org/origami}})
implements a more general functionality: values computed by its \lstinline!fold!-like
operations are themselves of monadic type. This is similar to using
\lstinline!FoldOp[Z, R, M[A]]! where \lstinline!M! is a monad and
the updater function has type \lstinline!(R, Z) => M[R]!.} It is interesting to note that the monadic fold fusion is compatible
with the applicative fold fusion. For instance, one could define \lstinline!average!
equivalently as:

\begin{wrapfigure}{l}{0.28\columnwidth}%
\vspace{-1\baselineskip}
\begin{lstlisting}
val average = for {
  acc <- sum
  n <- length
} yield acc / n
\end{lstlisting}

\vspace{-1\baselineskip}
\end{wrapfigure}%

\noindent This equivalence comes from the fact that the case class
\lstinline!FoldOp[Z, R, A]! depends on \lstinline!A! only through
the \lstinline!transform! value, which is of type \lstinline!R => A!.
So, the monadic behavior of \lstinline!FoldOp[Z, R, A]! with respect
to \lstinline!A! is similar to that of the \lstinline!Reader! monad
of type \lstinline!R => A!. Similarly to the \lstinline!Reader!
monad, the effects in the \lstinline!FoldOp! monad are independent.

\subsection{Parsing with applicative and monadic combinators\label{subsec:Parsing-with-applicative-and-monadic-parsers}}

Applicative functors are used for implementing parsers via the \textsf{``}parser
combinator\textsf{''} technique. In this technique, the programmer builds
a large parser out of smaller ones by using different functions (\textsf{``}combinators\textsf{''}).
We will now look at both applicative and monadic parser combinators.

The parsing techniques will be illustrated on some toy \textsf{``}languages\textsf{''}.
The first language encodes square roots of positive integers in an
XML-like format, as, for example, in the strings \lstinline!<sqrt>121</sqrt>!,
\lstinline!<sqrt><sqrt>10000</sqrt></sqrt>!, and \lstinline!123!.
Examples of \emph{invalid} strings are \lstinline!<sqrt>100</sqrt></sqrt>!
(junk at end of text), \lstinline!<sqrt>121<sqrt>! (tag not closed),
\lstinline!</sqrt>10</sqrt>! (tag not opened), and \lstinline!1.0!
(not an integer). The parser should output an integer result or an
error message.

To simplify the code, we define a parser as a function taking an input
string and returning either a value of some type \lstinline!A! (the
value \textsf{``}parsed out\textsf{''} of the input) or some error information. The
parser also returns the unused part of the input string. So, we define
the type constructor \lstinline!P[A]! by:
\begin{lstlisting}
final case class P[A](run: String => (Either[Err, A], String)) // A parser returns a result or fails.
type Err = List[String]           // A list of error messages.
\end{lstlisting}

In order to parse the toy language, we need to be able to parse integers,
opening tags, closing tags, and detect whether the end of input occurs
at a point where all tags are balanced. The parser combinator technique
begins by defining the simplest necessary parsers. We will need a
parser for integers and a parser that expects the input to start with
a given, fixed string:
\begin{lstlisting}
val intP: P[Int] = P { 
  val numRegex = "^([0-9]+)(.*)$".r
  s => s match {
    case numRegex(num, rest)   => (Right(num.toInt), rest)
    case s                     => (Left(List("no number")), s)
  } 
}
def constP(prefix: String, error: String = "no prefix"): P[String] = P { s =>
  if (s startsWith prefix) (Right(prefix), s.stripPrefix(prefix)) else (Left(List(error)), s)
}
\end{lstlisting}
Let us test that these parsers work as expected:
\begin{lstlisting}
scala> intP.run("123xyz")
res0: (Either[Err, Int], String) = (Right(123), "xyz")

scala> constP("<sqrt>").run("<sqrt>1</sqrt>")
res1: (Either[Err, String], String) = (Right("<sqrt>"), "1</sqrt>")
\end{lstlisting}

The next step is to define combinators that produce larger parsers
from smaller ones. We will define four different combinators corresponding
to a functor\textsf{'}s \lstinline!map!, an applicative \lstinline!zip!,
a monoid\textsf{'}s \lstinline!combine!, and a monad\textsf{'}s \lstinline!flatMap!
methods. For convenience, we implement the combinators as extension
methods on the type constructor \lstinline!P!. Start with \lstinline!zip!
and use the code from Section~\ref{subsec:Programs-that-accumulate-errors}:
\begin{lstlisting}
implicit class ParserZipOps[A](parserA: P[A]) {
   def zip[B](parserB: P[B]): P[(A, B)] = P { s =>
      val (resultA, rest) = parserA.run(s)
      val (resultB, restB) = parserB.run(rest)
      val result = (resultA, resultB) match { // Use the `zip` operation for Either[List[String], A].
        case (Left(x), Left(y))     => Left(x ++ y)
        case (Left(x), Right(_))    => Left(x)
        case (Right(_), Left(y))    => Left(y)
        case (Right(x), Right(y))   => Right((x, y))
      }
      (result, restB)
    }
}
\end{lstlisting}
To test this, let us define a parser \lstinline!p1! for strings of
the form \lstinline!<sqrt>123</sqrt>!:
\begin{lstlisting}
val openTag = constP("<sqrt>", "tag must be open")
val closeTag = constP("</sqrt>", "tag must be closed")
val p1 = openTag zip intP zip closeTag

scala> p1.run("<sqrt>123</sqrt>")
res4: (Either[Err, ((String, Int), String)], String) = (Right((("<sqrt>", 123), "</sqrt>")), "")

scala> p1.run("<sqrt></sqrt>")
res5: (Either[Err, ((String, Int), String)], String) = (Left(List("no number")), "")

scala> p1.run("abc")
res6: (Either[Err, ((String, Int), String)], String) = (Left(List("tag must be open", "no number", "tag must be closed")), "abc")
\end{lstlisting}
Because the parser \lstinline!p1! is defined via the applicative
\lstinline!zip! method, the code can report several errors at once.
With suitable definitions, parsers combined via \lstinline!zip! could
skip some invalid text and attempt to parse further data, possibly
reporting more errors to the user.

It is inconvenient that a \lstinline!zip!-combined parser returns
a deeply nested tuple structure as shown above; we are often interested
in reading only one value inside that structure. We would like \lstinline!p1.run("<sqrt>123</sqrt>")!
to return just \lstinline!Right(123, "")!. To achieve this, we define
the helper methods \lstinline!zipLeft! and \lstinline!zipRight!,
whose names indicate the part of the tuple that is being returned:
\begin{lstlisting}
implicit class ParserMoreZipOps[A](parserA: P[A]) {
   def map[B](f: A => B): P[B] = P { s =>
      val (result, rest) = parserA.run(s)
      (result.map(f), rest)
   } // 
   def zipLeft [B](parserB: P[B]): P[A] = (parserA zip parserB).map(_._1)
   def zipRight[B](parserB: P[B]): P[B] = (parserA zip parserB).map(_._2)
}

scala> (openTag zipRight intP zipLeft closeTag).run("<sqrt>123</sqrt>")
res7: (Either[Err, Int], String) = (Right(123), "")
\end{lstlisting}

The next parser combinator corresponds to a monoid\textsf{'}s \lstinline!combine!,
but we will call it \textsf{``}\lstinline!or!\textsf{''} for convenience. The result
of \lstinline!(p or q)! is a parser that succeeds if either \lstinline!p!
or \lstinline!q! succeed and returns the corresponding result (the
parser \lstinline!p! is tried first):
\begin{lstlisting}
implicit class ParserCombineOps[A](p: P[A]) {
   def or(q: => P[A]): P[A] = P { s =>       // Important: the argument `q` must be lazy.
   val (result, rest) = p.run(s)
   result match {                            // If `p` failed to parse the string `s`,
     case Left(err)  => q.run(s)             // ignore the error from `p` and run `q`.
     case Right(x)   => (Right(x), rest)     // Otherwise, use the result of running `p`.
   }
}    
\end{lstlisting}
We implement \lstinline!q! as a lazy argument because we should not
evaluate \lstinline!q! when the parser \lstinline!p! succeeds. This
will help us write recursively defined parsers, as we will see next.

Using these combinators, we write a first attempt at parsing the toy
language. Since the language allows us to have arbitrarily deep nesting
of the tags \lstinline!<sqrt>...</sqrt>!, we need to define the parser
as a recursive function (called \lstinline!p2!):
\begin{lstlisting}
def p2: P[Int] = intP or (openTag zipRight p2 zipLeft closeTag).map { x => math.sqrt(x).toInt }

scala> p2.run("121")._1.right.get
res8: Int = 121

scala> p2.run("<sqrt>121</sqrt>")._1.right.get
res9: Int = 11
\end{lstlisting}
The recursion stops only because the operation \textsf{``}\lstinline!or!\textsf{''}
does not evaluate its second parser when the first parser succeeds.
Otherwise the code would go into an infinite loop.

We are able to parse the toy language using only applicative combinators.
Let us now consider a more complicated language that \emph{requires}
monadic combinators to be parsed correctly. The language represents
equations between integers, such as \lstinline!1=1! or \lstinline!123=123!,
but does not admit \lstinline!1=2!. To parse this language, we need
to get the first integer, skip the equals sign, and then check that
the second integer is equal to the first one. So, we need to use a
parser that depends on the result of parsing a previous substring.
We would like to implement the parser using this straightforward code:
\begin{lstlisting}
val emptyP: P[Unit] = P { s => if (s.isEmpty) (Right(()), s) else (Left(List("junk at end")), s) }

val p3: P[Int] = for {
  x <- intP                // Expect an integer.
  _ <- constP("=")         // Expect an equals sign.
  _ <- constP(x.toString)  // Expect a string corresponding to the previously parsed integer `x`.
  _ <- emptyP              // Expect end of input.
} yield x 
\end{lstlisting}
To make this code work, we define \lstinline!flatMap! as an extension
method on \lstinline!P!:
\begin{lstlisting}
implicit class ParserMonadOps[A](parserA: P[A]) {
   def flatMap[B](f: A => P[B]): P[B] = P { s =>
      val (result, rest) = parserA.run(s)
      result match {
        case Left(err)   => (Left(err), rest)
        case Right(x)    => f(x).run(rest)
      }
   }
}
\end{lstlisting}
The result is a parser \lstinline!p3! that that stops at the first
error:
\begin{lstlisting}
scala> p3.run("123=123")
res10: (Either[Err, Int], String) = (Right(123), "")

scala> p3.run("1=2")
res11: (Either[Err, Int], String) = (Left(List("integer 1 not found")), "2")
\end{lstlisting}

The error reporting in \lstinline!p3! can be improved if we use both
monadic and applicative combinators in the same code. Let us implement
and use a special parser (called \textsf{``}\lstinline!expect!\textsf{''}) that will
fail with an informative message when a parsed integer is not equal
to a given value:
\begin{lstlisting}
val success: P[Unit] = P { s => (Right(()), s) }  // Always-succeeding parser that consumes no input.
def failure(message: String): P[Unit] = P { s => (Left(List(message)), s) } // Always-failing parser.

def expect(n: Int): P[Unit] = intP.flatMap { x =>
  if (x == n) success else failure(s"got $x but expected $n")
}

val p4: P[Int] = for {
   x <- intP zipLeft constP("=")
   _ <- expect(x) zip emptyP
} yield x

scala> p4.run("1=2mumbo-jumbo gobbledy-gook")
res12: (Either[Err, Int], String) = (Left(List("got 2 but expected 1", "junk at end")), "mumbo-jumbo gobbledy-gook")
\end{lstlisting}

Several parser combinator libraries\footnote{See, for example, the \textsf{``}\texttt{fastparse}\textsf{''} library: \texttt{\href{https://com-lihaoyi.github.io/fastparse}{https://com-lihaoyi.github.io/fastparse}}}
are designed after these principles and provide applicative, monadic,
filterable, monoidal, and other functionality for building up large
parsers from parts.

\subsection{Functor block syntax for applicative functors}

It is often desirable to combine applicative and monadic computations
in a single functor block. This can be done by using \lstinline!zip!,
\lstinline!zipLeft!, and \lstinline!zipRight! within source lines,
as shown in the previous section.

\begin{wrapfigure}{l}{0.3\columnwidth}%
\vspace{-0.95\baselineskip}

\begin{lstlisting}
for {
  x       <-   p
  (y, z)  <-   q(x) zip r(x)
  t       <-   s(x, y, z)
  ...
\end{lstlisting}
\vspace{-0.95\baselineskip}
\end{wrapfigure}%

\noindent In the code shown at left, \lstinline!q! and \lstinline!r!
cannot depend on each other\textsf{'}s returned values (\lstinline!y! and
\lstinline!z!). However, \lstinline!q! and \lstinline!r! may depend
on any of the previously computed values (such as \lstinline!x!).
Later computations (such as \lstinline!s!) may also depend on the
previous values (\lstinline!x!, \lstinline!y!, \lstinline!z!).

There have been several proposals\footnote{See, for example, \texttt{\href{https://contributors.scala-lang.org/t/4474}{https://contributors.scala-lang.org/t/4474}}}
to add a special block syntax for applicative functors in Scala. So
far, these proposals have \emph{not} been accepted as a part of the
Scala language. Some experimental projects implement a \lstinline!for!/\lstinline!yield!
syntax for applicative functors.\footnote{See, for example, \texttt{\href{https://github.com/bkirwi/applicative-syntax}{https://github.com/bkirwi/applicative-syntax}}
where a special macro is used, which may not work with Scala 3. The
blog post \texttt{\href{https://lptk.github.io/programming/2018/03/02/a-dual-to-iterator.html}{https://lptk.github.io/programming/2018/03/02/a-dual-to-iterator.html}}
proposes a \lstinline!for!/\lstinline!yield! syntax for applicative
functors and implements single-traversal applicative folds without
using macros.}

\subsection{Exercises\index{exercises}}

\subsubsection{Exercise \label{subsec:Exercise-applicative-I}\ref{subsec:Exercise-applicative-I}}

Implement \lstinline!map2! (or \lstinline!xmap2! if appropriate)
for the following type constructors $F^{A}$:

\textbf{(a)} $F^{A}\triangleq\bbnum 1+A+A\times A$.

\textbf{(b)} $F^{A}\triangleq E\rightarrow A\times A$.

\textbf{(c)} $F^{A}\triangleq Z\times A\rightarrow A$.

\textbf{(d)} $F^{A}\triangleq A\rightarrow A\times Z$ where $Z$
is a \lstinline!Monoid!.

\subsubsection{Exercise \label{subsec:Exercise-applicative-I-1-1}\ref{subsec:Exercise-applicative-I-1-1}}

Implement a \lstinline!zip! method for a ternary tree \lstinline!T3[A]!
with extra data on branches:
\begin{lstlisting}
sealed trait T3[A]
case class Leaf[A](a: A) extends T3[A]
case class Branch[A](label: A, left: T3[A], center: T3[A], right: T3[A]) extends T3[A]

def zip[A, B](ta: T3[A], tb: T3[B]): T3[(A, B)] = ???
\end{lstlisting}
Aim to preserve information and avoid changing the tree structures
unnecessarily.

\subsubsection{Exercise \label{subsec:Exercise-applicative-I-1}\ref{subsec:Exercise-applicative-I-1}}

Write a \lstinline!zip! function that defines an instance of the
\lstinline!Semigroup! type class for a tuple \lstinline!(A, B)!
where the types \lstinline!A! and \lstinline!B! are assumed to already
have a \lstinline!Semigroup! instance. Test on an example with \lstinline!Semigroup!
instances for \lstinline!A = Int! and \lstinline!B = String!.

\subsubsection{Exercise \label{subsec:Exercise-applicative-I-2}\ref{subsec:Exercise-applicative-I-2}}

Define a \lstinline!Monoid! instance for the type $F^{S}$ where
$F$ is an applicative functor that has \lstinline!map2! and \lstinline!pure!,
while $S$ is itself a monoid type.

\subsubsection{Exercise \label{subsec:Exercise-applicative-I-3}\ref{subsec:Exercise-applicative-I-3}}

Define a \textsf{``}regexp extractor\textsf{''} as a type constructor $R^{A}$ describing
extraction of various data from strings; the extracted data has type
\lstinline!Option[A]!. Implement \lstinline!zip! and \lstinline!map2!
for $R^{A}$.

\subsubsection{Exercise \label{subsec:Exercise-applicative-I-5}\ref{subsec:Exercise-applicative-I-5}}

Use fold fusion (Section~\ref{subsec:Single-traversal-fold-operations-applicative-fold-fusion})
to implement a \lstinline!FoldOp! that computes the standard deviation
of a sequence of type \lstinline!Seq[Double]! in one traversal.

\subsubsection{Exercise \label{subsec:Exercise-applicative-I-5-1}\ref{subsec:Exercise-applicative-I-5-1}}

Use parser combinators defined in Section~\ref{subsec:Parsing-with-applicative-and-monadic-parsers}
(alternatively, use a parser combinator library of your choice) to
implement a parser for a toy language that contains a single integer
in a sequence of arbitrary nested tags, such as \lstinline!<a><b>123</b></a>!.
Tag names are restricted to lowercase Latin letters. The parser should
return that integer or fail if tags are incorrectly nested.

\subsubsection{Exercise \label{subsec:Exercise-applicative-I-4}\ref{subsec:Exercise-applicative-I-4}}

\textbf{(a)} Use parser combinators to parse a toy arithmetic language
containing decimal digits and \textsf{``}\lstinline!+!\textsf{''} symbols, for example
\lstinline!1+321+20!. The parser should return the resulting integer.

\textbf{(b){*}} Extend the parser to support parentheses and \textsf{``}\lstinline!-!\textsf{''}
symbols, for example \lstinline!10-(2+3)!.

\section{Laws and structure of applicative functors}

\subsection{Equivalence of \texttt{map2}, \texttt{zip}, and \texttt{ap}\label{subsec:Equivalence-of-map2-zip-ap}}

We have seen in Section~\ref{subsec:Programs-that-accumulate-errors}
that the implementations of \lstinline!map2! and \lstinline!zip!
for \lstinline!Either[E, A]! are quite similar. This similarity holds
in the general case: \lstinline!map2! and \lstinline!zip! are equivalent
assuming a suitable naturality law of \lstinline!map2!. (Naturality
laws will always hold if the code is fully parametric.)

\subsubsection{Statement \label{subsec:Statement-map2-zip-equivalence}\ref{subsec:Statement-map2-zip-equivalence}
(equivalence of \lstinline!map2! and \lstinline!zip!)}

For any functor $L$ for which \lstinline!map2! or \lstinline!zip!
can be implemented, the type of functions \lstinline!zip! (type signature
$\text{zip}:L^{A}\times L^{B}\rightarrow L^{A\times B}$) is equivalent
to the type of functions \lstinline!map2! (type signature $L^{A}\times L^{B}\rightarrow\left(A\times B\rightarrow C\right)\rightarrow L^{C}$),
assuming that the functions \lstinline!map2! satisfy the naturality
law with respect to the type parameter $C$.

\subparagraph{Proof}

We need to show equivalence in two directions (from \lstinline!map2!
to \lstinline!zip! and back).

\textbf{(1)} Given any function $\text{map}_{2}:L^{A}\times L^{B}\rightarrow\left(A\times B\rightarrow C\right)\rightarrow L^{C}$
satisfying the naturality law:
\[
\text{map}_{2}\,(p^{:L^{A}}\times q^{:L^{B}})(f^{:A\times B\rightarrow C})\triangleright(g^{:C\rightarrow D})^{\uparrow L}=\text{map}_{2}\,(p\times q)(f\bef g)\quad,
\]
we first define a \lstinline!zip! method:
\[
\text{zip}:L^{A}\times L^{B}\rightarrow L^{A\times B}\quad,\quad\quad\text{zip}\,(p^{:L^{A}}\times q^{:L^{B}})\triangleq\text{map}_{2}\,(p\times q)(\text{id}^{:A\times B\rightarrow A\times B})\quad,
\]
and then define a new \lstinline!map2!$^{\prime}$ function through
that \lstinline!zip! method:
\[
\text{map}_{2}^{\prime}\,(p^{:L^{A}}\times q^{:L^{B}})(f^{:A\times B\rightarrow C})\triangleq(p\times q)\triangleright\text{zip}\triangleright f^{\uparrow L}\quad.
\]
Then we need to show that $\text{map}_{2}^{\prime}=\text{map}_{2}$.
We apply $\text{map}_{2}^{\prime}$ to arbitrary arguments and write:
\begin{align*}
 & \text{map}_{2}^{\prime}\,(p\times q)(f)=(p\times q)\triangleright\text{zip}\triangleright f^{\uparrow L}=\text{map}_{2}\,(p\times q)(\text{id})\,\gunderline{\triangleright\,f^{\uparrow L}}\\
{\color{greenunder}\text{naturality law of }\text{map}_{2}:}\quad & =\text{map}_{2}\,(p\times q)(\text{id}\bef f)=\text{map}_{2}\,(p\times q)(f)\quad.
\end{align*}

\textbf{(2)} Given any function $\text{zip}:L^{A}\times L^{B}\rightarrow L^{A\times B}$,
we first define a \lstinline!map2! method:
\[
\text{map}_{2}\,(p^{:L^{A}}\times q^{:L^{B}})(f^{:A\times B\rightarrow C})\triangleq(p\times q)\triangleright\text{zip}\triangleright f^{\uparrow L}\quad,
\]
and then define a new \lstinline!zip!$^{\prime}$ function through
that \lstinline!map2! method:
\[
\text{zip}^{\prime}\,(p^{:L^{A}}\times q^{:L^{B}})\triangleq\text{map}_{2}\,(p\times q)(\text{id}^{:A\times B\rightarrow A\times B})\quad.
\]
Then we need to show that $\text{zip}^{\prime}=\text{zip}$. We apply
$\text{zip}^{\prime}$ to arbitrary arguments $p$, $q$ and write:
\[
\text{zip}^{\prime}\,(p^{:L^{A}}\times q^{:L^{B}})=\text{map}_{2}\,(p\times q)(\text{id}^{:A\times B\rightarrow A\times B})=(p\times q)\triangleright\text{zip}\triangleright\text{id}^{\uparrow L}=(p\times q)\triangleright\text{zip}=\text{zip}\,(p\times q)\quad.
\]

It remains to show that the \lstinline!map2! method will satisfy
the naturality law if defined via \lstinline!zip!:
\begin{align*}
 & \gunderline{\text{map}_{2}\,(p\times q)(f)}\triangleright g^{\uparrow L}=(p\times q)\triangleright\text{zip}\triangleright f^{\uparrow L}\triangleright g^{\uparrow L}=(p\times q)\triangleright\text{zip}\triangleright(f\bef g)^{\uparrow L}\\
{\color{greenunder}\text{definition of }\text{map}_{2}\text{ via }\text{zip}:}\quad & =\text{map}_{2}\,(p\times q)(f\bef g)\quad.
\end{align*}
$\square$

The equivalence of \lstinline!map2! and \lstinline!zip! follows
a pattern similar to one shown in Section~\ref{subsec:Yoneda-identities}:
a function with one type parameter (a natural transformation) is equivalent
to a function with two type parameters if a naturality law holds with
respect to one of those type parameters. Here \lstinline!zip! is
playing the role of the natural transformation, and \lstinline!map2!
is a function obeying a naturality law. The proof of Statement~\ref{subsec:Statement-map2-zip-equivalence}
is similar to the proof of Statement~\ref{subsec:Statement-tr-equivalent-to-ftr}.

However, \lstinline!map2! is not a \textsf{``}lifting\textsf{''} in the sense of
Section~\ref{subsec:Yoneda-identities}. The method \textsf{``}\lstinline!ap!\textsf{''}
better fits the role of a lifting since its type signature ($\text{ap}:L^{A\rightarrow B}\rightarrow L^{A}\rightarrow L^{B}$)
transforms functions wrapped under $L$ (i.e., functions of type $L^{A\rightarrow B}$)
to functions of type $L^{A}\rightarrow L^{B}$. So, let us prove that
\lstinline!ap! is equivalent to \lstinline!map2!. Since the type
signature of \lstinline!ap! is curried, it is more convenient to
use the curried version (\lstinline!fmap2!) instead of \lstinline!map2!.
Let us write the relationship between \lstinline!ap! and \lstinline!fmap2!
in the code notation:%
\begin{comment}
precarious formatting
\end{comment}

\begin{wrapfigure}{i}{0.34\columnwidth}%
\vspace{-1.7\baselineskip}
\[
\xymatrix{\xyScaleY{0.4pc}\xyScaleX{2pc} & F^{B\rightarrow C}\ar[rd]\sp(0.5){\text{ap}^{B,C}}\\
F^{A}\ar[ru]\sp(0.45){f^{\uparrow L}}\ar[rr]\sb(0.43){\text{fmap}_{2}\,(f^{:A\rightarrow B\rightarrow C})} &  & \left(F^{B}\rightarrow F^{C}\right)
}
\]
\vspace{-2\baselineskip}
\end{wrapfigure}%

~\vspace{-0.8\baselineskip}
\begin{align*}
 & \text{fmap}_{2}\,(f^{:A\rightarrow B\rightarrow C})=f^{\uparrow L}\bef\text{ap}^{B,C}\quad,\\
 & \text{ap}^{A,B}=\text{fmap}_{2}\,(\text{id}^{:\left(A\rightarrow B\right)\rightarrow A\rightarrow B})\quad.
\end{align*}

We are now ready to prove the equivalence of \lstinline!ap! and \lstinline!fmap2!:

\subsubsection{Statement \label{subsec:Statement-fmap2-equivalence-to-ap}\ref{subsec:Statement-fmap2-equivalence-to-ap}
(equivalence of \lstinline!ap! and \lstinline!fmap2!)}

For any functor $L$ for which \lstinline!fmap2! or \lstinline!ap!
can be implemented, the type of functions \lstinline!ap! (type signature
$\text{ap}:L^{A\rightarrow B}\rightarrow L^{A}\rightarrow L^{B}$)
is equivalent to the type of functions \lstinline!fmap2! (type signature
$\left(A\rightarrow B\rightarrow C\right)\rightarrow L^{A}\rightarrow L^{B}\rightarrow L^{C}$),
assuming that the functions \lstinline!fmap2! satisfy the naturality
law with respect to the type parameter $C$.

\subparagraph{Proof}

We need to show equivalence in two directions (from \lstinline!fmap2!
to \lstinline!ap! and back).

\textbf{(1)} Given any function $\text{fmap}_{2}:\left(A\rightarrow B\rightarrow C\right)\rightarrow L^{A}\rightarrow L^{B}\rightarrow L^{C}$
satisfying the naturality law,
\[
\text{fmap}_{2}\,(g^{:X\rightarrow A}\bef f^{:A\rightarrow B\rightarrow C})(p^{:L^{X}})=\text{fmap}_{2}\,(f)(p\triangleright g^{\uparrow L})\quad,
\]
we first define an \lstinline!ap! method:
\[
\text{ap}:L^{A\rightarrow B}\rightarrow L^{A}\rightarrow L^{B}\quad,\quad\quad\text{ap}\,(r^{:L^{A\rightarrow B}})\triangleq\text{fmap}_{2}\,(\text{id}^{:(A\rightarrow B)\rightarrow A\rightarrow B})(r)\quad,
\]
and then define a new \lstinline!fmap2!$^{\prime}$ function through
that \lstinline!ap! method:
\[
\text{fmap}_{2}^{\prime}\,(f^{:A\rightarrow B\rightarrow C})(p^{:L^{A}})\triangleq p\triangleright f^{\uparrow L}\triangleright\text{ap}\quad.
\]
Then we need to show that $\text{fmap}_{2}^{\prime}=\text{fmap}_{2}$.
We apply $\text{fmap}_{2}^{\prime}$ to arbitrary arguments and write:
\begin{align*}
 & \text{fmap}_{2}^{\prime}\,(f^{:A\rightarrow B\rightarrow C})(p^{:L^{A}})=p\triangleright f^{\uparrow L}\triangleright\text{ap}=\text{fmap}_{2}\,(\text{id})(p\triangleright f^{\uparrow L})\\
{\color{greenunder}\text{naturality law of }\text{fmap}_{2}:}\quad & =\text{fmap}_{2}\,(f\bef\text{id})(p)=\text{fmap}_{2}\,(f)(p)\quad.
\end{align*}

\textbf{(2)} Given any function $\text{ap}:L^{A\rightarrow B}\rightarrow L^{A}\rightarrow L^{A}$,
we first define \lstinline!fmap2!:
\[
\text{fmap}_{2}\,(f^{:A\rightarrow B\rightarrow C})\triangleq f^{\uparrow L}\bef\text{ap}\quad,
\]
and then define a new \lstinline!ap!$^{\prime}$ function through
that \lstinline!fmap2! method:
\[
\text{ap}^{\prime}\triangleq\text{fmap}_{2}\,(\text{id})\quad.
\]
Then we need to show that $\text{ap}^{\prime}=\text{ap}$. We write:
\[
\text{ap}^{\prime}=\text{fmap}_{2}\,(\text{id})=\text{id}^{\uparrow L}\bef\text{ap}=\text{ap}\quad.
\]

It remains to show that the \lstinline!fmap2! method will satisfy
the naturality law if defined via \lstinline!ap!:
\begin{align*}
 & \text{fmap}_{2}\,(g\bef f)(p)=p\triangleright(g\bef f)^{\uparrow L}\bef\text{ap}=p\triangleright g^{\uparrow L}\triangleright\gunderline{f^{\uparrow L}\bef\text{ap}}\\
{\color{greenunder}\text{definition of }\text{fmap}_{2}\text{ via }\text{ap}:}\quad & =(p\triangleright g^{\uparrow L})\triangleright\text{fmap}_{2}\,(f)=\text{fmap}_{2}\,(f)(p\triangleright g^{\uparrow L})\quad.
\end{align*}
$\square$

Since \lstinline!zip! is equivalent to \lstinline!map2! and \lstinline!map2!
is equivalent to \lstinline!ap!, it follows (assuming suitable naturality
laws) that \lstinline!zip! is equivalent to \lstinline!ap!. In Scala,
the relationship between \lstinline!zip! and \lstinline!ap! is coded
like this:
\begin{lstlisting}
def zip[A, B](la: L[A], lb: L[B]): L[(A, B)] =    // Assuming that `ap` is already defined.
  ap[B, (A, B)](la.map {a => (b: B) => (a, b)} )(lb)

def ap[A, B](lf: L[A => B])(la: L[A]): L[B] =     // Assuming that `zip` is already defined.
  zip(lf, la).map { case (f, a) => f(a) }
\end{lstlisting}


\subsubsection{Statement \label{subsec:Statement-zip-ap-equivalence}\ref{subsec:Statement-zip-ap-equivalence}}

The functions \lstinline!zip! and \lstinline!ap! are equivalent
assuming the following naturality laws:
\[
(f^{\uparrow L}\boxtimes\text{id})\bef\text{zip}=\text{zip}\bef(f\boxtimes\text{id})^{\uparrow L}\quad,\quad\big(\text{ap}\,(r^{:L^{A\rightarrow B}})(p^{:L^{A}})\big)\triangleright(f^{:B\rightarrow C})^{\uparrow L}=\text{ap}\big(r\triangleright(x^{:A\rightarrow B}\rightarrow x\bef f)^{\uparrow L}\big)(p)\quad.
\]


\subparagraph{Proof}

We need to show equivalence in two directions (from \lstinline!zip!
to \lstinline!ap! and back). To write the relationships between \lstinline!zip!
and \lstinline!ap! more easily, let us denote by \lstinline!pair!
and \lstinline!eval! the following two functions:
\[
\text{pair}^{:A\rightarrow B\rightarrow A\times B}\triangleq a\rightarrow b\rightarrow a\times b\quad,\quad\quad\text{eval}^{:\left(A\rightarrow B\right)\times A\rightarrow B}\triangleq g^{:A\rightarrow B}\times a^{:A}\rightarrow g(a)\quad.
\]
The code of these fully parametric functions follows from their types.
Then we can write:
\[
\text{zip}\,(p^{:L^{A}}\times q^{:L^{B}})=\text{ap}\,(p\triangleright\text{pair}^{\uparrow L})(q)\quad,\quad\quad\text{ap}\,(r^{:L^{A\rightarrow B}})(p^{:L^{A}})=(r\times p)\triangleright\text{zip}\triangleright\text{eval}^{\uparrow L}\quad.
\]

The functions \lstinline!pair! and \lstinline!eval! satisfy the
following property that we will use in the proof:
\begin{equation}
(\text{pair}^{:A\rightarrow B\rightarrow A\times B}\boxtimes\text{id}^{:B\rightarrow B})\bef\text{eval}^{:\left(B\rightarrow A\times B\right)\times B\rightarrow A\times B}=\text{id}^{:A\times B\rightarrow A\times B}\quad.\label{eq:pair-and-eval-property-derivation1}
\end{equation}
To prove this property, apply both sides to arbitrary $a^{:A}$ and
$b^{:B}$:
\[
(a^{:A}\times b^{:B})\triangleright(\text{pair}\boxtimes\text{id})\bef\text{eval}=\big((z^{:B}\rightarrow a\times z)\times b\big)\triangleright\text{eval}=(z^{:B}\rightarrow a\times z)(b)=a\times b\quad.
\]

\textbf{(1)} Given any function $\text{zip}:L^{A}\times L^{B}\rightarrow L^{A\times B}$
satisfying the naturality law shown above, we define an \lstinline!ap!
method by $\text{ap}\,(r)(p)\triangleq(r\times p)\triangleright\text{zip}\triangleright\text{eval}^{\uparrow L}$
and then define a new \lstinline!zip!$^{\prime}$ function through
that \lstinline!ap!:
\[
\text{zip}^{\prime}\,(p^{:L^{A}}\times q^{:L^{B}})\triangleq\text{ap}\,(p\triangleright\text{pair}^{\uparrow L})(q)\quad.
\]
Then we need to show that $\text{zip}^{\prime}=\text{zip}$. We apply
$\text{zip}^{\prime}$ to arbitrary arguments and write:
\begin{align*}
{\color{greenunder}\text{expect to equal }\text{zip}\,(p\times q):}\quad & \text{zip}^{\prime}\,(p^{:L^{A}}\times q^{:L^{B}})=\text{ap}\,(p\triangleright\text{pair}^{\uparrow L})(q)\\
{\color{greenunder}\text{definition of }\text{ap}:}\quad & =\big((p\triangleright\text{pair}^{\uparrow L})\times q\big)\triangleright\text{zip}\bef\text{eval}^{\uparrow L}=(p\times q)\triangleright\gunderline{(\text{pair}^{\uparrow L}\times\text{id})\bef\text{zip}}\bef\text{eval}^{\uparrow L}\\
{\color{greenunder}\text{naturality law of }\text{zip}:}\quad & =(p\times q)\triangleright\text{zip}\bef\gunderline{(\text{pair}\boxtimes\text{id})^{\uparrow L}\bef\text{eval}^{\uparrow L}}\\
{\color{greenunder}\text{use Eq.~(\ref{eq:pair-and-eval-property-derivation1})}:}\quad & =(p\times q)\triangleright\text{zip}\bef\text{id}^{\uparrow L}=(p\times q)\triangleright\text{zip}\quad.
\end{align*}
It remains to show that the \lstinline!ap! method defined via \lstinline!zip!
will satisfy its required naturality law:
\begin{align*}
{\color{greenunder}\text{left-hand side}:}\quad & \big(\text{ap}\,(r)(p)\big)\triangleright f^{\uparrow L}=(r\times p)\triangleright\text{zip}\bef\text{eval}^{\uparrow L}\bef f^{\uparrow L}\quad,\\
{\color{greenunder}\text{right-hand side}:}\quad & \text{ap}\big(r\triangleright(x\rightarrow x\bef f)^{\uparrow L}\big)(p)=\big((r\triangleright(x\rightarrow x\bef f)^{\uparrow L})\times p\big)\triangleright\text{zip}\bef\text{eval}^{\uparrow L}\\
{\color{greenunder}\text{naturality law of }\text{zip}:}\quad & \quad=(r\times p)\triangleright\text{zip}\bef\big((x\rightarrow x\bef f)\boxtimes\text{id}\big)^{\uparrow L}\bef\text{eval}^{\uparrow L}\quad.
\end{align*}
The difference between the two sides of the law will disappear if
we show that
\[
\text{eval}\bef f\overset{?}{=}\big((x\rightarrow x\bef f)\boxtimes\text{id}\big)\bef\text{eval}\quad.
\]
Apply both sides to an arbitrary pair $g\times a$ of type $\left(A\rightarrow B\right)\times A$
and obtain equal results:
\begin{align*}
{\color{greenunder}\text{left-hand side}:}\quad & (g^{:A\rightarrow B}\times a^{:A})\triangleright\text{eval}\bef f=(g\times a)\triangleright\text{eval}\triangleright f=g(a)\triangleright f=a\triangleright g\triangleright f\quad,\\
{\color{greenunder}\text{right-hand side}:}\quad & (g\times a)\triangleright\big((x\rightarrow x\bef f)\boxtimes\text{id}\big)\bef\text{eval}=(g\times a)\triangleright\big((x\rightarrow x\bef f)\boxtimes\text{id}\big)\triangleright\text{eval}\\
 & \quad=\big((g\bef f)\times a\big)\triangleright\text{eval}=a\triangleright g\bef f=a\triangleright g\bef f\quad.
\end{align*}

\textbf{(2)} Given any function $\text{ap}:L^{A\rightarrow B}\rightarrow L^{A}\rightarrow L^{A}$,
we first define \lstinline!zip!:
\[
\text{zip}\,(p^{:L^{A}}\times q^{:L^{B}})\triangleq\text{ap}\big(p\triangleright\text{pair}^{\uparrow L}\big)(q)\quad,
\]
and then define a new \lstinline!ap!$^{\prime}$ function through
that \lstinline!zip! method:
\[
\text{ap}^{\prime}\,(r^{:L^{A\rightarrow B}})(p^{:L^{A}})\triangleq(r\times p)\triangleright\text{zip}\triangleright\text{eval}^{\uparrow L}\quad.
\]
Then we need to show that $\text{ap}^{\prime}=\text{ap}$. We write:
\begin{align*}
 & \text{ap}^{\prime}\,(r)(p)=(r\times p)\triangleright\text{zip}\triangleright\text{eval}^{\uparrow L}=\big(\text{ap}\,(r\triangleright\text{pair}^{\uparrow L})(p)\gunderline{\big)\triangleright\text{eval}^{\uparrow L}}\quad.\\
{\color{greenunder}\text{naturality law of }\text{ap}:}\quad & =\text{ap}\big(r\,\gunderline{\triangleright\,\text{pair}^{\uparrow L}\triangleright(x^{:A\rightarrow\left(A\rightarrow B\right)\times A}\rightarrow x\bef\text{eval})^{\uparrow L}}\big)(p)\\
{\color{greenunder}\text{composition under }^{\uparrow L}:}\quad & =\text{ap}\big(r\triangleright(\text{pair}\bef(x\rightarrow x\bef\text{eval}))^{\uparrow L}\big)(p)\quad.
\end{align*}
The remaining difference between $\text{ap}^{\prime}(r)(p)$ and $\text{ap}\,(r)(p)$
will disappear if we show that 
\[
\text{pair}^{:\left(A\rightarrow B\right)\rightarrow A\rightarrow\left(A\rightarrow B\right)\times A}\bef(x^{:A\rightarrow\left(A\rightarrow B\right)\times A}\rightarrow x\bef\text{eval})\overset{?}{=}\text{id}\quad.
\]
To see this, apply both sides to an arbitrary value $f$ of type $A\rightarrow B$:
\begin{align*}
{\color{greenunder}\text{expect to equal }f:}\quad & f^{:A\rightarrow B}\triangleright\text{pair}^{:\left(A\rightarrow B\right)\rightarrow A\rightarrow\left(A\rightarrow B\right)\times A}\bef(x^{:A\rightarrow\left(A\rightarrow B\right)\times A}\rightarrow x\bef\text{eval})\\
 & =(a^{:A}\rightarrow f\times a)\bef(x\rightarrow x\bef\text{eval})=a^{:A}\rightarrow(f\times a)\triangleright\text{eval}=a^{:A}\rightarrow f(a)=f\quad.
\end{align*}

Finally, we show that the \lstinline!zip! method will satisfy its
naturality law if defined via \lstinline!ap!:
\begin{align*}
{\color{greenunder}\text{left-hand side}:}\quad & (p^{:L^{A}}\times q^{:L^{B}})\triangleright(f^{\uparrow L}\boxtimes\text{id})\bef\text{zip}=\text{zip}\big((p\triangleright f^{\uparrow L})\times q\big)\\
{\color{greenunder}\text{express }\text{zip}\text{ via }\text{ap}:}\quad & \quad=\text{ap}\big(p\triangleright f^{\uparrow L}\triangleright\text{pair}^{\uparrow L})(q)=\text{ap}\big(p\triangleright(f\bef\text{pair})^{\uparrow L})(q)\quad,\\
{\color{greenunder}\text{right-hand side}:}\quad & (p^{:L^{A}}\times q^{:L^{B}})\triangleright\text{zip}\triangleright(f\boxtimes\text{id})^{\uparrow L}=\big(\text{ap}\,(p\triangleright\text{pair}^{\uparrow L})(q)\gunderline{\big)\triangleright(f\boxtimes\text{id})^{\uparrow L}}\\
{\color{greenunder}\text{naturality law of }\text{ap}:}\quad & \quad=\text{ap}\big(p\triangleright\text{pair}^{\uparrow L}\triangleright(x\rightarrow x\bef(f\boxtimes\text{id}))^{\uparrow L}\big)(q)\quad.
\end{align*}
The remaining difference will disappear if for any $f^{:Z\rightarrow A}$,
the following equation holds:
\[
f^{:Z\rightarrow A}\bef\text{pair}^{:A\rightarrow B\rightarrow A\times B}\overset{?}{=}\text{pair}^{:Z\rightarrow B\rightarrow Z\times B}\bef(x^{:B\rightarrow Z\times B}\rightarrow x\bef(f\boxtimes\text{id}))\quad.
\]
To verify this equation, substitute the definition of \lstinline!pair!
into both sides:
\begin{align*}
{\color{greenunder}\text{left-hand side}:}\quad & f\bef\text{pair}=(z^{:Z}\rightarrow f(z))\bef(a^{:A}\rightarrow b^{:B}\rightarrow a\times b)=z\rightarrow b\rightarrow f(z)\times b\quad,\\
{\color{greenunder}\text{right-hand side}:}\quad & \text{pair}\bef(x\rightarrow x\bef(f\boxtimes\text{id}))=(z^{:Z}\rightarrow b^{:B}\rightarrow z\times b)\bef(x^{:B\rightarrow Z\times B}\rightarrow x\bef(f\boxtimes\text{id}))\\
{\color{greenunder}\text{compute composition}:}\quad & \quad=z\rightarrow(b\rightarrow z\times b)\bef(f\boxtimes\text{id})=z\rightarrow b\rightarrow f(z)\times b\quad.
\end{align*}
Both sides are now equal. $\square$

\subsection{The \texttt{Zippable} and \texttt{Applicative} typeclasses\label{subsec:The-Zippable-and-Applicative-typeclass}}

The previous section showed that \lstinline!map2!, \lstinline!zip!,
and \lstinline!ap! are equivalent to each other, assuming that certain
naturality laws hold. (In practice, implementations of these methods
always satisfy the naturality laws with respect to each type parameter.)
We can encapsulate the functionality of these methods into a \textsf{``}zippable
functor\textsf{''} typeclass\footnote{A similar typeclass is called \textsf{``}\lstinline!Zip!\textsf{''} in \texttt{scalaz}
and \textsf{``}\lstinline!Semigroupal!\textsf{''} in \texttt{cats}.} whose \lstinline!map2! and \lstinline!ap! methods are defined via
\lstinline!zip!:\index{typeclass!Zippable@\texttt{Zippable}}
\begin{lstlisting}
abstract class Zippable[L[_]: Functor] {
  def zip[A, B](la: L[A], lb: L[B]): L[(A, B)]
  def map2[A, B, C](la: L[A], lb: L[B])(f: (A, B) => C): L[C] = zip(la, lb).map(Function.untupled(f))
  def ap[A, B](lf: L[A => B], la: L[A]): L[B] = zip(lf, la).map { case (f, a) => f(a) }
}
\end{lstlisting}
Instead of using \lstinline!zip!, either \lstinline!map2! or \lstinline!ap!
could be used to implement the other methods.

In addition to these methods, it is helpful to require a \lstinline!pure!
method for the functor \lstinline!L!. The resulting typeclass\index{typeclass!Applicative@\texttt{Applicative}}
is known as \lstinline!Applicative!. The \lstinline!pure! method
is equivalent to a \textsf{``}wrapped unit\textsf{''} (denoted \lstinline!wu!, see
Section~\ref{subsec:Pointed-functors-motivation-equivalence}). So,
the simplest definition of the \lstinline!Applicative! typeclass
contains just these two methods:
\begin{lstlisting}
trait Applicative[L[_]] {
  def zip[A, B](la: L[A], lb: L[B]): L[(A, B)]
  def wu: L[Unit]
}
\end{lstlisting}
Other methods (\lstinline!map2!, \lstinline!ap!, \lstinline!pure!)
can be defined separately (as extension methods) using the functor
instance for \lstinline!L!. However, this definition of the \lstinline!Applicative!
typeclass can be used also with type constructors \lstinline!L[A]!
that are not covariant in \lstinline!A!. So, we will use this definition
later in this chapter.

\subsection{Motivation for the laws of \texttt{map2}\label{subsec:Motivation-for-the-laws-of-map2}}

We now turn to laws that an applicative functor must satisfy. The
motivation for the laws comes from treating \lstinline!map2(la, lb)(f)!
as a replacement for the following code (assuming \lstinline!L! is
a monad):
\begin{lstlisting}
def monadicMap2(la: L[A], lb: L[B])(f: (A, B) => C): L[C] = for {
  x <- la
  y <- lb
} yield f(x, y)
\end{lstlisting}
Unlike the code of \lstinline!monadicMap2!, which uses the monad\textsf{'}s
methods, \lstinline!map2(la, lb)! assumes that the effects in \lstinline!la!
and \lstinline!lb! are independent of the wrapped values (such as
\lstinline!x! and \lstinline!y!). But the type signatures of \lstinline!monadicMap2!
and \lstinline!map2! are the same, and for some type constructors
(e.g., the \lstinline!Reader! monad) the results are also the same.
So, we will require that \lstinline!map2! should obey the same laws
as the code in \lstinline!monadicMap2!.

To derive the laws of \lstinline!map2!, we will specialize the laws
of monads (which are imposed on \lstinline!flatMap! and \lstinline!pure!)
to the code of the form of \lstinline!monadicMap2! and then express
those laws in terms of \lstinline!map2! and \lstinline!pure!.

We begin with the associativity law of monads, which says that these
three expressions are equal:

\vspace{0.2\baselineskip}

\noindent %
\begin{minipage}[c][1\totalheight][t]{0.3\columnwidth}%
\begin{lstlisting}
for {
  x <- la
  y <- lb
  z <- lc
} yield g(x, y, z)
\end{lstlisting}
%
\end{minipage}\hfill{}%
\begin{minipage}[c][1\totalheight][t]{0.3\columnwidth}%
\begin{lstlisting}
for {
  x <- la
  (y, z) <- for {
         yy <- lb
         zz <- lc
       } yield (yy, zz)
} yield g(x, y, z)
\end{lstlisting}
%
\end{minipage}\hfill{}%
\begin{minipage}[c][1\totalheight][t]{0.3\columnwidth}%
\begin{lstlisting}
for {
  (x, y) <- for {
              xx <- la
              yy <- lb
            } yield (xx, yy)
   z <- lc
} yield g(x, y, z)
\end{lstlisting}
%
\end{minipage}\\
Expressed via \lstinline!map2! and \lstinline!map3!, that law gives
an equation between the following function calls:
\begin{lstlisting}
map3(la, lb, lc) { (a, b, c) => g(a, b, c) }
  == map2(la, map2(lb, lc) { (b, c) => (b, c) } { case (x, (y, z)) => g(x, y, z) }
  == map2(map2(la, lb) { (a, b) => (a, b) }, lc) { case ((x, y), z)) => g(x, y, z) } 
\end{lstlisting}
In the code notation, this law is written as:
\begin{align*}
 & \text{map}_{3}\,(p^{:L^{A}}\times q^{:L^{B}}\times r^{:L^{C}})(g^{:A\times B\times C\rightarrow D})\\
 & =\text{map}_{2}\,(p\times\text{map}_{2}\,(q\times r)(\text{id}^{:B\times C\rightarrow B\times C}))\big(a^{:A}\times(b^{:B}\times c^{:C})\rightarrow g(a\times b\times c)\big)\\
 & =\text{map}_{2}\,(\text{map}_{2}\,(p\times q)(\text{id})\times r)\big((a^{:A}\times b^{:B})\times c^{:C}\rightarrow g(a\times b\times c)\big)\quad.
\end{align*}
This law guarantees that \lstinline!map3! can be expressed unambiguously
through \lstinline!map2! regardless of the order in which we group
the arguments. This is the \index{associativity law!of map2@of \texttt{map2}}\textbf{associativity
law} of \lstinline!map2!.

Next, we consider the monadic left identity law. That law says that
the following codes are equal:

\vspace{0.2\baselineskip}

\noindent %
\begin{minipage}[c][1\totalheight][t]{0.4\columnwidth}%
\begin{lstlisting}
for {
  x <- pure(a)
  y <- lb
} yield g(x, y)
\end{lstlisting}
%
\end{minipage}\hfill{}%
\begin{minipage}[c][1\totalheight][t]{0.4\columnwidth}%
\begin{lstlisting}
for {
  // No need to define x. 
  y <- lb
} yield g(a, y)
\end{lstlisting}
%
\end{minipage}

Writing this in terms of \lstinline!map2!, we obtain the equation:
\begin{lstlisting}
map2(pure(a), lb)(g) == lb.map { y => g(a, y) }
\end{lstlisting}

The monadic right identity law says that the following two expressions
are equal:

\vspace{0.2\baselineskip}

\noindent %
\begin{minipage}[c][1\totalheight][t]{0.4\columnwidth}%
\begin{lstlisting}
for {
  x <- la
  y <- pure(b)
} yield g(x, y)
\end{lstlisting}
%
\end{minipage}\hfill{}%
\begin{minipage}[c][1\totalheight][t]{0.4\columnwidth}%
\begin{lstlisting}
for {
  x <- la
  // No need to define y. 
} yield g(x, b)
\end{lstlisting}
%
\end{minipage}\\
Writing this in terms of \lstinline!map2!, we obtain the equation:
\begin{lstlisting}
map2(la, pure(b))(g) == la.map { x => g(x, b) }
\end{lstlisting}

In the code notation, the two \index{identity laws!of map2@of \texttt{map2}}\textbf{identity
laws} of \lstinline!map2! are written as:
\begin{align*}
{\color{greenunder}\text{left identity law}:}\quad & \text{map}_{2}\,(\text{pu}_{L}(a^{:A})\times q^{:L^{B}})(g^{:A\times B\rightarrow C})=q\triangleright(b^{:B}\rightarrow g(a\times b))^{\uparrow L}\quad,\\
{\color{greenunder}\text{right identity law}:}\quad & \text{map}_{2}\,(p^{:L^{A}}\times\text{pu}_{L}(b^{:B}))(g^{:A\times B\rightarrow C})=p\triangleright(a^{:A}\rightarrow g(a\times b))^{\uparrow L}\quad.
\end{align*}

To simplify and analyze the laws of \lstinline!map2!, we will now
derive the laws of \lstinline!zip! that will follow once we express
\lstinline!map2! via \lstinline!zip!.

\subsection{Deriving the laws of \texttt{zip} from the laws of \texttt{map2}\label{subsec:Deriving-the-laws-of-zip}}

To derive the laws of \lstinline!zip! that follow from the identity
and associativity laws of \lstinline!map2!, we express \lstinline!map2!
via \lstinline!zip! and then substitute into those laws:
\begin{equation}
\text{map}_{2}\,(p^{:L^{A}}\times q^{:L^{B}})(f^{:A\times B\rightarrow C})=\text{zip}\,(p\times q)\triangleright f^{\uparrow L}\quad.\label{eq:express-map2-via-zip}
\end{equation}

Begin with the associativity law and write its two sides separately:
\begin{align*}
{\color{greenunder}\text{left-hand side}:}\quad & \text{map}_{2}\,(p^{:L^{A}}\times\text{map}_{2}\,(q^{:L^{B}}\times r^{:L^{C}})(\text{id}^{:B\times C\rightarrow B\times C}))\big(a^{:A}\times(b^{:B}\times c^{:C})\rightarrow g(a\times b\times c)\big)\\
 & \quad=\text{zip}\,(p\times\text{zip}\,(q\times r))\triangleright(a\times(b\times c)\rightarrow g(a\times b\times c))^{\uparrow L}\\
{\color{greenunder}\text{right-hand side}:}\quad & \overset{!}{=}\text{map}_{2}\,(\text{map}_{2}\,(p\times q)(\text{id})\times r)\big((a^{:A}\times b^{:B})\times c^{:C}\rightarrow g(a\times b\times c)\big)\\
 & \quad=\text{zip}\,(\text{zip}\,(p\times q)\times r)\triangleright((a\times b)\times c)\rightarrow g(a\times b\times c))^{\uparrow L}\quad.
\end{align*}
Both sides now depend on an arbitrary function $g^{:A\times B\times C\rightarrow D}$
in almost the same way, except for a rearrangement of the nested products.
We can refactor the code to remove the dependency on $g$. Define
two functions $\varepsilon_{1,23}$ and $\varepsilon_{12,3}$ that
rearrange the tuples:
\[
\varepsilon_{1,23}\triangleq a^{:A}\times(b^{:B}\times c^{:C})\rightarrow a\times b\times c\quad,\quad\quad\varepsilon_{12,3}\triangleq(a^{:A}\times b^{:B})\times c\rightarrow a\times b\times c\quad.
\]
Then we can rewrite the two sides of the associativity law as:
\begin{align*}
{\color{greenunder}\text{left-hand side}:}\quad & \text{zip}\,(p\times\text{zip}\,(q\times r))\triangleright\varepsilon_{1,23}^{\uparrow L}\bef g^{\uparrow L}\\
{\color{greenunder}\text{right-hand side}:}\quad & \overset{!}{=}\text{zip}\,(\text{zip}\,(p\times q)\times r)\triangleright\varepsilon_{12,3}^{\uparrow L}\bef g^{\uparrow L}\quad.
\end{align*}
We can now omit $g^{\uparrow L}$ (see Exercise~\ref{subsec:Exercise-simplify-law-omit-lifted-function})
and obtain an equivalent law:
\begin{equation}
\text{zip}\big(p\times\text{zip}\left(q\times r\right)\big)\triangleright\varepsilon_{1,23}^{\uparrow L}\overset{!}{=}\text{zip}\big(\text{zip}\left(p\times q\right)\times r\big)\triangleright\varepsilon_{12,3}^{\uparrow L}\quad.\label{eq:zip-associativity-law-with-epsilons}
\end{equation}
A type diagram illustrating this law is shown below:
\[
\xymatrix{\xyScaleY{1.4pc}\xyScaleX{1.7pc} & L^{A\times B}\times L^{C}\ar[r]\sp(0.5){\text{zip}} & L^{(A\times B)\times C}\ar[rd]\sp(0.5){\varepsilon_{12,3}^{\uparrow L}} &  & L^{A\times(B\times C)}\ar[ld]\sb(0.5){\varepsilon_{1,23}^{\uparrow L}} & L^{A}\times L^{B\times C}\ar[l]\sb(0.5){\text{zip}}\\
L^{A}\times L^{B}\ar[r]\sp(0.55){\text{zip}} & L^{A\times B}\ar[u] &  & L^{A\times B\times C} &  & L^{B\times C}\ar[u] & L^{B}\times L^{C}\ar[l]\sb(0.5){\text{zip}}\\
 &  & p:L^{A}\ar[llu]\ar[rrruu] & q:L^{B}\ar[lllu]\ar[rrru] & r:L^{C}\ar[rru]\ar[llluu]
}
\]

The functions $\varepsilon_{1,23}$ and $\varepsilon_{12,3}$ implement
the conversion of the equivalent types $(A\times B)\times C$ and
$A\times(B\times C)$ to $A\times B\times C$. Since $L$ is a functor,
the lifted functions $\varepsilon_{1,23}^{\uparrow L}$ and $\varepsilon_{12,3}^{\uparrow L}$
produce equivalences between the types $L^{(A\times B)\times C}$,
$L^{A\times(B\times C)}$, and $L^{A\times B\times C}$. With these
equivalences in mind, and using the infix syntax for \lstinline!zip!,
we rewrite the associativity law in a simpler form:\index{associativity law!of zip@of \texttt{zip}}
\begin{equation}
p\,\,\text{zip}\,\,(q\,\,\text{zip}\,\,r)\cong(p\,\,\text{zip}\,\,q)\,\,\text{zip}\,\,r\quad.\label{eq:zip-associativity-law}
\end{equation}
In Eq.~(\ref{eq:zip-associativity-law}), the symbol $\cong$ denotes
equality up to the type equivalence. To obtain a real equation, one
would need to apply $\varepsilon_{1,23}^{\uparrow L}$ and $\varepsilon_{12,3}^{\uparrow L}$
at appropriate places. Apart from that, the law~(\ref{eq:zip-associativity-law})
has the usual form of an associativity law for a binary operation.

When writing laws with implied type equivalences, as in the formulation~(\ref{eq:zip-associativity-law}),
we help build the intuition about applicative laws. We also save time
because we do not write out a number of tuple-swapping functions.
To avoid errors, derivations using this technique must first check
that all types match up to tuple-swapping isomorphisms.

We now turn to the identity laws. Substitute Eq.~(\ref{eq:express-map2-via-zip})
into \lstinline!map2!\textsf{'}s left identity law:
\begin{align*}
{\color{greenunder}\text{left-hand side}:}\quad & \text{map}_{2}\,(\text{pu}_{L}(a^{:A})\times q^{:L^{B}})(g)=\text{zip}\,(\text{pu}_{L}(a)\times q)\triangleright g^{\uparrow L}\\
{\color{greenunder}\text{right-hand side}:}\quad & \overset{!}{=}q\triangleright(b^{:B}\rightarrow g(a\times b))^{\uparrow L}=q\triangleright(b^{:B}\rightarrow a\times b)^{\uparrow L}\triangleright g^{\uparrow L}\quad.
\end{align*}
We may remove the common function $g^{\uparrow L}$ from both sides
and get an equivalent law:
\[
\text{zip}\,(\text{pu}_{L}(a)\times q)\overset{!}{=}q\triangleright(b\rightarrow a\times b)^{\uparrow L}\quad.
\]
We express \lstinline!pure! through the \textsf{``}wrapped unit\textsf{''} value\index{wrapped@\textsf{``}wrapped unit\textsf{''} value}
(\lstinline!wu!) as we did in Section~\ref{subsec:Pointed-functors-motivation-equivalence}:
\[
\text{wu}:L^{\bbnum 1}\quad,\quad\quad\text{pu}_{L}(a)=\text{wu}\triangleright(1\rightarrow a)^{\uparrow L}\quad.
\]
We obtain:
\begin{align*}
{\color{greenunder}\text{left-hand side}:}\quad & \text{zip}\,(\text{pu}_{L}(a)\times q)=\text{zip}\big((\text{wu}\triangleright(1\rightarrow a)^{\uparrow L})\times q\big)\\
{\color{greenunder}\text{naturality law of }\text{zip}:}\quad & \quad=\text{zip}\,(\text{wu}\times q)\triangleright(1\times b^{:B}\rightarrow a\times b)^{\uparrow L}\\
{\color{greenunder}\text{right-hand side}:}\quad & \overset{!}{=}q\triangleright(b\rightarrow a\times b)^{\uparrow L}\quad.
\end{align*}
To simplify this equation further, we note that the function $(1\times b^{:B}\rightarrow a\times b)^{\uparrow L}$
in the left-hand side and the function $(b^{:B}\rightarrow a\times b)^{\uparrow L}$
in the right-hand side are similar. The only difference is between
the arguments $1\times b$ and $b$. But the types $\bbnum 1\times B$
and $B$ are equivalent. Let us define a function called \lstinline!ilu!
(\textsf{``}insert left unit\textsf{''}) implementing this equivalence:
\[
\text{ilu}:B\rightarrow\bbnum 1\times B\quad,\quad\quad\text{ilu}\triangleq b^{:B}\rightarrow1\times b\quad.
\]
Now we can express both sides of the law as:
\begin{equation}
\text{zip}\,(\text{wu}\times q)\triangleright(1\times b^{:B}\rightarrow a\times b)^{\uparrow L}\overset{!}{=}q\triangleright\text{ilu}^{\uparrow L}\triangleright(1\times b^{:B}\rightarrow a\times b)^{\uparrow L}\quad.\label{eq:left-identity-zip-derivation1}
\end{equation}
Both sides contain the function $(1\times b^{:B}\rightarrow a\times b)^{\uparrow L}$
with an arbitrary value $a^{:A}$ of arbitrary type $A$. Is it correct
to simplify the law by discarding that function? To show that it is,
we first set $A\triangleq\bbnum 1$ (since the type $A$ may be chosen
arbitrarily) and obtain the function 
\[
(1\times b^{:B}\rightarrow1\times b)^{\uparrow L}=\text{id}^{\uparrow L}=\text{id}^{:L^{\bbnum 1\times B}\rightarrow L^{\bbnum 1\times B}}\quad.
\]
Applying an identity function to both sides of Eq.~(\ref{eq:left-identity-zip-derivation1}),
we find:\index{identity laws!of zip@of \texttt{zip}}
\begin{equation}
\text{zip}\,(\text{wu}\times q)=q\triangleright\text{ilu}^{\uparrow L}\quad.\label{eq:zip-left-identity-law}
\end{equation}
This equation is a consequence of Eq.~(\ref{eq:left-identity-zip-derivation1}).
At the same time, we may apply the function $(1\times b^{:B}\rightarrow a\times b)^{\uparrow L}$,
this time with an arbitrary $a^{:A}$, to both sides of Eq.~(\ref{eq:zip-left-identity-law})
and recover Eq.~(\ref{eq:left-identity-zip-derivation1}). So, we
have justified the simplification of the left identity law to Eq.~(\ref{eq:zip-left-identity-law}).

Since $L$ is a functor, the conversion function $\text{ilu}^{\uparrow L}$
implements the type equivalence $L^{B}\cong L^{\bbnum 1\times B}$.
Denoting this equivalence by $\cong$ and using the infix syntax for
\lstinline!zip!, we rewrite the left identity law as:
\[
\text{wu}\,\,\text{zip}\,\,q\cong q\quad.
\]
Apart from the implied conversion function, this is a familiar form
of the left identity law: a value $q^{:L^{B}}$ remains unchanged
(up to type equivalence) when \textsf{``}zipping\textsf{''} it with the special value
\lstinline!wu!.

A similar derivation shows that the right identity law of \lstinline!zip!
is:
\[
\text{zip}\,(p^{:L^{A}}\times\text{wu})=p\triangleright\text{iru}^{\uparrow L}\quad,\quad\quad\text{iru}\triangleq a^{:A}\rightarrow a\times1\quad,
\]
or using the infix syntax and the implied equivalence between the
types $L^{A}$ and $L^{A\times\bbnum 1}$:
\[
p\,\,\text{zip}\,\,\text{wu}\cong p\quad.
\]
So, \lstinline!wu! is the \textsf{``}empty value\textsf{''} for the binary operation
\lstinline!zip!.

We have shown that the laws of \lstinline!map2! can be simplified
when formulated via \lstinline!zip!. In that formulation, the laws
of applicative functors are similar to the laws of a \emph{monoid}\index{monoid}
(see Example~\ref{subsec:tc-Example-Monoids}): the binary operation
is \lstinline!zip! and the empty value is \lstinline!wu!. 

\subsection{Commutative applicative functors and parallel computations\label{subsec:Commutative-applicative-functors}}

The main feature of applicative functors is that \lstinline!zip(p, q)!
uses the independence of effects expressed by \lstinline!p! and \lstinline!q!.
For some applicative functors, the effects of \lstinline!p! and \lstinline!q!
may be computed \emph{in parallel} without affecting the result of
\lstinline!zip!. For instance, assume that \lstinline!p! and \lstinline!q!
represent processes that compute values. Then \lstinline!zip(p, q)!
is a combined process that computes two values in parallel. We have
seen an implementation of this idea using the \lstinline!Future!
type in Section~\ref{subsec:Monadic-programs-with-independent-effects-future-applicative}.
The same property may hold for other applicative functors as long
as \lstinline!zip(p, q)! does not depend on the order in which the
effects of \lstinline!p! and \lstinline!q! are combined. The precise
condition is the \textsf{``}commutativity\textsf{''} property of the applicative functor.

\subsubsection{Definition \label{subsec:Definition-commutative-applicative}\ref{subsec:Definition-commutative-applicative}}

An applicative functor $L^{\bullet}$ is \textbf{commutative}\index{applicative functor!commutative}
if its \lstinline!zip! operation satisfies the commutativity law\index{commutativity law!of zip@of \texttt{zip}}
(in addition to the standard applicative laws):
\begin{equation}
\text{zip}\,(p^{:L^{A}}\times q^{:L^{B}})=\text{zip}\,(q\times p)\triangleright(b^{:B}\times a^{:A}\rightarrow a\times b)^{\uparrow L}\quad\text{or equivalently}:\quad\text{swap}\bef\text{zip}=\text{zip}\bef\text{swap}^{\uparrow L}\quad.\label{eq:commutativity-law-of-zip}
\end{equation}
Here we used the standard Scala function \lstinline!swap! defined
by $\text{swap}\triangleq a\times b\rightarrow b\times a$. 

The commutativity law says that \lstinline!zip! commutes with \lstinline!swap!.
If we use \lstinline!swap! as a type isomorphism between $A\times B$
and $B\times A$, we may write the commutativity law more concisely
and suggestively:
\[
p\,\,\text{zip}\,\,q\cong q\,\,\text{zip}\,\,p\quad.
\]

This definition agrees with that of the commutative monad (see Exercise~\ref{subsec:Exercise-1-monads-9-1}).
A monad\textsf{'}s commutativity is defined via code that is equivalent to
the \lstinline!map2! operation:\texttt{\textcolor{blue}{\footnotesize{}}}
\begin{lstlisting}
def map2[A, B, C](p: M[A], q: M[B])(f: (A, B) => C): M[C] = for {
    x <- p               // Effects of p and q are independent of
    y <- q               // the values x or y.
} yield f(x, y)
\end{lstlisting}
 So, the definition of a commutative monad is that the corresponding
\lstinline!map2! operation (defined via \lstinline!map! and \lstinline!flatMap!)
must satisfy the following commutativity property: 
\[
\text{map}_{2}(p\times q)(f)=\text{map}_{2}(q\times p)\big(b\times a\rightarrow f(a\times b)\big)\quad.
\]
Expressing \lstinline!map2! through \lstinline!zip! as $\text{map}_{2}(p\times q)(f)=\text{zip}\,(p\times q)\triangleright f^{\uparrow L}$,
we obtain Eq.~(\ref{eq:commutativity-law-of-zip}).

Most monads are not commutative (\lstinline!Option! and \lstinline!Reader!
are among the few exceptions). So, even when the effects are independent
of the previous values, it is not possible to parallelize monadic
code automatically. In contrast, most applicative functors turn out
to be commutative, so their \lstinline!zip! methods could (in principle)
have a parallel implementation. An exception is the parsing functor
from Section~\ref{subsec:Parsing-with-applicative-and-monadic-parsers}.
We cannot expect to be able to parse different parts of a file in
parallel, because correct parsing often depends on the success or
failure of parsing of previous portions of the data.

An example of a data structure that can automatically parallelize
computations is Apache Spark\textsf{'}s \lstinline!RDD! class.\index{Spark\textsf{'}s RDD data type@\texttt{Spark}\textsf{'}s \texttt{RDD} data type}
It is important that \lstinline!RDD[_]! is a commutative applicative
functor but \emph{not} a monad.\footnote{The \texttt{Spark} library does not support values of type \lstinline!RDD[RDD[A]]!.
The \lstinline!RDD! class\textsf{'}s \lstinline!cartesian! method has the
type signature corresponding to \lstinline!zip!. A method called
\textsf{``}\lstinline!flatMap!\textsf{''} exists but does not have the type signature
of a monad\textsf{'}s \lstinline!flatMap!. See \texttt{\href{https://spark.apache.org/docs/3.2.1/rdd-programming-guide.html}{https://spark.apache.org/docs/3.2.1/rdd-programming-guide.html}}} This agrees with the intuition that monadic programs are not automatically
parallelizable.

In many cases, a commutative applicative functor will also be a \emph{non-commutative}
monad. Some examples are the \lstinline!Either! and \lstinline!List!
functors. For these and other functors, the \lstinline!map2! and
\lstinline!zip! methods must be defined separately and not derived
from the monadic methods. If we \emph{did} choose to define the \lstinline!map2!
method via the monadic methods, there are two possible implementations:

\noindent %
\begin{minipage}[c][1\totalheight][t]{0.45\columnwidth}%
\begin{lstlisting}
map2(p, q)(f) == for { // Direct order.
    a <- p             // p: L[A]
    b <- q             // q: L[B]
} yield f(a, b)
\end{lstlisting}
%
\end{minipage}\hfill{}%
\begin{minipage}[c][1\totalheight][t]{0.45\columnwidth}%
\begin{lstlisting}
map2(p, q)(f) == for { // Inverse order.
    b <- q             // q: L[B]
    a <- p             // p: L[A]
} yield f(a, b)
\end{lstlisting}
%
\end{minipage}

\noindent The difference is only in the order in which the effects
of \lstinline!p! and \lstinline!q! are concatenated. The two corresponding
\lstinline!zip! functions differ by the permutation of the effects:
\[
\text{zip}\left(p\times q\right)\triangleq p\triangleright\text{flm}_{L}(a\rightarrow q\triangleright(b\rightarrow a\times b)^{\uparrow L})\quad,\quad\quad\text{zip}^{\prime}\left(p\times q\right)=\text{zip}\left(q\times p\right)\triangleright\text{swap}^{\uparrow L}\quad.
\]
For commutative monads, there will be no difference between $\text{zip}^{\prime}$
and $\text{zip}$. 

When an applicative functor is commutative, its associativity law
has a simplified form:

\subsubsection{Statement \label{subsec:Statement-associativity-law-of-zip-with-commutative}\ref{subsec:Statement-associativity-law-of-zip-with-commutative}}

The associativity law of a \emph{commutative} applicative functor
$L$ is equivalent to:
\begin{equation}
\text{zip}\big(p^{:L^{A}}\times\text{zip}\big(q^{:L^{B}}\times r^{:L^{C}}\big)\big)=\text{zip}\left(r\times\text{zip}\left(q\times p\right)\right)\triangleright(c^{:C}\times(b^{:B}\times a^{:A})\rightarrow a\times(b\times c))^{\uparrow L}\quad.\label{eq:associativity-law-of-zip-commutative}
\end{equation}
Up to the tuple-swapping isomorphism, Eq.~(\ref{subsec:Statement-associativity-law-of-zip-with-commutative})
has the form where only $p$ and $r$ are swapped:
\begin{equation}
p\,\,\text{zip}\,\,(q\,\,\text{zip}\,\,r)\cong r\,\,\text{zip}\,\,(q\,\,\text{zip}\,\,p)\quad.\label{eq:associativity-law-of-zip-commutative-short}
\end{equation}


\subparagraph{Proof}

We need to demonstrate two directions of the equivalence: \textbf{(1)}
Derive Eq.~(\ref{eq:associativity-law-of-zip-commutative-short})
assuming that the associativity law holds. \textbf{(2)} Derive the
associativity law assuming that Eq.~(\ref{eq:associativity-law-of-zip-commutative-short})
holds.

To save time, we will write $\cong$ to imply tuple-swapping isomorphisms
wherever necessary.

\textbf{(1)} Begin with the left-hand side of Eq.~(\ref{eq:associativity-law-of-zip-commutative-short})
and apply Eq.~(\ref{eq:zip-associativity-law}):
\begin{align*}
 & p\,\,\text{zip}\,\,(q\,\,\text{zip}\,\,r)\cong\gunderline{(p\,\,\text{zip}\,\,q)}\,\,\text{zip}\,\,\gunderline r\\
{\color{greenunder}\text{twice use the commutativity law of }\text{zip}:}\quad & \cong r\,\,\text{zip}\,\,(\gunderline p\,\,\text{zip}\,\,\gunderline q)\cong r\,\,\text{zip}\,\,(q\,\,\text{zip}\,\,p)\quad.
\end{align*}
This is the right-hand side of Eq.~(\ref{eq:associativity-law-of-zip-commutative-short})

\textbf{(2)} Begin with the right-hand side of Eq.~(\ref{eq:associativity-law-of-zip-commutative-short})
and apply the commutativity law:
\begin{align*}
 & r\,\,\text{zip}\,\,(\gunderline q\,\,\text{zip}\,\,\gunderline p)=\gunderline r\,\,\text{zip}\,\,\gunderline{(p\,\,\text{zip}\,\,q)}=(p\,\,\text{zip}\,\,q)\,\,\text{zip}\,\,r\\
{\color{greenunder}\text{use Eq.~(\ref{eq:associativity-law-of-zip-commutative-short})}:}\quad & \overset{!}{=}p\,\,\text{zip}\,\,(q\,\,\text{zip}\,\,r)\quad.
\end{align*}
We have derived Eq.~(\ref{eq:zip-associativity-law}). $\square$

This statement makes it easier to check the associativity law of \lstinline!zip!
if we know that the commutativity law holds. We just need to write
the code for $\text{zip}\left(p\times\text{zip}\left(q\times r\right)\right)$
and swap $p^{:L^{A}}$ and $r^{:L^{C}}$ in that code. The result
must be the same up to swapping the result values of types $A$ and
$C$.

For commutative applicative functors, it is also sufficient to check
only \emph{one} of the identity laws:

\subsubsection{Statement \label{subsec:Statement-identity-law-of-zip-with-commutative-applicative}\ref{subsec:Statement-identity-law-of-zip-with-commutative-applicative}}

The two identity laws of a commutative applicative functor $L$ are
equivalent.

\subparagraph{Proof}

We need to show the two directions of the isomorphism: 

\textbf{(1)} Derive $p\,\,\text{zip}\,\,\text{wu}\cong p$ from $\text{wu}\,\,\text{zip}\,\,q\cong q$.
Use the commutativity law:
\[
\gunderline p\,\,\text{zip}\,\,\gunderline{\text{wu}}\cong\text{wu}\,\,\text{zip}\,\,p\cong p\quad,
\]
where we use the left identity law with $q\triangleq p$.

\textbf{(2)} Derive $\text{wu}\,\,\text{zip}\,\,q\cong q$ from $p\,\,\text{zip}\,\,\text{wu}\cong p$.
Use the commutativity law:
\[
\gunderline{\text{wu}}\,\,\text{zip}\,\,\gunderline q\cong q\,\,\text{zip}\,\,\text{wu}\cong q\quad,
\]
where we use the right identity law with $p\triangleq q$. $\square$

\subsection{Constructions of applicative functors\label{subsec:Constructions-of-applicative-functors}}

We will now perform structural analysis of applicative functors, going
through all possible type constructions. In each case, we will implement
the \lstinline!zip! and \lstinline!wu! methods and verify their
laws, indicating whether the commutativity law holds. We will also
compare the applicative functor constructions to the corresponding
constructions for monads (Section~\ref{subsec:Structural-analysis-of-monads}).
As before, we say \textsf{``}$F$ is applicative\textsf{''} when there is at least
one lawful implementation of \lstinline!zip! and \lstinline!wu!.

Note that any monad will already have a lawful implementation of \lstinline!zip!
and \lstinline!wu! since the laws of \lstinline!map2! and \lstinline!zip!
were derived directly from the monad laws. If the monad is non-commutative,
there will be two inequivalent implementations of \lstinline!zip!.
So, each monad construction gives at least one applicative construction.
We will focus on constructions that generate applicative functors
and have no analogous monad constructions.

\paragraph{Type parameters}

Three type constructions are based on using just type parameters:
a constant functor, $L^{A}\triangleq Z$, the identity functor $L^{A}\triangleq A$,
and the functor composition, $L^{A}\triangleq F^{G^{A}}$.

Given a fixed monoid type $Z$, the constant functor $L^{A}\triangleq Z$
is applicative. To see this, consider the operation \lstinline!zip!
of type $L^{A}\times L^{B}\rightarrow L^{A\times B}$, which in this
case becomes just $Z\times Z\rightarrow Z$, the monoid $Z$\textsf{'}s \lstinline!combine!
function. The value \lstinline!wu! of type $L^{\bbnum 1}=Z$ is just
the monoid $Z$\textsf{'}s empty value ($e_{Z}$). The laws of \lstinline!zip!
then reduce to the monoid laws of $Z$. For the same reason, the applicative
functor $L^{A}\triangleq Z$ will be commutative if the monoid $Z$
is commutative.

Comparing this with the corresponding monad construction, we note
that $L^{A}\triangleq Z$ is not a monad unless $Z=\bbnum 1$. For
an arbitrary monoid type $Z$, the functor $L^{A}\triangleq Z$ is
only a semimonad.

The identity functor $L^{A}\triangleq A$ is applicative and commutative:
the \lstinline!zip! operation is the identity function of type $A\times B\rightarrow A\times B$,
and the wrapped unit (\lstinline!wu!) is just the unit value. All
required laws hold for the identity function.

The functor composition $L\triangleq F\circ G$ of two applicative
functors $F$ and $G$ is again applicative:

\subsubsection{Statement \label{subsec:Statement-applicative-composition}\ref{subsec:Statement-applicative-composition}}

For any two applicative functors $F$ and $G$ with lawful \lstinline!zip!
and \lstinline!pure! methods, the functor $L^{A}\triangleq F^{G^{A}}$
is also applicative. Its \lstinline!zip! and \lstinline!pure! methods
are defined by:
\begin{align*}
 & \text{zip}_{L}:F^{G^{A}}\times F^{G^{B}}\rightarrow F^{G^{A\times B}}\quad,\quad\quad\text{zip}_{L}\triangleq\text{zip}_{F}\bef\text{zip}_{G}^{\uparrow F}\quad,\\
 & \text{pu}_{L}:A\rightarrow F^{G^{A}}\quad,\quad\quad\text{pu}_{L}\triangleq\text{pu}_{G}\bef\text{pu}_{F}\quad,\quad\quad\text{wu}_{L}\triangleq\text{pu}_{F}(\text{wu}_{G})\quad.
\end{align*}
If $F$ and $G$ are commutative applicative functors, then so is
$L$.

\subparagraph{Proof}

We note that the lifting to $L$ is defined by $f^{\uparrow L}\triangleq f^{\uparrow G\uparrow F}=(f^{\uparrow G})^{\uparrow F}$.

To verify the left identity law~(\ref{eq:zip-left-identity-law}):
\begin{align*}
{\color{greenunder}\text{expect to equal }p\triangleright\text{ilu}^{\uparrow L}:}\quad & \text{zip}_{L}(\text{wu}_{L}\times p^{:F^{G^{A}}})=(\gunderline{\text{pu}_{F}(\text{wu}_{G})}\times p)\triangleright\gunderline{\text{zip}_{F}}\bef\text{zip}_{G}^{\uparrow F}\\
{\color{greenunder}\text{left identity law of }\text{zip}_{F}:}\quad & =p\triangleright(g^{:G^{A}}\rightarrow\gunderline{\text{wu}_{G}\times g)^{\uparrow F}\bef\text{zip}_{G}^{\uparrow F}}\\
{\color{greenunder}\text{composition under }^{\uparrow F}:}\quad & =p\triangleright\big(g^{:G^{A}}\rightarrow\gunderline{\text{zip}_{G}(\text{wu}_{G}\times g)}\big)^{\uparrow F}\\
{\color{greenunder}\text{left identity law of }\text{zip}_{G}:}\quad & =p\triangleright(\gunderline{g\rightarrow g}\triangleright\text{ilu}^{\uparrow G})^{\uparrow F}=p\triangleright(\text{ilu}^{\uparrow G})^{\uparrow F}=p\triangleright\text{ilu}^{\uparrow L}\quad.
\end{align*}

To verify the right identity law:
\begin{align*}
{\color{greenunder}\text{expect to equal }p\triangleright\text{iru}^{\uparrow L}:}\quad & \text{zip}_{L}(p^{:F^{G^{A}}}\times\text{wu}_{L})=(p\times\gunderline{\text{pu}_{F}(\text{wu}_{G})})\triangleright\gunderline{\text{zip}_{F}}\bef\text{zip}_{G}^{\uparrow F}\\
{\color{greenunder}\text{right identity law of }\text{zip}_{F}:}\quad & =p\triangleright(g^{:G^{A}}\rightarrow\gunderline{g\times\text{wu}_{G})^{\uparrow F}\bef\text{zip}_{G}^{\uparrow F}}=p\triangleright\big(g^{:G^{A}}\rightarrow\gunderline{\text{zip}_{G}(g\times\text{wu}_{G})}\big)^{\uparrow F}\\
{\color{greenunder}\text{right identity law of }\text{zip}_{G}:}\quad & =p\triangleright(\gunderline{g\rightarrow g\,\triangleright}\,\text{iru}^{\uparrow G})^{\uparrow F}=p\triangleright(\text{iru}^{\uparrow G})^{\uparrow F}=p\triangleright\text{iru}^{\uparrow L}\quad.
\end{align*}

To verify the associativity law, first substitute the definition of
$\text{zip}_{L}$ into one side:
\begin{align*}
{\color{greenunder}\text{left-hand side}:}\quad & \text{zip}_{L}\big(p\times\text{zip}_{L}(q\times r)\big)\triangleright\varepsilon_{1,23}^{\uparrow L}=\big(p\times\text{zip}_{L}(q\times r)\big)\triangleright\text{zip}_{F}\bef\text{zip}_{G}^{\uparrow F}\bef\gunderline{\varepsilon_{1,23}^{\uparrow L}}\\
{\color{greenunder}\text{definition of }^{\uparrow L}:}\quad & =\big(p\times\big(\text{zip}_{F}(q\times r)\triangleright\gunderline{\text{zip}_{G}^{\uparrow F}}\big)\big)\triangleright\gunderline{\text{zip}_{F}}\bef\text{zip}_{G}^{\uparrow F}\bef\varepsilon_{1,23}^{\uparrow G\uparrow F}\\
{\color{greenunder}\text{naturality law of }\text{zip}_{F}:}\quad & =\big(p\times\text{zip}_{F}(q\times r)\big)\triangleright\text{zip}_{F}\bef\big(g\times k^{:G^{B}\times G^{C}}\!\rightarrow\gunderline{g\times\text{zip}_{G}(k)\big)^{\uparrow F}\bef\text{zip}_{G}^{\uparrow F}\bef\varepsilon_{1,23}^{\uparrow G\uparrow F}}\\
{\color{greenunder}\text{composition under }^{\uparrow F}:}\quad & =\text{zip}_{F}\big(p\times\text{zip}_{F}(q\times r)\big)\triangleright\big(g\times(h\times j)\rightarrow\text{zip}_{G}(g\times\text{zip}_{G}(h\times j))\triangleright\varepsilon_{1,23}^{\uparrow G}\big)^{\uparrow F}\quad.
\end{align*}
Now rewrite the right-hand side in a similar way:
\begin{align*}
{\color{greenunder}\text{right-hand side}:}\quad & \text{zip}_{L}\big(\text{zip}_{L}(p\times q)\times r\big)\triangleright\varepsilon_{12,3}^{\uparrow L}=\big(\text{zip}_{L}(p\times q)\times r\big)\triangleright\text{zip}_{F}\bef\text{zip}_{G}^{\uparrow F}\bef\varepsilon_{12,3}^{\uparrow L}\\
{\color{greenunder}\text{definition of }^{\uparrow L}:}\quad & =\big(\big(\text{zip}_{F}(p\times q)\triangleright\gunderline{\text{zip}_{G}^{\uparrow F}}\big)\times r\big)\triangleright\gunderline{\text{zip}_{F}}\bef\text{zip}_{G}^{\uparrow F}\bef\varepsilon_{12,3}^{\uparrow G\uparrow F}\\
{\color{greenunder}\text{naturality law of }\text{zip}_{F}:}\quad & =\big(\text{zip}_{F}(p\times q)\times r\big)\triangleright\text{zip}_{F}\bef\big(k^{:G^{A}\times G^{B}}\!\times j\rightarrow\gunderline{\text{zip}_{G}(k)\times j\big)^{\uparrow F}\bef\text{zip}_{G}^{\uparrow F}\bef\varepsilon_{12,3}^{\uparrow G\uparrow F}}\\
{\color{greenunder}\text{composition under }^{\uparrow F}:}\quad & =\text{zip}_{F}\big(\text{zip}_{F}(p\times q)\times r\big)\triangleright\big((g\times h)\times j\rightarrow\text{zip}_{G}(\text{zip}_{G}(g\times h)\times j)\triangleright\varepsilon_{12,3}^{\uparrow G}\big)^{\uparrow F}\quad.
\end{align*}
The two sides become equal after using the associativity laws of $\text{zip}_{F}$
and $\text{zip}_{G}$:
\begin{align*}
 & \text{zip}_{F}\big(p\times\text{zip}_{F}(q\times r)\big)\triangleright\varepsilon_{1,23}^{\uparrow F}=\text{zip}_{F}\big(\text{zip}_{F}(p\times q)\times r\big)\triangleright\varepsilon_{12,3}^{\uparrow F}\quad,\\
 & \text{zip}_{G}(g\times\text{zip}_{G}(h\times j))\triangleright\varepsilon_{1,23}^{\uparrow G}=\text{zip}_{G}(\text{zip}_{G}(g\times h)\times j)\triangleright\varepsilon_{12,3}^{\uparrow G}\quad.
\end{align*}

To verify the commutativity law~(\ref{eq:commutativity-law-of-zip})
for $L$, we assume that the law holds for $F$ and $G$:
\begin{align*}
{\color{greenunder}\text{expect to equal }(\text{zip}_{L}\bef\text{swap}^{\uparrow L}):}\quad & \text{swap}\bef\text{zip}_{L}=\gunderline{\text{swap}\bef\text{zip}_{F}}\bef\text{zip}_{G}^{\uparrow F}\\
{\color{greenunder}\text{commutativity law of }F:}\quad & =\text{zip}_{F}\bef\gunderline{\text{swap}^{\uparrow F}\bef\text{zip}_{G}^{\uparrow F}}=\text{zip}_{F}\bef(\gunderline{\text{swap}\bef\text{zip}_{G}})^{\uparrow F}\\
{\color{greenunder}\text{commutativity law of }G\text{ under }^{\uparrow F}:}\quad & =\text{zip}_{F}\bef\text{zip}_{G}^{\uparrow F}\bef\text{swap}^{\uparrow G\uparrow F}=\text{zip}_{L}\bef\text{swap}^{\uparrow L}\quad.
\end{align*}
$\square$

\paragraph{Products}

Similarly to most other typeclasses, the product of applicative functors
is applicative:

\subsubsection{Statement \label{subsec:Statement-applicative-product}\ref{subsec:Statement-applicative-product}}

For any applicative functors $F$ and $G$, the functor $L^{A}\triangleq F^{A}\times G^{A}$
is also applicative. Its \lstinline!zip! and \lstinline!wu! methods
are defined by:
\begin{align*}
 & \text{zip}_{L}:(F^{A}\times G^{A})\times(F^{B}\times G^{B})\rightarrow F^{A\times B}\times G^{A\times B}\quad,\\
 & \text{zip}_{L}\big((m^{:F^{A}}\times n^{:G^{A}})\times(p^{:F^{B}}\times q^{:G^{B}})\big)\triangleq\text{zip}_{F}(m\times p)\times\text{zip}_{G}(n\times q)\quad,\\
 & \text{wu}_{L}:F^{\bbnum 1}\times G^{\bbnum 1}\quad,\quad\quad\text{wu}_{L}\triangleq\text{wu}_{F}\times\text{wu}_{G}\quad.
\end{align*}
If $F$ and $G$ are commutative applicative functors, then so is
$L$.

\subparagraph{Proof}

We note that the lifting to $L$ is defined by $f^{\uparrow L}\triangleq f^{\uparrow F}\boxtimes f^{\uparrow G}$.

To verify the left identity law of $\text{zip}_{L}$:
\begin{align*}
{\color{greenunder}\text{expect to equal }(p\times q)\triangleright\text{ilu}^{\uparrow L}:}\quad & \text{zip}_{L}\big(\gunderline{\text{wu}_{L}}\times(p^{:F^{A}}\times q^{:G^{A}})\big)=\gunderline{\text{zip}_{L}}((\text{wu}_{F}\times\text{wu}_{G})\times(p\times q))\\
{\color{greenunder}\text{definition of }\text{zip}_{L}:}\quad & =\text{zip}_{F}(\text{wu}_{F}\times p)\times\text{zip}_{G}(\text{wu}_{G}\times q)\\
{\color{greenunder}\text{left identity laws of }\text{zip}_{F}\text{ and }\text{zip}_{G}:}\quad & =(p\triangleright\text{ilu}^{\uparrow F})\times(q\triangleright\text{ilu}^{\uparrow G})=(p\times q)\triangleright(\text{ilu}^{\uparrow F}\boxtimes\text{ilu}^{\uparrow G})\\
{\color{greenunder}\text{definition of }^{\uparrow L}:}\quad & =(p\times q)\triangleright\text{ilu}^{\uparrow L}\quad.
\end{align*}

To verify the right identity law of $\text{zip}_{L}$:
\begin{align*}
{\color{greenunder}\text{expect to equal }(p\times q)\triangleright\text{iru}^{\uparrow L}:}\quad & \text{zip}_{L}\big((p^{:F^{A}}\times q^{:G^{A}})\times\gunderline{\text{wu}_{L}}\big)=\gunderline{\text{zip}_{L}}((p\times q)\times(\text{wu}_{F}\times\text{wu}_{G}))\\
{\color{greenunder}\text{definition of }\text{zip}_{L}:}\quad & =\text{zip}_{F}(p\times\text{wu}_{F})\times\text{zip}_{G}(q\times\text{wu}_{G})\\
{\color{greenunder}\text{right identity laws of }\text{zip}_{F}\text{ and }\text{zip}_{G}:}\quad & =(p\triangleright\text{iru}^{\uparrow F})\times(q\triangleright\text{iru}^{\uparrow G})=(p\times q)\triangleright(\text{iru}^{\uparrow F}\boxtimes\text{iru}^{\uparrow G})\\
{\color{greenunder}\text{definition of }^{\uparrow L}:}\quad & =(p\times q)\triangleright\text{iru}^{\uparrow L}\quad.
\end{align*}

To verify the associativity law, begin with its left-hand side and
use the definition of $\text{zip}_{L}$:
\begin{align*}
{\color{greenunder}\text{left-hand side}:}\quad & \text{zip}_{L}\big((p_{1}^{:F^{A}}\times p_{2}^{:G^{A}})\times\text{zip}_{L}\big((q_{1}^{:F^{B}}\times q_{2}^{:G^{B}})\times(r_{1}^{:F^{C}}\times r_{2}^{:G^{C}})\big)\big)\triangleright\gunderline{\varepsilon_{1,23}^{\uparrow L}}\\
 & =\text{zip}_{L}\big((p_{1}\times p_{2})\times\big(\text{zip}_{F}(q_{1}\times r_{1})\times\text{zip}_{G}(q_{2}\times r_{2})\big)\big)\triangleright\big(\varepsilon_{1,23}^{\uparrow F}\boxtimes\varepsilon_{1,23}^{\uparrow G}\big)\\
 & =\big(\gunderline{\text{zip}_{F}\big(p_{1}\times\text{zip}_{F}(q_{1}\times r_{1})\big)\triangleright\varepsilon_{1,23}^{\uparrow F}}\big)\times\big(\gunderline{\text{zip}_{G}\big(p_{2}\times\text{zip}_{G}(q_{2}\times r_{2})\big)\triangleright\varepsilon_{1,23}^{\uparrow G}}\big)\quad.
\end{align*}
The right-hand side is rewritten in a similar way:
\begin{align*}
{\color{greenunder}\text{right-hand side}:}\quad & \text{zip}_{L}\big(\text{zip}_{L}\big((p_{1}^{:F^{A}}\times p_{2}^{:G^{A}})\times(q_{1}^{:F^{B}}\times q_{2}^{:G^{B}})\big)\times(r_{1}^{:F^{C}}\times r_{2}^{:G^{C}})\big)\triangleright\gunderline{\varepsilon_{12,3}^{\uparrow L}}\\
 & =\text{zip}_{L}\big(\big(\text{zip}_{F}(p_{1}\times q_{1})\times\text{zip}_{G}(p_{2}\times q_{2})\big)\times(r_{1}\times r_{2})\big)\big)\triangleright\big(\varepsilon_{12,3}^{\uparrow F}\boxtimes\varepsilon_{12,3}^{\uparrow G}\big)\\
 & =\big(\gunderline{\text{zip}_{F}\big(\text{zip}_{F}(p_{1}\times q_{1})\times r_{1}\big)\triangleright\varepsilon_{12,3}^{\uparrow F}}\big)\times\big(\gunderline{\text{zip}_{G}\big(\text{zip}_{G}(p_{2}\times q_{2})\times r_{2}\big)\triangleright\varepsilon_{12,3}^{\uparrow G}}\big)\quad.
\end{align*}
The underlined expressions in both sides are equal due to associativity
laws of $\text{zip}_{F}$ and $\text{zip}_{G}$.

To verify the commutativity law of $L$ assuming it holds for $F$
and $G$:
\begin{align*}
{\color{greenunder}\text{expect to equal }\text{zip}_{L}\big((p\times q)\times(m\times n)\big):}\quad & \text{zip}_{L}\big((m\times n)\times(p\times q)\big)\triangleright\text{swap}^{\uparrow L}\\
{\color{greenunder}\text{definitions of }\text{zip}_{L}\text{ and }^{\uparrow L}:}\quad & =\big(\text{zip}_{F}(m\times p)\times\text{zip}_{G}(n\times q)\big)\triangleright(\text{swap}^{\uparrow F}\boxtimes\text{swap}^{\uparrow G})\\
{\color{greenunder}\text{definition of }\boxtimes:}\quad & =\big(\text{zip}_{F}(m\times p)\triangleright\text{swap}^{\uparrow F}\big)\times\big(\text{zip}_{G}(n\times q)\triangleright\text{swap}^{\uparrow G}\big)\\
{\color{greenunder}\text{commutativity laws of }F\text{ and }G:}\quad & =\text{zip}_{F}(p\times m)\times\text{zip}_{G}(q\times n)=\text{zip}_{L}\big((p\times q)\times(m\times n)\big)\quad.
\end{align*}
$\square$

\paragraph{Co-products}

The co-product $F^{A}+G^{A}$ of two arbitrary applicative functors
$F$ and $G$ is not always applicative (just as with monads). One
case where the \lstinline!zip! method cannot be implemented for $F^{A}+G^{A}$
was shown in Example~\ref{subsec:tc-Example-10}(b). However, the
following statements demonstrate that the type constructors $L^{A}\triangleq Z+F^{A}$
and $L^{A}\triangleq A+F^{A}$ are applicative functors.

\subsubsection{Statement \label{subsec:Statement-co-product-with-constant-functor-applicative}\ref{subsec:Statement-co-product-with-constant-functor-applicative}}

If $F^{\bullet}$ is applicative and $Z$ is a fixed monoid type then
$L^{A}\triangleq Z+F^{A}$ is applicative:
\begin{align*}
 & \text{zip}_{L}:(Z+F^{A})\times(Z+F^{B})\rightarrow Z+F^{A\times B}\quad,\quad\quad\text{zip}_{L}\triangleq\,\begin{array}{|c||cc|}
 & Z & F^{A\times B}\\
\hline Z\times Z & z_{1}\times z_{2}\rightarrow z_{1}\oplus z_{2} & \bbnum 0\\
F^{A}\times Z & \_^{:F^{A}}\times z\rightarrow z & \bbnum 0\\
Z\times F^{B} & z\times\_^{:F^{B}}\rightarrow z & \bbnum 0\\
F^{A}\times F^{B} & \bbnum 0 & \text{zip}_{F}
\end{array}\quad.
\end{align*}
The \textsf{``}wrapped unit\textsf{''} ($\text{wu}_{L}:Z+F^{\bbnum 1}$) is defined
as $\text{wu}_{L}\triangleq\bbnum 0^{:Z}+\text{wu}_{F}$. If $Z$
is a commutative monoid and $F^{\bullet}$ is commutative then $L^{\bullet}$
is also commutative.

\subparagraph{Proof}

We will verify the laws of $\text{zip}_{L}$ and $\text{wu}_{L}$,
assuming that $F$ is a lawful applicative functor.

The lifting to $L$ is defined in the standard way:
\[
(f^{:A\rightarrow B})^{\uparrow L}\triangleq\,\begin{array}{|c||cc|}
 & Z & F^{B}\\
\hline Z & \text{id} & \bbnum 0\\
F^{A} & \bbnum 0 & f^{\uparrow F}
\end{array}\quad.
\]

To verify the left identity law, we use the left identity law of $\text{zip}_{F}$:
\begin{align*}
 & \text{zip}_{L}(\text{wu}_{L}\times p^{:Z+F^{B}})=\text{zip}_{L}((\bbnum 0+\text{wu}_{F})\times p)=(\text{wu}_{F}\times p)\triangleright\,\begin{array}{|c||cc|}
 & Z & F^{\bbnum 1\times B}\\
\hline F^{\bbnum 1}\times Z & \_^{:F^{\bbnum 1}}\times z\rightarrow z & \bbnum 0\\
F^{\bbnum 1}\times F^{B} & \bbnum 0 & \text{zip}_{F}
\end{array}\\
 & =p\triangleright\,\begin{array}{|c||cc|}
 & Z & F^{\bbnum 1\times B}\\
\hline Z & \text{id} & \bbnum 0\\
F^{B} & \bbnum 0 & k^{:F^{B}}\rightarrow\gunderline{\text{zip}_{F}(\text{wu}_{F}\times k)}
\end{array}\,=p\triangleright\,\begin{array}{|c||cc|}
 & Z & F^{\bbnum 1\times B}\\
\hline Z & \text{id} & \bbnum 0\\
F^{B} & \bbnum 0 & k\rightarrow k\triangleright\text{ilu}^{\uparrow F}
\end{array}\,=p\triangleright\text{ilu}^{\uparrow L}\quad.
\end{align*}

To verify the right identity law, we write a similar calculation:
\begin{align*}
 & \text{zip}_{L}(p^{:Z+F^{A}}\times\text{wu}_{L})=\text{zip}_{L}(p\times(\bbnum 0+\text{wu}_{F}))=(p\times\text{wu}_{F})\triangleright\,\begin{array}{|c||cc|}
 & Z & F^{A\times\bbnum 1}\\
\hline Z\times F^{\bbnum 1} & z\times\_^{:F^{\bbnum 1}}\rightarrow z & \bbnum 0\\
F^{A}\times F^{\bbnum 1} & \bbnum 0 & \text{zip}_{F}
\end{array}\\
 & =p\triangleright\,\begin{array}{|c||cc|}
 & Z & F^{A\times\bbnum 1}\\
\hline Z & \text{id} & \bbnum 0\\
F^{A} & \bbnum 0 & k^{:F^{A}}\rightarrow\gunderline{\text{zip}_{F}(k\times\text{wu}_{F})}
\end{array}\,=p\triangleright\,\begin{array}{|c||cc|}
 & Z & F^{A\times\bbnum 1}\\
\hline Z & \text{id} & \bbnum 0\\
F^{A} & \bbnum 0 & k\rightarrow k\triangleright\text{iru}^{\uparrow F}
\end{array}\,=p\triangleright\text{iru}^{\uparrow L}\quad.
\end{align*}

To verify the associativity law, we use a trick to avoid long derivations.
The two sides of the associativity law are expressions of type $Z+F^{A\times B\times C}$:
\begin{align*}
{\color{greenunder}\text{left-hand side}:}\quad & \text{zip}_{L}\big(p^{:Z+F^{A}}\times\text{zip}_{L}(q^{:Z+F^{B}}\times r^{:Z+F^{C}})\big)\triangleright\varepsilon_{1,23}^{\uparrow L}\quad,\\
{\color{greenunder}\text{right-hand side}:}\quad & \text{zip}_{L}\big(\text{zip}_{L}(p^{:Z+F^{A}}\times q^{:Z+F^{B}})\times r^{:Z+F^{C}}\big)\triangleright\varepsilon_{12,3}^{\uparrow L}\quad.
\end{align*}
Since each of the arguments $p$, $q$, $r$ may be in one of the
two parts of the disjunction type $Z+F^{\bullet}$, we have 8 cases.
We note, however, that the code of $\text{zip}_{L}(p\times q)$ will
return a value of type $Z+\bbnum 0$ whenever at least one of the
arguments ($p$, $q$) is of type $Z+\bbnum 0$. So, a composition
of two \lstinline!zip! operations will also return a value of type
$Z+\bbnum 0$ whenever at least one of the arguments ($p$, $q$,
$r$) is of type $Z+\bbnum 0$. So, we need to consider the following
two cases: 

\textbf{(1)} At least one of $p$, $q$, $r$ is of type $Z+\bbnum 0$.
In this case, any arguments of type $\bbnum 0+F^{\bullet}$ are ignored
by $\text{zip}_{L}$, while the arguments of type $Z+\bbnum 0$ are
combined using the monoid $Z$\textsf{'}s binary operation ($\oplus$). So,
the result of the \lstinline!zip! operation is the same if we replace
any arguments ($p$, $q$, $r$) of type $\bbnum 0+F^{\bullet}$ by
the empty value $e_{Z}$. For example:
\[
\text{zip}_{L}\big((z+\bbnum 0)\times(\bbnum 0+k^{:F^{A}})\big)=z+\bbnum 0=\text{zip}_{L}\big((z+\bbnum 0)\times(e_{Z}+\bbnum 0)\big)\quad.
\]
After this replacement, we have three arguments ($z_{1}+\bbnum 0$,
$z_{2}+\bbnum 0$, $z_{3}+\bbnum 0$) instead of $p$, $q$, $r$,
and the function $\text{zip}_{L}$ reduces to the operation $\oplus$,
for which the associativity law holds by assumption.%
\begin{comment}
\begin{align*}
 & \text{zip}_{L}\big(p\times\text{zip}_{L}(q\times r)\big)\triangleright\varepsilon_{1,23}^{\uparrow L}=\big((z_{1}\oplus z_{2}\oplus z_{3})+\bbnum 0\big)\triangleright\varepsilon_{1,23}^{\uparrow L}=(z_{1}\oplus z_{2}\oplus z_{3})+\bbnum 0\quad,\\
 & \text{zip}_{L}\big(\text{zip}_{L}(p\times q)\times r\big)\triangleright\varepsilon_{12,3}^{\uparrow L}=\big((z_{1}\oplus z_{2}\oplus z_{3})+\bbnum 0\big)\triangleright\varepsilon_{12,3}^{\uparrow L}=(z_{1}\oplus z_{2}\oplus z_{3})+\bbnum 0\quad.
\end{align*}
\end{comment}

\textbf{(2)} All of $p$, $q$, $r$ are of type $\bbnum 0+F^{\bullet}$.
In this case, $\text{zip}_{L}$ reduces to $\text{zip}_{F}$, which
satisfies the associativity law by assumption.

To verify the commutativity law of $L$, use the code matrix for \lstinline!swap!
with the relevant types:
\begin{align*}
 & \text{swap}\bef\text{zip}_{L}=\,\begin{array}{|c||cccc|}
 & Z\times Z & F^{B}\times Z & Z\times F^{A} & F^{B}\times F^{A}\\
\hline Z\times Z & \text{swap} & \bbnum 0 & \bbnum 0 & \bbnum 0\\
F^{A}\times Z & \bbnum 0 & \bbnum 0 & \text{swap} & \bbnum 0\\
Z\times F^{B} & \bbnum 0 & \text{swap} & \bbnum 0 & \bbnum 0\\
F^{A}\times F^{B} & \bbnum 0 & \bbnum 0 & \bbnum 0 & \text{swap}
\end{array}\,\bef\,\begin{array}{|c||cc|}
 & Z & F^{B\times A}\\
\hline Z\times Z & z_{1}\times z_{2}\rightarrow z_{1}\oplus z_{2} & \bbnum 0\\
F^{B}\times Z & \_\times z\rightarrow z & \bbnum 0\\
Z\times F^{A} & z\times\_\rightarrow z & \bbnum 0\\
F^{B}\times F^{A} & \bbnum 0 & \text{zip}_{F}
\end{array}\\
 & =\,\begin{array}{|c||cc|}
 & Z & F^{B\times A}\\
\hline Z\times Z & z_{1}\times z_{2}\rightarrow z_{2}\oplus z_{1} & \bbnum 0\\
F^{A}\times Z & \_\times z\rightarrow z & \bbnum 0\\
Z\times F^{B} & z\times\_\rightarrow z & \bbnum 0\\
F^{A}\times F^{B} & \bbnum 0 & \text{swap}\bef\text{zip}_{F}
\end{array}\quad.
\end{align*}
By assumption, $\text{swap}\bef\text{zip}_{F}=\text{zip}_{F}\bef\text{swap}^{\uparrow F}$.
Next, we need the code for the lifted $\text{swap}^{\uparrow L}$:
\[
\text{swap}^{\uparrow L}=\,\begin{array}{|c||cc|}
 & Z & F^{B\times A}\\
\hline Z & \text{id} & \bbnum 0\\
F^{A\times B} &  & \text{swap}^{\uparrow F}
\end{array}\quad.
\]
We can now transform the right-hand side of the commutativity law:
\[
\text{zip}_{L}\bef\text{swap}^{\uparrow L}=\,\begin{array}{|c||cc|}
 & Z & F^{B\times A}\\
\hline Z\times Z & z_{1}\times z_{2}\rightarrow z_{1}\oplus z_{2} & \bbnum 0\\
F^{A}\times Z & \_\times z\rightarrow z & \bbnum 0\\
Z\times F^{B} & z\times\_\rightarrow z & \bbnum 0\\
F^{A}\times F^{B} & \bbnum 0 & \text{zip}_{F}\bef\text{swap}^{\uparrow F}
\end{array}\quad.
\]
The difference between the sides disappears if $Z$ is a commutative
monoid ($z_{1}\oplus z_{2}=z_{2}\oplus z_{1}$). $\square$

\subsubsection{Statement \label{subsec:Statement-co-product-with-identity-applicative}\ref{subsec:Statement-co-product-with-identity-applicative}}

If $F^{\bullet}$ is applicative then $L^{A}\triangleq A+F^{A}$ is
also applicative:
\begin{align*}
 & \text{zip}_{L}:(A+F^{A})\times(B+F^{B})\rightarrow A\times B+F^{A\times B}\quad,\quad\quad\text{zip}_{L}\triangleq\,\begin{array}{|c||cc|}
 & A\times B & F^{A\times B}\\
\hline A\times B & \text{id} & \bbnum 0\\
F^{A}\times B & \bbnum 0 & (\text{id}\boxtimes\text{pu}_{F})\bef\text{zip}_{F}\\
A\times F^{B} & \bbnum 0 & (\text{pu}_{F}\boxtimes\text{id})\bef\text{zip}_{F}\\
F^{A}\times F^{B} & \bbnum 0 & \text{zip}_{F}
\end{array}\quad.
\end{align*}
The \lstinline!wu! method is defined by $\text{wu}_{L}\triangleq1+\bbnum 0^{:F^{\bbnum 1}}$.
If $F^{\bullet}$ is commutative then $L^{\bullet}$ is also commutative.

\subparagraph{Proof}

We will use Statement~\ref{subsec:Statement-co-product-with-co-pointed-applicative},
where the same properties are demonstrated for a more general functor
$L^{A}\triangleq H^{A}+F^{A}$. We will set $H^{A}\triangleq A$ in
Statement~\ref{subsec:Statement-co-product-with-co-pointed-applicative}
and obtain the present statement because the compatibility law holds
automatically for $\text{ex}_{H}\triangleq\text{id}$ and $\text{zip}_{H}\triangleq\text{id}$.
 $\square$

It is important that Statements~\ref{subsec:Statement-co-product-with-constant-functor-applicative}\textendash \ref{subsec:Statement-co-product-with-identity-applicative}
define the methods \lstinline!zip! and \lstinline!wu! differently
for $L^{A}\triangleq Z+F^{A}$ and for $L^{A}\triangleq A+F^{A}$.
Only then the applicative laws of \lstinline!zip! and \lstinline!wu!
will hold for those functors. Also, those definitions agree with the
code of \lstinline!zip! and \lstinline!wu! for the \lstinline!Either!
functor (Section~\ref{subsec:Programs-that-accumulate-errors}).
The \lstinline!Either! functor ($L^{A}\triangleq Z+A$ with a monoid
type $Z$) is constructed as $L^{A}\triangleq Z+F^{A}$ with $F^{A}\triangleq A$
(via Statement~\ref{subsec:Statement-co-product-with-constant-functor-applicative})
or as $L^{A}\triangleq A+F^{A}$ with $F^{A}\triangleq Z$ (via Statement~\ref{subsec:Statement-co-product-with-identity-applicative}).

The following statement generalizes the construction $L^{A}\triangleq A+F^{A}$
to $L^{A}\triangleq H^{A}+F^{A}$ where $H^{\bullet}$ is applicative
and at the same time co-pointed (see Section~\ref{subsec:Co-pointed-functors}).
The code for $\text{zip}_{L}$ will use the co-pointed functor\textsf{'}s \lstinline!extract!
method. The next statement shows that $\text{zip}_{L}$ will obey
the applicative laws if a special \textbf{compatibility law}\index{compatibility law!of extract and zip@of \texttt{extract} and \texttt{zip}}
holds for \lstinline!extract! and \lstinline!zip!:
\begin{equation}
\text{ex}_{H}(\text{zip}_{H}(p^{:H^{A}}\times q^{:H^{B}}))=\text{ex}_{H}(p)\times\text{ex}_{H}(q)\quad,\quad\text{or equivalently}:\quad\text{zip}_{H}\bef\text{ex}_{H}=\text{ex}_{H}\boxtimes\text{ex}_{H}\quad.\label{eq:compatibility-law-of-extract-and-zip}
\end{equation}

A simple example of a co-pointed applicative functor is $H^{A}\triangleq A\times G^{A}$:

\subsubsection{Statement \label{subsec:Statement-co-pointed-applicative-example}\ref{subsec:Statement-co-pointed-applicative-example}}

Given any applicative functor $G^{\bullet}$, the functor $H^{A}\triangleq A\times G^{A}$
is applicative and co-pointed. The compatibility law~(\ref{eq:compatibility-law-of-extract-and-zip})
holds if the \lstinline!extract! method is defined by $\text{ex}_{H}\triangleq\pi_{1}=a^{:A}\times\_^{:G^{A}}\rightarrow a$.

\subparagraph{Proof}

The applicative and co-pointed properties of $H$ follow from the
product constructions of applicative functors (Statement~\ref{subsec:Statement-applicative-product})
and co-pointed functors (Section~\ref{subsec:Co-pointed-functors}).
To verify the compatibility law~(\ref{eq:compatibility-law-of-extract-and-zip}),
we write the definition of $\text{zip}_{H}$: 
\[
\text{zip}_{H}\big((a^{:A}\times g_{1}^{:G^{A}})\times(b^{:B}\times g_{2}^{:G^{B}})\big)=(a\times b)\times\text{zip}_{G}(g_{1}\times g_{2})\quad,
\]
and apply \lstinline!extract! to the last expression:
\begin{align*}
 & \text{ex}_{H}\big(\text{zip}_{H}\big((a^{:A}\times g_{1}^{:G^{A}})\times(b^{:B}\times g_{2}^{:G^{B}})\big)\big)=\text{ex}_{H}\big((a\times b)\times\text{zip}_{G}(g_{1}\times g_{2})\big)=a\times b\quad,\\
 & \text{ex}_{H}(a\times g_{1})\times\text{ex}_{H}(b\times g_{2})=a\times b\quad.
\end{align*}
$\square$

It is not known (Problem~\ref{subsec:Problem-co-pointed-applicative})
whether there exist co-pointed applicative functors that satisfy the
compatibility law~(\ref{eq:compatibility-law-of-extract-and-zip})
but are \emph{not} of the form $A\times G^{A}$. Here is an example
of an applicative functor not of the form $A\times G^{A}$ that violates
the compatibility law:

\subsubsection{Statement \label{subsec:Statement-co-pointed-applicative-example-failing-compatibility-law}\ref{subsec:Statement-co-pointed-applicative-example-failing-compatibility-law}}

Given any monoid type $Z$, the functor $F^{A}\triangleq Z\times\left(Z\rightarrow A\right)$
is applicative and co-pointed. However, the compatibility law~(\ref{eq:compatibility-law-of-extract-and-zip})
does not hold. 

\subparagraph{Proof}

The product construction (Statement~\ref{subsec:Statement-applicative-product})
shows that $F^{\bullet}$ is applicative: it is a product of a constant
functor ($Z$) and a \lstinline!Reader! functor ($Z\rightarrow A$).
The \lstinline!zip! method is:
\[
\text{zip}_{F}\big((z_{1}^{:Z}\times r_{1}^{:Z\rightarrow A})\times(z_{2}^{:Z}\times r_{2}^{:Z\rightarrow B})\big)\triangleq(z_{1}\oplus z_{2})\times(z^{:Z}\rightarrow r_{1}(z)\times r_{2}(z))\quad.
\]
 The functor $F^{\bullet}$ is co-pointed because it has a fully parametric
\lstinline!extract! method defined by: 
\[
\text{ex}_{F}\triangleq z^{:Z}\times r^{:Z\rightarrow A}\rightarrow r(z)\quad.
\]
Let us see whether the compatibility law~(\ref{eq:compatibility-law-of-extract-and-zip})
holds:
\begin{align*}
{\color{greenunder}\text{left-hand side}:}\quad & \text{ex}_{F}\big(\text{zip}_{F}\big((z_{1}\times r_{1})\times(z_{2}\times r_{2})\big)\big)=\text{ex}_{F}\big((z_{1}\oplus z_{2})\times(z^{:Z}\rightarrow r_{1}(z)\times r_{2}(z))\big)\\
 & \quad=r_{1}(z_{1}\oplus z_{2})\times r_{2}(z_{1}\oplus z_{2})\quad,\\
{\color{greenunder}\text{right-hand side}:}\quad & \text{ex}_{F}(z_{1}\times r_{1})\times\text{ex}_{F}(z_{2}\times r_{2})=r_{1}(z_{1})\times r_{2}(z_{2})\quad.
\end{align*}
The two sides are not equal (when $Z\neq\bbnum 1$). So, the compatibility
law does not hold. $\square$

We are now ready to prove the co-pointed co-product construction:

\subsubsection{Statement \label{subsec:Statement-co-product-with-co-pointed-applicative}\ref{subsec:Statement-co-product-with-co-pointed-applicative}}

The functor $L^{A}\triangleq H^{A}+F^{A}$ is applicative if $F^{\bullet}$
and $H^{\bullet}$ are applicative and in addition $H^{\bullet}$
is co-pointed with a method $\text{ex}_{H}:H^{A}\rightarrow A$ that
satisfies the compatibility law~(\ref{eq:compatibility-law-of-extract-and-zip}).
The applicative methods of $L^{\bullet}$ are defined by:
\begin{align*}
 & \text{zip}_{L}:(H^{A}+F^{A})\times(H^{B}+F^{B})\rightarrow H^{A\times B}+F^{A\times B}\quad,\\
 & \text{zip}_{L}\triangleq\,\begin{array}{|c||cc|}
 & H^{A\times B} & F^{A\times B}\\
\hline H^{A}\times H^{B} & \text{zip}_{H} & \bbnum 0\\
H^{A}\times F^{B} & \bbnum 0 & ((\text{ex}_{H}\bef\text{pu}_{F})\boxtimes\text{id})\bef\text{zip}_{F}\\
F^{A}\times H^{B} & \bbnum 0 & (\text{id}\boxtimes(\text{ex}_{H}\bef\text{pu}_{F}))\bef\text{zip}_{F}\\
F^{A}\times F^{B} & \bbnum 0 & \text{zip}_{F}
\end{array}\quad.
\end{align*}
The method $\text{wu}_{L}:H^{\bbnum 1}+F^{\bbnum 1}$ is defined by
$\text{wu}_{L}\triangleq\text{wu}_{H}+\bbnum 0$. If $F^{\bullet}$
and $H^{\bullet}$ are commutative applicative functors then $L^{\bullet}$
is also commutative.

\subparagraph{Proof}

The lifting to $L^{\bullet}$ is defined by:
\[
(f^{:A\rightarrow B})^{\uparrow L}\triangleq\,\begin{array}{|c||cc|}
 & H^{B} & F^{B}\\
\hline H^{A} & f^{\uparrow H} & \bbnum 0\\
F^{A} & \bbnum 0 & f^{\uparrow F}
\end{array}\quad.
\]

To verify the left identity law, we begin with its left-hand side:
\begin{align*}
 & \text{zip}_{L}(\text{wu}_{L}\times p)=\big((\text{wu}_{H}^{:H^{\bbnum 1}}+\bbnum 0^{:F^{\bbnum 1}})\times p^{:H^{B}+F^{B}}\big)\triangleright\,\begin{array}{|c||cc|}
 & H^{\bbnum 1\times B} & F^{\bbnum 1\times B}\\
\hline H^{\bbnum 1}\times H^{B} & \text{zip}_{H} & \bbnum 0\\
H^{\bbnum 1}\times F^{B} & \bbnum 0 & ((\text{ex}_{H}\bef\text{pu}_{F})\boxtimes\text{id})\bef\text{zip}_{F}\\
F^{\bbnum 1}\times H^{B} & \bbnum 0 & (\text{id}\boxtimes(\text{ex}_{H}\bef\text{pu}_{F}))\bef\text{zip}_{F}\\
F^{\bbnum 1}\times F^{B} & \bbnum 0 & \text{zip}_{F}
\end{array}\\
 & =p\triangleright\,\begin{array}{|c||cc|}
 & H^{\bbnum 1\times B} & F^{\bbnum 1\times B}\\
\hline H^{B} & h\rightarrow\text{zip}_{H}(\text{wu}_{H}\times h) & \bbnum 0\\
F^{B} & \bbnum 0 & f\rightarrow\text{zip}_{F}((\text{wu}_{H}\triangleright\text{ex}_{H}\triangleright\text{pu}_{F})\times f)
\end{array}\quad.
\end{align*}
Using Eq.~(\ref{eq:co-pointed-nondegeneracy-law-wu}) and the definition
of \lstinline!wu! through \lstinline!pure!, we find:
\[
\text{wu}_{H}\triangleright\text{ex}_{H}\triangleright\text{pu}_{F}=1\triangleright\text{pu}_{F}=\text{wu}_{F}\quad.
\]
Since the identity laws of $F$ and $H$ are assumed to hold, we can
transform the last matrix as:
\[
\begin{array}{|c||cc|}
 & H^{\bbnum 1\times B} & F^{\bbnum 1\times B}\\
\hline H^{B} & h\rightarrow\text{zip}_{H}(\text{wu}_{H}\times h) & \bbnum 0\\
F^{B} & \bbnum 0 & f\rightarrow\text{zip}_{F}((\text{wu}_{H}\triangleright\text{ex}_{H}\triangleright\text{pu}_{F})\times f)
\end{array}\,=\,\begin{array}{|c||cc|}
 & H^{\bbnum 1\times B} & F^{\bbnum 1\times B}\\
\hline H^{B} & \text{ilu}^{\uparrow H} & \bbnum 0\\
F^{B} & \bbnum 0 & \text{ilu}^{\uparrow F}
\end{array}\,=\text{ilu}^{\uparrow L}\quad.
\]
After this simplification, the left-hand side equals $p\triangleright\text{ilu}^{\uparrow L}$,
i.e., the right-hand side of the law.

The right identity law is verified in a similar way:
\begin{align*}
 & \text{zip}_{L}(p\times\text{wu}_{L})=\text{zip}_{L}\big(p^{:H^{A}+F^{A}}\times(\text{wu}_{H}^{:H^{\bbnum 1}}+\bbnum 0^{:F^{\bbnum 1}})\big)\\
 & =p\triangleright\,\begin{array}{|c||cc|}
 & H^{A\times\bbnum 1} & F^{A\times\bbnum 1}\\
\hline H^{A} & h\rightarrow\text{zip}_{H}(h\times\text{wu}_{H}) & \bbnum 0\\
F^{A} & \bbnum 0 & f\rightarrow\text{zip}_{F}(f\times(\text{wu}_{H}\triangleright\text{ex}_{H}\triangleright\text{pu}_{F}))
\end{array}
\end{align*}
\begin{align*}
 & =p\triangleright\,\,\begin{array}{|c||cc|}
 & H^{A\times\bbnum 1} & F^{A\times\bbnum 1}\\
\hline H^{A} & \text{iru}^{\uparrow H} & \bbnum 0\\
F^{A} & \bbnum 0 & \text{iru}^{\uparrow F}
\end{array}\,=p\triangleright\text{iru}^{\uparrow L}\quad.
\end{align*}

The associativity law is an equation between values of type $H^{A\times B\times C}+F^{A\times B\times C}$:
\[
\text{zip}_{L}(p^{:H^{A}+F^{A}}\times\text{zip}_{L}(q^{:H^{B}+F^{B}}\times r^{:H^{C}+F^{C}}))\triangleright\varepsilon_{1,23}^{\uparrow L}=\text{zip}_{L}(\text{zip}_{L}(p\times q)\times r)\triangleright\varepsilon_{12,3}^{\uparrow L}\quad.
\]
The operation $\text{zip}_{L}(p\times q)$ is defined in such a way
that it returns a value of type $H^{A\times B}+\bbnum 0$ only when
both $p$ and $q$ are in the left part of the disjunction:
\[
\text{zip}_{L}\big((a^{:H^{A}}+\bbnum 0^{:F^{A}})\times(b^{:H^{B}}+\bbnum 0^{:F^{B}})\big)=\text{zip}_{H}(a\times b)+\bbnum 0^{:F^{A\times B}}\quad.
\]
Otherwise, $\text{zip}_{L}(p\times q)$ returns a value of type $\bbnum 0^{:H^{A\times B}}+F^{A\times B}$.
So, we need to consider three cases:

\textbf{(1)} The arguments are $p=a^{:H^{A}}+\bbnum 0$, $q=b^{:H^{B}}+\bbnum 0$,
$r=c^{:H^{C}}+\bbnum 0$. In this case, $\text{zip}_{L}$ reduces
to $\text{zip}_{H}$:
\begin{align*}
{\color{greenunder}\text{left-hand side}:}\quad & \text{zip}_{L}(p\times\text{zip}_{L}(q\times r))\triangleright\varepsilon_{1,23}^{\uparrow L}=\text{zip}_{H}\big(a\times\text{zip}_{H}(b\times c)\big)\triangleright\varepsilon_{1,23}^{\uparrow H}+\bbnum 0\quad,\\
{\color{greenunder}\text{right-hand side}:}\quad & \text{zip}_{L}(\text{zip}_{L}(p\times q)\times r)\triangleright\varepsilon_{12,3}^{\uparrow L}=\text{zip}_{H}\big(\text{zip}_{H}(a\times b)\times c\big)\triangleright\varepsilon_{12,3}^{\uparrow H}+\bbnum 0\quad.
\end{align*}
The two sides are equal due to the associativity law of $\text{zip}_{H}$.

\textbf{(2)} The argument $q$ has type $\bbnum 0+F^{B}$. In this
case, $\text{zip}_{L}$ reduces to $\text{zip}_{F}$ after converting
arguments of type $H^{\bullet}+0$ to type $F^{\bullet}$ when needed.
We may define this conversion as a helper function \lstinline!toF!:
\begin{align*}
 & \text{toF}:H^{A}+F^{A}\rightarrow F^{A}\quad,\quad\quad\text{toF}\triangleq\,\begin{array}{|c||c|}
 & F^{A}\\
\hline H^{A} & \text{ex}_{H}\bef\text{pu}_{F}\\
F^{A} & \text{id}
\end{array}\quad.
\end{align*}
If $q$ has type $\bbnum 0+F^{B}$ then we have:
\[
\text{zip}_{L}(p\times q)=\bbnum 0+\text{zip}_{F}(\text{toF}\left(p\right)\times\text{toF}\left(q\right))\quad,\quad\quad\text{zip}_{L}(q\times r)=\bbnum 0+\text{zip}_{F}(\text{toF}\left(q\right)\times\text{toF}\left(r\right))\quad.
\]
Now the associativity law of $\text{zip}_{L}$ is reduced to the same
law of $\text{zip}_{F}$:
\begin{align*}
{\color{greenunder}\text{left-hand side}:}\quad & \text{zip}_{L}(p\times\text{zip}_{L}(q\times r))\triangleright\varepsilon_{1,23}^{\uparrow L}=\bbnum 0+\text{zip}_{F}\big(\text{toF}\left(p\right)\times\text{zip}_{F}(\text{toF}\left(q\right)\times\text{toF}\left(r\right))\big)\triangleright\varepsilon_{1,23}^{\uparrow F}\quad,\\
{\color{greenunder}\text{right-hand side}:}\quad & \text{zip}_{L}(\text{zip}_{L}(p\times q)\times r)\triangleright\varepsilon_{12,3}^{\uparrow L}=\bbnum 0+\text{zip}_{F}(\text{zip}_{F}(\text{toF}\left(p\right)\times\text{toF}\left(q\right))\times\text{toF}\left(r\right))\triangleright\varepsilon_{12,3}^{\uparrow F}\quad.
\end{align*}
The two sides are equal due to the associativity law of $\text{zip}_{F}$.

\textbf{(3)} Either $p=\bbnum 0+a^{:F^{A}}$ while $q=b^{:H^{B}}+\bbnum 0$
and $r=c^{:H^{C}}+\bbnum 0$; or $r=\bbnum 0+c^{:F^{C}}$ while $p=a^{:H^{A}}+\bbnum 0$
and $q=b^{:H^{B}}+\bbnum 0$. The two situations are symmetric, so
let us consider the first one:
\begin{align*}
{\color{greenunder}\text{left-hand side}:}\quad & \text{zip}_{L}(p\times\text{zip}_{L}(q\times r))\triangleright\varepsilon_{1,23}^{\uparrow L}=\bbnum 0+\text{zip}_{F}\big(\text{toF}\,(p)\times\text{toF}\,(\text{zip}_{H}(b\times c)+\bbnum 0)\big)\triangleright\varepsilon_{1,23}^{\uparrow F}\quad.
\end{align*}
Simplify the sub-expressions involving \lstinline!toF! separately:
\begin{align*}
 & \text{toF}\,(p)=\text{toF}\,(\bbnum 0+a)=a\quad,\\
 & \text{toF}\,(\text{zip}_{H}(b\times c)+\bbnum 0)=\text{pu}_{F}(\gunderline{\text{ex}_{H}(\text{zip}_{H}}(b\times c))\\
{\color{greenunder}\text{use Eq.~(\ref{eq:compatibility-law-of-extract-and-zip})}:}\quad & \quad=\text{pu}_{F}(\text{ex}_{H}(b)\times\text{ex}_{H}(c))\quad.
\end{align*}
So, we can rewrite the left-hand side as:
\begin{align*}
{\color{greenunder}\text{left-hand side}:}\quad & \text{zip}_{L}(p\times\text{zip}_{L}(q\times r))\triangleright\varepsilon_{1,23}^{\uparrow L}=\bbnum 0+\gunderline{\text{zip}_{F}}\big(a\times\gunderline{\text{pu}_{F}}(\text{ex}_{H}(b)\times\text{ex}_{H}(c))\big)\triangleright\varepsilon_{1,23}^{\uparrow F}\\
{\color{greenunder}\text{identity law of }\text{zip}_{F}:}\quad & =\bbnum 0+a\triangleright\big(k^{:A}\rightarrow k\times(\text{ex}_{H}(b)\times\text{ex}_{H}(c))\big)^{\uparrow F}\triangleright\varepsilon_{1,23}^{\uparrow F}\\
 & =\bbnum 0+a\triangleright\big(k^{:A}\rightarrow k\times\text{ex}_{H}(b)\times\text{ex}_{H}(c)\big)^{\uparrow F}\quad.
\end{align*}
The right-hand side can be transformed by using \lstinline!toF! on
all arguments:
\begin{align*}
{\color{greenunder}\text{right-hand side}:}\quad & \text{zip}_{L}(\text{zip}_{L}(p\times q)\times r)\triangleright\varepsilon_{12,3}^{\uparrow L}=\bbnum 0+\text{zip}_{F}(\text{zip}_{F}(\text{toF}\left(p\right)\times\text{toF}\left(q\right))\times\text{toF}\left(r\right))\triangleright\varepsilon_{12,3}^{\uparrow F}\\
 & =\bbnum 0+\text{zip}_{F}(\text{zip}_{F}(a\times\text{toF}\left(b+\bbnum 0\right))\times\text{toF}\left(r+\bbnum 0\right))\triangleright\varepsilon_{12,3}^{\uparrow F}\quad.
\end{align*}
Simplify the sub-expressions of the form $\text{zip}_{F}(a\times\text{toF}\left(b+\bbnum 0\right))$:
\begin{equation}
\text{zip}_{F}\big(a^{:F^{A}}\times\text{toF}\,(b^{:H^{B}}+\bbnum 0)\big)=\text{zip}_{F}(a\times\text{pu}_{F}(\text{ex}_{H}(b))=a\triangleright(k^{:A}\rightarrow k\times\text{ex}_{H}(b))^{\uparrow F}\quad.\label{eq:zip-copointed-construction-derivation1}
\end{equation}
Using this formula, we continue to transform the right-hand side:
\begin{align*}
{\color{greenunder}\text{right-hand side}:}\quad & \bbnum 0+\gunderline{\text{zip}_{F}}(\text{zip}_{F}(a\times\text{toF}\left(b+\bbnum 0\right))\times\gunderline{\text{toF}\left(c+\bbnum 0\right)})\triangleright\varepsilon_{12,3}^{\uparrow F}\\
{\color{greenunder}\text{use Eq.~(\ref{eq:zip-copointed-construction-derivation1})}:}\quad & =\bbnum 0+\gunderline{\text{zip}_{F}}(a\times\gunderline{\text{toF}\left(b+\bbnum 0\right)})\triangleright\big(k\rightarrow k\times\text{ex}_{H}(c)\big)^{\uparrow F}\triangleright\varepsilon_{12,3}^{\uparrow F}\\
{\color{greenunder}\text{use Eq.~(\ref{eq:zip-copointed-construction-derivation1})}:}\quad & =\bbnum 0+a\triangleright\big(k\rightarrow k\times\text{ex}_{H}(b)\big)^{\uparrow F}\triangleright\big(k\rightarrow k\times\text{ex}_{H}(c)\big)^{\uparrow F}\triangleright\varepsilon_{12,3}^{\uparrow F}\\
{\color{greenunder}\text{compute composition}:}\quad & =\bbnum 0+a\triangleright\big(k\rightarrow k\times\text{ex}_{H}(b)\times\text{ex}_{H}(c)\big)^{\uparrow F}\quad.
\end{align*}
The two sides are now equal. 

It remains to verify the commutativity law in case that law holds
for $F$ and $H$:
\[
\text{swap}\bef\text{zip}_{F}\overset{!}{=}\text{zip}_{F}\bef\text{swap}^{\uparrow F}\quad,\quad\quad\text{swap}\bef\text{zip}_{H}\overset{!}{=}\text{zip}_{H}\bef\text{swap}^{\uparrow H}\quad\quad.
\]
Begin with the left-hand side of the commutativity law for $\text{zip}_{L}$:
\[
\text{swap}\bef\text{zip}_{L}=\,\begin{array}{|c||cc|}
 & H^{B\times A} & F^{B\times A}\\
\hline H^{A}\times H^{B} & \text{swap}\bef\text{zip}_{H} & \bbnum 0\\
F^{A}\times H^{B} & \bbnum 0 & \text{swap}\bef((\text{ex}_{H}\bef\text{pu}_{F})\boxtimes\text{id})\bef\text{zip}_{F}\\
H^{A}\times F^{B} & \bbnum 0 & \text{swap}\bef(\text{id}\boxtimes(\text{ex}_{H}\bef\text{pu}_{F}))\bef\text{zip}_{F}\\
F^{A}\times F^{B} & \bbnum 0 & \text{swap}\bef\text{zip}_{F}
\end{array}\quad.
\]
Writing out the compositions of \lstinline!swap! and the pair product
functions, we get:
\begin{align*}
 & \text{swap}\bef((\text{ex}_{H}\bef\text{pu}_{F})\boxtimes\text{id})=(\text{id}\boxtimes(\text{ex}_{H}\bef\text{pu}_{F}))\bef\text{swap}\quad,\\
 & \text{swap}\bef(\text{id}\boxtimes(\text{ex}_{H}\bef\text{pu}_{F}))=(\text{pu}_{F}\boxtimes\text{id})\bef\text{swap}\quad.
\end{align*}
Using these simplifications, we rewrite the left-hand side as:
\[
\text{swap}\bef\text{zip}_{L}=\,\begin{array}{|c||cc|}
 & H^{B\times A} & F^{B\times A}\\
\hline H^{A}\times H^{B} & \text{swap}\bef\text{zip}_{H} & \bbnum 0\\
F^{A}\times H^{B} & \bbnum 0 & (\text{id}\boxtimes(\text{ex}_{H}\bef\text{pu}_{F}))\bef\text{swap}\bef\text{zip}_{F}\\
H^{A}\times F^{B} & \bbnum 0 & ((\text{ex}_{H}\bef\text{pu}_{F})\boxtimes\text{id})\bef\text{swap}\bef\text{zip}_{F}\\
F^{A}\times F^{B} & \bbnum 0 & \text{swap}\bef\text{zip}_{F}
\end{array}\quad.
\]
The right-hand side is rewritten to the same code after using the
laws of $\text{zip}_{F}$ and $\text{zip}_{H}$:
\begin{align*}
 & \text{zip}_{L}\bef\text{swap}^{\uparrow L}=\,\begin{array}{|c||cc|}
 & H^{A\times B} & F^{A\times B}\\
\hline H^{A}\times H^{B} & \text{zip}_{H} & \bbnum 0\\
F^{A}\times H^{B} & \bbnum 0 & (\text{id}\boxtimes(\text{ex}_{H}\bef\text{pu}_{F}))\bef\text{zip}_{F}\\
H^{A}\times F^{B} & \bbnum 0 & ((\text{ex}_{H}\bef\text{pu}_{F})\boxtimes\text{id})\bef\text{zip}_{F}\\
F^{A}\times F^{B} & \bbnum 0 & \text{zip}_{F}
\end{array}\,\bef\,\begin{array}{|c||cc|}
 & H^{B\times A} & F^{B\times A}\\
\hline H^{A\times B} & \text{swap}^{\uparrow H} & \bbnum 0\\
F^{A\times B} & \bbnum 0 & \text{swap}^{\uparrow F}
\end{array}\\
 & =\,\begin{array}{|c||cc|}
 & H^{B\times A} & F^{B\times A}\\
\hline A\times B & \text{swap}\bef\text{zip}_{H} & \bbnum 0\\
F^{A}\times B & \bbnum 0 & (\text{id}\boxtimes(\text{ex}_{H}\bef\text{pu}_{F}))\bef\text{swap}\bef\text{zip}_{F}\\
A\times F^{B} & \bbnum 0 & ((\text{ex}_{H}\bef\text{pu}_{F})\boxtimes\text{id})\bef\text{swap}\bef\text{zip}_{F}\\
F^{A}\times F^{B} & \bbnum 0 & \text{swap}\bef\text{zip}_{F}
\end{array}\quad.
\end{align*}
The two sides are now equal. $\square$

The constructions shown so far define applicative methods for all
polynomial functors.

\subsubsection{Statement \label{subsec:Statement-polynomial-functor-applicative}\ref{subsec:Statement-polynomial-functor-applicative}}

\textbf{(a)} Any polynomial functor $L^{A}$ whose fixed types are
monoids can be made into an applicative functor. 

\textbf{(b)} If $P^{A}$ and $Q^{A}$ are polynomial functors with
monoidal fixed types and $R^{A}$ is any applicative functor then
$L^{A}\triangleq P^{A}+Q^{A}\times R^{A}$ is applicative.

\textbf{(c)} A recursive polynomial functor $L^{\bullet}$ defined
by $L^{A}\triangleq S^{A,L^{F^{A}}}$ is applicative if $S^{\bullet,\bullet}$
and $F^{\bullet}$ are polynomial functors with monoidal fixed types.

\textbf{(d)} If all fixed types used in parts \textbf{(a)}\textendash \textbf{(c)}
are commutative monoids, $L^{\bullet}$ will be also commutative.

\subparagraph{Proof}

\textbf{(a)} Any polynomial functor $L^{A}$ is built (in at least
one way) by combining fixed types and the type parameter $A$ using
products and co-products. By rearranging the type expression $L^{A}$,
we may bring it to the following equivalent form:
\[
L^{A}\cong Z_{0}+A\times(Z_{1}+A\times(...\times(Z_{n-1}+A\times Z_{n})...))\quad.
\]
Here, the fixed types $Z_{i}$ are all monoids by assumption. Statements~\ref{subsec:Statement-applicative-product}
and~\ref{subsec:Statement-co-product-with-constant-functor-applicative}
are then sufficient to make $L$ into a lawful applicative functor. 

\textbf{(b)} We can express $P^{A}$ and $Q^{A}$ equivalently as:
\[
P^{A}=Z_{1}+A\times S_{1}^{A}\quad,\quad\quad Q^{A}=Z_{2}+A\times S_{2}^{A}\quad,
\]
with some fixed types $Z_{i}$ and some polynomial functors $S_{i}^{A}$(here
$i=1,2$). We can then express the type $L^{A}$ equivalently as:
\[
P^{A}+Q^{A}\times R^{A}=Z_{1}+A\times S_{1}^{A}+(Z_{2}+A\times S_{2}^{A})\times R^{A}\cong(Z_{1}+Z_{2}\times R^{A})+A\times(S_{1}^{A}+S_{2}^{A}\times R^{A})\quad.
\]
Since $S_{1}^{A}$ and $S_{2}^{A}$ are polynomial functors of smaller
degree than $P^{A}$ and $Q^{A}$, we may assume by induction that
the property we are proving will already hold for the functor $G^{A}\triangleq S_{1}^{A}+S_{2}^{A}\times R^{A}$.
Statements~\ref{subsec:Statement-applicative-product}, \ref{subsec:Statement-co-product-with-constant-functor-applicative},
\ref{subsec:Statement-co-product-with-identity-applicative}, and~\ref{subsec:Statement-co-product-with-co-pointed-applicative}
then show that $F^{A}\triangleq Z_{1}+Z_{2}\times R^{A}$ is applicative
and $F^{A}+A\times G^{A}\cong L^{A}$ is applicative.

\textbf{(c)} The polynomial functor $S^{\bullet,\bullet}$ can be
expressed as
\[
S^{A,R}=P_{0}^{A}+R^{A}\times(P_{1}^{A}+R^{A}\times(...\times(P_{n-1}^{A}+R^{A}\times P_{n}^{A})...))\quad,
\]
where $P_{0}^{A}$, ..., $P_{n}^{A}$ are some polynomial functors
(with respect to the type parameter $A$) with monoidal fixed types.
Let us assume for now that $R^{A}$ is an applicative functor. Then
it follows from part \textbf{(b)} that $P_{n-1}^{A}+R\times P_{n}^{A}$
is applicative. In the same way, it follows that $P_{n-2}^{A}+R\times(P_{n-1}^{A}+R\times P_{n}^{A})$
is applicative, and so on, until we show that $S^{A,R}$ is applicative.

Now we write the recursive definition $L^{A}\triangleq S^{A,L^{F^{A}}}$.
To prove that $S^{A,L^{F^{A}}}$ is an applicative functor, we may
use the inductive assumption that $L^{A}$ is applicative when used
in the recursive position (i.e., as the second argument of $S^{\bullet,\bullet}$).
By Statement~\ref{subsec:Statement-applicative-composition}, $L^{F^{A}}$
is applicative when used in that position. Denoting $R^{A}\triangleq L^{F^{A}}$,
we now deduce that $S^{A,R^{A}}\cong L^{A}$ is applicative.

\textbf{(d)} The proofs of parts \textbf{(a)}\textendash \textbf{(c)}
use only Statements~\ref{subsec:Statement-applicative-composition},
\ref{subsec:Statement-applicative-product}, \ref{subsec:Statement-co-product-with-constant-functor-applicative},
\ref{subsec:Statement-co-product-with-identity-applicative}, and~\ref{subsec:Statement-co-product-with-co-pointed-applicative}.
These statements guarantee that $L$ will be commutative if all relevant
fixed types are commutative monoids. $\square$

\paragraph{Function types}

We have seen in Section~\ref{subsec:The-applicative-Reader-functor}
that the \lstinline!Reader! functor ($L^{A}\triangleq R\rightarrow A$)
has a \lstinline!zip! operation. That \lstinline!zip! operation
can be derived from the monadic methods of the \lstinline!Reader!
monad (which is commutative). Statement~\ref{subsec:Statement-monad-construction-2}
generalized the \lstinline!Reader! monad to a wider class of monads
with type $L^{A}\triangleq H^{A}\rightarrow A$, where $H^{\bullet}$
is an arbitrary contrafunctor. The lawful monad gives up to two definitions
of a \lstinline!zip! method for the functors $L$ of this type. However,
commutativity is not guaranteed for arbitrary $H^{\bullet}$. One
can implement a \lstinline!zip! method with type signature
\[
\text{zip}:(H^{A}\rightarrow A)\times(H^{B}\rightarrow B)\rightarrow H^{A\times B}\rightarrow A\times B\quad,
\]
such that the commutativity law~(\ref{eq:associativity-law-of-zip-commutative})
always holds. That definition of \lstinline!zip! will, however, fail
the associativity law (we omit the proof for brevity). So, there are
no new constructions of lawful applicative functors of function type.

\paragraph{Recursive types}

There exist two different constructions of applicative functors based
on recursive types. These constructions generalize the tree-like \textsf{``}free
monad\textsf{''} (see Statement~\ref{subsec:Statement-monad-construction-4-free-monad})
and the standard \lstinline!zip! implementation for the \lstinline!List!
functor.

The first construction generalizes tree-like functors. A simple example
is the binary tree:
\[
T^{A}\triangleq A+T^{A}\times T^{A}\quad.
\]
This data structure describes trees with leaves of type $A$ and branches
that contain two sub-trees. We could generalize this type by adding
extra data of type $A$ for each branch:
\[
T^{A}\triangleq A+A\times T^{A}\times T^{A}\quad.
\]
The following statement generalizes this type further by using two
arbitrary applicative functors to describe the branch shapes and any
extra data on branches.

\subsubsection{Statement \label{subsec:Statement-applicative-recursive-type}\ref{subsec:Statement-applicative-recursive-type}}

For any applicative functors $F$ and $H$, the functor $L$ defined
recursively by $L^{A}\triangleq A+H^{A}\times F^{L^{A}}$ is applicative.
If $F$ and $H$ are commutative then $L$ is also commutative.

\subparagraph{Proof}

It turns out that we can verify the required laws without long derivations
by using type constructions whose validity we already proved. Let
us define $N^{A}\triangleq H^{A}\times F^{L^{A}}$ for brevity. Then
the definition of $L$ is equivalently written as $L^{A}=A+N^{A}=A+H^{A}\times F^{L^{A}}$.
This type is built up from applicative functors $H$, $F$, and (recursively)
$L$ itself. The functor $A+N^{A}$ is applicative due to Statement~\ref{subsec:Statement-co-product-with-identity-applicative},
while $N^{A}=H^{A}\times F^{L^{A}}$ is a functor product (Statement~\ref{subsec:Statement-applicative-product}).
Finally, $F^{L^{A}}$ is applicative because it is a functor composition
(Statement~\ref{subsec:Statement-applicative-composition}), and
because we may assume that $L^{A}$ is already a lawful applicative
functor when we use its methods in recursive calls.

Since all these constructions preserve commutative applicative functors,
we conclude that $L$ will be commutative if $F$ and $H$ are.

The constructions also give us the code for the applicative methods
\lstinline!zip! and \lstinline!wu! of $L$. This code is similar
to the code for a tree-like applicative shown in Example~\ref{subsec:Example-applicative-tree}.
When combining a \textsf{``}leaf\textsf{''} value ($A+\bbnum 0$) with a \textsf{``}sub-tree\textsf{''}
value ($\bbnum 0+N^{A}$), we duplicate the \textsf{``}leaf\textsf{''} value as many
times as needed to cover the \textsf{``}sub-tree\textsf{''}:
\begin{align*}
 & \text{zip}_{L}:(A+N^{A})\times(B+N^{B})\rightarrow A\times B+N^{A\times B}\quad,\\
 & \text{zip}_{L}\triangleq\,\begin{array}{|c||cc|}
 & A\times B & N^{A\times B}\\
\hline A\times B & \text{id} & \bbnum 0\\
N^{A}\times B & \bbnum 0 & n\times b\rightarrow n\triangleright(a^{:A}\rightarrow a\times b)^{\uparrow N}\\
A\times N^{B} & \bbnum 0 & a\times n\rightarrow n\triangleright(b^{:B}\rightarrow a\times b)^{\uparrow N}\\
N^{A}\times N^{B} & \bbnum 0 & \overline{\text{zip}_{N}}
\end{array}\quad.
\end{align*}
The \textsf{``}wrapped unit\textsf{''} is $\text{wu}_{L}\triangleq1+\bbnum 0^{:N^{\bbnum 1}}$.
The liftings to $L$ and $N$ are defined by:
\[
(f^{:A\rightarrow B})^{\uparrow L}=\,\begin{array}{|c||cc|}
 & B & N^{B}\\
\hline A & \text{id} & \bbnum 0\\
N^{A} & \bbnum 0 & f^{\uparrow N}
\end{array}\quad,\quad\quad f^{\uparrow N}=f^{\uparrow H}\times f^{\overline{\uparrow L}\uparrow F}=h^{:H^{A}}\times k^{:F^{L^{A}}}\rightarrow(h\triangleright f^{\uparrow H})\times(k\triangleright f^{\overline{\uparrow L}\uparrow F})\quad.
\]
For brevity, we denoted by $\overline{\text{zip}_{N}}$ the following
function:
\[
\overline{\text{zip}_{N}}\triangleq(h_{1}^{:H^{A}}\times k_{1}^{:F^{L^{A}}})\times(h_{2}^{:H^{B}}\times k_{2}^{:F^{L^{B}}})\rightarrow\text{zip}_{H}(h_{1}\times h_{2})\times\big(\text{zip}_{F}(k_{1}\times k_{2})\triangleright\overline{\text{zip}_{L}}^{\uparrow F}\big)\quad.
\]
The overline reminds us that $\overline{\text{zip}_{N}}$ uses a recursive
call to $\overline{\text{zip}_{L}}$. %
\begin{comment}
$\text{zip}\left(p\times q\right)$ is simplified when one its arguments
($p$, $q$) are in the left part of the disjunction. If $p\triangleq a^{:A}+\bbnum 0^{:N^{A}}$
and $q^{:L^{B}}$ is arbitrary, we have:
\begin{align}
 & \text{zip}_{L}(p\times q)=(p\times q)\triangleright\text{zip}_{L}=(p\times q)\triangleright\,\begin{array}{|c||cc|}
 & A\times B & N^{A\times B}\\
\hline A\times B & \text{id} & \bbnum 0\\
A\times N^{B} & \bbnum 0 & a\times n\rightarrow n\triangleright(b^{:B}\rightarrow a\times b)^{\uparrow N}
\end{array}\nonumber \\
 & =q\triangleright\,\begin{array}{|c||cc|}
 & A\times B & N^{A\times B}\\
\hline B & b^{:B}\rightarrow a\times b & \bbnum 0\\
N^{B} & \bbnum 0 & n\rightarrow n\triangleright(b^{:B}\rightarrow a\times b)^{\uparrow N}
\end{array}\,=q\triangleright(b^{:B}\rightarrow a\times b)^{\uparrow N}\quad.\label{eq:zip-p-q-case1-derivation1}
\end{align}
If $p^{:L^{A}}$ is arbitrary and $q\triangleq b^{:B}+\bbnum 0^{:N^{B}}$,
we have:
\begin{align}
 & \text{zip}_{L}(p\times q)=(p\times q)\triangleright\text{zip}_{L}=(p\times q)\triangleright\,\begin{array}{|c||cc|}
 & A\times B & N^{A\times B}\\
\hline A\times B & \text{id} & \bbnum 0\\
N^{A}\times B & \bbnum 0 & n\times b\rightarrow n\triangleright(a^{:A}\rightarrow a\times b)^{\uparrow N}
\end{array}\nonumber \\
 & =p\triangleright\,\begin{array}{|c||cc|}
 & A\times B & N^{A\times B}\\
\hline A & a^{:A}\rightarrow a\times b & \bbnum 0\\
N^{A} & \bbnum 0 & n\rightarrow n\triangleright(a^{:A}\rightarrow a\times b)^{\uparrow N}
\end{array}\,=p\triangleright(a^{:A}\rightarrow a\times b)^{\uparrow N}\quad.\label{eq:zip-p-q-case2-derivation1}
\end{align}

To verify the left identity law:
\begin{align*}
 & \text{zip}_{L}(\text{wu}_{L}\times p)=\text{zip}_{L}\\
\end{align*}
To verify the associativity law~(\ref{eq:zip-associativity-law-with-epsilons})
by making use of Eqs.~(\ref{eq:zip-p-q-case1-derivation1})\textendash (\ref{eq:zip-p-q-case2-derivation1}),
we will consider separately the three cases when one of $p$, $q$,
$r$ is in the left part of their disjunctive types, and the remaining
case where all of $p$, $q$, $r$ are in the right parts.

We begin with the case $p=a^{:A}+\bbnum 0$ and write the left-hand
side of the associativity law:
\begin{align*}
 & \gunderline{\text{zip}_{L}(p}\times\text{zip}_{L}(q\times r))\triangleright\varepsilon_{1,23}^{\uparrow L}\\
{\color{greenunder}\text{use Eq.~(\ref{eq:zip-p-q-case1-derivation1})}:}\quad & =\text{zip}_{L}(q\times r)\triangleright(b^{:B}\times c^{:C}\rightarrow a\times(b\times c))^{\uparrow L}\bef\varepsilon_{1,23}^{\uparrow L}\\
 & =(q\times r)\triangleright\text{zip}_{L}\bef(b^{:B}\times c^{:C}\rightarrow a\times b\times c)^{\uparrow L}\quad.
\end{align*}
Compare this with the right-hand side of the associativity law:
\begin{align*}
 & \text{zip}_{L}(\gunderline{\text{zip}_{L}(p\times q)}\times r)\triangleright\varepsilon_{12,3}^{\uparrow L}\\
{\color{greenunder}\text{use Eq.~(\ref{eq:zip-p-q-case1-derivation1})}:}\quad & =\big((q\triangleright(b^{:B}\rightarrow a\times b))\times r\big)\triangleright\text{zip}_{L}\bef\varepsilon_{12,3}^{\uparrow L}\\
{\color{greenunder}\text{naturality law of }\text{zip}_{L}:}\quad & =(q\times r)\triangleright\text{zip}_{L}\bef(b^{:B}\times c^{:C}\rightarrow a\times b\times c)^{\uparrow L}\quad.
\end{align*}
The two sides are now equal. We are allowed to use the naturality
law of $\text{zip}_{L}$ since it is a fully parametric function (and
assuming that naturality laws hold for $\text{zip}_{F}$ and $\text{zip}_{H}$).

In the second case, $q=b^{:B}+\bbnum 0$. The two sides of the associativity
law become equal after simplification:
\begin{align*}
{\color{greenunder}\text{left-hand side}:}\quad & \text{zip}_{L}(p\times\gunderline{\text{zip}_{L}(q\times r)})\triangleright\varepsilon_{1,23}^{\uparrow L}=(p\times(r\triangleright(c^{:C}\rightarrow b\times c)^{\uparrow L}))\triangleright\text{zip}_{L}\bef\varepsilon_{1,23}^{\uparrow L}\\
{\color{greenunder}\text{naturality law of }\text{zip}_{L}:}\quad & \quad=(p\times r)\triangleright\text{zip}_{L}\bef(a^{:A}\times c^{:C}\rightarrow a\times b\times c)^{\uparrow L}\quad,\\
{\color{greenunder}\text{right-hand side}:}\quad & \text{zip}_{L}(\gunderline{\text{zip}_{L}(p\times q)}\times r)\triangleright\varepsilon_{12,3}^{\uparrow L}=\big((p\triangleright(a^{:A}\rightarrow a\times b)^{\uparrow L})\times r\big)\triangleright\text{zip}_{L}\bef\varepsilon_{12,3}^{\uparrow L}\\
{\color{greenunder}\text{naturality law of }\text{zip}_{L}:}\quad & \quad=(p\times r)\triangleright\text{zip}_{L}\bef(a^{:A}\times c^{:C}\rightarrow a\times b\times c)^{\uparrow L}\quad.
\end{align*}

In the third case, $r=c^{:C}+\bbnum 0$. Write the two sides of the
associativity law using Eq.~(\ref{eq:zip-p-q-case2-derivation1}):
\begin{align*}
{\color{greenunder}\text{left-hand side}:}\quad & \text{zip}_{L}(p\times\gunderline{\text{zip}_{L}(q\times r)})\triangleright\varepsilon_{1,23}^{\uparrow L}=(p\times(q\triangleright(b^{:B}\rightarrow b\times c)^{\uparrow L}))\triangleright\text{zip}_{L}\bef\varepsilon_{1,23}^{\uparrow L}\\
{\color{greenunder}\text{naturality law of }\text{zip}_{L}:}\quad & \quad=(p\times q)\triangleright\text{zip}_{L}\bef(a^{:A}\times b^{:B}\rightarrow a\times b\times c)^{\uparrow L}\quad,\\
{\color{greenunder}\text{right-hand side}:}\quad & \gunderline{\text{zip}_{L}}(\text{zip}_{L}(p\times q)\times\gunderline r)\triangleright\varepsilon_{12,3}^{\uparrow L}=\text{zip}_{L}(p\times q)\triangleright(a^{:A}\times b^{:B}\rightarrow(a\times b)\times c)^{\uparrow L}\bef\varepsilon_{12,3}^{\uparrow L}\\
{\color{greenunder}\text{naturality law of }\text{zip}_{L}:}\quad & \quad=(p\times r)\triangleright\text{zip}_{L}\bef(a^{:A}\times c^{:C}\rightarrow a\times b\times c)^{\uparrow L}\quad.
\end{align*}
 

It remains to consider the case when the three arguments are of the
form 
\[
p\triangleq\bbnum 0+h_{1}^{:H^{A}}\times k_{1}^{:F^{L^{A}}}\quad,\quad\quad q\triangleq\bbnum 0+h_{2}^{:H^{B}}\times k_{2}^{:F^{L^{B}}}\quad,\quad\quad r\triangleq\bbnum 0+h_{3}^{:H^{C}}\times k_{3}^{:F^{L^{C}}}\quad.
\]
Simplify some sub-expressions used in the associativity law:
\begin{align*}
 & \text{zip}_{L}(p\times q)=\bbnum 0+\text{zip}_{H}(h_{1}\times h_{2})\times\big(\text{zip}_{F}(k_{1}\times k_{2})\triangleright\overline{\text{zip}}_{L}^{\uparrow F}\big)\quad,\\
 & \text{zip}_{L}(q\times r)=\bbnum 0+\text{zip}_{H}(h_{2}\times h_{3})\times\big(\text{zip}_{F}(k_{2}\times k_{3})\triangleright\overline{\text{zip}}_{L}^{\uparrow F}\big)\quad.
\end{align*}
Write the two sides of the associativity law separately, omitting
$\varepsilon_{1,23}$ and $\varepsilon_{12,3}$ for now:
\begin{align*}
 & \text{zip}_{L}(p\times\gunderline{\text{zip}_{L}(q\times r)})=\bbnum 0+\text{zip}_{L}\big((h_{1}\times k_{1})\times\big(\text{zip}_{H}(h_{2}\times h_{3})\times(\text{zip}_{F}(k_{2}\times k_{3})\triangleright\overline{\text{zip}}_{L}^{\uparrow F})\big)\big)\\
 & \quad=\bbnum 0+\text{zip}_{H}(h_{1}\times\text{zip}_{H}(h_{2}\times h_{3}))\times\gunderline{\text{zip}_{F}}(k_{1}\times(\text{zip}_{F}(k_{2}\times k_{3})\gunderline{\triangleright\overline{\text{zip}}_{L}^{\uparrow F}}))\triangleright\overline{\text{zip}}_{L}^{\uparrow F}\quad,\\
 & \text{zip}_{L}(\text{zip}_{L}(p\times q)\times r)=\bbnum 0+\text{zip}_{L}\big(\big(\text{zip}_{H}(h_{1}\times h_{2})\times\big(\text{zip}_{F}(k_{1}\times k_{2})\triangleright\overline{\text{zip}}_{L}^{\uparrow F}\big)\big)\times(h_{3}\times k_{3})\big)\\
 & \quad=\bbnum 0+\text{zip}_{H}(\text{zip}_{H}(h_{1}\times h_{2})\times h_{3})\times\text{zip}_{F}\big(\big(\text{zip}_{F}(k_{1}\times k_{2})\triangleright\overline{\text{zip}}_{L}^{\uparrow F}\big)\times k_{3}\big)\triangleright\overline{\text{zip}}_{L}^{\uparrow F}\quad.
\end{align*}
We assume that the associativity laws of $\text{zip}_{H}$, $\text{zip}_{F}$,
and $\overline{\text{zip}_{L}}$ already hold. The sub-expressions
involving $\text{zip}_{H}$ become equal after applying $\varepsilon_{1,23}^{\uparrow H}$
and $\varepsilon_{12,3}^{\uparrow H}$. The remaining difference between
the two parts is:
\[
\text{zip}_{F}(k_{1}\times(\text{zip}_{F}(k_{2}\times k_{3})\triangleright\overline{\text{zip}}_{L}^{\uparrow F}))\triangleright\overline{\text{zip}}_{L}^{\uparrow F}\bef\varepsilon_{1,23}^{\uparrow L\uparrow F}\overset{?}{=}\text{zip}_{F}\big(\big(\text{zip}_{F}(k_{1}\times k_{2})\triangleright\overline{\text{zip}}_{L}^{\uparrow F}\big)\times k_{3}\big)\triangleright\overline{\text{zip}}_{L}^{\uparrow F}\bef\varepsilon_{12,3}^{\uparrow L\uparrow F}\quad.
\]
To show that this equation holds, we use the naturality law of $\text{zip}_{F}$
and write for one side:
\begin{align*}
 & \gunderline{\text{zip}_{F}}(k_{1}\times(\text{zip}_{F}(k_{2}\times k_{3})\gunderline{\triangleright\overline{\text{zip}}_{L}^{\uparrow F}}))\triangleright\overline{\text{zip}}_{L}^{\uparrow F}\bef\varepsilon_{1,23}^{\uparrow L\uparrow F}=\text{zip}_{F}(k_{1}\times\text{zip}_{F}(k_{2}\times k_{3}))\triangleright(\text{id}\boxtimes\overline{\text{zip}}_{L})^{\uparrow F}\bef\overline{\text{zip}}_{L}^{\uparrow F}\bef\varepsilon_{1,23}^{\uparrow L\uparrow F}\\
 & =\text{zip}_{F}(k_{1}\times\text{zip}_{F}(k_{2}\times k_{3}))\triangleright\varepsilon_{1,23}^{\uparrow F}\bef\big(l_{1}\times l_{2}\times l_{3}\rightarrow\overline{\text{zip}}_{L}(l_{1}\times\overline{\text{zip}}_{L}(l_{2}\times l_{3}))\triangleright\varepsilon_{1,23}^{\uparrow L}\big)^{\uparrow F}\quad.
\end{align*}
For the other side:
\begin{align*}
 & \gunderline{\text{zip}_{F}}\big(\big(\text{zip}_{F}(k_{1}\times k_{2})\gunderline{\triangleright\overline{\text{zip}}_{L}^{\uparrow F}}\big)\times k_{3}\big)\triangleright\overline{\text{zip}}_{L}^{\uparrow F}\bef\varepsilon_{12,3}^{\uparrow L\uparrow F}=\text{zip}_{F}\big(\text{zip}_{F}(k_{1}\times k_{2})\times k_{3}\big)\triangleright\big((\overline{\text{zip}}_{L}\boxtimes\text{id})\bef\overline{\text{zip}}_{L}\bef\varepsilon_{12,3}^{\uparrow L}\big)^{\uparrow F}\\
 & =\text{zip}_{F}\big(\text{zip}_{F}(k_{1}\times k_{2})\times k_{3}\big)\triangleright\varepsilon_{12,3}^{\uparrow F}\bef\big(l_{1}\times l_{2}\times l_{3}\rightarrow\overline{\text{zip}}_{L}(\overline{\text{zip}}_{L}(l_{1}\times l_{2})\times l_{3})\triangleright\varepsilon_{12,3}^{\uparrow L}\big)^{\uparrow F}\quad.
\end{align*}
The two sides are now equal because of the associativity laws of $\text{zip}_{F}$
and $\text{zip}_{H}$.

Finally, we check that the commutativity law holds under appropriate
assumptions:
\begin{align*}
\\
\end{align*}
\end{comment}
$\square$

Statement~\ref{subsec:Statement-applicative-recursive-type} covers
tree-like functors $L^{A}\triangleq A+H^{A}\times F^{L^{A}}$ with
\textsf{``}leaves\textsf{''} of type $A$ and \textsf{``}branch shapes\textsf{''} described by a functor
$F$. In addition, each branch may carry a value of type $H^{A}$.

When $H^{A}$ is a constant functor, the resulting $L^{\bullet}$
is a monad (Statement~\ref{subsec:Statement-monad-construction-4-free-monad}).
It is important that the applicative implementations of \lstinline!map2!
and \lstinline!zip! for this tree-like monad are \emph{not} compatible
with its monad methods. To see this, it is sufficient to note that
even the simplest tree-like monad (the binary tree, $L^{A}\triangleq A+L^{A}\times L^{A}$)
is not commutative. The applicative functor $L$, however, is commutative
because it is built from $H^{A}\triangleq1$ and $F^{A}\triangleq A\times A$,
which are both commutative.

The next statement generalizes the \lstinline!List! functor, 
\[
\text{List}^{A}\triangleq\bbnum 1+A\times\text{List}^{A}\quad,
\]
by introducing \emph{three} arbitrary applicative functors into the
above formula for $\text{List}^{A}$.

\subsubsection{Statement \label{subsec:Statement-applicative-recursive-type-1}\ref{subsec:Statement-applicative-recursive-type-1}}

\textbf{(a)} For any applicative functors $F$, $G$, and $H$, the
functor $L$ defined recursively by $L^{A}\triangleq G^{A}+H^{A}\times F^{L^{A}}$
is applicative as long as $H$ is co-pointed and the compatibility
law~(\ref{eq:compatibility-law-of-extract-and-zip}) holds. If $F$,
$G$, and $H$ are commutative then $L$ is also commutative.

\textbf{(b)} The same properties hold for the functor $P$ defined
by $P^{A}\triangleq F^{G^{A}+H^{A}\times P^{A}}$ instead.

\subparagraph{Proof}

\textbf{(a)} We will avoid long derivations if we show that the functor
$L$ is built via known type constructions. At the top level, $L^{A}$
is the co-pointed co-product construction (Statement~\ref{subsec:Statement-co-product-with-co-pointed-applicative})
with functors $G^{\bullet}$ and $H^{\bullet}\times F^{L^{\bullet}}$.
The functor $H^{\bullet}\times F^{L^{\bullet}}$ is co-pointed because
$H$ is (Section~\ref{subsec:Co-pointed-functors}). The compatibility
law holds for $H^{\bullet}\times F^{L^{\bullet}}$ due to Exercise~\ref{subsec:Exercise-applicative-II-4-1}.
The functor $F^{L^{\bullet}}$ is applicative because it is a composition
of $F$ and the recursively used $L$. As usual, we may assume that
recursive uses of $L$\textsf{'}s methods will satisfy all required laws.

The constructions give us the code of $L$\textsf{'}s methods \lstinline!zip!
and \lstinline!wu!. Define $N^{A}\triangleq H^{A}\times F^{L^{A}}$
and write: 
\begin{align*}
 & \text{ex}_{N}\triangleq\pi_{1}\bef\text{ex}_{H}=\big(h^{:H^{A}}\times k^{:F^{L^{A}}}\rightarrow\text{ex}_{H}(h)\big)\quad,\\
 & \overline{\text{zip}_{N}}\triangleq(h_{1}^{:H^{A}}\times k_{1}^{:F^{L^{A}}})\times(h_{2}^{:H^{B}}\times k_{2}^{:F^{L^{B}}})\rightarrow\text{zip}_{H}(h_{1}\times h_{2})\times\big(\text{zip}_{F}(k_{1}\times k_{2})\triangleright\overline{\text{zip}_{L}}^{\uparrow F}\big)\quad.
\end{align*}
The function \lstinline!toG! converts values of type $N^{A}$ to
values of type $G^{A}$:
\[
\text{toG}:N^{A}\rightarrow G^{A}\quad,\quad\quad\text{toG}\triangleq\text{ex}_{N}\bef\text{pu}_{G}\quad.
\]
Then we have the following code:
\begin{align*}
 & \text{zip}_{L}:(G^{A}+N^{A})\times(G^{B}+N^{B})\rightarrow G^{A\times B}+N^{A\times B}\quad,\\
 & \text{zip}_{L}\triangleq\,\begin{array}{|c||cc|}
 & G^{A\times B} & N^{A\times B}\\
\hline G^{A}\times G^{B} & \text{zip}_{G} & \bbnum 0\\
N^{A}\times G^{B} & (\text{toG}\boxtimes\text{id})\bef\text{zip}_{G} & \bbnum 0\\
G^{A}\times N^{B} & (\text{id}\boxtimes\text{toG})\bef\text{zip}_{G} & \bbnum 0\\
N^{A}\times N^{B} & \bbnum 0 & \overline{\text{zip}_{N}}
\end{array}\quad,\\
 & \text{wu}_{L}:G^{\bbnum 1}+H^{\bbnum 1}\times F^{L^{\bbnum 1}}\quad,\quad\quad\text{wu}_{L}\triangleq\bbnum 0^{:G^{\bbnum 1}}+\text{wu}_{H}\times\text{pu}_{F}(\text{wu}_{L})\quad.
\end{align*}

\textbf{(b)} Let us use the \index{recursive types!unrolling trick}\index{unrolling trick for recursive types}unrolling
trick to compare the recursive types $L$ and $P$:
\[
L^{A}=G^{A}+H^{A}\times F^{G^{A}+H^{A}\times F^{G^{A}+...}}\quad,\quad\quad P^{A}=F^{G^{A}+H^{A}\times F^{G^{A}+H^{A}\times F^{G^{A}+...}}}\quad.
\]
It suggests that $P^{A}\cong F^{L^{A}}$. Then the required properties
hold for $P$ due to the functor composition construction (Statement~\ref{subsec:Statement-applicative-composition}).
To establish the type equivalence $P^{A}\cong F^{L^{A}}$ rigorously,
we use Statement~\ref{subsec:Statement-unrolling-trick} below, where
we need to set $R^{\bullet}\triangleq F^{\bullet}$, $S^{T}\triangleq G^{A}+H^{A}\times T$,
$U\triangleq P^{A}$, and $V\triangleq L^{A}$. $\square$

Note that the construction shown in Statement~\ref{subsec:Statement-applicative-recursive-type-1}
needs to define \emph{both} \lstinline!zip! and \lstinline!wu! recursively.
This leads to a problem in case \lstinline!wu! does not have the
type of a function. A simple example when this problem occurs is with
the \lstinline!List! functor. The value $\text{wu}_{\text{List}}$
is then defined as:
\[
\text{List}^{A}\triangleq\bbnum 1+A\times\text{List}^{A}\quad,\quad\quad\text{wu}_{\text{List}}\triangleq\bbnum 0+1\times\text{wu}_{\text{List}}\quad.
\]
This value apparently represents an \emph{infinite} list of unit values:
\[
\text{wu}_{L}=\bbnum 0+1\times\left(\bbnum 0+\bbnum 1\times\left(\bbnum 0+1\times\left(...\right)\right)\right)\quad\quad???
\]
We need a \textsf{``}lazy list\index{lazy collection}\textsf{''} (or another lazy
collection) if we want to define \lstinline!wu! recursively by the
code shown above. However, the data structure $\text{List}^{A}$ is
eagerly evaluated and can only hold a finite number of values. So,
we must conclude that \lstinline!wu! is undefined for the functor
\lstinline!List!, which then cannot be considered a fully lawful
applicative functor. Nevertheless, the associativity and the commutativity
laws hold for the \lstinline!zip! method of \lstinline!List!. The
absence of a well-defined value \lstinline!wu! does not lead to practical
disadvantages when working with finite lists. 

For certain choices of $F$, $G$, and $H$, the functor $L$ will
be a monad. It is important that the implementation of applicative
methods \lstinline!pure!, \lstinline!map2!, and \lstinline!zip!
for this functor will be, in general, incompatible with its monad
methods. The \lstinline!List! functor again gives an example where
the monad is not commutative while the applicative functor is. So,
the commutative \lstinline!zip! method of \lstinline!List! cannot
be defined via its \lstinline!map! and \lstinline!flatMap! methods.
Also, the \lstinline!List! monad\textsf{'}s \lstinline!pure! method (returning
a list with a single value) differs from the applicative \lstinline!pure!
method (returning an infinite list of values, as we just saw).

There could exist other recursive constructions that produce lawful
applicative functors. For instance, one could assume an \textsf{``}applicative
bifunctor\textsf{''} $P^{A,R}$ having a \lstinline!zip! method with type
signature $P^{A,R}\times P^{B,S}\rightarrow P^{A\times B,R\times S}$
and defining a functor $L$ via the recursive type equation $L^{A}\triangleq P^{A,L^{A}}$.
The functor $L^{\bullet}$ will be applicative when suitable laws
are imposed on $P$. To find what applicative bifunctors $P$ exist,
one could continue with structural analysis (considering products
$P_{1}^{\bullet,\bullet}\times P_{2}^{\bullet,\bullet}$, co-products
$P_{1}^{\bullet,\bullet}+P_{2}^{\bullet,\bullet}$, and so on). This
book will not pursue this further, because it is not clear what practical
use one will get out of those constructions, and because there seems
to be no end to this sort of exploration. (A recursive construction
for the applicative bifunctor $P$ will involve an \textsf{``}applicative
trifunctor\textsf{''}, etc.)

Note that any functor defined via a recursive type equation $L^{A}\triangleq S^{A,L^{\bullet}}$
will be applicative if $S^{A,R^{\bullet}}$ produces an applicative
functor whenever $R^{\bullet}$ is applicative. Statements~\ref{subsec:Statement-applicative-recursive-type}\textendash \ref{subsec:Statement-applicative-recursive-type-1}
are obtained with $S^{A,R^{\bullet}}\triangleq A+H^{A}\times F^{R^{A}}$
and $S^{A,R^{\bullet}}\triangleq G^{A}+H^{A}\times F^{R^{A}}$. Other
examples of such $S^{\bullet,\bullet}$ can be found by combining
some type constructions that are known to produce applicative functors.

We conclude this section with a proof of one version of the \textsf{``}unrolling
trick\textsf{''} for recursive types:\index{unrolling trick for recursive types!proof}\index{recursive types!unrolling trick!proof}

\subsubsection{Statement \label{subsec:Statement-unrolling-trick}\ref{subsec:Statement-unrolling-trick}}

Given two functors $R^{\bullet}$ and $S^{\bullet}$, define two recursive
types $U$ and $V$ by $U\triangleq R^{S^{U}}$ and $V\triangleq S^{R^{V}}$.
The \textsf{``}unrolling trick\textsf{''} writes (non-rigorously) $U=R^{S^{R^{S^{\iddots}}}}\!$
and $V=S^{R^{S^{R^{\iddots}}}}\!$, which suggests that $U$ and $R^{V}$
are the same type. In fact, the type $U$ is rigorously equivalent
to the type $R^{V}$.

\subparagraph{Proof}

We will show that $U\cong R^{V}$ by implementing the isomorphisms
in two directions. 

By definition of the type $U$, we must have some isomorphisms (called
$\text{fix}_{U}$ and $\text{unfix}_{U}$) between types $U$ and
$R^{S^{U}}$, and similarly for the type $V$. So, we assume that
the following functions are known and satisfy the properties of isomorphisms:
\begin{align*}
 & \text{fix}_{U}:R^{S^{U}}\rightarrow U\quad,\quad\text{unfix}_{U}:U\rightarrow R^{S^{U}}\quad,\quad\text{fix}_{U}\bef\text{unfix}_{U}=\text{id}\quad,\quad\text{unfix}_{U}\bef\text{fix}_{U}=\text{id}\quad;\\
 & \text{fix}_{V}:S^{R^{V}}\rightarrow V\quad,\quad\text{unfix}_{V}:V\rightarrow S^{R^{V}}\quad,\quad\text{fix}_{V}\bef\text{unfix}_{V}=\text{id}\quad,\quad\text{unfix}_{V}\bef\text{fix}_{V}=\text{id}\quad.
\end{align*}

We implement the isomorphism functions (called \lstinline!toU! and
\lstinline!toRV!) recursively:
\begin{align*}
 & \text{toU}:R^{V}\rightarrow U\quad,\quad\quad\text{toU}\triangleq\text{unfix}_{V}^{\uparrow R}\bef\overline{\text{toU}}^{\uparrow S\uparrow R}\bef\text{fix}_{U}\quad;\\
 & \text{toRV}:U\rightarrow R^{V}\quad,\quad\quad\text{toRV}\triangleq\text{unfix}_{U}\bef\overline{\text{toRV}}^{\uparrow S\uparrow R}\bef\text{fix}_{V}^{\uparrow R}\quad.
\end{align*}

To verify the isomorphism properties ($\text{toU}\bef\text{toRV}=\text{id}$
and $\text{toRV}\bef\text{toU}=\text{id}$), we use the inductive
assumption that those properties already hold for any recursive calls
of these functions:
\begin{align*}
 & \text{toU}\bef\text{toRV}=\text{unfix}_{V}^{\uparrow R}\bef\overline{\text{toU}}^{\uparrow S\uparrow R}\bef\gunderline{\text{fix}_{U}\bef\text{unfix}_{U}}\bef\overline{\text{toRV}}^{\uparrow S\uparrow R}\bef\text{fix}_{V}^{\uparrow R}\\
 & \quad=\text{unfix}_{V}^{\uparrow R}\bef\gunderline{\overline{\text{toU}}^{\uparrow S\uparrow R}\bef\overline{\text{toRV}}^{\uparrow S\uparrow R}}\bef\text{fix}_{V}^{\uparrow R}=\text{unfix}_{V}^{\uparrow R}\bef\text{fix}_{V}^{\uparrow R}=\text{id}\quad,\\
 & \text{toRV}\bef\text{toU}=\text{unfix}_{U}\bef\overline{\text{toRV}}^{\uparrow S\uparrow R}\bef\gunderline{\text{fix}_{V}^{\uparrow R}\bef\text{unfix}_{V}^{\uparrow R}}\bef\overline{\text{toU}}^{\uparrow S\uparrow R}\bef\text{fix}_{U}\\
 & \quad=\text{unfix}_{U}\bef\gunderline{\overline{\text{toRV}}^{\uparrow S\uparrow R}\bef\overline{\text{toU}}^{\uparrow S\uparrow R}}\bef\text{fix}_{U}=\text{unfix}_{U}\bef\text{fix}_{U}=\text{id}\quad.
\end{align*}


\section{Applicative contrafunctors and profunctors\label{sec:Applicative-contrafunctors-and-profunctors}}

We have seen in Example~\ref{subsec:Example-applicative-profunctor}
that a \lstinline!zip! method can be implemented for some type constructors
that are not covariant. In this section, we will apply structural
analysis systematically to discover the non-covariant type constructors
(contrafunctors and profunctors) that admit a lawful \lstinline!zip!
method.

\subsection{Applicative contrafunctors: Laws and constructions}

Contrafunctors (see Section~\ref{subsec:Contrafunctors}) support
a \lstinline!contramap! method instead of \lstinline!map!. So, contrafunctors
cannot have the \lstinline!map2! or \lstinline!ap! methods. Nevertheless,
the applicative laws can be formulated via \lstinline!zip! and \lstinline!wu!
methods using \lstinline!contramap!. That is the formulation we will
use for contrafunctors.

\subsubsection{Definition \label{subsec:Definition-applicative-contrafunctor}\ref{subsec:Definition-applicative-contrafunctor}}

A contrafunctor $C^{\bullet}$ is \textbf{applicative} if there exist
methods \lstinline!zip! and \lstinline!wu! such that:
\begin{align}
 & \text{zip}_{C}:C^{A}\times C^{B}\rightarrow C^{A\times B}\quad,\quad\quad\text{wu}_{C}:C^{\bbnum 1}\quad,\nonumber \\
{\color{greenunder}\text{associativity law}:}\quad & \text{zip}_{C}(p\times\text{zip}_{C}(q\times r))\triangleright\tilde{\varepsilon}_{1,23}^{\downarrow C}=\text{zip}_{C}(\text{zip}_{C}(p\times q)\times r)\triangleright\tilde{\varepsilon}_{12,3}^{\downarrow C}\quad,\label{eq:applicative-contrafunctor-associativity-law}\\
{\color{greenunder}\text{left and right identity laws}:}\quad & \text{zip}_{C}(\text{wu}_{C}\times p)\triangleright\text{ilu}^{\downarrow C}=p\quad,\quad\quad\text{zip}_{C}(p\times\text{wu}_{C})\triangleright\text{iru}^{\downarrow C}=p\quad.\label{eq:applicative-contrafunctor-identity-laws}
\end{align}
Here the tuple-rearranging isomorphisms $\tilde{\varepsilon}_{1,23}$,
$\tilde{\varepsilon}_{12,3}$, \lstinline!ilu!, and \lstinline!iru!
are defined by:
\begin{align*}
 & \tilde{\varepsilon}_{1,23}\triangleq a\times b\times c\rightarrow a\times\left(b\times c\right)\quad,\quad\quad\tilde{\varepsilon}_{12,3}\triangleq a\times b\times c\rightarrow\left(a\times b\right)\times c\quad,\\
 & \text{ilu}\triangleq a\rightarrow1\times a\quad,\quad\quad\text{iru}=a\rightarrow a\times1\quad.
\end{align*}

It is important to require the laws to hold. Otherwise, a function
with the type signature of \lstinline!zip! could be implemented for
any contrafunctor $C$:
\[
\text{badZip}:C^{A}\times C^{B}\rightarrow C^{A\times B}\quad,\quad\quad\text{badZip}\triangleq p^{:C^{A}}\times\_^{:C^{B}}\rightarrow p\triangleright(a\times b\rightarrow a)^{\downarrow C}\quad.
\]
This function loses information because it ignores one of its arguments.
Intuitively, we expect that some laws will fail for this function.
Indeed, the left identity law cannot hold: \lstinline!badZip(wu, p)!
ignores \lstinline!p!, so \lstinline!badZip(wu, p).contramap(ilu)!
cannot be equal to \lstinline!p!. 

Once the \lstinline!zip! and \lstinline!wu! methods are known, we
can define \lstinline!cpure! and \lstinline!cmap2!:
\begin{align*}
 & \text{cpu}_{C}:\forall A.\,C^{A}\quad,\quad\quad\text{cpu}_{C}\triangleq\text{wu}_{C}\triangleright(\_^{:A}\rightarrow1)^{\downarrow C}\quad,\\
 & \text{cmap}_{2}:(D\rightarrow A\times B)\rightarrow C^{A}\times C^{B}\rightarrow C^{D}\quad,\quad\quad\text{cmap}_{2}(f^{:D\rightarrow A\times B})\triangleq\text{zip}_{C}\bef f^{\downarrow C}\quad.
\end{align*}

The commutativity law is formulated for applicative contrafunctors
like this:
\[
\text{zip}_{C}(q\times p)=\text{zip}_{C}(p\times q)\triangleright\text{swap}^{\downarrow C}\quad,\quad\quad\text{or equivalently}:\quad\text{swap}\bef\text{zip}_{C}=\text{zip}_{C}\bef\text{swap}^{\downarrow C}\quad.
\]

The rest of this section proves some constructions that produce lawful
applicative contrafunctors.

\paragraph{Type parameters}

A constant contrafunctor $C^{A}\triangleq Z$ (where $Z$ is a fixed
type) is at the same time a constant functor. We already showed that
a constant functor is applicative when $Z$ is a monoid. 

Another construction for applicative contrafunctors is the composition
with functors:

\subsubsection{Statement \label{subsec:Statement-applicative-contrafunctor-composition}\ref{subsec:Statement-applicative-contrafunctor-composition}}

If $F^{\bullet}$ is an applicative \emph{functor} and $G^{\bullet}$
is an applicative contrafunctor then the contrafunctor $C^{A}\triangleq F^{G^{A}}$
(equivalently written as $C\triangleq F\circ G$) is applicative.

\subparagraph{Proof}

We follow the proof of Statement~\ref{subsec:Statement-applicative-composition}.
The methods \lstinline!zip! and \lstinline!wu! are defined by:
\begin{align*}
 & \text{zip}_{C}:F^{G^{A}}\times F^{G^{B}}\rightarrow F^{G^{A\times B}}\quad,\quad\quad\text{zip}_{C}\triangleq\text{zip}_{F}\bef\text{zip}_{G}^{\uparrow F}\quad,\\
 & \text{wu}_{C}\triangleq\text{pu}_{F}(\text{wu}_{G})\quad.
\end{align*}
Note that the lifting to $C$ is defined by $f^{\downarrow C}\triangleq f^{\downarrow G\uparrow F}=(f^{\downarrow G})^{\uparrow F}$. 

To verify the left identity law in Eq.~(\ref{eq:applicative-contrafunctor-identity-laws}):
\begin{align*}
{\color{greenunder}\text{expect to equal }p:}\quad & \text{zip}_{C}(\text{wu}_{C}\times p^{:F^{G^{A}}})\triangleright\text{ilu}^{\downarrow C}=(\gunderline{\text{pu}_{F}(\text{wu}_{G})}\times p)\triangleright\gunderline{\text{zip}_{F}}\bef\text{zip}_{G}^{\uparrow F}\bef\text{ilu}^{\downarrow G\uparrow F}\\
{\color{greenunder}\text{left identity law of }\text{zip}_{F}:}\quad & =p\triangleright(g^{:G^{A}}\rightarrow\gunderline{\text{wu}_{G}\times g)^{\uparrow F}\bef\text{zip}_{G}^{\uparrow F}\bef\text{ilu}^{\downarrow G\uparrow F}}\\
{\color{greenunder}\text{composition under }^{\uparrow F}:}\quad & =p\triangleright\big(g^{:G^{A}}\rightarrow\gunderline{\text{zip}_{G}(\text{wu}_{G}\times g)\triangleright\text{ilu}^{\downarrow G}}\big)^{\uparrow F}\\
{\color{greenunder}\text{left identity law of }\text{zip}_{G}:}\quad & =p\triangleright(\gunderline{g\rightarrow g})^{\uparrow F}=p\triangleright\text{id}^{\uparrow F}=p\quad.
\end{align*}

To verify the right identity law:
\begin{align*}
{\color{greenunder}\text{expect to equal }p:}\quad & \text{zip}_{C}(p^{:F^{G^{A}}}\times\text{wu}_{C})\triangleright\text{iru}^{\downarrow C}=(p\times\gunderline{\text{pu}_{F}(\text{wu}_{G})})\triangleright\gunderline{\text{zip}_{F}}\bef\text{zip}_{G}^{\uparrow F}\bef\text{iru}^{\downarrow G\uparrow F}\\
{\color{greenunder}\text{right identity law of }\text{zip}_{F}:}\quad & =p\triangleright(g^{:G^{A}}\rightarrow\gunderline{g\times\text{wu}_{G})^{\uparrow F}\bef\text{zip}_{G}^{\uparrow F}\bef\text{iru}^{\downarrow G\uparrow F}}\\
{\color{greenunder}\text{composition under }^{\uparrow F}:}\quad & =p\triangleright\big(g^{:G^{A}}\rightarrow\gunderline{\text{zip}_{G}(g\times\text{wu}_{G})\triangleright\text{iru}^{\downarrow G}}\big)^{\uparrow F}\\
{\color{greenunder}\text{right identity law of }\text{zip}_{G}:}\quad & =p\triangleright(\gunderline{g\rightarrow g})^{\uparrow F}=p\triangleright\text{id}^{\uparrow F}=p\quad.
\end{align*}

To verify the associativity law, first substitute the definition of
$\text{zip}_{C}$ into one side:
\begin{align*}
{\color{greenunder}\text{left-hand side}:}\quad & \text{zip}_{C}\big(p\times\text{zip}_{C}(q\times r)\big)\triangleright\tilde{\varepsilon}_{1,23}^{\downarrow C}=\big(p\times\gunderline{\text{zip}_{C}(q\times r)}\big)\triangleright\text{zip}_{F}\bef\text{zip}_{G}^{\uparrow F}\bef\tilde{\varepsilon}_{1,23}^{\downarrow C}\\
 & =\big(p\times\big(\text{zip}_{F}(q\times r)\triangleright\gunderline{\text{zip}_{G}^{\uparrow F}}\big)\big)\triangleright\gunderline{\text{zip}_{F}}\bef\text{zip}_{G}^{\uparrow F}\bef\tilde{\varepsilon}_{1,23}^{\downarrow G\uparrow F}\\
{\color{greenunder}\text{naturality law of }\text{zip}_{F}:}\quad & =\big(p\times\text{zip}_{F}(q\times r)\big)\triangleright\text{zip}_{F}\bef\big(g\times k^{:G^{B}\times G^{C}}\!\rightarrow\gunderline{g\times\text{zip}_{G}(k)\big)^{\uparrow F}\bef\text{zip}_{G}^{\uparrow F}\bef\tilde{\varepsilon}_{1,23}^{\downarrow G\uparrow F}}\\
{\color{greenunder}\text{composition under }^{\uparrow F}:}\quad & =\text{zip}_{F}\big(p\times\text{zip}_{F}(q\times r)\big)\triangleright\big(g\times(h\times j)\rightarrow\text{zip}_{G}(g\times\text{zip}_{G}(h\times j))\triangleright\tilde{\varepsilon}_{1,23}^{\downarrow G}\big)^{\uparrow F}\quad.
\end{align*}
Rewrite the right-hand side of the associativity law in a similar
way:
\begin{align*}
{\color{greenunder}\text{right-hand side}:}\quad & \text{zip}_{C}\big(\text{zip}_{C}(p\times q)\times r\big)\triangleright\tilde{\varepsilon}_{12,3}^{\downarrow C}=\big(\gunderline{\text{zip}_{C}(p\times q)}\times r\big)\triangleright\text{zip}_{F}\bef\text{zip}_{G}^{\uparrow F}\bef\tilde{\varepsilon}_{12,3}^{\downarrow C}\\
 & =\big(\big(\text{zip}_{F}(p\times q)\triangleright\gunderline{\text{zip}_{G}^{\uparrow F}}\big)\times r\big)\triangleright\gunderline{\text{zip}_{F}}\bef\text{zip}_{G}^{\uparrow F}\bef\tilde{\varepsilon}_{12,3}^{\downarrow G\uparrow F}\\
{\color{greenunder}\text{naturality law of }\text{zip}_{F}:}\quad & =\big(\text{zip}_{F}(p\times q)\times r\big)\triangleright\text{zip}_{F}\bef\big(k^{:G^{A}\times G^{B}}\!\times j\rightarrow\gunderline{\text{zip}_{G}(k)\times j\big)^{\uparrow F}\bef\text{zip}_{G}^{\uparrow F}\bef\tilde{\varepsilon}_{12,3}^{\downarrow G\uparrow F}}\\
{\color{greenunder}\text{composition under }^{\uparrow F}:}\quad & =\text{zip}_{F}\big(\text{zip}_{F}(p\times q)\times r\big)\triangleright\big((g\times h)\times j\rightarrow\text{zip}_{G}(\text{zip}_{G}(g\times h)\times j)\triangleright\tilde{\varepsilon}_{12,3}^{\downarrow G}\big)^{\uparrow F}\quad.
\end{align*}
The two sides become equal after using the associativity laws of $\text{zip}_{F}$
and $\text{zip}_{G}$:
\begin{align*}
 & \text{zip}_{F}\big(p\times\text{zip}_{F}(q\times r)\big)\triangleright\varepsilon_{1,23}^{\uparrow F}=\text{zip}_{F}\big(\text{zip}_{F}(p\times q)\times r\big)\triangleright\varepsilon_{12,3}^{\uparrow F}\quad,\\
 & \text{zip}_{G}(g\times\text{zip}_{G}(h\times j))\triangleright\tilde{\varepsilon}_{1,23}^{\downarrow G}=\text{zip}_{G}(\text{zip}_{G}(g\times h)\times j)\triangleright\tilde{\varepsilon}_{12,3}^{\downarrow G}\quad.
\end{align*}

To verify the commutativity law for $C$, we assume that the law holds
for $F$ and $G$:
\begin{align*}
{\color{greenunder}\text{expect to equal }(\text{zip}_{C}\bef\text{swap}^{\downarrow C}):}\quad & \text{swap}\bef\text{zip}_{C}=\gunderline{\text{swap}\bef\text{zip}_{F}}\bef\text{zip}_{G}^{\uparrow F}\\
{\color{greenunder}\text{commutativity law of }F:}\quad & =\text{zip}_{F}\bef\gunderline{\text{swap}^{\uparrow F}\bef\text{zip}_{G}^{\uparrow F}}=\text{zip}_{F}\bef(\gunderline{\text{swap}\bef\text{zip}_{G}})^{\uparrow F}\\
{\color{greenunder}\text{commutativity law of }G\text{ under }^{\uparrow F}:}\quad & =\text{zip}_{F}\bef\text{zip}_{G}^{\uparrow F}\bef\text{swap}^{\downarrow G\uparrow F}=\text{zip}_{C}\bef\text{swap}^{\downarrow C}\quad.
\end{align*}
$\square$

\paragraph{Products}

This construction for applicative contrafunctors is similar to Statement~\ref{subsec:Statement-applicative-contrafunctor-product}:

\subsubsection{Statement \label{subsec:Statement-applicative-contrafunctor-product}\ref{subsec:Statement-applicative-contrafunctor-product}}

If $F^{\bullet}$ and $G^{\bullet}$ are applicative contrafunctors
then the contrafunctor $C^{A}\triangleq F^{A}\times G^{A}$ is also
applicative. If $F$ and $G$ are commutative then $C$ is also commutative.

\subparagraph{Proof}

Exercise~\ref{subsec:Exercise-applicative-II-5}. $\square$

\paragraph{Co-products}

The co-product of applicative contrafunctors is always applicative: 

\subsubsection{Statement \label{subsec:Statement-applicative-contrafunctor-co-product}\ref{subsec:Statement-applicative-contrafunctor-co-product}}

If $F^{\bullet}$ and $G^{\bullet}$ are applicative contrafunctors
then the contrafunctor $C^{A}\triangleq F^{A}+G^{A}$ is also applicative.
If $F$ and $G$ are commutative then $C$ is also commutative.

\subparagraph{Proof}

We need to implement the methods $\text{zip}_{C}$ and $\text{wu}_{C}$.
Since the type $F^{A}+G^{A}$ is completely symmetric in $F$ and
$G$ and the requirements for $F$ and $G$ are the same, there are
two ways of defining the applicative methods for $F^{A}+G^{A}$ where
either $F$ or $G$ is in some sense \textsf{``}preferred\textsf{''}.

To implement $\text{zip}_{C}$, we first transform its type signature
to an equivalent type:
\begin{align*}
 & C^{A}\times C^{B}\rightarrow C^{A\times B}\cong(F^{A}+G^{A})\times(F^{B}+G^{B})\rightarrow F^{A\times B}+G^{A\times B}\\
 & \cong F^{A}\times F^{B}+F^{A}\times G^{B}+G^{A}\times F^{B}+G^{A}\times G^{B}\rightarrow F^{A\times B}+G^{A\times B}\quad.
\end{align*}
The code must fill a matrix of the following form:
\[
\text{zip}_{C}\triangleq\,\begin{array}{|c||cc|}
 & F^{A\times B} & G^{A\times B}\\
\hline F^{A}\times F^{B} & \text{zip}_{F} & \bbnum 0\\
F^{A}\times G^{B} & \text{???} & \text{???}\\
G^{A}\times F^{B} & \text{???} & \text{???}\\
G^{A}\times G^{B} & \bbnum 0 & \text{zip}_{G}
\end{array}\quad.
\]
The line with type $F^{A}\times G^{B}\rightarrow F^{A\times B}+G^{A\times B}$
must hard-code the decision of whether to return $F^{A\times B}+\bbnum 0$
or $\bbnum 0+G^{A\times B}$. Choosing arbitrarily to \textsf{``}prefer\textsf{''}
$F$ over $G$, we decide to always return $F^{A\times B}+\bbnum 0$
in that line. To convert $F^{A}$ into $F^{A\times B}$, we use \lstinline!contramap!
with the projection function $\pi_{1}:A\times B\rightarrow A$, obtaining
$\pi_{1}^{\downarrow F}:F^{A}\rightarrow F^{A\times B}$. The line
with type $G^{A}\times F^{B}\rightarrow F^{A\times B}+G^{A\times B}$
is handled similarly:
\[
\text{zip}_{C}\triangleq\,\begin{array}{|c||cc|}
 & F^{A\times B} & G^{A\times B}\\
\hline F^{A}\times F^{B} & \text{zip}_{F} & \bbnum 0\\
F^{A}\times G^{B} & p^{:F^{A}}\times\_^{:G^{B}}\rightarrow p\triangleright\pi_{1}^{\downarrow F} & \bbnum 0\\
G^{A}\times F^{B} & \_^{:G^{A}}\times q^{:F^{B}}\rightarrow q\triangleright\pi_{2}^{\downarrow F} & \bbnum 0\\
G^{A}\times G^{B} & \bbnum 0 & \text{zip}_{G}
\end{array}\quad.
\]
This function will sometimes ignore its argument when that argument
has type $\bbnum 0+G^{\bullet}$.

Looking at the possible implementations of $\text{wu}_{C}$ (of type
$F^{\bbnum 1}+G^{\bbnum 1}$), we find two choices:
\[
\text{wu}_{C}\triangleq\text{wu}_{F}+\bbnum 0\quad,\quad\quad\text{or alternatively}:\quad\text{wu}_{C}\triangleq\bbnum 0+\text{wu}_{G}\quad.
\]
Let us check whether the identity laws hold with any of these choices.
The left identity law is:
\[
\text{zip}_{C}(\text{wu}_{C}\times p)\triangleright\text{ilu}^{\downarrow C}\overset{?}{=}p\quad.
\]
We know that $\text{zip}_{C}$ will sometimes ignore its argument
of type $\bbnum 0+G^{\bullet}$, and yet we need to guarantee that
no information is lost from the argument $p$. At the same time, it
is acceptable if $\text{zip}_{C}(\text{wu}_{C}\times p)$ ignored
the argument $\text{wu}_{C}$. So, we need to choose $\text{wu}_{C}\triangleq\bbnum 0+\text{wu}_{G}$. 

With this choice, we can now verify the left identity law:
\begin{align*}
 & \text{zip}_{C}(\text{wu}_{C}\times p)\triangleright\text{ilu}^{\downarrow C}=\big((\bbnum 0^{:F^{\bbnum 1}}+\text{wu}_{G})\times p\big)\triangleright\,\begin{array}{|c||cc|}
 & F^{\bbnum 1\times B} & G^{\bbnum 1\times B}\\
\hline F^{\bbnum 1}\times F^{B} & \text{zip}_{F} & \bbnum 0\\
F^{\bbnum 1}\times G^{B} & p\times\_^{:G^{B}}\rightarrow p\triangleright\pi_{1}^{\downarrow F} & \bbnum 0\\
G^{\bbnum 1}\times F^{B} & \_^{:G^{\bbnum 1}}\times q^{:F^{B}}\rightarrow q\triangleright\pi_{2}^{\downarrow F} & \bbnum 0\\
G^{\bbnum 1}\times G^{B} & \bbnum 0 & \text{zip}_{G}
\end{array}\bef\text{ilu}^{\downarrow C}\\
 & =p\triangleright\,\begin{array}{|c||cc|}
 & F^{\bbnum 1\times B} & G^{\bbnum 1\times B}\\
\hline F^{B} & \pi_{2}^{\downarrow F} & \bbnum 0\\
G^{B} & \bbnum 0 & g\rightarrow\text{zip}_{G}(\text{wu}_{G}\times g)
\end{array}\,\bef\,\begin{array}{|c||cc|}
 & F^{B} & G^{B}\\
\hline F^{\bbnum 1\times B} & \text{ilu}^{\downarrow F} & \bbnum 0\\
G^{\bbnum 1\times B} & \bbnum 0 & \text{ilu}^{\downarrow G}
\end{array}\\
 & =p\triangleright\,\,\begin{array}{|c||cc|}
 & F^{B} & G^{B}\\
\hline F^{B} & \gunderline{\pi_{2}^{\downarrow F}\bef\text{ilu}^{\downarrow F}} & \bbnum 0\\
G^{B} & \bbnum 0 & g\rightarrow\gunderline{\text{zip}_{G}(\text{wu}_{G}\times g)\triangleright\text{ilu}^{\downarrow F}}
\end{array}\,=p\triangleright\,\begin{array}{|c||cc|}
 & F^{B} & G^{B}\\
\hline F^{B} & \text{id} & \bbnum 0\\
G^{B} & \bbnum 0 & g\rightarrow g
\end{array}\,=p\triangleright\text{id}=p\quad.
\end{align*}

The right identity law is verified in a similar way:
\begin{align*}
 & \text{zip}_{C}(p\times\text{wu}_{C})\triangleright\text{iru}^{\downarrow C}=p\triangleright\,\begin{array}{|c||cc|}
 & F^{A} & G^{A}\\
\hline F^{A} & \gunderline{\pi_{1}^{\downarrow F}\bef\text{iru}^{\downarrow F}} & \bbnum 0\\
G^{A} & \bbnum 0 & g\rightarrow\gunderline{\text{zip}_{G}(g\times\text{wu}_{G})\triangleright\text{iru}^{\downarrow F}}
\end{array}\\
 & =p\triangleright\,\begin{array}{|c||cc|}
 & F^{A} & G^{A}\\
\hline F^{A} & \text{id} & \bbnum 0\\
G^{A} & \bbnum 0 & g\rightarrow g
\end{array}\,=p\triangleright\text{id}=p\quad.
\end{align*}

If we defined $\text{wu}_{C}\triangleq\text{wu}_{F}+\bbnum 0$, the
identity laws would have failed to hold. It follows that the choice
of \textsf{``}preferred\textsf{''} $F$ in the code of $\text{zip}_{C}$ needs to
be accompanied by the definition $\text{wu}_{C}\triangleq\bbnum 0+\text{wu}_{G}$.
The other choice (\textsf{``}prefer $G$\textsf{''}) in $\text{zip}_{C}$ would require
us to define $\text{wu}_{C}\triangleq\text{wu}_{F}+\bbnum 0$. Both
definitions yield lawful \lstinline!zip! and \lstinline!wu! methods
for the contrafunctor $C$ and differ only by swapping $F$ and $G$.
In the rest of this proof, we stick to the definition that \textsf{``}prefers\textsf{''}
$F$.

Next, we verify the associativity law~(\ref{eq:applicative-contrafunctor-associativity-law})
of $\text{zip}_{C}$. The expression $\text{zip}_{C}(p\times\text{zip}_{C}(q\times r))$
has eight cases depending on whether $p$, $q$, and $r$ are in the
$F$ or in the $G$ parts of their disjunctive types. Let us look
at the conditions for the result of a \lstinline!zip! operation to
be of type $F^{\bullet}+\bbnum 0$ or $\bbnum 0+G^{\bullet}$. According
to the code matrix of $\text{zip}_{C}$, the result of computing $\text{zip}_{C}(p\times q)$
is type $\bbnum 0+G^{\bullet}$ only when both $p$ and $q$ are in
their $G$ parts. In that case, we find that $\text{zip}_{C}$ is
reduced to $\text{zip}_{G}$:
\[
\text{zip}_{C}\big(p^{:\bbnum 0+G^{A}}\times q^{:\bbnum 0+G^{B}}\big)=\text{zip}_{C}\big((\bbnum 0+g^{:G^{A}})\times(\bbnum 0+h^{:G^{B}})\big)=\bbnum 0^{:F^{A\times B}}+\text{zip}_{G}(g\times h)\quad.
\]
So, if all of $p$, $q$, $r$ are of type $\bbnum 0+G^{\bullet}$,
the associativity law of $\text{zip}_{C}$ is reduced to the associativity
law of $\text{zip}_{G}$. Similarly, if all of $p$, $q$, $r$ are
of type $F^{\bullet}+\bbnum 0$, the associativity law of $\text{zip}_{C}$
is reduced to the associativity law of $\text{zip}_{F}$. Since the
laws of $\text{zip}_{F}$ and $\text{zip}_{G}$ hold by assumption,
we will not need to consider these two cases any further. 

To verify the associativity law in the remaining cases, write $p\times\text{zip}_{C}(q\times r)$
separately:
\begin{align*}
 & p\times\text{zip}_{C}(q\times r)=(q\times r)\triangleright\,\begin{array}{|c||cc|}
 & C^{A}\times F^{B\times C} & C^{A}\times G^{B\times C}\\
\hline F^{B}\times F^{C} & g\times h\rightarrow p\times\text{zip}_{F}(g\times h) & \bbnum 0\\
F^{B}\times G^{C} & f\times\_\rightarrow p\times(f\triangleright\pi_{1}^{\downarrow F}) & \bbnum 0\\
G^{B}\times F^{C} & \_\times f\rightarrow p\times(f\triangleright\pi_{2}^{\downarrow F}) & \bbnum 0\\
G^{B}\times G^{C} & \bbnum 0 & g\times h\rightarrow p\times\text{zip}_{G}(g\times h)
\end{array}\quad.
\end{align*}
We can now compute $\text{zip}_{C}(p\times\text{zip}_{C}(q\times r))$,
which always returns values of type $F^{\bullet}+\bbnum 0$:
\begin{align*}
 & \text{zip}_{C}(p\times\text{zip}_{C}(q\times r))=(p\times q\times r)\triangleright\,\begin{array}{|c||cc|}
 & F^{A\times(B\times C)} & G^{A\times(B\times C)}\\
\hline F^{A}\times F^{B}\times G^{C} & f\times g\times\_\rightarrow\text{zip}_{F}(f\times(g\triangleright\pi_{1}^{\downarrow F})) & \bbnum 0\\
F^{A}\times G^{B}\times F^{C} & f\times\_\times h\rightarrow\text{zip}_{F}(f\times(h\triangleright\pi_{2}^{\downarrow F})) & \bbnum 0\\
F^{A}\times G^{B}\times G^{C} & f\times\_\times\_\rightarrow f\triangleright\pi_{1}^{\downarrow F} & \bbnum 0\\
G^{A}\times F^{B}\times F^{C} & \_\times g\times h\rightarrow\text{zip}_{F}(g\times h)\triangleright\pi_{2}^{\downarrow F} & \bbnum 0\\
G^{A}\times F^{B}\times G^{C} & \_\times g\times\_\rightarrow g\triangleright\pi_{1}^{\downarrow F}\bef\pi_{2}^{\downarrow F} & \bbnum 0\\
G^{A}\times G^{B}\times F^{C} & \_\times\_\times h\rightarrow h\triangleright\pi_{2}^{\downarrow F}\bef\pi_{2}^{\downarrow F} & \bbnum 0
\end{array}\quad.
\end{align*}
The expression $\text{zip}_{C}(\text{zip}_{C}(p\times q)\times r)$
in the right-hand side of the associativity law is written as:
\begin{align*}
 & \text{zip}_{C}(\text{zip}_{C}(p\times q)\times r)=(p\times q\times r)\triangleright\,\begin{array}{|c||cc|}
 & F^{(A\times B)\times C} & G^{(A\times B)\times C}\\
\hline F^{A}\times F^{B}\times G^{C} & f\times g\times\_\rightarrow\text{zip}_{F}(f\times g)\triangleright\pi_{1}^{\downarrow F} & \bbnum 0\\
F^{A}\times G^{B}\times F^{C} & f\times\_\times h\rightarrow\text{zip}_{F}((f\triangleright\pi_{1}^{\downarrow F})\times h) & \bbnum 0\\
F^{A}\times G^{B}\times G^{C} & f\times\_\times\_\rightarrow f\triangleright\pi_{1}^{\downarrow F}\bef\pi_{1}^{\downarrow F} & \bbnum 0\\
G^{A}\times F^{B}\times F^{C} & \_\times g\times h\rightarrow\text{zip}_{F}((g\triangleright\pi_{2}^{\downarrow F})\times h) & \bbnum 0\\
G^{A}\times F^{B}\times G^{C} & \_\times g\times\_\rightarrow g\triangleright\pi_{2}^{\downarrow F}\bef\pi_{1}^{\downarrow F} & \bbnum 0\\
G^{A}\times G^{B}\times F^{C} & \_\times\_\times h\rightarrow h\triangleright\pi_{2}^{\downarrow F} & \bbnum 0
\end{array}\quad.
\end{align*}
 To show that the two sides are equal, it remains to use the naturality
law of $\text{zip}_{F}$:
\[
\text{zip}_{F}\big((p\triangleright s^{\downarrow F})\times q\big)=\text{zip}_{F}(p\times q)\triangleright(a\times b\rightarrow s(a)\times b)^{\downarrow F}\quad,
\]
and to apply the tuple-rearranging isomorphisms. For instance, in
the row for $F^{A}\times G^{B}\times F^{C}$ we get:
\begin{align*}
 & \text{zip}_{F}(f\times(h\triangleright\pi_{2}^{\downarrow F}))\triangleright\tilde{\varepsilon}_{1,23}^{\downarrow F}=\text{zip}_{F}(f\times h)\triangleright(a\times(b\times c)\rightarrow a\times c)^{\downarrow F}\bef(a\times b\times c\rightarrow a\times(b\times c))^{\downarrow F}\\
 & \quad=\text{zip}_{F}(f\times h)\triangleright(a\times b\times c\rightarrow a\times c)^{\downarrow F}\quad.\\
 & \text{zip}_{F}((f\triangleright\pi_{1}^{\downarrow F})\times h)\triangleright\tilde{\varepsilon}_{12,3}^{\downarrow F}=\text{zip}_{F}(f\times h)\triangleright((a\times b)\times c\rightarrow a\times c)^{\downarrow F}\bef(a\times b\times c\rightarrow(a\times b)\times c)^{\downarrow F}\\
 & \quad=\text{zip}_{F}(f\times h)\triangleright(a\times b\times c\rightarrow a\times c)^{\downarrow F}\quad.
\end{align*}
In a similar way, we can show that the two sides are equal for all
other rows of the matrices.

To verify the commutativity law of $\text{zip}_{C}$ when that law
holds for $\text{zip}_{F}$ and $\text{zip}_{G}$:
\begin{align*}
{\color{greenunder}\text{left-hand side}:}\quad & \text{swap}\bef\text{zip}_{C}=\,\begin{array}{|c||cc|}
 & F^{B\times A} & G^{B\times A}\\
\hline F^{A}\times F^{B} & \text{swap}\bef\text{zip}_{F} & \bbnum 0\\
F^{A}\times G^{B} & p^{:F^{A}}\times\_^{:G^{B}}\rightarrow p\triangleright\pi_{2}^{\downarrow F} & \bbnum 0\\
G^{A}\times F^{B} & \_^{:G^{A}}\times q^{:F^{B}}\rightarrow q\triangleright\pi_{1}^{\downarrow F} & \bbnum 0\\
G^{A}\times G^{B} & \bbnum 0 & \text{swap}\bef\text{zip}_{G}
\end{array}\quad,\\
{\color{greenunder}\text{right-hand side}:}\quad & \text{zip}_{C}\bef\text{swap}^{\downarrow C}=\,\begin{array}{|c||cc|}
 & F^{B\times A} & G^{B\times A}\\
\hline F^{A}\times F^{B} & \text{zip}_{F}\bef\text{swap}^{\downarrow F} & \bbnum 0\\
F^{A}\times G^{B} & p^{:F^{A}}\times\_^{:G^{B}}\rightarrow p\triangleright\pi_{1}^{\downarrow F}\bef\text{swap}^{\downarrow F} & \bbnum 0\\
G^{A}\times F^{B} & \_^{:G^{A}}\times q^{:F^{B}}\rightarrow q\triangleright\pi_{2}^{\downarrow F}\bef\text{swap}^{\downarrow F} & \bbnum 0\\
G^{A}\times G^{B} & \bbnum 0 & \text{zip}_{G}\bef\text{swap}^{\downarrow G}
\end{array}\quad.
\end{align*}
The two sides are equal due to the commutativity laws of $\text{zip}_{F}$
and $\text{zip}_{G}$, and due to the properties
\[
\text{swap}\bef\pi_{1}=(a\times b\rightarrow b\times a)\bef(c\times d\rightarrow c)=a\times b\rightarrow b=\pi_{2}\quad,\quad\quad\text{swap}\bef\pi_{2}=\pi_{1}\quad.
\]
$\square$

\paragraph{Function types}

The construction $P^{A}\triangleq H^{A}\rightarrow G^{A}$ for applicative
contrafunctors has no analog for applicative \emph{functors}. Exercise~\ref{subsec:Exercise-function-type-construction-not-applicative}
shows simple examples where a function type construction fails to
produce applicative functors. However, the function type construction
works for a wide class of applicative contrafunctors:

\subsubsection{Statement \label{subsec:Statement-applicative-contrafunctor-exponential}\ref{subsec:Statement-applicative-contrafunctor-exponential}}

If $G^{\bullet}$ is an applicative contrafunctor and $H^{\bullet}$
is \emph{any functor} then the contrafunctor $P^{A}\triangleq H^{A}\rightarrow G^{A}$
is applicative. If $G^{\bullet}$ is commutative then $P^{\bullet}$
is also commutative.

\subparagraph{Proof}

We implement the \lstinline!zip! and \lstinline!wu! methods like
this:
\begin{align*}
 & \text{zip}_{P}:(H^{A}\rightarrow G^{A})\times(H^{B}\rightarrow G^{B})\rightarrow H^{A\times B}\rightarrow G^{A\times B}\quad,\\
 & \text{zip}_{P}\big(p^{:H^{A}\rightarrow G^{A}}\times q^{:H^{B}\rightarrow G^{B}}\big)\triangleq h^{:H^{A\times B}}\rightarrow\text{zip}_{G}\big(p(h\triangleright\pi_{1}^{\uparrow H})\times q(h\triangleright\pi_{2}^{\uparrow H})\big)\quad,\\
 & \text{wu}_{P}\triangleq\_^{:H^{\bbnum 1}}\rightarrow\text{wu}_{G}\quad.
\end{align*}
We can equivalently write the definition of $\text{zip}_{P}$ in a
point-free form, omitting the argument $h$:
\[
\text{zip}_{P}(p\times q)\triangleq\Delta\bef(\pi_{1}^{\uparrow H}\boxtimes\pi_{2}^{\uparrow H})\bef(p\boxtimes q)\bef\text{zip}_{G}\quad.
\]

The code for lifting to $P$ is standard:
\[
f^{\downarrow P}=p^{:H^{A}\rightarrow G^{A}}\rightarrow f^{\uparrow H}\bef p\bef f^{\downarrow G}\quad.
\]

To verify the left identity law, we need to show that
\[
h^{:H^{A}}\triangleright\big(\text{zip}_{P}(\text{wu}_{P}\times p^{:H^{A}\rightarrow G^{A}})\triangleright\text{ilu}^{\downarrow P}\big)\overset{?}{=}h\triangleright p=p(h)\quad.
\]
We use the properties such as $\text{ilu}\bef\pi_{2}=\text{id}$ and
compute:
\begin{align*}
 & h\triangleright\text{ilu}^{\uparrow H}\triangleright\text{zip}_{P}(\text{wu}_{P}\times p)\triangleright\text{ilu}^{\downarrow G}=\text{zip}_{G}\big(\text{wu}_{P}(h\triangleright\text{ilu}^{\uparrow H}\triangleright\pi_{1}^{\uparrow H})\times p(h\triangleright\text{ilu}^{\uparrow H}\triangleright\pi_{2}^{\uparrow H})\big)\triangleright\text{ilu}^{\downarrow G}\\
 & =\text{zip}_{G}(\text{wu}_{G}\times p(h))\triangleright\text{ilu}^{\downarrow G}=p(h)\quad.
\end{align*}

The right identity law is verified by a similar calculation:
\begin{align*}
 & h\triangleright\text{iru}^{\uparrow H}\triangleright\text{zip}_{P}(p\times\text{wu}_{P})\triangleright\text{iru}^{\downarrow G}=\text{zip}_{G}(p(h\triangleright\text{iru}^{\uparrow H}\triangleright\pi_{1}^{\uparrow H})\times\text{wu}_{P}(h\triangleright\text{iru}^{\uparrow H}\triangleright\pi_{2}^{\uparrow H}))\triangleright\text{iru}^{\downarrow G}\\
 & =\text{zip}_{G}(p(h)\times\text{wu}_{G})\triangleright\text{iru}^{\downarrow G}=p(h)\quad.
\end{align*}

To verify the associativity law, we use properties such as $\varepsilon_{12,3}\bef\pi_{2}=\pi_{3}$
and so on:
\begin{align*}
 & h^{:H^{A\times B\times C}}\triangleright\tilde{\varepsilon}_{1,23}^{\uparrow H}\triangleright\text{zip}_{P}(p\times\text{zip}_{P}(q\times r))\triangleright\tilde{\varepsilon}_{1,23}^{\downarrow G}=\text{zip}_{G}\big(p(h\triangleright\gunderline{\tilde{\varepsilon}_{1,23}^{\uparrow H}\bef\pi_{1}^{\uparrow H}})\times\text{zip}_{P}(q\times r)(h\triangleright\gunderline{\tilde{\varepsilon}_{1,23}^{\uparrow H}\bef\pi_{2}^{\uparrow H}})\big)\triangleright\tilde{\varepsilon}_{1,23}^{\downarrow G}\\
 & \quad=\text{zip}_{G}\big(p(h\triangleright\pi_{1}^{\uparrow H})\times\text{zip}_{G}\big(q(h\triangleright\pi_{2}^{\uparrow H})\times r(h\triangleright\pi_{3}^{\uparrow H})\big)\big)\triangleright\tilde{\varepsilon}_{1,23}^{\downarrow G}\quad,\\
 & h^{:H^{A\times B\times C}}\triangleright\tilde{\varepsilon}_{12,3}^{\uparrow H}\triangleright\text{zip}_{P}(\text{zip}_{P}(p\times q)\times r)\triangleright\tilde{\varepsilon}_{12,3}^{\downarrow G}=\text{zip}_{G}\big(\text{zip}_{P}(p\times q)(h\triangleright\gunderline{\tilde{\varepsilon}_{12,3}^{\uparrow H}\bef\pi_{1}^{\uparrow H}})\times r(h\triangleright\gunderline{\tilde{\varepsilon}_{12,3}^{\uparrow H}\bef\pi_{2}^{\uparrow H}})\big)\triangleright\tilde{\varepsilon}_{12,3}^{\downarrow G}\\
 & \quad=\text{zip}_{G}\big(\text{zip}_{G}\big(p(h\triangleright\pi_{1}^{\uparrow H})\times q(h\triangleright\pi_{2}^{\uparrow H})\big)\times r(h\triangleright\pi_{3}^{\uparrow H})\big)\triangleright\tilde{\varepsilon}_{12,3}^{\downarrow G}\quad.
\end{align*}
The two sides are now equal due to the assumed associativity law of
$\text{zip}_{G}$.

It remains to verify the commutativity law, assuming that $\text{zip}_{G}$
satisfies that law:
\begin{align*}
 & \text{zip}_{P}(q\times p)=\Delta\bef(\pi_{1}^{\uparrow H}\boxtimes\pi_{2}^{\uparrow H})\bef(q\boxtimes p)\bef\text{zip}_{G}\quad,\\
 & \text{zip}_{P}(p\times q)\triangleright\text{swap}^{\downarrow P}=\text{swap}^{\uparrow H}\bef\Delta\bef(\pi_{1}^{\uparrow H}\boxtimes\pi_{2}^{\uparrow H})\bef(p\boxtimes q)\bef\gunderline{\text{zip}_{G}\bef\text{swap}^{\downarrow G}}\\
 & \quad=\Delta\bef\gunderline{(\text{swap}^{\uparrow H}\boxtimes\text{swap}^{\uparrow H})\bef(\pi_{1}^{\uparrow H}\boxtimes\pi_{2}^{\uparrow H})}\bef\gunderline{(p\boxtimes q)\bef\text{swap}}\bef\text{zip}_{G}\\
 & \quad=\gunderline{\Delta\bef(\pi_{2}^{\uparrow H}\boxtimes\pi_{1}^{\uparrow H})\bef\text{swap}}\bef(q\boxtimes p)\bef\text{zip}_{G}=\Delta\bef(\pi_{1}^{\uparrow H}\boxtimes\pi_{2}^{\uparrow H})\bef(q\boxtimes p)\bef\text{zip}_{G}\quad.
\end{align*}
$\square$

The constructions shown in this section cover all exponential-polynomial
contrafunctors built up from monoidal fixed types. It follows that
\emph{all} exponential-polynomial contrafunctors with monoidal fixed
types are applicative. The applicative instance will not be unique
if the contrafunctor involves a co-product, because the construction
in Statement~\ref{subsec:Statement-applicative-contrafunctor-co-product}
admits two alternative implementations of the applicative methods.

\subsection{Applicative profunctors: Laws and constructions}

The word \textsf{``}profunctor\index{profunctor}\textsf{''} is used in two ways in
this book (see Section~\ref{subsec:f-Profunctors}). We call a type
constructor $P^{X,Y}$ a profunctor when it is contravariant in $X$
and covariant in $Y$. Given such a profunctor $P^{X,Y}$, we can
set $X=Y$ and obtain a type constructor $Q^{X}\triangleq P^{X,X}$,
which is neither covariant nor contravariant in $X$. This new type
constructor ($Q^{\bullet}$) is also called a \textsf{``}profunctor\textsf{''}. 

When a given type constructor $Q^{A}$ is fully parametric, we can
always separate the covariant and the contravariant occurrences of
the type parameter $A$ in $Q^{A}$. We can then rename the contravariant
occurrences to $X$ and the covariant ones to $Y$, obtaining a type
constructor $P^{X,Y}$ that fits the definition of a profunctor. In
this sense, all fully parametric type constructors are profunctors.

An example of a type constructor that \emph{cannot} be a profunctor
is an \textsf{``}unfunctor\index{unfunctor}\textsf{''}:

\begin{wrapfigure}{l}{0.47\columnwidth}%
\vspace{-1\baselineskip}

\begin{lstlisting}
sealed trait U[A]                // Unfunctor.
final case class U1(s: Double) extends U[Unit]
final case class U2(b: String) extends U[Int]
final case class U3(b: String) extends U[Long]
\end{lstlisting}

\vspace{0.2\baselineskip}
\end{wrapfigure}%

~\vspace{-0.8\baselineskip}
\[
U^{A}\triangleq\text{Double}^{:U^{\bbnum 1}}+\text{String}^{:U^{\text{Int}}}+\text{String}^{:U^{\text{Long}}}\quad.
\]

\noindent The definition of this type constructor is \emph{not} fully
parametric: values of type $U^{A}$ cannot be created for arbitrary
type parameter $A$ (only for $A=\text{Int}$ or $A=\text{Long}$
or $A=\bbnum 1$). This prevents us from implementing any \lstinline!map!-like
methods for $U^{\bullet}$.

Profunctors have an \lstinline!xmap! method instead of a \lstinline!map!
method:
\begin{lstlisting}
def xmap[A, B](f: A => B, g: B => A): P[A] => P[B]
\end{lstlisting}
 For brevity, we will denote Scala code such as \lstinline!xmap(f, g)(p)!
or \lstinline!p.xmap(f, g)! by:
\[
p\triangleright f^{\uparrow P}g^{\downarrow P}\quad,\quad\text{or equivalently}:\quad p\triangleright g^{\downarrow P}f^{\uparrow P}\quad.
\]

The applicative laws for profunctors are formulated via \lstinline!zip!
and \lstinline!wu! methods using \lstinline!xmap!:

\subsubsection{Definition \label{subsec:Definition-applicative-profunctor}\ref{subsec:Definition-applicative-profunctor}}

A profunctor $P^{\bullet}$ is \textbf{applicative} if there exist
methods \lstinline!zip! and \lstinline!wu! such that:
\begin{align*}
 & \text{zip}_{P}:P^{A}\times P^{B}\rightarrow P^{A\times B}\quad,\quad\quad\text{zip}_{P}(p\times\text{zip}_{P}(q\times r))\cong\text{zip}_{P}(\text{zip}_{P}(p\times q)\times r)\quad,\\
 & \text{wu}_{P}:P^{\bbnum 1}\quad,\quad\text{zip}_{P}(\text{wu}_{P}\times p)\cong p\quad,\quad\quad\text{zip}_{P}(p\times\text{wu}_{P})\cong p\quad.
\end{align*}
Here we imply tuple-rearranging isomorphisms $P^{(A\times B)\times C}\cong P^{A\times(B\times C)}$
and $P^{\bbnum 1\times A}\cong P^{A}\cong P^{A\times\bbnum 1}$, which
are implemented via the functions $\varepsilon_{1,23}$, $\varepsilon_{12,3}$,
$\tilde{\varepsilon}_{1,23}$, $\tilde{\varepsilon}_{12,3}$, \lstinline!ilu!,
and \lstinline!iru!:
\begin{align*}
 & \text{zip}_{P}(p\times\text{zip}_{P}(q\times r))\triangleright\varepsilon_{1,23}^{\uparrow P}\tilde{\varepsilon}_{1,23}^{\downarrow P}=\text{zip}_{P}(\text{zip}_{P}(p\times q)\times r)\triangleright\varepsilon_{12,3}^{\uparrow P}\tilde{\varepsilon}_{12,3}^{\downarrow P}\quad,\\
 & \text{zip}_{P}(\text{wu}_{P}\times p)=p\triangleright\text{ilu}^{\uparrow P}\pi_{2}^{\downarrow P}\quad,\quad\quad\text{zip}_{P}(p\times\text{wu}_{P})=p\triangleright\text{iru}^{\uparrow P}\pi_{1}^{\downarrow P}\quad.
\end{align*}

Once the \lstinline!zip! and \lstinline!wu! methods are known, we
can define \lstinline!pure! and \lstinline!xmap2!:
\begin{align*}
 & \text{pu}_{P}:A\rightarrow P^{A}\quad,\quad\quad\text{pu}_{P}(a^{:A})\triangleq\text{wu}_{P}\triangleright(\_^{:A}\rightarrow1)^{\downarrow P}(\_^{:\bbnum 1}\rightarrow a)^{\uparrow P}\quad,\\
 & \text{xmap}_{2}:(A\times B\rightarrow D)\times(D\rightarrow A\times B)\rightarrow P^{A}\times P^{B}\rightarrow P^{D}\quad,\\
 & \text{xmap}_{2}(f^{:A\times B\rightarrow D}\times g^{:D\rightarrow A\times B})\triangleq\text{zip}_{P}\bef(g^{\downarrow C}f^{\uparrow C})\quad.
\end{align*}

It is important that the \lstinline!pure! method ($\text{pu}_{P}$)
for profunctors is defined via the wrapped unit ($\text{wu}_{P}$).
The presence of a value $\text{wu}_{P}:P^{\bbnum 1}$ means that the
profunctor $P^{\bullet}$ is pointed\index{profunctor!pointed} (see
Section~\ref{subsec:Pointed-functors-motivation-equivalence}). For
functors and contrafunctors, the naturality law of \lstinline!pure!
is enough to enforce the equivalence of \lstinline!pure! and \lstinline!wu!.
For profunctors, however, the value $\text{wu}_{P}:P^{\bbnum 1}$
is \emph{not} equivalent to the type of fully parametric functions
with the type signature of \lstinline!pure!. The following example
illustrates this (see also Exercise~\ref{subsec:Exercise-profunctor-pure-not-equivalent-1}).

\subsubsection{Example \label{subsec:Example-profunctor-pure-not-equivalent}\ref{subsec:Example-profunctor-pure-not-equivalent}\index{solved examples}}

For the profunctor $P^{A}\triangleq\left(A\rightarrow A\right)\rightarrow A$,
show that the type $P^{\bbnum 1}$ is \emph{not} equivalent to the
type of fully parametric functions $\text{pu}_{P}:A\rightarrow P^{A}$.

\subparagraph{Solution}

We can rewrite the type of $\text{wu}_{P}$ equivalently as $P^{\bbnum 1}=\left(\bbnum 1\rightarrow\bbnum 1\right)\rightarrow\bbnum 1\cong\bbnum 1$.
So, there is only one value of this type (a function that ignores
its argument and always returns the unit value $1$). The corresponding
\lstinline!pure! method is a function that ignores its argument always
returns the given value:
\[
\text{pu}_{P}\triangleq a^{:A}\rightarrow\_^{:A\rightarrow A}\rightarrow a\quad.
\]
But there are many more functions with the same type signature as
$\text{pu}_{P}$. To see this, it is convenient swap the curried arguments
of $\text{pu}_{P}$ and obtain an equivalent type: 
\[
A\rightarrow P^{A}=A\rightarrow\left(A\rightarrow A\right)\rightarrow A\cong\left(A\rightarrow A\right)\rightarrow A\rightarrow A\quad.
\]
Examples of functions of this type are
\[
f_{1}\triangleq k^{:A\rightarrow A}\rightarrow k\quad,\quad\quad f_{2}\triangleq k^{:A\rightarrow A}\rightarrow(k\bef k)\quad,
\]
and so on. For any non-negative integer $n=0,1,2,...$, we can define
the function $f_{n}$ that applies its argument $n$ times (similar
functions were defined in Example~\ref{subsec:Example-hof-derive-types-2}
and Exercise~\ref{subsec:Exercise-hof-simple-4}). The function \lstinline!pure!
defined above is the same as $f_{0}$. All functions $f_{n}$ are
fully parametric. So, the type of fully parametric functions with
type signature $A\rightarrow P^{A}$ contains many more values than
the type $P^{\bbnum 1}$. $\square$

The commutativity law of applicative profunctors is written like this:
\[
\text{zip}_{P}(q\times p)\triangleright\text{swap}^{\downarrow P}\text{swap}^{\uparrow P}=\text{zip}_{P}(p\times q)\quad.
\]

The rest of this section proves some constructions that produce lawful
applicative profunctors.

\subsubsection{Statement \label{subsec:Statement-applicative-profunctor-composition}\ref{subsec:Statement-applicative-profunctor-composition}}

If $F^{\bullet}$ is an applicative \emph{functor} and $G^{\bullet}$
is an applicative profunctor then the profunctor $P^{A}\triangleq F^{G^{A}}$
(equivalently written as $P\triangleq F\circ G$) is applicative.
In addition, if $G$ is commutative then so is $P$.

\subparagraph{Proof}

Exercise~\ref{subsec:Exercise-applicative-profunctor-composition}.

\subsubsection{Statement \label{subsec:Statement-applicative-profunctor-product}\ref{subsec:Statement-applicative-profunctor-product}}

If $G^{\bullet}$ and $H^{\bullet}$ are applicative profunctors then
so is $P^{A}\triangleq G^{A}\times H^{A}$. If both $G$ and $H$
are commutative then so is $P$.

\subparagraph{Proof}

We follow the proof of Statement~\ref{subsec:Statement-applicative-product}.
Begin by implementing the \lstinline!zip! and \lstinline!wu! methods
for $P$:
\begin{align*}
 & \text{zip}_{P}:(F^{A}\times G^{A})\times(F^{B}\times G^{B})\rightarrow F^{A\times B}\times G^{A\times B}\quad,\\
 & \text{zip}_{P}\big((m^{:F^{A}}\times n^{:G^{A}})\times(p^{:F^{B}}\times q^{:G^{B}})\big)\triangleq\text{zip}_{F}(m\times p)\times\text{zip}_{G}(n\times q)\quad,\\
 & \text{wu}_{P}:F^{\bbnum 1}\times G^{\bbnum 1}\quad,\quad\quad\text{wu}_{P}\triangleq\text{wu}_{F}\times\text{wu}_{G}\quad.
\end{align*}

The lifting to $P$ is defined by $(f^{\uparrow P}g^{\downarrow P})\triangleq(f^{\uparrow F}g^{\downarrow F})\boxtimes(f^{\uparrow G}g^{\downarrow G})$.

To verify the left identity law of $\text{zip}_{P}$:
\begin{align*}
{\color{greenunder}\text{expect to equal }(p\times q)\triangleright\text{ilu}^{\uparrow P}\pi_{2}^{\downarrow P}:}\quad & \text{zip}_{P}\big(\gunderline{\text{wu}_{L}}\times(p^{:F^{A}}\times q^{:G^{A}})\big)=\gunderline{\text{zip}_{P}}((\text{wu}_{F}\times\text{wu}_{G})\times(p\times q))\\
{\color{greenunder}\text{definition of }\text{zip}_{P}:}\quad & =\text{zip}_{F}(\text{wu}_{F}\times p)\times\text{zip}_{G}(\text{wu}_{G}\times q)\\
{\color{greenunder}\text{left identity laws of }\text{zip}_{F}\text{ and }\text{zip}_{G}:}\quad & =(p\triangleright\text{ilu}^{\uparrow F}\pi_{2}^{\downarrow F})\times(q\triangleright\text{ilu}^{\uparrow G}\pi_{2}^{\downarrow G})\\
{\color{greenunder}\text{definition of }\boxtimes:}\quad & =(p\times q)\triangleright\big((\text{ilu}^{\uparrow F}\pi_{2}^{\downarrow F})\boxtimes(\text{ilu}^{\uparrow G}\pi_{2}^{\downarrow G})\big)=(p\times q)\triangleright\text{ilu}^{\uparrow P}\pi_{2}^{\downarrow P}\quad.
\end{align*}

To verify the right identity law of $\text{zip}_{P}$:
\begin{align*}
{\color{greenunder}\text{expect to equal }(p\times q)\triangleright\text{iru}^{\uparrow P}\pi_{1}^{\downarrow P}:}\quad & \text{zip}_{P}\big((p^{:F^{A}}\times q^{:G^{A}})\times\gunderline{\text{wu}_{L}}\big)=\gunderline{\text{zip}_{P}}((p\times q)\times(\text{wu}_{F}\times\text{wu}_{G}))\\
{\color{greenunder}\text{definition of }\text{zip}_{P}:}\quad & =\text{zip}_{F}(p\times\text{wu}_{F})\times\text{zip}_{G}(q\times\text{wu}_{G})\\
{\color{greenunder}\text{right identity laws of }\text{zip}_{F}\text{ and }\text{zip}_{G}:}\quad & =(p\triangleright\text{iru}^{\uparrow F}\pi_{1}^{\downarrow F})\times(q\triangleright\text{iru}^{\uparrow G}\pi_{1}^{\downarrow G})\\
{\color{greenunder}\text{definition of }\boxtimes:}\quad & =(p\times q)\triangleright\big((\text{iru}^{\uparrow F}\pi_{1}^{\downarrow F})\boxtimes(\text{iru}^{\uparrow G}\pi_{1}^{\downarrow G})\big)=(p\times q)\triangleright\text{iru}^{\uparrow P}\pi_{1}^{\downarrow P}\quad.
\end{align*}

To verify the associativity law, begin with its left-hand side and
use the definition of $\text{zip}_{P}$:
\begin{align*}
 & \text{zip}_{P}\big((p_{1}^{:F^{A}}\times p_{2}^{:G^{A}})\times\text{zip}_{P}\big((q_{1}^{:F^{B}}\times q_{2}^{:G^{B}})\times(r_{1}^{:F^{C}}\times r_{2}^{:G^{C}})\big)\big)\triangleright\varepsilon_{1,23}^{\uparrow P}\tilde{\varepsilon}_{1,23}^{\downarrow P}\\
 & =\text{zip}_{P}\big((p_{1}\times p_{2})\times\big(\text{zip}_{F}(q_{1}\times r_{1})\times\text{zip}_{G}(q_{2}\times r_{2})\big)\big)\triangleright\big((\varepsilon_{1,23}^{\uparrow F}\tilde{\varepsilon}_{1,23}^{\downarrow F})\boxtimes(\varepsilon_{1,23}^{\uparrow G}\tilde{\varepsilon}_{1,23}^{\downarrow G})\big)\\
 & =\big(\gunderline{\text{zip}_{F}\big(p_{1}\times\text{zip}_{F}(q_{1}\times r_{1})\big)\triangleright\varepsilon_{1,23}^{\uparrow F}\tilde{\varepsilon}_{1,23}^{\downarrow F}}\big)\times\big(\gunderline{\text{zip}_{G}\big(p_{2}\times\text{zip}_{G}(q_{2}\times r_{2})\big)\triangleright\varepsilon_{1,23}^{\uparrow G}\tilde{\varepsilon}_{1,23}^{\downarrow G}}\big)\quad.
\end{align*}
The right-hand side is rewritten in a similar way:
\begin{align*}
 & \text{zip}_{P}\big(\text{zip}_{P}\big((p_{1}^{:F^{A}}\times p_{2}^{:G^{A}})\times(q_{1}^{:F^{B}}\times q_{2}^{:G^{B}})\big)\times(r_{1}^{:F^{C}}\times r_{2}^{:G^{C}})\big)\triangleright\varepsilon_{12,3}^{\uparrow P}\tilde{\varepsilon}_{12,3}^{\downarrow P}\\
 & =\text{zip}_{P}\big(\big(\text{zip}_{F}(p_{1}\times q_{1})\times\text{zip}_{G}(p_{2}\times q_{2})\big)\times(r_{1}\times r_{2})\big)\big)\triangleright\big((\varepsilon_{12,3}^{\uparrow F}\tilde{\varepsilon}_{12,3}^{\downarrow F})\boxtimes(\varepsilon_{12,3}^{\uparrow G}\tilde{\varepsilon}_{12,3}^{\downarrow G})\big)\\
 & =\big(\gunderline{\text{zip}_{F}\big(\text{zip}_{F}(p_{1}\times q_{1})\times r_{1}\big)\triangleright\varepsilon_{12,3}^{\uparrow F}\tilde{\varepsilon}_{12,3}^{\downarrow F}}\big)\times\big(\gunderline{\text{zip}_{G}\big(\text{zip}_{G}(p_{2}\times q_{2})\times r_{2}\big)\triangleright\varepsilon_{12,3}^{\uparrow G}\tilde{\varepsilon}_{12,3}^{\downarrow G}}\big)\quad.
\end{align*}
The underlined expressions in both sides are equal due to associativity
laws of $\text{zip}_{F}$ and $\text{zip}_{G}$.

To verify the commutativity law of $P$ assuming it holds for $F$
and $G$:
\begin{align*}
{\color{greenunder}\text{expect }\text{zip}_{P}\big((p\times q)\times(m\times n)\big):}\quad & \text{zip}_{P}\big((m\times n)\times(p\times q)\big)\triangleright\text{swap}^{\uparrow P}\text{swap}^{\downarrow P}\\
{\color{greenunder}\text{definitions of }\text{zip}_{P}\text{ and }^{\uparrow P}:}\quad & =\big(\text{zip}_{F}(m\times p)\times\text{zip}_{G}(n\times q)\big)\triangleright\big((\text{swap}^{\uparrow F}\text{swap}^{\downarrow F})\boxtimes(\text{swap}^{\uparrow G}\text{swap}^{\downarrow G})\big)\\
{\color{greenunder}\text{definition of }\boxtimes:}\quad & =\big(\text{zip}_{F}(m\times p)\triangleright\text{swap}^{\uparrow F}\text{swap}^{\downarrow F}\big)\times\big(\text{zip}_{G}(n\times q)\triangleright\text{swap}^{\uparrow G}\text{swap}^{\downarrow G}\big)\\
{\color{greenunder}\text{commutativity of }F\text{ and }G:}\quad & =\text{zip}_{F}(p\times m)\times\text{zip}_{G}(q\times n)=\text{zip}_{P}\big((p\times q)\times(m\times n)\big)\quad.
\end{align*}
$\square$

\subsubsection{Statement \label{subsec:Statement-applicative-profunctor-co-product-1}\ref{subsec:Statement-applicative-profunctor-co-product-1}}

If $F^{\bullet}$ is an applicative profunctor and $Z$ is a fixed
monoid type then the profunctor $P^{A}\triangleq Z+F^{A}$ is also
applicative:
\begin{align*}
 & \text{zip}_{P}:(Z+F^{A})\times(Z+F^{B})\rightarrow Z+F^{A\times B}\quad,\quad\quad\text{zip}_{P}\triangleq\,\begin{array}{|c||cc|}
 & Z & F^{A\times B}\\
\hline Z\times Z & z_{1}\times z_{2}\rightarrow z_{1}\oplus z_{2} & \bbnum 0\\
F^{A}\times Z & \_^{:F^{A}}\times z\rightarrow z & \bbnum 0\\
Z\times F^{B} & z\times\_^{:F^{B}}\rightarrow z & \bbnum 0\\
F^{A}\times F^{B} & \bbnum 0 & \text{zip}_{F}
\end{array}\quad.
\end{align*}
The method $\text{wu}_{P}:Z+F^{\bbnum 1}$ is defined by $\text{wu}_{P}\triangleq\bbnum 0^{:Z}+\text{wu}_{F}$.
If $Z$ is a commutative monoid and $F^{\bullet}$ is commutative
then $P^{\bullet}$ is also commutative.

\subparagraph{Proof}

We follow the proof of Statement~\ref{subsec:Statement-co-product-with-constant-functor-applicative}
\emph{mutatis mutandis}.

The lifting to $P$ is defined in the standard way:
\[
(f^{:A\rightarrow B})^{\uparrow P}(g^{:B\rightarrow A})^{\downarrow P}\triangleq\,\begin{array}{|c||cc|}
 & Z & F^{B}\\
\hline Z & \text{id} & \bbnum 0\\
F^{A} & \bbnum 0 & f^{\uparrow F}g^{\downarrow F}
\end{array}\quad.
\]

To verify the left identity law, we use the left identity law of $\text{zip}_{F}$:
\begin{align*}
 & \text{zip}_{P}(\text{wu}_{P}\times p^{:Z+F^{B}})=\text{zip}_{P}((\bbnum 0+\text{wu}_{F})\times p)=(\text{wu}_{F}\times p)\triangleright\,\begin{array}{|c||cc|}
 & Z & F^{\bbnum 1\times B}\\
\hline F^{\bbnum 1}\times Z & \_^{:F^{\bbnum 1}}\times z\rightarrow z & \bbnum 0\\
F^{\bbnum 1}\times F^{B} & \bbnum 0 & \text{zip}_{F}
\end{array}\\
 & =p\triangleright\,\begin{array}{|c||cc|}
 & Z & F^{\bbnum 1\times B}\\
\hline Z & \text{id} & \bbnum 0\\
F^{B} & \bbnum 0 & k^{:F^{B}}\rightarrow\gunderline{\text{zip}_{F}(\text{wu}_{F}\times k)}
\end{array}\,=p\triangleright\,\begin{array}{|c||cc|}
 & Z & F^{\bbnum 1\times B}\\
\hline Z & \text{id} & \bbnum 0\\
F^{B} & \bbnum 0 & k\rightarrow k\triangleright\text{ilu}^{\uparrow F}\pi_{2}^{\downarrow F}
\end{array}\,=p\triangleright\text{ilu}^{\uparrow P}\pi_{2}^{\downarrow P}\quad.
\end{align*}

To verify the right identity law, we write a similar calculation:
\begin{align*}
 & \text{zip}_{P}(p^{:Z+F^{A}}\times\text{wu}_{P})=\text{zip}_{P}(p\times(\bbnum 0+\text{wu}_{F}))=(p\times\text{wu}_{F})\triangleright\,\begin{array}{|c||cc|}
 & Z & F^{A\times\bbnum 1}\\
\hline Z\times F^{\bbnum 1} & z\times\_^{:F^{\bbnum 1}}\rightarrow z & \bbnum 0\\
F^{A}\times F^{\bbnum 1} & \bbnum 0 & \text{zip}_{F}
\end{array}\\
 & =p\triangleright\,\begin{array}{|c||cc|}
 & Z & F^{A\times\bbnum 1}\\
\hline Z & \text{id} & \bbnum 0\\
F^{A} & \bbnum 0 & k^{:F^{A}}\rightarrow\gunderline{\text{zip}_{F}(k\times\text{wu}_{F})}
\end{array}\,=p\triangleright\,\begin{array}{|c||cc|}
 & Z & F^{A\times\bbnum 1}\\
\hline Z & \text{id} & \bbnum 0\\
F^{A} & \bbnum 0 & k\rightarrow k\triangleright\text{iru}^{\uparrow F}\pi_{1}^{\downarrow F}
\end{array}\,=p\triangleright\text{iru}^{\uparrow P}\pi_{1}^{\downarrow P}\quad.
\end{align*}

To verify the associativity law, we use a trick to avoid long derivations.
The two sides of the associativity law are expressions of type $Z+F^{A\times B\times C}$:
\begin{align*}
{\color{greenunder}\text{left-hand side}:}\quad & \text{zip}_{P}\big(p^{:Z+F^{A}}\times\text{zip}_{P}(q^{:Z+F^{B}}\times r^{:Z+F^{C}})\big)\triangleright\varepsilon_{1,23}^{\uparrow P}\tilde{\varepsilon}_{1,23}^{\downarrow P}\quad,\\
{\color{greenunder}\text{right-hand side}:}\quad & \text{zip}_{P}\big(\text{zip}_{P}(p^{:Z+F^{A}}\times q^{:Z+F^{B}})\times r^{:Z+F^{C}}\big)\triangleright\varepsilon_{12,3}^{\uparrow P}\tilde{\varepsilon}_{12,3}^{\downarrow P}\quad.
\end{align*}
Since each of the arguments $p$, $q$, $r$ may be in one of the
two parts of the disjunction type $Z+F^{\bullet}$, we have 8 cases.
We note, however, that the code of $\text{zip}_{P}(p\times q)$ will
return a value of type $Z+\bbnum 0$ whenever at least one of the
arguments ($p$, $q$) is of type $Z+\bbnum 0$. So, a composition
of two \lstinline!zip! operations will also return a value of type
$Z+\bbnum 0$ whenever at least one of the arguments ($p$, $q$,
$r$) is of type $Z+\bbnum 0$. It remains to consider only two cases: 

\textbf{(1)} At least one of $p$, $q$, $r$ is of type $Z+\bbnum 0$.
In this case, any arguments of type $\bbnum 0+F^{\bullet}$ are ignored
by $\text{zip}_{P}$, while the arguments of type $Z+\bbnum 0$ are
combined using the monoid $Z$\textsf{'}s binary operation ($\oplus$). So,
the result of the \lstinline!zip! operation is the same if we replace
any arguments ($p$, $q$, $r$) of type $\bbnum 0+F^{\bullet}$ by
the empty value $e_{Z}$. For example:
\[
\text{zip}_{P}\big((z+\bbnum 0)\times(\bbnum 0+k^{:F^{A}})\big)=z+\bbnum 0=\text{zip}_{P}\big((z+\bbnum 0)\times(e_{Z}+\bbnum 0)\big)\quad.
\]
After this replacement, we have three arguments ($z_{1}+\bbnum 0$,
$z_{2}+\bbnum 0$, $z_{3}+\bbnum 0$) instead of $p$, $q$, $r$,
and the function $\text{zip}_{P}$ reduces to the operation $\oplus$,
for which the associativity law holds by assumption.

\textbf{(2)} All of $p$, $q$, $r$ are of type $\bbnum 0+F^{\bullet}$.
In this case, $\text{zip}_{P}$ reduces to $\text{zip}_{F}$, which
satisfies the associativity law by assumption.

To verify the commutativity law of $P$, use the code matrix for \lstinline!swap!
with the relevant types:
\begin{align*}
 & \text{swap}\bef\text{zip}_{P}=\,\begin{array}{|c||cccc|}
 & Z\times Z & F^{B}\times Z & Z\times F^{A} & F^{B}\times F^{A}\\
\hline Z\times Z & \text{swap} & \bbnum 0 & \bbnum 0 & \bbnum 0\\
F^{A}\times Z & \bbnum 0 & \bbnum 0 & \text{swap} & \bbnum 0\\
Z\times F^{B} & \bbnum 0 & \text{swap} & \bbnum 0 & \bbnum 0\\
F^{A}\times F^{B} & \bbnum 0 & \bbnum 0 & \bbnum 0 & \text{swap}
\end{array}\,\bef\,\begin{array}{|c||cc|}
 & Z & F^{B\times A}\\
\hline Z\times Z & z_{1}\times z_{2}\rightarrow z_{1}\oplus z_{2} & \bbnum 0\\
F^{B}\times Z & \_\times z\rightarrow z & \bbnum 0\\
Z\times F^{A} & z\times\_\rightarrow z & \bbnum 0\\
F^{B}\times F^{A} & \bbnum 0 & \text{zip}_{F}
\end{array}\\
 & =\,\begin{array}{|c||cc|}
 & Z & F^{B\times A}\\
\hline Z\times Z & z_{1}\times z_{2}\rightarrow z_{2}\oplus z_{1} & \bbnum 0\\
F^{A}\times Z & \_\times z\rightarrow z & \bbnum 0\\
Z\times F^{B} & z\times\_\rightarrow z & \bbnum 0\\
F^{A}\times F^{B} & \bbnum 0 & \text{swap}\bef\text{zip}_{F}
\end{array}\quad.
\end{align*}
By assumption, $\text{swap}\bef\text{zip}_{F}=\text{zip}_{F}\bef\text{swap}^{\uparrow F}\text{swap}^{\downarrow F}$.
The code for $\text{swap}^{\uparrow P}\text{swap}^{\downarrow P}$
is:
\[
\text{swap}^{\uparrow P}\text{swap}^{\downarrow P}=\,\begin{array}{|c||cc|}
 & Z & F^{B\times A}\\
\hline Z & \text{id} & \bbnum 0\\
F^{A\times B} &  & \text{swap}^{\uparrow F}\text{swap}^{\downarrow F}
\end{array}\quad.
\]
We can now transform the right-hand side of the commutativity law:
\[
\text{zip}_{L}\bef\text{swap}^{\uparrow P}\text{swap}^{\downarrow P}=\,\begin{array}{|c||cc|}
 & Z & F^{B\times A}\\
\hline Z\times Z & z_{1}\times z_{2}\rightarrow z_{1}\oplus z_{2} & \bbnum 0\\
F^{A}\times Z & \_\times z\rightarrow z & \bbnum 0\\
Z\times F^{B} & z\times\_\rightarrow z & \bbnum 0\\
F^{A}\times F^{B} & \bbnum 0 & \text{zip}_{F}\bef\text{swap}^{\uparrow F}\text{swap}^{\downarrow F}
\end{array}\quad.
\]
The difference between the sides disappears if $Z$ is a commutative
monoid ($z_{1}\oplus z_{2}=z_{2}\oplus z_{1}$). $\square$

\subsubsection{Statement \label{subsec:Statement-applicative-profunctor-co-product-2}\ref{subsec:Statement-applicative-profunctor-co-product-2}}

If $F^{\bullet}$ and $H^{\bullet}$ are applicative profunctors and
$H^{\bullet}$ is co-pointed with the method $\text{ex}_{F}:H^{A}\rightarrow A$
such that the compatibility law~(\ref{eq:compatibility-law-of-extract-and-zip})
holds, then $P^{A}\triangleq H^{A}+F^{A}$ is also applicative:
\begin{align*}
 & \text{zip}_{P}:(H^{A}+F^{A})\times(H^{B}+F^{B})\rightarrow H^{A\times B}+F^{A\times B}\quad,\\
 & \text{zip}_{P}\triangleq\,\begin{array}{|c||cc|}
 & H^{A\times B} & F^{A\times B}\\
\hline H^{A}\times H^{B} & \text{zip}_{H} & \bbnum 0\\
H^{A}\times F^{B} & \bbnum 0 & ((\text{ex}_{H}\bef\text{pu}_{F})\boxtimes\text{id})\bef\text{zip}_{F}\\
F^{A}\times H^{B} & \bbnum 0 & (\text{id}\boxtimes(\text{ex}_{H}\bef\text{pu}_{F}))\bef\text{zip}_{F}\\
F^{A}\times F^{B} & \bbnum 0 & \text{zip}_{F}
\end{array}\quad;
\end{align*}
The method $\text{wu}_{P}:H^{\bbnum 1}+F^{\bbnum 1}$ is defined by
$\text{wu}_{P}\triangleq\text{wu}_{H}+\bbnum 0^{:F^{\bbnum 1}}$.
If $F^{\bullet}$ and $H^{\bullet}$ are commutative then $P^{\bullet}$
is also commutative.

\subparagraph{Proof}

We follow the proof of Statement~\ref{subsec:Statement-co-product-with-co-pointed-applicative}.
The lifting to $P^{\bullet}$ is defined by:
\[
(f^{:A\rightarrow B})^{\uparrow P}(g^{:B\rightarrow A})^{\downarrow P}\triangleq\,\begin{array}{|c||cc|}
 & H^{B} & F^{B}\\
\hline H^{A} & f^{\uparrow H}g^{\downarrow H} & \bbnum 0\\
F^{A} & \bbnum 0 & f^{\uparrow F}g^{\downarrow F}
\end{array}\quad.
\]

To verify the left identity law, we begin with its left-hand side:
\begin{align*}
 & \text{zip}_{P}(\text{wu}_{P}\times p)=\big((\text{wu}_{H}^{:H^{\bbnum 1}}+\bbnum 0^{:F^{\bbnum 1}})\times p^{:H^{B}+F^{B}}\big)\triangleright\,\begin{array}{|c||cc|}
 & H^{\bbnum 1\times B} & F^{\bbnum 1\times B}\\
\hline H^{\bbnum 1}\times H^{B} & \text{zip}_{H} & \bbnum 0\\
H^{\bbnum 1}\times F^{B} & \bbnum 0 & ((\text{ex}_{H}\bef\text{pu}_{F})\boxtimes\text{id})\bef\text{zip}_{F}\\
F^{\bbnum 1}\times H^{B} & \bbnum 0 & (\text{id}\boxtimes(\text{ex}_{H}\bef\text{pu}_{F}))\bef\text{zip}_{F}\\
F^{\bbnum 1}\times F^{B} & \bbnum 0 & \text{zip}_{F}
\end{array}\\
 & =p\triangleright\,\begin{array}{|c||cc|}
 & H^{\bbnum 1\times B} & F^{\bbnum 1\times B}\\
\hline H^{B} & h\rightarrow\text{zip}_{H}(\text{wu}_{H}\times h) & \bbnum 0\\
F^{B} & \bbnum 0 & f\rightarrow\text{zip}_{F}((\text{wu}_{H}\triangleright\text{ex}_{H}\triangleright\text{pu}_{F})\times f)
\end{array}\quad.
\end{align*}
Using Eq.~(\ref{eq:co-pointed-nondegeneracy-law-wu}) and the definition
of \lstinline!wu! through \lstinline!pure!, we find:
\[
\text{wu}_{H}\triangleright\text{ex}_{H}\triangleright\text{pu}_{F}=1\triangleright\text{pu}_{F}=\text{wu}_{F}\quad.
\]
Since the identity laws of $F$ and $H$ are assumed to hold, we can
transform the last matrix as:
\begin{align*}
 & \begin{array}{|c||cc|}
 & H^{\bbnum 1\times B} & F^{\bbnum 1\times B}\\
\hline H^{B} & h\rightarrow\text{zip}_{H}(\text{wu}_{H}\times h) & \bbnum 0\\
F^{B} & \bbnum 0 & f\rightarrow\text{zip}_{F}((\text{wu}_{H}\triangleright\text{ex}_{H}\triangleright\text{pu}_{F})\times f)
\end{array}\,=\,\begin{array}{|c||cc|}
 & H^{\bbnum 1\times B} & F^{\bbnum 1\times B}\\
\hline H^{B} & \text{ilu}^{\uparrow H}\pi_{2}^{\downarrow H} & \bbnum 0\\
F^{B} & \bbnum 0 & \text{ilu}^{\uparrow F}\pi_{2}^{\downarrow F}
\end{array}\\
 & =\text{ilu}^{\uparrow P}\pi_{2}^{\downarrow P}\quad.
\end{align*}
After this simplification, the left-hand side equals $p\triangleright\text{ilu}^{\uparrow P}\pi_{2}^{\downarrow P}$
(the right-hand side of the law).

The right identity law is verified in a similar way:
\begin{align*}
 & \text{zip}_{P}(p\times\text{wu}_{L})=\text{zip}_{P}\big(p^{:H^{A}+F^{A}}\times(\text{wu}_{H}^{:H^{\bbnum 1}}+\bbnum 0^{:F^{\bbnum 1}})\big)\\
 & =p\triangleright\,\begin{array}{|c||cc|}
 & H^{A\times\bbnum 1} & F^{A\times\bbnum 1}\\
\hline H^{A} & h\rightarrow\text{zip}_{H}(h\times\text{wu}_{H}) & \bbnum 0\\
F^{A} & \bbnum 0 & f\rightarrow\text{zip}_{F}(f\times(\text{wu}_{H}\triangleright\text{ex}_{H}\triangleright\text{pu}_{F}))
\end{array}
\end{align*}
\begin{align*}
 & =p\triangleright\,\,\begin{array}{|c||cc|}
 & H^{A\times\bbnum 1} & F^{A\times\bbnum 1}\\
\hline H^{A} & \text{iru}^{\uparrow H}\pi_{1}^{\downarrow H} & \bbnum 0\\
F^{A} & \bbnum 0 & \text{iru}^{\uparrow F}\pi_{1}^{\downarrow F}
\end{array}\,=p\triangleright\text{iru}^{\uparrow P}\pi_{1}^{\downarrow P}\quad.
\end{align*}

The associativity law is an equation between values of type $H^{A\times B\times C}+F^{A\times B\times C}$:
\[
\text{zip}_{P}(p^{:H^{A}+F^{A}}\times\text{zip}_{P}(q^{:H^{B}+F^{B}}\times r^{:H^{C}+F^{C}}))\triangleright\varepsilon_{1,23}^{\uparrow P}\tilde{\varepsilon}_{1,23}^{\downarrow P}=\text{zip}_{P}(\text{zip}_{P}(p\times q)\times r)\triangleright\varepsilon_{12,3}^{\uparrow P}\tilde{\varepsilon}_{12,3}^{\downarrow P}\quad.
\]
The operation $\text{zip}_{P}(p\times q)$ is defined in such a way
that it returns a value of type $H^{A\times B}+\bbnum 0$ only when
both $p$ and $q$ are in the left part of the disjunction:
\[
\text{zip}_{P}\big((a^{:H^{A}}+\bbnum 0^{:F^{A}})\times(b^{:H^{B}}+\bbnum 0^{:F^{B}})\big)=\text{zip}_{H}(a\times b)+\bbnum 0^{:F^{A\times B}}\quad.
\]
Otherwise, $\text{zip}_{P}(p\times q)$ returns a value of type $\bbnum 0^{:H^{A\times B}}+F^{A\times B}$.
So, let us consider three cases:

\textbf{(1)} The arguments are $p=a^{:H^{A}}+\bbnum 0$, $q=b^{:H^{B}}+\bbnum 0$,
$r=c^{:H^{C}}+\bbnum 0$. In this case, $\text{zip}_{P}$ reduces
to $\text{zip}_{H}$:
\begin{align*}
{\color{greenunder}\text{left-hand side}:}\quad & \text{zip}_{P}(p\times\text{zip}_{P}(q\times r))\triangleright\varepsilon_{1,23}^{\uparrow P}\tilde{\varepsilon}_{1,23}^{\downarrow P}=\text{zip}_{H}\big(a\times\text{zip}_{H}(b\times c)\big)\triangleright\varepsilon_{1,23}^{\uparrow H}\tilde{\varepsilon}_{1,23}^{\downarrow H}+\bbnum 0\quad,\\
{\color{greenunder}\text{right-hand side}:}\quad & \text{zip}_{P}(\text{zip}_{P}(p\times q)\times r)\triangleright\varepsilon_{12,3}^{\uparrow P}\tilde{\varepsilon}_{12,3}^{\downarrow P}=\text{zip}_{H}\big(\text{zip}_{H}(a\times b)\times c\big)\triangleright\varepsilon_{12,3}^{\uparrow H}\tilde{\varepsilon}_{12,3}^{\downarrow H}+\bbnum 0\quad.
\end{align*}
The two sides are equal due to the associativity law of $\text{zip}_{H}$.

\textbf{(2)} The argument $q$ has type $\bbnum 0+F^{B}$. In this
case, $\text{zip}_{P}$ reduces to $\text{zip}_{F}$ after converting
arguments of type $H^{\bullet}+0$ to type $F^{\bullet}$ when needed.
We may define this conversion as a helper function \lstinline!toF!
in the same way as in the proof of Statement~\ref{subsec:Statement-co-product-with-co-pointed-applicative}.
The associativity law of $\text{zip}_{P}$ is then reduced to the
same law of $\text{zip}_{F}$:
\begin{align*}
 & \text{zip}_{P}(p\times\text{zip}_{P}(q\times r))\triangleright\varepsilon_{1,23}^{\uparrow P}\tilde{\varepsilon}_{1,23}^{\downarrow P}=\bbnum 0+\text{zip}_{F}\big(\text{toF}\left(p\right)\times\text{zip}_{F}(\text{toF}\left(q\right)\times\text{toF}\left(r\right))\big)\triangleright\varepsilon_{1,23}^{\uparrow F}\tilde{\varepsilon}_{1,23}^{\downarrow F}\quad,\\
 & \text{zip}_{P}(\text{zip}_{P}(p\times q)\times r)\triangleright\varepsilon_{12,3}^{\uparrow P}\tilde{\varepsilon}_{12,3}^{\downarrow P}=\bbnum 0+\text{zip}_{F}(\text{zip}_{F}(\text{toF}\left(p\right)\times\text{toF}\left(q\right))\times\text{toF}\left(r\right))\triangleright\varepsilon_{12,3}^{\uparrow F}\tilde{\varepsilon}_{12,3}^{\downarrow F}\quad.
\end{align*}
The two sides are equal due to the associativity law of $\text{zip}_{F}$.

\textbf{(3)} Either $p=\bbnum 0+a^{:F^{A}}$ while $q=b^{:H^{B}}+\bbnum 0$
and $r=c^{:H^{C}}+\bbnum 0$; or $r=\bbnum 0+c^{:F^{C}}$ while $p=a^{:H^{A}}+\bbnum 0$
and $q=b^{:H^{B}}+\bbnum 0$. The two situations are symmetric, so
let us consider the first one:
\begin{align*}
{\color{greenunder}\text{left-hand side}:}\quad & \text{zip}_{P}(p\times\text{zip}_{P}(q\times r))\triangleright\varepsilon_{1,23}^{\uparrow P}\tilde{\varepsilon}_{1,23}^{\downarrow P}\\
 & =\bbnum 0+\text{zip}_{F}\big(\text{toF}\,(p)\times\text{toF}\,(\text{zip}_{H}(b\times c)+\bbnum 0)\big)\triangleright\varepsilon_{1,23}^{\uparrow F}\tilde{\varepsilon}_{1,23}^{\downarrow F}\quad.
\end{align*}
Simplify the sub-expressions involving \lstinline!toF! separately:
\begin{align*}
 & \text{toF}\,(p)=\text{toF}\,(\bbnum 0+a)=a\quad,\\
 & \text{toF}\,(\text{zip}_{H}(b\times c)+\bbnum 0)=\text{pu}_{F}(\gunderline{\text{ex}_{H}(\text{zip}_{H}}(b\times c))\\
{\color{greenunder}\text{use Eq.~(\ref{eq:compatibility-law-of-extract-and-zip})}:}\quad & \quad=\text{pu}_{F}(\text{ex}_{H}(b)\times\text{ex}_{H}(c))\quad.
\end{align*}
For profunctors, the right identity law of \lstinline!zip! and \lstinline!pure!
has the form:
\[
\text{zip}_{F}(a^{:F^{A}}\times\text{pu}_{F}(b^{:B}))=a\triangleright(k^{:A}\rightarrow k\times b)^{\uparrow F}\pi_{1}^{\downarrow F}\quad.
\]
So, we can rewrite the left-hand side of the associativity law like
this:
\begin{align*}
{\color{greenunder}\text{left-hand side}:}\quad & \text{zip}_{P}(p\times\text{zip}_{P}(q\times r))\triangleright\varepsilon_{1,23}^{\uparrow P}\tilde{\varepsilon}_{1,23}^{\downarrow P}=\bbnum 0+\gunderline{\text{zip}_{F}}\big(a\times\gunderline{\text{pu}_{F}}(\text{ex}_{H}(b)\times\text{ex}_{H}(c))\big)\triangleright\varepsilon_{1,23}^{\uparrow F}\tilde{\varepsilon}_{1,23}^{\downarrow F}\\
{\color{greenunder}\text{identity law of }\text{zip}_{F}:}\quad & =\bbnum 0+a\triangleright\big(k^{:A}\rightarrow k\times(\text{ex}_{H}(b)\times\text{ex}_{H}(c))\big)^{\uparrow F}\pi_{1}^{\downarrow F}\triangleright\varepsilon_{1,23}^{\uparrow F}\tilde{\varepsilon}_{1,23}^{\downarrow F}\\
 & =\bbnum 0+a\triangleright\big(k^{:A}\rightarrow k\times\text{ex}_{H}(b)\times\text{ex}_{H}(c)\big)^{\uparrow F}\pi_{1}^{\downarrow F}\quad.
\end{align*}
The right-hand side can be transformed by using \lstinline!toF! on
all arguments:
\begin{align*}
{\color{greenunder}\text{right-hand side}:}\quad & \text{zip}_{P}(\text{zip}_{P}(p\times q)\times r)\triangleright\varepsilon_{12,3}^{\uparrow P}\tilde{\varepsilon}_{12,3}^{\downarrow P}\\
 & =\bbnum 0+\text{zip}_{F}(\text{zip}_{F}(\text{toF}\left(p\right)\times\text{toF}\left(q\right))\times\text{toF}\left(r\right))\triangleright\varepsilon_{12,3}^{\uparrow F}\tilde{\varepsilon}_{12,3}^{\downarrow F}\\
 & =\bbnum 0+\text{zip}_{F}(\text{zip}_{F}(a\times\text{toF}\left(b+\bbnum 0\right))\times\text{toF}\left(r+\bbnum 0\right))\triangleright\varepsilon_{12,3}^{\uparrow F}\tilde{\varepsilon}_{12,3}^{\downarrow F}\quad.
\end{align*}
Simplify the sub-expressions of the form $\text{zip}_{F}(a\times\text{toF}\left(b+\bbnum 0\right))$:
\begin{equation}
\text{zip}_{F}\big(a^{:F^{A}}\times\text{toF}\,(b^{:H^{B}}+\bbnum 0)\big)=\text{zip}_{F}(a\times\text{pu}_{F}(\text{ex}_{H}(b))=a\triangleright(k^{:A}\rightarrow k\times\text{ex}_{H}(b))^{\uparrow F}\pi_{1}^{\downarrow F}\quad.\label{eq:zip-copointed-construction-derivation1-1}
\end{equation}
Using this formula, we continue to transform the right-hand side:
\begin{align*}
{\color{greenunder}\text{right-hand side}:}\quad & \bbnum 0+\gunderline{\text{zip}_{F}}(\text{zip}_{F}(a\times\text{toF}\left(b+\bbnum 0\right))\times\gunderline{\text{toF}\left(c+\bbnum 0\right)})\triangleright\varepsilon_{12,3}^{\uparrow F}\tilde{\varepsilon}_{12,3}^{\downarrow F}\\
{\color{greenunder}\text{use Eq.~(\ref{eq:zip-copointed-construction-derivation1-1})}:}\quad & =\bbnum 0+\gunderline{\text{zip}_{F}}(a\times\gunderline{\text{toF}\left(b+\bbnum 0\right)})\triangleright\big(k\rightarrow k\times\text{ex}_{H}(c)\big)^{\uparrow F}\pi_{1}^{\downarrow F}\triangleright\varepsilon_{12,3}^{\uparrow F}\tilde{\varepsilon}_{12,3}^{\downarrow F}\\
{\color{greenunder}\text{use Eq.~(\ref{eq:zip-copointed-construction-derivation1-1})}:}\quad & =\bbnum 0+a\triangleright\big(k\rightarrow k\times\text{ex}_{H}(b)\big)^{\uparrow F}\pi_{1}^{\downarrow F}\triangleright\big(k\rightarrow k\times\text{ex}_{H}(c)\big)^{\uparrow F}\pi_{1}^{\downarrow F}\triangleright\varepsilon_{12,3}^{\uparrow F}\tilde{\varepsilon}_{12,3}^{\downarrow F}\\
{\color{greenunder}\text{compute composition}:}\quad & =\bbnum 0+a\triangleright\big(k\rightarrow k\times\text{ex}_{H}(b)\times\text{ex}_{H}(c)\big)^{\uparrow F}\pi_{1}^{\downarrow F}\quad.
\end{align*}
The two sides are now equal. 

It remains to verify the commutativity law in case that law holds
for $F$ and $H$:
\[
\text{swap}\bef\text{zip}_{F}\overset{!}{=}\text{zip}_{F}\bef\text{swap}^{\uparrow F}\text{swap}^{\downarrow F}\quad,\quad\text{swap}\bef\text{zip}_{H}\overset{!}{=}\text{zip}_{H}\bef\text{swap}^{\uparrow H}\text{swap}^{\downarrow H}\quad\quad.
\]
Begin with the left-hand side of the commutativity law for $\text{zip}_{P}$:
\[
\text{swap}\bef\text{zip}_{P}=\,\begin{array}{|c||cc|}
 & H^{B\times A} & F^{B\times A}\\
\hline H^{A}\times H^{B} & \text{swap}\bef\text{zip}_{H} & \bbnum 0\\
F^{A}\times H^{B} & \bbnum 0 & \text{swap}\bef((\text{ex}_{H}\bef\text{pu}_{F})\boxtimes\text{id})\bef\text{zip}_{F}\\
H^{A}\times F^{B} & \bbnum 0 & \text{swap}\bef(\text{id}\boxtimes(\text{ex}_{H}\bef\text{pu}_{F}))\bef\text{zip}_{F}\\
F^{A}\times F^{B} & \bbnum 0 & \text{swap}\bef\text{zip}_{F}
\end{array}\quad.
\]
Writing out the compositions of \lstinline!swap! and the pair product
functions, we get:
\begin{align*}
 & \text{swap}\bef((\text{ex}_{H}\bef\text{pu}_{F})\boxtimes\text{id})=(\text{id}\boxtimes(\text{ex}_{H}\bef\text{pu}_{F}))\bef\text{swap}\quad,\\
 & \text{swap}\bef(\text{id}\boxtimes(\text{ex}_{H}\bef\text{pu}_{F}))=(\text{pu}_{F}\boxtimes\text{id})\bef\text{swap}\quad.
\end{align*}
Using these simplifications, we rewrite the left-hand side as:
\[
\text{swap}\bef\text{zip}_{P}=\,\begin{array}{|c||cc|}
 & H^{B\times A} & F^{B\times A}\\
\hline H^{A}\times H^{B} & \text{swap}\bef\text{zip}_{H} & \bbnum 0\\
F^{A}\times H^{B} & \bbnum 0 & (\text{id}\boxtimes(\text{ex}_{H}\bef\text{pu}_{F}))\bef\text{swap}\bef\text{zip}_{F}\\
H^{A}\times F^{B} & \bbnum 0 & ((\text{ex}_{H}\bef\text{pu}_{F})\boxtimes\text{id})\bef\text{swap}\bef\text{zip}_{F}\\
F^{A}\times F^{B} & \bbnum 0 & \text{swap}\bef\text{zip}_{F}
\end{array}\quad.
\]
The right-hand side is rewritten to the same code after using the
laws of $\text{zip}_{F}$ and $\text{zip}_{H}$:
\begin{align*}
 & \text{zip}_{P}\bef\text{swap}^{\uparrow P}\text{swap}^{\downarrow P}\\
 & =\,\begin{array}{|c||cc|}
 & H^{A\times B} & F^{A\times B}\\
\hline H^{A}\times H^{B} & \text{zip}_{H} & \bbnum 0\\
F^{A}\times H^{B} & \bbnum 0 & (\text{id}\boxtimes(\text{ex}_{H}\bef\text{pu}_{F}))\bef\text{zip}_{F}\\
H^{A}\times F^{B} & \bbnum 0 & ((\text{ex}_{H}\bef\text{pu}_{F})\boxtimes\text{id})\bef\text{zip}_{F}\\
F^{A}\times F^{B} & \bbnum 0 & \text{zip}_{F}
\end{array}\,\bef\,\begin{array}{|c||cc|}
 & H^{B\times A} & F^{B\times A}\\
\hline H^{A\times B} & \text{swap}^{\uparrow H}\text{swap}^{\downarrow H} & \bbnum 0\\
F^{A\times B} & \bbnum 0 & \text{swap}^{\uparrow F}\text{swap}^{\downarrow F}
\end{array}\\
 & =\,\,\begin{array}{|c||cc|}
 & H^{B\times A} & F^{B\times A}\\
\hline A\times B & \text{swap}\bef\text{zip}_{H} & \bbnum 0\\
F^{A}\times B & \bbnum 0 & (\text{id}\boxtimes(\text{ex}_{H}\bef\text{pu}_{F}))\bef\text{swap}\bef\text{zip}_{F}\\
A\times F^{B} & \bbnum 0 & ((\text{ex}_{H}\bef\text{pu}_{F})\boxtimes\text{id})\bef\text{swap}\bef\text{zip}_{F}\\
F^{A}\times F^{B} & \bbnum 0 & \text{swap}\bef\text{zip}_{F}
\end{array}\quad.
\end{align*}
The two sides are now equal. $\square$

\subsubsection{Statement \label{subsec:Statement-applicative-profunctor-exponential}\ref{subsec:Statement-applicative-profunctor-exponential}}

If $G^{\bullet}$ is an applicative profunctor and $H^{\bullet}$
is \emph{any functor} then the profunctor $P^{A}\triangleq H^{A}\rightarrow G^{A}$
is applicative.

\subparagraph{Proof}

We follow the proof of Statement~\ref{subsec:Statement-applicative-contrafunctor-exponential}.
Implement the \lstinline!zip! and \lstinline!wu! methods for $P$:
\begin{align*}
 & \text{zip}_{P}:(H^{A}\rightarrow G^{A})\times(H^{B}\rightarrow G^{B})\rightarrow H^{A\times B}\rightarrow G^{A\times B}\quad,\\
 & \text{zip}_{P}\big(p^{:H^{A}\rightarrow G^{A}}\times q^{:H^{B}\rightarrow G^{B}}\big)\triangleq h^{:H^{A\times B}}\rightarrow\text{zip}_{G}\big(p(h\triangleright\pi_{1}^{\uparrow H})\times q(h\triangleright\pi_{2}^{\uparrow H})\big)\quad,\\
 & \text{wu}_{P}\triangleq\_^{:H^{\bbnum 1}}\rightarrow\text{wu}_{G}\quad.
\end{align*}
We will also use the definition of $\text{zip}_{P}$ in a point-free
form, which omits the argument $h$:
\[
\text{zip}_{P}(p\times q)\triangleq\Delta\bef(\pi_{1}^{\uparrow H}\boxtimes\pi_{2}^{\uparrow H})\bef(p\boxtimes q)\bef\text{zip}_{G}\quad.
\]

The code for lifting to $P$ is standard:
\[
(p^{:H^{A}\rightarrow G^{A}})\triangleright f^{\downarrow P}g^{\uparrow P}=f^{\uparrow H}\bef p\bef(f^{\downarrow G}g^{\uparrow G})\quad.
\]

To verify the left identity law of $P$, we use the left identity
law of $G$:
\begin{align*}
 & h^{:H^{\bbnum 1\times A}}\triangleright\text{zip}_{P}(\text{wu}_{P}\times p^{:H^{A}\rightarrow G^{A}})=\text{zip}_{G}\big(\gunderline{\text{wu}_{P}(h\triangleright\pi_{1}^{\uparrow H})}\times p(h\triangleright\pi_{2}^{\uparrow H})\big)\\
{\color{greenunder}\text{definition of }\text{wu}_{P}:}\quad & =\text{zip}_{G}\big(\text{wu}_{G}\times p(h\triangleright\pi_{2}^{\uparrow H})\big)\\
{\color{greenunder}\text{left identity law of }G:}\quad & =p(h\triangleright\pi_{2}^{\uparrow H})\triangleright\text{ilu}^{\uparrow G}\pi_{2}^{\downarrow G}=h\triangleright\gunderline{\pi_{2}^{\uparrow H}\bef p\bef\text{ilu}^{\uparrow G}\pi_{2}^{\downarrow G}}=h\triangleright(p\bef\text{ilu}^{\uparrow P}\pi_{2}^{\downarrow P})\quad.
\end{align*}

To verify the right identity law of $P$:
\begin{align*}
 & h^{:H^{A\times\bbnum 1}}\triangleright\text{zip}_{P}(p^{:H^{A}\rightarrow G^{A}}\times\text{wu}_{P})=\text{zip}_{G}\big(p(h\triangleright\pi_{1}^{\uparrow H})\times\text{wu}_{P}(h\triangleright\pi_{2}^{\uparrow H})\big)=\text{zip}_{G}\big(p(h\triangleright\pi_{1}^{\uparrow H})\times\text{wu}_{G}\big)\\
 & =p(h\triangleright\pi_{1}^{\uparrow H})\triangleright\text{iru}^{\uparrow G}\pi_{1}^{\downarrow G}=h\triangleright\gunderline{\pi_{1}^{\uparrow H}\bef p\bef\text{iru}^{\uparrow G}\pi_{1}^{\downarrow G}}=h\triangleright(p\bef\text{iru}^{\uparrow P}\pi_{1}^{\downarrow P})\quad.
\end{align*}

To verify the associativity law, we use the definition of $^{\uparrow P}$:
\begin{align*}
{\color{greenunder}\text{left-hand side}:}\quad & h^{:H^{A\times B\times C}}\triangleright\big(\text{zip}_{P}(p\times\text{zip}_{P}(q\times r))\,\gunderline{\triangleright\,\varepsilon_{1,23}^{\uparrow P}\tilde{\varepsilon}_{1,23}^{\downarrow P}}\big)\\
 & \quad=h^{:H^{A\times B\times C}}\triangleright\tilde{\varepsilon}_{1,23}^{\uparrow H}\triangleright\text{zip}_{P}(p\times\text{zip}_{P}(q\times r))\triangleright\varepsilon_{1,23}^{\uparrow G}\tilde{\varepsilon}_{1,23}^{\downarrow G}\\
 & \quad=\text{zip}_{G}\big(p(h\triangleright\gunderline{\tilde{\varepsilon}_{1,23}^{\uparrow H}\bef\pi_{1}^{\uparrow H}})\times\text{zip}_{P}(q\times r)(h\triangleright\tilde{\varepsilon}_{1,23}^{\uparrow H}\bef\pi_{2}^{\uparrow H})\big)\triangleright\varepsilon_{1,23}^{\uparrow G}\tilde{\varepsilon}_{1,23}^{\downarrow G}\\
 & \quad=\text{zip}_{G}\big(p(h\triangleright\pi_{1}^{\uparrow H})\times\text{zip}_{G}\big(q(h\triangleright\pi_{2}^{\uparrow H})\times r(h\triangleright\pi_{3}^{\uparrow H})\big)\big)\triangleright\varepsilon_{1,23}^{\uparrow G}\tilde{\varepsilon}_{1,23}^{\downarrow G}\quad,\\
{\color{greenunder}\text{right-hand side}:}\quad & h^{:H^{A\times B\times C}}\triangleright\tilde{\varepsilon}_{12,3}^{\uparrow H}\triangleright\text{zip}_{P}(\text{zip}_{P}(p\times q)\times r)\triangleright\varepsilon_{12,3}^{\uparrow G}\tilde{\varepsilon}_{12,3}^{\downarrow G}\\
 & \quad=\text{zip}_{G}\big(\text{zip}_{P}(p\times q)(h\triangleright\tilde{\varepsilon}_{12,3}^{\uparrow H}\bef\pi_{1}^{\uparrow H})\times r(h\triangleright\tilde{\varepsilon}_{12,3}^{\uparrow H}\bef\pi_{2}^{\uparrow H})\big)\triangleright\varepsilon_{12,3}^{\uparrow G}\tilde{\varepsilon}_{12,3}^{\downarrow G}\\
 & \quad=\text{zip}_{G}\big(\text{zip}_{G}\big(p(h\triangleright\pi_{1}^{\uparrow H})\times q(h\triangleright\pi_{2}^{\uparrow H})\big)\times r(h\triangleright\pi_{3}^{\uparrow H})\big)\triangleright\varepsilon_{12,3}^{\uparrow G}\tilde{\varepsilon}_{12,3}^{\downarrow G}\quad.
\end{align*}
The two sides are now equal due to the associativity law of $\text{zip}_{G}$.

It remains to verify the commutativity law, assuming that $\text{zip}_{G}$
satisfies that law:
\begin{align*}
 & \text{zip}_{P}(q\times p)=\Delta\bef(\pi_{1}^{\uparrow H}\boxtimes\pi_{2}^{\uparrow H})\bef(q\boxtimes p)\bef\text{zip}_{G}\quad,\\
 & \text{zip}_{P}(p\times q)\triangleright\text{swap}^{\uparrow P}\text{swap}^{\downarrow P}=\text{swap}^{\uparrow H}\bef\Delta\bef(\pi_{1}^{\uparrow H}\boxtimes\pi_{2}^{\uparrow H})\bef(p\boxtimes q)\bef\gunderline{\text{zip}_{G}\bef\text{swap}^{\uparrow G}\text{swap}^{\downarrow G}}\\
 & \quad=\Delta\bef\gunderline{(\text{swap}^{\uparrow H}\boxtimes\text{swap}^{\uparrow H})\bef(\pi_{1}^{\uparrow H}\boxtimes\pi_{2}^{\uparrow H})}\bef\gunderline{(p\boxtimes q)\bef\text{swap}}\bef\text{zip}_{G}\\
 & \quad=\gunderline{\Delta\bef(\pi_{2}^{\uparrow H}\boxtimes\pi_{1}^{\uparrow H})\bef\text{swap}}\bef(q\boxtimes p)\bef\text{zip}_{G}=\Delta\bef(\pi_{1}^{\uparrow H}\boxtimes\pi_{2}^{\uparrow H})\bef(q\boxtimes p)\bef\text{zip}_{G}\quad.
\end{align*}
$\square$

Note that the functor $H$ must be \emph{covariant} in the last construction
($P^{A}\triangleq H^{A}\rightarrow G^{A}$). Otherwise, the profunctor
$P$ will not be applicative even the simple cases such as $P^{A}\triangleq H^{A}\rightarrow A$
where $H$ is an arbitrary profunctor. (We omit the proof of that
statement.)

\section{Discussion and further developments}

\subsection{Equivalence of typeclass methods with laws}

In this and the previous chapters, we have seen that certain typeclass
methods are equivalent. The equivalence of \lstinline!map2!, \lstinline!zip!,
and \lstinline!ap! was proved in Section~\ref{subsec:Equivalence-of-map2-zip-ap}.
For filterable functors, we have shown that the methods \lstinline!filter!,
\lstinline!deflate!, and \lstinline!liftOpt! are equivalent if some
naturality laws hold (Sections~\ref{subsec:Equivalence-of-filter-and-deflate}
and~\ref{subsec:Motivation-and-laws-for-liftopt-and-equivalence}).
For monads, we proved that \lstinline!flatMap! is equivalent to \lstinline!flatten!
(Statement~\ref{subsec:Statement-flatten-equivalent-to-flatMap}).
Another case of equivalence between certain function types was proved
in Section~\ref{subsec:Yoneda-identities}. 

Perhaps the simplest example of this sort of equivalence is that between
the \lstinline!pure! method and the \textsf{``}wrapped unit\textsf{''} (\lstinline!wu!),
as we saw in Section~\ref{subsec:Pointed-functors-motivation-equivalence}.
A function $\text{pu}_{F}:A\rightarrow F^{A}$ satisfying a naturality
law is equivalent to a value $\text{wu}_{F}:F^{\bbnum 1}$. 

After seeing those detailed proofs, we can now clarify the meaning
of \textsf{``}equivalence under laws\textsf{''}. The goal of this subsection is to
find a rigorous formulation of that equivalence.

In each case seen so far, we have two functions with two different
type signatures (usually with type parameters), and we assume that
certain naturality laws hold. It is important to keep in mind that
the naturality laws are equations whose form \emph{automatically}
follows from type signatures. If a function has more than one type
parameter, there is one naturality law per type parameter (the full
details are in Section~\ref{subsec:Naturality-laws-and-natural-transformations}
and Appendix~\ref{app:Proofs-of-naturality-parametricity}). For
example, the type signature \lstinline!pure[A]: A => F[A]! (in the
type notation, $\text{pu}_{F}:\forall A.\,A\rightarrow F^{A}$) gives
rise to the naturality law $f\bef\text{pu}_{F}=\text{pu}_{F}\bef f^{\uparrow F}$.
In Scala code, this law is written as \lstinline!pure(f(x)) == pure(x).map(f)!.

What we have proved in Section~\ref{subsec:Pointed-functors-motivation-equivalence}
is a one-to-one correspondence between two \emph{sets}: the set of
all values \lstinline!wu! of type \lstinline!F[Unit]!, and the set
of all functions with type \lstinline!pure[A]: A => F[A]! satisfying
the naturality law \lstinline!pure(f(x)) == pure(x).map(f)!. The
proof defines explicit mappings between \lstinline!wu! and \lstinline!pure!
in both directions and verifies that their compositions are identity
maps. We also made sure to exclude all functions of type \lstinline!A => F[A]!
that fail to satisfy the naturality law. For instance, we proved that
if \lstinline!pure! is defined via \lstinline!wu! then the naturality
law will hold for \lstinline!pure!.

So, if \lstinline!wu: F[Unit]! is defined in some way for a functor
\lstinline!F!, we will always have a corresponding definition of
\lstinline!pure! that satisfies the naturality law. Conversely, for
any given lawful definition of \lstinline!pure! we will have a corresponding
value \lstinline!wu!.

The same pattern was used in the proofs of equivalence in other chapters.
Let us formulate this pattern in a more abstract way. We would like
to prove that some function $p$ (of type $P^{A,B,...}$) is equivalent
to some $q$ (of type $Q^{A,B,...}$) when suitable naturality laws
hold. These functions may have any number of type parameters ($A$,
$B$, etc.). More explicitly, the types of $p$ and $q$ are:
\[
p:\forall(A,B,...).\,P^{A,B,...}\quad,\quad\quad q:\forall(A,B,...).\,Q^{A,B,...}\quad.
\]
We view the type $\forall(A,B,...).\,P^{A,B,...}$ as a set of all
functions $p$ (implementable in Scala) with this code:
\begin{lstlisting}
def p[A, B, ...]: P[A, B, ...] = { ... }
\end{lstlisting}
and similarly for $q$. To establish the equivalence, we first write
some code that expresses $p$ through $q$ and back. This requires
two functions, $f:P^{A,B,...}\rightarrow Q^{A,B,...}$ and $g:Q^{A,B,...}\rightarrow P^{A,B,...}$.
These functions are mappings (in both directions) between the set
of all functions $p$ and the set of all functions $q$ of the corresponding
types. Then we prove that the mappings are one-to-one: for any $p$
and $q$ we have $g(f(p))=p$ and $f(g(q))=q$. Finally, we check
that the required naturality laws still hold after applying the mappings
$f$ and $g$. In other words, for any $p$ satisfying its naturality
law, we prove that the function $f(p)$ of type $Q^{A,B,...}$ will
satisfy \emph{its} naturality law; and similarly for $q$ and $g(q)$.

Without the naturality laws, the types $\forall(A,B,...).\,P^{A,B,...}$
and $\forall(A,B,...).\,Q^{A,B,...}$ are usually \emph{not} equivalent:
there is no one-to-one correspondence between functions $p$ and functions
$q$ of those types. Imposing a naturality law means that we exclude
functions $p$ of type $\forall(A,B,...).\,P^{A,B,...}$ that do not
satisfy the naturality law. This gives us a smaller set of all lawful
functions $p$ (and similarly for $q$). The equivalence holds only
for those smaller sets of lawful functions.

To see this in a simple example, consider the equivalence of $p$=\lstinline!pure!
and $q$=\lstinline!wu! for the identity functor ($L^{A}\triangleq A$).
The method \lstinline!pure! has type $\forall A.\,A\rightarrow A$,
while \lstinline!wu! has the \lstinline!Unit! type ($L^{\bbnum 1}=\bbnum 1$).
There is only one value \lstinline!wu!, but there are many possible
functions $p$ of type $\forall A.\,A\rightarrow A$. For instance,
we could define a function $p$ like this:

\begin{wrapfigure}{l}{0.295\columnwidth}%
\vspace{-1\baselineskip}
\begin{lstlisting}
def p[A]: A => A = {
  case a: Int   => (a + 123)
            .asInstanceOf[A]
  case a        => a
}
\end{lstlisting}

\vspace{-1\baselineskip}
\end{wrapfigure}%

\noindent This function is defined for all types $A$, so it fits
the type $\forall A.\,A\rightarrow A$. The set of all functions of
type $\forall A.\,A\rightarrow A$ contains a large number of similarly
defined functions. For example, we may replace \lstinline!123! by
another number or select another type instead of \lstinline!Int!.
However, all those functions fail to satisfy the naturality law, which
has the form $f^{:A\rightarrow B}\bef p^{B}=p^{A}\bef f$. The only
function $p$ that satisfies this naturality law is the identity function,
$p^{A}=\text{id}^{A}\triangleq a^{:A}\rightarrow a$ (see Exercise~\ref{subsec:Exercise-hof-composition-1}).
We find that imposing the naturality law removes all functions $p$
except one ($p=\text{id}$) from the set of functions $p:\text{\ensuremath{\forall A.\,A\rightarrow A}}$.
There remains a set containing a single element ($\text{id}$), which
is in a one-to-one correspondence with the set of values of type $\bbnum 1$.

So, the rigorous meaning of an \textsf{``}equivalence between functions $p$
and $q$ assuming some laws\textsf{''} is a one-to-one correspondence between
the set of all functions $p:\forall(A,B,...).\,P^{A,B,...}$ that
satisfy the given laws of $p$, and the set of all functions $q:\forall(A,B,...).\,Q^{A,B,...}$
that satisfy the given laws of $q$.

Note that the Scala compiler is unable to check automatically that
the naturality laws hold for a function $p$ with type parameters.
Indeed, those laws are equations for $p$ that involve arbitrary functions
$f^{:A\rightarrow B}$ with arbitrary types $A$ and $B$. Such equations
cannot be enforced by the type signature of $p$. Neither can we prove
naturality laws by running tests: each test will have to use specific
types as the type parameters, while the purpose of naturality laws
is to verify that the function works in the same way for \emph{all}
types. Naturality laws can be verified only through a proof that uses
symbolic reasoning. (The proof can be omitted if the code is fully
parametric, as shown in Appendix~\ref{app:Proofs-of-naturality-parametricity}.)

The Scala compiler also cannot verify the laws specific to a given
typeclass (e.g., the identity and composition laws of filterable functors,
or the identity and associativity laws of monads and applicative functors).
These laws must be also verified by symbolic reasoning. It is often
easier to verify laws if the type signature of the method has fewer
type parameters. For instance, the \lstinline!zip! method has two
type parameters while \lstinline!map2! has three; \lstinline!flatten!
has one type parameter while \lstinline!flatMap! has two. For this
reason, we have systematically derived the laws of all the equivalent
typeclass methods. In many cases, we found a formulation of the laws
that was either conceptually simpler or more straightforward to verify. 

\subsection{Relationship between monads and applicative functors}

Any lawful monad gives its type constructor at least one applicative
functor definition: as we showed in Section~\ref{subsec:Commutative-applicative-functors},
we may define the \lstinline!map2! method via \lstinline!map! and
\lstinline!flatMap! in two ways (which will give the same code if
the monad is commutative). The \lstinline!map2! methods defined in
this way will have the right type signature and will satisfy the laws
of applicative functors. This is due to the fact that the laws of
\lstinline!map2! are derived (as shown in Section~\ref{subsec:Motivation-for-the-laws-of-map2})
from the monad laws precisely by considering the \lstinline!map2!
method defined via the monad\textsf{'}s \lstinline!flatMap!.

However, in many cases we need to define the \lstinline!map2! method
in a different way because expressing \lstinline!map2! via \lstinline!flatMap!
does not give us the required functionality. We have seen several
examples of this. For instance, the standard behavior of \lstinline!map2!
for sequences, trees, and \lstinline!Either! is not compatible with
the standard \lstinline!flatMap! methods of those functors.

So, it is rarely useful to define an \lstinline!Applicative! typeclass
instance automatically for all monads. In most cases, we need to define
the applicative instance separately from the monad instance. (Automatic
derivation of \lstinline!Applicative! instances is made difficult
also by the fact that many type constructors will admit more than
one lawful implementation of \lstinline!map2!.)

In addition, some applicative functors are not monads. We have shown
in Section~\ref{subsec:Constructions-of-applicative-functors} that
a lawful \lstinline!Applicative! instance exists for all polynomial
functors with monoidal fixed types. (Accordingly, all our examples
of non-applicative functors involve non-polynomial functors.) This
does not hold for monads; not all polynomial functors are monadic.\footnote{It is unknown how to characterize or enumerate all polynomial functors
that are monads (see Problems~\ref{par:Problem-monads}\textendash \ref{par:Problem-monads-1}).} Example~\ref{subsec:Example-applicative-not-monad} shows a simple
applicative functor ($L^{A}\triangleq\bbnum 1+A\times A$) that \emph{cannot}
have a lawful monad implementation (Exercise~\ref{subsec:Exercise-1-monads-7-not-a-monad}).

We have also seen that the \lstinline!zip! and \lstinline!wu! operations
exist for some type constructors that are not covariant. We conclude
that the \lstinline!Applicative! typeclass is larger than the \lstinline!Monad!
typeclass.

\subsection{Applicative morphisms}

One of the applicative constructions (Statement~\ref{subsec:Statement-co-product-with-co-pointed-applicative})
needs a compatibility law~(\ref{eq:compatibility-law-of-extract-and-zip})
between the methods $\text{zip}_{H}$ and $\text{ex}_{H}$. This law
is understood better if we rewrite the type signature $\text{ex}_{H}:H^{A}\rightarrow A$
as $\text{ex}_{H}:H^{A}\rightarrow\text{Id}^{A}$, where $\text{Id}^{A}\triangleq A$
is the identity functor (which is also applicative). Viewed in this
way, the \lstinline!extract! method is an example of a mapping between
two applicative functors ($H$ and $\text{Id}$). So, the compatibility
law requires that the operation $\text{zip}_{H}$ be mapped to the
tupling operation, which is the same as the \lstinline!zip! operation
of the identity functor: $\text{zip}_{\text{Id}}(a\times b)=a\times b$.

This suggests considering a more general mapping $\phi:H^{A}\rightarrow K^{A}$
where $H$ and $K$ are two applicative functors, and formulating
a \textbf{composition law}\index{composition law!of applicative morphisms}
of $\phi$ by analogy with Eq.~(\ref{eq:compatibility-law-of-extract-and-zip}):
\[
\phi\big(\text{zip}_{H}(p\times q)\big)=\text{zip}_{K}(\phi(p)\times\phi(q))\quad,\quad\text{or equivalently}:\quad\text{zip}_{H}\bef\phi=(\phi\boxtimes\phi)\bef\text{zip}_{K}\quad.
\]
This law reproduces Eq.~(\ref{eq:compatibility-law-of-extract-and-zip})
when $\phi=\text{ex}_{H}$ and $K=\text{Id}$ because we will then
have $\text{zip}_{K}=\text{id}$. 

It seems reasonable to require also that $\phi$ should map the method
$\text{pu}_{H}$ to $\text{pu}_{K}$. This gives the \textbf{identity
law}: $\text{pu}_{H}\bef\phi=\text{pu}_{K}$.\index{identity laws!of applicative morphisms}
Functions $\phi$ satisfying these two laws are called \textbf{applicative
morphisms}\index{applicative morphism} between two applicative functors
$H$ and $K$. The two laws make applicative morphisms fully analogous
to monad morphisms (Section~\ref{subsec:Monads-in-category-theory-monad-morphisms}).

\subsection{Deriving the laws of \texttt{ap} using category theory}

Statement~\ref{subsec:Statement-fmap2-equivalence-to-ap} shows that
the additional capabilities of \lstinline!map2! compared with \lstinline!map!
can be expressed via the function \lstinline!ap!:
\[
\text{map}_{2}\,(p^{:L^{A}}\times q^{:L^{B}})(f^{:A\times B\rightarrow C})=\text{ap}\,(p\triangleright(a^{:A}\rightarrow b^{:B}\rightarrow f(a\times b))^{\uparrow L})(q)\quad.
\]
How can we derive the laws of \lstinline!ap! that correspond to the
laws of \lstinline!map2!? If we substitute the equation above into
the laws of \lstinline!map2!, we will obtain a relationship involving
arbitrary values $p^{:L^{A}}$, $q^{:L^{B}}$, and $f^{:A\times B\rightarrow C}$.
However, the type signature of \lstinline!ap! is $L^{A\rightarrow B}\rightarrow L^{A}\rightarrow L^{B}$,
and so we expect the laws of \lstinline!ap! to have arbitrary arguments
of types $L^{A\rightarrow B}$ and $L^{B}$. It is not clear how to
choose the types in the laws of \lstinline!map2! in order to derive
a complete set of laws for \lstinline!ap!.

To resolve this issue, we turn to category theory for guidance. The
type signature of \lstinline!ap! looks like a \textsf{``}lifting\textsf{''} from
values of type $L^{A\rightarrow B}$ to functions of type $L^{A}\rightarrow L^{B}$.
Category theory suggests to describe this lifting as part of a suitable
categorical functor.\index{functor!in category theory}\index{category theory!functor}
A categorical functor can exist only between two categories. So, we
need to show that values of type $L^{A\rightarrow B}$ (\textsf{``}wrapped
functions\textsf{''}) can play the role of morphisms in a suitably defined
category. To define that category, we need to produce objects, morphisms,
the identity morphism, and the composition operation, and prove their
laws. Finally, we will need to prove that \lstinline!ap! satisfies
the identity and composition laws appropriate for a (categorical)
functor.

This section will follow these considerations in order to derive and
verify the laws of \lstinline!ap!.

The first step is to define two categories between which we will establish
a categorical functor. We would like the morphisms of the first category
to be values of type $L^{A\rightarrow B}$, and the morphisms of the
second category to be functions of type $L^{A}\rightarrow L^{B}$.
The second category was called \textsf{``}$L$-lifted\textsf{''} in Section~\ref{subsec:Motivation-for-using-category-theory}.
Functions of type $L^{A}\rightarrow L^{B}$ already satisfy the required
properties of morphisms: the identity morphism is the function $\text{id}^{:L^{A}\rightarrow L^{A}}$,
and the ordinary function composition is associative and respects
the identity. The objects of the first category may be chosen as ordinary
types ($A$, $B$, etc.). To establish that values of type $L^{A\rightarrow B}$
also satisfy the properties of morphisms, we need to define a composition
operation for those values and prove its properties.

It turns out that there are two ways of defining the composition operation,
as shown in Statement~\ref{subsec:Statement-ap-category-laws} (a)
and (b) below. The difference between the two definitions disappears
if $L$ is a \emph{commutative} applicative functor. We will use the
definition (b); the choice is dictated by compatibility with the \lstinline!ap!
operation, as we will demonstrate in Statement~\ref{subsec:Statement-ap-functor-laws}.

\subsubsection{Statement \label{subsec:Statement-ap-category-laws}\ref{subsec:Statement-ap-category-laws}}

Given a functor $L$ with lawful \lstinline!zip! and \lstinline!pure!
methods, we define an \index{applicative composition (odot)@applicative composition ($\odot$)}\textbf{$L$-applicative
composition} operation (denoted by $\odot$). For any $p:L^{A\rightarrow B}$
and $q:L^{B\rightarrow C}$, the value $p\odot q$ will be of type
$L^{A\rightarrow C}$. The operation $\odot$ may be defined in two
different ways:
\begin{align*}
{\color{greenunder}\text{definition (a) of }\odot:}\quad & p^{:L^{A\rightarrow B}}\odot q^{:L^{B\rightarrow C}}\triangleq(p\times q)\triangleright\text{zip}\triangleright(f^{:A\rightarrow B}\times g^{:B\rightarrow C}\rightarrow f\bef g)^{\uparrow L}\quad,\\
{\color{greenunder}\text{definition (b) of }\odot:}\quad & p^{:L^{A\rightarrow B}}\odot q^{:L^{B\rightarrow C}}\triangleq(q\times p)\triangleright\text{zip}\triangleright(g^{:B\rightarrow C}\times f^{:A\rightarrow B}\rightarrow f\bef g)^{\uparrow L}\quad.
\end{align*}
Moreover, the special \textsf{``}wrapped identity\textsf{''} value, \lstinline!wid!,
of type $L^{A\rightarrow A}$, is defined by $\text{wid}_{L}^{A}\triangleq\text{pu}_{L}(\text{id}^{A})$.
The following properties then hold for each of the two definitions:\index{identity laws!of applicative composition}\index{associativity law!of applicative composition}
\begin{align*}
{\color{greenunder}\text{left identity law of }\odot:}\quad & \text{wid}_{L}^{A}\odot p^{:L^{A\rightarrow B}}=p\quad,\\
{\color{greenunder}\text{right identity law of }\odot:}\quad & p^{:L^{A\rightarrow B}}\odot\text{wid}_{L}^{B}=p\quad,\\
{\color{greenunder}\text{associativity law of }\odot:}\quad & (p^{:L^{A\rightarrow B}}\odot q^{:L^{B\rightarrow C}})\odot r^{:L^{C\rightarrow D}}=p\odot(q\odot r)\quad.
\end{align*}


\subparagraph{Proof}

We will prove the laws separately for the definitions (a) and (b).

\textbf{(a)} To verify the left identity law, substitute the definition
(a) of $\odot$:
\begin{align*}
{\color{greenunder}\text{expect to equal }p:}\quad & \text{wid}_{L}\odot p=\text{pu}_{L}(\text{id})\odot p=(\text{pu}_{L}(\text{id})\times p)\triangleright\text{zip}\bef(f\times g\rightarrow f\bef g)^{\uparrow L}\\
{\color{greenunder}\text{left identity law of }\text{zip}:}\quad & =p\triangleright(s^{:A\rightarrow B}\rightarrow\text{id}\times s)^{\uparrow L}\bef(f\times g\rightarrow f\bef g)^{\uparrow L}\\
{\color{greenunder}\text{function composition under }^{\uparrow L}:}\quad & =p\triangleright(\gunderline{s\rightarrow\text{id}\bef s})^{\uparrow L}=p\triangleright\text{id}^{\uparrow L}=p\quad.
\end{align*}

To verify the right identity law:
\begin{align*}
{\color{greenunder}\text{expect to equal }p:}\quad & p\odot\text{wid}_{L}=p\odot\text{pu}_{L}(\text{id})=(p\times\text{pu}_{L}(\text{id}))\triangleright\text{zip}\bef(f\times g\rightarrow f\bef g)^{\uparrow L}\\
{\color{greenunder}\text{right identity law of }\text{zip}:}\quad & =p\triangleright(s^{:A\rightarrow B}\rightarrow s\times\text{id})^{\uparrow L}\bef(f\times g\rightarrow f\bef g)^{\uparrow L}\\
{\color{greenunder}\text{function composition under }^{\uparrow L}:}\quad & =p\triangleright(\gunderline{s\rightarrow s\bef\text{id}})^{\uparrow L}=p\triangleright\text{id}^{\uparrow L}=p\quad.
\end{align*}

To verify the composition law, begin rewriting its left-hand side:
\begin{align*}
{\color{greenunder}\text{left-hand side}:}\quad & (p\odot q)\odot r=\big((p\times q)\triangleright\text{zip}\bef(f\times g\rightarrow f\bef g)^{\uparrow L}\big)\odot r\\
 & =\big(\big(\text{zip}\left(p\times q\right)\triangleright\gunderline{(f\times g\rightarrow f\bef g)^{\uparrow L}}\big)\times r\big)\triangleright\gunderline{\text{zip}}\bef(k\times h\rightarrow k\bef h)^{\uparrow L}\\
{\color{greenunder}\text{naturality law of }\text{zip}:}\quad & =(\text{zip}\left(p\times q\right)\times r)\triangleright\text{zip}\bef\gunderline{((f\times g)\times h\rightarrow(f\bef g)\times h)^{\uparrow L}\bef(k\times h\rightarrow k\bef h)^{\uparrow L}}\\
{\color{greenunder}\text{composition under }^{\uparrow L}:}\quad & =\text{zip}\left(\text{zip}\left(p\times q\right)\times r\right)\triangleright((f\times g)\times h\rightarrow f\bef g\bef h)^{\uparrow L}\quad.
\end{align*}
The right-hand side is rewritten similarly:
\begin{align*}
{\color{greenunder}\text{right-hand side}:}\quad & p\odot(q\odot r)=p\odot\big(\text{zip}\left(q\times r\right)\triangleright(f\times g\rightarrow f\bef g)^{\uparrow L}\big)\\
 & =\big(p\times\big(\text{zip}\left(q\times r\right)\triangleright\gunderline{(g\times h\rightarrow g\bef h)^{\uparrow L}}\big)\big)\triangleright\gunderline{\text{zip}}\bef(f\times k\rightarrow f\bef k)^{\uparrow L}\\
{\color{greenunder}\text{naturality law of }\text{zip}:}\quad & =(p\times\text{zip}\left(q\times r\right))\triangleright\text{zip}\bef\gunderline{(f\times(g\times h)\rightarrow f\times(g\bef h))^{\uparrow L}\bef(f\times k\rightarrow f\bef k)^{\uparrow L}}\\
{\color{greenunder}\text{composition under }^{\uparrow L}:}\quad & =\text{zip}\left(p\times\text{zip}\left(q\times r\right)\right)\triangleright(f\times(g\times h)\rightarrow f\bef g\bef h)^{\uparrow L}\quad.
\end{align*}
It is clear that we need to use the associativity law of \lstinline!zip!.
To be able to use that law, we express the functions $(f\times g)\times h\rightarrow f\bef g\bef h$
and $f\times(g\times h)\rightarrow f\bef g\bef h$ through the conversion
functions $\varepsilon_{1,23}$ and $\varepsilon_{12,3}$ (defined
in Section~\ref{subsec:Deriving-the-laws-of-zip}):
\begin{align*}
 & \big((f\times g)\times h\rightarrow f\bef g\bef h\big)=\varepsilon_{12,3}\bef(f\times g\times h\rightarrow f\bef g\bef h)\quad,\\
 & \big(f\times(g\times h)\rightarrow f\bef g\bef h\big)=\varepsilon_{1,23}\bef(f\times g\times h\rightarrow f\bef g\bef h)\quad.
\end{align*}
Using these equations, we show that the two sides of the associativity
law of $\odot$ are equal:
\begin{align*}
{\color{greenunder}\text{left-hand side}:}\quad & \gunderline{\text{zip}\left(\text{zip}\left(p\times q\right)\times r\right)\triangleright\varepsilon_{12,3}}\bef(f\times g\times h\rightarrow f\bef g\bef h)\\
{\color{greenunder}\text{right-hand side}:}\quad & =\gunderline{\text{zip}\left(p\times\text{zip}\left(q\times r\right)\right)\triangleright\varepsilon_{1,23}}\bef(f\times g\times h\rightarrow f\bef g\bef h)\quad.
\end{align*}
The underlined expressions are equal due to the associativity law~(\ref{eq:zip-associativity-law-with-epsilons})
of \lstinline!zip!.

\textbf{(b)} To verify the left identity law, substitute the definition
(b) of $\odot$:
\begin{align*}
{\color{greenunder}\text{expect to equal }p:}\quad & \text{wid}_{L}\odot p=\text{pu}_{L}(\text{id})\odot p=(p\times\text{pu}_{L}(\text{id}))\triangleright\text{zip}\bef(g\times f\rightarrow f\bef g)^{\uparrow L}\\
{\color{greenunder}\text{right identity law of }\text{zip}:}\quad & =p\triangleright(s^{:A\rightarrow B}\rightarrow s\times\text{id})^{\uparrow L}\bef(g\times f\rightarrow f\bef g)^{\uparrow L}\\
{\color{greenunder}\text{function composition under }^{\uparrow L}:}\quad & =p\triangleright(\gunderline{s\rightarrow\text{id}\bef s})^{\uparrow L}=p\triangleright\text{id}^{\uparrow L}=p\quad.
\end{align*}

To verify the right identity law:
\begin{align*}
{\color{greenunder}\text{expect to equal }p:}\quad & p\odot\text{wid}_{L}=p\odot\text{pu}_{L}(\text{id})=(\text{pu}_{L}(\text{id})\times p)\triangleright\text{zip}\bef(g\times f\rightarrow f\bef g)^{\uparrow L}\\
{\color{greenunder}\text{left identity law of }\text{zip}:}\quad & =p\triangleright(s^{:A\rightarrow B}\rightarrow\text{id}\times s)^{\uparrow L}\bef(g\times f\rightarrow f\bef g)^{\uparrow L}\\
{\color{greenunder}\text{function composition under }^{\uparrow L}:}\quad & =p\triangleright(\gunderline{s\rightarrow s\bef\text{id}})^{\uparrow L}=p\triangleright\text{id}^{\uparrow L}=p\quad.
\end{align*}

To verify the composition law, begin rewriting its left-hand side:
\begin{align*}
{\color{greenunder}\text{left-hand side}:}\quad & (p\odot q)\odot r=\big((q\times p)\triangleright\text{zip}\bef(g\times f\rightarrow f\bef g)^{\uparrow L}\big)\odot r\\
 & =\big(r\times\big(\text{zip}\left(q\times p\right)\triangleright\gunderline{(g\times f\rightarrow f\bef g)^{\uparrow L}}\big)\big)\triangleright\gunderline{\text{zip}}\bef(h\times k\rightarrow k\bef h)^{\uparrow L}\\
{\color{greenunder}\text{naturality law of }\text{zip}:}\quad & =(r\times\text{zip}\left(q\times p\right))\triangleright\text{zip}\bef\gunderline{(h\times(g\times f)\rightarrow h\times(f\bef g))^{\uparrow L}\bef(h\times k\rightarrow k\bef h)^{\uparrow L}}\\
{\color{greenunder}\text{composition under }^{\uparrow L}:}\quad & =\text{zip}\left(r\times\text{zip}\left(q\times p\right)\right)\triangleright(h\times(g\times f)\rightarrow f\bef g\bef h)^{\uparrow L}\quad.
\end{align*}
The right-hand side is rewritten similarly:
\begin{align*}
{\color{greenunder}\text{right-hand side}:}\quad & p\odot(q\odot r)=p\odot\big(\text{zip}\left(q\times r\right)\triangleright(g\times f\rightarrow f\bef g)^{\uparrow L}\big)\\
 & =\big(\big(\text{zip}\left(q\times r\right)\triangleright\gunderline{(h\times g\rightarrow g\bef h)^{\uparrow L}}\big)\times p\big)\triangleright\gunderline{\text{zip}}\bef(k\times f\rightarrow f\bef k)^{\uparrow L}\\
{\color{greenunder}\text{naturality law of }\text{zip}:}\quad & =(\text{zip}\left(q\times r\right)\times p)\triangleright\text{zip}\bef\gunderline{((h\times g)\times f\rightarrow(g\bef h)\times f)^{\uparrow L}\bef(k\times f\rightarrow f\bef k)^{\uparrow L}}\\
{\color{greenunder}\text{composition under }^{\uparrow L}:}\quad & =\text{zip}\left(\text{zip}\left(q\times r\right)\times p\right)\triangleright((h\times g)\times f\rightarrow f\bef g\bef h)^{\uparrow L}\quad.
\end{align*}
To apply the associativity law of \lstinline!zip!, we use the conversion
functions $\varepsilon_{1,23}$ and $\varepsilon_{12,3}$ and write:
\begin{align*}
 & \big(h\times(g\times f)\rightarrow f\bef g\bef h\big)=\varepsilon_{1,23}\bef(h\times g\times f\rightarrow f\bef g\bef h)\quad,\\
 & \big((h\times g)\times f\rightarrow f\bef g\bef h\big)=\varepsilon_{12,3}\bef(h\times g\times f\rightarrow f\bef g\bef h)\quad.
\end{align*}
Using these equations, we show that the two sides of the associativity
law of $\odot$ are equal:
\begin{align*}
{\color{greenunder}\text{left-hand side}:}\quad & \gunderline{\text{zip}\left(r\times\text{zip}\left(q\times p\right)\right)\triangleright\varepsilon_{1,23}}\bef(h\times g\times f\rightarrow f\bef g\bef h)^{\uparrow L}\\
{\color{greenunder}\text{right-hand side}:}\quad & =\gunderline{\text{zip}\left(\text{zip}\left(q\times r\right)\times p\right)\triangleright\varepsilon_{12,3}}\bef(h\times g\times f\rightarrow f\bef g\bef h)\quad.
\end{align*}
The underlined expressions are equal due to the associativity law
of \lstinline!zip!. $\square$

The laws of the composition operation ($\odot$) show that values
of type $L^{A\rightarrow B}$ are indeed morphisms. Let us call \textsf{``}$L$-applicative\textsf{''}
the category with those morphisms. We may now write the laws of a
(categorical) functor between the $L$-applicative and the $L$-lifted
categories. Such a functor consists of two mappings: a mapping between
objects: each type $A$ is mapped to the type $L^{A}$; and a mapping
between morphisms: each $L$-applicative morphism $f:L^{A\rightarrow B}$
is mapped to the morphism $\text{ap}\left(f\right):L^{A}\rightarrow L^{B}$
in the $L$-lifted category. In other words, we expect the function
\lstinline!ap! to play the role of the morphism mapping of that functor.
This mapping must obey the laws of identity and composition.

The identity law says that the function \lstinline!ap! must map the
identity morphism of the $L$-applicative category into the identity
morphism of the $L$-lifted category:\index{identity laws!of ap@of \texttt{ap}}
\begin{align}
{\color{greenunder}\text{left identity law of }\text{ap}:}\quad & \text{ap}\,(\text{pu}_{L}(\text{id}^{:A\rightarrow A}))=\text{id}^{:L^{A}\rightarrow L^{A}}\quad.\label{eq:identity-law-of-ap}
\end{align}

The composition law says that the composition $p\odot q$ of any two
$L$-applicative morphisms $p^{:L^{A\rightarrow B}}$ and $q^{:L^{B\rightarrow C}}$
must be mapped by \lstinline!ap! into the composition $\text{ap}\left(p\right)\bef\text{ap}\left(q\right)$:
\begin{align}
{\color{greenunder}\text{composition law of }\text{ap}:}\quad & \text{ap}\big(p^{:L^{A\rightarrow B}}\odot q^{:L^{B\rightarrow C}}\big)=\text{ap}\left(p\right)\bef\text{ap}\left(q\right)\quad.\label{eq:composition-law-of-ap}
\end{align}
In this law, the operation $\odot$ needs to be defined as in Statement~\ref{subsec:Statement-ap-category-laws}(b).
Expressing that operation via \lstinline!ap!, we obtain the following
formulation of the composition law of\index{composition law!of ap@of \texttt{ap}}
\lstinline!ap!:
\[
\text{ap}\big(\text{ap}\big(q^{:L^{B\rightarrow C}}\triangleright(g^{:B\rightarrow C}\rightarrow f^{:A\rightarrow B}\rightarrow f\bef g)^{\uparrow L}\big)(p^{:L^{A\rightarrow B}})\big)=\text{ap}\left(p\right)\bef\text{ap}\left(q\right)\quad.
\]
Instead of proving this complicated law, we will prove the functor
laws of \lstinline!ap!:

\subsubsection{Statement \label{subsec:Statement-ap-functor-laws}\ref{subsec:Statement-ap-functor-laws}}

Given a functor $L$ with lawful \lstinline!zip! and \lstinline!pure!
methods, we define the \lstinline!ap! method as in Statement~\ref{subsec:Statement-zip-ap-equivalence}:
\[
\text{ap}\,(r^{:L^{A\rightarrow B}})(p^{:L^{A}})\triangleq\text{zip}\,(r\times p)\triangleright\text{eval}^{\uparrow L}=(r\times p)\triangleright\text{zip}\bef(f^{:A\rightarrow B}\times a^{:A}\rightarrow f(a))^{\uparrow L}\quad.
\]
The operation $\odot$ is defined by Statement~\ref{subsec:Statement-ap-category-laws}(b).
Then the \lstinline!ap! method satisfies Eqs.~(\ref{eq:identity-law-of-ap})\textendash (\ref{eq:composition-law-of-ap}). 

\subparagraph{Proof}

To verify the identity law~(\ref{eq:identity-law-of-ap}), apply
both sides to some $p^{:L^{A}}$:
\begin{align*}
{\color{greenunder}\text{expect to equal }p:}\quad & \text{ap}\,(\text{pu}_{L}(\text{id}))(p^{:L^{A}})=(\gunderline{\text{pu}_{L}(\text{id})}\times p)\triangleright\gunderline{\text{zip}}\bef\text{eval}^{\uparrow L}(f^{:A\rightarrow A}\times a^{:A}\rightarrow f(a))^{\uparrow L}\\
{\color{greenunder}\text{left identity law of }\text{zip}:}\quad & =p\triangleright\gunderline{(a^{:A}\rightarrow\text{id}\times a)^{\uparrow L}\bef(f^{:A\rightarrow A}\times a^{:A}\rightarrow f(a))^{\uparrow L}}\\
{\color{greenunder}\text{composition under }^{\uparrow L}:}\quad & =p\triangleright(a^{:A}\rightarrow\text{id}\,(a))^{\uparrow L}=p\triangleright(a\rightarrow a)^{\uparrow L}=p\triangleright\text{id}=p\quad.
\end{align*}

To verify the composition law~(\ref{eq:composition-law-of-ap}),
apply both sides to an arbitrary value $r^{:L^{A}}$:
\begin{align*}
{\color{greenunder}\text{left-hand side}:}\quad & r\triangleright\text{ap}\big(p^{:L^{A\rightarrow B}}\odot q^{:L^{B\rightarrow C}}\big)=\text{ap}\,(p\odot q)(r)=((p\odot q)\times r)\triangleright\text{zip}\bef\text{eval}^{\uparrow L}\\
{\color{greenunder}\text{definition (b) of }\odot:}\quad & =\big(\big(\text{zip}\left(q\times p\right)\triangleright\gunderline{(h\times g\rightarrow g\bef h)^{\uparrow L}}\big)\times r\big)\triangleright\gunderline{\text{zip}}\bef\text{eval}^{\uparrow L}\\
{\color{greenunder}\text{naturality law of }\text{zip}:}\quad & =\big(\text{zip}\left(q\times p\right)\times r\big)\triangleright\text{zip}\triangleright\gunderline{((h\times g)\times a\rightarrow(g\bef h)\times a)^{\uparrow L}\bef\text{eval}^{\uparrow L}}\\
{\color{greenunder}\text{composition under }^{\uparrow L}:}\quad & =\text{zip}\big(\text{zip}\left(q\times p\right)\times r\big)\triangleright((h\times g)\times a\rightarrow a\triangleright g\bef h)^{\uparrow L}\quad.
\end{align*}
Now rewrite the right-hand side of Eq.~(\ref{eq:composition-law-of-ap})
applied to $r$:
\begin{align*}
{\color{greenunder}\text{right-hand side}:}\quad & r\triangleright\text{ap}\,(p)\bef\text{ap}\,(q)=\text{ap}\,(q)\big(\text{ap}\,(p)(r)\big)=\text{zip}\big(q\times\text{ap}\,(p)(r)\big)\triangleright\text{eval}^{\uparrow L}\\
 & =\gunderline{\text{zip}}\big(q\times\big(\text{zip}\,(p\times r)\triangleright\gunderline{\text{eval}^{\uparrow L}}\big)\big)\triangleright\text{eval}^{\uparrow L}\\
{\color{greenunder}\text{naturality law of }\text{zip}:}\quad & =\text{zip}\big(q\times\text{zip}\left(p\times r\right)\big)\triangleright\gunderline{(h\times(g\times a)\rightarrow h\times g(a))^{\uparrow L}\bef\text{eval}^{\uparrow L}}\\
{\color{greenunder}\text{composition under }^{\uparrow L}:}\quad & =\text{zip}\big(q\times\text{zip}\left(p\times r\right)\big)\triangleright\big(h\times(g\times a)\rightarrow h(g(a))\big)^{\uparrow L}\quad.
\end{align*}
The remaining difference is reduced to the associativity law of \lstinline!zip!:
\begin{align*}
{\color{greenunder}\text{left-hand side}:}\quad & \gunderline{\text{zip}\big(\text{zip}\left(q\times p\right)\times r\big)\triangleright\varepsilon_{12,3}}\bef(h\times g\times a\rightarrow a\triangleright g\triangleright h)^{\uparrow L}\\
{\color{greenunder}\text{right-hand side}:}\quad & =\gunderline{\text{zip}\big(q\times\text{zip}\left(p\times r\right)\big)\triangleright\varepsilon_{1,23}}\bef\big(h\times g\times a\rightarrow a\triangleright g\triangleright h\big)^{\uparrow L}\quad.
\end{align*}
The underlined expressions are equal due to the associativity law~(\ref{eq:zip-associativity-law-with-epsilons})
of \lstinline!zip!. $\square$

It is important to use the definition (b) of the operation $\odot$
from Statement~\ref{subsec:Statement-ap-category-laws}. The definition~(a)
does create a valid category but describes a reversed order of effects
and cannot be used to write the law of \lstinline!ap! in the form
of a (categorical) functor composition law. So, we will use the definition~(b)
for the category we call \textsf{``}$L$-applicative\textsf{''}. Statements~\ref{subsec:Statement-ap-category-laws}\textendash \ref{subsec:Statement-ap-functor-laws}
show that the laws of \lstinline!ap!, viewed as the laws of the $L$-applicative
category together with the laws of a functor from the $L$-applicative
to the $L$-lifted category, are a consequence of the laws of \lstinline!zip!.

In this way, we have used the guidance of category theory to formulate
the laws of \lstinline!ap!.

The proofs of Statements~\ref{subsec:Statement-ap-category-laws}\textendash \ref{subsec:Statement-ap-functor-laws}
use the left identity law of \lstinline!zip! but not the right identity
law. That law is equivalent to an additional law of \lstinline!ap!
(see Exercise~\ref{subsec:Exercise-additional-law-of-ap}).

\subsection{The pattern of \textquotedblleft functorial\textquotedblright{} typeclasses
in view of category theory\label{subsec:The-pattern-of-functorial-typeclasses}}

In the previous chapters, we have derived several equivalent formulations
of the laws of various typeclasses (such as functor, contrafunctor,
filterable, monad, applicative). We found that some of these formulations
are easier to use when verifying the laws by hand. However, in every
case we have found a certain formulation of the typeclass laws in
terms of the laws of a \textsf{``}lifting\textsf{''} (namely, the identity and composition
laws). This formulation is important because it makes contact with
category theory, which provides assurance that the laws are chosen
reasonably and correctly. Let us now summarize what we have learned
about the laws of the typeclasses.

In many cases considered so far, we were able to find a certain typeclass
method whose type signature looks like a \textsf{``}lifting\textsf{''} between functions
of one sort and functions of another sort. For instance, the \lstinline!Filterable!
typeclass has the \lstinline!liftOpt! method (Section~\ref{subsec:Motivation-and-laws-for-liftopt-and-equivalence})
with type signature
\[
\text{liftOpt}:\left(A\rightarrow\bbnum 1+B\right)\rightarrow(F^{A}\rightarrow F^{B})\quad.
\]
Here, we wrote the optional parentheses around $(F^{A}\rightarrow F^{B})$
to emphasize that \lstinline!liftOpt! maps from one sort of functions
to another. Category theory generalizes functions (with type $A\rightarrow B$)
to \textsf{``}morphisms\textsf{''}, which in most cases are functions with a modified
type signature (such as $A\rightarrow\bbnum 1+B$, or $F^{A}\rightarrow F^{B}$,
or something else). Table~\ref{tab:functorial-typeclasses} lists
the type signatures of the \textsf{``}lifting\textsf{''} functions and the corresponding
morphism types for some typeclasses. We see that in each case the
\textsf{``}lifting\textsf{''} method maps functions of a selected morphism type to
functions of type $F^{A}\rightarrow F^{B}$.

\begin{table}
\begin{centering}
\begin{tabular}{|c|c|c|}
\hline 
\textbf{\footnotesize{}Typeclass} & \textbf{\footnotesize{}Type signature of the \textsf{``}lifting\textsf{''}} & \textbf{\footnotesize{}Morphism type}\tabularnewline
\hline 
\hline 
{\footnotesize{}functor} & {\footnotesize{}$\text{fmap}:\left(A\rightarrow B\right)\rightarrow(F^{A}\rightarrow F^{B})$} & {\footnotesize{}$A\rightarrow B$}\tabularnewline
\hline 
{\footnotesize{}filterable} & {\footnotesize{}$\text{liftOpt}:\left(A\rightarrow\bbnum 1+B\right)\rightarrow(F^{A}\rightarrow F^{B})$} & {\footnotesize{}$A\rightarrow\bbnum 1+B$}\tabularnewline
\hline 
{\footnotesize{}monad} & {\footnotesize{}$\text{flm}:\left(A\rightarrow F^{B}\right)\rightarrow(F^{A}\rightarrow F^{B})$} & {\footnotesize{}$A\rightarrow F^{B}$}\tabularnewline
\hline 
{\footnotesize{}applicative} & {\footnotesize{}$\text{ap}:F^{A\rightarrow B}\rightarrow(F^{A}\rightarrow F^{B})$} & {\footnotesize{}$F^{A\rightarrow B}$}\tabularnewline
\hline 
{\footnotesize{}contrafunctor} & {\footnotesize{}$\text{cmap}:\left(B\rightarrow A\right)\rightarrow(F^{A}\rightarrow F^{B})$} & {\footnotesize{}$B\rightarrow A$}\tabularnewline
\hline 
{\footnotesize{}profunctor} & {\footnotesize{}$\text{xmap}:\left(A\rightarrow B\right)\times\left(B\rightarrow A\right)\rightarrow(F^{A}\rightarrow F^{B})$} & {\footnotesize{}$\left(A\rightarrow B\right)\times\left(B\rightarrow A\right)$}\tabularnewline
\hline 
{\footnotesize{}contrafilterable} & {\footnotesize{}$\text{liftOpt}:\left(B\rightarrow\bbnum 1+A\right)\rightarrow(F^{A}\rightarrow F^{B})$} & {\footnotesize{}$B\rightarrow\bbnum 1+A$}\tabularnewline
\hline 
{\footnotesize{}comonad} & {\footnotesize{}$\text{coflm}:\left(F^{A}\rightarrow B\right)\rightarrow(F^{A}\rightarrow F^{B})$} & {\footnotesize{}$F^{A}\rightarrow B$}\tabularnewline
\hline 
\end{tabular}
\par\end{centering}
\caption{Some typeclasses that follow the functorial pattern.}
\label{tab:functorial-typeclasses}
\end{table}

We\index{typeclass!functorial pattern} call \textbf{functorial} all
typeclasses that follow this pattern. The word \textsf{``}functorial\textsf{''} will
remind us that there exists a certain (categorical) functor at the
core of the typeclass. Having defined that functor, we can derive
the methods and the laws of the typeclass via the following steps:
\begin{enumerate}
\item Define each category\textsf{'}s composition operation and identity morphism.
\item Verify that the category laws hold.
\item Write the type signature of the \textsf{``}lifting\textsf{''} method corresponding
to the functor.
\item Impose the naturality laws and the functor laws on the \textsf{``}lifting\textsf{''}
method.
\item Derive other typeclass methods (and their laws) from the \textsf{``}lifting\textsf{''}
method and its laws.
\end{enumerate}
Let us go through this pattern for the applicative functor typeclass
that we studied in this chapter. Following the functorial pattern,
we say that a covariant type constructor $F^{\bullet}$ is applicative
if there exists a (categorical) functor between the $F$-applicative
and $F$-lifted categories. The morphisms in the $F$-applicative
category are values of type $F^{A\rightarrow B}$; the morphisms in
the $F$-lifted category are values of type $F^{A}\rightarrow F^{B}$.
This is all we need to start with. All other properties of applicative
functors are then derived in a systematic way. With this approach,
we do not need to memorize the complicated type signatures and laws
of the methods \lstinline!map2! and \lstinline!ap!.

The first step is to define the $F$-applicative category. We define
identity morphisms (\lstinline!wid[A]! of type $F^{A\rightarrow A}$)
and the composition operation ($\odot$) that composes $F^{A\rightarrow B}$
with $F^{B\rightarrow C}$ to obtain $F^{A\rightarrow C}$. The second
step is to verify that the category laws hold with these definitions;
this is done in Statement~\ref{subsec:Statement-ap-category-laws}.
The $F$-lifted category with morphisms $F^{A}\rightarrow F^{B}$
is shared by all functorial typeclasses we have seen, and we already
checked that it satisfies the category laws (Section~\ref{subsec:Motivation-for-using-category-theory}).
The third step is to write the type signature of the \textsf{``}lifting\textsf{''}
function of the functor:
\[
\text{ap}:F^{A\rightarrow B}\rightarrow F^{A}\rightarrow F^{B}\quad.
\]
The fourth step is to require that \lstinline!ap! obey the naturality
and the functor laws. The last step is to derive other methods that
are equivalent to \lstinline!ap! but more convenient to use. We have
shown that the \lstinline!map2! and \lstinline!zip! methods with
suitable laws are equivalent to \lstinline!ap! (see Section~\ref{subsec:Equivalence-of-map2-zip-ap}
and Statement~\ref{subsec:Statement-ap-functor-laws}). 

In this way, we demonstrate that the applicative typeclass follows
the functorial typeclass pattern. 

Many typeclasses can be derived from this pattern, but some cannot.
For example, applicative contrafunctors and profunctors (Section~\ref{sec:Applicative-contrafunctors-and-profunctors})
or traversable functors (Chapter~\ref{chap:9-Traversable-functors-and})
are not functorial typeclasses. Their laws must be motivated by other
considerations.

\subsection{Exercises\index{exercises}}

\subsubsection{Exercise \label{subsec:Exercise-applicative-II-3}\ref{subsec:Exercise-applicative-II-3}}

Implement an applicative instance for $F^{A}=\bbnum 1+\text{Int}\times A+A\times A\times A$.

\subsubsection{Exercise \label{subsec:Exercise-function-type-construction-not-applicative}\ref{subsec:Exercise-function-type-construction-not-applicative}}

Show that the following functors $F^{\bullet}$ \emph{cannot} be applicative: 

\textbf{(a)} $F^{A}\triangleq(A\rightarrow P)\rightarrow Q\quad.$

\textbf{(b)} $F^{A}\triangleq\left(A\rightarrow P\right)\rightarrow\bbnum 1+A\quad.$

Here $P$ and $Q$ are some fixed monoidal types that are not related
in any way to each other.

\subsubsection{Exercise \label{subsec:Exercise-applicative-II-4-2}\ref{subsec:Exercise-applicative-II-4-2}}

Show that the ternary tree functor from Exercise~\ref{subsec:Exercise-applicative-I-1-1}
is a lawful commutative applicative functor. Use applicative functor
constructions to avoid long proofs.

\subsubsection{Exercise \label{subsec:Exercise-applicative-II-4}\ref{subsec:Exercise-applicative-II-4}}

Show that $F^{A}\triangleq G^{A}+H^{G^{A}}$ is a lawful applicative
functor if $G$ and $H$ are. Show that $F$ is commutative if $G$
and $H$ are. Use applicative functor constructions to avoid long
proofs.

\subsubsection{Exercise \label{subsec:Exercise-zip-pure-pure}\ref{subsec:Exercise-zip-pure-pure}}

For an arbitrary applicative functor $F$ with a lawful \lstinline!zip!
method, prove this law:
\[
\text{zip}_{F}\big(\text{pu}_{F}(a^{:A})\times\text{pu}_{F}(b^{:B})\big)=\text{pu}_{F}(a\times b)\quad.
\]


\subsubsection{Exercise \label{subsec:Exercise-simplify-law-omit-lifted-function}\ref{subsec:Exercise-simplify-law-omit-lifted-function}}

It is given that two functions $u:A\rightarrow F^{B}$ and $v:A\rightarrow F^{B}$
(where $F$ is some functor) satisfy the law $u\bef f^{\uparrow F}=v\bef f^{\uparrow F}$
with an \emph{arbitrary} function $f:B\rightarrow C$. Show that the
function $f$ can be omitted from the law: an equivalent law is simply
$u=v$.

\subsubsection{Exercise \label{subsec:Exercise-applicative-II}\ref{subsec:Exercise-applicative-II}}

Show that $\text{pu}_{L}(f)\odot\text{pu}_{L}(g)=\text{pu}_{L}\left(f\bef g\right)$
for an applicative functor $L$. The operation $\odot$ is defined
in Statement~\ref{subsec:Statement-ap-category-laws}(b).

\subsubsection{Exercise \label{subsec:Exercise-additional-law-of-ap}\ref{subsec:Exercise-additional-law-of-ap}}

Show that if \lstinline!ap! is defined via a lawful \lstinline!zip!
then \lstinline!ap! satisfies the following law:\index{identity laws!of ap@of \texttt{ap}}
\begin{align*}
{\color{greenunder}\text{right identity law of }\text{ap}_{L}:}\quad & \text{ap}_{L}\,(r^{:L^{A\rightarrow B}})(\text{pu}_{L}(a^{:A}))=r\triangleright(f^{:A\rightarrow B}\rightarrow f(a))^{\uparrow L}\quad.
\end{align*}


\subsubsection{Exercise \label{subsec:Exercise-applicative-of-monoid-is-monoid}\ref{subsec:Exercise-applicative-of-monoid-is-monoid}}

Show that $P^{S}$ is a monoid if $S$ is a fixed monoidal type and
$P^{\bullet}$ is any applicative functor, contrafunctor, or profunctor.

\subsubsection{Exercise \label{subsec:Exercise-applicative-II-1}\ref{subsec:Exercise-applicative-II-1}}

Prove the following statements (which complement Statement~\ref{subsec:Statement-applicative-composition}):

\textbf{(a)} If $F^{\bullet}$ is an applicative functor and $G^{\bullet}$
is an applicative contrafunctor then the contrafunctor $L^{A}\triangleq G^{F^{A}}$
is applicative.

\textbf{(b)} If $F^{\bullet}$ and $G^{\bullet}$ are both applicative
contrafunctors then $L^{A}\triangleq F^{G^{A}}$ is an applicative
\emph{functor}.

\textbf{(c)} In both parts \textbf{(a)} and \textbf{(b)}, if $F^{\bullet}$
and $G^{\bullet}$ are commutative then $L^{\bullet}$ is also commutative.

\subsubsection{Exercise \label{subsec:Exercise-applicative-II-4-1}\ref{subsec:Exercise-applicative-II-4-1}}

Show that $F^{A}\triangleq H^{A}\times G^{A}$ is applicative and
co-pointed if $G$ and $H$ are applicative functors and $H$ is co-pointed.
Show that the compatibility law~(\ref{eq:compatibility-law-of-extract-and-zip})
holds for $\text{zip}_{F}$ and $\text{ex}_{F}$ if it holds for $\text{zip}_{H}$
and $\text{ex}_{H}$.

\subsubsection{Exercise \label{subsec:Exercise-applicative-II-5}\ref{subsec:Exercise-applicative-II-5}}

Explicitly prove the laws in contrafunctor product construction (Statement~\ref{subsec:Statement-applicative-contrafunctor-product}).

\subsubsection{Exercise \label{subsec:Exercise-applicative-II-7}\ref{subsec:Exercise-applicative-II-7}}

Show that the recursive functor $F^{A}\triangleq\bbnum 1+G^{A\times F^{A}}$
is applicative if $G^{A}$ is applicative and $\text{wu}_{F}$ is
defined recursively as $\text{wu}_{F}\triangleq\bbnum 0+\text{pu}_{G}\left(1\times\text{wu}_{F}\right)$.
Use applicative functor constructions.

\subsubsection{Exercise \label{subsec:Exercise-applicative-profunctor-composition}\ref{subsec:Exercise-applicative-profunctor-composition}}

Prove Statement~\ref{subsec:Statement-applicative-profunctor-composition}.

\subsubsection{Exercise \label{subsec:Exercise-profunctor-example}\ref{subsec:Exercise-profunctor-example}}

The type constructor $Q^{\bullet}$ is defined by 
\[
Q^{A}\triangleq\left(A\rightarrow\text{Int}\right)\times A\times\left(A\rightarrow A\right)\quad.
\]
Show that $Q^{A}$ is neither covariant nor contravariant in $A$,
and express $Q^{A}$ via a profunctor.

\subsubsection{Exercise \label{subsec:Exercise-applicative-II-11}\ref{subsec:Exercise-applicative-II-11}}

\textbf{(a)} For any given profunctor $P^{A}$, implement a function
of type $A\times P^{B}\rightarrow P^{A\times B}$. 

\textbf{(b)} Show that one \emph{cannot} implement a function of type
$A\times P^{B}\rightarrow P^{A}$ for some profunctors $P^{A}$.

\subsubsection{Exercise \label{subsec:Exercise-applicative-II-10}\ref{subsec:Exercise-applicative-II-10}}

Implement profunctor and applicative instances for $P^{A}\triangleq A+Z\times G^{A}$
where $G^{A}$ is a given applicative profunctor and $Z$ is a monoid.

\subsubsection{Exercise \label{subsec:Exercise-profunctor-pure-not-equivalent-1}\ref{subsec:Exercise-profunctor-pure-not-equivalent-1}}

For the profunctor $P^{A}\triangleq A\rightarrow A$:

\textbf{(a)} Show that $P^{A}$ is pointed: there exist a value \lstinline!wu!
of type $P^{\bbnum 1}$ and a method \lstinline!pure! of type $A\rightarrow P^{A}$. 

\textbf{(b)} Show that the type of fully parametric functions $\text{pu}_{P}:A\rightarrow P^{A}$
is \emph{not} equivalent to the type $P^{\bbnum 1}$.

\begin{comment}
this is chapter eight devoted to applicative hunters and pro functors
the first part will be practical examples main motivation for applicati
factors comes from considering kinetic computations or computations
in the factor block as I color in case when effects are independent
and competitive or in case when these effects could be executed in
parallel while the result is still correct here is an example consider
a portion of a functor block or a for yield block in Scala that looks
like this there are three future values and X will be waiting until
with this future value is ready imagine that these are some long-running
computations and after these three lines we can use XYZ and further
computations any further computations will be waiting for these three
futures now if we write code like this then these three futures will
be created sequentially that is first this future will be created
and scheduled on some thread and then when it\textsf{'}s done you get the value
X and then this second future will be created and scheduled even though
it doesn't use the value X but the monadic block or the for yield
block or the Thunder block whatever you want to call it I call it
a factor block the Thunder block is such that every generator line
locks everything else until it\textsf{'}s done so in the future we'll be created
one at a time and so obviously this is not optimal if you translate
this into flat map code and map that\textsf{'}s the code so you have a first
future but to which you append this flat map so you schedule it is
further computation only when the first computation is ready the second
future will be then started and this will be waiting until the second
future is ready and then the third future will be started and this
will be ready will be waiting until the third future is ready clearly
this is not optimal would like to parallelize these things and we
have seen in a previous tutorial that a very easy way of paralyzing
such computations is to create the futures before starting the factor
block but this is actually a specific feature of scholar where futures
already start computing when you create them there is no separate
method to start computing them which is actually a design flaw and
we would like to express more carefully that certain computations
can be done in parallel because they're independent whereas other
computations really have to wait one for another now in this example
like I said C 1 C 2 and C 3 are just some fixed computations that
don't depend on the values computed previously if we do have this
dependency and of course there\textsf{'}s no way except wait until each previous
computation is done but in case they're they're independent we would
like to be able to compute them in parallel and we would like to express
that in a better way you know code rather than just a hacky way where
you create these futures separately and then put into some variables
and then you write this code because that will work for future will
not work for some other factor another use case where moon ads or
monadic effects are not exactly what we want is a case when we perform
computations that may give errors now we have seen in the previous
chapter that using a moon ad such as either moon ad option will not
we could stop at the first error very easily but sometimes we don't
want to stop at the first error we want to maybe accumulate all errors
as much as possible and give the user more information so {[}Music{]}
monads cannot do that in general they cannot accumulate all errors
because if one monadic step fails then the next one won't be executed
so if as I said effects are independent then we would like to have
a different way of computing than the moon at computation shown here
so monads are inconvenient for expressing independent effects also
they're inconvenient for expressing commutative effects because typically
when you change the order of generator lines like these two you change
the results even though these generator lines seem to be independent
of each other they aren't the computation will first iterate with
X going over this list and for each X Y it will be going through that
list in that order first X will be iterated over in this list then
Y over that list now order is fixed and you can do nothing to change
it so for example if you interchange these two lines you would get
a different list as a result the first iteration would be interchange
with the second one and the list will have a different value now for
lists the order of the order of elements is important perhaps for
other monads usually this is also the case that you cannot just interchange
lines in a factor block and expect the result not to change however
there are some cases where these computations logically speaking are
independent and what we would like to do is express this very clearly
this is how it is done we would like to have a function that computes
something like this but assuming that the effects are commutative
how would that function work well look at what it does we have here
two containers and we compute some function from values that are stored
in these two containers and put all the results in a new container
so if we just express this computation as a function that function
would have the following type signature it will take two containers
let\textsf{'}s call them FA and FB so let\textsf{'}s say F is list and aliened b are
going to be the types of the elements in these two containers in this
example both integers but in general we might have two different types
a and B so this function takes these two containers it also takes
a function f which is a function from a pair a B to some type C as
a result we get a container of type C now clearly this kind of function
let\textsf{'}s call this map to because it\textsf{'}s like map except you have two containers
and two arguments of a function that you use now of course this computation
has this type signature so if you have a monad for F and then you
can define this function using this code very easily X left arrow
F a YF arrow left arrow FB yield F of x1 so that is the code that
implements map to in terms of flat map and map now the key observation
here is that we would like to not use the flat map because flat map
will force this ordering of effects and it will prevent us from having
commutative effects whereas this function we could define differently
we could define it separately from flat map in a different way such
that this function is symmetric in a and B for example we could do
that and once we are able to do that we can use that function by itself
without writing furniture blocks may be using a different syntax but
basically just using this function to prepare this data in this way
so to process several containers in a symmetric way that is to endure
more independent containers have independent effects and a function
that maps their values to some new type and the result is a new container
so that is the key idea behind applicative functors applicative factors
are factors F for which this function is available somehow also the
function P or the same function as in monads with the type signature
a going to f of a must be available but that function just like in
the case of Moonen is much less important for practical coding than
this function so applicative functor is a factor that has these but
first of all it does not have to be a monad and even if it is a monad
in some way the definition of map 2 does not have to come from the
definition of flat map like this so the reason is that we want a different
logic we don't want the calculation going like this with flat maps
we actually want to avoid using flat map so that\textsf{'}s why we would define
map 2 separately not via flat map and use that function map to directly
on some containers in this way for example we would be able to define
a function of three futures that takes also a function of three numbers
and compute the future of the result now that would be map 3 as it
were so let\textsf{'}s consider this as a generalization so as I said for a
monad we can define these functions through a flat map and let\textsf{'}s just
use that as a guide we're not going to define through flatmap really
but we should be able to get a lot of intuition by just looking at
the functions that we get from flat map because the types are going
to be the same and so a lot of intuition comes from looking at the
type of a function so if we just have one container then it\textsf{'}s a ordinary
map which is equivalent to this code in Scala I remind you that the
left arrow or the generator arrow is translated into either flat map
or map it\textsf{'}s if there are several generator arrows it\textsf{'}s the first until
the last one so the last but one all of them are flat maps and the
last one is map so the last left arrow before yield is just translated
into a map so that\textsf{'}s a map now this would be first a flat map than
an ordinary map but we want to replace that with this kind of syntax
perhaps where we have map - it takes a tuple of containers it also
takes a function f with two arguments and it returns a container here
is a similar thing with three containers where we have map three so
the function f needs to have three arguments and the result is again
a single container so how do we generalize this let\textsf{'}s write down the
type signatures so map the usual map let\textsf{'}s call it map one just to
be systematic then it\textsf{'}s like this this is the standard type signature
for the map function map to has this type signature takes two containers
as we already saw on the previous slide map three takes three containers
takes a function with three arguments instead of two and it again
returns a new container so clearly this is how we can generalize map
one map to map three and so on the type signatures are clear now applicative
factors have all of these functions map in in fact we will see in
a later tutorial that once you have map one and map two it is sufficient
in order to be able to define all the others you don't need to have
separate definitions for all the other maps in you can define map
in taking a list of these a list of these it\textsf{'}s easy once you have
met one and map - we will not dwell on it in this tutorial that\textsf{'}s
the subject of the next tutorial because it requires us to consider
the laws that must be satisfied by these map end functions and then
it will be much easier to understand why all these functions are interrelated
to each other but at the same time it will require us to go much deeper
into the relationships between these functions the equations and the
laws so that\textsf{'}s going to be the subject of the next tutorial part 2
of chapter 8 in this part one I'm just going to talk about practical
considerations practical examples of how you use this map in concentrating
mostly on map to map n will be used in a very similar way so what
are examples where we want to use such things let\textsf{'}s look at the example
code so the first example is the easier type so imagine we're doing
some computations when results could be given as values of type a
or as you could have a failure an error message that is represented
by a string an example of such a computation would be the safe divide
function where I divide but I check whether it is not 0 and if it\textsf{'}s
0 then I don't divide I'll give an error now this is just an example
of this kind of approach where your computations have the type up
of double instead of double and this is a type constructor that encapsulate
some effect some error accumulation or some other effect so in this
case this constructor represents errors usually as we have seen before
so let\textsf{'}s implement map 2 in a way that would help us understand how
we can collect all errors so map 2 needs to have this type signature
it takes up of a up of B it also takes a function from a B to Z and
it returns up of Z so it needs to have type parameters a B and Z how
do we compute this OP of Z well let\textsf{'}s find out we have up of a and
up of B now op is an either so we can imagine it let\textsf{'}s match right
away on both and the now that there\textsf{'}s a case when we have two errors
and in this case it will be interesting for us to accumulate these
error messages so let\textsf{'}s not throw away information here and let\textsf{'}s
do this so we can catenate the strings in more general situation this
type instead of string it could be some other type which could be
a monoid and then we could combine two values of a mono it into a
larger value in case we have one error and one result you can't really
do much except return that error and in case we have two results we
can actually perform a computation with this requesting which is this
function f and return a right so in these two cases there isn't enough
data to call F because we only have one value but not the other so
we cannot call F and the only thing we can return is an error so this
is how we implement map two now notice that this map 2 does more than
a monadic factor block would would do because the magnetic factor
block would stop after the first error this one doesn't stop not to
the first air it takes also look looks at the second computation and
if it also is an error then it accumulates the errors into into this
value this is something a monadic factor block cannot do cannot accumulate
errors because once it finds the first there it stops and returns
that error so here\textsf{'}s an example how we could compute for example here
is map - I'm just going to apply this function to the values you find
like this as an example now in a real computation this doesn't seem
to be very convenient to always have to write computations like this
Scala can give you a lot of convenience by defining domain-specific
language with syntax and operators so that you don't have to write
all that stuff but under the hood it must do this stuff so this is
the way it works under the hood for any kind of library that would
do this more conveniently so you could call this function map two-on-two
computations now notice they are both dividing by 0 so these two computations
will give you two error messages and now the function that you pass
is a subtraction and that function by itself doesn't give any errors
but it never gets called because you actually have errors in both
of these two computations so the tests asserts we actually have two
error messages telling you what you're dividing by zero now if you
instead wrote this kind of thing which has the same type signature
now you would have just one error message the first one would have
stopped the functor block from continuing so this is how it works
now we can define map three let\textsf{'}s take a look is a very simple type
can't be so bad now however we now have three computations each could
be an error or a result so there are eight possible cases we're not
going to write them down that would be really not great so how do
we do that well let\textsf{'}s think about how we could use map to instead
of writing this function map 3 from scratch map to only works by applying
it to two containers ok let\textsf{'}s do that let\textsf{'}s apply it to a and B and
well the function f requires three arguments we can't apply it let\textsf{'}s
not apply it let\textsf{'}s just apply it to some function that doesn't do
anything to the arguments just accumulate them in a tuple which is
almost an identity function just to remind you that in the Scala syntax
this is not actually a tuple this is anonymous function that takes
two arguments and if you wanted a tuple you'd have to write like this
which is the to pull of X while I'm going to the tuple of X Y this
is the actual identity function on tuples but this is not what we
defined so it\textsf{'}s almost an identity function up to some syntax we could
have defined the syntax so that it is really an identity so we don't
we just combine these two what\textsf{'}s the result the result is this so
it\textsf{'}s an oak of a tuple now we can use again map 2 to combine this
up of a tuple with C and then the map 2 will have this type of argument
now we can apply F so this kind of thing is how you could use me up
to to define map 3 now clearly this can be generalized a little awkward
to generalize but it can be generalized the awkwardness comes from
here it'll be hard to generalize this case expression obviously if
you have map n you need to apply map 2 then again map 2 again map
2 when you do that every time I have a more more 2 pulls in your case
expression so that\textsf{'}s a little awkward can be done with some more work
but not going to look at this right now yeah easier ways of defining
all these map and functions here is an example of how it works a map
3 working on three computations of this type accumulates the two errors
that you see the third computation does not give an error so that\textsf{'}s
fine now a use case for this kind of type is where you want to validate
value so imagine if has some case class with a few values you want
to you want to validate them and each validation is separate from
other validations just like each of these computations is separate
so only when all of the three parts of a tuple or a case class pass
validation you want to create actually a value of the case class otherwise
you want to fail but when you want when you fail you want to gather
all the errors this is how you would do that you would say you do
a map three of these three validated computations safe and divided
and then you would apply that map three on that to a function that
takes three values and maps and makes C case class value out of them
which is C don't apply no see don't apply is a function that takes
three arguments and returns case class I've made out of them so this
C dot apply is defined automatically by Scala very convenient so in
this case in this example obviously there aren't any divisions by
zero and so we get the result second so we have gone through these
two examples the third example is when we prefer a future our computations
concurrently and here\textsf{'}s how we would do that we would define lab 2
for futures now since these arguments are eagerly evaluated in there
there are already given when you call up to these futures already
were started on some threads and so we can just write the free on
block like we did before these futures are not really sequential because
they already have started so they run most likely on separate parallel
threads already when we are starting to evaluate this function so
that\textsf{'}s why it\textsf{'}s ok then you find map 2 through a flat map for futures
now that is a test when we do a map - like this on these two features
and then there\textsf{'}s another addition so it\textsf{'}s 1 + 2 3 + 4 3 that will
be 10 and map n can be defined like this so there\textsf{'}s a list of futures
and you returned the future of alistel doesn't really map in there
must be still a function so let\textsf{'}s call it something else in the standard
library it\textsf{'}s called future that sequence sequence is a kind of confusing
name it refers to changing of the order of these two type constructors
first you had a list of futures and then you have the future of the
list very important and useful function in the standard library that
essentially implements most of map in you still need to map this list
to some other value so the real map and we'll take a list of futures
they also take a function from a list of A to B so we need a type
of Z and we will return a future of B and the reason yeah and then
we would just do that in order to get it working so that was the example
of the futures another example where applicative factors could be
useful is when you have a reader mu net worth or functions with some
standard arguments and you are getting tired of passing these arguments
over and over to all kinds of functions so as we have seen in the
previous chapter the reader moment does that kind of thing and it
turns out that the reader monad whose effect is this constant value
D that you could always read which is kind of an standard argument
of all computations that you are doing now this e is a constant it
is an immutable given value and so and that\textsf{'}s the only effect of this
moment so this monad always has independent and commutative effects
so in this monad you don't have to worry about the order of effects
it is already independent and in other words we can define map to
buy a flat map there\textsf{'}s no penalty for doing so here\textsf{'}s an example imagine
we have some application that needs to use a logger now a logger we
will just do a quick and dirty logger which has a side effect which
takes whatever value of an any type and prints it in some way and
he returns unit now this is a very dirty way of doing login but it\textsf{'}s
quick it\textsf{'}s dirty in the sense that it\textsf{'}s hard to see that you have
logged anything or not because this function doesn't return any useful
values but let\textsf{'}s continue with that for now a functional logger of
this clean is the writer monad it tells you explicitly that there
is an extra value being created with each computation which is perhaps
a log message and it tells you also that you can bind the log messages
together using a Monod so that\textsf{'}s a clean way of logging but let\textsf{'}s
just for the sake of this example consider this logger so we have
an empty logger and maybe some non empty loggers and now suppose that
every computation we want is going to be logged so every computation
is going to use the slogger and call this print function many times
for whatever reason and so every computation needs a logger as an
argument because that logger will provide the printing functionality
and if you do that also it becomes much easier to test such code then
you can pass an empty lager or you can pass a test only debugging
lager or anything like that but then of course all your computations
become more cumbersome because now you have functions that have these
in this argument so here\textsf{'}s for example a computation that adds two
integers it returns a logged integer it is a function that takes a
logger it does this computation who logs it and returns the result
so it\textsf{'}s a typical kind of code that you would use now suppose I wanted
to combine these computations so I'm loading one plus two I'm logging
10 plus 20 the results are X\&Y and I have X plus y now changing the
order of computations gives the same result except for the side effect
printed of course that is not going to be commutative first it\textsf{'}s going
to print this and then it\textsf{'}s going to print that but the side effect
is invisible in the types of these expressions and so the expressions
are going to be equal after interchanging the order as this test shows
and this is another illustration of the bad nature of side effects
you don't see them you have no assurance that the side effects have
been performed in the order you want so for this example it will be
okay so now let\textsf{'}s define map to just define it via flat map like this
or we can define it by hand like this which is kind of obvious because
these this is a function from logger to a this is a function from
blogger to B and this is a function from logger to Z so we obviously
just call these two functions with a and B provided by those two functions
so echo F on this that\textsf{'}s the only way to implement map to now this
way is exactly the same actually we can verify this symbolically but
the code is exactly the same because maverick is this for the reader
munna flat map is that there\textsf{'}s no freedom here the types fix the code
uniquely and then we can calculate what is mapped to of a B which
is going to be this is this is the translation of the for yield construction
we can do that by first looking at the map substituting the definition
of the map right here and applying the function to the argument so
then you get this function and finally we need to put that function
into flat map substitute the definition of flat map which is this
and we get this code so by just symbolically transforming the code
I step-by-step arrive at the code of the other function so in this
way you could use map to map N and so on and have your standard arguments
passed to functions but so I do that with the reader Mona if you can
just do the for yield construction well what if you can't yeah there
is no Timon on somewhere in case your type is more complicated than
this you might be in a situation where you don't have a monad but
still you need to pass standard arguments and you can do that with
map in another example is the list which is a monad and we have seen
that every minute is already an applicative because you can always
define map to buy a flat map but there might be a different definition
map to in case of list there are various definitions we'll just use
a standard one right now and show how we can transpose a matrix so
let\textsf{'}s define map - first of all on map - is a very simple thing for
this is the zip so first we can zip see what we need is a function
from list a and list B - a list of pairs a B no look you just may
up with F and that\textsf{'}s what we write here so we do an a sub B which
gives us a list of pairs a B and we can map that now the zip function
has a specific implementation we could change that especially when
a and B do not have the same length then there are lots of choices
about what to do do you want to cut short so you want to fill with
some default values or something like that but what let\textsf{'}s not discuss
it right now we will see in in the second part of this tutorial how
to define map 2 in various ways in case there are several definitions
so for now we just use the standards library in Scala to have the
zip function so that\textsf{'}s clearly what map - does so you can see that
map 2 has a close connection to the zip function for lists that takes
two lists and returns a list of pairs and then you just map over that
list with a function f so in up to on a pair of these two lists with
a plus function will give you pairwise sums so how do we transpose
a matrix now the matrix represented here is a list of lists needs
to be transposed so in fact we need to understand how we represent
a {[}Music{]} matrix as list of lists so let me have a an example
so this matrix is a list of three lists and the transpose matrix is
a list of two lists but the first has these three elements together
and the second list has these three elements together so to transpose
them what we need is to take the heads of each list and put these
heads into a list of their own and then we can use transpose on the
tails of each list in the same way so we are we're going to have a
recursive function obviously so how do we do that so suppose that
this list of lists has heads and tails so what does it mean heads
so this is the heads now actually {[}Music{]} what you want is to
append this to this and to this you want to have a list that goes
like that so clearly the head of that list is going to be the head
of lips and let\textsf{'}s transpose the tails so tails are these so sorry
tails are these if we transpose them then we will have a list of these
two and then a list of these two now what we need to do is we need
to append this to that list and this to that list now this append
looks like a component wise operation on this list and on the list
of those so that\textsf{'}s where we use map to we use map to on heads which
is this and on transposed tails which is it transposed this to append
the first elements to the rest which means that we will append this
to that will append this to that and so on so it remains to transpose
the tails now it\textsf{'}s very easy to transpose the tails you just call
transpose on the tails right here everything else in this code is
bookkeeping that is designed carefully to avoid problems where you
have empty lists now that code is kind of cumbersome and error-prone
you do a map with a function that returns empty lists and here you
just return an empty list it took me a few tries to get it right this
code in this code but the other thing is more important so this is
how we use map tool in order to concatenate component-wise this with
this at the same time this with this and so on so this is the kind
of typical computation that the clickity factors do they do component
wise computations component wise computations are independent of each
other\textsf{'}s results so this is independent of this and that\textsf{'}s where you
use applicative factors and some tests to show that this is correct
so after these examples in principle there are a few other examples
where you use applicative factors and before going to those examples
that are a little more compact complicated I'd like to talk about
applicative contractors an applicative crew factors now the reason
I'm talking about this is that actually they are sometimes quite useful
you can have an applicative instance or a function such as map 2 or
zip for a tight constructor that is not a factor we have seen in a
previous tutorial that type constructors could fail to be factors
in a number of ways one of them is when they are contractors that
has contravariant type constructors but there\textsf{'}s another way where
they can be neither factors nor contra factors and that happens when
your type parameter is both in a covariant and in a contravariant
position in the type constructor sorry this is an example so if you
have a as your main type prime Z let\textsf{'}s say is a constant type it is
your main type parameter then this a is to the right of the function
arrow and this is to the left of the function area so this a and this
a are in a covariant position and this is in a contravariant position
because this type constructor contains both covariant and contravariant
position for a it cannot be a functor and it cannot be a control factor
for in terms of a but nevertheless it has still properties that are
quite nice it is called the true factor when you can see all type
parameters either in a covariant or in a contravariant position I
will talk about Pro factors later in more detail but for now just
keep in mind that it\textsf{'}s very easy to see which position is coherent
and which is controlling and just look at the function errors in your
type and everything to the left of a function arrow becomes contravariant
if there is an arrow inside that then it can again become covariant
so we have seen examples of this so here\textsf{'}s how Pro factors can become
useful this is an example where we can define a semigroup typeclass
instance for a type that has parts such as a tuple or a case class
where each part also has a semigroup typeclass instance and let\textsf{'}s
see how that works just for brevity I will define semigroup here like
this it\textsf{'}s a very simple tight class it just has one method and this
method basically is equivalent to this data type a product of two
S\textsf{'}s going to 1s so clearly this data type considered as a data type
is not a factor and not a control funky in the type parameter S now
we usually don't consider typeclass traits as containers as data types
but we can we could and what will happen if we do is we will notice
that usually there would not be factors and not become true factors
because they have too many things in both covariant and control positions
so these two are contravariant position this is a covariant position
but it is a pro functor because it has nothing but things in covariant
and contravariant positions in that case it\textsf{'}s going to be a pro factor
in fact all exponential polynomial types are going to be Pro factors
there\textsf{'}s no way you could fail to be be a profounder as long as your
stables and exponential polynomial types now the interesting thing
is that we could define a zip function for this type constructor so
let\textsf{'}s forget front at the moment that we're going to be using this
type constructor just to define a few implicit values for the typeclass
and let\textsf{'}s just consider that as a type constructor and try to see
what zip function would be for that type constructor well zip function
would have this signature we'll take one semi group another semi group
and we'll return a semi group of a pair how do we define that wallets
pretty straightforward you need to define a combined function that
combines a B and a B into a B or you just combine this a with this
a into this one and you combine this B with this B into this one using
the combine operation from the semigroups P\&Q and that\textsf{'}s the code
and now we can use that to define semi grouped typeclass instance
for a pair it\textsf{'}s just same I just I'm just going to use their syntax
and to test that this works i define semigroup instances for integer
and double in some way and i have a pair of int double and now I am
able to use the syntax so after this implicit definition I am able
to use this index so this is a zip function is very closely related
to map to we did not actually use map tool right now but once you
have zip then you get your container of pairs and all you need to
do is to map your function of two arguments over this container to
get the map tool so zip and map two are very closely related once
you implement one you can implement the other the second example I'd
like to give is that let\textsf{'}s consider this cofactor and let\textsf{'}s just implement
an app to for it now map to for a profounder has a different type
signature than map to for a factor and also it\textsf{'}s called imap2 in the
cat\textsf{'}s library and in any case to indicate that this is no this is
not a functor map to they call it invariant so prou factors are called
invariant which is quite confusing to me because invariant has so
many other meanings in different contexts such as how you make a transformation
but something doesn't change then you say something is invariant with
respect to the transformation but none of that is happening here nothing
is being transformed such that it remains invariant in geometry for
example the length of line segment is invariant under rotations the
the area of a triangle is invariant under rotations so that\textsf{'}s the
kind of context that I'm familiar with and calling this invariant
would lead me to to ask what is the transformation that you're applying
such that this does not change and that makes no sense at all in this
context or for instance in the computer science there is something
called loop invariant which is an expression that remains constant
throughout each iteration between each iterations of the loop there
is no loop here and nothing remains constant so again that meaning
of the word invariant does not apply so that\textsf{'}s confusing to me so
I don't want to talk about invariant factors I want to say Pro factors
it\textsf{'}s shorter anyway so let\textsf{'}s look at how we can implement an analogue
of map to function for this type constructor now this type constructor
as I said already it\textsf{'}s not a functor cannot be possibly in a factor
because it has the type parameter a in a contravariant position and
also in the covariant position so it cannot be a contra factor either
but it is nevertheless what I call zero ball you could define zip
for it and you can define imap2 for it let us see how do we define
i map to Fred so here I define this type constructor F now in order
to do an app - we use the function from a B to C and we applied that
to a container with a and container with D so here we have a container
with le here we have a container with B and here we have a function
from A to B to C but it turns out that\textsf{'}s insufficient you also need
a function from C back to a B only then it makes sense to define mapping
functions for a profounder and the reason is that when you want to
define a mapping function and you transform the type parameter say
a into some other type frame and say C but when you have your type
in a covariant position then you just compose with the function that
transforms when you have your type in a contravariant position you
have to have a function that goes backwards from C to a side and then
you would take a C you will use that function to get an A and when
you put that a as an argument in a contravariant position so for this
reason the profanity requires a function from C back to a B now if
this were purely a control factor then we would just use this function
from C to a B and we will be done I remind you that contra factors
are just very similar to functions except the map function takes the
opposite direction of transformation in types but here it is neither
a functor nor a contra factor and actually we need both functions
F and G F goes from a B to C and G goes from C back to an a B and
then it turns out we can do what we need how do we do that well so
let\textsf{'}s write the code being guided by types and by the intuition of
what we need to do so we need to produce an F of C and Z so that\textsf{'}s
going to be a function from Z C to C C so we need to return a tuple
of C see how do we do that well it\textsf{'}s obvious that we need to use these
functions F and G somehow we have a see if we use G on that we get
a pair of any bit let\textsf{'}s do that so we get some new a and new be my
back transforming C so now these new and new B need to be substituted
into the contravariant positions which are these positions so we take
F AZ and we substitute Z and a now Z there\textsf{'}s only one Z and that\textsf{'}s
not going to change if those are not transforming that type parameter
Z at all so Z stays the same but here we substitute new and new B
into the contravariant position of the F the result is going to be
pairs of a a and B B so let\textsf{'}s call them new a nu B B so now we can
apply F to this data to get a new value of C we need to have two different
values of C and it makes sense that we would use the first two and
the second to like that to substitute that into F so that\textsf{'}s how we
would do that let me try to improvise and define a zip function for
this type constructor so what would be a zip function so will be F
is e type fpz of type F of Z and the result will be F of pair a B
and Z Z is not going to change we're just by the transform and what
do I need I need type parameters a B and Z so how am I going to do
that well obviously F is this on e to have a function that takes Z
and a pair of a B in Scala I cannot just have a function that takes
that as arguments I need to do a case expression so that Ida structure
these arguments as it\textsf{'}s called or a match from them so now I need
to produce a pair of actually a B a B I need to return a pair of tuples
like that what I have is a function from za to a a and from Z B to
B so clearly I need to use those functions if I apply faz to Z and
a here I get an a a and it makes sense that I will put these days
here a and a and I'll do the same with B so let me write it down new
a a is fa z of new baby is f DZ of Z B so now I've got my Paris baby
and a and now I can return this which is going to be basically this
I'm going to return this Pinner and I'm going to return that pin now
if you compare this and the code appears to be quite similar except
that I don't have an F and I don't have a G and that\textsf{'}s quite typical
of these pairs of functions one of them is equal to the other when
we put identities instead of types instead of some arguments and the
other is obtained using some kind of F map so we've seen this pattern
before and that\textsf{'}s what it will be but that will be in the next part
of this tutorial so in this way we find that this kind of type constructor
has a zip like function which is exactly the same type signature is
a typical zip we'll begin with this if you ignore the extra type parameter
and will be just a typical zip F of a F of B going to F apparently
just like lists instead of f however this F is not a functor is a
much more difficult object perhaps to work with so that was that was
the example I was interested in showing how you can define zip and
imap2 for those profounder now and call non disjunctive and the reason
is these type expression do not contain any disjunctions and actually
if they do contain these junctions and it\textsf{'}s not always possible to
define zip so I call them zip herbal so not all of them are zip about
so it\textsf{'}s just a lot of them are but not all but when they do not contain
any disjunctions and they're always zip alone so with this understood
let us now consider an interesting practical case where we can use
this knowledge and actually make a code simpler in this case is the
so called fusion for fold and the idea is that fold is a computation
that iterates typically over a container and if you need to iterate
several times and then do a computation on the results that\textsf{'}s kind
of wasteful that\textsf{'}s better to iterate once and accumulate more intermediate
results but when you write code for this B it becomes cumbersome if
you have to do it every time you have to write a different complicated
fold function so the idea here is that we can actually automatically
merge several of these fold functions into one so that everything
is computed in one traversal so there are several libraries so one
library is called Scala fooled which is this one where this is implemented
where you can combine different folds of as an example I will show
you if you want to compute the average of a list then you'll have
to traverse the list twice once to define its length and another one
another time to make a summation and that\textsf{'}s wasteful and so you can
merge these folds into one so you can define a fold separately and
then apply merged merge several folds into one and apply that one
fold to a list traversing just once so let\textsf{'}s see how that works so
fold is an operation that takes several parameters so let\textsf{'}s remind
ourselves for instance a list the standard would be our running example
of data so hold left for instance what does it require requires an
initial value of some type B and an operation that takes the previous
accumulated value a new element of the list which is going to be of
type double and returns a new accumulated value so a fold essentially
takes these two values apply it when you apply that to the list so
let\textsf{'}s just have a type that encapsulates these two arguments over
fold left so this is going to be a very simple type just going to
be a tuple of these two values of this just like that so Z is the
type of values in your collection and R is a type of the result so
in order to call the full left you need to provide this data and the
list of course so let\textsf{'}s put this data into a type constructor of its
own so here I call it fold 0 so it\textsf{'}s kind of a 0 version of version
zero of this implementation I put it in Turkey\textsf{'}s class so the case
class is a product and I just put names on it for convenience and
very simple syntax extension will make it possible for us to apply
this fold value to a collection that is foldable now the foldable
typeclass we haven't looked at yet but basically just it\textsf{'}s a it\textsf{'}s
some items it\textsf{'}s a collection it has fold left basically all polynomial
type constructors are foldable and no others so it\textsf{'}s I prefer to think
of foldable as just a property of polynomial type constructors now
it\textsf{'}s interesting that we can define an instance of what Catalan recalls
in variant semigroup oh and what I just called in a previous code
snippet as if Abel pro factor neither of these names are particularly
nice as I said in variant just gives me all kinds of wrong associations
and semi Drupal is a difficult thing to do a difficult thing to understand
because it is not a semi group so semi group all suggests that it
is a semi group but it isn't so zip abou proof factor is the same
as invariant semi global I'm not sure what terminology is better neither
is standard neither terminologies widely used right so let\textsf{'}s look
at this obviously we have our in the contravariant position and also
we have our in a covariant position so again this is going to be a
pro functor not a functor no matter we can do a zip on it and I just
use my career Howard library to implement the IMAP and the product
methods now the product is the same as zip Justin cats library uses
the name product instead of zip IMAP is this Pro functor property
where you you can map fold 0 of Z a two fold 0 of Z B if you have
functions from A to B and from B to a so this is a typical thing that
the proof factor requires you cannot map a pro factor of a into profounder
of B unless you have functions that map in both directions because
you would use this function to substitute in the covariant positions
and you would use this function to substitute in the contravariant
positions and since you have both positions for your type parameter
you need both functions but luckily enough this type is sufficiently
straightforward so that my Harvard library can automatically implement
these methods and then I can implement zip as a syntax by just using
this product because zip is exactly the same type signature as this
product in the Katz library and now how will we use that so let\textsf{'}s
define some actual folding operations for numeric data for instance
the length of a list or sum of elements so the length is going to
be a fold with initial value 0 and updater that just adds 1 to the
accumulated value ignoring the value that is in the collection and
here i'm i've used the numeric typeclass which comes from the spirit
library let me see what was so ridiculous oh yeah here\textsf{'}s the spire
mass library so I'm using the spire math numeric because I found that
standard scholars numeric they are very hard to use inspire has a
good library it has lots of interesting types and I could recommend
using that so for numeric n we have operations such as plus minus
and divided and multiplying so on so for those it makes sense to do
a sum by folding with this update function that accumulates the Sun
starting from 0 now we can combine these two now these two must be
diffs rather than vowels because they have type constraints that cost
constraints so typeclass constraint is really a implicit argument
of a function so it must be a function is not not a value and it has
a type parameter so it cannot be a value anyway can be a valence color
it has a type parameter and/or it if it has an argument type typeclass
constraint but it doesn't really matter you can just do it like that
now if we apply the zip operation and we get a fold that accumulates
a pair of two numbers so the chemo is the length and the sum separately
so we can apply this fold to a list we get this pair and then we can
divide the sum by the length and get the average of the sequence so
in this way we already realize what we wanted we have combined folds
and this is a single traversal because this is a single at full left
operation now this is however inconvenient because we need to do this
combining and then we have a tuple and we have to take parts of this
tuple we would like to incorporate this final computation somehow
already into the fold so when we apply this then all of this is done
automatically and also we don't want to worry about this so much this
we want two things to be automatic how do we do that well so the idea
is that this final computation that takes the accumulated value does
something to it and gives you a different result perhaps of a different
type like here the type of the accumulated value was tuple and in
and the type of the result was just in we would like to put this final
computation also into the data structure that is the fold so that
when we apply the fold to a list all this is done automatically well
easy to do we define this fold one which is the same as before it
has the initial value it has an update here and then it has this final
transform which takes the accumulated value and returns some result
value and the types our za and our so we now we have three type travelers
in this data structure but so what we can have as many as we need
we find a syntax I do this fold l1 now unfortunately the syntax cannot
clash with other define syntax I need to do this full l1 I cannot
just do fold left I would clash with Scala standard from left and
in order to apply this a new comprehensive fold so to speak first
we need to apply it ordinary fold to the initial value and the updater
and the result of that needs to be transformed using the transform
so that is how we apply the phone there is still just one traversal
because there is just one call to actual fold this is the fold left
the foldable class has this hold help function so there\textsf{'}s only one
traversal and we'd like to keep it that way so there won't be any
even if we combine mini folds together it will just be one traversal
so how do we combine the fold while we do the zip and I just told
it to implement nice now the interesting thing that happened after
we added this transform a element to the case class is that now the
type parameter R is only in the covariant position so in the covariant
position the a of course it is a contravariant position here but R
is only in the covariant in the covariant position which means that
with respect to the parameter R this type constructor is a functor
so we use fixed values of other type parameters and only vary the
type parameter R with respect to that it is a functor so we just implement
automatically have a functor instance and now how did that happen
why is it a functor well strictly speaking formally speaking it\textsf{'}s
because the typewriter are only occurs in a covariant position here
but actually the implementation of map for this factor is very easy
you want to transform our to some T just modify this element compose
this function with the function from R to T and you get a function
from a to T and that\textsf{'}s it so basically the map operation is the same
as and then applied to the transform element of the tuple or part
of the case class so that\textsf{'}s why I did not write code here I just said
implement it works it is completely fixed by types now defining the
length and the some operations as folds now we have these type parameters
which I wrote down because you have to not on the right-hand side
but you need to write them on the left and so it\textsf{'}s a function definition
so now you see same thing except I put identity here because it\textsf{'}s
a transforms as identity I need to put it in let\textsf{'}s combine the sum
and the length like this we we do a sum zip line so that\textsf{'}s going to
be accumulator of this type and then we apply this function which
will divide the sum by the length and get the average so the test
works now actually this is still quite cumbersome to write this kind
of thing we would why can't we write just this we can we just define
syntax instead of writing this syntax that would allow us to write
that so here\textsf{'}s the syntax as an syntax extension we define all these
operations an arbitrary binary operation basically does this just
X\textsf{'}s if y exactly like what we did here and then a function and this
is the binary operation so in this way we can easily define all the
binary operations that we want and here\textsf{'}s the code now very easy to
write the double type parameters unfortunately are required or you
could put the type parameters here that would be less intuitive so
this is the way that we can combine folds in a single traversal note
that the same structure is required for scans of scan is like fold
except all the intermediate accumulated values are still kept in in
the sequence or in a container and so that\textsf{'}s how you would apply a
fold operation that we defined as a skin it is very easy because scan
left has exactly the same types of arguments and these two arguments
are what the fold structure gives and then you just need to map with
transform now unfortunately this is going to be twice the traversal
that we had because we do a scan left that\textsf{'}s going to create one sequence
and then we do a map is going to be a second traversal there might
be a way of avoiding it but that\textsf{'}s less important that\textsf{'}s certainly
just to traversals we can combine as many folds as we want in a complicated
way they're still going to be just two traversals so if we can somehow
refactor this to have just one traversal maybe by refactoring the
scan lift itself that would solve that problem that\textsf{'}s not the focus
of this tutorial so here\textsf{'}s an example so if we use the average and
do the scan instead of fold and you have all the intermediate averages
accumulated as you iterate over the list so finally I would like to
point out the difference between applicative and one addict factors
so we have seen that fold could be seen as a not function as a proof
factor and yet it has a zipper ball property and so you can merge
these folds into one now this merging is similar to component wise
information so for example average of sequins is a division of Sun
of a sequence and length with sequins and some and links are completely
independent of each other so that\textsf{'}s a component wise operation in
a sense and so that\textsf{'}s a click ative by our intuition now monadic operation
on the other hand would be such that we depend on the previous results
in order to compute the next iteration of the fold and so for example
we could compute a running average that depends on running average
that we just computed previously so we can combine folds together
but each new iteration would depend on the previous accumulated result
so that would be a monadic fold so let\textsf{'}s see how that works we are
going to continue to use this type constructor because it\textsf{'}s a functor
in harm and as we know a mu naught cannot be not a function could
not be contravariant or pro factor it must be a function so if we
want a monad so that there is a useful kind of flat map then {[}Music{]}
you need to use this we cannot use fold 0 that was not not fall to
fold 0 all right but actually it is more difficult than that because
when you combine two different folds the type of the value that you
accumulate must change it must be a tuple of values that you previously
accumulated let\textsf{'}s look at the type signature of the combined so you
see you combine I skipped over this but if you look carefully the
first fold takes values of type Z from your sequence accumulates values
of type a and returns results of Thailand a result of Type R the second
fold is a different accumulated type and so the final combined fold
needs to accumulate a pair of a B and it returns a pair of our team
and so this is not quite the same as the type signature of a usual
zip because not only this type parameter is modified but also this
one but this is necessary so it is a slight generalization of a zipper
ball or or zip or type type signature but this small generalization
is important and necessary and similarly for the flat map if we want
to define flat map it would have to change the type of the accumulated
value a so here\textsf{'}s what it would be if you want to do a flat map for
fold then it will take three it will take two two arguments one and
be some initial foldable then you take the result of that fold and
the function will compute some other fold using that result and then
you want some how to combine these two into a fold that returns T
but this fold certainly needs to accumulate a as well otherwise you
won't be able to get ours and are necessary for this one to work so
the resulting fold must accumulate here maybe there\textsf{'}s no way around
that so this is going to be again slightly generalized type signature
for flatmap where usually you would not see this these type constructors
would be absent and you would just have f of r r2 f of t returning
f of t but now you have to return this kind of thing and that is as
I just explained unavoidable so how do we do that well we basically
very carefully look at what is going to be computed when we apply
this combined fold so first of all what is going to be the initial
value of this type while the initial value needs to be of this type
or clearly we need to use the initial value of the first fold and
where will we get the initial value of the second type the only way
to get it if is when we apply F to something but to what we need some
R well the only way to get R is when you transform with the first
fold from some a and there\textsf{'}s only one a here which is this initial
value so f of this is the new fold that has initial value B so let\textsf{'}s
take that and put that into the tuple let\textsf{'}s now look at the update
how would they update work now the update needs to be of this type
clearly the only way for us to get a new value of a is to use the
update from the first fold so let\textsf{'}s call it new a now the only way
to get anything of type B is to go through this function f so we need
to apply F to something to what now clearly to some R and the only
way to get an R is to transform at the first fold and the only reasonably
correct thing to transform is in new a well could transform the old
a but then and this is not right because we want the second fall to
depend on the result computed by the first but this is debatable we
could have a situation when this could be a 1 here it just seems to
me that this is better to use an update in value and in a specific
application you want to see you this is so so this is a new fold so
so this is a new value of this type now what we need is a new B so
we need to update obviously this B we need to update with new fold
we get a new B so now this is the new alien new B that we accumulate
so that\textsf{'}s how the the updater works there\textsf{'}s just one place here where
we have some some some ambiguity some choice everything else is pretty
much fixed the new transform is going to be from a beta T and that\textsf{'}s
again pretty easy to understand that we have to get a new fold somehow
and the only way to get it to do this and then we transform with b1
and we get a tea and so let me return a fold that has this in it this
update and this transform so let\textsf{'}s see how this fold works let\textsf{'}s have
the running average of the running average so one way of doing this
would be to do double traversal which is actually quadruple traversal
since each scan is the two traversals but let\textsf{'}s ignore this for now
so it\textsf{'}s a double traversal so first average gives you this and then
when you again do running average of this another scan with the same
fold then you get this now let\textsf{'}s combine the two averages together
in the melodic way so how would that work we do the average one do
a flat map so X is this running average after transformation so it\textsf{'}s
a running average and for each running average return a fold that
accumulates that running average does not actually accumulate values
of Z from the initial sequence it accumulates this running average
for each fold and then we divide that by length so see this is how
we would write that and now a single scan gives us exactly the same
results so in this way we have combined two folds into one in a monadic
way we can use the for yield syntax the functor block which is easier
to read and that would look like this so X is the running average
from that accumulator is the result of this accumulation which depends
on X which accumulates the values of exit actually ignores the Z it
accumulates the values of X n is the running average or running value
of this fold so these variables are to be visualized as running values
of the fold and finally we divide this accumulator by very visual
and we have again the same result so I would like to emphasize the
difference between negative fold combination and magnetic fold combination
is that applicative folds cannot depend on each other\textsf{'}s intermediate
results but monadic fold combinations can that is the main intuition
behind understanding the difference between monads and applicative
factors the packet of factors Express computations that are independent
of each other there are kind of component wise computations but monadic
combinations describe computations that are possibly dependent on
each other\textsf{'}s running values so the previous value can influence what
you get in the next map now there is a library called origami that
gives you a monadic fold but they are actually not magnetic they're
moon advanced if you take a look at that you would see it says monadic
fold but actually they are not the same as what I just described they
must operate on a monad and their update has this type which means
that your updater includes a monad as a as a result and every iteration
of your fold must flatmap over this monad so that\textsf{'}s why I would call
this a mu not valued fold so the result of the fold is a monad value
and so every time you accumulate your result you updated you actually
have to evaluate a flat map on that moment so that\textsf{'}s a very different
thing which is useful in different ways it is not the same as the
Magnetic composition of food so take a look at these libraries in
more detail if you're interested and the final worked example that
I'd like to explain is the difference between applicative parsers
and one addict parsers so again this is really a difference between
how we compose these parsers together just like with folds how you
compose phones together you can compose them in a negative way so
when they don't depend on each other but you must compose them in
a melodic way when they do so the same thing happens with parsers
as I will show now parsing is a very big topic with a lot of different
algorithms and complicated grammars that you can parse in different
ways and I'm going to explain very basic things that you can do easily
while trying to do this on your own you will certainly run into trouble
but that\textsf{'}s because parsing is hard easy languages can be parsed easily
so that\textsf{'}s why I take examples of very easy languages for parsing so
my first language for parsing looks like this it\textsf{'}s either a number
so this is end of file or end of string either either a number or
it\textsf{'}s this HTML tag that I invented with a number inside and a closed
tag or or it\textsf{'}s several of these tags and there\textsf{'}s always one number
inside and the tags must be balanced there must be only one number
and the idea is that you take a square root as many times as you put
these tags so you evaluate this to a number by taking square root
as many times out of this number as you have the tags opened and the
tags must be balanced so here\textsf{'}s an error for example not closed not
opened open closed but then there\textsf{'}s junk at the end which is also
not allowed so those must be errors those must be signaled as errors
so here I just created a type for errors which is just a list of strings
and these are the typical errors that I wanted to detect tags are
closed now I'm not closed not or not opened or there\textsf{'}s no number and
so on so how do I even approach this situation well it\textsf{'}s a very simple
idea is that a parser is a function that takes a string and it tries
to get something out of that string and if it succeeds it gives you
a value that it computes and it also returns the rest of the string
that did not consume so it consumes some part of it and returns the
rest of the string here\textsf{'}s a type that could be used for very simple
situations like this where you take a string and you return a tuple
which is either of error and some result type and also unconsumed
remain remaining portion of the string now if you failed to parse
then you return error of some kind and if you did not fail to first
on your return the value that you found computed in some way and so
this is the first idea is that you will define the values of this
type and you will combine them and the language parser will be a combination
of these smaller simpler parsers so here are the simpler parsers that
I found necessary for this language the first the simplest parser
is that it\textsf{'}s the end of file so the sparser succeeds only when the
string is empty now that returns unit so it\textsf{'}s a parser with unit type
parameter it returns unit otherwise it returns on here so I'm going
to use that when I require that there should be nothing else anymore
in the string and the second parser I found you the necessary is the
one that actually gives an error if there is no content in the string
so it kind of says there must be content in the string otherwise it\textsf{'}s
not right then I have a parser for a number so I have a regular expression
I'm just using very basic tools it\textsf{'}s certainly not the best way to
parse a large amount of data quickly but I'm interested in the principles
of how this works soap our servant will take a string it will match
the regular expression on the string so I'm using Scala regular expression
standard library which operates on string as as if it\textsf{'}s a case match
and each variable here the pattern variable will be equal to the group
that is matched so after this if this is matched and I have this number
and I have the rest of the string and I return this integer and the
rest of the string and if I did not match then I say there\textsf{'}s no number
and I return the entire string I didn't consume anything so in the
same way I define other persons so for example open tag if it\textsf{'}s this
then I return that otherwise I'll return errors close tag all right
so these are my test strings now how do I combine parsers this is
the most important question so the first Combinator is the zip like
Combinator\textsf{'}s applicative come to enter so let\textsf{'}s call it like that
and the idea is that I will combine two parsers let\textsf{'}s say parser a
and parser B may combine them by letting first parser a parse something
then also letting parser be parsed remainder and then I gather their
errors together in an applicative fashion so if they gave a result
great I'll return a tuple if they gave errors and I collect all these
errors and if there are two errors from both of them then I collect
all the errors just like we did initially in the example with either
where we could define the applicative instance or map to by collecting
errors rather than stopping at the first error so that\textsf{'}s the implicative
Combinator or zip of the two processors however note that the second
parser depends on the first because the second parser runs on the
rest that is remaining after the first parser has run so there is
a dependence of the second parser on the first just that this dependence
is not in the type a it it depends on the other things on the context
and the string that is passed around invisibly invisibly that is to
the types a and B the type a doesn't know that there is a string being
passed around so from the point of view of types a and B these are
independent but actually they are not independent so the parser B
is run after parser a has has run however it is applicative nevertheless
because parser a failing does not make parser be necessarily fail
parser B could succeed when parser a fails and then WordPress really
could also fail and then both errors would be collected so that\textsf{'}s
the difference there\textsf{'}s also a monadic like Combinator that uses flat
map where we first do parser a when we discard its result and then
we do parser B so if the parser a fails then nothing will be done
so this will not collect errors from person B another important Combinator
that has nothing to do with monads or applicatives is this alternative
come to meter which is that parser a will be run and if it succeeds
parser B will be round but if bursary does not succeed sorry if if
parser a fails parser B will be run if parser a succeeds parser B
will not be run and so parser B is actually here passed by lazy evaluation
for reasons that I will explain shortly so how do we do that so first
we run parser a we get the result and the rest of the string if the
result is not empty then we just return that if it is an error well
that error is ignored and we try parser B so that\textsf{'}s the alternative
and when we define map and flatmap because parsers are functors let\textsf{'}s
check that is a factor there\textsf{'}s a parser type parser type is type parameter
a is here to the right of the function error so yes it is a factor
now let\textsf{'}s define the language the language is defined like this so
first of all it must be not empty so and this is the melodic Combinator
so everything will fail if this fails so if the string is empty then
everything fails right away nothing else is tried if it\textsf{'}s not empty
and we try to parse the number if so we mount it to double if it fails
then we try open tag and then again we use a monadic Combinator because
if that fails that we should fail that\textsf{'}s either a number or an open
tag and it should not even try every anything else so it what happens
later is again we use language your exits a recursive call open tag
can contain any other thing in the language it can again contain openText
so we do a recursive call and then close that now this is an applicative
Combinator so if this fails we can still check that this fails or
not and the result is then mapped into a square root because we have
here if this succeeded then we had a tag which is a square root tag
so we have to compute a square root so in this way we define just
the language of these tags and then the final parser is this recursive
language parts are followed by end-of-file and this is again an applicative
combination so we can detect junk even if this failed junk at the
end of file will be detected and then we map to the first value returned
by this so then it\textsf{'}s a parser of double finally we define the function
parse language it takes a string and runs this language parser on
the string and takes its value which is going to be either of error
or double and here are the tests it works for example parse language
of this is 123 first language of this is 11 square root of this first
language of that is 10 this is square root of square root these are
not closed so in their junk at end and here we actually find several
errors there\textsf{'}s not opened not closed and here is not opened not closed
so this for example we find that the tag is not open the tag is not
closed and there is a junk at the end so obviously we are able to
find all those errors at once so as a second example of a language
consider this a number surrounded by tags but now the tags are arbitrary
their tag names are arbitrary they just have to be balanced so there
it must be B and B C and C now this is a different language where
the parser depends on the result of the previous parse now here the
first row did not depend on the result because this is parsed independently
of whether it is inside a spirit or not this by itself is parsed and
gives you hundred regardless of where it is it does not depend on
that but in the second language that I consider that is not true because
this needs to be followed by C which so if they said if this were
C that\textsf{'}s an error so you need to know that you have first be here
in order to parse this correctly so parsing this depends depends on
parsing this and that\textsf{'}s why we need to use a monadic Combinator so
I again define a few different small parsers and combine them into
larger parser so I find two parsers that parse any tag and return
the tag name as the result and then a parser that takes a specified
tag that should be closed as an argument so now this is a function
that returns a parser and I'm going to use a flat map so that it is
very similar parsers before I have an end the file is error than I
person number and then if not a number then it\textsf{'}s any tag but the tag
that I parts it needs to be closed right here so in this way I depend
on the results of the previous bars so this is when an addict combination
is is required and here are the tests and just to highlight the difference
so we do find several errors here as well because we use applicative
Combinator here and here as well but now we find fewer errors at once
because for example here remember this incorrect input that was tested
here here we got that it is not opened not closed and junkit and here
we only see that it\textsf{'}s not opened and jumpa then we do not see that
this is not closed because the closing tag unethically depends on
the result of the opening tag which failed and so nothing was tried
so that the parser a little parser that was here never got called
and so it never had a chance to fail and to tell us that the tag was
not closed so unfortunately there is no other way to parse this language
in this in this simple minded approach because we have to use this
flat map we have to depend on the result of this in order to check
that this has the same name so this is a difference between magnetic
and duplicative functors applicative factors describe computations
that so to speak they they occur component by component independently
so there are several parts and each part is processed independently
and the results are all accumulated so map to visa is a good example
so you have several parts that are all processed independently component
by component and the results are put into another container or as
monadic combination means that each next step here depends on the
result of the previous step and that is sometimes necessary however
the parsers is an interesting example because actually as I just showed
parsers do depend on previous results because the parser that is next
gets the remaining string and so it depends on how much of the previous
how much of the input string was consumed by the previous parsers
and so for this reason even if we only have applicative parser Combinator\textsf{'}s
you still cannot say they are fully independent and so for instance
you wouldn't be able to parallelize the person but in many cases applicative
combinations can be parallelized as we showed in the example with
the futures so here are some exercises on this material so here you
should implement map to or IMAP to is appropriate for these type constructors
some of them are not funky and as I showed for the semigroup but you
should do the same for the Montoya to you find a typeclass instance
for the pair then you should define a monoid instance for the type
FS now this is s is another type parameter is a fixed mono ed type
and so it\textsf{'}s just applying a type constructor to a fixed type should
show that this is a monoid this was similar to what was done in the
previous chapter for monads but now for applicative so sure you don't
need F to be a moon and for this and some exercises for parser and
folding Combinator\textsf{'}s this concludes part 1 

all right so this is about goals and the main question has well I
will talk about rules and output about parsers platform is so how
would we compute standard deviation of this list of data well we can
compute it in simple way and beautiful ends so this is always the
length of this list don't give the average refused average of squares
and you standard deviation well this is a simple way of doing it three
men were statistic widths it to give the way of doing this which is
to introduce the creation factor which is n divided by n minus one
due to sample variance all right this is it now the problem with this
competition is that which reverse is list three times every time you
do a length sum or the map traverse this list so actually here matures
it four times because map traverses and then assembles of traverses
so the way to avoid traversing many times is to further wait these
operations as some kind of full bridges and so let\textsf{'}s look at the type
signature of fold Oh blood for example what is the type signature
of net it takes a value B IV and it takes a function that updates
the accumulated value and so it goes over your list starting with
the initial value v the initial value of the accumulator and then
for each value age from the list it holds this function to update
the accumulated value and finally you have the accumulated value that
you output so that\textsf{'}s the type signature of fold so you can do this
with a fold start with for example the computer sound sort of doing
data to the sum you say you fold left so but the Sun let\textsf{'}s put this
into your blog okay so how do we compute them in for example you fold
with initial value 0 and the function that takes helpfully tells me
what it takes so accumulator and bullets would X going to well I'm
going to add 1 to B that went to be the length so I can I can express
those things very easily some is this sum squared is this right RZ
inz right now however we still if we combine those were still ready
to have multiple universes so they are yet but we want to pursue is
that we want to have some way of combining these folds automatically
and the single universal so that we don't have to revolve include
of course just do more work and make a single fold which will do all
of this will accumulate a complicated round will have to collect a
lot of data you cannot just do it in simple traversal like this have
to accumulate the length separately we have to accumulate the sum
separately we have to avoid the average squares sum of squares separately
and the end we'll have to perform this computation so that\textsf{'}s kind
of difficult we'll have to accumulate a triple at least and then at
the end we'll have to do this so we could do that by him but instead
we want to write some code that will automatically combine folds like
these I wanted basically you'd be able to combine these things together
automatic so that it automatically decides what needs to be accumulated
and we don't want to see all of this so the first attempt to do that
would be just very straightforward so let\textsf{'}s look at what we want to
combine so we want to combine bowls so once at forward but one of
them that we render them by the idea is that this data data that you
have to pass before life that is what you want to come you want to
encapsulate this data in a new data type and combine those somehow
and then at the end he will pass the combined data to the faultless
be wonderful left in the end and that\textsf{'}s how we will accomplish we
want combined the day time that both left that\textsf{'}s the idea what is
the data that the form of days the data consists of two pieces or
two parts the first part is a value of type B the second part is dysfunctional
others therefore define a type it has these two values and call it
foap 0 it has a type parameter B and so it has initial value IV and
update function of type the easy one is who I call it Z just so I
need to type parameters actually right so this is the signature of
boatlift f of E and I have some type of their sequence element both
of them need to be type parameters now because I'm trying to generalize
it alright let\textsf{'}s let\textsf{'}s call this a instantly white hole it'll be okay
so this is the data that fold needs to perform its operation now we
can define a syntax to apply our to apply the full duration today
and so we want to want to be able to define values of this type and
fold with them and then we also want to combine them so let\textsf{'}s do one
first first our syntax to perform old love using and that\textsf{'}s obviously
going to be a sin tax extension right it\textsf{'}s going to be implicit class
hold you syntax some type parameter which is going to take a sequence
of a let\textsf{'}s say right let\textsf{'}s that\textsf{'}s miss good for a listing same time
take a list of a and define a function which is going to be all 0
only taking actually messy here take an age old hold 0 of Z of a Z
and the result is going to be a B right so that\textsf{'}s going to be s hold
left of fall 0 needs all all 0 and a done so now we have this syntax
and now let\textsf{'}s define these parents for example playing plan 0 links
in some type actually with but what type would be we fold 0 of some
Z and say double that\textsf{'}s not not very generic now that do need is the
odds be too generic it can be done with the generic and now length
of its list or double right it\textsf{'}s going to be 4 0 of 0 and simulator
bar or interview waiting plus one right so that\textsf{'}s how you find a phone
now we can apply data don't hold 0 of things that\textsf{'}s should be all
the thin I'm going to run this right now for the main errors okay
Sergei what you've done is are you taking this whole operation when
you were originally running as you look at the other nine you stored
that operation itself with case class they executed bigger right so
I started a time that the full apparition needs and my idea is that
I want to be able to combine that big so that they their presents
the folder hasn't yet been done right it\textsf{'}s waiting to be applied and
here\textsf{'}s I still want around this test ok right so so this is how I
will later apply these fools now for mine so what does it mean to
combine I want to have a full 0 of za I don't have a full zero of
B I want to get full 0 see what of a big maybe pair baby right so
if I can do this then I can combine arbitrary folds into a big one
automatically but I can just apply that that would be a single traversal
so let\textsf{'}s see how we can do this so they come that combine function
has a type signature that\textsf{'}s very similar to see lists so I call it
let me call it zip zip zero I have folds your outer for one later
I'm permits I hopefully everybody will show that because that\textsf{'}s also
quite interesting so what is M 0 0 is going to take folds here on
the a40 of CV but to give me from 0 of saying I need a parameters
C B right how do you combine that kind of thing well let\textsf{'}s look at
the type Oh so we haven't - for basically we have a slot for this
basically named to having two parts listen this so we have this for
a and we have this for me we have a we have this function and this
function for me right we need a B and if we seem to be right so we
just want to return new in it and new update and so this is going
to be new in it type a B and you update of this type and if we can
do this we're done we have combined how to combine this well obviously
so much we can do to combine this a and this B alright so that\textsf{'}s all
0 in it or 1 right Elsie\textsf{'}s latest is done now how do we do this well
each return a function that takes this and returns that bullets start
writing the function is see check the types now we need to return
a tuple of Av by using those we can get an aid we get an if you have
an N Z plus F 0 of AZ every one of these sorry F 1 of easy right what
is oh oh yeah f 0 update ok so in this way we have returned the correct
ID and we have used F 0 and F 1 somehow by combining them which makes
sense now is that really what we want well yes because how would this
updater work it will take the previous pair of a B it will update
the first one using the first operator and now then the second one
using the second nominator and it reuses the same C which makes sense
is the same sequence verb folding over so this code in cups awaits
the idea that we're folding two things at once but doing only one
traversal but let\textsf{'}s now define another fold so we define length so
far let\textsf{'}s define some H we know how to do let\textsf{'}s define some squared
which is going to be like this let\textsf{'}s use this zip zero to combine
what will happen some and length which is going to be super serum
of some zero length here and look at the type the type is correct
now it\textsf{'}s going to fold and produce a repair Dallas apply using zip
all right so I'm going to apply holes you on this sum and length and
that\textsf{'}s going to give me a tuple write us all is something length zero
and then I say sum divided by length five point five because that\textsf{'}s
the average of this yes okay so this is how we succeeded already we've
already combined two folds and there\textsf{'}s only one chairs of course this
is very ugly have to do all this and this two plane and all this stuff
and then we have still two separate calculation after this is inconvenient
but we would like the code to be something like this you know data
does fold and then some divided my ladies wouldn't that be nice we
could have code like this and then we could have here a larger computation
some example you know some squared minus some time some something
like this no why can't we do it like this we'll be great we'll be
very declarative you can't do it so the way to do it is to understand
first what he want this once you understand what she wants design
your declarative language then it\textsf{'}s always what you need to do we
already know how to combine both so all these need to be false you
just need to define the creations that combine them and at the same
time perform the final computation after the forward so far our folds
always will give you some tools he wants to combine the tuples in
the final computation into single value once we realize that we want
this the natural thing to do is to put that final computation into
the full data structure let\textsf{'}s define fold one so I'm going to define
fold one by adding final computation of type a to me let\textsf{'}s call it
hard to be more visual so a is accumulating our is result so we're
going to have a structure fold one that encapsulate s-{}- both the
folding and the final computation after falling so now if we combine
these folds then this a could be a complicated tuple type but this
result computation will perform all this extraction out of the two
whole automatically and here\textsf{'}s single value that you want so in this
case we will be able to implement things like this and that what will
be remaining is just indexed we find syntax for this which is reasonably
easy so this is the main idea of how to combine false good thing else
is now just implementation we need to define again sin tax reform
for left we need to define the same things define the zip and need
to esta let\textsf{'}s work let\textsf{'}s just that anything so for one syntax or have
an old one right now old one still does the same things as anything
has updated but then has an extra type track which is going to be
any part right for the result and the result is going to be its own
either side round here and the result is going to be old one the result
of this and the this is of type our okay so everything else follows
pretty much so double-double int chickens well actually I can do this
or I can just say it\textsf{'}s an instant I accumulate and then I have okay
so let\textsf{'}s check that this works actually the things won't compile them
know yet but right now onion is except one which under to define and
okay so let\textsf{'}s do what\textsf{'}s story I remarkable I held upon a needs if
one here he\textsf{'}s someone length warm and is it ones to be old he paid
are one let\textsf{'}s call it a 1 R 1 with a two or two it\textsf{'}s going to be a
1 into R 1 R 2 now the zip still needs to needs to do this needs to
take a to talk and automatically create for me the accumulated type
which is going to be a two-fold will not use to kill plate and the
result type which is good to use it to turn right now let\textsf{'}s implement
the zip it\textsf{'}s a little more complicated it\textsf{'}s basically the same code
as we have the previous tip except we're going to have a transform
a deal right so it\textsf{'}s exactly the same game same code will work except
I need to do a new result and so new result of Type R 1 or 2 is going
to be an f0 result sorry what doesn't he result not the right type
and it\textsf{'}s going to be in one it\textsf{'}s you going to our 140 so it\textsf{'}s going
to be kiss anyone to going to have zero result one diamond one is
not to that you remain right done now just for your reference all
of this I have I have implemented all of this automatically another
another file which I'm not showing I have implemented this is zip
function on Z automatically using a quick outline room keys of marinus
basically implements functions based on type signature and the type
signature of this function is sufficiently respective so that only
wild reasonable implementation can implement is this Margaret I have
very Howard library anyway let me just show you where this this one
so I could just implement it it implemented hold this automatically
this code is in some sense were late it\textsf{'}s boilerplate in the sense
that the types we paid what I have to do there\textsf{'}s not much choice alright
let\textsf{'}s check that this works I'm just going to do some some squares
I need to have more type parameters here or one need add identity
are not transformational one all right done of one some live one and
see this works due to the test name this man in yes alright so it
works but now we can do much better we can actually define average
is it for I'm going to skip the types actually these types are not
necessary if I do a sip one of someone length one and then I what
I want is I want to add a transformation to this alright so how do
I add a transformation now I I can the type of this is this so I could
have a transformation I will take this result and divide it made the
final result let\textsf{'}s have a function of as a transformation so let\textsf{'}s
do syntax extension we have a full one syntax yes where they have
one x what doesn't matter what that means okay so I take a fold old
one so I need these type parameters and I returns something that has
a committed which is going to be and then some function are going
to see is name for one see 80 which is going to be full one don't
open result falls for one result and then okay so that\textsf{'}s going to
be my crew so I'm just I'm going to keep the fold as it is except
I'm going to change its result by a new function f so now I can do
and then some length points with some dividers what is wrong ooh a
hole in it oh because I already defined some but there\textsf{'}s a type of
this this is a two-fold why is that a tooth oh oh yeah because it\textsf{'}s
I can't I have to do in case that\textsf{'}s right okay it\textsf{'}s done so now I
can do this I can just compute this image it is still a bit verbose
right if you have to do all this stuff so in order to make it less
if it was is what you can do let me just show you the code on the
length of time here\textsf{'}s what I don't do I can I can do it there is magic
extension on the phone it\textsf{'}s exactly the same hole except now it\textsf{'}s
permit rise on numeric result and if the result is numeric I can do
arithmetic I can do plus minus and the result is that I can have I
don't have food like this where I directly fold all the one using
this as my and the Nationals and the result is so the syntax is now
almost like we wanted except I have to put the double in here because
in this code it\textsf{'}s completely genetic it\textsf{'}s a generic over numeric numeric
item in whereas in my example quote here I have double everywhere
explicitly so I don't need any type parameters but if you want to
have your code generic then that\textsf{'}s what you need you define your syntax
which is so all this code is going to be in the library easily and
your your user calls upon me like this so this is how you combine
folds into well you can also do you can also use can left with the
same data because scan is like a fold except you keep all the intermediate
accumulated values and so you can you can do scan left or the fold
data and in this way you can just use these fold things as the control
your scam in traditions in so for example you can do an running average
running average is this so I'm just doing a scan with this average
one with a defined can I get this so now if I combine several folds
into one I have put them into the argument of this scan one and there
will be one traversable technically speaking there to traversals with
mmm but I think it really and it there\textsf{'}s only one logically speaking
there\textsf{'}s only one fold with your appliance folders come together and
it encompasses the entire computation so I could express all kinds
of things like standard deviation like that I can also define pure
for fall so for example if you look at my plan here not sure if I
don't have time to go to do this this computation has Falls but I
might need to have constants as well for example a standard deviation
here requires this constant one so I might need to have a constant
as full fold that always returns one so I can define that but me we
do that here in a chamber or one so I headed a transformation defined
and constant old so this is what we call this pure one it\textsf{'}s going
to be some X and it\textsf{'}s going to give old one are see our heart so this
fold one is going to be starting from X 1 to have function that just
gives me X and it\textsf{'}s going to have a function that returns an X whatever
it\textsf{'}s it\textsf{'}s whatever I want I just want to return X I don't care about
any of a connection so if I have a phone like this then I can say
for example here of one or pure of two and I won't have a full that
always returns that constant and I can combine that with other fools
way that I combine for it is in the zero and so then I could amend
the computations isn't one let me show you okay any questions at this
point so we have basically achieved this syntax yeah basically achieved
our goal we can compute like this but I would like to show you now
is that actually this fold is a moment in lavender arm and so you
can implement in in a sense it\textsf{'}s frankly speaking it\textsf{'}s kind of a more
not I you can implement flat map and the only problem with flat mount
is that flat map could change the type both there\textsf{'}s some other folks
and so the accumulator type needs to change now you might have to
accumulate more than one value so a flat map they can in the usual
way does not allow you to change other type parameters it changes
one type parameter but not others it would have a flat map in the
usual way to take a full of art and function from R to full of T and
you're returning full of T however it won't work because you need
to accumulate more Nathan so you're forced to have a different type
parameter and there\textsf{'}s an inner zone so this is not really implement
in a sense and it is a generalized kind of that but the flat map is
a flat map so it\textsf{'}s usable in exactly the same way so after I implemented
this loudmouth which again it\textsf{'}s implemented kind of automatically
just try to try to see what what needs to be done and now first hit
accumulate the first fold and then you transform that when you take
the army apply to function you take the second fold accumulate that
is it\textsf{'}s kind of automatic so the result of this of assure you this
code very interesting-looking I can now write fold combining code
in this syntax now these are different combinators them zip because
they can depend on each other so first I take the average X is the
running average and then I can define a full that uses that X to accumulate
that running average in another accumulated and then I also have the
length which is a previous fall and then I divide this by them so
this gives me a running average over a running average I have a first
Pole and the second fall depends on the first event on the results
from the first I remind you that average or way to compute the result
divides already sung by the ladies this is already complicated for
I'm combining it with other tools so using this syntax it\textsf{'}s much more
visual what I do the result is a fold and if I apply this fold to
a list I get this which is you see it\textsf{'}s growing much slower because
it\textsf{'}s a running average of the running average so in this way I have
combined folds phonetically which means the second fold depends on
the results from the person using zip I combine forms in a way that
don't that they don't depend on each other they all running in parallel
from my signal when I combine them using flat map I can't make them
run so to speak sequentially later for and now this can be an arbitrary
updater but it depends upon the results of the previous fold an arbitrary
way this is this X is not the accumulated value inside this form is
the result is a running result after transformation it\textsf{'}s very powerful
way of combining computations and the result of this is still just
full it\textsf{'}s not yet running it\textsf{'}s just a fall I don't care about its
type here it\textsf{'}s different from zip so saving is duplicative convolution
and for a flat map is the melodic information so this illustrates
the difference between applicative and all not a complicated combination
means that the structure of the computation doesn't depend on previous
results and phonetic combination could depend and this could be a
different form each time depending on this X so that\textsf{'}s a much more
powerful way but yet and yet I have a single traversal so you see
this food inside it there is just a single big updater of motion that
is going to be substituted into the scan left it\textsf{'}s going to be just
a single Traverse with that the beta version and all of that is automatic
all right now I'd like to leave time for questions anything here needs
to be clarified let\textsf{'}s see the code for a cold one again yes for one
is this so it is a initial value of the accumulator it\textsf{'}s the updater
function and the final transform from a college transfer from a to
R so if you look at this pipe constructing the type parameter R is
in a covariant position but the type parameter a is not so it\textsf{'}s not
that functor if you spective are so sorry with respective a so it
isn't funky with respect to R so we could make it it will not only
with respect to honor so unless we add the transform here we could
lose him like we did with full zero not possibly do flatmap is a the
typewriter a occurs in a contravariant position here and the coherent
position here so zip can be implemented nevertheless zip does not
require being so it can be implemented many places but so see I'm
using automatic implementation for words all that I'm using also this
syntax or factor instance this is not not very important for this
you like if you want to know about this asking basically this has
a functor instances of one of instance but only the monarch with this
I change so the time traveller in the middle used to change otherwise
this given to us by Allah so there are libraries but implement these
foods to rest on unfortunately none of them were published in a at
the same time they're very small there\textsf{'}s not much code rather more
more than I showed here but much much more than this there are some
convenience is there are some predefined falls from the Marek data
basically it\textsf{'}s not a lot more functionality on this there are maybe
some more conveniences there are some libraries in differences I don't
think it is in the standard of living of any kind yet not sure something
in it so there are libraries I would say well maybe there aren't so
many use cases where you want to combine these fold but I think it\textsf{'}s
instructive to look at this to get an idea what you can do the basic
idea here is that we wanted to combine communications wanted to do
that you first define a data type that represents your computation
but does not get performance so your computation is fully defined
and specified but not yet performed when you define combinators from
that so you can combine your computation you can combine them as negatives
whereas once you have a pure flatmap have map yep zip so you can combine
your computations the variety of ways and then at the empty run once
you have combined all this your code is done at the end you run the
computation so you can organize your program as these computations
that are really data structures that are combined in flexible ways
but you run it at the end so that\textsf{'}s that\textsf{'}s the main idea and that\textsf{'}s
what enables is a musician all right anybody if remote having a question
alright so that was mine folks I didn't get to talk about our things
but that\textsf{'}s that\textsf{'}s quite seen in in spirit you present Mercer as a
data structure and combine them and you run my question is how you
combine them and the order to use and that you look for civility of
zipping them together or flat map and you know it you know how to
do with those once you find those we thought we have a pure and they
were flat nothing in the map I was here and once again we find that
which is combined in a very flexible way combine all of the divisions
we have a commune specific language of sorts and then you run it all
right well we're out of time thank you very much 
\end{comment}

\begin{comment}
this is part 2 of chapter 8 continuing applicative functors and pro
functors in part 1 we looked at practical examples of applicative
factors and pro founders and their use in part 2 we concentrate on
the theoretical properties of these factors to begin recall that applicative
factors have the map to operation but also they have map 3 not 4 and
so on do we need to define them separately for all n map in which
would be unfeasible perhaps or can we have it in some other way and
the answer to that question leads us to an operation called app before
we look at the properties of this operation let\textsf{'}s try to define map
in on a specific function either of string a so consider this type
constructor which is a factor in a and let\textsf{'}s try to define map in
and use it let\textsf{'}s see if we can do something better than just doing
up in here is map 2 it takes top of a up of B it also takes a function
from a B to some Z so a B and Z or arbitrary types parametrized here
and we return up of Z so how do we do that well for an either that\textsf{'}s
straightforward computation we match if there are two left we use
the monoi against a some string we were just concatenate the two strings
and otherwise if one of them is left one is right then the left remains
because that\textsf{'}s an error of some kind so we propagate the error and
only if we have both in the right and then we can apply the function
if so then we return the right of f of X 1 X 2 so it\textsf{'}s obvious how
to generalize this function to n arguments instead of 2 but the code
would have been very complicated to write and we'd have a lot of cases
so one solution would be to use a list of arguments and record over
it in some way and the second solution is to use curried arguments
which is going to lead us to the app method so let\textsf{'}s look at the first
solution so this map in one takes a list of up aim and returns an
OP of list a because can't without we don't have a function that\textsf{'}s
just instead of a function of n arguments let\textsf{'}s just put all these
arguments into a list so that will be sufficient we can always add
a function after that so we can see how that works if there\textsf{'}s an empty
list we go into an empty list and for a list having a head and tail
we do map to his head and then the rest we use map and one recursively
and the function that maps to uses is just appending X which is the
head to the list T which is in the tail after we already did my own
one so we have an optimist a so T is placed and X is an a so that\textsf{'}s
working but it\textsf{'}s not very great it\textsf{'}s still quite clunky to use because
then we would have to have a function that takes a list of arguments
so using this in practice would be quite inconvenient so let\textsf{'}s try
to see how we can do it better and the starting point is to define
a current version of map to lab two takes two arguments like this
and the function f takes two arguments like this now if we instead
use the curried version then a function f would have a type signature
like this so we would instead of taking a tuple of KB we will take
a and return the function that takes B and returns Z so that\textsf{'}s equivalent
but it allows us to do an interesting thing namely we have now a function
from A to B to Z and instead of taking a pair of a B and returning
of Z we also carry that so we take up a and return a function that
takes open B and return of C so that\textsf{'}s clearly just a rewriting of
map to using a different APA a different type which is equivalent
to the previous type so there is nothing new really except for this
current and a function type instead of unhurried now f map does not
have this type signature but here\textsf{'}s the definition now let\textsf{'}s also
introduce this syntax which will be more convenient remember we in
the short notation we often use this kind of notation when a function
is f mapped over a container so then we can just define that with
this index it\textsf{'}s just to write this instead of that so now once we
do that let\textsf{'}s apply F map to F of this type so why does F may not
give us this why do we need F map to that\textsf{'}s the question well F map
doesn't map this to this what does it do f map would take this which
is of the form a to something and after F mapping we would have op
of a to op of that something so the type will be like this OP of a
going to op of B to Z and that\textsf{'}s not what we wanted to have over here
we wanted above a going to op of be going to off of the so that\textsf{'}s
why F map doesn't work this way and we need F map - so what is missing
forgive me to be able to work this way so what is missing in a map
that F map - does here well clearly what is missing is to be able
to transform this into this so this is the transformation that f map
doesn't have but I've mapped two hands somehow AK map - already does
this let\textsf{'}s denote this transformation by app so this is how we usually
define F so this is the transformation that if added to F map will
give us F map - well we can define this transformation for either
just in the same way most more or less it\textsf{'}s a slightly less code to
write but basically the same code just there there\textsf{'}s one fewer argument
because the function f isn't here see the function if would be in
the F map so by considering app we have simplified our life F map
to has two concerns it takes this F it needs to map it to be and also
sorry I'm looking at that map I've map - it has two concerns it needs
to do this thing with two arguments and also it needs to apply F whereas
map only has one concern it only disentangled the two arguments somehow
and f is handled by map so let\textsf{'}s define F map - through app and F
map to see how works now for convenience that\textsf{'}s defined in fix syntax
for help which will be this it\textsf{'}s just so that we can write we can
write fa b star fa instead of up if i be okay so that\textsf{'}s just syntax
it doesn't change what this function does now with the syntax which
is defined here we can now define f map to via app and death map so
how do we do that well we need to return this function so we take
open up any take open B we need to return an OP see so let\textsf{'}s first
apply F map to up a so f has this type signature applying F map of
F to pop a gives us up of B to Z because F maps a to a function of
time bitters so now suppose we have this X so we could define it like
this now we can use app on that X and transform that X into this so
app of X of app V would be of type opposite so this will be out of
this of OB now that\textsf{'}s kind of clunky but if we write it in this in
fix syntax we write this instead of that and we write this instead
of that so in the Scala syntax these two operators will associate
to the left because the only way to associate to the right is to use
that colon as the first character of the special syntax of the of
the operator so therefore we can simply write this instead of F map
of f of opa of app and so on so the syntax is that first we compute
this so it\textsf{'}s associates to the left so it\textsf{'}s as if we had parentheses
around this around F F map OPA app OPB so and this syntax is very
similar to just as if we had have been able to directly apply a function
f notice this function has this type as if we could apply it directly
to OPA and OPB although we can't because F needs an argument of type
a and this is an Okie of it so it\textsf{'}s a functor of a so we can't directly
apply F to OPA this would be incorrect in terms of types so with these
special separators we can do it now so here\textsf{'}s how we can define F
map 3 OPA will be be obviously going to this what\textsf{'}s the final map
for like this so you see with this weird-lookin syntax we can define
f map 3 of map 4 and so on in a very easy way and we can actually
use that directly we don't need to define f map for and call it because
this code is so concise already so let\textsf{'}s see how we can just use directly
these operations now I just remind you that these operations are f
map and app they're just written in an infix syntax there is nothing
new about them or just F map and app so having these operations let\textsf{'}s
have a little test and do the safe divide so we divide by a number
but if that number is 0 we give an error message so if if it\textsf{'}s not
0 we divide so imagine we have a function of this type double to double
to double because now we need curried functions so now we can just
write code like this F safe divides 2 1 safe divide for 2 and another
test is we want to create a validated case class so what we need to
do is to have a current class constructor now the ordinary classical
structure would be uncared it will take a pair of arguments what we
need is 2 so this will be C to apply so this is the ordinary class
constructor with two arguments and dot curried is a standard method
on a function in Scala so this is in the standard library and the
result is a function of type double - double - C - which is a constructor
of the type C - but it\textsf{'}s clarity so now since it\textsf{'}s carrot we can just
use like that and we can use it like that so in this syntax we don't
need to call map - for example directly we just write if these are
directly applicable this is 2 divided by 1 this is 4 divided by 2
we could just implement safe divide as a syntax as well if we wanted
to and then we would have code that\textsf{'}s maybe easier to read in any
case now we can see that using the method app and in the syntax so
it\textsf{'}s basically app which is defined as this kind of function it\textsf{'}s
it becomes easier to write code with the clickety factors so this
app seems to be an important method that is simpler than F map - or
map - it allows us to define nap 3 in that 4 and so on and actually
it allows us to write code quicker without explicitly calling those
earth map three if not four and so on so basically what we have found
is that F map - can be defined through app if we have app using this
code and we just use F map F and then we apply up to the result that
was our code essentially defined first we do f9f of this and then
we apply up to the result to do this so that\textsf{'}s what we just found
now can we also define app through F map - yes and the way to do it
is to set a the type parameter a here to be the function B to Z we
are allowed to do that because these parameters are arbitrary so we
can just have a special case where a is equal to that and then we
here we would have a type like this be to Z going to be to Z so we
have an identity function of this type always and we can just apply
F map to to that identity function and the result would be a function
of type FA - FB 2 FZ but a is B - Z so the result will be of type
like this and so that\textsf{'}s just F map to applied to the identity function
of this type now because we have defined the type parameter a through
B in Z we have one fewer type parameter in app only two type parameters
in app or as is in F nap - we had three type parameters so it is in
this way that F map - is more complicated than the app so actually
F map to an app are computational equivalent we can define F map to
through app we can define up through a snap - and this diagram illustrates
how that works the type diagram starting from FA go by F naught F
F lowercase F has this type so if mapping it over F a gives you f
of B to Z now you app that you get this function of B to F Z or you
can directly F map to from here to here and that should be the same
and so clearly app is equal to something like F map - of identity
if this is identity so if we take F to be identity then this arrow
is just identity here so these are the same and therefore then app
is the same as f map - over identity so this diagram shows you at
once the two equations the expression expressing of f map to through
up and the expressing of app through F negative and similarly we can
define F map three of my four and so on and here\textsf{'}s a diagram for ff3
it is slightly more complicated because we need to first map to this
then we need to do app with these type parameters and then we need
to do app on this argument we need to take keep this argument constant
and app on this argument with another cz so in order to do that we
need to f map in other words we need to lift up from its ordinary
type which which is this into this type which is kind of a reader
monad with this as the environment so that as a reader functor F map
yeah so so this diagram shows how f map 3 is really defined but in
the code we don't need to worry about this because it\textsf{'}s automatic
because we already carry this argument so we don't need to explicitly
F map when we use the in fix syntax which we have just seen in the
code I would like to call your attention to this pattern that we have
seen before that we have some kind of equivalence between two functions
or two methods and the equivalence works by taking F map and and composing
it with one of those methods and usually one of those methods is a
natural transformation and another is a kind of lifting so we have
seen this pattern several times before where we were able to get two
functions that are computationally equivalent but one is simpler than
the other because there\textsf{'}s this F map but one of these functions already
does and the other doesn't so this function becomes simpler it has
fewer type parameters and fewer arguments so let\textsf{'}s recall the zip
operation that we have seen before and let\textsf{'}s see how that operation
is related to map to to find that and also let\textsf{'}s think about these
two types that are equivalent and their equivalents is given by the
curry and unclear methods in the scholars can standard library as
we have seen so we can take this F NAB tool that we had in the previous
slide the sect map to and we can unclear it and if we incur it will
have a type signature like this which is a tuple and here is also
a tuple so now let\textsf{'}s do the same trick we did in the previous slide
when we substituted an identity function into this F map to in order
to obtain a simpler natural transformation so in this case the identity
function will be of this type rather than what we had before we had
before was this type but we carry no sorry we uncaring now so instead
of a function type will have a tuple type like this so let\textsf{'}s let\textsf{'}s
do that and the result will be we have F map to with this now we take
C in other words equal to the tuple 8 or in B so we set the type C
to be the tuple a B and then we have a function of this type which
is FA of be going to F it'll be because C is tuple a B now this function
is called zip because it\textsf{'}s a very similar type signature to the standard
function zip different sequences where you take two lists you zip
them and you obtain a list of pairs so here F is the list type constructor
so then zip function can be seen as taking two lists of maybe two
different types not necessarily but maybe and returning a list of
pairs so let\textsf{'}s check that again zip and F map to our computational
equivalent so we define zip like this can we define F map to value
zip yes all we need to do is we first need to do zip that will give
us this and then we need to apply F F to F which will take F a B into
F C so this is the type diagram that illustrates this so we do is
if we get that and we apply enough map F and we get F C or we can
go directly from here to here using F map 2 or F which is the uncut
version so this diagram at once expresses the equivalence of the two
equations here because you can take this to be identity this function
and when we do that these two become identical because F map of identity
is identity by the factor law therefore these two are identical and
so zip is equal to F map to all that F of the identity so that is
this equation and otherwise you get the second equation so we can
think now that applicative functors that were initially defined as
something that has mapped to could be equivalently defined as something
that has zip of course work or something that has out there was a
roster equivalent and certainly laws should apply and hold but we
will look at Louis very shortly we could call factors a pebble if
this zip function exists for it and that would be actually weaker
than an applicative as we will see but there doesn't seem to be like
a good name that everybody uses for functors that just have the zip
method and nothing else there are typeclasses for that in different
libraries are called differently for example in the cats library it\textsf{'}s
called semi Drupal in this ecology library it\textsf{'}s called apply or something
like that notice also the same pattern which is a natural transformation
is computationally equivalent to lifting and the second one\textsf{'}s certainly
needs to be demonstrated rigorously we have not done this in the slide
here also have not done this we have indicated that they are defined
through each other but that\textsf{'}s insufficient to show that there are
computationally equivalent you have to show that this is actually
they're isomorphic in other words if you take app define F map to
from it and you take that off map tool and define a new app from it
then you have to show that that new app is the same as the app you
started from similarly here you would need to show that if you take
a zip say and you define as map tool through it and then you use that
F map to to define a new zip then that new zip will be exactly the
same as the own zip that you started with and you have to also show
that if you start from F nap to and go in the other direction F map
to define zip you find new f-type to ensure it\textsf{'}s equal to the old
F nectar now these proofs are very straight forward and we have seen
one of these proofs in a previous chapter in detail so I'm not going
to go through them again especially since it\textsf{'}s exactly the same pattern
where two functions are equivalent and they differ by applying I compose
a composition with F map F so this is a pattern that we have seen
time and time again and so it\textsf{'}s sufficient right now to recognize
it and understand that the proof is exactly equivalent analogous to
the proof we have seen in chapter 7 I believe where this was written
out in full is this equivalence proof finally we can also ask are
the operations app and zip equivalent they seem to be two sides of
the applicative coin indeed they are in order to figure that out let\textsf{'}s
remember that we started out by setting a to B to C function so let\textsf{'}s
do the same with zip we will get this transformation this is what
zip does now can we get an FC out of this and if we get if we do then
we will get an app because app is basically this going to this going
to FC so we could just carry this to get the correct type signature
for app if we could only convert this into an FC now obviously we
can convert this to FC because this is a function from B to C and
we have a B so you can clearly just apply this function to this B
and get a C let\textsf{'}s call that function eval which is this function and
then f map of eval will be a function of this type it is trivial to
define that function then we just do F map of it and we get this so
then we transform from here to here and we just uncurl we get an app
ID write it with two piece just so that it\textsf{'}s different name from app
but this is just the uncured version of app so the the uncured version
of app is equal to zip composed with F map of evil so we see young
exactly the same pattern that app is a zip with some F map although
it this is not an arbitrary function it\textsf{'}s a specific function but
it\textsf{'}s a very similar pattern and so it suggests computational curveballs
so let\textsf{'}s try to define in the other direction how to define zip through
app well that\textsf{'}s done like that so app and its functions that operate
on two different types instead of zip let\textsf{'}s prepare for this we would
need a function that makes appear out of two elements so let\textsf{'}s call
this function here and f map of parallels just denote f map like this
for brevity which I have already done in in previous chapters seems
to be a good notation and then clearly we get this kind of situation
if we apply this F map of pair to some F a of type F of a and we get
this and that is something that now resembles what app likes to take
as a as an argument so then let\textsf{'}s see zip of two different values
of a and F be you'll be equal to app applied to a pair so app has
this type signature we need to give it a product F b2c is this where
C is a to B and this is FB so then applying that app we get F C and
C is a to be so that\textsf{'}s exactly the right type namely F of a times
B so C is 8 times B knotted with sorry C is a times B and so we have
expressed zip through F and we have expressed up through Z and here
is the type diagram we need to take care about what types were using
so that\textsf{'}s why I have written about all these types in foam here the
type parameters for app for example you see app is defined like this
but now we need to use as its first type parameter we need to use
actually B to C so {[}Music{]} actually you know that is a mistake
sorry this needs to be deleted a because it\textsf{'}s not a B to C B this
is AB BC because this is how I defined it so this is just a PC I will
correct this slide so we start with this type we do a zip on it we
get this and then with your F map of eval and then we get F C or we
start with this type we do an app on it which is going to give us
F C directly so I need to delete these two symbols and then and also
these two symbols and then this diagram will be correct and these
type others are correct so clearly we can also do this with carried
arguments we just need to define current versions of zip which which
is like this and then we have exactly the same relationship between
up and fzp so I just call this F zip where it\textsf{'}s great and then f z
PQ is that and it\textsf{'}s exactly the same except we don't need to do the
two plane of arguments that we just need to use the arguments one
by one and now we see that this function takes this argument and this
function also takes that argument and so we can carry that away we
can omit the argument Q and we can write it like this which is nice
because it allows you to reason quicker about what is equal to what
you would have to write fewer symbols you see we can also omit the
argument P because this is now app of pair of P which is firstly apply
pair and a new clean up in my notation let\textsf{'}s function composition
in Y notation goes left to right and here are the explicit type parameters
that you need for this to work having looked at all this we still
don't know what the laws are now we have mapped to we have app and
we have zip let\textsf{'}s now derive the laws for these methods the motivation
for laws comes from our initial idea web map tool is basically a replacement
for a monadic block or a functor block with independent effects and
in other words map two with two arguments and a third argument which
is a function replaces the functor block of this comment so we started
with this we expect their fourth that whatever laws of the Monad hold
for this kind of construction should also hold for map 2 therefore
we just will take moon at louis which we considered in in the previous
chapter and write them replacing the functor block with map to where
where we can so the first two laws to consider are naturality laws
they come from manipulating data in one of the containers so for example
we can manipulate first in this container and that should be the same
as if we acted with the function f on X like this so if we rewrite
that in terms of map tool when we get on the left we get this map
to of this and this and of G on the right we have map to of this and
this and a modified G which takes X\&Y and returns this and similarly
if we apply map on cons to instead of count one we will get the right
natural T law which is like this very similar except acting on Y instead
of acting in X here also the monads have identity laws and associativity
laws so let\textsf{'}s look at those the associative law is that we can inline
the generators in a four yield block or a functor block that is the
right hand side and that should be the same as when we in one when
we write all of these in a single flat for yield block and we can
inline in two different ways and that should be the same so that that\textsf{'}s
the associativity law for the Monad now if we really write that so
the idea is that we have three containers and we have a function of
three arguments let\textsf{'}s say and then we first go over the first container
and then Y Z go over the second container now we need to do this rewriting
because we we need to formulate everything in terms of for yield blocks
with just two lines because that\textsf{'}s map to and for yields blocks blocks
that don't depend on containers don't depend on each other on previous
owners so this is a map - and then this is also a map - so we have
a map - of count 1 and map - of Constituent 3 and then the function
that just takes two arguments puts them into the tuple that\textsf{'}s the
Scala syntax for that and then the function that takes X \& Y Z and
returns G of X Y Z so we need two unto pole like this as the result
of map tool will be a tuple on the right we first apply the for yield
block to container 1 and 2 and so this becomes mapped to of point
one con 2 with the two-point function and then we do map 2 of this
and country with this function which now has two n tuple X Y first
so therefore this is the associativity law for map - let\textsf{'}s consider
the identity laws identity laws expressed in the relationship between
map 2 and the P were united so it appears that applicative factors
also need the pyramid it\textsf{'}s very useful to need to demand that the
pure method exists and satisfy the identity laws turns out that this
constraints so negative factors in an interesting way a useful way
without identity laws they're just too many ways in which you can
define not to satisfying just associativity law so the identity laws
are that if you do a pure on the right hand side that\textsf{'}s the same as
if you just had X equal to a because that\textsf{'}s an empty context or empty
effect so then we have the left and the right identity laws because
the pure or the empty context can precede a generator or it can fall
it can follow a generator service on both of these cases we have this
equivalence and so these two laws are written like that so for example
here is a map to of pure a and the container and then the function
G and on the right is just a simple container map with this function
that substitutes a instead of X and similarly the right identity law
so now let\textsf{'}s derive the laws for zip the reason we want to do it right
away is that these laws are quite complicated so these are very complicated
combinations are all kinds of things that are complicated so basically
map two of conto on map two of County two point three that is simple
so you just change the order in which you apply map to but everything
else is just some kind of bookkeeping that we should be able to get
rid of and simplify somehow so for example we don't want to talk about
arbitrary function G in the law we'd like to simply find that so that
law doesn't contain arbitrary functions so we have seen before that
this kind of simplification can be obtained if you instead of lifting
as you consider natural transformations and so that\textsf{'}s computationally
equivalent but the laws are much simpler we have seen that before
that\textsf{'}s for example the laws for flatten in the Minard are much simpler
than the law for flat map a lot of point but it\textsf{'}s the same thing will
happen here and it\textsf{'}s reasonable to expect that this will happen so
let\textsf{'}s do that and derive the laws for zip so what are the natural
T laws now naturality laws written like this we here needed to deal
with this kind of function where we modify the function by taking
some arguments and so on now it\textsf{'}s not very nice to reason about such
functions so I would like to introduce a short notation for that the
source of difficulty or the source of lack of elegance is that we
need to write these arguments and then all we do is we put these arguments
right back into the function G except one of them gets modified so
whatever we need to modify both so that\textsf{'}s something we would like
to be able to write more concisely and so that\textsf{'}s why I introduced
this notation so first I need use this function product notation which
is a function on a product defined like this so this is just syntax
so to speak and this function is trivial and doesn't let me do very
much but using this notation now we can do a lot of interesting things
for example we can now rewrite the laws from mapped like this F map
tool is just map to where the first argument is the function with
this type and the second argument is a tuple is like as a pair just
swap the arguments as compared to my job now this is the short notation
for first container dot map F you see that then it\textsf{'}s a map to with
this tuple so we read that as a product like this and on the right
hand side it\textsf{'}s a lab two of that tuple unmodified but the function
G is modified in that the first argument gets acted upon by function
f and this is expressed like this in the short notation so we have
a function composition on tuple so this is a function that acts on
the tuple a B and this function applies F to a and identity to be
so that\textsf{'}s a function that takes a tuple a B and returns a tuple F
of a B and then we apply G to that therefore we get G of F of a B
that\textsf{'}s precisely what we have here and so therefore this notation
can be used somewhat shorter then the code and easier to reason about
so these are the two laws that we have so far for natural T this is
the associativity law where I just have this notation G 1 2 3 G 1
2 3 to indicate that this is the function G with three arguments but
this second and third arguments are in a tuple and here the first
and second arguments are in the tuple so I have just rewritten associativity
law in a short notation it\textsf{'}s slightly shorter than this but it\textsf{'}s still
quite ugly and finally the two identity laws which I rewrite like
this pure acting a and then tuple with Q and then this function and
so on that\textsf{'}s exactly the law here and then I write Y goes to jr. a
1 as B goes to G of a B so these are the laws for map 2 so let\textsf{'}s Express
F map to through zip we can express it certain like that written out
with all the arguments in folds like this F map to acting on G and
also acting on this tuple is zip followed by lifted G or F map of
G acting on this tuple and so since this tuple is the same we can
omit that argument in the cred equation where the both sides of this
equation are now functions acting on a tuple q1 q2 so this is how
we can express F map through zip and we can substitute this F map
into the laws and then we will obtain laws for zip to simplify things
we can combine the two natural G laws into one where we use two functions
F and G or F 1 and F 2 acting on Q so that\textsf{'}s how we can rewrite these
laws and naturality then follows four zip because F map G is zip followed
by lifted G so we can just write it here and the right-hand side of
the natural law is zip of this followed by lift G so now you see we
have this followed by left G equal to this followed by lift G \& G
is an arbitrary function so clearly we can just substitute G to be
identity and get rid of it so that\textsf{'}s therefore the cat\textsf{'}s reality naturality
well for zip if this law holds we can add an arbitrary function like
this and we restore this law so therefore the laterality low for zip
is equivalent to the natural go for f- so what does this naturality
law say it says we can first transform both arguments of the tuple
using some functions and then we can zip them or we can first zip
and then we can transform both arguments of the tuple using some functions
that\textsf{'}s a typical form of the natural reality law that says you can
f map something before your natural transformation or you can f map
it after natural transformation the results are the same let\textsf{'}s look
at the associativity law it\textsf{'}s more complex but if we do this substitution
so we take that law as written here and we just substitute F map G
as zip followed by lifted G everywhere so the result is that we have
G of zip and so on equals that so the arbitrary function G has a different
set of arguments on the left and on the right now you see as G 1 2
3 was actually this kind of function where we first I'm to pull 1
and 2 and then we have 3 or we have firstly on tuple 2 and 3 and then
we have 1 so this is what I am indicating in this very informal notation
I don't want to write a lot of parenthesis and so on I just want to
indicate what we want so this is just shorthand for kind of function
and that kind of function but if we substitute zip followed by G here
so now we'll have zip followed by G which means that we actually need
to substitute the full term zip followed by G on the queue so zip
on that followed by G that\textsf{'}s how every written it here now these identity
sorry these tuple transformations all come from this isomorphism which
is trivial that\textsf{'}s just we can undo pull this or we can untap on that
they're equivalent so let\textsf{'}s not have to let\textsf{'}s let\textsf{'}s not write all
of this explicitly every time we are on to playing and so on let\textsf{'}s
just do it when needed whenever we have a tuple of a tuple like this
we just unto pull and rearrange as required and this operation i want
to denote by a special symbol so this is the symbol which is kind
of equivalent it\textsf{'}s not precisely equal but it\textsf{'}s equivalent up to this
a trivial isomorphism so in other words very simple isomorphism that
doesn't require a lot of work and so if we get rid of that then the
G is an arbitrary function so let\textsf{'}s substitute identity instead of
G and the result is this law so we first do a zip on q1 and the zip
of q2 q3 or you first do a zip on q2 q1 q2 and then zip of that and
q3 and these should be equivalent up to rearranging the tuples like
this because the types are going to be different so let\textsf{'}s look at
the type diagram to see what types they are this type diagram is quite
large and so let\textsf{'}s look at look at it in this way so it starts from
these three if a FB FC which I have put into the diagram in this order
just arbitrarily it doesn't really matter now the first thing we do
is we make tuples say of FA and FB that that is indicated by these
two arrows so this is just to make a tuple out of these two also we
make a tuple of B and C here then we apply zip to fafb we get F a
B we apply the epitome of bit of C we get F BC now we can make a tuple
from this and this we get this tuple we make a tuple from this and
this we get this tuple now we apply zip again to this tuple so this
gives us F of that and this after applying zip gives us F of that
now actually they are equivalent to some F of this because of this
isomorphism so this corresponds to that and this corresponds to that
so this is the associativity law for zip its identity laws have complicated
form in particular because we have this arbitrary value a and actually
we can simplify them if we replace this pure emitted by an equivalent
method that doesn't have an arbitrary value a which is an interesting
trick actually so let me explain this simplification in some more
detail because this is a kind of a complicated law we want to simplify
it so here\textsf{'}s how we simplify it let\textsf{'}s consider pure of the unit just
apply pure to the value of type unit there\textsf{'}s only one now you have
that unit let\textsf{'}s apply we get a value of type f of a unit so I called
it w u because it\textsf{'}s a wrapped unit it\textsf{'}s a unit that\textsf{'}s wrapped in the
type constructor F in some way so with empty effect so it\textsf{'}s an empty
value unit wrapped in an empty effect so that\textsf{'}s why I called it wrapped
unit now it\textsf{'}s a pure of one so in order to restore the pure function
all we need to do is to map this one into a inside the container F
so that is what we need to do so in order to express pure through
wrapped unit so these are equivalent you can see that if we use wrapped
unit instead of here then things are actually certified and this is
how we do it so let\textsf{'}s substitute first instead of pure a we substitute
this now this we rewrite using this notation because argument B is
unchanged and instead of one we have argument a so we will call G
on this so you see so this actually is equal to that no way we can
pull out out of zip we can pull out this function using left natural
reality and then we get zip of W u cross Q where this now is outside
the zip it needs to act on the first argument and so this is how I
express that this function needs to act on the first argument of a
tuple second argument of a tuple is unchanged so now it looks like
we have a lot of this bookkeeping where we just put units and so on
so let\textsf{'}s denote the these things temporarily just so that we can reason
about that a little shorter so fine is this function and beta a is
this function so this is a trivial kind of code that just adds unit
and this code just takes a two of unit and B and returns a tuple of
a and B given a fixed value a so just substitutes a instead of unit
and the B is unchanged so this identity function is acting on B and
this function is acting on unit and this product this function product
gives me a function of this type so now this function is a composition
of this kind because we we first take a B we apply feet fight to it
which gives us 1 B so 1 times B when we apply beta a to it which gives
us a times B and when we apply G to it so that\textsf{'}s exactly what\textsf{'}s happening
here if we use that the advantage is that these are function compositions
we can reason very easily about function compositions so if we substitute
into that naturality law which we just had in this form then it becomes
G so this is beta le this is actually beta in my definitions of G
beta a zip now clearly data a can be composed with G and then lift
it because that\textsf{'}s just a functor law that we can lift after composition
and then the right hand side of the natural T of the identity law
like this which is what we simplified here as this composition therefore
the naturality law sorry the identity law becomes this equals that
but here we can also put Phi inside so we can lift Phi first and that
would act on the Q first so now we have an equality that has a common
prefix of some functions we can commit it and the law becomes much
simpler it\textsf{'}s like this so now Phi is this isomorphism between B and
the tuple of unit and B again this is a kind of a trivial isomorphism
but we can apply whenever needed we don't want to keep writing file
all over the place if I lifted to F is the neither morphism between
these two again what\textsf{'}s imagine it is applied whenever necessary we
will express that using this symbol but these are not really equal
but they're equivalent up to applying these isomer films isomorphisms
whenever necessary so the left identity law therefore can be rewritten
like this and the right identity law similarly like this so then let\textsf{'}s
actually simplify the notation some more {[}Music{]} instead of zip
of PQ let\textsf{'}s write this now this is just a symbol I invented it doesn't
matter what it is but it\textsf{'}s kind of a zip zipper symbol if we write
it like this then associativity and identity law look like that now
these laws are basically laws of a monoid up two different types that
I did not write so this is of type FA the size of type FB because
of type FC and assumed transformations that are isomorphisms here
and here natural reality law is written like this so you can lift
F act it on q1 lift f2 acted on q2 it\textsf{'}s the same as if you lifted
this and acting it on the Zipp now the wrapped unit has no laws at
all it\textsf{'}s just a fixed value of type F 1 F of a unit the natural reality
law for pure will follow automatically from the definition of pure
through the wrapped unit this is again the pattern we have seen where
one covalent method is simpler than another and it is it is equivalent
nevertheless and the other method is expressed through the first method
using some F map so that\textsf{'}s very similar tangent and so actually we
see that the laws of applicative have become extremely simple and
suggested there similar to the loss of a monoid there is a social
activity and to identity laws now this syntax of code is just zip
right so this is P zip Q in Scala you can write zip in fix notation
already so you can just say q1 zip keep to zip q3 and you shouldn't
you don't have to worry about parentheses that\textsf{'}s the essence of the
associativity law and identity laws mean that you can have enough
something that you zip with and it doesn't change that then you're
zipping with up to and isomorphism of course it\textsf{'}s they're not actually
equal they're isomorphic they're equal when you apply those isomorphic
transformations in the right places so obviously this is much simpler
than the laws when formulated in terms of map to I already discussed
in a previous tutorial that naturality usually follows from parameter
ECT in code that has type parameters so when we want to check laws
for specific factors then we don't usually need to check naturality
it will be obvious if our code is fully parametric and generic and
all type parameters then it will natural tea will be automatic associativity
of course needs to be checked and identity laws need to be checked
so actually it\textsf{'}s interesting that we have not seen third natural to
move from up to we have seen to natural reality laws and we have derived
naturality laws per zip but actually when we define map to three zip
there\textsf{'}s an there\textsf{'}s one more law one more natural T law to we'll see
shortly and now it became obvious that we are we were missing a law
so our consideration when we derived the laws they gave us two laws
but they didn't give us all all the laws we could have been more clever
right here but we weren't however once we understood how to formulate
the laws in the best way then obviously the right thing to do is to
start with these laws and then derive the laws from map to from those
this also gives us assurance that we haven't missed any laws this
is a very common construction I'm annoyed and generally this idea
of having a social tivity laws and identity laws is a very is a very
common pattern that happens time and again once we see that we have
assurance that the laws are complete this is the complete set of laws
for an applicative functor if we formulate the laws in terms of zip
zip and wrapped unit so we have now great assurance that we are on
the right path we have found the correct laws for duplicative factors
now factors that are zip able as I said before they did not have peer
necessarily they do not have wrapped unit they only have the associativity
and naturality laws but negative factors must have both this and these
laws so as I started from monads well clearly if a functor is a monad
then it will satisfy all these laws - if we define map - through the
mullet construction then we will automatically satisfy all aplicativo
since since those laws were motivated from monadic clause but there
are some applicative founders that cannot be monads and so actually
all monadic factors are negative but not vice versa and this is actually
a mistake it\textsf{'}s a strict superset so there is strictly more applicative
functors than Muniz and as another way in which negative factors are
a superset is that bucket of fun term employment asian may disagree
with the moon and implementation of the function map - in other words
we may want to define lab - and zip and so on in a different way for
a functor that already has a monad implementation but for some purposes
we might want to define it in a different way and we have already
seen that in the first part of this tutorial when a monad would for
example for the either factor and one odd implementation would take
only the first error in a computation and we want to gather all errors
so we define an applicative factor differently so that it collects
all errors so that definition is disagrees with the definition of
map - that would follow from the walnut so strictly speaking we should
rename this factor some and into some other name just to distinguish
it so that we don't get confused because we might want to write applicative
code in the magnetic for yield block and then they would accidentally
have the wrong implementation of map 2 so strictly speaking we should
avoid defining map 2 at the same time as a flat map so that they disagree
but in practice this does not happen very often but it but this is
another way in which applicative factors are a strict superset so
sorry about this mistake it\textsf{'}s a superset so what is the third natural
T law it\textsf{'}s a law where we transform the result of map two we have
not done this we have not thought of doing this transforming the result
of map two should be equivalent to doing map tool like this and transforming
the result of G so when we write it in terms of map two we get this
curve once and if we write it in a short notation then map 2 of G
followed by some lifted function must be mapped to of G on which we
act with that function or if we rewrite the same thing by substituting
explicitly the arguments P and Q which are here count one and count
two then we get this now this law follows automatically if we define
map to through Z because that\textsf{'}s a definition map 2 of G is this but
then clearly if we apply some function f to the result then it\textsf{'}s the
same as if we lifted a composition of G in F because this is the functor
composition law so this is a very obvious property then that is not
reality with respect to transformation of the result of the map so
we could have noticed that we're missing a naturality law firm map
- if we looked at map - type signature we see it has three type parameters
and usually there is a natural T law for each type parameter because
nationality law means we aren't we aren't changing the results of
any transformations applied to a certain type and for each type parameter
we can transform that value of that type separately from transforming
values of all other time parameters and so zip has two natural G laws
because it has two type parameters but my up two has three type parameters
that should have three naturality laws so that\textsf{'}s the third natural
table now I would like to go a little deeper in analyzing what the
properties of app turned out to be so that\textsf{'}s a very interesting direction
because it will show us more deeply what are the properties of app
and what are the laws of app now the laws of app are not so easy to
derive and we will have to prepare ourselves for that so app is a
function of this type now we can consider it as a kind of lifting
so we have this function and we lift it into this but actually this
is not a function this is a type constructor whose type parameter
is a function type so this is not really a lifting of a function into
some other functor but we can think about it like this we can think
well this is just like a function it could be except it\textsf{'}s a little
twisted so it\textsf{'}s kind of a lifting from a twisted function type into
the function of a type constructor to type constructor type now a
lifting should have identity and composition laws just like a lifting
of a factor the factor would have type signature a lifting from F
map going like lips a B to F a FB so that is a classic lifting where
we get our intuition about lifting now lifting in the functor case
has identity and composition laws so just two laws which means that
if we lift identity we get identity and if we lift composition of
two functions we get a composition of two functions that\textsf{'}s very reasonable
can we find the same laws for this lifting can we find identity and
composition laws for it now if we could then we would first of all
need a value of this type F a to a which would represent the identity
for for this lifting and then it would we would demand the function
takes that identity value and returns an identity transformation on
a PHA so how do we get a value of this type well let\textsf{'}s call it by
this symbol identity with this dot in a circle well we have a pure
method on an applicative factor so we can easily create an identity
transformation like this and put it into a pillar and that would be
a functor type value with empty effect because we're pure we were
using pure and identity transformation so that looks like a good candidate
for the kind of value that should not transform anything we will see
that this is indeed the case now the second question is how do we
check the composition lon well unusual functor lifting is clear what
the composition of two functions a to B B to C they compose only and
get a function e to C similarly here you can pose a fail to FB FB
to FC and you get FA to FC so here we have this type however so how
do we compose this and this so that the result is this well it is
not easy to do that necessarily but let\textsf{'}s try to use map to to implement
that composition can we maybe do that because we almost have it we
that we need with map to or we could use zip maybe zip is actually
even easier to visualize here if we zip this this we get an F with
a product a B and BC we can just compose those things in the product
and F map so that\textsf{'}s what the f map 2 gives us in one go with it takes
two functions P and Q now let\textsf{'}s denote by this symbol this kind of
twisted composition where we can post this with this and yield that
so we will define it now using this code which is f map of G H F map
to of G H where the function takes a to B and B to C and returns just
a composition of these two P compose key P and thank you now it seems
that we have defined a reasonable candidate for identity and the reasonable
candidate for composition let\textsf{'}s check the laws what are the laws well
it turns out there are precisely the laws of identity and composition
namely these the composition of this identity with something does
not change that something and composition is associative also there
are two naturality laws which basically say that f map or lifting
plays well with this composition if you can lift first and then you
compose or you can first compose and then lift this and act on the
result and so so you can you can check that these types make sense
the first three laws or actually the identity and associativity laws
of a category where the morphism type is this twisted function type
the identity is this and the composition is that I remind you that
the category laws for identity and composition laws you need to have
just these laws associative 'ti of composition and identity laws the
second the last two laws are natural reality was there connecting
this twisted category composition with the ordinary F map which is
the ordinary lifting that the F must already have because F must already
be a factor so these are the five laws that are very interesting because
they show that app is actually kind of a lifting is it satisfies functor
laws as if we had a functor between these two categories so in this
way we have justified calling up and lifting it\textsf{'}s it\textsf{'}s it\textsf{'}s like a
F map function except the category has a twisted type this is another
way in which we have assurance that our set of laws for the applicative
functor is reasonable it is not too large not too small because this
is just a category law in some suitably defined category in other
words in a category of functions with this type morphisms of the category
have this type I remind you what I call categories just morphisms
where this arrow is just a general symbol and you have to define what
it what it is in any particular category and then you need identity
laws and Composition laws which are these so having defined that let\textsf{'}s
check that this is actually true I haven't actually derived that these
laws must hold they are follow they follow from the map to laws so
I I will now go through this derivation for example let\textsf{'}s start with
the identity was let\textsf{'}s consider this we substitute the definition
of the product dog sorry of the composition which was here and also
we need to look at slide 7 because that defines laws for the F net
so if we substitute that we get F map to of this which is list definition
here but we are working on pure of identity function so according
to the law on on slide 7 this is equal to that so we have F map to
of something and pure so that\textsf{'}s supposed to be equal to this so let\textsf{'}s
just copy this over and we get that but now you see identity function
followed by B is just B so now we have be going to be which is an
identity function lift that and still an identity function so that\textsf{'}s
equal to H so that\textsf{'}s why they are tentative or holds similarly we
derive the right identity law associated with T law we also need to
substitute then according to the third naturality law now this I'm
just using a very short notation where this is a function that takes
two arguments and returns the composition of these two arguments as
functions that\textsf{'}s what I want to use here this is this function this
I just denote this by that it\textsf{'}s a single now we use the third naturality
law and move the function this function out of f map to so then F
map to with this function H K is equal to this function lifted of
have not evolved identity HK so that\textsf{'}s the third naturality law and
once we do that we can do we can just see that whenever we have the
category composition it\textsf{'}s just basically f map of this composition
function and so then every time you have this you have a composition
so then you basically have F map two of these and this is mapped to
that and that is mapped to that but this is basically the same as
a formulation of a socially divisive o4f map to where we had to use
these functions that rearrange the tuples now this function is the
same this is the G in the F map to law here this is the G and these
are precisely the right hand sides and the left hand sides of this
law this is that and that precisely that long and so because F map
2 has that associativity law this must be equal to that because so
we find that these are equivalent so so associative et offereth map
to a group is equivalent to relativity law for this the naturality
laws can be derived by writing out again what is the morality law
for map tool that we have now three natural tables come up to so for
example you do this you act with something on G we write a definition
of the category composition there is no name for this category by
the way I don't think there is but I just think about it as the category
allows me to think of applicative factors as having a lifting sometimes
it\textsf{'}s just a category where this is the composition so this becomes
my left hand side and then I can transform it so just put this outside
put this function outside so I get that put it in here and I have
GH then I I put that is the is the mother this F map to of this alright
I have another naturality right where I act with some function on
the G so that\textsf{'}s the third natural T so I can pull this out of F map
to which is this part I pull it out with and the result is that I
have just XY going to X and then Y which is this so then I pull this
out and this is a definition of G dot H so then I have my natural
to look for yo dot H and similarly the other natural tool so now the
laws for app let\textsf{'}s write them out so app has this type and is defined
like this identity law is that app of identity is identity right so
the laws for app are just the laws of of lift and we like that lifted
identity must become identity and lifted composition must become composition
so these are the laws for app identity law let\textsf{'}s derive it well we
can derive it in a pedestrian way and let\textsf{'}s do it first so app of
identity you applied to some Q so what is that so let\textsf{'}s write the
definition let\textsf{'}s f map of identity and then this is a definition of
this single applied to Q so let\textsf{'}s write it all out we have this F
map two of this functional identity of this type is actually this
function if you think about it but we uncured it so we we need to
incur it because we want to have map to which is uncured you see here
I have this F map subscript to just clear it and F map without substrate
is uncared applied to a product of F X is just this this is equivalent
to this identity and now I'm going to just simplify this by using
the identity law firm F map to which is going to give me this if I
substitute we identity law for F map to which is this and therefore
I have this code now identity of X is just X we have X to X lifted
still identity the result is Q now there\textsf{'}s an easier derivation which
I would like to use now because it\textsf{'}s interesting and simpler consider
these isomorphisms they are obviously they are not the same obviously
because this is a value and this is not yet evaluated you have to
give it an argument and so on but they're equivalent computationally
and then this app becomes basically after this equivalence if you
look at this app it basically becomes a categorical composition of
Q and P so Q has this type you cannot do categorical composition unless
both have function types but because of our P isomorphisms we can
replace a by 1 to a here and so we do that this must have been a double
arrow I will correct the slide so Q F 1 to B can be composed with
pfb to Z and you get something of type F of 1 to Z which is equivalent
to F Z so that\textsf{'}s F Z so that\textsf{'}s in this way we can equivalently rewrite
app as just this categorical composition and then everything becomes
just very easy because then app of identity on Q is Q composed with
categorical identity and that\textsf{'}s just Q by categorical laws so composition
law again they are easy we say AB PQ is just Q P and we just rewrite
that as q GH and then this is where rewritten as app of each of app
of G of Q which is f of h q g q GH now these are equivalent because
of associativity so in other words once we establish the category
laws app becomes a lawful lifting and all the other laws follow so
it\textsf{'}s sufficient to establish the category laws or it\textsf{'}s sufficient
to establish the zip laws and everything else follows so these are
the all the possible ways of looking at the laws for placated functors
so we can choose the one you like best and different ones have different
utility for instance the category laws are not directly so useful
for for coding and they're not so easy to check perhaps because you
need this complicated categorical composition and all these type parameters
I basically have three type parameters in this composition which is
harder to check than zip that has two type parameters but it it allows
us to look at those things in a very general way which we'll see later
so now I will go on to define various constructions that you can use
to make applicative functors out of other types we have seen in the
previous chapter a large number of constructions for monads since
all monads are applicative they all those one etic constructions also
hold but sometimes they hold for weaker conditions in other words
not as an magnetic construction that for example this was also magnetic
construction but it requires both of them to be Mona\textsf{'}s but now we
only require them to be applicatives negatives is a superset of monads
so this is a similar construction but it is a superset and so on so
we'll now go through these constructions and show that the laws of
the implicative hold for each of them assuming that the laws hold
for the parts after which you build 

the first construction is constant factor and identity factor both
of these factors are also Mullins and their applicative instance is
following from the moon unit instance nevertheless let\textsf{'}s look at the
code the constant factor is a factor that takes a type parameter a
and returns a unit type so we could define it as this type function
that takes the type parameter and returns always a unit type independently
of the type parameter so that\textsf{'}s why it\textsf{'}s a constant factor it\textsf{'}s like
a constant type function now for the constant factor there\textsf{'}s only
one value that data can be if the data is of type F of a namely the
unit value and so all methods of the factor including monadic methods
flat map pure map zip we can only return the unit value since all
methods return unit value all laws are trivially satisfied because
the laws say that one combination of methods should be equal to another
combination but all of these always return the unit value so they're
always equal so in this case it is trivial we don't need to check
any any laws here let\textsf{'}s consider the identity function it\textsf{'}s a type
function that takes a type parameter a and returns the same type a
so you can look at it as a type function that is the identity function
at type level let\textsf{'}s define this type constructor like this let\textsf{'}s define
the factor instance there\textsf{'}s a standard very simple code that just
applies the function f to the data there is nothing else we could
do and let\textsf{'}s look at the applicative instance now I define this typeclass
that I called Mujib which is just a typeclass that expresses the implicative
property of the function f by defining methods wrapped unit and zip
i made this typeclass require a typeclass constraint that the type
constructor f should be a factor i also defined a pure metal because
we can now using the factor instance for f we can define pure through
wrapped unit i also define some convenience methods such as getting
the zip evidence value of this type getting the wrapped unit value
for a given type constructor so I can just write W U of F at any point
it\textsf{'}s very quick to get the wrapped unit value I also defined a converter
to catch- instance now on the CAD Slimer II applicative instance requires
pure and app but when you reason about the properties of Lickety factors
and when you check the laws it\textsf{'}s much easier to reason about functions
zip and wrapped unit then about alkanes pure and so they're equivalent
as we have seen but it\textsf{'}s much easier to reason about applicative filters
in terms of zip and wrapped unit so that\textsf{'}s why I defined this as an
as an adapter but I don't actually use the cat lick ative I use my
top class with zip instead and I also define a syntax it allows me
to say if a zip FB which is shorter and easier to read which is equivalent
to this slightly more verbose so to define a typeclass instance for
zip it is required to have a functor type of instance and so that\textsf{'}s
why I will always define first a functor instance for a type constructor
and then a with zip instance so how do we define a with zip instance
well we just need to define the wrapped unit the whoo and the zip
in this case the wrapped unit is just unit there\textsf{'}s no structure that
wraps here anything so it\textsf{'}s just unit so we have nothing else to do
except write these functions like this and we could have written them
using the carry Howard library implementing them automatically because
this code is completely determined by the types there is nothing else
you can do for example in order to return a value of type a tuple
a B if you're given a value of a and the value of B there\textsf{'}s nothing
else you can do in this so you could just say well this is shorter
than implement it\textsf{'}s the same let\textsf{'}s check the laws now the associativity
law says that this combination let\textsf{'}s undo my changes to see what the
code is since we need to know zip of a and the result of the zip of
FB and FC is just that and the zip and the other order is this so
we need to show that these are equivalent but this is exactly the
definition of equivalence which is rearrangement of nested tuples
so there is nothing else we need to do here to prove and similarly
identity laws and say that zip of wrapped unit and FA must be equal
to FA but wrapped here it is just unit and zip is just a tuple and
so this is a tuple of unit and FA and that should be equivalent to
FA but this is actually the definition of our equivalence that we
have equivalence that could rearrange two poles when necessary and
also add or remove a unit in a tuple when necessary so that\textsf{'}s just
a definition of our records so the laws are satisfied by definition
without much kind of calculation here consider now the second construction
this is a product of two factors so if two functions are replicated
then the product like this is a factor that is also applicative now
the product construction is seen in pretty much every typeclass in
the functors product of two factors is a functor monads product of
two minutes there is a moment filter balls product of two filter rules
as a filterable same as for application it\textsf{'}s a it\textsf{'}s a product is always
going to be a construction for every of these typeclasses however
the sum or the disjunction is not a construction for applicative so
the disjunction of two applicative factors in general is not effective
and also it wasn't so Ramona we'll see examples shortly non-intuitive
factors of this shape so let\textsf{'}s look at the code for construction to
now we need to define the type constructor somehow now we can't just
say type F of a because we want G and H as parameters the way around
that is to use the syntax of the so called kind projector in Scala
which I've been using in this tutorial and this is the syntax that
represents the type constructor that is the product of G and H so
it takes that type of parameter and returns this type so this is just
a syntax for type function written like this the lambda is a key word
that the plug-in defines so first we need to implement the function
instance here it is G is a factor H is a factor then we define a factor
of this so we need to define just a founder instance for product which
is a standard code that you take component by component the first
component of a fail you map in the second component of a fail you
map and these maps this is in G and this is an age so component by
component you define it the same way we define the rap unit and the
zip we defined component by component so for example zip from map
things like this we can just take this and this together which will
be the first component of each of them when we take this and this
together to zip which will be the second component of each of them
let\textsf{'}s just like that it\textsf{'}s the first value and the second value so
the first component on the first value is zipped with the first component
of the second value and you get this which is the first component
of the result so component by component we define wrapped unit in
Z the wrapped unit is just a tuple of the two wrapped units for G
and H zip is a tuple of - zips from the image this is in zip regime
and this is a fridge the walls will hold separately in each part of
the Pinner because the computations proceeded in each part of the
pair independently so the first component of the result depends only
on first components of the data the first component of the result
only depends on the first component of the data and so on for the
second component so it is kind of easy to understand the laws will
all hold very easily let\textsf{'}s nevertheless take a look at how that can
be written down so let\textsf{'}s write a social tickity so a pair of gah a
so I assume G X of type G of a and H a is of type h ma so I so this
will be of type F of a when F is this this type constructor we don't
use the name F here but I used it in the slide so if we first zip
the first two together and then take the result and zip it with G
CHC so what does it give you give us first we zip like this the first
two and then we zip the result with the rest now if we apply the definition
we need to zip the first component with the first component so that
will give you this now I stop you writing parentheses here because
this is the zip in the G factor which is associative by assumption
we already assumed that G and H have associated so I don't need to
write parenthesis like this I don't have to write it like this it
doesn't matter if I write it like this or if I write it like this
result is guaranteed to be the same already by circulating in the
function G so also here in the frontal H I don't write the parentheses
and clearly these expressions don't depend on the order in which we
zip so let\textsf{'}s {[}Music{]} let\textsf{'}s zip together the second two and the
result won't be like this we have this pair and then we zip it with
this ugha so again we zip the first component with the second with
the first endpoint the second component for the second component and
the result is this so we get exactly the same result after using a
social activity for G and H so that proves the associative law more
rigorously only identity laws are proving similarly we take the wrapped
unit which is defined like this by our code and we zip it with some
arbitrary GEHA and the result is the zipping of wrapped unit in each
component so we assume that this is equivalent to just gah a in each
component because there are the entity laws as we assumed cold for
G and H already and so this shows equivalence but we need this is
equivalence - yes and similarly we can show the right identity law
the 3rd construction is a free point advantage I already talked about
this in the previous chapter actually this construction as well as
this one of three constructions which I will talk about in a later
tutorial now it\textsf{'}s just a name that I'm using to invite myself to do
this and for this tutorial I am just using this definition this is
a factor Construction defined like this I'm not going to use the fact
that it is a free construction because we haven't yet gone through
this in the tutorials so for the function defined like this assuming
that G is duplicative we need to show that F is also implicated so
let\textsf{'}s see how that works so this is this factor again we will use
the kind ejector in order to denote this type construction and so
syntax will be like this since the type function that takes the type
parameter a and return the disjunction of either a or G of a note
that in the syntax for the type function a is a type parameter that
is only defined in the argument of this type function so this is not
a type parameter here it is not a type parameter of this function
it is a type parameter only of the type expression here so that\textsf{'}s
that\textsf{'}s important to keep in mind so I could use a different letter
here in principle and it\textsf{'}s not a type parameter in in the value here
so the function the factor instance is standard we take a function
f from A to B and we apply it to a here or to a here depending on
where we are in the disjunction and if we are here that we need to
map it over the G function so this is a melting the G factor so we
certainly need to assume that G is a function for this to work this
is Stanley let\textsf{'}s define the zip {[}Music{]} instance so I have put
here a type constraint so that they're both a with zip and a factor
and this is just so that I could more easily use map G if I ever need
it but it seems I don't really need it so let me remove it from from
here how do i define the wrapped unit and the zip so that\textsf{'}s actually
an interesting question because wrapped unit is required to be a value
of type F of unit which is either unit or geo unit so we have two
possibilities to implement this method we could have the left of unit
which is this or we could have the right of G of unit which is the
wrapped unit of G which is this so we could have either this or that
we actually I don't know up front which one is correct we need to
find out so it turns out that only this for fit fulfills identity
laws this one doesn't we'll see why very shortly but at the beginning
I don't know so you know that there are two possibilities you should
have to explore both of them how do we define the zip method well
as if needs to take either a G of a either B G of B and return either
of this a B G of a B so clearly we have several possibilities here
we can have a left left we can have a left right and so on so let\textsf{'}s
match on all these possibilities so we have four cases if both are
on the left and we have a value of type any and the value of type
B what can we return what we need to return a value of this type so
we could return the left of a be like this or we also could return
the right of G of a B where we could easily lift the pair a B into
the function G by using the pure method of the top of the function
G and very easily just left so the question is should we do that we
return this or we return a right of G pure of this turns out that
we have to return this in other words the identity laws will not hold
we'll see why now let\textsf{'}s see what happens when we have a mixed turn
left and there right now in that case we have an a and we have a G
of B what can we return well we can't possibly return a pair of a
B because we don't have a B we have a GOP so we must return the right
part of the disjunction so we have to return right of G of nd now
clearly we have to combine somehow a with G and B and the way to get
G of a B is to use it on G because that\textsf{'}s the only thing we have that
seems to be the right way but then we need to lift a into the function
G and we have this and this is actually according to the identity
law in the function G which should hold this is equivalent to just
F mapping or mapping over G over G B of a function that takes B and
returns a pair in D so let\textsf{'}s just write it like this because it\textsf{'}s
easier to understand what\textsf{'}s happening so similarly if we have a right
on the Left we must return the right of this combination where we
lift the B into the function G and then zip together with G finally
if we have both on the right then we just zip them together and they
remain on the right so you see only when both of the data values are
in the left part of the disjunction only then we can return the left
in other cases we must return it so that\textsf{'}s how it must be now in my
intuition it will be already suspicious if we wanted to return right
in all four cases it means that somehow we are losing information
never returning a left and that\textsf{'}s going to lose information and certainly
that could be an obstacle to satisfying the laws we'll see that indeed
that is so let\textsf{'}s check associativity not verify so objectivity we
could have just written F a zip FB zip FC and substitute with this
code but that would be very cumbersome because there will be 8 cases
to consider it will be a lot of writing in the work let\textsf{'}s instead
try to visualize how this operation zip works in this function and
the way to verify associativity easier is to consider zip of three
values like this with DIF different parentheses first in this way
down in that way but try to define zip in such a way that it is manifestly
independent of the way that you put parentheses so try to reformulate
how the zip is defined so that we can somehow define this operation
directly for three elements rather than defining it only for two if
we are able to define a zip operation directly for three elements
like this in a way that is clearly independent of where the parentheses
are in that a B and C all enter in the same way in this computation
lady there isn't any precedence is not that a and B are first and
then you see somehow then it will be manifestly associative so let\textsf{'}s
see how that works in this case so consider this this situation now
we could have all of them on the left and then we would be in this
situation or return the left after the first pair and also after the
second zip or in the other word will be exactly the same so if all
of them are on the left then it\textsf{'}s just going to be the result is just
going to be to this triple with parentheses here or with parentheses
here but that\textsf{'}s equivalent according to our definition of equivalence
and so that\textsf{'}s associative so if they are all on the left then this
operation is associative the result is manifestly associative now
if even just one of them is on the right then we know that the full
result is always going to be on the right and we know that what what
will happen is that all the parts that are on the left and maybe for
example this is on the left but this is on the right and this is again
on the left all of those that are on the left are going to be lifted
into the front of G using the pure operation and so basically you
could imagine that first we lift all of these that are on the left
into the pure into the G factor and then we have all of them from
the right as if and then we just sit them together so the result would
be as if in the G factor of three values from the G factor and that\textsf{'}s
associative so therefore we have formulated a computation in a way
that is manifestly associative it\textsf{'}s independent of the order of parentheses
and so in this way we figure out that it\textsf{'}s associative it\textsf{'}s much faster
to write out this code let\textsf{'}s check the identity law so the wrapped
unit is on the Left it\textsf{'}s the unit when the left part of the disjunction
which I denote like this in my short notation consider the right identity
law for example so we have some fa which is this type we zip it with
this now if a face on the left then the result will be according to
our definition this which is going to be this left of a and the unit
and that\textsf{'}s equivalent to left away according to our convention because
we can always add or remove units from the tuple and so then the result
is this which is equivalent to that so that\textsf{'}s an identity identity
law if a face on the right you need to lift this into the right and
then zip with FA when we lift unit into G using pure the result is
the wrapped unit which is pure of unit that\textsf{'}s the definition of what
wrapped unities so in relation to the pyramid and so we have the zip
equal to this right of this with a wrapped unit of G now the identity
law for G tells us that this is equivalent to G a therefore the result
is equivalent to right of G which is exactly FA so in other words
zipping FA with this always give you gives you something equivalent
to FA and that\textsf{'}s the identity law and clearly in the same way we check
their left identity if we define the wrapped unit like this instead
then we can see that the identity law will be broken here\textsf{'}s why consider
the zip of a left value with this wrapped unit defined as the right
of wrap unit of G according to our code whenever something is on the
right the result is on the right so the result of this is going to
be right of something now whatever we compute in here we couldn't
possibly or produce a left of anything because it\textsf{'}s already on the
right so it cannot be equal to the left of a our equivalence rules
are that we can rearrange tuples and we can add or remove units unit
values from tuples our equivalence rules do not allow us to interchange
anything else or to change at all anything else so we could not change
left into right and so that\textsf{'}s not going to be covered and similarly
here if we define this to be right then zipping left with anything
could never give a left and so the identity law that would have no
chance and so in this way we see that this implementation is the only
one that respects the laws there are several implementations that
are of the right type signatures but only one implementation respects
the laws so that\textsf{'}s an interesting situation which happens for this
construction now the next construction is the Freeman and over a function
G so G here is an arbitrary contract and this is a recursive type
that\textsf{'}s defined language now we have seen in the previous chapter that
this is a monad and anyone odd is an applicative as well we can define
zip function for a monad like this and we can define pure will not
already has peer and we can define the wrapped unit as pure of one
pure unit so if we have a moon at construction then we already have
the Monad laws the implicative laws are consequences of the Monad
laws when the zip and the pure are defined like this through flat
map so we don't actually need to check any laws here for this construction
because we already checked the Monad laws and the applicable laws
or a consequence when when we defined implicative instances through
melodic instances and the same codes for this construction this construction
was also a monadic construction so we don't need to consider it separately
in this chapter notice that we did have to provide proofs for these
constructions because even though we had anodic constructions of the
same form those required for instance G and H both to be monads now
we don't require this we only require them to be applicatives similarly
here we only require this to be applicative so we have relaxed recall
some requirements as compared with the constructions in the monad
and so since applicatives are a strict superset of monads and we have
relaxed those requirements we have to provide a proof for these constructions
and does not assume units from the user just assume duplicative and
that\textsf{'}s what we have done we provided proofs of the implicit of laws
but only assumed that G and H satisfy applicative laws we did not
assume that G H permanent now in these two constructions there aren't
any constraints on G and H that we changed these are exactly the same
constructions with exactly the same conditions from G H was in the
Monod constructions and therefore we don't need to prove anything
you know they are exactly the same these constructions are stronger
than we need they prove that these are mu nuts and it follows that
these are also ticket ifs but that\textsf{'}s okay that they are stronger and
they prove more than we need in this chapter but that\textsf{'}s fine now the
takeaway is that these constructions also work for negatives because
they are monadic constructions now here are some constructions that
do not correspond to any mimetic constructions these are the last
three the construction six is the constant factor giving a monoid
value so it\textsf{'}s a type Z some type that we know is the 108 it is not
a unit actually we will see this shortly so the type constructor is
defined like this and we don't want to define it as a type F because
we want to keep this Z as a parameter so we again use the kind projector
and we write this kind of type constructor for the type function that
takes type parameter a and returns the type Z independent of a that\textsf{'}s
okay quite fine so as usual first we define the function instance
the functor instance is standard there\textsf{'}s just one remark that I'd
like to make here which is that this function f is not used in the
map function so the map function is supposed to take a value of type
F of a and return a value of type F of B but both F of a and F of
B are just the type Z they don't depend on a and B so we just take
this Z and we need to return a zero we return the same thing it can't
do much else reasonably and we can't use F because we don't have any
age to apply F to and also we don't have any other thing that would
apply to F or any function that would consume this type we don't have
any of them so there\textsf{'}s no way what we could use F in the code of map
so we're losing information but that\textsf{'}s okay so we're losing this F
that\textsf{'}s fine let\textsf{'}s look up the implicative instance the wrapped unit
is monoid empty value this seems to be reasonable what what else can
we return an iterator in a value of type Z but we don't have any data
to compute that so the only value we can return is the empty value
of the monoid and the zip operation is just a monoid combination mono
in operation on a fan of B and we could have defined it in the opposite
order and I will still be valid so this is somewhat this was an arbitrary
choice really so why do the laws hold for this it is because if you
look at the applicative laws formulated like this they look exactly
like monoid laws except that in the applicative laws all of these
are values with type parameters so this is some F of a is a summary
of B the system F of C however if F is the constant factor then all
of these are just the same type Z and then this becomes exactly 1
2 1 1 oide follows there aren't any more type transformations here
they're just this becomes equality and this B\textsf{'}s 3 laws become exactly
the monoid lost so therefore this operation instead of zip exactly
satisfies all the laws and if we interchange B and a it would be exactly
the same laws with being they interchanged where appropriate so this
would be exactly equivalent let us see why this type constructor cannot
be imminent so we could define the pure function by again returning
1 with empty there is nothing else we can return because we can't
reuse use a value of type a to compute a monoid value of type Z so
we could implement pure like this it could implement flatmap by again
returning this F a unmodified now we can't use F at all just like
we couldn't use it in the map so we are losing information in this
function we could also return here an empty value of the monoid or
we could return this value but in any case we would have to lose information
about the function f now the left identity law for a monad is this
which is a flat map with F which is applied to pure should be F but
flat map loses all information about F as we have just seen so this
can't possibly recover F whatever you compute here however you define
here it cannot possibly recover F from a function that ignores its
argument F and so this law could not possibly hold and it could not
hold however you define this you could define it in this way or you
could define it as Z dot empty but still you could not recover F and
this law would not hold so the only constant functor that is a monad
is this factor is the unit because then you're losing information
but there is no information to lose there is only the unit value there
is nothing else that you could possibly return and so you aren't actually
losing any information because there wasn't any information to lose
to begin with so for this reason this is a construction that does
not correspond to an another construction next construction also doesn't
it\textsf{'}s similar to this one except we're having here instead of the pain
we have demantoid Zee now G must be a positive just like here so why
is this construction not magnetic because even if you take for G the
same first factor for example this one the constant constant function
then we will have Z plus 1 which is a monoid but not a moaner as we
just have seen the only constant factor that is one note as a monad
is this the unit so this is going to be not a unit it\textsf{'}s going to be
Z plus unit and so it it\textsf{'}s not going to be Amana so it is impossible
to have a construction like this for a unit where even if we require
that G is a monad still this is not going to be imminent for all G
let'em units let\textsf{'}s see how this construction works for applicative
families so first we define the function instance which is standard
we just apply function f to a under G using the map or if we're in
the left we don't apply it it\textsf{'}s the Z value remains unchanged so that\textsf{'}s
the standard implementation of the functor instance so let\textsf{'}s now look
at the construction so again I have put here perhaps constraint that
I don't need now we need to define the wrapped unit which is of this
type and we have again just like in the other construction we have
two possibilities we could return the left of the empty mandroid value
like this or we could return the right of the wrapped unit of G so
it turns out in this case that the correct way is to return the right
and the left of not know where the empty breaks that the identity
laws will see one define the zip again we need to on the four cases
in the left we just use the monoidal pressure to add together two
values of demantoid now let\textsf{'}s look at what happens when we have the
cross term so a Z on FA is on the left as a Z but FD is on the right
now we have a value of type G of B and we have a value of times Z
we're supposed to return this I'm supposed to return either of Z and
gob can we return this no because we don't have an A we have a Z and
we return and we have G of B we can't possibly get an A from anywhere
so we can't possibly return G of a B therefore we must return left
and the left has only seen so you must return left of this of this
Z and we must ignore whatever was on the right similarly in the other
kids and only when both are on the right then we can zip them together
using the zip in the G factor so in this case it seems we don't have
much choice but how to implement zip we still have a choice about
implementing rap unit let\textsf{'}s check the laws and see how that works
so again we will just do a consideration that zips together with three
values and formulates a result in a way that is manifestly associated
so consider these values if at least some of these three are on the
left then according to our code result is going to be on the left
and we're going to ignore everything that\textsf{'}s on the right so therefore
we will obtain a result which is going to be a monoidal operation
home all of the Z\textsf{'}s that are on the left ignoring everything that\textsf{'}s
on the right we could have three things - Z\textsf{'}s or ones but that\textsf{'}s all
so that\textsf{'}s obviously associative because it doesn't depend on the order
of parentheses and because Z is amyloid and so this operation is associative
in the monoid now the other case is that we have all of these three
on the right in that case the result is the right of zipping of these
three which is associative because by assumption zip is associative
in the function G so we have reformulated as computation in a way
that is manifestly associative the wrapped unit we need to make a
short computation here so for example we have this zipping it with
that if this is on the left result is again the left which is exactly
the same and so that\textsf{'}s identity law if we have G a on the right and
we're dipping it with this then we're zipping this ga with the wrapped
unit which is equivalent to GA by assumption since the identity law
holds for G so we have again right of GA which is equivalent to writing
G in both cases they identity law holds and the same way we check
the left identity law now if we define the wrapped unit as a left
side of the disjunction when we consider zipping of left and right
and the result must be on the left because anything that has at least
one of them on the left it turns the left and therefore it cannot
possibly be equivalent to the right of GE because whatever you compute
is going to be left of something and it\textsf{'}s not going to be right of
G so that breaks the identity law so in this way we find that this
is the only correct definition for the wrapped in it finally look
at the construction 8 is functor composition when G and H are both
applicative factors when their composition is which is G of H of a
is again an applicative factor now this is not a monadic construction
because composition of two Moniz is not necessarily Amanat it may
be or may not be in some cases so it is not guaranteed whereas with
applicatives it is guaranteed I will see an example explicitly in
this tutorial where you have a composition of monads and the composition
is not alone odd but you couldn't get such an example for applicatives
let\textsf{'}s see how that works so we have two factors we define their composition
which is defined like this as a type function and then we define the
map which is standard we just map over H under G so we have a smashed-up
map now let\textsf{'}s define the zip so how do we define a zip we need to
define a transformation like this we need to define also the wrapped
unit so the only way to define the wrapped unit is to do first take
the wrapped unit of H and then lift it into G by using the pure from
G that\textsf{'}s the only way we can get a value of this type the definition
of zip is straightforward we first zip the two genes together using
the zip energy factor the result is a G of the product H a and H B
and then we map under the function f function G we map H a and H B
into H of a times B which is this function which uses the zip in the
edge let\textsf{'}s check the laws for this so I simplified and so let\textsf{'}s check
the associativity the definition is like this like I'm just going
to rewrite it a little shorter because it\textsf{'}s easier to reason about
when when notation is shorter once we have the zip we map that with
essentially zip in the age factor and when we have three values JH
h HB HC when we accept them like this then this is the result now
in order to simplify this use natural tea loafers if the naturality
law says that for this zip if one of these argument one of its arguments
has a map working on it we could put this map over here and work on
the first type inside the zip leaving the second type unchanged that\textsf{'}s
that\textsf{'}s a natural allottee law let\textsf{'}s remind ourselves of the naturality
law that is here if we have a zip that on which some function works
on the argument then we could pull this out we have a zip of those
things without anything acting on it but we can act on the result
of zip by the product of functions so that\textsf{'}s maybe it\textsf{'}s a little easier
to see here so zip yes so zip on which some functions have worked
is equivalent to zip followed by some map so we will do that and we
will pull this outside so how do we put it outside well the result
is that we we still have this back here but before that we have this
map of zip H which takes the first tool it\textsf{'}s a product of two functions
one the zip age and the other is identity so this isn't changed and
zip age is now acting on these two because it was acting on these
two so I'm just using the shorthand but me and it\textsf{'}s kind of it it\textsf{'}s
the same as map of case image being so this map of the page here it
was like this but now we need to apply this to the result of zip which
is going to be of type G of H of M be sr g of h a times H B and H
so then HC needs to be outside of this tuple so that is the result
now we can combine these two maps together using the function composition
law just take this we first compute this and we match on that with
this case and we have this result now obviously this is a associative
combination for zip age where we quickly could commit in parentheses
and so when we do the same computation for this expression we get
a similar function with parentheses in the other place and if we compare
these two we see that these are equivalent because the G factor satisfies
associativity and these are equivalent because that each function
satisfies associative ET and these are just equivalent by definition
and so we have found that the equivalence holds the identity law is
satisfied as well for the in the same way we do a zip with map we
use the identity law for zip in the g-factor first identity law says
that this is equal to gh a dot map of this function and then we just
add zip H on top now this becomes the zip H of wrapped unit H which
is equivalent to H a so this becomes H a to H it becomes identity
function so mapping with identity function does not change the result
so therefore zipping gh a with wrapped unit is equivalent to G eg
similarly we can check the right identity laws so these are the constructions
that I was able to find that build new applicative factors out of
old ones now I would like to give another example of an applicative
factor that disagrees with its own net so what what does it mean exactly
well this is typically a situation when you have a function that has
a moon an instance and also it has an applicative instance and sometimes
you would like the implicit of instance to do something different
than what what a moon adds definition would do we have seen this in
the first part of the tutorial where you would have either moment
which stops up first error but you can define an applicative on either
that accumulates all errors so this is an example of a positive that
disagrees with its moment it\textsf{'}s often the case that you want will not
to agree with applicative but sometimes we want them conditionally
and here\textsf{'}s an interesting example of type constructor where it is
easy to see that it\textsf{'}s really reasonable for them to disagree this
type constructor is basically a lazy list so I will explain what that
means if you delete this function from unit to that and you would
have a definition of the list it\textsf{'}s a recursive definition of the list
factor adding this function arrow means that elements of the list
are not yet computed necessarily you could compute them by calling
this function you can always call it it doesn't require any extra
date and just a unit value which you always have so you can always
call this function and get the next element and the tail of the list
on which you can again call this function get again next elements
and maybe some computations could be encoded in this way where maybe
it\textsf{'}s expensive to compute these elements and this is only done when
you want them so in this in this sense it\textsf{'}s a lazy list it does not
have all its elements already evaluated it\textsf{'}s waiting until you need
them when you need them you call this function and you get your next
element so let\textsf{'}s see how we define this is a list as application factor
this is the short type notation that we have just seen and since this
is a recursive type we cannot use the kind projecting we have to use
a class also cannot use a type alias we have to use a class can you
find it in Scala so let\textsf{'}s do that and I'm just using a very direct
encoding of this like so it\textsf{'}s either unit or a function from unit
to this now in Scala this syntax does not actually mean a function
with unit argument it\textsf{'}s a function with no arguments with an empty
list of arguments Scala has this syntax it doesn't really change anything
for us just a little less typing if we wanted to have a function with
unit argument it would be like this just more typing for no game in
particular let\textsf{'}s define some utility functions so that we can easily
work with data of this type first what\textsf{'}s doing empty list which is
just returning this first unit which is left of unit let\textsf{'}s define
a function that takes an ordinary list and creates this lazy list
it would be useful for testing the idea is that we would create F
with a right which would have a function that will return a tuple
of the first element and the rest the rest will be again F of right
and the gamma function will be the second element and so on and finally
would be an empty list so that\textsf{'}s easily done we match on the list
if it\textsf{'}s empty we return the empty which will you find here if it\textsf{'}s
not empty we match on head and tail will return a F with a right inside
and the function like this which will return a tuple of head and again
a value of type F of a which is the result of applying the same function
list to the tail of the list so that is easily tested now we can create
value like this and in order to get anything out of those lazy lists
we need to actually call the function so this fetches all those things
and calls the function and the result is that you have a tuple with
one element which was a and the rest which is again the same story
again you have to do this in order to get anything out of it so let\textsf{'}s
for convenience convert to ordinary list so that\textsf{'}s how we convert
straightforward function I would call and then call itself again so
these functions are not tail recursive just using it for tests right
now so these tests check that we need these lists work first you find
a factory instance let us straight forward because it\textsf{'}s just a function
so we need to create a new function where we replace arguments and
do a map recursively and let\textsf{'}s create with zip instance and applicative
instance and that\textsf{'}s where the interesting things happen so the first
question is what is the wrapped unit no we could in principle return
a left or we could return a right because they're the type has an
either so we could return the left or we could return the right we
return the left and it will be just an empty list so the wrapped unit
would be an empty list of type unit if we return a right then we could
return a function that has this a which is a unit and then again returns
the same wrapped unit that\textsf{'}s what we do in fact it\textsf{'}s a recursive definition
that in effect it is equivalent to a never-ending sequence of unit
values in the list because the list is lazy it doesn't actually evaluate
infinitely many elements but if you request the next element if you
evaluate this function and you would get again the same function that
would generate the rest of the list so the rest of the list is exactly
the same as the list itself and so in other words it\textsf{'}s a never-ending
sequence it\textsf{'}s not actually an infinite sequence in memory of course
it\textsf{'}s conceptually equivalent to the infinite sequence because you
can request next elements as many times as you want you will still
have always next elements that I found the same elements which is
empty so converting this to list would be a stack overflow or memory
over out of memory error fix the Stack Overflow first now zip operation
is interesting but it\textsf{'}s really trivial in a sense because if one of
them is empty will return empty and if we have two functions so GA
and GB are functions of this type and we again return a function which
just has the tuple of the two first elements and then zip in the rest
recursively so this is a typical implementation of a zip of two lists
except that we need to dance around with these function calls we have
these functions and we need to match in a different way and and so
on but other world other than that it\textsf{'}s the same and so zipping a
list of two elements with a list of three elements cuts this list
of three elements to length two it gives you this as a result so this
is a standard behavior of the zip method on lists but also this wrapped
unit acts as a unit if you zip it with some list the result is the
same as the initial list after the equivalence transformation so how
does it work well since the wrapped unit logically represents and
never-ending Western unterminated sequence in this list is finite
the zip function will cut the longer sequence because that\textsf{'}s the code
if one of them is is on the Left which is an empty list then the result
is an empty list so whenever get an empty list we cut and so that\textsf{'}s
going to be the result indeed zip defined in this way I can't do much
more than cut because if it didn't cut we would have to produce here
another element of type string unit but there\textsf{'}s no string to get to
right here logically you would have to cut certainly in some applications
this is not what you wanted but in many applications that\textsf{'}s what you
want you to cut the laundry list and you could conceivably you could
do other things but what if the first list is empty when do you didn't
there aren't any values for you to fill you have to cut and sew however
if you use the list monad the definition of pure is not the same it
is not an infinite sequence it is a sequence of length one so the
pure would be just this be in one element list and the standard zip
function zipping with this would certainly not do what you what you
want it would cut at length one now certainly if you define zip through
the munna instance and not in this way and you would have a very different
behavior then of course the laws will hold just zip defined from the
monad will take each element from the first list and each element
from the second list and put all those pairs as a result in waste
so for example zip of waste and one to waste of 1020 it would be a
list of 1/10 1:22 10 to 20 so certainly this is very far from the
standard function zip on lists it could be reasonable for some applications
but the standard zip function doesn't do this it does that and therefore
for the standard zip function the correct wrapped unit is this infinite
sequence P not this and so this is an example where the standard zip
function which suggests an applicative disagrees with the mu naught
which is also standard for the standard flat map and disagrees with
that and so that\textsf{'}s but but actually they're both useful in different
contexts so this is exactly the kind of example I was talking about
and the reason that this disagreement might be troublesome is that
if you defined for example code that has a for yield construction
when you might be able to simplify it using zip what if you notice
that you have this kind of construction and you say all that this
is obviously a map - this is a map - or this is a zip like this zip
and this would be a map - and then you will be tempted to refactor
this code by doing something like this but if the zip implementation
disagrees with the Munna instance then you would have changed the
functionality you cannot replace this code with this code unless the
applicative instance defines map to in the way that is exactly the
same as what would follow from flat Mac now if you are using a type
constructor for which the model instance and the implicative instance
disagree then you should never refactor code like this but you might
be tempted to or you might just do it without thinking it\textsf{'}s very confusing
but these things are not the same and so it is recommended therefore
to avoid these situations are not to define applicative instances
that disagree with model instances if you need that the easiest way
out is to rename then a type constructor to some other rename have
an alias or some different type would only have the implicative instance
but not a model instance and the first type would be a monad and you
would only use the first type when we when you use as a monad in the
second when you use as applicative so but in this way you avoid potential
for bugs that would come out of this kind of confusion so finally
let\textsf{'}s look at some examples of non applicative factors so these are
the examples now the first example is the disjunction of two reader
mode ads essentially so reader mode on of course is applicative also
it is this construction but their disjunction is in general not implicative
another example of not duplicative is this this is a functor in a
because a is here in a covariant position so this is to the left of
the function arrow so this entire group is contravariant in the controller
in position but inside it a is also in a contravariant position so
that cancels out and the result a is covariant here similarly here
is covariant and also a is here covariant to the right of the function
error so these are functors but they're not applicative let\textsf{'}s see
how we can verify that so i defined these type constructors i also
defined this G which is I've just reversed these two errors the result
is that it\textsf{'}s a contra factor it\textsf{'}s not a fuck function anymore but
it\textsf{'}s very similar in structure so now these are our three examples
F H and Kane I'm going to try to implement the zip function using
the curry Harvard library so I'm going to ask a Craig Howard library
to implement any code that has this type signature and this is the
Olav type method so I'm going to find all of these so that the result
is going to be a sequence of implementations and then I'm going to
look at those sequences so this is all happening at compile time and
so these tests run now the length of zip F is 1 which means that there
is one implementation of this function which is this type but when
we look at the code which is printed in a short notation we see that
the code is that as function returns always none so it always returns
an empty option a zip function that always return an empty option
is obviously going to violate laws as for instance identity laws well
it won't violate a social Timothy because always returns the same
thing but it\textsf{'}s going to violate identity laws that require it to preserve
information and since the identity laws say that zip of something
with the ident with the wrapped unit must not change the something
so it should not lose any information if zip losses information then
it\textsf{'}s certainly going to violate the identity walls and so the only
implementation we have of this type signature which is the zip type
signature from this funding is going to violate identity laws so there
is no good implementation and for these other factors there\textsf{'}s not
a single implementation at all of this type so those types could not
be instantiated could not be implemented by any code so there\textsf{'}s no
code that\textsf{'}s generic and all these parameters that implements this
type signature and also this type signature but for the G there are
actually two implementations G\textsf{'}s a country factor I just took this
and I reverse the arrows and it turns out there are two implementations
so that\textsf{'}s interesting and so let\textsf{'}s look at contra factors now all
a blicket of puncture laws that are formulated via unzip and wrapped
unit actually don't use the map function so we can formulate the same
laws for contra factors so we can say an applicative contra factor
is a contra function that has a zip and wrapped unit methods with
the same type signature and the same laws identity laws law and social
dignity law so that\textsf{'}s the definition so we will look at constructions
shortly but one thing we need to keep in mind as a contra funders
are different from functions and that they don't have map they have
contour map so for instance you cannot take a wrapped unit and map
it to get a pure out and because you don't have a map you have a contra
map and so it slightly it\textsf{'}s slightly different another thing is that
we always have a function like this just drop B and take the first
element out of a tuple if you control map with this function of CA
then you get C of a B because that goes and you're in the opposite
direction CA with this function goes from here to here and to see
a B which looks like what you want to implement the zip you take the
CA you just contra map it and you get c but that\textsf{'}s invalid as an implementation
of zip because it loses information about CB and we know that would
violate the left identity law if you put identity here you should
reproduce CB on the right hand side but you lost all information about
it so naturally T must hold but with contra mapping set of map so
that\textsf{'}s another difference now if you try to control map this with
a function from one to a you can't it\textsf{'}s the wrong direction you can
confirm map it with a function from a to 1 and you get C away for
any a so that but that\textsf{'}s kind of that\textsf{'}s different from pure but it
can be seen as an analog pure has a type signature a going to F a
this does not have a going to anything you don't well you could imagine
that you have it but you don't use that information so you don't use
any values of type a to create this already can't read them so that\textsf{'}s
yet another difference and also there is no contra app there is cerebral
mind there is no analog of app for control factors so these are the
these are the control factors and indeed we'll see this is one of
the constructions that the disjunction of to a positive control factor
is again a pretty check this is this is this construction which we
will look at shortly so all right so now we verified using the curry
Howard library the these factors are not applicable by trying to implement
the type signature of zip and finding in this way that for this factor
there is one implementation of this type signature but it loses information
and so it could not possibly satisfy laws and for these two types
we found that there aren't any implementations at all for the type
signature of zip now I should comment here that actually is quite
difficult to prove that there are no implementations of zip it\textsf{'}s actually
quite difficult to prove so I'm using the very hard library that performs
exhaustive search of possible implementations but certainly it\textsf{'}s not
really a proof I'm just running some code maybe it has bugs so it\textsf{'}s
hard to actually produce a good proof and I'm not going to do it it\textsf{'}s
it\textsf{'}s if you if you go back to the Carey Hubbard correspondence tutorial
you will see but this is equivalent to finding a proof in the constructive
logic or or showing that there is no proof so there are methods for
showing that there is no proof proof theory gives you tools for doing
this but it is difficult and cumbersome and so that\textsf{'}s why I'm using
the curry covered library in which I have a certain degree of confidence
where I can just ask it how many implementations and given type signature
hands so having finished with constructions we notice that some of
them contain one nodes so you know it\textsf{'}s seem to play an important
role they are very similar to applicative function in some interesting
ways so let\textsf{'}s actually ask what are more node types so what are the
types that I could use here in this construction the answer is surprisingly
in Scala any type ism or not all non parametrized exponential polynomial
types are mono it\textsf{'}s what does it mean we have mono at constructions
and in the previous chapter and I could have made this observation
already in the previous chapter that these three constructions give
new monoids out of previous ones and also that all the primitive types
integers floating points and so on sequences and strings unit all
of these types have at least one way in which they can be implemented
as monoids instances of the monomial typeclass for instance these
are all like numbers they can be added together and that\textsf{'}s a fundamental
operation these are all like sets where you can have an union of two
sets so here is a sequence and union is just concatenation and this
is a set Union and this is a map merging Union but basically they
are set like their MA nodes and strings can be concatenated because
they are grooving to sequence of integers and unit is a trivial annoyed
and case classes you can define using these constructions and function
types you can define using this construction so a function from say
float to a sequence of strings it\textsf{'}s just one of these constructions
applied to one of these types and so they're monoids so all exponential
polynomial types that is all types constructed from these three operations
starting from the primitive types are going to be always one with
at least in one way sometimes in more than one way so here\textsf{'}s an example
consider this type expression in Scala code this would be implemented
as a sealed trait with three case classes because there are three
parts of the disjunction this is a monoid I don't have to worry about
how to implement it because I have these constructions and once I
decompose this into some of these constructions starting from primitive
types I can just generate the monorail instance automatically or mechanically
notice that this does not have any type parameters if I have type
parameters then I don't know if that type is a monoid I don't know
how to implement and the moment instance for it maybe it will be able
know it when this type expression is actually used in my code but
I don't know that so I don't know how to implement and here\textsf{'}s an example
of an envelope type with type parameters so if this were a to a that
would be already a monoid it\textsf{'}s a function one weight with function
composition but here I cannot compose two functions of type A to B
and get a third function again of type A to B no way I don't know
how to combine A\textsf{'}s I don't know how to combine B\textsf{'}s there are unknown
types if I knew that they were more nodes but at least be if I knew
that B is I don't know that\textsf{'}s enough already I have this construction
but I don't know that I don't know how to combine so that\textsf{'}s that is
to say all types that don't have type parameters and our exponential
formula such as this one they're all Minuit\textsf{'}s so it is in this sense
that I say all non traumatised exponential trinomial types and paranoids
very interesting conclusion follows from it namely that constructions
one two six and seven give us a way of expressing all polynomial factors
with monoidal coefficients as applicative factors starting from one
nodes we can build applicative instance for any polynomial factor
so how to do that the short summary of this of this construction is
that we need to rewrite the polynomial in this form which we always
can do isomorphic Li we can transform the type into this form just
like in school algebra this form is called Horner\textsf{'}s scheme for polynomials
so a is the argument and these are coefficients in school algebra
that would be numbers constants and this is the variable in the polynomial
and so because we can write it like this these are just a sequence
of constructions that you apply and you get an applicative function
so let\textsf{'}s see how that works suppose you have a polynomial factor and
which you can write in a short type notation like this you have some
coefficient that is constant type could be complicated but it\textsf{'}s constantly
you know it could be like this but it does not have type parameters
it\textsf{'}s a constant type multiplied by some number of A\textsf{'}s or some power
of a but a fixed number and another monoidal coefficient so we assume
they're all one which is Z and Y because they are of this kind so
they do not have type parameters we can start with the highest power
of the polynomial like we do in school algebra and then some smaller
power and then finally we go down to power one and zero with some
constant types here so this is one way of writing a polynomial which
we can always do another way would be then to put parentheses in like
this so starting from the lowest power we take a out of parentheses
and then in fact factor it out the result is a polynomial of lower
power which we can write again factor out in the same way and finally
in the middle of it there will be some last a and Z Z is the coefficient
and the highest power of the point number so we can always transform
the polynomial into this into this shape into corner scheme and notice
here some steps could contain more than one a so for example it could
be like this because some coefficients could be zero conceptually
speaking or zero type as also something we could use here for generality
but we don't have to each of these steps corresponds to construction
7 which is this Z plus some G of a actually I think it was called
and construction 2 which is multiplying a and G of a where G is already
negative and a Z can be then used like this they can add Z and or
you can multiply by a and that\textsf{'}s still applicative these are the constructions
construction 2 and construction 7 so indeed we can find that it is
a picketing but how does the implicative instance actually operate
so what does the applicative method wrapped unit for example do or
zip what do they do two types like this how can we how can we visualize
those things so here\textsf{'}s how the wrapped unit for construction 7 is
this we have found and wrapped unit for construction - is this and
so the wrapped unit for the entire polynomial is going to be every
time you step through construction 7 you discard the Z so you you
go to the right like this in other words every time in construction
7 when you go you discard this then you keep a discard this you keep
any discard that keep any discard that keep a and finally you're here
so you have discarded everything but Z eh eh eh in other words everything
but the term of the highest power of the polynomial so therefore the
wrapped unit for FFA is of this form is the highest power term the
polynomial where you take instead of Z the unit value or empty value
of the monoi oil and instead of a you put the unit values so that
is going to be the wrapped unit let\textsf{'}s look at how zip is defined so
it would be good if we visualize for example zip of these two terms
then if we can do it then since zip is distributive one zip of this
and some other polynomial essentially is just one zip of this train
because a value of this type must be in one of the parts of the disjunction
so it\textsf{'}s one of these either this or maybe it\textsf{'}s this so it\textsf{'}s sufficient
to be able to compute zip for two monomials like this so let\textsf{'}s see
so construction seven says that the result of these is of type RA
a because construction seven says if one of them was on the Left we
discard what is on the right in other words if we are here for example
we discard this so if two polynomials are zipped together we discard
the one that has higher power so we discard this as discard the power
so we discard the coefficient at the higher power and the result is
the coefficient of the lower power that that remains and we need to
discard so we will go through these and we will discard as many of
these B\textsf{'}s as we need to in order to have a pair for each a so in other
words we'll discard B 3 and before here and we'll keep B 1 B 2 actually
it might be that we discard B 1 B 2 and we keep B 3 before that could
be an equivalent equivalent definition doesn't matter right now we
just want to understand the principle so the principle is first of
all we choose the polynomial of the lower of the two powers we discard
the monoidal value of the other one and we discard the extra values
that cannot be paired up with ours because we need to zip those together
and we return this so we treat them as lists we zip them together
as lists so we cut the longer list when we do that and we discard
a coefficient at the longer list also keep a coefficient and the shortened
list if the two lists are the same length we don't discard them as
we use the monoidal operation from the coefficients and we just zip
the lists as before so in this way we can visualize how constructions
2 and 7 as well as constructions 1 and 6 as far as I remember correctly
yes as we need a constant factor sometimes we need the entity factor
sometimes and we need these two constructions so in this way we have
defined how they work through constructions and we have visualized
how they actually work on specific terms of these types so that certainly
is plausible that you could take any polynomial factor and more or
less mechanically transform it into this way into this form check
that all the coefficients are monoids and then generate noise the
Romero instance for each of them and then generate the applicative
instance for effect here are examples of polynomial factors but are
applicative because they are point omean so the first example is interesting
because this type constructor cannot be defined as a walnut cannot
have a model instance you can define the methods pure and flatmap
with the right type signatures but they will not satisfy the laws
there would not be associative 'ti and identity laws for an unsatisfied
no matter how you try people several people have verified this by
explicit calculations but it is applicative so the implicit of instance
is very easy and just this construction of adding Z to an applicative
factor which is a product of two identity and factors so obviously
this is an applicative factor also it\textsf{'}s interesting to look at this
as a composition of two functions one option so one plus something
is option and then a factor which is the pair a a so that\textsf{'}s clearly
a polynomial factor so the composition of option option of prae is
not am honored but option by itself is a monad and tear a a by itself
is also a moment so this is an example where composition of two units
is not among that the composition in the opposite order will be imminent
the pair of two models one plus eight times one plus eight that is
a moment but not in not composition in this order and in this example
is just polynomial is just to visualize what I mean by polynomial
with monoidal coefficients it\textsf{'}s like a polynomial but the coefficients
must be unknown so any types that have no parameter a in them so like
the Z that must be a monument now this this is a one that this this
factor I believe is a Mona but it\textsf{'}s not obvious how you need to find
Mona constructions but for example we have here a writer Monette obviously
and then you you have a plus so you multiply identity with the writer
Monat so that is a product construction for bullets and then you add
a that\textsf{'}s a three-pointed construction and that\textsf{'}s also a moment so
through finding a sequence of constructions starting with a writer
mu naught which we know is a minute we can show that this is a moment
without explicitly having to prove the blows hold and notice that
our examples of non applicative factors were all non polynomial so
they all had function in them christmas they're not polynomial factors
indeed there aren't any examples of polynomial but not applicative
factors that have no type parameters and only monoidal coefficient
is very easy to drop in something like this and say all this is not
monoidal and if i have a coefficient like this then obviously I don't
have applicative factors so that\textsf{'}s certainly would be an example so
polynomial factor with non monoidal coefficients that\textsf{'}s very interesting
perhaps but still it\textsf{'}s a valid example so let\textsf{'}s continue and take
a look at Concha funky constructions I already described how Concha
front was are defined so in the next and final portion of part 2 of
chapter 8 I will talk about the applicative control under constructions
as well as Pro factors and applicative profounder constructions 

the first construction is the constant factor where the type Z must
be alone right now this is exactly the same as we just had here against
construction six and in fact construction six was examined and we
have found construction six like this with we have verified the walls
but we did not actually use the fact that the type constructor was
considered to be a factor so we could have defined exactly the same
code without using this typeclass instance of furniture and in fact
I have defined also typeclasses that I pass control Mujib which is
the same methods except it expects a contra factor instead of factor
but other than that it\textsf{'}s exactly the same type signatures so the result
is that we actually don't need to prove any more than we already have
the previous proof goes exactly the same word for word for a contra
function here as it went for the function because we never used the
fact that we consider this type constructor to be a factor of this
type constructor is of course somewhat trivial it does not depend
on type a but never but this exactly the same logic will apply to
many of our constructions that we just considered four factors for
example the product construction is exactly exactly the same kind
the product construction is defined here for a factor but we actually
don't use a factor instance as a constraint when we define the zip
and the wrapped unit and if we look at the truth or associativity
and identity laws we never use map in any of these proofs we actually
never assumed that G or H are factors we assume that for example equivalences
can be established between these these values but these grow answers
can be established for contra factors equally easily as four factors
so for example for a contra functor if you if you want to compact
if you want to convert for example C of a C into c of a b c like this
this is an equivalence that is required in proven social TV t for
certain functions on constructions what you need is a function that
goes the opposite way from here to here and of course this function
exists it\textsf{'}s a trivial reordering of the tuples or as for functor we
needed a function that went in the other way but also that function
was clearly available so there is not another problem going through
exactly the same proof and just substitute in this kind of equivalence
which is exactly similar whenever we need equivalence between list
of tuples and also the equivalence between a tuple with unit value
and a single value which is necessary for it you know the identity
was the non-trivial construction is this one because there is no analogue
of this construction for applicative functors or for Mullens product
matter as we have just seen in general the disjunction of to placated
factors is not applicable in all cases and this junction of two monads
is not abundant in all cases so let\textsf{'}s look at the implementation of
this construction we use the cats library for contravariant which
is their name for contra factor and we define this type constructors
in either of G\&H assuming that both G and H are contravariant and
this is also controlling it this is how we establish that we define
continuity so contra map should map A to B using a function from B
to a well this goes exactly like in the functor case except we use
Concha maps instead of apps so if we're on the Left we use left and
we map in the G country furniture where on the right we return right
and we map in the H control factor so now we can use applicative instance
and we will use both contravariant and contract lucrative typeclasses
we actually will need that here\textsf{'}s how it goes so the wrapped unit
first needs to be defined so that\textsf{'}s a value of this type now here
we could return the left with wrapped unit of G or we could return
a right with wrapped unit of H and actually this choice is arbitrary
it could return a left or we could return a right we could then define
the zip accordingly and laws would hold the reason this is so is because
this construction is completely symmetric there is no difference between
G and H and so if we are able to define things with this choice and
just by swapping G and H we will be able to define this construction
with the right choice so let\textsf{'}s make this choice arbitrarily so that
we're on the left with wrapped unit so now how do we define a zip
we need to transform this and this into this if we are on the Left
then clearly we just turn the left we have two left G of a and G of
B we can just zip them together using the zip from the G country factor
and if we're in the right if both of them are on the right we can
just zip in the H country factor so that\textsf{'}s clear now what do we do
if one of them is on the left and the other is on the right well we
need to return either left of G or right of age what do we do well
we have now seen two cases when we had a disjunction and we implemented
the zip function for this Junction in each of these two cases what
we had to do is that if the wrapped unit was on the left then the
mixed case needs to be on the right if the wrapped unit wood is on
the right and the mixed case needs to be on the left and that was
kind of the pattern we have seen in the previous two examples so let\textsf{'}s
follow that pattern and we'll see that this actually works so how
do we return the right of H a B now we have only an H of B and we
have also a G of a now we can't possibly combine G and H there\textsf{'}s no
method for that so we have to ignore G and we have to transform H
of B into H of a B but that\textsf{'}s possible H is a control factor so we
can confirm mark it with this function that transforms a B I call
that x and y here but it would be easier negative read the code I
called it a and B are because then the types will be more clear so
I always have this function that transforms a B into B it actually
ignores eight so I can kill that and that function transforming a
B into B contri Maps HB into H of a B and that\textsf{'}s what I need I need
to transform HB into H of a B and so that\textsf{'}s why this code has the
right type and I do the same thing in the other mixed case I return
a right of H a and I contour map it like this so I transform H a into
H of a B so having implemented it let\textsf{'}s check the laws how do we reformulate
it so if we consider this kind of combination if all of them are on
one side so both all three on the right or all three on the left then
we we can do immediately we can see what happens it will be either
on the right of h HB HC whole zip or on the left of GH g BG c all
zipped so that\textsf{'}s clearly associative because we are now doing zip
into in the factor H or an F and Q G and that\textsf{'}s associative by assumption
now if some of these are on the left and others are in the right when
according to our code all the left ones are ignored and all the right
ones are control mapped so that they have the right type and the Contra
mapping is just with the trivial substitution of tuples so clearly
all the right ones are going to be just converted to this using the
trivial konchem up and then zipped together and so the result would
be something like this it will be a contra map with this where what
say C was on the left so it was ignored and a and B were on the right
so they were not ignored and then we we need to zip so that\textsf{'}s associative
because the condition that we are ignoring all the ones on the left
that condition is independent of the order which is no matter what
we add parentheses we put in first around these two or around these
two the condition of dropping all of these that are on the left that\textsf{'}s
associative not independent of the order of parentheses and then we
will come to map it finally into this type unzip them all together
{[}Music{]} that\textsf{'}s associative as well for with the zip ages out of
place here now identity laws are actually checked using a cop as an
explicit computation because you cannot just argue about it a lot
of some symmetry consideration we have to actually compute and verify
that our intuition was right that we we had a left here therefore
we need a right here - let\textsf{'}s check so let\textsf{'}s take some arbitrary FA
and zip it with the wrapped unit which is this and the result is that
we need to match according to this code you can match like this not
only two pieces remain because we have a left here if they're both
on the left we do a zip now this one GB is actually this so that\textsf{'}s
going to be equivalent if it\textsf{'}s on the right then we're actually ignoring
this GB according to our code we need to ignore this and we are on
the right we do a contour map which is like this and this is just
a contour map that is the isomorphism or the equivalents that we allow
so that\textsf{'}s the equivalence expressed by this symbol as I'm using it
and that\textsf{'}s equivalent to right away J cuz contour map with this isomorphism
is precisely the equivalence and so then right away J here we have
right over J here so identity was cold so clearly this would not hold
if we didn't have a right in this in these two places so in this way
we verify this construction now this construction says that for any
factor and applicative contrapuntal G this function is applicative
as contractor let\textsf{'}s see how that works this is the type constructor
so as a type function it\textsf{'}s contravariant because obviously this is
a factor by by assumption H was the front and this is a control factor
so the concha functor is here in a covariant position and functor
is a in a contravariant position so the result is contravariant to
implement that contravariance is easy you return a function which
takes HB now you have to compute some GFP so how do you do that you
can get G of B if you first get G of a and then control map it with
this so how do you get G of a you just substitute into a fey some
HIV but how do you get H of a you take H of B and map it with F because
H is a factor so that\textsf{'}s what we do we map HB with F substitute that
into FA and then confirm map the result with that and this could be
generated automatically for us now let\textsf{'}s implement the zip now we
are using the function instance on H to do that but we're actually
not using a contravariant instance on G for this let\textsf{'}s delete this
we are not using concern up in this code we're using map on H so we
need the function constraint on H but we don't need contraband two-factor
constraint on G for this code the wrapped unit it\textsf{'}s a value of this
type so how do we generate a function of this type role we can take
some H of unit but there\textsf{'}s no way for us to use that H of unit to
make G of unit we already have G of unit anyway it\textsf{'}s the wrapped unit
so let\textsf{'}s ignore the argument and return that wrapped unit how do we
do this so that\textsf{'}s a bit of a complication we have an H of a - G of
a H of B 2 G of B and we need to return this function so let\textsf{'}s return
this function so we return a function that takes HJ b and returns
G a B so how do we get a GLB well clearly we need GA and GB for that
so we need H a and H B so that we can substitute those into F a and
every how do we get h + HP while we take H a B and project out side
B we drop the beat so that\textsf{'}s just a map with this function look drops
the be out of a tuple we could write this function more concisely
like this but anyway so that\textsf{'}s how we do that so now we get H a I
get H beam and we obtain G of a we obtain G of B and we zip them in
the country function G we're done what type is correct what seems
like it was the only possibility to implement these types let\textsf{'}s check
the laws so here it\textsf{'}s hard to reason about these these values in some
hand waving in fashion and and we formulate them explicitly it has
when you firstly applicative I'm sorry manifestly associative it\textsf{'}s
it\textsf{'}s hard let\textsf{'}s write down the code so let\textsf{'}s consider this expression
first so this expression has this type from this to this we write
down those things we just in line just in line hjb map 1 HK be mapped
to I in line them so let\textsf{'}s now zip this with FC so that would be more
complicated we have FA zip FB and then we have apply this function
because this is this 2h a BC map one zip FC h ABC map 2 so if we just
substitute the definition of zip again and we have this expression
now notice we have here H ABC map 1 map to map one map 1 map 1 map
2 and H ABC map to now our H ABC is actually of this type so it has
a nest tuple let\textsf{'}s simplify that let\textsf{'}s map this each ABC as the flat
tuple and convert it into our non flat tuple first so we convert it
and then we would have these expressions which are basically just
taking first element of this tuple which is this then again taking
the first element which is this so this is basically projecting ABC
onto a which is to be equal to that so that simplifies our expression
into into this and then we contra map the result in in this way so
that we know well that that will transform the nested to fall back
into non-listed since it\textsf{'}s a country map so here\textsf{'}s the code that results
from this operation we start from each ABC which is flattened and
then we I just substituted all of this country mapped at the end and
then I have simplified these things so I have F a of this zip FB of
this zip FC of this and so I first have to zipped together and then
a zip it with this FC now the last step is actually zip in the G contractor
because this is zip between values of F of Sun and F is a function
from H to G so this zip is associative and I'm just applying some
kind of a isomorphism which is equivalence don't care about that I
can always apply it whenever necessary and so the main result is that
I have here at this expression that is manifestly associative by assumption
because this is a zip in that country functor G and everything else
is perfectly symmetric so if a HIV seen that one as being HIV seen
up to FC hn be seen up straight so there\textsf{'}s no asymmetry and so I expect
when I start with the other order of parentheses to get exactly the
same thing with different parentheses and so then because of associativity
of G these two are equivalent and these two rows I can always add
or remove whenever necessary so that verifies the associativity law
and let\textsf{'}s look at identity the identity law is that this zip of some
arbitrary FA with this function that we define that always returns
wrapped unit of G ignoring its argument so let\textsf{'}s find out what that
does so the code is like this so we have some H a and H beam that
we define right here and then we do FH a zip F PHP now HB is actually
this but it\textsf{'}s the function FB that is acting on it is this function
that ignores HB so it means actually ignored and result is always
this so the result is going to be fa h a zip this now this is the
zip in the g control function which we assume satisfies the identity
law and therefore this is just mapped into G of a into G of a unit
which is the awesome orphism and so basically we start with function
FA and we have a function that takes H a B maps it into H a and applies
FA to that so that\textsf{'}s basically if you look at how its how its map
its mapped using this isomorphism which is going to be the isomorphism
between a common unit and a that\textsf{'}s our equivalence and so that\textsf{'}s basically
H a B is equivalent to H a and therefore and so FA of H a is the same
as FA of HIV well of the equivalents and so we take FA and we return
a function that takes H a and applies Ephrata HJ so we take a favorite
turn effect so that\textsf{'}s identity so therefore this returns a function
that\textsf{'}s equivalent to FA up to there is some morphisms that we have
such as this one and so the left identity the right identity holds
that we just found and left identity holds in the same way just H
beans to the AJ so so much for this construction now this construction
is a functor G at the contrapunto edge so if we look at the corresponding
construction for functors this one you see we are not using the funk
terminology you're using functor on the edge actually I think that
is a subtype this is a mistake so we use a map on on G G must be a
functor but we are not using map on H so we need to we can delete
this we don't need type constraint constraint that each is a factor
and when we check the laws we use map but always on G so we we take
out these maps but these are values of the function G so we never
use map on H and here also we use a map but only for G you never use
map on any values of H and because of this the proof that we gave
for this construction for functors where G was a functor and H also
a functor now G must remain a funkier because were using map on G
all over the place but it doesn't have to be a frontier it could be
a country functor so this construction is very similar to this one
and the proof goes through exactly the same because we don't actually
need the functor property of H we only need the lucrative property
but not the functor property so we are not using map or country map
on each hello they're only using zip and rapped in the proof will
go through exactly the same word for word so these are the constructions
that I was able to find for country factors now the interesting thing
here is that we have constructions that have constant factor we don't
have identity factor because it\textsf{'}s a con is not a country factor it\textsf{'}s
a factor but we have a constant culture factor we have product disjunction
or some and function or exponentiation so we have exponential polynomial
constructions all of them are here we also have composition but this
is not necessary if we have these three constructions we can come
we can build up an arbitrary exponential polynomial culture factor
with monoidal coefficients as long as we have these all the constant
types that occur must be money or it\textsf{'}s so this is what means to have
constant coefficients for exponential polynomial country factors and
then we have covered all possible country funders so essentially all
exponential trinomial country factors with manorial coefficients have
an applicative instance through these constructions so as long as
we found any exponential polynomial country factor with monoidal coefficients
we know it\textsf{'}s applicative there\textsf{'}s no there\textsf{'}s no question and no counter
examples of non duplicative such country factors so this is very interesting
so we did not have this four factors that all exponential point normal
are factors but that\textsf{'}s the example I have seen so I was trying to
do counter example so this is a counter example four factors this
is not applicable but if you reverse these two arrows you get a contra
function and it\textsf{'}s applicative and that we have seen that it it had
implementations and we know why it\textsf{'}s this construction so the conclusion
is that all the clickety of contra ventures are basically you can
write down are going to be exponential polynomial contra factors and
vice versa all the exponential polynomial contra factors are going
to be negative now let\textsf{'}s talk about true factors in the first part
of this chapter we have seen type constructors that are not functors
but applicative and these were option actually also not control factors
and there are two main examples that we have seen one was the typeclass
from annoyed and the other was the fold the type transfer fold or
rather the type just the data structures were fold we had full diffusion
and the data structures were fold the first variant of it was neither
a function or a contra factor and yet it was applicatives we could
have an negative Combinator for it so those are proof factors informally
speaking pro functors are type constructors that have the type parameter
and the type parameter occurs both in contravariant and covariant
positions because it occurs in both of these positions you cannot
have a map and you cannot have a contra map for this type parameter
here\textsf{'}s an here some examples typically this would be an example so
you have a function from this to this and the type parameter occurs
both in contravariant and covariant positions another example you
have a disjunction or some type sum of a and a function but so here
a is covariant and here is contravariant and so these are typical
Pro factors what would be an example of a non-pro factor no clearly
anything you can write using exponential polynomial operations there\textsf{'}s
going to be a pro factor because the type type parameter is going
to be either on the right or or on the left of the function in room
and the only way to have a non-pro factor is to have a non exponential
polynomial type constructor an example this would be a so-called generalised
generalized algebra data type which I find not a helpful name but
these are basically I would say these are type functions that are
partial partial type functions we have seen that concept in the chapter
on typeclasses these are just partial type functions they are not
defined on all types they're only defined on certain types in in some
specific ways and here\textsf{'}s an example in Scala so you have a trait which
is parameterize by type a but there are only two case classes that
that implement this trait and they have specific type parameters values
here int and unit and so it means you cannot instantiate a value of
type say F of double it\textsf{'}s impossible you can only instantiate F of
int and F of the unit and as a result you also cannot instantiate
F of a for arbitrary a and so there\textsf{'}s no hope for you to have napkins
from the app or anything like that because map and confirm happily
require you to be able to transform F of A to F of B for arbitrary
a and B and you can't even instantiate those F of a and F would be
for arbitrary amb and so no hope for that kind of property for these
types so these are partial type functions no hope for them to satisfy
good properties I'm trying to invent a good notation for these kind
of type functions I'm not sure this is a good notation but this basically
means a string but it is ascribed that this type F event and an int
but it\textsf{'}s considered of type F of unit just by by hand we just say
this is a for unit and this is evident maybe this notation is useful
but in any case at this point I have very little to say about these
partial type functions except that they are not true factors this
is an example of a non-pro factor now notice I have not yet actually
give it even a good definition of a profounder I've Illustrated what
I wanted to achieve these kind of things are pro functors and have
a in a certain position but this is not a good definition because
it depends on being able to write a type in a certain dessert away
and what if we don't know how to write it in a certain way maybe what
if this is actually a profounder just I don't know how to write it
correctly we need a better definition a rigorous definition there\textsf{'}s
a typo I'll fix it a rigorous definition or a profounder which is
that it is a type function with two type parameters such that it is
a contra filter in a and the factor in B so a is contravariant b is
covariant if such a type function exists that PA is defined like this
when we put a and to be the same type then that type function is a
pro factor so for each Pro functor there must be a way to split it
into a function of type function of two type parameters one of them
is strictly called contra variant and the other strictly covariant
and if you can do that then defining P like this then P is a contra
fun it is a pro factor that\textsf{'}s the definition of a pro factor so obviously
we can do this here we can say well this is going to be a and this
is going to be B in the Q so we define Q of a be like this so Q is
1 plus int times a going to be and here we put here be because this
is covariant and here a because this is contra here so that will become
the Q that corresponds to the P so basically the idea of this definition
is that a pro functor is a type function where really you can split
it into a function with two type parameters each purely covariant
or purely contravariant and that\textsf{'}s and then then obviously for the
contravariant you have a conscience where the covariant you have a
layup with the Loess separately holding for a and for me with a usual
laws and since you have that then you can apply something called X
map which is just a combination of map and contra map map in being
in control map in a accepted then after that you said the same type
triangle the result is that in order to map a profounder from one
type to another you need functions from A to B and from B to a going
in both directions if you have that and you use this in the map use
this in the Contra map for Q it\textsf{'}s a bit confusing here I use a and
B both for this and for this so I will fix this here I would use this
as x and y instead of a you'll be on the temp side x and y then PX
is and where P a is Q a h2o but this is going to be x and y so then
this end B plays very different role from this so because we assumed
that Q exists and so it has a map and a contra map then obviously
P will have an X map which is called in the scholars Eli rains the
cats lair is called I map not sure what is a good name but let\textsf{'}s go
let me call it X map so this will be defined this follows from a definition
and the laws identity and composition laws oops there is a mistake
must be G 2 followed by G 1 because their composition for the contravariant
art is the opposite order okay I'll fix that in the slide so this
is G 2 and G 1 the important idea is that once you have this cue it
means you already have map in B and country map in a and the ones
for those called separately and so it can be derived but these laws
hold because they're just positions of those functions and so therefore
these laws can be seen as consequence of this definition and it\textsf{'}s
not a new a new assumption or new requirement it\textsf{'}s a consequence of
the definition and so that\textsf{'}s why we believe it\textsf{'}s reasonable to impose
these laws so as I said all exponential polynomial type constructors
are pro founders because in all of them there will be type expressions
of this kind the type parameter must be either on the right or on
the left and so then you can easily relate the covariant and contravariant
by different type parameters and then we can define the Q that you
are required to have and then you have your pro functor defined and
you prove that easily just applicative proof functor is defined in
exactly the same way using zip and wrapped unit with the same boss
there is no corresponding app method but there is a pure method from
a to P because you can take wrapped unit which is p1 and you can X
map it with a function from a to one and the function from 1 to a
which you will always have since you have an a then you have a function
from 1 to a and the function from a to 1 you always have and so you
can X map your rapped unit from P 1 to P a given a so that\textsf{'}s the pyramid
but you cannot derive app or any analog event I believe so what are
the constructions all the previous constructions still work because
profounder is a superset of both a functor and contra factor is any
factor is a profounder any control hunter is a pro factor is a superset
of both of them so all of the constructions we had before also cold
as proof under constructions in the trivial sense however there are
more construction so the product construction again then there is
the monoid addition then there is the free pointed and a function
now a function works and all these constructions work in exactly the
same way as they work for the factor and this works in exactly the
same way as it works for contra factor you see here it\textsf{'}s a functor
here it\textsf{'}s not it\textsf{'}s not control factor it\textsf{'}s a factor it\textsf{'}s important
it doesn't work doesn't seem to work for control factors really confer
pro factors I tried but I couldn't make this construction work I could
implement the types of zip and the wrapped Union but I could not get
associativity law to hold for for this kind of construction so I don't
think this construction works but these constructions certainly work
in the composition also in order to find that they work you don't
need any more proofs actually you just look at your old proofs and
you find that you aren't using the property of G as functor or contra
factor or anything you don't use map on G when you do these things
you don't use contra map either in this construction we use pure but
you have pure for the pro functor in this construction we use only
zip and wrapped in it and nothing else with this construction you
only use zip and wrapped unit in this construction you only use a
functor for F but you don't use anything for Q except wrapped in it
and zip and similarly here and so all these constructions actually
go through with no further proofs necessary I was able to find out
that these constructions work by pretty much try on air I hope I found
all the important constructions and that exists improves so for example
here we don't use we use map on H but we don't use anything on G we
only do mark on each not never on G so I'm pretty sure that this construction
does not work and I'm pretty sure these constructions all work because
these proofs do not use properties of pro factor other than zip and
wrapped unit let me just check the sum of a and govt so this this
is construction that uses pure so let me just look at the proof of
that construction here this it does not use the functor instance on
G just uses the zip and rap unit there\textsf{'}s never any map there\textsf{'}s a pure
so we need a pure on G that\textsf{'}s true and it needs to satisfy identity
law that\textsf{'}s true other than that we do not use map on anywhere so we
we use the equivalences yes but those are available for pro factors
in the same way they are available for country factors we can for
example use x map and rearrange tuples in the forward direction in
the backward direction and then this will give us a rearrangement
of tuples in the type of the pro factor so yeah so to summarize these
are the constructions i believe exist for applicative functors contra
hunters and cro factors now i'd like to have a little comment on an
interesting property of applicative factors which is symmetry or commutativity
but before symmetry because it\textsf{'}s not really committed to beauty literally
speaking monoidal operation can be commutative or symmetric with respect
to the Pregnant\textsf{'}s it\textsf{'}s probably better to say cumulative it\textsf{'}s just
this property now if you want to apply the same property to zip so
you I would use this notation for the zip operation you cannot literally
say it FA of being must be equal to if BFA because there are different
types so you need to map the types by rearranging the types because
this is going to be of type F of a B and this is going to be of type
F of B a but if you implicitly say that this isomorphism is included
in the equivalence and you could write it like this so that is the
same symmetry property not not all applicative factors are symmetric
what does it mean that it is symmetric well it means that the effects
somehow are independent in such a way that the second effect is independent
of the first now in the implicative factor the second effect is independent
of the value returned by the first container of the value it is an
independent but it may not be independent of the effect and if it
is so then this symmetry will not hold but there are some examples
where it holds for example list is symmetric manifestly so because
you can just permit the list dominance a good example of non symmetric
applicative frontier is parsers we have looked at this briefly parsers
are not symmetric because when you do applicative composition of parsers
the first parser already might have consumed some part of the input
string and the second parser starts from the place left over by the
first parser well the first parser stopped and so the second parser
depends on this effect now the value returned by the first parser
does not indicate necessarily a position where it stopped the value
is the value it parsed out of the input and the second part is independent
of that value but it depends implicitly on where the first parser
stopped and so the parser combination using applicative composition
would not satisfy this requirement of symmetry also if you define
anything through the mu net most likely it\textsf{'}s not going to be symmetric
well for lists we know it is not symmetric we have seen that the order
is different in the result when you define zip through through the
moment but we know that for lists the mu not defined zip is not usually
what you want you want applicative defined zip which is incompatible
for lists all polynomial factors with monoidal coefficients that are
symmetric will be symmetric applicative entries because we have seen
how polynomial factors combine their values and then so the minute
they put commute these elements in a different order if you change
the order of these two so the moon at zip is not symmetric now the
polynomial one is symmetric because it basically combines elements
like this so if you put this first it doesn't matter it will still
take these two and combine with these two so similarly here so the
typical polynomial factor with symmetrical B with symmetric when one
of the coefficients like integer which is a symmetrical mono at boolean
of symmetrical node with strings and string concatenation is not as
commutative monoid so that would not be commutative and so the polynomial
factor with strength coefficients would not be a symmetric or neither
or commutative may be gathered aside but I've seen usage of the word
symmetric so I would say the commutative or symmetric negative funky
will be probably the same meaning now it\textsf{'}s interesting to say that
first of all most of our constructions preserve symmetry so if if
you think about how we defined all these constructions of the last
portion there was deep in this functor and it was symmetric in how
you provided arguments for this zip so if G is symmetric then this
is also symmetric and G is symmetric this is symmetric if G and H
are symmetrical commutative then this is also commutative and so on
so there are some constructions that would not be symmetric usually
coming from units but all these constructions are symmetric so if
you have symmetric coefficients and if you have symmetric applicative
factors and you combine them using any of these constructions that
we have seen and you get again symmetric wickety factors and control
factors and pro functions work exactly the same way and they have
the same commutativity property because well we have formulated the
cumulative 'ti with symmetry property using this which is a map but
if this were a contra factor that we would just use contra map here
and this function is an isomorphism so it is available in both directions
and so we can use X map as well because we have this in both directions
commutativity or symmetry makes it easier to prove associativity for
negative factors I have not used this in my proofs but mostly because
I have specific instructions I didn't want to assume symmetry but
if you had a commutative applicati factor that you could just rearrange
this into this by permuting and and permitting this with this so you
have that which is almost what you need to prove associative you just
need to swap FC and FA and so that\textsf{'}s perhaps much less work to do
that we just compute this and then you swap FA and I've seen it and
you demand that the result be the same up to a swapping of types and
that\textsf{'}s less work for symmetry for symmetric code then another wise
that\textsf{'}s otherwise you'd have to first compute this separately then
you compute this separately but in most cases it\textsf{'}s significantly easier
to prove I think it was proved that we can see so I would like to
finish this tutorial with an overview of standard filter classes using
category theory so strictly speaking this is just kind of a theoretical
perspective which doesn't add much practically to programming it\textsf{'}s
more or less just a justification as a an explanation of why these
factors exist why these specific properties exist and why we have
assurance that we have the right laws for them and here\textsf{'}s how it goes
so consider typeclasses such as functor filterable monad applicative
contra functor and so on each of these when we consider them depth
the laws for them we found a function with type signature that looked
like a lifting and by lifting I mean it was a function of higher order
that a function like this and produced another function and the function
on the left was of one type and the function on the right was of a
different type and on the right it was in the function f so always
this lifting took something and produced a function from FA to FB
so it took this produced a fatal FB took this was his flat map produced
a fatal be missus app it took this produced a fatal of me contra factor
took this produced a fatal FB this should be X map rather than Dyna
prevented own I will fix this in the slides it took something it gives
a fate of B and so on so this is what I mean by lifting it is this
kind of functions type signature that takes some kind of function
as argument and returns this kind of function FA to FB let\textsf{'}s look
carefully at the types of functions that are taken as arguments by
these liftings all of these our functions of a types contain a and
B their functions from something to something that contains a and
B in some way this is what we consider to be a category in the previous
lecture namely it\textsf{'}s a description of what is the type of a function
that we twist so we kind of twist a function type we don't just take
a to be like here we twist it a bit we add something on top of it
to be like or we can even reverse the area we don't have B to a instead
of A to B or we have a to F of B or we have a to one plus B or B to
1 plus a or somewhere so it\textsf{'}s kind of twisted it\textsf{'}s a twisted function
type and so all these liftings take a twisted function type and give
me a function type in the functor so functor transformation second
thing we notice is that when we looked at the laws that need to be
acquired for these typeclasses each of these strange twisted function
types had identity and composition laws there was always a way to
compose two of these and get another one of those so for example we
composed a to B and B to C and we got a to C we can post a to one
plus B beta 1 plus C and we got a to 1 plus C we can pose a to FB
b to FC we get a 2 FC we can pose F of A to B and F of B to C we get
F of A to C and so we compose functions of this type also we had an
identity function of this twisted type here it was the ordinary identity
this was a pure option that creates so 8 goes to 0 plus a so it creates
a sum away this was the pure function of the mullet this was the pure
of identity this is identity we did not consider those things in this
way but they have exactly the same structure there is an identity
function of this twisted type and there is a composition for functions
of twisted type so in our simplified definition of category that\textsf{'}s
what category is is a twisted type for functions such that we can
compose these twisted functions and we have a twisted identity so
here in applicative i noted it like this this is the twisted identity
and this is a twisted composition and so for example a function of
a value of type f a to b can be composed with other values of type
say F B to C and you get a value of type F a to B and F a to sir sorry
so that\textsf{'}s the general scheme of things it seems for each of these
examples we have this twisted function type a twisted composition
of these twisted function types and the twisted identity and the laws
of identity and composition hold in other words composition of identity
and something is equal to that something and composition is associative
these are the axioms of a category the category laws so to speak a
category is described by saying what is the type of the twisted function
that it has a twisted function is also called morphism so what is
the type of morphism that the category has this type should somehow
depend on a and B so there must be a and B in this type somewhere
other than that we don't know what that type is it could be many different
type expressions having a and B in it and it better be some kind of
function it could be like this as well it\textsf{'}s not clear exactly what
qualifies is fun is this a function we don't know one example of applicative
is a monoid with no dependence of type parameter that\textsf{'}s what the functions
prickly speaking but it doesn't matter a morphism is just a type such
that you have an identity element of this type and the composition
of elements of this type and the type is indexed by two type parameters
a and B what\textsf{'}s all it needs to be it\textsf{'}s just the morphism is a type
expression depending on two parameters and that\textsf{'}s it each category
has its own different definition of what that morphism is some categories
have names this category is the plain type category which is not twist
it\textsf{'}s just a function it was not Christian so this is the category
of ordinary functions this is a class Li category of the function
f this is the class the category of the font option so it\textsf{'}s a specific
example actually of this with the option is a specific F this is the
applicative category but that\textsf{'}s just what I call it there is no accepted
name for it it seems this is the opposite category category that all
functions with opposite type signature I don't know what this is called
this is the opposite to option class Lee so the generalization that
category theory cell gives us here is that all of these typeclasses
are seen in one kind of way you define them by specifying the type
of the morphism defines the special some special category that you're
interested in automatically you demand laws the category laws is also
you specified identity morphisms once you have specified this and
demanded that laws of category and also natural T there must be in
all of these naturality laws which relate like we have here in here
the naturality laws which relate this twisted composition and the
ordinary usual F map so that must be their natural T because we are
now we need to always relate this category and the category or working
with so but once you specify these was which are always the same always
the same laws so they're not different somehow in each in each case
there are the same laws just for a different category it\textsf{'}s a very
powerful generalization where you basically say everything is fixed
except for the type of the morphism and you can choose that type in
many different ways and you get different typeclasses automatically
with all the period clause automatically chosen for you remember the
filterable functor had four laws which were kind of ad hoc if you
look at them like that moon had also had laws that we kind of guessed
applicative we kind of guessed what these laws must be but if you
look at those from the category point of view they're not at all arbitrary
once you specified this type expression for the morphism type or the
twisted function type everything is fixed all the laws are fixed you
give you derive all the possible constructions for the factors that
satisfy these laws all the laws for example for zip and wrapped unit
are equivalent as we have shown to the category laws for this category
and the same thing we have shown in previous chapters about filter
balls and moments the category laws are equivalent to the laws obtained
previously in terms of different functions so because the category
laws are always the same it\textsf{'}s just that the types are different we
have assurance that we found the right laws that all these classes
are somehow correctly defined we didn't forget some law for filterable
also we didn't have too many laws we have exactly the laws we want
and that\textsf{'}s one thing the second thing we we obtained from the systematic
picture is we can try to generalize and find more typeclasses I mean
we have found these typeclasses are there any more have we forgotten
some interesting typeclasses how do we generalize well in this scheme
the only way to generalize is to change the type of morphism in the
category and let\textsf{'}s see we had a to be we had a to F of B we had F
of a to b but we didn't have F of a going to B we didn't have that
why not indeed excellent question and if we consider that as the type
of the categories morphism and demand that these functions twisted
functions or morphisms have a composition law and identity one we
obtained something called the comonad we have not yet considered comonad
and we did not find any other motivation to consider it but this is
a this is a motivation that kind of falls out of this consideration
so we have a to be it says a to FB with some specific f we could take
some other specific F and see what happens then maybe it\textsf{'}s not something
interesting and I've generalized filterable so from option to something
else maybe it\textsf{'}s useful maybe not monad applicative now doesn't seem
to be any other way of putting F on here we already exhausted everything
we have a to be we have F a to F of B F of a to b f a to B now let\textsf{'}s
reverse we have B to a we have B to one plus a what about B to F of
a what about F of B to a and what about F B to a I tried some of them
and they don't work so actually this scheme does not always work so
you can't choose an arbitrary type expression here and expect it to
work in what in what says it do I say that it does not work in the
sense that you cannot find a composition law for this category that
would at the same time fit with this scheme so because what you want
is not just some arbitrary category with some arbitrary composition
law but you also want to lift it then to your functor and that\textsf{'}s the
that\textsf{'}s what doesn't work so you basically find that there aren't any
factors that satisfy this property so another thing that is not great
about it is that some typeclasses don't seem to be covered by this
scheme such as for example contra applicative and pro functor replicative
i haven't been able to find a formulation of them using this scheme
so maybe there is a trick that I'm missing but right now and especially
since for example contract leakages don't have an app they don't so
you cannot write like this you know you could think contra black Egyptians
contra functor means this category and applicative means this category
contra placated means f of B to e no so you can't have this kind of
thing with F of B to a so the conclusion is we have a very interesting
and general scheme that shows that there are some typeclasses like
these that are in some sense natural there\textsf{'}s some sense they are all
part of one approach which is define a category lifts from that category
morphism to this morphism for some functor f derive the laws from
the category laws we must have here liftings laws our function was
they're fixed as well I didn't talk about this but the laws for this
are fixed because it\textsf{'}s of just a functor from one category to another
and so the laws are identity must go to identity composition must
go to composition so again no freedom in choosing the laws we have
fixed laws they're fixed laws here we derive then what are the properties
of the function f such that these laws hold and we formalize this
as a typeclass so in this way we can make a case that all these classes
all these typeclasses are in some sense standard they are all obtained
from the same method and their laws are fixed so there we have good
assurance that we we have found the right louis and the right typeclasses
and we can generalize so communal is one example that follows from
this with this choice of the morphism type some possibilities like
contra monad you know when you try to do moon ad and do B to F FA
for example it doesn't seem to work and some classes like contra functors
applicatives don't seem to be covered by risk but other than this
seems to be a very powerful generalization and an elegant way of conceptualizing
and justifying that the laws are correct and understanding why the
laws must be like that and how to analyze those factors and so for
each of these types we have thoroughly analyzed what factors have
the properties that they have that they must have and in the later
tutorial we might do the same for como nuts so this is what category
theory brings it\textsf{'}s kind of an conceptual generalization it is not
so much in terms of specific code that we couldn't write until we
saw this table but in a certain sense it shows a direction in which
we could go on and that\textsf{'}s so far what I found category theory gives
so to finish off this tutorial here are some exercises for you along
the lines of what we have been doing and some of these exercises require
proof and others don't indicate where so here you do need to prove
and here you don't here you don't and here you do good luck so this
concludes the tutorial in Chapter eight 
\end{comment}

\global\long\def\gunderline#1{\mathunderline{greenunder}{#1}}%
\global\long\def\bef{\forwardcompose}%
\global\long\def\bbnum#1{\custombb{#1}}%
\global\long\def\pplus{{\displaystyle }{+\negmedspace+}}%


\chapter{Traversable functors\label{chap:9-Traversable-functors-and}}

\section{Motivation}

The \lstinline!map!/\lstinline!reduce! style\index{map/reduce programming style@\texttt{map}/\texttt{reduce} programming style}
is a major industrial use case of functional programming. Previous
chapters examined systematically the properties of functions used
in that style (\lstinline!map!, \lstinline!filter!, \lstinline!flatMap!,
and \lstinline!zip!) and generalized them to many different type
constructors. This chapter adopts the same approach to study and generalize
the \lstinline!reduce! method. In this way, we will obtain a complete
understanding of the \lstinline!map!/\lstinline!reduce! programming
style.

\subsection{From \texttt{reduce} and \texttt{foldLeft} to \texttt{foldMap} and
\texttt{traverse}\label{subsec:From-reduce-and-foldleft-to-foldmap}}

The type signatures of \lstinline!reduce! and \lstinline!foldLeft!
for a sequence of type \lstinline!Seq[A]! can be written as:
\begin{lstlisting}
def reduce[A](xs: Seq[A])(update: (A, A) => A): A
def foldLeft[A, B](xs: Seq[A])(b0: B)(update: (B, A) => B): B
\end{lstlisting}
\[
\text{reduce}:\text{Seq}^{A}\rightarrow(A\times A\rightarrow A)\rightarrow A\quad,\quad\quad\text{foldLeft}:\text{Seq}^{A}\rightarrow B\rightarrow(B\times A\rightarrow B)\rightarrow B\quad.
\]
In \lstinline!foldLeft!, the accumulated result is of a different
type ($B$) than the type of sequence elements ($A$). So, \lstinline!foldLeft!
is a more general version of \lstinline!reduce!. We can implement
\lstinline!reduce! through \lstinline!foldLeft!:
\begin{lstlisting}
def reduce[A](xs: Seq[A])(update: (A, A) => A): A = xs.tail.foldLeft(xs.head)(update)
\end{lstlisting}

The \lstinline!foldLeft! operation goes over each element of a sequence
and accumulates the updates, returning the final updated value. We
have seen many examples of using \lstinline!foldLeft! in Chapter~\ref{chap:2-Mathematical-induction}.
Now we would like to generalize \lstinline!foldLeft! to data structures
other than sequences. Such a generalization must be based on imposing
suitable laws, but it is not obvious what laws \lstinline!foldLeft!
must satisfy (other than naturality laws that will hold automatically
for any fully parametric code). To make progress, we will generalize
\lstinline!foldLeft! further to obtain a new function called \lstinline!traverse!,
for which we can motivate non-trivial laws. We will achieve this generalization
in two steps, first transforming \lstinline!foldLeft! to a function
called \lstinline!foldMap! and then to \lstinline!traverse!. 

The way from \lstinline!foldLeft! to \lstinline!foldMap! \index{foldMap function@\texttt{foldMap} function}starts
by changing the type of \lstinline!update: (B, A) => B! to an equivalent
curried form with a flipped order of arguments:
\[
\text{upd}:A\rightarrow B\rightarrow B\quad.
\]
So, $\text{upd}\,(x^{:A})$ has type $B\rightarrow B$. The key observation
is that the set of all functions of type $B\rightarrow B$ is a monoid
(denoted by \lstinline!monoidFunc! in Example~\ref{subsec:tc-Example-Monoids}
and Section~\ref{subsec:Monoids-constructions}).\index{monoid}
For brevity, we will here denote that monoid by \lstinline!MF!. The
empty element of that monoid is the identity function ($e_{\text{MF}}\triangleq\text{id}^{:B\rightarrow B}$),
and the \lstinline!compose! operation is the function composition
($f^{:B\rightarrow B}\oplus_{\text{MF}}g^{:B\rightarrow B}\triangleq f\bef g$).
Now the evaluation of a \lstinline!foldLeft! operation can be visualized
like this:
\begin{align*}
 & \text{foldLeft}\left(\left[x_{1},x_{2},x_{3}\right]\right)\left(b_{0}\right)\left(\text{upd}\right)=b_{0}\triangleright\text{upd}\left(x_{1}\right)\triangleright\text{upd}\left(x_{2}\right)\triangleright\text{upd}\left(x_{3}\right)\\
 & =b_{0}\triangleright\text{upd}\left(x_{1}\right)\bef\text{upd}\left(x_{2}\right)\bef\text{upd}\left(x_{3}\right)=b_{0}\triangleright\big(\text{upd}\left(x_{1}\right)\oplus_{\text{MF}}\text{upd}\left(x_{2}\right)\oplus_{\text{MF}}\text{upd}\left(x_{3}\right)\big)\quad.
\end{align*}
This formulation suggests that we could replace the specific monoid
(\lstinline!MF!) used in \lstinline!foldLeft! by an arbitrary monoid
($M$). The new, more general function is called \lstinline!foldMap!.
To obtain that function, we first change the order of the curried
arguments for convenience:
\[
\text{foldMap}\,(\text{upd})(\text{xs})\left(b_{0}\right)\triangleq\text{foldLeft}\,(\text{xs})\left(b_{0}\right)(\text{upd})\quad.
\]
Omitting the argument $b_{0}$, we may visualize the computation performed
by \lstinline!foldMap! as:
\[
\text{foldMap}\,(\text{upd})\left(\left[x_{1},x_{2},x_{3}\right]\right)=\text{upd}\left(x_{1}\right)\oplus_{M}\text{upd}\left(x_{2}\right)\oplus_{M}\text{upd}\left(x_{3}\right)\quad.
\]
The type of \lstinline!upd! is now $A\rightarrow M$ instead of $A\rightarrow B\rightarrow B$.
So, the type signature of \lstinline!foldMap! must be:
\[
\text{foldMap}:(A\rightarrow M)\rightarrow\text{Seq}^{A}\rightarrow M\quad.
\]
This type signature assumes that $M$ is a monoid type. We can now
implement \lstinline!foldMap! via \lstinline!foldLeft!:
\begin{lstlisting}
def foldMap[M: Monoid, A](f: A => M): Seq[A] => M =
  _.foldLeft(Monoid[M].empty) { (m, a) => m |+| f(a) }
\end{lstlisting}
We can also implement \lstinline!foldLeft! via \lstinline!foldMap!
by using the monoid instance of \lstinline!MF! (see Example~\ref{subsec:tc-Example-Monoids}):
\begin{lstlisting}
implicit def monoidFunc[A]: Monoid[A => A] = ...
def foldLeft[A, B](xs: Seq[A])(b0: B)(update: (B, A) => B): B =
  foldMap[B => B, A](a => b => update(b, a))(xs)(b0)
\end{lstlisting}
We will prove later that \lstinline!reduce!, \lstinline!foldLeft!,
and \lstinline!foldMap! are in fact equivalent.

The type signature of \lstinline!foldMap! suggests that we could
replace \lstinline!Seq[_]! by an arbitrary type constructor \lstinline!L[_]!.
We call \textbf{foldable}\index{foldable functor} a type constructor
\lstinline!L[_]! for which the \lstinline!foldMap! method is defined:
\begin{lstlisting}
def foldMap_L[M: Monoid, A](f: A => M): L[A] => M
\end{lstlisting}
\[
\text{foldMap}_{L}:\left(A\rightarrow M\right)\rightarrow L^{A}\rightarrow M\quad,\quad\text{assuming that }M\text{ is a monoid}\quad.
\]
The function $\text{foldMap}_{L}$ works in the same way for all monoids
$M$ (but differently for each $L$). 

The final step towards the \lstinline!traverse! method is to replace
an arbitrary monoid $M$ by an arbitrary \emph{applicative} functor
$F^{B}$ with a new type parameter $B$. This step is not straightforward:
values of applicative functor types are not monoids (we do not necessarily
have an operation of type $F^{B}\times F^{B}\rightarrow F^{B}$).
So, we will give an independent motivation for the \lstinline!traverse!
method in the next subsection.

\subsection{The \texttt{traverse} operation\label{subsec:The-traverse-operation}}

Consider the task of waiting for several concurrent operations to
finish. As an example, suppose we have a list of type \lstinline!List[A]!
and a processing function of type \lstinline!A => Future[B]!. We
need to apply the processing function in parallel to all items from
the list and wait until the entire data set is processed.

The Scala standard library has a special method (\lstinline!Future.traverse!)
to use in this situation:
\begin{lstlisting}
val data: List[A] = ???
val processing: A => Future[B] = ???
implicit val ec: ExecutionContext = ???
val results: Future[List[B]] = Future.traverse(data)(processing)
\end{lstlisting}
Since the processing of each element is asynchronous, the result is
wrapped in a \lstinline!Future! type constructor. The \lstinline!traverse!
method will mark that \lstinline!Future! value as finished when all
processing is done.

If we ignore the implicit argument of type \lstinline!ExecutionContext!
(which is specific to the \lstinline!Future! class) and also flip
the first two curried arguments, we will arrive at the following type
signature of \lstinline!traverse!:\vspace{-0.5\baselineskip}
\[
\text{traverse}:(A\rightarrow\text{Future}^{B})\rightarrow\text{List}^{A}\rightarrow\text{Future}^{\text{List}^{B}}\quad.
\]
Generalizing to other type constructors $F$, $L$ instead of \lstinline!Future!
and \lstinline!List!, we obtain:
\begin{lstlisting}
def traverse[A, B, F[_]: Applicative: Functor](f: A => F[B]): L[A] => F[L[B]]
\end{lstlisting}
\[
\text{trav}_{L}^{F,A}:(A\rightarrow F^{B})\rightarrow L^{A}\rightarrow F^{L^{B}}\quad.
\]
In this type signature, we have chosen the parameter names $F$ and
$L$ specifically to help us avoid errors (e.g., writing $L^{F^{B}}$
instead of $F^{L^{B}}$) by recalling the example with $F=$ \lstinline!Future!
and $L=$ \lstinline!List!.

It turns out that the \lstinline!traverse! method can work in the
same way for any \emph{applicative} functor $F$. A functor $L$ having
a \lstinline!traverse! method of this type is called \index{traversable functor}\textbf{traversable}.

The type signatures of \lstinline!traverse! and \lstinline!foldMap!
are similar except that $M$ is replaced by $F^{B}$ or $F^{L^{B}}$
as needed. To see that \lstinline!foldMap! is a special case of \lstinline!traverse!,
take a constant functor $F^{A}\triangleq M$, which is applicative
if $M$ is a monoid (as shown in Section~\ref{subsec:Constructions-of-applicative-functors}).
With this choice of $F$, the type signature of \lstinline!traverse!
reduces to that of \lstinline!foldMap!.

Before studying the properties of \lstinline!traverse!, we will look
at some examples of its practical use.

\section{Practical use of folding and traversing operations}

\subsection{Implementing \texttt{traverse} for various data types}

To understand how \lstinline!traverse! works, let us implement it
for some data types, including lists and trees.

\subsubsection{Example \label{subsec:Example-traverse-for-1+a*a}\ref{subsec:Example-traverse-for-1+a*a}\index{solved examples}}

Implement \lstinline!traverse! for the type constructor $L$ defined
by:

\begin{wrapfigure}{l}{0.5\columnwidth}%
\vspace{-1\baselineskip}
\begin{lstlisting}
type L[A] = Option[(A, A)]
\end{lstlisting}
\vspace{-0.5\baselineskip}
\end{wrapfigure}%

~\vspace{-0.5\baselineskip}
\[
L^{A}\triangleq\bbnum 1+A\times A\quad.
\]


\subparagraph{Solution}

For the given type constructor $L$, write the type signature of \lstinline!trav!
as:
\begin{lstlisting}
def trav[A, B, F[_]: Applicative : Functor](f: A => F[B]): Option[(A, A)] => F[Option[(B, B)]]
\end{lstlisting}
In the short type notation, this type signature is:
\[
\text{trav}_{L}:(A\rightarrow F^{B})\rightarrow(\bbnum 1+A\times A)\rightarrow F^{\bbnum 1+B\times B}\quad.
\]
 Let us implement \lstinline!trav!, aiming to preserve information.
We need to compute a value of type $F^{\bbnum 1+B\times B}$. Begin
by pattern matching on the argument of type $\bbnum 1+A\times A$:
\begin{lstlisting}
def trav[A, B, F[_]: Applicative : Functor](f: A => F[B]): Option[(A, A)] => F[Option[(B, B)]] = {
  case None             => Applicative[F].pure(None)  // No other choice here.
  case Some((a1, a2))   => ???                        // Need a value of type F[Some[(B, B)]] here.
}
\end{lstlisting}
The only way of getting values of type $B$ is by applying \lstinline!f!
to the values \lstinline!a1! and \lstinline!a2!. We will then get
two values of type $F^{B}$. We may merge them into a single value
of type $F^{B\times B}$ via $F$\textsf{'}s \lstinline!zip! method. It remains
to use $F$\textsf{'}s \lstinline!map! in order to transform $F^{B\times B}$
to the required type $F^{\bbnum 1+B\times B}$:
\begin{lstlisting}
def trav[A, B, F[_]: Applicative : Functor](f: A => F[B]): Option[(A, A)] => F[Option[(B, B)]] = {
  case None             => Applicative[F].pure(None)          // No other choice here.
  case Some((a1, a2))   => Applicative[F].zip(f(a1), f(a2)).map { case (b1, b2)  => Some((b1, b2)) }
}
\end{lstlisting}
In the code notation, this is written as:
\[
\text{trav}_{L}(f^{:A\rightarrow F^{B}})\triangleq\,\begin{array}{|c||c|}
 & F^{\bbnum 1+B\times B}\\
\hline \bbnum 1 & \_\rightarrow\text{pu}_{F}(1+\bbnum 0^{:B\times B})\\
A\times A & (f\boxtimes f)\bef\text{zip}_{F}\bef(b_{1}^{:B}\times b_{2}^{:B}\rightarrow\bbnum 0^{:\bbnum 1}+b_{1}\times b_{2})^{\uparrow F}
\end{array}\quad.
\]
The code must use $F$\textsf{'}s methods \lstinline!map!, \lstinline!pure!,
and \lstinline!zip!. This makes it clear why \lstinline!traverse!
requires $F$ to be an applicative functor. Apart from that requirement,
the code works in the same way for all $F$.

\subsubsection{Example \label{subsec:Example-traversable-seq}\ref{subsec:Example-traversable-seq}}

Implement \lstinline!traverse! for Scala\textsf{'}s \lstinline!List! type
constructor. 

\subparagraph{Solution}

We need to implement the following type signature:
\begin{lstlisting}
def trav[A, B, F[_]: Applicative : Functor](f: A => F[B]): List[A] => F[List[B]]
\end{lstlisting}
A list may be empty, or may have a head and a tail. Similarly to the
code in Example~\ref{subsec:Example-traverse-for-1+a*a}, we use
$F$\textsf{'}s methods \lstinline!pure!, \lstinline!zip!, and \lstinline!map!
to compute a suitable value of type \lstinline!F[Seq[B]]! in these
cases.
\begin{lstlisting}
def trav[A, B, F[_]: Applicative : Functor](f: A => F[B])(la: List[A]): F[List[B]] = la match {
  case Nil            => Applicative[F].pure(Nil)
  case head :: tail   => (f(head) zip trav(f)(tail)).map { case (headB, tailB)  => headB :: tailB }
}
\end{lstlisting}
This code applies \lstinline!f! to each element of the list and accumulates
the resulting $F$-effects of type $F^{B}$. The $F$-effects are
merged using $F$\textsf{'}s \lstinline!zip! function (implemented as an extension
method).

In the code notation, the function \lstinline!trav! looks like this:
\[
\text{trav}_{\text{List}}(f^{:A\rightarrow F^{B}})\triangleq\,\begin{array}{|c||c|}
 & F^{\bbnum 1+B\times\text{List}^{B}}\\
\hline \bbnum 1 & \_\rightarrow\text{pu}_{F}(1+\bbnum 0^{:B\times\text{List}^{B}})\\
A\times\text{List}^{A} & \big(f\boxtimes\overline{\text{trav}}_{\text{List}}(f)\big)\bef\text{zip}_{F}\bef(h^{:B}\times t^{:\text{List}^{B}}\rightarrow\bbnum 0^{:\bbnum 1}+h\times t)^{\uparrow F}
\end{array}\quad.
\]


\subsubsection{Example \label{subsec:Example-traverse-tree}\ref{subsec:Example-traverse-tree}}

Implement a \lstinline!traverse! operation for a binary tree.

\subparagraph{Solution}

We will use the following binary tree data type denoted by \lstinline!T2!:
\begin{lstlisting}
sealed trait T2[A]
final case class Leaf[A](a: A)                 extends T2[A]
final case class Branch[A](l: T2[A], r: T2[A]) extends T2[A]
\end{lstlisting}
We implement \lstinline!trav! similarly to the implementation of
\lstinline!foldLeft! in Section~\ref{subsec:Binary-trees}:
\begin{lstlisting}
def trav[A, B, F[_]: Applicative : Functor](f: A => F[B])(t: T2[A]): F[T2[B]] = t match {
  case Leaf(a)          => f(a).map(b => Leaf(b))            // Reproduce the Leaf structure under F.
  case Branch(t1, t2)   =>
    val (r1, r2) = (trav(f)(t1), trav(f)(t2))    // Traverse the two branches and obtain two results.
    (r1 zip r2).map { case (b1, b2)  => Branch(b1, b2) }   // Reproduce the Branch structure under F.
}
\end{lstlisting}
In the code notation, this function looks like this:
\[
\text{trav}\,(f^{:A\rightarrow F^{B}})\triangleq\,\begin{array}{|c||c|}
 & F^{B+\text{T2}^{B}\times\text{T2}^{B}}\\
\hline A & f\bef(b^{:B}\rightarrow b+\bbnum 0^{:\text{T2}^{B}\times\text{T2}^{B}})^{\uparrow F}\\
\text{T2}^{A}\times\text{T2}^{A} & \big(\overline{\text{trav}}\,(f)\boxtimes\overline{\text{trav}}\,(f)\big)\bef\text{zip}_{F}\bef(l^{:\text{T2}^{B}}\times r^{:\text{T2}^{B}}\rightarrow\bbnum 0^{:B}+l\times r)^{\uparrow F}
\end{array}\quad.
\]

The code first traverses the two sub-branches (using recursive calls
to \lstinline!trav!) and then combines the resulting values. So,
this implementation represents a depth-first, left-to-right traversal
of the tree.

\subsubsection{Example \label{subsec:Example-traversal-perfect-shaped-tree}\ref{subsec:Example-traversal-perfect-shaped-tree}}

Implement a \lstinline!traverse! operation for a perfect-shaped tree
(\lstinline!PTree!, Section~\ref{subsec:Perfect-shaped-trees}).

\subparagraph{Solution}

We begin writing code as in Example~\ref{subsec:Example-traverse-tree}
by pattern matching:
\begin{lstlisting}[numbers=left]
def trav[A, B, F[_]: Applicative : Functor](f: A => F[B])(t: PTree[A]): F[PTree[B]] = t match {
  case Leaf(a)     => f(a).map(b => Leaf(b))           // Reproduce the Leaf structure under F.
  case Branch(t)   => ???  // Here t has type PTree[(A, A)].
}
\end{lstlisting}
In line 3, we have a pattern variable \lstinline!t! of type \lstinline!PTree[(A, A)]!,
but we need to obtain a value of type \lstinline!F[Branch[B]]! in
that scope. Since a \lstinline!Branch! contains a recursive instance
of the type \lstinline!PTree!, it seems we need to use a recursive
call of \lstinline!trav! here. However, we cannot apply \lstinline!trav(f)!
to \lstinline!t! because the types do not match. We will be able
to apply \lstinline!trav! if instead of \lstinline!f: A => F[B]!
we had a function of type \lstinline!(A, A) => F[(B, B)]!. We can
create such a function out of \lstinline!f! using $F$\textsf{'}s \lstinline!zip!
method:
\begin{lstlisting}
{ case (a1, a2) => f(a1) zip f(a2) }
\end{lstlisting}
 So, we write:
\begin{lstlisting}
  case Branch(t)   => trav { case (a1, a2) => f(a1) zip f(a2) }(t); ??? } // Have F[PTree[(B, B)]].
\end{lstlisting}
We can now use $F$\textsf{'}s \lstinline!map! method to restore the \lstinline!Branch!
structure under $F$. The complete code is:
\begin{lstlisting}
def trav[A, B, F[_]: Applicative : Functor](f: A => F[B])(t: PTree[A]): F[PTree[B]] = t match {
  case Leaf(a)     => f(a).map(b => Leaf(b))           // Reproduce the Leaf structure under F.
  case Branch(t)   => (trav[(A, A), (B, B), F] { case (a1, a2) => f(a1) zip f(a2) }(t)
                      ).map(x => Branch(x))          // Reproduce the Branch structure under F.
}
\end{lstlisting}
Here we assumed that the \lstinline!zip! function is defined on \lstinline!F!
as an extension method. $\square$

These examples show how to implement the \lstinline!traverse! operation
for a container-like data structure $L^{A}$ that stores some values
of type $A$. Each stored value of type $A$ is processed using the
given function $f:A\rightarrow F^{B}$. All of the resulting $F$-effects
need to be collected and merged (using $F$\textsf{'}s \lstinline!zip! and
\lstinline!map!) into a single $F$-effect that wraps a value of
type $L^{B}$. The wrapped value should reproduce the shape of the
data structure that was present in the input value of type $L^{A}$. 

In all examples above, we merged all $F$-effects together by collecting
each $F$-effect exactly once. It would be strange if the \lstinline!traverse!
operation repeated some $F$-effects, as in the following code:
\begin{lstlisting}
def badtrav[A, B, F[_]: Applicative : Functor](f: A => F[B])(la: List[A]): F[List[B]] = la match {
  case Nil            => Applicative[F].pure(Nil)
  case head :: tail   => (f(head) zip f(head) zip badtrav(f)(tail))     // Use the F-effect twice.
                           .map { case (headB, (_, tailB))  => headB :: tailB }
}
\end{lstlisting}
Although this implementation fits the type signature of \lstinline!traverse!,
it contradicts the intuition of traversing the sequence only once.
When $F=$ \lstinline!Future!, the code \lstinline!badtrav(f)(xs)!
will start \emph{two} parallel computations for each element of the
sequence \lstinline!xs! and ignore one of the results. When $F$\textsf{'}s
effect describes parsing (see Section~\ref{subsec:Parsing-with-applicative-and-monadic-parsers})
or other computations that maintain an internal state, invoking such
an $F$-effect twice will likely lead to incorrect results. Below
we will see that the laws of \lstinline!traverse! prohibit implementations
that either repeat some of the $F$-effects or ignore some of the
values.

\subsection{Aggregating tree-like data by folding. Breadth-first traversal\label{subsec:Aggregating-tree-like-data-bfs}}

Although \lstinline!foldMap! (and \lstinline!foldLeft!) are special
cases of \lstinline!traverse!, it is often easier to implement \lstinline!foldMap!
directly. Let us look at some examples of \lstinline!foldMap! that
implements different kinds of tree traversals.

One implementation of \lstinline!foldMap! corresponds to the \lstinline!traverse!
operation defined in Example~\ref{subsec:Example-traverse-tree}.
The code of \lstinline!foldMap! is:
\begin{lstlisting}
def foldMap[A, M: Monoid](f: A => M)(t: T2[A]): M = t match {
  case Leaf(a)          => f(a)
  case Branch(t1, t2)   => foldMap(f)(t1) |+| foldMap(f)(t2)
}
\end{lstlisting}

A special case of a monoid is \lstinline!List[A]!; its \lstinline!combine!
operation is the concatenation of lists. If we use \lstinline!foldMap!
with the monoid \lstinline!M = List[A]! and the function \lstinline!f: A => List[A]!
that creates a single-element list, we obtain a function \lstinline!toList!
that converts a tree \lstinline!T2[A]! into \lstinline!List[A]!:
\begin{lstlisting}
def toList[A]: T2[A] => List[A] = foldMap[A, List[A]](List(_)) // Need a Monoid instance for List[A].
\end{lstlisting}

In the same way, we may implement \lstinline!toList: L[A] => List[A]!
for any foldable functor \lstinline!L[_]!.

The function \lstinline!toList! captures the requirement that a foldable
functor \lstinline!L! must have a well-defined way of iterating over
the data items stored inside it. So, \lstinline!L! is foldable if
there is a known way of extracting its data items in the form of a
list. We may then use standard aggregation operations on lists (\lstinline!foldLeft!,
\lstinline!reduce!, and so on). Below, we will prove that \lstinline!toList!
is equivalent to \lstinline!foldMap!. So, a foldable functor allows
us to apply any aggregation operation to the data stored in the functor. 

An aggregation\textsf{'}s results may depend on the order of data traversal.
It is important that the function \lstinline!toList! can be implemented
in different ways by choosing a specific order in which the data from
\lstinline!L! is copied to a list. Some data structures have a \textsf{``}natural\textsf{''}
traversal order that is easiest to implement. Let us see some examples
of implementing different traversal orders for lists and trees.

The above code of \lstinline!foldMap! for the binary tree implements
a depth-first traversal of the tree. We can see that by applying \lstinline!toList!
to an example tree (\lstinline!t2!):

\begin{wrapfigure}{l}{0.76\columnwidth}%
\vspace{-0.65\baselineskip}
\begin{lstlisting}
val t2: T2[Int] = Branch(Leaf(8), Branch(Branch(Leaf(3), Leaf(5)), Leaf(4)))

scala> toList(t2)
res0: List[Int] = List(8, 3, 5, 4)
\end{lstlisting}

\vspace{0.5\baselineskip}
\end{wrapfigure}%

\noindent {\footnotesize{}\vspace{-0.3\baselineskip}
}{\footnotesize\par}

\noindent ~~\lstinline!toList(!{\tiny{} \Tree[ 8 [ [ 3 5 ] 4 ] ] }\lstinline!)!{\small{}}\\
{\small{}~~~ $=\left[8,3,5,4\right]$}{\small\par}

Let us now implement another version of \lstinline!toList!, called
\lstinline!toList2!, that performs the \emph{breadth-first} traversal
of a binary tree. The idea is to prepare a list of all leaf values
at level $0$ in the tree, then a list of all values at level $1$,
and so on. At each level, values must be collected left to right.
The result will be a list of lists. For example, the tree \lstinline!t2!
shown above has no values at level $0$, value $8$ at level $1$,
value $4$ at level $2$, and values $3$ and $5$ at level $3$.
So, applying \lstinline!toList2! to the tree \lstinline!t2! will
give the nested list $\left[\left[\right],\left[8\right],\left[4\right],\left[3,5\right]\right]$.
Then we can \lstinline!flatten! that list and obtain the list $\left[8,4,3,5\right]$,
which is the breadth-first traversal order of the tree \lstinline!t2!. 

Begin writing code of \lstinline!toList2! by pattern matching on
the tree structure:
\begin{lstlisting}
def toListBFS[A]: T2[A] => List[A] = toList2 andThen (_.flatten)

def toList2[A]: T2[A] => List[List[A]] = {
  case Leaf(a)        => List(List(a))    // Only one value at level 0.
  case Branch(l, r)   => ???         // Have toList2(l) and toList2(r).
}
\end{lstlisting}
After using recursive calls of \lstinline!toList2! on the left and
right sub-trees (\lstinline!l! and \lstinline!r!), we need somehow
to combine the resulting lists. For the tree \lstinline!t2!, applying
\lstinline!toList2! to the sub-trees will return the lists $\left[\left[8\right]\right]$
and $\left[\left[\right],\left[4\right],\left[3,5\right]\right]$.
We need to concatenate the corresponding nested lists and obtain $\left[\left[\right],\left[8\right],\left[4\right],\left[3,5\right]\right]$.
(An initial empty list needs to be added because the sub-trees begin
one level deeper.) The standard \lstinline!zip! operation would not
work correctly here: \lstinline!zip! would truncate the longer list.
Instead, we need to keep the longer list\textsf{'}s elements. So, let us implement
a special helper function (\lstinline!listMerge!) for merging the
nested lists in this way:
\begin{lstlisting}
def listMerge[A](l: List[List[A]], r: List[List[A]]): List[List[A]] = (l, r) match {
  case (Nil, r)               => r        // Keep the elements from the longer list.
  case (l, Nil)               => l
  case (lh :: lt, rh :: rt)   => (lh ++ rh) :: listMerge(lt, rt)
}
\end{lstlisting}
Now we can complete the code of \lstinline!toList2! and run some
tests:
\begin{lstlisting}
def toList2[A]: T2[A] => List[List[A]] = {
  case Leaf(a)        => List(List(a)) // Only one value at level 0.
  case Branch(l, r)   => listMerge(Nil :: toList2(l), Nil :: toList2(r))
}

scala> toList2(t2)
res1: List[List[Int]] = List(List(), List(8), List(4), List(3, 5))

scala> toListBFS(t2)
res2: List[Int] = List(8, 4, 3, 5)
\end{lstlisting}

For lists and other sequences, the easiest traversal orders are either
direct or reverse. A direct traversal is implemented by the standard
\lstinline!foldLeft! operation on \lstinline!Seq!, while a reverse-order
traversal is done simply by applying the \lstinline!reverse! method,
such as \lstinline!la.reverse.foldLeft(...)(...)!.

Similarly to \lstinline!foldLeft! and \lstinline!foldMap!, the \lstinline!traverse!
method may be implemented with different traversal orders. For comparison,
we show the code for the direct-ordered and reverse-ordered \lstinline!List!
traversals:
\begin{lstlisting}
// This code assumes that F has typeclass instances for Applicative and Functor.
def travList[A, B](f: A => F[B])(la: List[A]): F[List[B]] = la match {
  case Nil            => Applicative[F].pure(Nil)
  case head :: tail   => f(head).map2(travList(f)(tail)) { (x, y) => x +: y }
}

def travRevList[A, B](f: A => F[B])(la: List[A]): F[List[B]] = la.reverse match {
  case Nil            => Applicative[F].pure(Nil)
  case head :: tail   => f(head).map2(travRevList(f)(tail)) { (h, t) => t :+ h }
}
\end{lstlisting}
The function \lstinline!travRevList! first reverses the given list,
so that $F$\textsf{'}s effects are merged in the reverse order. Then it swaps
\lstinline!h! and \lstinline!t! using \lstinline!map2!, which restores
the original list structure wrapped under $F$.

Implementing a breadth-first \lstinline!traverse! for a tree is more
difficult. It is not sufficient to convert the tree to a list in the
breadth-first order and collect $F$\textsf{'}s effects. We also need to restore
the original tree structure wrapped under the functor $F$. Section~\ref{subsec:Decorating-a-tree-breadth-first-traversal}
will show an example of implementing a breadth-first \lstinline!traverse!
for binary trees.

\subsection{Decorating a tree. I. Depth-first traversal}

A \lstinline!traverse! method for a given tree-like data structure
may be used to perform various operations on trees, as long as those
operations are described by the effect of some applicative functor
$F$. 

An example of a tree operation is \textsf{``}decorating\textsf{''}, i.e., adding labels
of some sort to each leaf. Since tree-like data types are functors
that have a \lstinline!map! method, a simple decorating operation
replaces each leaf with a function of the data at that leaf. For instance,
we may add the value $20$ to each leaf:
\begin{lstlisting}
val t2 = Branch(Leaf(8), Branch(Branch(Leaf(3), Leaf(5)), Leaf(4)))

scala> t2.map(x => x + 20)  // Assuming a Functor instance for T2[_].
res0: T2[Int] = Branch(Leaf(28), Branch(Branch(Leaf(23), Leaf(25)), Leaf(24)))
\end{lstlisting}
This transforms the tree {\tiny{}\Tree[ 8 [ [ 3 5 ] 4 ] ] } into
{\tiny{}\Tree[ 28 [ [ 23 25 ] 24 ] ]}. However, the \lstinline!map!
method works separately with each leaf and cannot use any previously
computed values. E.g., we cannot use \lstinline!map! to implement
a \lstinline!zipWithIndex! function for trees, transforming the tree
{\tiny{}\Tree[ 8 [ [ 3 5 ] 4 ] ] } into {\tiny{}\Tree[ (8,0) [ [ (3,1) (5,2) ] (4,3) ] ]}
with added indices ($0$, $1$, $2$, $3$). For such tasks, we need
the additional functionality of the \lstinline!traverse! operation.

Labeling the leaves of a tree with their traversal index requires
us to update an internal state (the number of leaves seen so far)
while traversing the tree. Recall that updating an internal state
is the effect of a \lstinline!State! monad. Since all monads can
also implement the applicative methods (\lstinline!pure! and \lstinline!zip!),
we may use the \lstinline!State! monad as the applicative functor
$F$ for the \lstinline!traverse! function.

In this section, we will implement \lstinline!zipWithIndex! for a
\emph{depth-first} tree traversal. Let us call that function \lstinline!zipWithIndexDF!.
The code of \lstinline!zipWithIndexDF! will have the form:
\begin{lstlisting}
final case class St[A](run: Int => (A, Int))    // A State monad with internal state of type Int.
                         // Assume that we have defined Applicative and Functor instances for St.
def computeIndex[A]: A => St[(A, Int)] = ???                 // Define the "decoration" function.
def zipWithIndexDF[A](tree: T2[A]): T2[(A, Int)] = {
  val afterTraverse: St[T2[(A, Int)]] = trav[A, (A, Int), St](computeIndex)(tree)
  afterTraverse.run(0)._1                        // Run the State monad and get the result value.
}
\end{lstlisting}
This will be a depth-first traversal if \lstinline!trav! is the function
shown in Example~\ref{subsec:Example-traverse-tree}. It remains
to define the function \lstinline!computeIndex! describing the \textsf{``}decoration
logic\textsf{''}. Begin writing the code as:
\begin{lstlisting}
def computeIndex[A]: A => St[(A, Int)] = a => St { i => ???: ((A, Int), Int) }
\end{lstlisting}
Given a leaf value (\lstinline!a: A!) and a previous internal state
(\lstinline!i: Int!), we must compute the \textsf{``}decorated\textsf{''} leaf value
of type $A\times\text{Int}$ as well as the new internal state. Since
our goal is to label the leaves by the current leaf count, we use
the internal state to store the number of leaves seen so far. So,
we just need to increment the internal state after copying it to a
leaf:
\begin{lstlisting}
def computeIndex[A]: A => St[(A, Int)] = a => St { i => ((a, i), i + 1) }
\end{lstlisting}
This completes the implementation of \lstinline!zipWithIndexDFS!,
which we can now test:
\begin{lstlisting}
scala> zipWithIndexDF(t2)
res0: T2[(Int, Int)] = Branch(Leaf((8,0)), Branch(Branch(Leaf((3,1)), Leaf((5,2))), Leaf((4,3))))
\end{lstlisting}


\subsection{Decorating a tree. II. Breadth-first traversal\label{subsec:Decorating-a-tree-breadth-first-traversal}}

In Section~\ref{subsec:Aggregating-tree-like-data-bfs}, we have
implemented a breadth-first folding for the tree data structure \lstinline!T2[A]!.
What additional code does  \lstinline!traverse! need? Folding over
\lstinline!T2[A]! merely needs to arrange the leaf values of type
\lstinline!A! in the breadth-first order, but a \lstinline!traverse!
function must return a value of type \lstinline!F[T2[B]]!. This requires
us to sequence the effects of an arbitrary applicative functor \lstinline!F[_]!
in the breadth-first order, while gathering the values of type \lstinline!B!
into a tree structure (\lstinline!T2[B]!) wrapped under \lstinline!F!.
The function \lstinline!toListBFS! shown in the previous section
is not sufficient for that purpose, because the tree structure cannot
be reproduced if we only have a list of leaf values. Even the nested
list computed by \lstinline!toList2! is not sufficient. We need additional
information about each leaf\textsf{'}s position in the tree.

So, let us begin by adding the required position information to the
nested list of leaf values. For each leaf, we need to store the path
from the root of the tree. We can use the \lstinline!Either! type
constructor and its subtypes \lstinline!Left! and \lstinline!Right!
for marking that path. For example, the position information for the
tree \lstinline!t2 =!{\tiny{}\Tree[ 8 [ [ 3 5 ] 4 ] ]} could be
represented by the following nested list:
\begin{lstlisting}
List(
  List(),                                                // No leaves at level 0.
  List( Left(8) ),                                       // One leaf at level 1.
  List( Right(Right(4)) ),                               // One leaf at level 2.
  List( Right(Left(Left(3))), Right(Left(Right(5))) ),   // Two leaves at level 3.
)
\end{lstlisting}
This data structure makes it easy to iterate over the leaf values
in the breadth-first order. At the same time, it keeps enough information
for restoring the original tree since each leaf value comes with the
full path from the root of the tree.

An immediate problem with this data structure is a type mismatch in
the outer \lstinline!List!: its elements are not of the same type
because each subsequent sub-list contains data wrapped under \lstinline!Either!
more deeply. To make the types match, we need to implement a special
list-like data structure whose first element has type \lstinline!List[A]!,
the second element has type \lstinline!List[Either[A, A]]!, the third
element has type \lstinline!List[Either[Either[A, A], Either[A, A]]]!,
and so on. (Exercise~\ref{subsec:Exercise-disjunctive-EvenList-1}
shows a similar data type.) Let us call this data type \lstinline!TD!
(\textsf{``}tree descriptor\textsf{''}):
\begin{lstlisting}
sealed trait TD[A]
final case class Last[A](a: List[A]) extends TD[A]
final case class More[A](a: List[A], tail: TD[Either[A, A]]) extends TD[A]  
\end{lstlisting}
The tree descriptor for the tree \lstinline!t2! shown above will
then look like this:
\begin{lstlisting}
val td2: TD[Int] = More(List(), More(                          // No leaves at level 0.
    List( Left(8) ), More(                                     // The leaf at level 1.
      List( Right(Right(4)) ), Last(                           // The leaf at level 2.
        List( Right(Left(Left(3))), Right(Left(Right(5))) ),   // Two leaves at level 3.
))))
\end{lstlisting}
Now the types match, while a wrong number of \lstinline!Left! or
\lstinline!Right! wrappers will be a type error.

This representation of the tree descriptor is certainly not efficient
in terms of memory consumption and processing speed. An \textsf{``}industry-strength\textsf{''}
breadth-first tree traversal will use quite different data structures
while implementing the same logic. For instance, one could use a bit
vector $\left[1,0,1\right]$ instead of the wrapper \lstinline!Right(Left(Right()))!,
avoiding the allocation of many objects in memory. However, our present
goal is not to achieve high efficiency but to obtain correct code
quickly.

The next step is to write functions transforming between trees \lstinline!T2[A]!
and descriptors \lstinline!TD[A]!:
\begin{lstlisting}
def t2ToTD[A]: T2[A] => TD[A] = ???
def tdToT2[A]: TD[A] => T2[A] = ???
\end{lstlisting}
Begin implementing the first function by pattern matching:
\begin{lstlisting}[numbers=left]
def t2ToTD[A]: T2[A] => TD[A] = {
  case Leaf(a)        => Last(List(a))                      // A leaf at level 0.
  case Branch(l, r)   => ( t2ToTD(l),  t2ToTD(r) ); ???     // How to combine the subtrees?
}
\end{lstlisting}
It seems reasonable that we need to apply \lstinline!t2ToTD! recursively
to the subtrees \lstinline!l! and \lstinline!r! in line $3$, obtaining
the tree descriptors of type \lstinline!TD[A]! for those subtrees.
We still have to merge these tree descriptors in the correct way.
Inspired by the \lstinline!listMerge! function from Section~\ref{subsec:Aggregating-tree-like-data-bfs},
we implement the descriptor-merging code by concatenating the lists
separately at each depth level:
\begin{lstlisting}
def tdMerge[A](l: TD[A], r: TD[A]): TD[A] = (l, r) match {
  case (Last(la), Last(lb))                 => Last(la ++ lb)
  case (Last(la), More(lb, tail))           => More(la ++ lb, tail)
  case (More(la, tail), Last(lb))           => More(la ++ lb, tail)
  case (More(la, tailA), More(lb, tailB))   => More(la ++ lb, tdMerge(tailA, tailB))
}
\end{lstlisting}
However, writing just \lstinline!tdMerge(t2ToTD(l), t2ToTD(r))! would
not be correct in line $3$ of \lstinline!t2ToTD!. We need somehow
to store the information that the subtrees \lstinline!l! and \lstinline!r!
are located one level deeper at the left and at the right respectively.
To see how that information needs to be stored, let us look at the
tree \lstinline!t2! whose left and right subtrees are \lstinline!l = !{\tiny{}\Tree[ 8  ]}
and \lstinline!r =!{\tiny{}\Tree[  [ 3 5 ] 4  ]}. The descriptors
of these subtrees are computed by recursive calls of \lstinline!t2ToTD!,
which (if implemented correctly) should yield the following:
\begin{lstlisting}
t2ToTD(l) == Last(List(8))                   // One leaf at level 0.

t2ToTD(r) == More(List(), More(              // No leaves at level 0.
  List( Right(4) ), Last(                    // One leaf at level 1.
    List( Left(Left(3)), Left(Right(5)) )    // Two leaves at level 2.
)))
\end{lstlisting}
The correct merging of these tree descriptors requires wrapping all
data in the subtree \lstinline!l! in an additional \lstinline!Left()!
layer and all data in the subtree \lstinline!r! in an additional
\lstinline!Right()! layer. Let us implement two helper functions
(\lstinline!addLeft! and \lstinline!addRight!) for that transformation:
\begin{lstlisting}
def addLeft[A]: TD[A] => TD[Either[A, A]] = {
  case Last(a)         => Last(a.map(Left(_)))
  case More(a, tail)   => More(a.map(Left(_)), addLeft[Either[A, A]](tail))
}

def addRight[A]: TD[A] => TD[Either[A, A]] = {
  case Last(a)         => Last(a.map(Right(_)))
  case More(a, tail)   => More(a.map(Right(_)), addRight[Either[A, A]](tail))
}
\end{lstlisting}
 We expect that \lstinline!addLeft! and \lstinline!addRight! will
transform the left and right subtree descriptors like this:
\begin{lstlisting}
addLeft(t2ToTD(l)) == Last(List(Left(8)))

addRight(t2ToTD(r)) == More( List(), More(
  List( Right(Right(4)) ), Last(
    List( Right(Left(Left(3))), Right(Left(Right(5))) )
)))
\end{lstlisting}
It remains to mark the subtrees as being one layer deeper. We do this
by inserting an empty list at the beginning of a \lstinline!TD! structure.
The types will then require that the subsequent lists must have one
more layer of \lstinline!Either! wrappers, which they will indeed
have after applying \lstinline!addLeft! and \lstinline!addRight!.
So, the correct merging of the two subtree descriptors is:
\begin{lstlisting}
More(List(), tdMerge( addLeft(t2ToTD(l)), addRight(t2ToTD(r)) ))
\end{lstlisting}
The complete code of \lstinline!t2ToTD! becomes:
\begin{lstlisting}
def t2ToTD[A]: T2[A] => TD[A] = {
  case Leaf(a)        => Last(List(a))
  case Branch(l, r)   => More(List(), tdMerge( addLeft(t2ToTD(l)), addRight(t2ToTD(r)) ))
}
\end{lstlisting}
We can test this code on an example tree:
\begin{lstlisting}
val t2 = Branch(Leaf(8), Branch(Branch(Leaf(3), Leaf(5)), Leaf(4)))

scala> t2ToTD(t2)
res0: TD[Int] = More(List(), More(List(Left(8)), More(List(Right(Right(4))), Last(List(Right(Left(Left(3))), Right(Left(Right(5))))))))
\end{lstlisting}

To implement the inverse function (\lstinline!tdToT2!), we need to
keep in mind that \lstinline!t2ToTD! will only create a certain subset
of possible values of type \lstinline!TD[A]!. Looking at the code
of \lstinline!t2ToTD!, we can see, for instance, that the first list
is always empty when the tree has a branch. The first list is either
empty or contains only one element (the single leaf of the tree).
We will never create values of type \lstinline!TD[A]! where the first
list has more than one element, or where the first list is nonempty
and there are also nonempty subsequent lists. Such values of type
\lstinline!TD[A]! do not correspond to any trees of type \lstinline!T2[A]!.
So, the code of \lstinline!tdToT2! can be a \emph{partial} function
of type \lstinline!TD[A] => T2[A]! that only works with valid tree
descriptors, that is, values of type \lstinline!TD[A]! that were
obtained by applying \lstinline!t2ToTd! to some trees of type \lstinline!T2[A]!.
The composition \lstinline!t2ToTD andThen tdToT2! should be the identity
function.

Begin writing the code of \lstinline!tdToT2! by pattern matching:
\begin{lstlisting}
def tdToT2[A]: TD[A] => T2[A] = {
  case Last(List(a))   => Leaf(a)    // Tree has a single leaf at level 0.
  case _               => ???
}
\end{lstlisting}
Valid tree descriptors can only have a non-empty first list if the
tree has a single leaf at level $0$. So, we may ignore all other
cases and continue with patterns that assume an empty first list:
\begin{lstlisting}[numbers=left]
def tdToT2[A]: TD[A] => T2[A] = {
  case Last(List(a))        => Leaf(a)    // Tree has a single leaf at level 0.
  case More(List(a), _)     => Leaf(a)    // Tree has a single leaf at level 0.         
  case More(List(), tail)   => ???        // Tree has two branches. 
}
\end{lstlisting}
In line $4$, we need to restore two branches of the tree from the
given value \lstinline!tail! of type \lstinline!TD[Either[A, A]]!.
How can we do that? The left branch contains all the leaf data from
\lstinline!tail! where the outermost wrapper is a \lstinline!Left()!.
We can select just that subset of leaf data and remove the outermost
\lstinline!Left()! wrapper by implementing the helper functions \lstinline!removeLeft!
and \lstinline!filterLeft! as shown here:
\begin{lstlisting}
def removeLeft[A]: List[Either[A, A]] => List[A] = _.collect { case Left(x) => x }

def filterLeft[A]: TD[Either[A, A]] => TD[A] = {
  case Last(la)         => Last(removeLeft(la))
  case More(la, tail)   => More(removeLeft(la), filterLeft(tail))
}
\end{lstlisting}
For example, if we apply \lstinline!filterLeft! to the descriptor
\lstinline!t2ToTD(t2).tail! then we will get:
\begin{lstlisting}
filterLeft(t2ToTD(t2).tail) == More(List(8), More(List(), Last(List())))
\end{lstlisting}
This tree descriptor corresponds to the left subtree of \lstinline!t2!.

In a similar way, we implement \lstinline!filterRight! that extracts
the descriptor data for the right subtree:
\begin{lstlisting}
def removeRight[A]: List[Either[A, A]] => List[A] = _.collect { case Right(x) => x }
def filterRight[A]: TD[Either[A, A]] => TD[A] = {
  case Last(la)         => Last(removeRight(la))
  case More(la, tail)   => More(removeRight(la), filterRight(tail))
}

scala> filterRight(t2ToTD(t2).tail)
res1: TD[Int] = More(List(), More(List(Right(4)), Last(List(Left(Left(3)), Left(Right(5))))))
\end{lstlisting}

Now we are ready to complete the implementation of \lstinline!tdToT2!.
We use \lstinline!filterLeft! and \lstinline!filterRight! to obtain
the descriptors for the left and the right subtrees. Applying \lstinline!tdToT2!
recursively to those descriptors will restore the two subtrees:
\begin{lstlisting}
def tdToT2[A]: TD[A] => T2[A] = {
  case Last(List(a))        => Leaf(a)
  case More(List(a), _)     => Leaf(a)
  case More(List(), tail)   => Branch(tdToT2(filterLeft(tail)), tdToT2(filterRight(tail)))
}

scala> tdToT2(t2ToTD(t2)) == t2
res2: Boolean = true
\end{lstlisting}

The next step is to implement \lstinline!traverse! for the type constructor
\lstinline!TD[_]!, which makes \lstinline!TD[_]! into a traversable
functor. We will call its \lstinline!traverse! operation \textsf{``}\lstinline!travTD!\textsf{''}
for clarity. Since \lstinline!TD[A]! is essentially a decorated list
of lists, we will need \lstinline!List!\textsf{'}s \lstinline!traverse! (Example~\ref{subsec:Example-traversable-seq}),
which we here denote by \lstinline!travList!. We will also need a
\lstinline!traverse! function for the functor \lstinline!Either[A, A]!:
\begin{lstlisting}
def travEither[A, B, F[_]: Functor](f: A => F[B])(e: Either[A, A]): F[Either[B, B]] = e match {
  case Left(a)    => f(a).map(Left(_))
  case Right(a)   => f(a).map(Right(_))
}
\end{lstlisting}
Using \lstinline!travList! and \lstinline!travEither!, we can now
implement \lstinline!travTD! like this:
\begin{lstlisting}
def travTD[A, B, F[_]: Applicative : Functor](f: A => F[B])(td: TD[A]): F[TD[B]] = td match {
  case Last(a)         => travList(f)(a).map(Last(_))
  case More(a, tail)   =>
    val headFB: F[List[B]] = travList(f)(a)
    val tailFB: F[TD[Either[B, B]]] = travTD { x: Either[A, A] => travEither(f)(x) }(tail)
    (headFB zip tailFB).map { case (headB, tailB) => More(headB, tailB) }
}
\end{lstlisting}

The final step is to convert \lstinline!TD!\textsf{'}s \lstinline!traverse!
into \lstinline!T2!\textsf{'}s \lstinline!traverse! by using \lstinline!t2ToTD!
and \lstinline!tdToT2!. We first compute the tree descriptor of a
given tree and use \lstinline!travTD! to perform a traversal. The
result is a value of type \lstinline!F[TD[B]]!, which we convert
into a value of type \lstinline!F[T2[B]]! using \lstinline!tdToT2!
lifted to \lstinline!F!:
\begin{lstlisting}
def travBF[A, B, F[_]: Applicative : Functor](f: A => F[B])(tree: T2[A]): F[T2[B]] =
  travTD(f)(t2ToTD(tree)).map(tdToT2) 
\end{lstlisting}

This completes our implementation of a breadth-first traversal for
binary trees. To test the code, we run the example of decorating a
tree with the breadth-first traversal index:
\begin{lstlisting}
def zipWithIndexBF[A](tree: T2[A]): T2[(A, Int)] = {
  val afterTraverse: St[T2[(A, Int)]] = travBF[A, (A, Int), St](computeIndex)(tree)
  afterTraverse.run(0)._1                        // Run the State monad and get the result value.
}

scala> zipWithIndexBF(t2)
res3: T2[(Int, Int)] = Branch(Leaf((8,0)), Branch(Branch(Leaf((3,2)), Leaf((5,3))), Leaf((4,1))))
\end{lstlisting}

The only difference between \lstinline!zipWithIndexDF! and \lstinline!zipWithIndexBF!
is the choice of the \lstinline!traverse! operation (\lstinline!trav!
or \lstinline!travBF!). The \textsf{``}decoration logic\textsf{''} is described by
the function \lstinline!computeIndex! and does not depend on the
traversal order. As a result, we gain flexibility: an arbitrary traversal
order may be used with an arbitrary \textsf{``}decoration logic\textsf{''}. 

\subsection{The \texttt{Traversable} typeclass. Implementing \texttt{scanLeft}
via \texttt{traverse}}

We define a \textsf{``}traversable functor\textsf{''} typeclass (called \lstinline!Traversable!)
by specifying a \lstinline!traverse! operation, which is convenient
to provide as an extension method:
\begin{lstlisting}
trait Traversable[L[_]] {
  def trav[A, B, F[_]: Applicative: Functor](f: A => F[B])(la: L[A]): F[L[B]]
}
implicit class TraversableOps[L[_], A](la: L[A])(implicit tr: Traversable[L]) {
  def traverse[B, F[_]: Applicative : Functor](f: A => F[B]): F[L[B]] = tr.trav(f)(la)
}
\end{lstlisting}

As we have mentioned in Section~\ref{subsec:The-traverse-operation},
the \lstinline!foldLeft! operation can be defined via \lstinline!traverse!.
The \lstinline!foldLeft! operation transforms an initial sequence
into a result value by updating some internal state. Another function
similar to \lstinline!foldLeft! is \lstinline!scanLeft!; the difference
is that in \lstinline!scanLeft!, the result value is the \emph{sequence}
of all intermediate state values rather than just the last state value.
Let us see how the functionality of \lstinline!scanLeft! can be expressed
using \lstinline!traverse! with a \lstinline!State! monad that handles
the state updates. This will allow us to implement \lstinline!scanLeft!
automatically for every traversable functor.

Assume that \lstinline!L[_]! is a traversable functor whose \lstinline!trav!
function is available:
\begin{lstlisting}
def trav[A, B, F[_]: Applicative: Functor](f: A => F[B])(la: L[A]): F[L[B]] = ...
\end{lstlisting}
We would like to implement \lstinline!scanLeft! with the following
type signature:
\begin{lstlisting}
def scanLeft[A, Z](la: L[A])(init: Z)(f: (A, Z) => Z): L[Z] = ???
\end{lstlisting}
The standard \lstinline!scanLeft! method for lists produces a list
with an extra initial element, which is always equal to the initial
value \lstinline!init!. This initial element is not essential and
may be omitted from \lstinline!scanLeft!. In fact, we \emph{need}
to omit that initial element when we generalize \lstinline!scanLeft!
to data types \lstinline!L[A]! other than \lstinline!List[A]!, since
those data types will not necessarily support adding one more data
item.

In order to update the state of type \lstinline!Z! and also store
the state\textsf{'}s value as the result of the traversal, we will use the
\lstinline!State! monad and compute a value of type \lstinline!State[Z, Z]!.
The code of \lstinline!scanLeft! calls \lstinline!traverse! with
a suitable function \lstinline!accum! that creates the required value
of type \lstinline!State[Z, Z]!:
\begin{lstlisting}
case class State[Z, A](run: Z => (A, Z)) // Assume Applicative and Functor instances for State[Z, *].
def accum[A, Z](a: A, f: (A, Z) => Z): State[Z, Z] = State { z =>
  val newZ = f(a, z)  // Update the internal state using `f`.
  (newZ, newZ)        // Store the internal state, and also return it as a result value.
}
implicit class TraversableScanOps[L[_] : Traversable, A](la: L[A]) {
  def scanLeft[Z](init: Z)(f: (A, Z) => Z): L[Z] = la.traverse(a => accum(a, f)).run(init)._1
}
\end{lstlisting}
In this way, \lstinline!scanLeft! is made available for all traversable
functors.

To test this code, we implement \lstinline!zipWithIndex! via depth-first
traverse for a tree of type \lstinline!T2!:
\begin{lstlisting}
def zipWithIndexDFS[A]: T2[A] => T2[(A, Int)] = 
  _.scanLeft[(A, Int)]((null.asInstanceOf[A], -1)) { case (a, (_, i)) => (a, i + 1) }
val t2 = Branch(Leaf(8), Branch(Branch(Leaf(3), Leaf(5)), Leaf(4)))

scala> zipWithIndexDFS(t2)
res0: T2[(Int, Int)] = Branch(Leaf((8,0)), Branch(Branch(Leaf((3,1)), Leaf((5,2))), Leaf((4,3))))
\end{lstlisting}


\subsection{Tasks that cannot be performed via \texttt{traverse}\label{subsec:Tasks-not-implementable-via-traverse}}

The \lstinline!traverse! function is powerful since it can use an
arbitrary applicative functor \lstinline!F[_]!. However, some computations
are still not expressible via \lstinline!traverse! because they require
information that \lstinline!traverse! cannot have. We will now look
at two examples of this.

The first example is the depth labeling of a tree: each leaf gets
a value equal to its depth. For instance, the tree {\tiny{} \Tree[ 8 [ [ 3 5 ] 4 ] ] }
becomes {\tiny{}}{\tiny{} \Tree[ (8,1) [ [ (3,3) (5,3) ] (4,2) ] ] }
after depth labeling. This cannot be implemented via \lstinline!traverse!
because it cannot detect nodes that have the same depth in the tree.
To see this in more detail, recall that the code of \lstinline!traverse!
always collects and merges all the effects of a given functor \lstinline!F!.
If the effect of \lstinline!F! describes incrementing a counter (as
in our examples involving the \lstinline!State! monad), the code
of \lstinline!traverse! will increment that counter at \emph{each}
leaf. We may traverse the tree so that leaves at the same depth are
traversed next to each other (as in the breadth-first traversal).
But the code of \lstinline!traverse! cannot skip the incrementing
when a leaf is at the same depth: \lstinline!traverse! does not receive
any information about the position of values in the tree. So, we cannot
label certain nodes with the same depth value but other nodes with
a different depth value.

Depth labeling can be implemented as a special operation such as \lstinline!zipWithDepth!
for the tree type \lstinline!T2!:
\begin{lstlisting}
def zipWithDepth[A](initial: Int = 0): T2[A] => T2[(A, Int)] = {
  case Leaf(a)        => Leaf((a, initial))
  case Branch(l, r)   => Branch(zipWithDepth(initial + 1)(l), zipWithDepth(initial + 1)(r))
}
val t2: T2[Int] = Branch(Leaf(8), Branch(Branch(Leaf(3), Leaf(5)), Leaf(4)))

scala> zipWithDepth()(t2)
res0: T2[(Int, Int)] = Branch(Leaf((8,1)), Branch(Branch(Leaf((3,3)), Leaf((5,3))), Leaf((4,2))))
\end{lstlisting}

More generally, traversals cannot perform computations that depend
on the \emph{position} of data in the tree (e.g., whether the data
is in a left or in a right branch and at what depth). An example of
such a computation is \textsf{``}pretty-printing\textsf{''} that converts trees into
a text form. In the \LaTeX{} format used to typeset this book, the
tree {\tiny{} \Tree[ 8 [ [ 3 5 ] 4 ] ] } is represented by the text
string \lstinline!"\Tree[ 8 [ [ 3 5 ] 4 ] ]"!. We may write the following
code for converting trees to this format:
\begin{lstlisting}
def printLaTeX[A](t: T2[A])(toString: A => String): String = {
  def printLaTeXSubtree: T2[A] => String = {
    case Leaf(a)        => toString(a)
    case Branch(l, r)   => "[ " + printLaTeXSubtree(l) + " " + printLaTeXSubtree(r) + " ]"
  }
  "\\Tree" + printLaTeXSubtree(t)
} 

scala> printLaTeX(t2)(_.toString)
res1: String = \Tree[ 8 [ [ 3 5 ] 4 ] ]
\end{lstlisting}


\subsection{Recursion schemes. I. Folding operations\label{subsec:Recursion-schemes.-folding}}

The previous section showed two examples of folding and traversing
operations that cannot be expressed through the standard \lstinline!foldMap!
or \lstinline!traverse! methods and are instead implemented via custom
recursive code. If we needed to implement the same operations for
trees of different shapes or for other recursive data types, we would
need to write new custom code for each new data type. That code will
contain a certain common pattern: it will use recursive calls at the
points where the recursive data type refers to itself. It turns out
that we can separate this common pattern from the custom code, reducing
the effort required for implementing the folding and traversing operations
for different recursive data types.

To begin, write the recursive definition of the tree \lstinline!T2!
as:
\[
\text{T2}^{A}\triangleq A+\text{T2}^{A}\times\text{T2}^{A}\quad.
\]
This type refers recursively to itself in two places. To express that,
define a bifunctor $S^{A,R}$:
\[
S^{A,R}\triangleq A+R\times R\quad.
\]
We can now rewrite the definition of \lstinline!T2! as a recursive
type equation: $\text{T2}^{A}\triangleq S^{A,\text{T2}^{A}}$. The
corresponding Scala code is:
\begin{lstlisting}
type S[A, R] = Either[A, (R, R)]
final case class T2[A](run: S[A, T2[A]])
\end{lstlisting}

The bifunctor\index{bifunctor} $S^{\bullet,\bullet}$ is called the
\textbf{recursion scheme}\index{recursion scheme} of the type \lstinline!T2!.
The recursion scheme describes the places where the recursive type
refers to itself in its definition. All the recursive uses correspond
to occurrences of the type parameter $R$ in $S^{A,R}$. 

The folding operation \lstinline!foldMap! takes a function $f$ of
type $A\rightarrow Z$ as a parameter:
\[
\text{foldMap}_{L}(f^{:A\rightarrow Z}):L^{A}\rightarrow Z\quad.
\]
The function $f$ will be applied to all values of type $A$ stored
inside $L^{A}$. As we have seen in the previous section, the function
$f^{:A\rightarrow Z}$ cannot receive any information about the location
of values of type $A$ inside $L^{A}$. In order to access that information
and make the folding operation \textsf{``}location-aware\textsf{''}, we need to change
the type signature of $f$. To figure out the new type signature,
let us look at the code of \lstinline!printLaTeXSubtree! (short notation
\textsf{``}\lstinline!pls!\textsf{''}) from the previous section:
\[
\text{pls}\triangleq\,\begin{array}{|c||c|}
 & \text{String}\\
\hline A & \text{toString}\\
\text{T2}^{A}\times\text{T2}^{A} & (\overline{\text{pls}}\boxtimes\overline{\text{pls}})\bef(l\times r\rightarrow\text{"[ "}+l+\text{" "}+r+\text{" ]"})
\end{array}\quad.
\]
We can rewrite this code using the bifunctor $S^{A,R}\triangleq A+R\times R$.
In this example, the result type $Z$ is \lstinline!String!. By calling
\lstinline!printLaTeXSubtree! recursively on the left and the right
subtrees, we obtain two values of type $Z$. We can separate the recursive
calls to \lstinline!printLaTeXSubtree! from the custom string processing
and express \lstinline!printLaTeXSubtree! as the following function
composition:
\[
\text{pls}\triangleq\,\begin{array}{|c||cc|}
 & A & Z\times Z\\
\hline A & \text{id} & \bbnum 0\\
\text{T2}^{A}\times\text{T2}^{A} & \bbnum 0 & \overline{\text{pls}}\boxtimes\overline{\text{pls}}
\end{array}\,\bef\,\begin{array}{|c||c|}
 & Z\\
\hline A & \text{toString}\\
Z\times Z & l\times r\rightarrow\text{"[ "}+l+\text{" "}+r+\text{" ]"}
\end{array}\quad.
\]
The intermediate result is a data structure of type $A+Z\times Z$,
to which we need to apply some string manipulations. We note that
the type $A+Z\times Z$ is the same as $S^{A,Z}$, while the first
matrix in the code above is the lifting $\overline{\text{pls}}^{\uparrow S^{A,\bullet}}\,$with
respect to the type parameter $R$ of $S^{A,R}$. So, we may express
the custom string manipulations specific to \lstinline!printLaTeXSubtree!
via a function \lstinline!toLaTeX! of type $S^{A,Z}\rightarrow Z$:
\[
\text{toLaTeX}:S^{A,Z}\rightarrow Z\quad,\quad\quad\text{toLaTeX}\triangleq\,\begin{array}{|c||c|}
 & Z\\
\hline A & \text{toString}\\
Z\times Z & l\times r\rightarrow\text{"[ "}+l+\text{" "}+r+\text{" ]"}
\end{array}\quad.
\]
Using this function, the implementation of \lstinline!printLaTeXSubtree!
can be rewritten as:
\[
\text{pls}\triangleq\overline{\text{pls}}^{\uparrow S^{A,\bullet}}\bef\text{toLaTeX}\quad.
\]
All custom \textsf{``}location-aware\textsf{''} logic is now encapsulated in the (non-recursive!)
function \lstinline!toLaTeX!. So, we can now generalize the calculation
by defining a new \lstinline!fold! function (denoted by $\text{fold}_{S}$):
\begin{equation}
\text{fold}_{S}:(S^{A,Z}\rightarrow Z)\rightarrow L^{A}\rightarrow Z\quad,\quad\quad\text{fold}_{S}(f)\triangleq\overline{\text{fold}_{S}(f)}^{\uparrow S^{A,\bullet}}\bef f\quad.\label{eq:fold-via-recursion-scheme-1}
\end{equation}
We then obtain $\text{pls}=\text{fold}_{S}(\text{toLaTeX})$. The
corresponding Scala code is:
\begin{lstlisting}
def fmapR[A, R, T](f: R => T): S[A, R] => S[A, T] = _.map { case (r1, r2) => (f(r1), f(r2)) }

def foldS[A, Z](f: S[A, Z] => Z)(tree: T2[A]): Z = f(fmapR(foldS(f))(tree.run))

def toLaTeX[A]: S[A, String] => String = {
  case Left(a)         => a.toString
  case Right((l, r))   => s"[ $l $r ]"
}
def printLaTeX[A](tree: T2[A]): String = "\\Tree" + foldS[A, String](toLaTeX)(tree)

val t2: T2[Int] = T2(Right((T2(Left(8)), T2(Right((T2(Right((T2(Left(3)), T2(Left(5))))), T2(Left(4))))))))

scala> printLaTeX(t2)
res0: String = \Tree[ [ 8 [ 3 5 ] ] 4 ]
\end{lstlisting}

It is important that the function $\text{fold}_{S}$ is parametric
in the recursion scheme $S^{\bullet,\bullet}$ and the result type
$Z$ (which is not required to be a monoid). Different recursion schemes
$S^{\bullet,\bullet}$ may be used to define lists, trees, and other
recursive data types. The same code of $\text{fold}_{S}$ will work
for all those data types, as long as we have the recursion scheme
$S^{\bullet,\bullet}$ and the corresponding lifting function (\lstinline!fmapR!
in the Scala code shown above, or $^{\uparrow S^{A,\bullet}}$ in
the code notation).

To illustrate the general applicability of $\text{fold}_{S}$ to different
data types, let us implement a \lstinline!printLaTeX! function for
ordinary lists, for non-empty lists, and for rose trees (Section~\ref{subsec:Rose-trees}).

Scala\textsf{'}s standard \lstinline!List! type and the corresponding recursion
scheme $S^{A,R}$ may be defined by:
\[
\text{List}^{A}\triangleq\bbnum 1+A\times\text{List}^{A}\quad,\quad\quad S^{A,R}\triangleq\bbnum 1+A\times R\quad,\quad\quad\text{List}^{A}\triangleq S^{A,\text{List}^{A}}\quad.
\]

A non-empty list type (\lstinline!NEL!) and the corresponding recursion
scheme $S^{A,R}$ may be defined by:
\[
\text{NEL}^{A}\triangleq A+A\times\text{NEL}^{A}\quad,\quad\quad S^{A,R}\triangleq A+A\times R\quad,\quad\quad\text{NEL}^{A}\triangleq S^{A,\text{NEL}^{A}}\quad.
\]

A rose tree (\lstinline!TreeN!) and the corresponding recursion scheme
$S^{A,R}$ may be defined by:
\[
\text{TreeN}^{A}\triangleq A+\text{NEL}^{\text{TreeN}^{A}}\quad,\quad\quad S^{A,R}\triangleq A+\text{NEL}^{R}\quad,\quad\quad\text{TreeN}^{A}\triangleq S^{A,\text{TreeN}^{A}}\quad.
\]
For rose trees, the recursion scheme $S^{A,R}$ is itself a recursively
defined type because it uses the non-empty list (NEL). This is not
a problem: $S^{A,R}$ is still polynomial, which guarantees that any
value of type $S^{A,R}$ contains a finite number of elements of types
$A$ and $R$. So, any value of type \lstinline!TreeN[A]! will contain
a finite number of values of type $A$, assuring that the folding
operation will terminate.

To avoid repetitive code, let us define all three data types (\lstinline!List!,
\lstinline!NEL!, and \lstinline!TreeN!) at once through a universal
recursive class \lstinline!Fix! that takes the recursion scheme $S^{\bullet,\bullet}$
as a type parameter:
\begin{lstlisting}
type S1[A, R] = Option[(A, R)]             // For List.
type S2[A, R] = Either[A, (A, R)]          // For NEL.
type S3[A, R] = Either[A, NEL[R]]          // For TreeN.
final case class Fix[S[_, _], A](unfix: S[A, Fix[S, A]])
\end{lstlisting}
Using the class \lstinline!Fix!, we may write the equivalent definitions
\lstinline!List[A]! $\cong$ \lstinline!Fix[S1, A]!, \lstinline!NEL[A]!
$\cong$ \lstinline!Fix[S2, A]!, and \lstinline!TreeN[A]! $\cong$
\lstinline!Fix[S3, A]!. Some example values of these types are:
\begin{lstlisting}
val x1 = Fix[S1, Int](Some((1, Fix[S1, Int](Some((2, Fix[S1, Int](Some((3, Fix[S1, Int](None))))))))))  // Equivalent to List(1, 2, 3) .
val x2 = Fix[S2, Int](Right((1, Fix[S2, Int](Right((2, Fix[S2, Int](Left(3))))))))  // NEL(1, 2, 3) .
val x3 = Fix[S3, Int](Right(NEL(Fix[S3, Int](Right(NEL(Fix[S3, Int](Left(10)), Fix[S3, Int](Right(NEL(Fix[S3, Int](Left(20)), Fix[S3, Int](Left(30)))))))), Fix[S3, Int](Left(40)))))
\end{lstlisting}

In practice, defining data structures via \lstinline!Fix! is both
inconvenient and inefficient. We show these definitions only to clarify
how one can generalize \textsf{``}location-aware\textsf{''} folding operations to
arbitrary recursive data types. (For simplicity, the definition of
\lstinline!S3! uses \lstinline!NEL! rather than \lstinline!Fix[S2, A]!.)

To make the code of $\text{fold}_{S}$ general, we add a functor typeclass
constraint to $S^{A,\bullet}$:
\begin{lstlisting}
def fold[A, Z, S[_,_]](f: S[A, Z] => Z)(t: Fix[S, A])(implicit fs: Functor[S[A, *]]): Z =
  f(t.unfix.map(fold(f)))
\end{lstlisting}
Below we will assume that the appropriate functor instances are defined
for \lstinline!S1!, \lstinline!S2!, and \lstinline!S3!.

It remains to implement functions with type signatures \lstinline!S[A, String] => String!:
\begin{lstlisting}
def toLaTeX1[A]: S1[A, String] => String = {
  case None                 => "Nil"
  case Some((head, tail))   => head.toString + ", " + tail
}
def toLaTeX2[A]: S2[A, String] => String = {
  case Left(a)               => a.toString
  case Right((head, tail))   => head.toString + ", " + tail
}
def toLaTeX3[A]: S3[A, String] => String = {
  case Left(a)      => a.toString
  case Right(nel)   => "[ " + nel.mkString(" ") + " ]" // Assume mkString() is defined for NEL.
}
\end{lstlisting}
We can now use $\text{fold}_{S}$ to convert some data structures
to a \LaTeX{} form:
\begin{lstlisting}
def listToLaTeX[A](t: Fix[S1, A]): String = "[ " + fold[A, String, S1](toLaTeX1)(t) + " ]"
def nelToLaTeX[A](t: Fix[S2, A]): String = "[ " + fold[A, String, S2](toLaTeX2)(t) + " ]"
def treeNToLaTeX[A](t: Fix[S3, A]): String = "\\Tree" + fold[A, String, S3](toLaTeX3)(t)

scala> listToLaTeX(x1)
res1: String = [ 1, 2, 3, Nil ]

scala> nelToLaTeX(x2)
res2: String = [ 1, 2, 3 ]

scala> treeNToLaTeX(x3)
res3: String = \Tree[ [ 10 [ 20 30 ] ] 40 ]
\end{lstlisting}

The definition of $\text{fold}_{S}$ calls itself recursively under
the lifting ($\overline{\text{fold}_{S}(f)}^{\uparrow S^{A,\bullet}}$).
Is it guaranteed that this recursion terminates? The only way it can
terminate is when the lifting $f^{\uparrow S^{A,\bullet}}$ does not
\emph{always} call the function $f$. This will happen if $S^{A,R}$
is a disjunctive type with some parts that do not depend on $R$.
This is indeed the case for lists ($S^{A,R}\triangleq\bbnum 1+A\times R$
or $S^{A,R}\triangleq A+A\times R$) and trees ($S^{A,R}\triangleq A+R\times R$).
Applying $f^{\uparrow S^{A,\bullet}}$ to values of $R$-independent
types will be an identity function; it will not actually call $f$.
This will be the base case of the recursion.

It remains to assure that the recursion reaches the base case with
every value of type $L^{A}$. That is, no values of type $L^{A}$
should cause an infinite loop in $\text{fold}_{S}$. A simple example
where $\text{fold}_{S}$ enters an infinite loop is the recursion
scheme $S^{A,R}\triangleq A+(\bbnum 1\rightarrow R)$. Note that $S^{A,R}$
is non-polynomial due to the function type $\bbnum 1\rightarrow R$,
which delays the evaluation of a value of type $R$. This allows us
to implement a well-defined, finite value \lstinline!x: L[A]! which
refers to itself under the delayed evaluation:
\begin{lstlisting}
type S[A, R] = Either[A, Unit => R]
final case class Looping[A](run: S[A, Looping[A]])
lazy val x: Looping[Int] = Looping(Right(_ => x))

scala> x.run.right.get(())                      // No stack overflows.
val res0: Looping[Int] = Looping(Right($Lambda$1123/1058984040@753aca85)) 

scala> x.run.right.get(()).run.right.get(())    // This is again the same value:
val res1: Looping[Int] = Looping(Right($Lambda$1123/1058984040@753aca85))
\end{lstlisting}
Trying to compute \lstinline!fold(f)(x)! with any \lstinline!f!
will result in an infinite loop.

It seems that we need to restrict recursion schemes $S^{A,R}$ to
\emph{polynomial} bifunctors. Such $S^{A,R}$ will define recursive
polynomial functors $L^{A}$ that support no delayed evaluation of
stored values of type $A$. So, any value \lstinline!x! of type $L^{A}$
will have to contain a finite number of values of type $A$, and \lstinline!fold(f)(x)!
is guaranteed to terminate for any terminating function $f:S^{A,Z}\rightarrow Z$.

Rather than working with the general $\text{fold}_{S}$ function and
redefine all recursive types via \lstinline!Fix!, it is more convenient
to implement and use specialized versions of $\text{fold}_{S}$ for
already defined recursive types. The general implementation of $\text{fold}_{S}$
in Eq.~(\ref{eq:fold-via-recursion-scheme-1}) can be translated
mechanically (e.g., using macros or code generators) into code specialized
for a given data type and recursion scheme.

For instance, while the type \lstinline!TreeN[A]! is equivalent to
\lstinline!Fix[S3[A, *]]! shown above, it is easier to work with
\lstinline!TreeN[A]!. The specialized version of $\text{fold}_{S}$
for \lstinline!TreeN[A]! has the type signature:
\begin{lstlisting}
def foldTreeN[A, Z](f: S3[A, Z] => Z): TreeN[A] => Z = ???
\end{lstlisting}
The general definition of $\text{fold}_{S}$ in Eq.~(\ref{eq:fold-via-recursion-scheme-1})
shows us how to write the code of \lstinline!foldTreeN!: 
\begin{lstlisting}
def foldTreeN[A, Z](f: Either[A, NEL[Z]] => Z): TreeN[A] => Z = {
  case Leaf(a)      => f(Left(a))
  case Branch(ts)   => f(Right(ts.map(foldTreeN(f))))
}
\end{lstlisting}
Then we can implement \lstinline!printLaTeX! for \lstinline!TreeN[A]!
like this:
\begin{lstlisting}
def printLaTeX[A](tree: TreeN[A]): String = "\\Tree" + foldTreeN[A, String](toLaTeX3)(tree)
\end{lstlisting}

Another simple example of an aggregation operation that cannot be
expressed as a traversal is the task of determining the maximum branching
number of a given rose tree. The function \lstinline!foldTreeN! now
allows us to implement that computation:
\begin{lstlisting}
def maxBranching[A]: TreeN[A] => Int = foldTreeN[A, Int] {
  case Left(_)      => 0
  case Right(nel)   => math.max(nel.max, nel.length)    // NEL must have `max` and `length` methods.
}

scala> maxBranching(x3)
res2: Int = 2
\end{lstlisting}


\subsection{Recursion schemes. II. Unfolding operations}

A folding operation converts a collection to a single value. The opposite
operation is \textsf{``}unfolding\textsf{''}: converting a single value into a collection.
By reversing the direction of certain function arrows in the type
signature of $\text{fold}_{S}$, we can define a general \textsf{``}unfolding\textsf{''}
method that uses an arbitrary recursion scheme $S^{A,R}$ and an arbitrary
function of type $Z\rightarrow S^{A,Z}$:
\begin{equation}
\text{unfold}_{S}:(Z\rightarrow S^{A,Z})\rightarrow Z\rightarrow L^{A}\quad,\quad\text{unfold}_{S}(f)\triangleq f\bef\overline{\text{unfold}_{S}(f)}^{\uparrow S^{A,\bullet}}\quad.\label{eq:unfold-via-recursion-scheme}
\end{equation}
Section~\ref{sec:ch2Converting-a-single} showed an unfolding operation
for sequences: starting from an initial value, a function is applied
repeatedly to compute further elements of the sequence. The \lstinline!unfold!
operation generalizes that computation to an arbitrary recursive type
$L^{A}$ whose recursion scheme $S^{A,R}$ is given. 

To get more intuition, we look at some examples using \lstinline!unfold!
with lists and binary trees.

\subsubsection{Example \label{subsec:Example-unfold-list}\ref{subsec:Example-unfold-list}\index{solved examples}}

Use \lstinline!unfold! to create a \lstinline!List! of consecutive
powers of $2$ up to a given value $n$: 
\begin{lstlisting}
type S[A, R] = ???
type Z = ???

def unfoldList[A, Z](f: Z => S[A, Z])(init: Z): List[A] = ???

def powersOf2UpTo(n: Long): List[Long] = unfoldList(???)(???)

scala> powersOf2UpTo(1000)
res0: List[Long] = List(1, 2, 4, 8, 16, 32, 64, 128, 256, 512)
\end{lstlisting}


\subparagraph{Solution}

The recursion scheme for \lstinline!List[A]! is $S^{A,R}\triangleq\bbnum 1+A\times R$.
Let us specialize the code of \lstinline!unfold! from Eq.~(\ref{eq:unfold-via-recursion-scheme})
to the type \lstinline!List[A]!:
\begin{lstlisting}
type S[A, R] = Option[(A, R)]
def unfoldList[A, Z](f: Z => S[A, Z])(init: Z): List[A] = f(init) match {
  case None           => Nil
  case Some((a, z))   => a :: unfoldList(f)(z)
}
\end{lstlisting}

We now need to determine a suitable type $Z$ and a suitable function
$f:Z\rightarrow\bbnum 1+A\times Z$ so that the unfolding would produce
the required sequence. If we are in the middle of unfolding, we need
to produce the remaining portion of the list, say, $\left[128,256,512\right]$,
given only a current value of type $Z$. We can do that if the current
value of type $Z$ is $128$ (the smallest remaining power of $2$).
So, let us choose $Z\triangleq$ \lstinline!Long! to represent the
smallest remaining power of $2$. The type $A$ will be also \lstinline!Long!. 

How can we implement the function $f$? It should return \lstinline!None!
(denoted by $1+\bbnum 0$) when the list is finished. When we are
in the middle of generating the list, the call $f(z)$ should return
$\bbnum 0+a\times z^{\prime}$ with some values $a$ and $z^{\prime}$.
The value $a$ must be the new element of the list. The value $z^{\prime}$
will be passed to the next call of $f$. Since the next element must
be twice the previous one, we must have $a=z$ and $z^{\prime}=2*z$.
So, the code of $f$ is:
\begin{lstlisting}
def f(n: Long): Long => Option[(Long, Long)] = { z => if (z >= n) None else Some((z, z * 2)) }
\end{lstlisting}
Note that the code of $f$ is \emph{not} recursive, and the value
$n$ is captured inside the nameless function returned by $f(n)$.
We can now complete the solution:
\begin{lstlisting}
def powersOf2UpTo(n: Long): List[Long] = unfoldList(f(n))(1)
\end{lstlisting}


\subsubsection{Example \label{subsec:Example-unfold-tree}\ref{subsec:Example-unfold-tree}}

Implement \lstinline!unfoldT2! for the data type \lstinline!T2[A]!
from Section~\ref{subsec:Recursion-schemes.-folding}. Use it to
create \textsf{``}full\textsf{''} binary trees of given depth, e.g., {\tiny{}\Tree[ [ 0 1 ] [ 2 3 ] ]}
(depth $2$) and {\tiny{}\Tree[ [ [ 0 1 ] [ 2 3 ] ] [ [ 4 5 ] [ 6 7 ] ] ]}
(depth $3$).

\subparagraph{Solution}

We adapt the general code in Eq.~(\ref{eq:unfold-via-recursion-scheme})
to obtain the code of \lstinline!unfoldT2!:
\begin{lstlisting}
type S[A, R] = Either[A, (R, R)]
def unfoldT2[A, Z](f: Z => S[A, Z])(init: Z): T2[A] = f(init) match {
  case Left(a)           => Leaf(a)
  case Right((z1, z2))   => Branch(unfoldT2(f)(z1), unfoldT2(f)(z2))
}
\end{lstlisting}
We plan to implement the function \lstinline!fullBinaryTree! like
this:
\begin{lstlisting}
def fullBinaryTree(n: Int): T2[Int] = {
  type Z = ???
  val init: Z = ???
  val f: Z => Either[Int, (Z, Z)] = ???
  unfoldT2[Int, Z](f)(init)
}
\end{lstlisting}
The next step is to choose a suitable type \lstinline!Z! such that
the unfolding procedure can generate trees of the required shape.
To figure out what \lstinline!Z! must be, we need to consider an
intermediate step that generates a subtree at some point in the middle
of \textsf{``}unfolding\textsf{''}. For the tree of depth $3$ as shown above, an
example of a subtree in the middle is {\tiny{}\Tree[ 4 5 ]}. This
subtree must be computed as \lstinline!unfoldT2(f)(z)! with some
\lstinline!z! of type \lstinline!Z!. The information needed for
this computation is the initial value ($4$) and the total number
($2$) of the subtree\textsf{'}s leaves. So, the type \lstinline!Z! needs
to contain two integers: the value of the first leaf and the size
of the remaining subtree (which will always be a power of $2$).
\begin{lstlisting}[mathescape=true]
type Z = (Int, Int)        // (k, m) where k is the first leaf$\color{dkgreen}\texttt{'}$s value and m is the subtree size.
val init: Z = (0, 1 << n)  // The size of the entire tree is 2 to the power n.
\end{lstlisting}
The initial value for the entire tree shown above will be \lstinline!(0, 8)!.

Next, we figure out the function \lstinline!f!. When \lstinline!f!
is applied to a value \lstinline!(k, m)! of type \lstinline!Z!,
it should generate the corresponding subtree. If the subtree size
\lstinline!m! is $1$, the return value is \lstinline!Left(k)!.
Otherwise, the return value should give two new values of type \lstinline!Z!
corresponding to the two subtrees one level deeper. Those two subtrees
have sizes \lstinline!m / 2! and initial values \lstinline!k! and
\lstinline!m / 2 + k!. So, the code of \lstinline!f! is:
\begin{lstlisting}
val f: Z => Either[Int, (Z, Z)] = {
  case (k, m) if m == 1   => Left(k)
  case (k, m)             => Right(((k, m / 2), (m / 2 + k, m / 2)))
}
\end{lstlisting}

This completes the implementation of \lstinline!fullBinaryTree!.
To test the resulting code, compute a full tree

\begin{wrapfigure}{l}{0.74\columnwidth}%
\vspace{-0.85\baselineskip}
\begin{lstlisting}
scala> fullBinaryTree(2)
res0: T2[Int] = Branch(Branch(Leaf(0), Leaf(1)), Branch(Leaf(2), Leaf(3)))
\end{lstlisting}

\vspace{0.5\baselineskip}
\end{wrapfigure}%

\noindent of depth $2$:~{\tiny{}\Tree[ [ 0 1 ] [ 2 3 ] ]}\\
~

\subsubsection{Example \label{subsec:Example-unfold-tree-evenodd}\ref{subsec:Example-unfold-tree-evenodd}}

Use \lstinline!unfoldT2! from Example~\ref{subsec:Example-unfold-tree}
to implement the function \lstinline!evenOdd(n)! that generates binary
trees of type \lstinline!T2[Int]! where the leaves have descending
numbers from \lstinline!n! to \lstinline!0!, but all odd numbers
are on the left and all even numbers on the right. For example, \lstinline!evenOdd(3)!
should generate the tree {\tiny{}\Tree[ 3 [ [ 1 0 ] 2 ] ]} , while
\lstinline!evenOdd(4)! should give the tree {\tiny{}\Tree[ [ 3 [ [ 1 0 ] 2 ] ] 4 ]}
.

\subparagraph{Solution}

We plan to implement \lstinline!evenOdd! like this:
\begin{lstlisting}
def evenOdd(n: Int): T2[Int] = {
  type Z = ???
  val init: Z = ???
  val f: Z => Either[Int, (Z, Z)] = ???
  unfoldT2[Int, Z](f)(init)
}
\end{lstlisting}
The examples with \lstinline!evenOdd(3)! and \lstinline!evenOdd(4)!
suggest that \lstinline!evenOdd(n)! is a tree containing a leaf with
value \lstinline!n! and a subtree \lstinline!evenOdd(n - 1)!. Can
we use the integer \lstinline!n! as the initial value for unfolding? 

To figure this out, consider an intermediate stage of the unfolding
process where \lstinline!unfoldT2! will apply the function \lstinline!f!
to some value \lstinline!z! of type \lstinline!Z!. If \lstinline!f(z) == Left(k)!,
we will get a \lstinline!Leaf! with value \lstinline!k!. The other
possibility is \lstinline!f(z) == Right((z1, z2))!, where \lstinline!z1!
and \lstinline!z2! are some values of type \lstinline!Z!. This will
generate a \lstinline!Branch! with two subtrees. The function \lstinline!unfoldT2!
will then apply \lstinline!f! to \lstinline!z1! and \lstinline!z2!
in order to create the left and the right subtrees. The difference
between those subtrees must come entirely from the difference between
the values \lstinline!z1! and \lstinline!z2!. 

In our case, we need to make a \lstinline!Leaf! either at the left
or at the right depending on whether the initial leaf value is odd
or even. The function \lstinline!f! must know whether \lstinline!f(z)!
should return a subtree or a \lstinline!Leaf!. This information can
only come from the value \lstinline!z!. So, the type \lstinline!Z!
must contain a \lstinline!Boolean! flag saying whether we need a
\lstinline!Leaf! at the current place. For clarity, let us define
the type \lstinline!Z! as a case class:
\begin{lstlisting}
final case class Z(startAt: Int, makeLeaf: Boolean)
val init = Z(startAt = n, makeLeaf = false)
\end{lstlisting}
When \lstinline!makeLeaf! is \lstinline!true!, we must create a
\lstinline!Leaf!, so \lstinline!f! should return a \lstinline!Left()!.
Otherwise \lstinline!f! should return a \lstinline!Right((z1, z2))!,
where \lstinline!z1! and \lstinline!z2! should set \lstinline!makeLeaf!
depending on \lstinline!startAt! being odd or even. Looking at the
examples, we find that \lstinline!f! must also return a \lstinline!Left()!
when \lstinline!startAt == 0!. So, the code of \lstinline!f! is:
\begin{lstlisting}
val f: Z => Either[Int, (Z, Z)] = {
  case Z(n, false) if n > 0 && n % 2 == 0  => Right((Z(n - 1, false), Z(n, true)))
  case Z(n, false) if n > 0 && n % 2 == 1  => Right((Z(n, true), Z(n - 1, false)))
  case Z(n, _)                             => Left(n) // Make a leaf when n == 0 or makeLeaf == true.
}
\end{lstlisting}
The implementation of \lstinline!evenOdd! is now complete. To test:

\begin{wrapfigure}{l}{0.74\columnwidth}%
\vspace{-0.95\baselineskip}
\begin{lstlisting}
scala> evenOdd(3)
res0: T2[Int] = Branch(Leaf(3), Branch(Branch(Leaf(1), Leaf(0)), Leaf(2)))
\end{lstlisting}

\vspace{0\baselineskip}
\end{wrapfigure}%

\noindent ~{\tiny{}\Tree[ 3 [ [ 1 0 ] 2 ] ]}

\noindent $\square$

The kind of reasoning shown in Examples~\ref{subsec:Example-unfold-list}\textendash \ref{subsec:Example-unfold-tree-evenodd}
is called \textbf{co-induction}\index{co-induction}. It is related
to mathematical induction but is significantly different from the
reasoning required to write the code for a folding operation (which
is directly modeled on induction). In co-induction, the base cases
are not at the beginning of the computation but \textsf{``}in the future\textsf{''}.
Note that \lstinline!unfold(f)(z)! will call itself whenever the
value $f(z)$ of type $S^{A,Z}$ contains additional values of type
$Z$. The programmer must carefully choose a suitable type $Z$ and
a suitable function $f$ such that \lstinline!unfold(f)(z)! stops
the recursion at the required places. For instance, if $S^{A,Z}\triangleq A+Z\times Z$,
the function $f$ must sometimes return a value of type $A+\bbnum 0$
to stop the unfolding.

Is the recursion guaranteed to stop while evaluating \lstinline!unfold!?
The type signature $f:Z\rightarrow S^{A,Z}$ itself does not guarantee
that $f(z)$ returns values that will stop the unfolding at the right
places. If $S^{A,Z}\triangleq A+Z\times Z$ and $f(z)$ always returns
values of type $\bbnum 0+Z\times Z$ (for example, $f(z)\triangleq\bbnum 0+z\times z$),
the unfolding operation \lstinline!unfold(f)! will enter an infinite
loop trying to constructing a tree of infinite size. This will, of
course, fail since data structures in a computer cannot have infinite
size.

Unfolding will always terminate if we use a data type that \emph{delays}
the evaluation of its recursively defined parts. Those parts will
be computed only on demand. To obtain further data, the code needs
to call certain functions. Data types of this kind are sometimes called
\textsf{``}infinite\textsf{''},\index{infinite data types} which is misleading since
no infinite amount of data is involved. A function may be called many
times and produce any number of result values, but it does not mean
that a function stores an infinite amount of data.

As an example, consider the recursion scheme $S^{A,R}\triangleq A+(\bbnum 1\rightarrow R\times R)$.
The corresponding data type $L^{A}\triangleq S^{A,L^{A}}$ is a binary
tree whose branches are evaluated on demand (but leaves are evaluated
eagerly). This data structure supports unfolding with \emph{any} function
$f:Z\rightarrow S^{A,Z}$ because the recursive evaluation of $f$
at the branches is always delayed. For instance, \lstinline!unfold!
can generate a value of type $L^{A}$ whose tree structure has the
form {\tiny{}\Tree[ 1 [ 2  [3 ... ] ] ]} with unbounded size:
\begin{lstlisting}
type S[A, R] = Either[A, Unit => (R, R)]

sealed trait UT[A]      // A tree with on-call evaluation of branches and eager evaluation of leaves.
case class ULeaf[A](a: A)                          extends UT[A]
case class UBranch[A](run: Unit => (UT[A], UT[A])) extends UT[A] // Call run(()) to get the branches.

def unfoldUT[A, Z](f: Z => S[A, Z])(init: Z): UT[A] = f(init) match {
  case Left(a)       => ULeaf(a)
  case Right(func)   => UBranch { _ =>     // It is important to delay the evaluation of func(()).
    val (z1, z2) = func(())                // Force the evaluation of branches at this level.
    (unfoldUT(f)(z1), unfoldUT(f)(z2))     // `unfold` will delay the evaluation of further branches.
  }
}

val tree1toInf = unfoldUT[Int, (Int, Boolean)] { case (z, makeLeaf) =>
  if (makeLeaf) Left(z) else Right(_ => ((z + 1, true), (z + 1, false)))
}((0, false))
\end{lstlisting}
The value \lstinline!tree1toInf! is finite but can compute on demand
a potentially unbounded number of leaves. To visualize \lstinline!tree1toInf!,
we write a function that converts \lstinline!UT[A]! to \lstinline!T2[A]!
by stopping at a given maximum depth. The unevaluated parts of the
tree will be marked with a value called \lstinline!default!:
\begin{lstlisting}
def toT2[A](maxDepth: Int, default: A): UT[A] => T2[A] = {
  case ULeaf(a)        => Leaf(a)e
  case UBranch(func)   => if (maxDepth == 0) Leaf(default) else {
    val (z1, z2) = func(())
    Branch(toT2(maxDepth - 1, default)(z1), toT2(maxDepth - 1, default)(z2))
  }
}
\end{lstlisting}
To test this code, let us truncate the infinite structure \lstinline!tree1toInf!
at depth $4$. The result is a finite tree of type \lstinline!T2[Int]!,
where the unevaluated part of the tree is shown as \textsf{``}\lstinline!-1!\textsf{''}: 

\begin{wrapfigure}{l}{0.65\columnwidth}%
\vspace{-0.95\baselineskip}
\begin{lstlisting}
scala> toT2(maxDepth = 3, default = -1)(tree1toInf)
res0: T2[Int] = Branch(Leaf(1), Branch(Leaf(2), Branch(Leaf(3), Branch(Leaf(4), Leaf(-1)))))
\end{lstlisting}

\vspace{-0.5\baselineskip}
\end{wrapfigure}%

\noindent ~{\tiny{}\Tree[ 1 [ 2  [3 [ 4 -1 ] ] ] ]}

\subsection{Recursion schemes. III. Traversing operations}

Folding with a recursion scheme ($\text{fold}_{S}$) allows us to
implement operations such as \lstinline!printLaTeXSubtree! (Section~\ref{subsec:Recursion-schemes.-folding})
that cannot be expressed via ordinary \lstinline!fold! or \lstinline!traverse!
functions. Another operation not expressible via \lstinline!traverse!
is \lstinline!zipWithDepth!, which we implemented in Section~\ref{subsec:Tasks-not-implementable-via-traverse}
through custom code. We will now implement \lstinline!zipWithDepth!
via a more general traversal operation ($\text{trav}_{S}$) parameterized
by an arbitrary recursion scheme $S^{\bullet,\bullet}$ and an arbitrary
functor $F^{\bullet}$ (not necessarily applicative).

To figure out the type signature of $\text{trav}_{S}$, consider the
relationship between \lstinline!foldMap!, $\text{fold}_{S}$, and
the ordinary \lstinline!traverse! (denoted $\text{trav}_{L}$):
\begin{align*}
 & \text{foldMap}_{L}(f^{:A\rightarrow Z}):L^{A}\rightarrow Z\quad,\quad\quad\text{trav}_{L}(f^{:A\rightarrow F^{B}}):L^{A}\rightarrow F^{L^{B}}\quad,\\
 & \text{fold}_{S}(f^{:S^{A,Z}\rightarrow Z}):L^{A}\rightarrow Z\quad,\quad\quad\text{trav}_{S}(f^{:???}):L^{A}\rightarrow F^{L^{B}}\quad.
\end{align*}
We recover \lstinline!foldMap! from \lstinline!traverse! by setting
the applicative functor $F$ as $F^{B}\triangleq Z$. So, we expect
to obtain the type signature of $\text{trav}_{S}$ from the type signature
of $\text{fold}_{S}$ if we replace $Z$ by $F^{L^{B}}$. The first
argument $f$ of $\text{trav}_{S}$ will then have the type $S^{A,F^{L^{B}}}\rightarrow F^{L^{B}}$:
\[
\text{trav}_{S}:(S^{A,F^{L^{B}}}\rightarrow F^{L^{B}})\rightarrow L^{A}\rightarrow F^{L^{B}}\quad.
\]
To implement this method, begin with the type equivalence $L^{A}\cong S^{A,L^{A}}$.
We can apply $\text{trav}_{S}(f)$ recursively to the values of type
$L^{A}$ stored inside $S^{A,L^{A}}$ and obtain a value of type $S^{A,F^{L^{B}}}$:
\[
\big(s^{:S^{A,L^{A}}}\triangleright\,\overline{\text{trav}_{S}(f)}^{\uparrow S^{A,\bullet}}\big):S^{A,F^{L^{B}}}\quad.
\]
It remains to apply $f$ to the last obtained value. This completes
the code of $\text{trav}_{S}$: 
\[
\text{trav}_{S}\big(f^{:S^{A,F^{L^{B}}}\rightarrow F^{L^{B}}}\big)\triangleq\overline{\text{trav}_{S}(f)}^{\uparrow S^{A,\bullet}}\bef f\quad.
\]

The method $\text{trav}_{S}$ works in the same way for all recursion
schemes $S$ and for all type constructors $F$. This makes $\text{trav}_{S}$
powerful but hard to use because we need to work with data structures
defined via the \lstinline!Fix! type constructor. It is easier to
use a specialized version of $\text{trav}_{S}$ for the data structure
at hand, similarly to what we did in the previous sections for folding
and unfolding. 

To illustrate this, let us implement \lstinline!zipWithDepth! via
a traversal with a binary tree recursion scheme. We will use the \lstinline!State!
monad as the functor $F$:
\begin{lstlisting}
final case class St[A](run: Int => (A, Int))  { // A State monad with internal state of type Int.
  import io.chymyst.ch.implement                // Implement the monad methods automatically.
  def flatMap[B](f: A => St[B]): St[B] = implement
  def map[B](f: A => B): St[B] = implement
}
def incrementAndGet: St[Int] = St(s => (s + 1, s + 1))  // Increment the current state value.
def get: St[Int] = St(s => (s, s))             // Fetch the current state value.
def set(s: Int): St[Unit] = St(_ => ((), s))   // Set the state, ignore previous state value.
\end{lstlisting}
Next, define the recursion scheme $S^{A,R}$ and a specialized version
of $\text{trav}_{S}$ for trees of type \lstinline!T2[A]!:
\begin{lstlisting}
type S[A, R] = Either[A, (R, R)]          // Recursion scheme for T2[A].
def travT2[A, B, F[_]](f: S[A, F[T2[B]]] => F[T2[B]]): T2[A] => F[T2[B]] = {
  case Leaf(a)        => f(Left(a))
  case Branch(l, r)   => f(Right((travT2(f)(l), travT2(f)(r))))
}
\end{lstlisting}
It remains to apply \lstinline!travT2! to a suitable function \lstinline!f!.
The value of the internal state will represent the current depth of
the tree element. We need to increment the depth whenever we find
a branch and then traverse the two subtrees starting from the same
depth value. The code is:
\begin{lstlisting}
def zipWithDepth[A](tree: T2[A]): T2[(A, Int)] = travS[A, (A, Int), St] {
  case Left(a) => for { s <- get } yield Leaf((a, s))   // Put the current depth into the Leaf value.
  case Right((l, r))   => for {
    s <- incrementAndGet   // Read the current depth after incrementing it.
    x <- l                 // Traverse the left branch starting from depth `s`.
    _ <- set(s)            // Set the same initial depth `s` for traversing the right branch.       
    y <- r                 // Traverse the right branch.
  } yield Branch(x, y)
}(tree).run(0)._1
\end{lstlisting}
To test the code, apply \lstinline!zipWithDepth! to a sample tree
\lstinline!t2! used earlier in Sections~\ref{subsec:Decorating-a-tree-breadth-first-traversal}
and~\ref{subsec:Tasks-not-implementable-via-traverse}:

\begin{wrapfigure}{l}{0.7\columnwidth}%
\vspace{-0.75\baselineskip}
\begin{lstlisting}
scala> zipWithDepth(t2)
res0: T2[(Int, Int)] = Branch(Leaf((8, 1)), Branch(Branch(Leaf((3, 3)), Leaf((5, 3))), Leaf((4, 2))))
\end{lstlisting}

\vspace{-0.5\baselineskip}
\end{wrapfigure}%

\noindent ~{\tiny{}\Tree[ (8,1) [ [ (3,3) (5,3) ] (4,2) ] ]}

\noindent ~

Here we need to use the \lstinline!State! monad\textsf{'}s method \lstinline!set(s)!
with a value \lstinline!s! obtained from a previous monadic computation.
So, the code involves $\text{trav}_{S}(f)$ with a function $f:S^{A,F^{L^{B}}}\rightarrow F^{L^{B}}$
whose $F$-effect is \emph{not} equivalent to an applicative functor\textsf{'}s
effect (which cannot depend on a previously computed value). We can
see that the recursion scheme-based traversal ($\text{trav}_{S}$)
is more powerful than the plain traversal, $\text{trav}_{L}(f^{:A\rightarrow F^{B}})$,
that may only use $F$\textsf{'}s applicative functor methods.

\subsection{Exercises\index{exercises}}

\subsubsection{Exercise \label{subsec:Exercise-traversables-7-1}\ref{subsec:Exercise-traversables-7-1}}

Implement \lstinline!foldMap! and \lstinline!traverse! for the following
type constructors:

\textbf{(a)} $F^{A}\triangleq\text{Int}+A+A\times A\quad.$

\textbf{(b)} $F^{A}\triangleq(\text{String}+A)\times(\bbnum 1+A\times A)\quad.$

\textbf{(c)} $F^{A}\triangleq A+A\times F^{A}$ (a recursive definition
equivalent to \lstinline!NEL!, the non-empty list).

\textbf{(d)} $F^{A}\triangleq\bbnum 1+A+A\times A\times F^{A}$ (a
recursive definition).

\textbf{(e)} $F^{A}\triangleq(\bbnum 1+F^{A})\times(\bbnum 1+A\times F^{A})$
(a recursive definition).

\subsubsection{Exercise \label{subsec:Exercise-traversables-7}\ref{subsec:Exercise-traversables-7}}

For the binary tree \lstinline!T2[A]! (Section~\ref{subsec:Recursion-schemes.-folding}),
implement a \lstinline!Traversable! instance for \emph{right-to-left}
depth-first traversal order. Use that to implement \lstinline!zipWithIndex!
for the type \lstinline!T2[A]!. Verify that \lstinline!zipWithIndex!
transforms the tree {\tiny{} \Tree[ [ 8 [ 3 5 ] ] 4 ] } into {\tiny{} \Tree[ [ (8,3) [ (3,2) (5,1) ] ] (4,0) ] }
.

\subsubsection{Exercise \label{subsec:Exercise-traversables-8}\ref{subsec:Exercise-traversables-8}}

Implement a \lstinline!Traversable! instance for the data type \lstinline!T3!
defined in Exercise~\ref{subsec:Exercise-applicative-I-1-1}.

\subsubsection{Exercise \label{subsec:Exercise-traversables-8-1}\ref{subsec:Exercise-traversables-8-1}}

Implement a \lstinline!Functor! instance for the data type \lstinline!TD!
defined in Section~\ref{subsec:Decorating-a-tree-breadth-first-traversal}.
Use \lstinline!TD!\textsf{'}s \lstinline!map! method to re-implement the
functions \lstinline!addLeft! and \lstinline!addRight! from that
section. Show that the new implementations are equivalent to the old
ones.

\subsubsection{Exercise \label{subsec:Exercise-traversables-12}\ref{subsec:Exercise-traversables-12}}

Use the specialized version of $\text{fold}_{S}$ for the binary tree
\lstinline!T2[A]! (Section~\ref{subsec:Recursion-schemes.-folding})
to compute the maximum depth of a tree:
\begin{lstlisting}
def maxDepth[A](tree: T2[A]): Int = foldT2(???)(???)

scala> maxDepth(Branch(Leaf(0), Branch(Branch(Leaf(0), Leaf(0)), Leaf(0))))
res0: Int = 3
\end{lstlisting}


\subsubsection{Exercise \label{subsec:Exercise-traversables-11}\ref{subsec:Exercise-traversables-11}}

For the data type \lstinline!T3[A]! defined in Exercise~\ref{subsec:Exercise-applicative-I-1-1},
define a recursion scheme and implement a specialized version of \lstinline!unfold!
as \lstinline!unfoldT3!. Using \lstinline!unfoldT3!, write a function
that generates ternary trees of the form {\tiny{}}{\tiny{} \Tree[.3  0  [.2 0   1   0  ]   0  ] }
starting from the given integer $n$ at the root.

\section{Laws and structure}

To study the laws of the folding and traversing operations, it helps
to choose simpler but equivalent versions of these operations. We
have seen four methods that implement folding operations: \lstinline!foldLeft!,
\lstinline!foldMap!, \lstinline!reduce!, and \lstinline!toList!.
With suitable naturality laws, these methods are equivalent.

\subsection{Equivalence of \texttt{reduce}, \texttt{foldLeft}, \texttt{foldMap},
and \texttt{toList}. Monoid morphisms\label{subsec:Equivalence-of-foldLeft,foldMap,reduce,and-toList}}

Scala\textsf{'}s standard \lstinline!reduce! method assumes a non-empty sequence
and will fail otherwise:
\begin{lstlisting}
def reduce[A](la: Seq[A])(red: (A, A) => A): A
\end{lstlisting}
For the purposes of this section, we will modify \lstinline!reduce!
to supply a default value for empty sequences:
\begin{lstlisting}
def reduceE[A](la: Seq[A])(default: A)(red: (A, A) => A): A =
if (la.isEmpty) default else reduce(la)(red)
\end{lstlisting}
Having a default value of type $A$ and a binary operation of type
$A\times A\rightarrow A$ suggest that $A$ may be a monoid (assuming
that the monoid laws hold, see Example~\ref{subsec:tc-Example-Monoids}).
So, we simplify the type signature of \lstinline!reduceE! using a
\lstinline!Monoid! typeclass constraint:
\begin{lstlisting}
def reduceE[M: Monoid](la: Seq[M]): M
\end{lstlisting}

After this change, let us compare the type signatures of the four
methods:
\begin{lstlisting}
def foldLeft[A, B](la: L[A])(init: B)(update: (B, A) => B): B
def foldMap[M: Monoid, A](f: A => M): L[A] => M
def reduceE[M: Monoid]: L[M] => M
def toList[A]: L[A] => List[A]
\end{lstlisting}
\begin{align*}
 & \text{foldLeft}_{L}:L^{A}\rightarrow B\rightarrow(B\times A\rightarrow B)\rightarrow B\quad,\quad\quad\text{foldMap}_{L}:(A\rightarrow M)\rightarrow L^{A}\rightarrow M\quad,\\
 & \text{reduceE}_{L}:L^{M}\rightarrow M\quad,\quad\quad\text{toList}_{L}:L^{A}\rightarrow\text{List}^{A}\quad.
\end{align*}
We will now show that these four functions are equivalent, assuming
certain naturality laws.

The formulation of naturality laws is different for functions whose
type parameters have typeclass constraints. For example, consider
\lstinline!reduceE! whose type parameter $M$ is constrained to be
a monoid. The naturality law for functions $\phi$ with type signature
$\forall M.\,L^{M}\rightarrow M$ (but without the typeclass constraint)
is:
\[
\text{for all }f^{:M\rightarrow N}\quad:\quad\quad\phi^{:L^{M}\rightarrow M}\bef f=f^{\uparrow L}\bef\phi^{:L^{N}\rightarrow N}\quad.
\]
Does this law hold with $\phi=$ \lstinline!reduceE!? Since \lstinline!reduceE!
would be used with type parameters $M$ and $N$, those types have
to be monoids for the law to make sense. Also, it turns out that the
law must be used only with functions $f$ that satisfy certain conditions
with respect to the monoids $M$ and $N$. 

To see why, let us set $L=$ \lstinline!List!. We expect that \lstinline!List!\textsf{'}s
\lstinline!reduceE! method will satisfy the naturality law as long
as we formulate that law correctly. We now apply \lstinline!reduceE!
to an empty list and to a list with two elements of type $M$:
\[
\text{reduceE}\left(\left[\right]\right)=e_{M}\quad,\quad\quad\text{reduceE}\left(\left[x,y\right]\right)=x\oplus_{M}y\quad.
\]
Similar values are found when \lstinline!reduceE! is used with the
monoid $N$. Then we have:
\begin{align*}
 & \left[\right]\triangleright\text{reduceE}\triangleright f=f(e_{M})\quad,\quad\quad\left[\right]\triangleright f^{\uparrow L}\triangleright\text{reduceE}=\left[\right]\triangleright\text{reduceE}=e_{N}\quad,\\
 & \left[x,y\right]\triangleright\text{reduceE}\triangleright f=(x\oplus_{M}y)\triangleright f=f(x\oplus_{M}y)\quad,\\
 & \left[x,y\right]\triangleright f^{\uparrow L}\triangleright\text{reduceE}=\left[f(x),f(y)\right]\triangleright\text{reduceE}=f(x)\oplus_{N}f(y)\quad.
\end{align*}
It follows that the law holds for \lstinline!reduceE! only if the
function $f$ has the properties: 
\[
f(e_{M})=e_{N}\quad,\quad\quad f(x\oplus_{M}y)=f(x)\oplus_{N}f(y)\quad.
\]
These properties mean that $f:M\rightarrow N$ preserves the operations
of the monoids $M$ and $N$, in the sense that $f$ maps the empty
value $e_{M}$ to the empty value $e_{N}$ and $M$\textsf{'}s binary operation
to $N$\textsf{'}s. Functions with these properties are called \textsf{``}monoid morphisms\textsf{''}.

\subsubsection{Definition \label{subsec:Definition-monoid-morphism}\ref{subsec:Definition-monoid-morphism}}

For any two monoids $M$ and $N$, a function $f:M\rightarrow N$
is a \textbf{monoid morphism}\index{monoid morphism|textit} if $f$
satisfies the following laws:
\begin{align*}
{\color{greenunder}\text{identity law}:}\quad & f(e_{M})=e_{N}\quad,\\
{\color{greenunder}\text{composition law}:}\quad & f(x^{:M}\oplus_{M}y^{:M})=f(x)\oplus_{N}f(y)\quad.
\end{align*}

The \textbf{monoidal naturality law} \index{monoidal naturality law}of
\lstinline!reduceE! is then formulated as:

\begin{wrapfigure}{l}{0.2\columnwidth}%
\vspace{-2.15\baselineskip}
\[
\xymatrix{\xyScaleY{1.4pc}\xyScaleX{3.5pc}L^{M}\ar[r]\sp(0.5){\ \text{reduceE}^{M}}\ar[d]\sp(0.45){\,f^{\uparrow L}} & M\ar[d]\sp(0.45){\,f}\\
L^{N}\ar[r]\sp(0.5){~\text{reduceE}^{N}} & N
}
\]
\vspace{-0.6\baselineskip}
\end{wrapfigure}%

~\vspace{-0.4\baselineskip}
\begin{equation}
\text{reduceE}^{M}\bef f^{:M\rightarrow N}=f^{\uparrow L}\bef\text{reduceE}^{N}\quad.\label{eq:monoidal-naturality-law-of-reduceE}
\end{equation}
Here the types $M$, $N$ are arbitrary monoids and $f:M\rightarrow N$
is an arbitrary monoid morphism between $M$ and $N$. 

The monoidal naturality law expresses a programmer\textsf{'}s expectation that
the code of \lstinline!reduceE[M]! should work in the same way for
every monoid \lstinline!M!. The code of \lstinline!reduceE! should
be fully parametric and may use the monoid \lstinline!M!\textsf{'}s methods
but may not, e.g., inspect the type parameter \lstinline!M! via run-time
reflection and make any decisions based on that. 

With these definitions, we can now give a precise formulation of the
following equivalences:

\subsubsection{Statement \label{subsec:Statement-foldleft-foldmap-equivalence}\ref{subsec:Statement-foldleft-foldmap-equivalence}}

\textbf{(a)} The functions \lstinline!foldMap! and \lstinline!reduceE!
are equivalent as long as \lstinline!foldMap! satisfies the naturality
law with respect to its type parameter $A$. In addition, Exercise~\ref{subsec:Exercise-traversables-laws-1-1}
will show that the monoidal naturality laws of \lstinline!foldMap!
and \lstinline!reduceE! are equivalent.

\textbf{(b)} The function \lstinline!foldLeft! is equivalent to a
new method we call \lstinline!foldFn! (\textsf{``}fold with function\textsf{''}):
\[
\text{foldFn}:L^{B\rightarrow B}\rightarrow B\rightarrow B\quad,
\]
as long as \lstinline!foldLeft! satisfies the naturality law with
respect to its type parameter $A$.

\textbf{(c)} The functions \lstinline!foldFn! and \lstinline!reduceE!
are equivalent as long as \lstinline!foldFn! obeys the laws~(\ref{eq:foldFn-first-special-law})
and~(\ref{eq:foldFn-second-special-law}) shown below and \lstinline!reduceE!
satisfies the monoidal naturality law. 

We will show in Statement~\ref{subsec:relational-property-for-foldFn}
below that the special laws~(\ref{eq:foldFn-first-special-law})
and~(\ref{eq:foldFn-second-special-law}) follow from parametricity.
So, these laws will hold automatically when the code of \lstinline!foldFn!
is fully parametric. However, formulating these special laws allows
us to prove the equivalence of \lstinline!foldFn! and \lstinline!reduceE!
without assuming parametricity.

\subparagraph{Proof}

\textbf{(a)} The equivalence of \lstinline!foldMap! and \lstinline!reduceE!
follows from Statement~\ref{subsec:Statement-tr-equivalent-to-ftr}
if we assume that \lstinline!foldMap! satisfies the naturality law
with respect to the type parameter $A$. To prove that equivalence,
we just need to set $F^{A}\triangleq L^{A}$, $G^{B}\triangleq M$,
and $H^{B}\triangleq M$ in Statement~\ref{subsec:Statement-tr-equivalent-to-ftr}.

\textbf{(b)} We first reformulate \lstinline!foldLeft!\textsf{'}s type signature
in a form more similar to that of \lstinline!foldMap!. We change
the order of curried arguments in \lstinline!foldLeft! and also replace
the updater function of type $B\times A\rightarrow B$ by an equivalent
curried function of type $A\rightarrow B\rightarrow B$ (similarly
to what was done in Section~\ref{subsec:From-reduce-and-foldleft-to-foldmap}).
The result is a function we denote by \lstinline!fldl!:
\begin{align*}
 & \text{fldl}:(A\rightarrow B\rightarrow B)\rightarrow L^{A}\rightarrow B\rightarrow B\quad,\quad\quad\text{fldl}\,(u^{:A\rightarrow B\rightarrow B})(p^{:L^{A}})(z^{:B})\triangleq\text{foldLeft}\,(p)(z)(\tilde{u})\quad,\\
 & \text{where we defined}:\quad\tilde{u}^{:B\times A\rightarrow B}\triangleq b^{:B}\times a^{:A}\rightarrow u\left(a\right)(b)\quad.
\end{align*}
The function \lstinline!fldl! is equivalent to \lstinline!foldLeft!
since we only replaced some types with equivalent ones. Setting $F^{A}\triangleq L^{A}$,
$G^{A}\triangleq B\rightarrow B$, and $H^{A}\triangleq B\rightarrow B$
in Statement~\ref{subsec:Statement-tr-equivalent-to-ftr}, we obtain
an equivalence between \lstinline!fldl! and \lstinline!foldFn!:
\[
\text{foldFn}:L^{B\rightarrow B}\rightarrow B\rightarrow B\quad,\quad\quad\text{foldFn}\,(p^{:L^{B\rightarrow B}})\triangleq\text{fldl}\,(\text{id}^{:(B\rightarrow B)\rightarrow B\rightarrow B})(p)\quad.
\]

\textbf{(c)} The type signatures of \lstinline!foldFn! and \lstinline!reduceE!
are similar except that \lstinline!foldFn! uses the type $B\rightarrow B$
where \lstinline!reduceE! uses an arbitrary monoid $M$ (having an
empty value $e_{M}$ and a binary operation $\oplus_{M}$). The type
of functions $B\rightarrow B$ is itself a monoid that we here denote
by $\text{MF}^{B}\triangleq B\rightarrow B$. The empty value of that
monoid is the identity function ($e_{_{\text{MF}^{B}}}=\text{id}^{:B\rightarrow B}$),
and the binary operation ($\ensuremath{\oplus}_{_{\text{MF}^{B}}}$)
is the \emph{forward} function composition: 
\[
p^{:B\rightarrow B}\oplus_{_{\text{MF}^{B}}}q^{:B\rightarrow B}\triangleq p\bef q\quad.
\]

To relate \lstinline!foldFn! and \lstinline!reduceE!, we need to
map an arbitrary monoid $M$ to the monoid $\text{MF}^{M}$:
\[
\text{inMF}:M\rightarrow\text{MF}^{M}\quad,\quad\quad\text{inMF}\left(m\right)\triangleq n^{:M}\rightarrow n\oplus_{M}m\quad.
\]
The code of \lstinline!inMF! is just the monoid $M$\textsf{'}s binary operation
($\oplus_{M}$) in a flipped and curried form:
\[
\text{inMF}:M\rightarrow M\rightarrow M\quad,\quad\quad\text{inMF}\triangleq m\rightarrow n\rightarrow n\oplus_{M}m\quad.
\]
In particular, for any value $m^{:M}$ we have, due to the monoid
$M$\textsf{'}s identity law: 
\begin{equation}
\text{inMF}\,(m)(e_{M})=e_{M}\oplus m=m\quad.\label{eq:identity-law-of-inMF}
\end{equation}

The function \lstinline!inMF! is a monoid morphism because the two
laws of Definition~\ref{subsec:Definition-monoid-morphism} hold:
\begin{align*}
 & \text{inMF}\,(e_{M})=n^{:M}\rightarrow n\oplus_{M}e_{M}=n^{:M}\rightarrow n=\text{id}^{:M\rightarrow M}=e_{_{\text{MF}^{M}}}\quad,\\
 & \text{inMF}\,(x\oplus_{M}y)=n^{:M}\rightarrow n\oplus_{M}x\oplus_{M}y\quad,\\
 & \text{inMF}\left(x\right)\oplus_{_{\text{MF}^{M}}}\text{inMF}\left(y\right)=\text{inMF}\left(x\right)\bef\text{inMF}\left(y\right)=(n\rightarrow n\oplus_{M}x)\bef(n\rightarrow n\oplus_{M}y)\\
 & \quad=n\rightarrow n\oplus_{M}x\oplus_{M}y=\text{inMF}\,(x\oplus_{M}y)\quad.
\end{align*}
So, we may use $f\triangleq\text{inMF}$ in the monoidal naturality
law~(\ref{eq:monoidal-naturality-law-of-reduceE}) with monoids $M$
and $N\triangleq\text{MF}^{M}$:
\begin{equation}
\text{reduceE}\bef\text{inMF}=\text{inMF}^{\uparrow L}\bef\text{reduceE}\quad.\label{eq:monoidal-naturality-law-of-reduceE-inMF}
\end{equation}

Now we can express \lstinline!foldFn! and \lstinline!reduceE! through
each other:
\begin{align}
 & \text{foldFn}^{B}=\text{reduceE}^{\text{MF}^{B}}\quad\quad\text{where}\quad\quad\text{MF}^{B}\triangleq B\rightarrow B\quad,\label{eq:foldFn-via-reduceE}\\
 & \text{reduceE}\,(p^{:L^{M}})=\text{foldFn}\,(p\triangleright\text{inMF}^{\uparrow L})(e_{M})\quad.\label{eq:reduceE-via-foldFn}
\end{align}

It remains to demonstrate an isomorphism between \lstinline!foldFn!
and \lstinline!reduceE! in both directions.

\textbf{1)} For a given function \lstinline!foldFn! with the type
signature shown above, we define \lstinline!reduceE! via Eq.~(\ref{eq:reduceE-via-foldFn})
and then define a new \lstinline!foldFn!$^{\prime}$ via Eq.~(\ref{eq:foldFn-via-reduceE}).
We need to show that \lstinline!foldFn!$^{\prime}=$ \lstinline!foldFn!:
\begin{align*}
 & \text{foldFn}^{\prime}(p^{:L^{B\rightarrow B}})=\text{reduceE}\,(p)=\text{foldFn}\,(p\triangleright\text{inMF}^{\uparrow L})(e_{M})\quad.
\end{align*}
Here the function \lstinline!inMF! is used with the type signature
corresponding to the monoid $M\triangleq B\rightarrow B$:
\begin{align*}
 & \text{inMF}:\left(B\rightarrow B\right)\rightarrow\text{MF}^{B\rightarrow B}\cong\left(B\rightarrow B\right)\rightarrow\left(B\rightarrow B\right)\rightarrow B\rightarrow B\quad,\\
 & \text{inMF}\triangleq g^{:B\rightarrow B}\rightarrow h^{:B\rightarrow B}\rightarrow h\bef g\quad,\quad\quad e_{\text{MF}}\triangleq\text{id}^{:B\rightarrow B}\quad.
\end{align*}

We find that $\text{foldFn}^{\prime}(p)=\text{foldFn}\,(p)$ if the
following equation holds:
\begin{equation}
\text{foldFn}^{B}(p^{:L^{B\rightarrow B}})=\text{foldFn}^{B\rightarrow B}(p\triangleright\text{inMF}^{\uparrow L})(\text{id}^{:B\rightarrow B})\quad.\label{eq:foldFn-first-special-law}
\end{equation}

It remains to prove that the law~(\ref{eq:monoidal-naturality-law-of-reduceE})
will hold when \lstinline!reduceE! is defined via Eq.~(\ref{eq:reduceE-via-foldFn}):
\begin{align*}
{\color{greenunder}\text{left-hand side}:}\quad & p^{:L^{M}}\triangleright\text{reduceE}^{M}\bef f^{:M\rightarrow N}=\big(\text{foldFn}\,(p\triangleright\text{inMF}^{\uparrow L})(e_{M})\big)\triangleright f\quad,\\
{\color{greenunder}\text{right-hand side}:}\quad & p^{:L^{M}}\triangleright f^{\uparrow L}\bef\text{reduceE}^{N}=\text{foldFn}\,(p\triangleright f^{\uparrow L}\triangleright\text{inMF}^{\uparrow L})(e_{N})\quad.
\end{align*}
The two sides are equal if the following law holds for any $p^{:L^{M}}$
and any monoid morphism $f^{:M\rightarrow N}$:
\begin{equation}
e_{M}\triangleright\big(p\triangleright\text{inMF}^{\uparrow L}\triangleright\text{foldFn}\big)\triangleright f=e_{N}\triangleright\big(p\triangleright f^{\uparrow L}\triangleright\text{inMF}^{\uparrow L}\triangleright\text{foldFn}\big)\quad.\label{eq:foldFn-second-special-law}
\end{equation}
Since both laws~(\ref{eq:foldFn-first-special-law}) and~(\ref{eq:foldFn-second-special-law})
are assumed to hold, we have demonstrated the equivalence between
\lstinline!reduceE! and \lstinline!foldFn! in one direction.

\textbf{2)} For a given function \lstinline!reduceE! with the type
signature shown above, we define \lstinline!foldFn! via Eq.~(\ref{eq:foldFn-via-reduceE})
and then define a new \lstinline!reduceE!$^{\prime}$ via Eq.~(\ref{eq:reduceE-via-foldFn}).
We need to show that \lstinline!reduceE!$^{\prime}=$ \lstinline!reduceE!:
\begin{align*}
{\color{greenunder}\text{expect to equal }\text{reduceE}\left(p\right):}\quad & \text{reduceE}^{\prime}(p^{:L^{M}})=\text{foldFn}\,(p\triangleright\text{inMF}^{\uparrow L})(e_{M})=\text{reduceE}\,(p\triangleright\text{inMF}^{\uparrow L})(e_{M})\\
{\color{greenunder}\text{rewrite using }\triangleright\text{-notation}:}\quad & =e_{M}\triangleright\big(p\triangleright\gunderline{\text{inMF}^{\uparrow L}\bef\text{reduceE}}\big)\\
{\color{greenunder}\text{use Eq.~(\ref{eq:monoidal-naturality-law-of-reduceE-inMF})}:}\quad & =e_{M}\triangleright\big(p\triangleright\text{reduceE}\bef\text{inMF}\big)=\gunderline{\text{inMF}}\,(p\triangleright\text{reduceE})\gunderline{(e_{M})}\\
{\color{greenunder}\text{use Eq.~(\ref{eq:identity-law-of-inMF})}:}\quad & =p\triangleright\text{reduceE}=\text{reduceE}\,(p)\quad.
\end{align*}

It remains to prove that \lstinline!foldFn! defined via Eq.~(\ref{eq:foldFn-via-reduceE})
will always satisfy the laws~(\ref{eq:foldFn-first-special-law})
and~(\ref{eq:foldFn-second-special-law}). To verify that the law~(\ref{eq:foldFn-first-special-law})
holds:
\begin{align*}
{\color{greenunder}\text{expect to equal }\text{foldFn}^{B}(p):}\quad & \text{foldFn}^{B\rightarrow B}(p^{:L^{B\rightarrow B}}\triangleright\text{inMF}^{\uparrow L})(e_{\text{MF}})=\text{reduceE}\,(p\triangleright\text{inMF}^{\uparrow L})(e_{\text{MF}})\\
 & =e_{\text{MF}}\triangleright(p\triangleright\gunderline{\text{inMF}^{\uparrow L}\bef\text{reduceE}})\\
{\color{greenunder}\text{use Eq.~(\ref{eq:monoidal-naturality-law-of-reduceE-inMF})}:}\quad & =e_{\text{MF}}\triangleright\big(p\triangleright\text{reduceE}\bef\text{inMF}\big)=\gunderline{\text{inMF}}\,(p\triangleright\text{reduceE})\gunderline{(e_{\text{MF}})}\\
{\color{greenunder}\text{use Eq.~(\ref{eq:identity-law-of-inMF})}:}\quad & =p\triangleright\text{reduceE}=\text{reduceE}^{B\rightarrow B}(p)=\text{foldFn}^{B}(p)\quad.
\end{align*}

To verify the law~(\ref{eq:foldFn-second-special-law}), write the
two sides separately and substitute Eq.~(\ref{eq:foldFn-via-reduceE}):
\begin{align*}
{\color{greenunder}\text{left-hand side}:}\quad & e_{M}\triangleright\big(p\triangleright\gunderline{\text{inMF}^{\uparrow L}\triangleright\text{foldFn}}\big)\triangleright f=e_{M}\triangleright(p\triangleright\gunderline{\text{inMF}^{\uparrow L}\bef\text{reduceE}}\big)\triangleright f\\
{\color{greenunder}\text{use Eq.~(\ref{eq:monoidal-naturality-law-of-reduceE-inMF})}:}\quad & \quad=e_{M}\triangleright\big(p\triangleright\text{reduceE}\bef\text{inMF}\big)\triangleright f=\big(\gunderline{\text{inMF}}\,(p\triangleright\text{reduceE})\gunderline{(e_{M})}\big)\triangleright f\\
{\color{greenunder}\text{use Eq.~(\ref{eq:identity-law-of-inMF})}:}\quad & \quad=p\triangleright\text{reduceE}\triangleright f\quad,\\
{\color{greenunder}\text{right-hand side}:}\quad & e_{N}\triangleright\big(p\triangleright\gunderline{f^{\uparrow L}\triangleright\text{inMF}^{\uparrow L}\triangleright\text{foldFn}}\big)=e_{N}\triangleright\big(p\triangleright f^{\uparrow L}\bef\gunderline{\text{inMF}^{\uparrow L}\bef\text{reduceE}}\big)\\
{\color{greenunder}\text{use Eq.~(\ref{eq:monoidal-naturality-law-of-reduceE-inMF})}:}\quad & \quad=e_{N}\triangleright\big(p\triangleright\gunderline{f^{\uparrow L}\bef\text{reduceE}}\bef\text{inMF}\big)\\
{\color{greenunder}\text{use Eq.~(\ref{eq:monoidal-naturality-law-of-reduceE})}:}\quad & \quad=e_{N}\triangleright\big(p\triangleright\text{reduceE}\bef f\bef\text{inMF}\big)=\gunderline{\text{inMF}}\,\big(p\triangleright\text{reduceE}\bef f\big)\gunderline{(e_{N})}\\
{\color{greenunder}\text{use Eq.~(\ref{eq:identity-law-of-inMF})}:}\quad & \quad=p\triangleright\text{reduceE}\triangleright f\quad.
\end{align*}
The two sides are now equal.

This proves that the equivalence between \lstinline!reduceE! and
\lstinline!foldFn! holds also in the other direction. $\square$

Let us comment on the special laws~(\ref{eq:foldFn-first-special-law})
and~(\ref{eq:foldFn-second-special-law}). The next statement will
show that those laws follow from parametricity. If \lstinline!foldFn(p)!
uses any functions of type $B\rightarrow B$ contained in $p:L^{B\rightarrow B}$
in a fully parametric manner, the code of \lstinline!foldFn! cannot
inspect the type $B$ or the code of the functions of type $B\rightarrow B$.
The only way of using those functions is to compose them with each
other in some order, obtaining again a function of type $B\rightarrow B$.
The laws~(\ref{eq:foldFn-first-special-law}) and~(\ref{eq:foldFn-second-special-law})
express this property in different ways by writing the function composition
explicitly as part of the code of the function \lstinline!inMF!.

\subsubsection{Statement \label{subsec:relational-property-for-foldFn}\ref{subsec:relational-property-for-foldFn}}

Denote for brevity $E^{A}\triangleq A\rightarrow A$. Let $L$ be
any \emph{polynomial} functor. (This restriction is acceptable because,
as we will see below, only polynomial functors are traversable.)

\textbf{(a)} Any fully parametric function $\phi:\forall A.\,L^{E^{A}}\rightarrow E^{A}$
satisfies the law~(\ref{eq:foldFn-first-special-law}):
\begin{align}
 & \phi\,(p^{:L^{E^{A}}})=\phi\,(p\triangleright\text{inF}^{\uparrow L})(\text{id}^{:A\rightarrow A})\quad,\label{eq:first-special-law-of-phi}\\
{\color{greenunder}\text{where we defined}:}\quad & \quad\text{inF}:E^{A}\rightarrow E^{A}\rightarrow E^{A}\quad,\quad\quad\text{inF}\triangleq h^{:A\rightarrow A}\rightarrow k^{:A\rightarrow A}\rightarrow k\bef h\quad.\nonumber 
\end{align}

\textbf{(b)} Any fully parametric function $\text{foldFn}:\forall A.\,L^{E^{A}}\rightarrow E^{A}$
satisfies Eq.~(\ref{eq:foldFn-second-special-law}).

\subparagraph{Proof}

\textbf{(a)} Example~\ref{subsec:Example-strong-dinaturality-proof-of-foldFn-law}
(in Appendix~\ref{app:Proofs-of-naturality-parametricity}) shows
that $\phi$ satisfies the strong dinaturality law. The law says that
\begin{comment}
Begin by formulating the relational naturality law of $\phi$: for
all $r^{:A\leftrightarrow B}$,
\[
\text{if }(x^{:L^{A\rightarrow A}},y^{:L^{B\rightarrow B}})\in r^{\updownarrow(L\circ E)}\text{ then }(\phi(x),\phi(y))\in r^{\updownarrow E}\quad.
\]
In order to use this law to derive Eq.~(\ref{eq:first-special-law-of-phi}),
we need to choose a specific relation $r$ and specific types $A$
and $B$. Note that the value $p\triangleright\text{inMF}^{\uparrow L}$
in Eq.~(\ref{eq:first-special-law-of-phi}) has type $L^{E^{A}\rightarrow E^{A}}$.
So, the only hope of using Eq.~(\ref{eq:relational-law-of-g}) is
by choosing the type $B$ as $B\triangleq E^{A}=A\rightarrow A$. 

It remains to choose a suitable relation $r$ between the types $A$
and $B\triangleq A\rightarrow A$. In Eq.~(\ref{eq:relational-law-of-g}),
the relation $r$ is being lifted to the functor $L\circ E$, which
is equivalent to the lifting $(r^{\updownarrow E})^{\updownarrow L}$.
Since $L$ is an unknown functor, we do not have a general formula
for lifting an arbitrary relation $r$ to $L\circ E$. However, we
do have a formula for lifting a relation $r^{:A\leftrightarrow B}$
that comes from a function (either of type $A\rightarrow B$ or of
type $B\rightarrow A$). For instance, given a function $f:B\rightarrow A$,
we define $r\triangleq\text{rev}\left<f\right>$. First we lift $\left<f\right>^{\updownarrow E}$
as:
\[
(g^{:B\rightarrow B},h^{:A\rightarrow A})\in\left<f\right>^{\updownarrow E}\text{ means }g\bef f=f\bef h\quad.
\]
This is a relation of pullback form:
\[
\left<f\right>^{\updownarrow E}=\text{pull}\,\big(g^{:B\rightarrow B}\rightarrow g\bef f,\quad h^{:A\rightarrow A}\rightarrow f\bef h\big)\quad.
\]
A pullback relation can then be lifted to a functor $L$ as shown
in Statement~\ref{subsec:Statement-pullback-as-composition}:
\[
\left<f\right>^{\updownarrow E\updownarrow L}=\text{pull}\,\big((g^{:B\rightarrow B}\rightarrow g\bef f)^{\uparrow L},\quad(h^{:A\rightarrow A}\rightarrow f\bef h)^{\uparrow L}\big)\quad.
\]
We can now rewrite Eq.~ in the form of a chain of pullback relations: 
\end{comment}
for any $f^{:B\rightarrow A}$, $x^{:L^{A\rightarrow A}}$, and $y^{:L^{B\rightarrow B}}$:
\begin{equation}
\text{if }x\triangleright(h^{:A\rightarrow A}\rightarrow f\bef h)^{\uparrow L}=y\triangleright(g^{:B\rightarrow B}\rightarrow g\bef f)^{\uparrow L}\quad\text{then}\quad f\bef\phi(x)=\phi(y)\bef f\quad.\label{eq:strong-dinaturality-law-of-phi-with-f}
\end{equation}
We need to choose $A$, $B$, and $f$ appropriately in order to be
able to derive Eq.~(\ref{eq:first-special-law-of-phi}). After some
guessing and trying, one finds that a good choice is $B\triangleq A\rightarrow A$
and $f^{:B\rightarrow A}$ defined by:
\begin{equation}
f:(A\rightarrow A)\rightarrow A\quad,\quad\quad f\triangleq k^{:A\rightarrow A}\rightarrow k(a_{0})\quad\text{with a fixed }a_{0}:A\quad.\label{eq:f-for-relational-law-of-foldFn-derivation1}
\end{equation}
Each value $a_{0}$ of type $A$ gives us a function $f_{a_{0}}:(A\rightarrow A)\rightarrow A$
for which the law~(\ref{eq:strong-dinaturality-law-of-phi-with-f})
holds. 

We will now show that Eq.~(\ref{eq:first-special-law-of-phi}) follows
from Eq.~(\ref{eq:strong-dinaturality-law-of-phi-with-f}) with this
choice of $f$. The value $a_{0}$ will be held fixed throughout most
of this proof, so we will write just $f$ instead of $f_{a_{0}}$.

In order to use Eq.~(\ref{eq:strong-dinaturality-law-of-phi-with-f})
with this choice of $f$, we need to find suitable values $x$ and
$y$. Note that the conclusion of Eq.~(\ref{eq:strong-dinaturality-law-of-phi-with-f})
relates $\phi(x)$ with $\phi(y)$, while Eq.~(\ref{eq:first-special-law-of-phi})
requires us to relate $\phi(p)$ with $\phi(p\triangleright\text{inF}^{\uparrow L})$.
So, we must choose $x=p$ and $y=p\triangleright\text{inF}^{\uparrow L}$.
This choice would work with Eq.~(\ref{eq:strong-dinaturality-law-of-phi-with-f})
only if its precondition is satisfied with these $x$ and $y$ and
with $f$ defined by Eq.~(\ref{eq:f-for-relational-law-of-foldFn-derivation1}):
\begin{align*}
 & p\triangleright(h^{:A\rightarrow A}\rightarrow f\bef h)^{\uparrow L}\overset{?}{=}y\triangleright(g^{:B\rightarrow B}\rightarrow g\bef f)^{\uparrow L}=p\triangleright\text{inF}^{\uparrow L}\triangleright(g^{:B\rightarrow B}\rightarrow g\bef f)^{\uparrow L}\\
 & \quad=p\triangleright\big(\text{inF}\bef(g^{:B\rightarrow B}\rightarrow g\bef f)\big)^{\uparrow L}\quad.
\end{align*}
We will show that this equation holds for any $p:L^{A\rightarrow A}$
if we show that:
\[
h^{:A\rightarrow A}\rightarrow f\bef h\overset{?}{=}\text{inF}\bef(g^{:B\rightarrow B}\rightarrow g\bef f)\quad.
\]
We simplify separately each side of this equation and find that they
are equal:
\begin{align*}
{\color{greenunder}\text{left-hand side}:}\quad & h^{:A\rightarrow A}\rightarrow f\bef h=h\rightarrow(k\rightarrow\gunderline{k(a_{0}))\bef h}=h\rightarrow k\rightarrow h(k(a_{0}))\quad,\\
{\color{greenunder}\text{right-hand side}:}\quad & \gunderline{\text{inF}}\bef(g^{:B\rightarrow B}\rightarrow g\bef\gunderline f)=(h\rightarrow k\rightarrow k\bef h)\bef(g\rightarrow g\bef(k\rightarrow k(a_{0}))\\
{\color{greenunder}\text{compute composition}:}\quad & \quad=h\rightarrow(k\rightarrow k\bef h)\bef(k\rightarrow k(a_{0}))=h\rightarrow k\rightarrow\gunderline{(k\bef h)(a_{0})}\\
 & \quad=h\rightarrow k\rightarrow h(k(a_{0})\quad.
\end{align*}

Since the precondition of Eq.~(\ref{eq:strong-dinaturality-law-of-phi-with-f})
is satisfied, its conclusion holds too:
\begin{align*}
 & f\bef\phi(x)\overset{!}{=}\phi(y)\bef f\quad,\\
{\color{greenunder}\text{equivalently}:}\quad & (k^{:A\rightarrow A}\rightarrow k(a_{0}))\bef\phi(p)\overset{!}{=}\phi(p\triangleright\text{inF}^{\uparrow L})\bef(k^{:A\rightarrow A}\rightarrow k(a_{0}))\quad,\\
{\color{greenunder}\text{equivalently}:}\quad & k^{:A\rightarrow A}\rightarrow\phi(p)(k(a_{0}))\overset{!}{=}k^{:A\rightarrow A}\rightarrow\phi(p\triangleright\text{inF}^{\uparrow L})(k)(a_{0})\quad.
\end{align*}
The last equation can be applied to any function $k$ of type $A\rightarrow A$:\vspace{-0.3\baselineskip}
\[
\phi(p)(k(a_{0}))\overset{!}{=}\phi(p\triangleright\text{inF}^{\uparrow L})(k)(a_{0})\quad.
\]
In Eq.~(\ref{eq:first-special-law-of-phi}), the function $\phi(p\triangleright\text{inF}^{\uparrow L})$
is applied to the identity function of type $A\rightarrow A$. So,
we set $k\triangleq\text{id}$ in the line above:\vspace{-0.3\baselineskip}
\[
\phi(p)(a_{0})\overset{!}{=}\phi(p\triangleright\text{inF}^{\uparrow L})(\text{id})(a_{0})\quad.
\]
Since this holds for any $a_{0}$, we get $\phi(p)\overset{!}{=}\phi(p\triangleright\text{inF}^{\uparrow L})(\text{id})$,
which proves Eq.~(\ref{eq:first-special-law-of-phi}).

\textbf{(b)} We already showed in part \textbf{(a)} that \lstinline!foldFn!
$=\phi$ satisfies Eq.~(\ref{eq:strong-dinaturality-law-of-phi-with-f}).
We will now show that Eq.~(\ref{eq:foldFn-second-special-law}) follows
if we apply Eq.~(\ref{eq:strong-dinaturality-law-of-phi-with-f})
with $A$, $B$, $x$, and $y$ chosen as:
\[
A\triangleq N\quad,\quad\quad B\triangleq M\quad,\quad\quad x^{:L^{N\rightarrow N}}\triangleq p\triangleright f^{\uparrow L}\triangleright\text{inMF}^{\uparrow L}\quad,\quad\quad y^{:L^{M\rightarrow M}}\triangleq p\triangleright\text{inMF}^{\uparrow L}\quad.
\]
Before we can use Eq.~(\ref{eq:strong-dinaturality-law-of-phi-with-f}),
we need to verify that its precondition is satisfied:
\[
x\triangleright(h\rightarrow f\bef h)^{\uparrow L}=p\triangleright\big(f\bef\text{inMF}\bef(h\rightarrow f\bef h)\big)^{\uparrow L}\overset{?}{=}y\triangleright(g\rightarrow g\bef f)^{\uparrow L}=p\triangleright\big(\text{inMF}\bef(g\rightarrow g\bef f)\big)^{\uparrow L}\quad.
\]
To show that this equation holds, we compare separately the lifted
functions that are applied to $p$:
\begin{align*}
 & f\bef\text{inMF}\bef(h\rightarrow f\bef h)=\big(m^{:M}\rightarrow n^{:N}\rightarrow n\oplus_{N}f(m)\big)\bef(h\rightarrow f\bef h)=m^{:M}\rightarrow l^{:M}\rightarrow f(l)\oplus_{N}f(m)\quad,\\
 & \text{inMF}\bef(g\rightarrow g\bef f)=(m^{:M}\rightarrow l^{:M}\rightarrow l\oplus_{M}m)\bef(g\rightarrow g\bef f)=m^{:M}\rightarrow l^{:M}\rightarrow f(l\oplus_{M}m)\quad.
\end{align*}
The two functions are equal because of the composition law of the
monoid morphism $f$. So, the precondition of Eq.~(\ref{eq:strong-dinaturality-law-of-phi-with-f})
holds, and we may use its conclusion:
\[
f\bef\text{foldFn}\,(x)\overset{!}{=}\text{foldFn}(y)\bef f\quad\text{or equivalently}:\quad\text{foldFn}\,(p\triangleright\text{inMF}^{\uparrow L})\bef f=f\bef\text{foldFn}(p\triangleright f^{\uparrow L}\bef\text{inMF}^{\uparrow L}).
\]
This gives the law~(\ref{eq:foldFn-second-special-law}) if we use
$f$\textsf{'}s monoid morphism identity law: $e_{M}\triangleright f=e_{N}$.

\subsubsection{Statement \label{subsec:Statement-reduceE-toList-equivalence}\ref{subsec:Statement-reduceE-toList-equivalence}}

Functions \lstinline!reduceE! and \lstinline!toList! are equivalent
if expressed through each other as:
\begin{align}
 & \text{toList}:L^{A}\rightarrow\text{List}^{A}\quad,\quad\quad\text{toList}=\text{pu}_{\text{List}}^{\uparrow L}\bef\text{reduceE}^{\text{List}^{A}}\quad,\label{eq:toList-via-reduceE}\\
 & \text{reduceE}:L^{M}\rightarrow M\quad,\quad\quad\text{reduceE}=\text{toList}\bef\text{reduceList}\quad,\label{eq:reduceE-via-toList}
\end{align}
assuming the naturality law of \lstinline!toList! and the monoidal
naturality law~(\ref{eq:monoidal-naturality-law-of-reduceE}) of
\lstinline!reduceE!. Here the function $\text{reduceE}^{\text{List}^{A}}$
is \lstinline!reduceE! applied to the monoidal type \lstinline!List[A]!
whose binary operation is the list concatenation. The monoidally natural
function \lstinline!reduceList[M]! is defined for monoids $M$ by:
\begin{lstlisting}
def reduceList[M: Monoid]: List[M] => M = {
  case Nil            => Monoid[M].empty
  case head :: tail   => head |+| reduceList(tail)
}
\end{lstlisting}
\[
\text{reduceList}^{M}:\text{List}^{M}\rightarrow M\quad,\quad\quad\text{reduceList}\triangleq\,\begin{array}{|c||c|}
 & M\\
\hline \bbnum 1 & \_\rightarrow e_{M}\\
M\times\text{List}^{M} & h^{:M}\times t^{:\text{List}^{M}}\rightarrow h\oplus_{M}\overline{\text{reduceList}}\left(t\right)
\end{array}\quad.
\]


\subparagraph{Proof}

We begin by showing that \lstinline!reduceList! obeys the monoidal
naturality law:
\[
f^{\uparrow\text{List}}\bef\text{reduceList}=\text{reduceList}\bef f\quad.
\]
This law assumes that $f^{:M\rightarrow N}$ is a monoid morphism
between two monoids $M$ and $N$. 

Simplify the left-hand side of the law:
\begin{align*}
 & f^{\uparrow\text{List}}\bef\text{reduceList}=\,\begin{array}{|c||cc|}
 & \bbnum 1 & N\times\text{List}^{N}\\
\hline \bbnum 1 & \text{id} & \bbnum 0\\
M\times\text{List}^{M} & \bbnum 0 & f\boxtimes f^{\uparrow\text{List}}
\end{array}\,\bef\,\begin{array}{|c||c|}
 & N\\
\hline \bbnum 1 & \_\rightarrow e_{N}\\
N\times\text{List}^{N} & h\times t\rightarrow h\oplus_{N}(t\triangleright\overline{\text{reduceList}})
\end{array}\\
 & =\,\begin{array}{|c||c|}
 & N\\
\hline \bbnum 1 & \_\rightarrow e_{N}\\
M\times\text{List}^{M} & h\times t\rightarrow f(h)\oplus_{N}(t\triangleright f^{\uparrow\text{List}}\bef\overline{\text{reduceList}})
\end{array}\quad.
\end{align*}
We can use the inductive assumption that recursive calls to \lstinline!reduceList!
already obey the monoidal naturality law. So, we can simplify:
\[
t\triangleright f^{\uparrow\text{List}}\bef\overline{\text{reduceList}}=t\triangleright\overline{\text{reduceList}}\bef f=f(\overline{\text{reduceList}}\left(t\right))\quad.
\]
The bottom-row expression in the last matrix is then rewritten to:
\begin{align*}
 & f(h)\oplus_{N}(t\triangleright f^{\uparrow\text{List}}\bef\overline{\text{reduceList}})=f(h)\oplus_{N}f(\overline{\text{reduceList}}\left(t\right))\\
{\color{greenunder}\text{monad morphism law}:}\quad & =f\big(h\oplus_{M}\overline{\text{reduceList}}\left(t\right)\big)\quad.
\end{align*}
Using the monad morphism identity law ($f(e_{M})=e_{N}$), we find:
\begin{align*}
 & f^{\uparrow\text{List}}\bef\text{reduceList}=\,\begin{array}{|c||c|}
 & N\\
\hline \bbnum 1 & \_\rightarrow f(e_{M})\\
M\times\text{List}^{M} & h\times t\rightarrow f\big(h\oplus_{M}\overline{\text{reduceList}}\left(t\right)\big)
\end{array}\\
 & =\,\begin{array}{|c||c|}
 & N\\
\hline \bbnum 1 & \_\rightarrow e_{M}\\
M\times\text{List}^{M} & h\times t\rightarrow h\oplus_{M}\overline{\text{reduceList}}\left(t\right)
\end{array}\bef f=\text{reduceList}\bef f\quad.
\end{align*}
 This is equal to the right-hand side of the law.

We are now ready to show the isomorphism between \lstinline!reduceE!
and \lstinline!toList! in both directions:

\textbf{(1)} Given a function \lstinline!reduceE! that satisfies
Eq.~(\ref{eq:monoidal-naturality-law-of-reduceE}), define \lstinline!toList!
via Eq.~(\ref{eq:toList-via-reduceE}) and then define a new function
\lstinline!reduceE!$^{\prime}$ via Eq.~(\ref{eq:reduceE-via-toList}).
To show that \lstinline!reduceE!$^{\prime}=$ \lstinline!reduceE!,
we write:
\begin{align*}
 & \text{reduceE}^{\prime}=\text{toList}\bef\text{reduceList}=\gunderline{\text{pu}_{\text{List}}^{\uparrow L}\bef\text{reduceE}^{\text{List}^{M}}}\bef\text{reduceList}\\
{\color{greenunder}\text{Eq.~(\ref{eq:monoidal-naturality-law-of-reduceE}) with }N\triangleq\text{List}^{A}:}\quad & =\text{reduceE}^{M}\bef\text{pu}_{\text{List}}\bef\text{reduceList}\quad.
\end{align*}
It remains to show that the composition $\text{pu}_{\text{List}}\bef\text{reduceList}$
is equal to the identity function:
\[
\text{pu}_{\text{List}}\bef\text{reduceList}=\text{id}^{:M\rightarrow M}\quad.
\]
We use the definition of \lstinline!reduceList[M]! and simplify the
matrix composition:
\begin{align*}
 & \text{pu}_{\text{List}}\bef\text{reduceList}=(m^{:M}\rightarrow\bbnum 0+m\times\left[\right])\bef\text{reduceList}\\
 & =\,\begin{array}{|c||cc|}
 & \bbnum 1 & M\times\text{List}^{M}\\
\hline M & \bbnum 0 & m\rightarrow m\times\left[\right]
\end{array}\,\bef\,\begin{array}{|c||c|}
 & M\\
\hline \bbnum 1 & \_\rightarrow e_{M}\\
M\times\text{List}^{M} & h\times t\rightarrow h\oplus_{M}\overline{\text{reduceList}}\left(t\right)
\end{array}\\
 & =\,\,\begin{array}{|c||c|}
 & M\\
\hline M & m\rightarrow m\oplus_{M}\gunderline{\overline{\text{reduceList}}\left(\left[\right]\right)}
\end{array}\,=m\rightarrow\gunderline{m\oplus e_{M}}=m\rightarrow m=\text{id}\quad.
\end{align*}
In the last line, we used the fact that \lstinline!reduceList[M]!
applied to an empty list gives $e_{M}$.

\textbf{(2)} Given a function \lstinline!toList!, define \lstinline!reduceE!
via Eq.~(\ref{eq:reduceE-via-toList}) and then define a new function
\lstinline!toList!$^{\prime}$ via Eq.~(\ref{eq:toList-via-reduceE}).
To show that \lstinline!toList!$^{\prime}=$ \lstinline!toList!,
we write:
\begin{align*}
 & \text{toList}^{\prime}=\text{pu}_{\text{List}}^{\uparrow L}\bef\text{reduceE}^{\text{List}^{A}}=\gunderline{\text{pu}_{\text{List}}^{\uparrow L}\bef\text{toList}}\bef\text{reduceList}^{\text{List}^{A}}\\
{\color{greenunder}\text{naturality of }\text{toList}:}\quad & =\text{toList}\bef\text{pu}_{\text{List}}^{\uparrow\text{List}}\bef\text{reduceList}^{\text{List}^{A}}\quad.
\end{align*}
It remains to show that the composition $\text{pu}_{\text{List}}^{\uparrow\text{List}}\bef\text{reduceList}$
is equal to the identity function:
\begin{equation}
\text{pu}_{\text{List}}^{\uparrow\text{List}}\bef\text{reduceList}=\text{id}^{:\text{List}^{M}\rightarrow\text{List}^{M}}\quad.\label{eq:identity-for-reduceList-and-pure}
\end{equation}
We use the definition of \lstinline!reduceList[M]! and set \lstinline!M!
to the monoid type \lstinline!List[A]!:
\begin{align*}
 & \text{pu}_{\text{List}}^{\uparrow\text{List}}\bef\text{reduceList}^{\text{List}^{A}}\\
 & =\,\begin{array}{|c||cc|}
 & \bbnum 1 & \text{List}^{M}\times\text{List}^{\text{List}^{M}}\\
\hline \bbnum 1 & \text{id} & \bbnum 0\\
M\times\text{List}^{M} & \bbnum 0 & h\times t\rightarrow\text{pu}_{\text{List}}(h)\times(t\triangleright\text{pu}_{\text{List}}^{\uparrow\text{List}})
\end{array}\,\bef\,\begin{array}{|c||c|}
 & \text{List}^{M}\\
\hline \bbnum 1 & \_\rightarrow1+\bbnum 0\\
\text{List}^{M}\times\text{List}^{\text{List}^{M}} & h\times t\rightarrow h\pplus\overline{\text{reduceList}}\left(t\right)
\end{array}\\
 & =\,\,\begin{array}{|c||c|}
 & \text{List}^{M}\\
\hline \bbnum 1 & \_\rightarrow1+\bbnum 0\\
M\times\text{List}^{M} & h\times t\rightarrow\text{pu}_{\text{List}}(h)\pplus(t\triangleright\text{pu}_{\text{List}}^{\uparrow\text{List}}\triangleright\overline{\text{reduceList}})
\end{array}\quad.
\end{align*}
By the inductive assumption, a recursive call to \lstinline!reduceList!
already satisfies Eq.~(\ref{eq:identity-for-reduceList-and-pure}):
\[
t\triangleright\gunderline{\text{pu}_{\text{List}}^{\uparrow\text{List}}\triangleright\overline{\text{reduceList}}}=t\triangleright\text{id}=t\quad.
\]
We can now simplify the code and obtain the identity function:
\begin{align*}
 & \text{pu}_{\text{List}}^{\uparrow\text{List}}\bef\text{reduceList}^{\text{List}^{A}}=\,\begin{array}{|c||c|}
 & \text{List}^{M}\\
\hline \bbnum 1 & \_\rightarrow1+\bbnum 0\\
M\times\text{List}^{M} & h^{:M}\times t^{:\text{List}^{M}}\rightarrow\text{pu}_{\text{List}}(h)\pplus t
\end{array}\\
 & =\,\begin{array}{|c||c|}
 & \text{List}^{M}\\
\hline \bbnum 1 & \_\rightarrow1+\bbnum 0\\
M\times\text{List}^{M} & h^{:M}\times t^{:\text{List}^{M}}\rightarrow\bbnum 0+h\times t
\end{array}\,=\,\begin{array}{|c||cc|}
 & \bbnum 1 & M\times\text{List}^{M}\\
\hline \bbnum 1 & \text{id} & \bbnum 0\\
M\times\text{List}^{M} & \bbnum 0 & h^{:M}\times t^{:\text{List}^{M}}\rightarrow h\times t
\end{array}\\
 & =\text{id}\quad.
\end{align*}

It remains to prove that \lstinline!reduceE! will satisfy the monoidal
naturality law~(\ref{eq:monoidal-naturality-law-of-reduceE}) when
defined via Eq.~(\ref{eq:reduceE-via-toList}) through any \lstinline!toList!
function. For any monoid morphism $f:M\rightarrow N$ between arbitrary
monoids $M$ and $N$, we write the two sides of the monoidal naturality
law~(\ref{eq:monoidal-naturality-law-of-reduceE}):
\begin{align*}
{\color{greenunder}\text{left-hand side}:}\quad & \text{reduceE}\bef f=\text{toList}\bef\gunderline{\text{reduceList}\bef f}\\
{\color{greenunder}\text{monoidal naturality of }\text{reduceList}:}\quad & \quad=\gunderline{\text{toList}\bef f^{\uparrow\text{List}}}\bef\text{reduceList}\\
{\color{greenunder}\text{naturality of }\text{toList}:}\quad & \quad=f^{\uparrow L}\bef\text{toList}\bef\text{reduceList}\quad,\\
{\color{greenunder}\text{right-hand side}:}\quad & f^{\uparrow L}\bef\text{reduceE}=f^{\uparrow L}\bef\text{toList}\bef\text{reduceList}\quad.
\end{align*}
The two sides are now equal. $\square$

\subsection{The missing laws of \texttt{foldMap} and \texttt{reduce}}

The equivalence of \lstinline!foldLeft!, \lstinline!foldMap!, \lstinline!reduce!,
and \lstinline!toList! is ensured by assuming various naturality
laws. However, those naturality laws do not fully describe the programmer\textsf{'}s
expectations about the behavior of folding operations. 

To see why, let us consider \lstinline!toList! with the type signature
$L^{A}\rightarrow\text{List}^{A}$. If \lstinline!toList! is fully
parametric, it will satisfy the naturality law:
\[
(f^{:A\rightarrow B})^{\uparrow L}\bef\text{toList }=\text{toList}\bef f^{\uparrow\text{List}}\quad.
\]
This law describes the property that \lstinline!toList[A]! works
in the same way for all types \lstinline!A!. Certainly, programmers
expect this property to hold. But the main intent of \lstinline!toList!
is to extract values of type $A$ out of $L^{A}$ and store them in
a list. Naturality laws do not express this intent.

More precisely, programmers expect that for any \textsf{``}container\textsf{''} value
\lstinline!p! of type \lstinline!L[A]!, the value \lstinline!toList(p)!
should be a list of \emph{all} values of type \lstinline!A! stored
in \lstinline!p!. As an example of unexpected behavior, imagine implementing
the type signature of \lstinline!toList! by a function that always
returns an empty list. For \textsf{``}containers\textsf{''} that store some values
of type \lstinline!A!, this implementation (although it is fully
parametric) would be unacceptable since it loses information: \lstinline!toList(x)!
always ignores its argument \lstinline!x!.

To prohibit such implementations, we would like to impose a law that
holds only when \lstinline!toList! extracts every value of type \lstinline!A!
stored in \lstinline!p!. Unfortunately, it seems to be impossible
to express this property in the form of an equation satisfied by \lstinline!toList!.
Often, a loss of information is prevented by imposing an identity
law. Can we formulate an identity law for \lstinline!toList!? Such
a law could state that \lstinline!toList(p)! should extract some
known values contained in \lstinline!p!. If $L$ were a pointed functor\index{pointed functor}
(see Section~\ref{subsec:Pointed-functors-motivation-equivalence}),
we could use its \lstinline!pure! method to inject a known value
\lstinline!x! into the container \lstinline!p! and then require
\lstinline!toList! to extract the same value \lstinline!x!. But
the definition of \lstinline!toList! does not require the functor
$L$ to be pointed. So, in general we cannot inject values into $L^{A}$
in a way that is guaranteed to preserve information. This prevents
us from formulating an identity law for \lstinline!toList!.

Another approach to finding laws is to look for type signatures in
the form of a \textsf{``}lifting\textsf{''} that transforms a function of one type
into a function of another type. We have summarized the methods of
several standard typeclasses as \textsf{``}liftings\textsf{''} in Section~\ref{subsec:The-pattern-of-functorial-typeclasses}.
The laws of a \textsf{``}lifting\textsf{''} are the functor laws (identity and composition).
Could we apply this approach to the folding operations? The type signature
of the \lstinline!foldMap! function is:\index{foldMap function@\texttt{foldMap} function}
\[
\text{foldMap}:\left(A\rightarrow M\right)\rightarrow L^{A}\rightarrow M\quad.
\]
This resembles a \textsf{``}lifting\textsf{''} from functions of type $A\rightarrow M$
to functions of type $L^{A}\rightarrow M$. However, it is not possible
to impose the functor laws on those liftings. For instance, the functor
composition law involves applying \lstinline!foldMap! to a composition
of arguments. But the type signature $A\rightarrow M$ (where $M$
is a fixed type) does not support composition since we cannot compose
$A\rightarrow M$ with $B\rightarrow M$.

So, the \textsf{``}lifting\textsf{''} approach also fails to yield a suitable law
for folding operations. However, we will see below that lifting-like
laws may be imposed on the \lstinline!traverse! operation, whose
type signature is a generalization of that of \lstinline!foldMap!.
We will show that the laws of \lstinline!traverse! forbid information-losing
implementations. Since \lstinline!foldMap! can be derived from \lstinline!traverse!,
we may take the position that the only acceptable implementations
of \lstinline!foldMap! are those derived from a lawful \lstinline!traverse!
function.

\subsection{All polynomial functors are foldable}

It turns out that the lack of available laws does not prevent us from
finding correct implementations of folding operations. The reason
is that folding operations are available only for polynomial functors,
such as \lstinline!Option[A]! and \lstinline!List[A]! (note that
\lstinline!List! is a recursively defined polynomial functor\index{polynomial functor!recursive}).
These functors represent containers that store a finite number of
values of type \lstinline!A!. So, it is clear what it means to extract
\textsf{``}all values of type $A$\textsf{''} from such containers. To show that all
polynomial functors are foldable, we will define \lstinline!toList!
inductively via structural analysis. The definition will ensure that
\lstinline!toList! extracts exactly as many values as stored in the
polynomial functor.

Let us first show that non-polynomial functors are \emph{not} foldable.
An example of a non-polynomial functor is the \lstinline!Reader!
monad:
\[
\text{Reader}^{R,A}\triangleq R\rightarrow A\quad,
\]
where $R$ is a fixed but arbitrarily chosen type. A value $p$ of
type $R\rightarrow A$ can be viewed as a container that stores one
value of type $A$ for each value of type $R$. We can extract all
values of type $A$ from $p$ only if we can enumerate all possible
values of type $R$. In general, it is not practical to enumerate
all values of a given type (as an example, consider the type $R\triangleq\text{String}\rightarrow\text{String}$).
So, we will not be able to implement \lstinline!toList!, \lstinline!foldMap!,
\lstinline!foldLeft!, or \lstinline!reduce! for the \lstinline!Reader!
monad. The only exceptions are types $R$ that have a known finite
set of distinct values, such as $R=$ \lstinline!Boolean!. However,
in those cases the type $R\rightarrow A$ is equivalent to a polynomial
functor:
\[
\text{Reader}^{\text{Boolean},A}=\bbnum 2\rightarrow A\cong A\times A\quad.
\]

We are now ready to show that all polynomial functors are foldable.
It is convenient to use the \lstinline!toList! operation to define
the \textsf{``}\index{foldable functor}foldable functor\textsf{''} typeclass:
\begin{lstlisting}
trait Foldable[L[_]] {
  def toList[A]: L[A] => List[A]
}
\end{lstlisting}
Other folding operations (\lstinline!foldLeft!, \lstinline!foldMap!,
\lstinline!reduce!) can be derived from \lstinline!toList! (Section~\ref{subsec:Equivalence-of-foldLeft,foldMap,reduce,and-toList}).

Polynomial functors are built via the five standard type constructions
(Table~\ref{subsec:Type-notation-and-standard-type-constructions}
without the function types). Defining \lstinline!toList! for these
constructions will provide an implementation of folding operations
for all polynomial functors.

\paragraph{Fixed type}

Constant functors $L^{A}\triangleq Z$, where $Z$ is a fixed type,
are viewed as containers that are always empty and cannot store any
values of type $A$. So, \lstinline!toList! for $L$ always returns
an empty list.

\paragraph{Type parameter}

The identity functor $L^{A}\triangleq A$ is viewed as a container
holding a single value of type $A$. So, we simply define $\text{toList}\triangleq\text{id}$.

\paragraph{Products}

If $K$ and $L$ are foldable functors that support $\text{toList}_{K}$
and $\text{toList}_{L}$, we define the product $M^{A}\triangleq K^{A}\times L^{A}$
and the corresponding function $\text{toList}_{M}$ as:
\begin{lstlisting}
def toList_M[A]: ((K[A], L[A])) => List[A] = { case (p, q) => toList_K(p) ++ toList_L(q) }
\end{lstlisting}
\[
\text{toList}_{M}:K^{A}\times L^{A}\rightarrow\text{List}^{A}\quad,\quad\quad\text{toList}_{M}\triangleq p^{:K^{A}}\times q^{:L^{A}}\rightarrow\text{toList}_{K}(p)\,\pplus\,\text{toList}_{L}(q)\quad.
\]
This implementation contains an arbitrary choice: the values stored
in $p$ are listed before the values stored in $q$. We could equally
well write $\text{toList}_{L}(q)\,\pplus\,\text{toList}_{K}(p)$ in
the function body. This arbitrary choice corresponds to the arbitrariness
of the order in which a folding operation may traverse the values
stored in a container. Different instances of the \lstinline!Foldable!
typeclass may define different traversal orders, as we have seen in
the code examples earlier in this chapter.

\paragraph{Co-products}

If $K$ and $L$ are foldable functors, we define the co-product $M^{A}\triangleq K^{A}+L^{A}$
and the corresponding function $\text{toList}_{M}:K^{A}+L^{A}\rightarrow\text{List}^{A}$
as:

\begin{wrapfigure}{l}{0.6\columnwidth}%
\vspace{-0.25\baselineskip}
\begin{lstlisting}
def toList_M[A]: Either[K[A], L[A]] => List[A] = {
  case Left(p)    => toList_K(p)
  case Right(q)   => toList_L(q)
}
\end{lstlisting}

\vspace{-1\baselineskip}
\end{wrapfigure}%

\noindent ~\vspace{-2\baselineskip}
\[
\text{toList}_{M}\triangleq\,\begin{array}{|c||c|}
 & \text{List}^{A}\\
\hline K^{A} & \text{toList}_{K}\\
L^{A} & \text{toList}_{L}
\end{array}\quad.
\]
\vspace{-0.6\baselineskip}


\paragraph{Recursive types}

We need to implement \lstinline!toList! for a functor $L$ defined
recursively by $L^{A}\triangleq S^{A,L^{A}}$, where the recursion
scheme $S^{\bullet,\bullet}$ is a bifunctor that is itself foldable.
A bifunctor $S^{\bullet,\bullet}$ is \textbf{foldable}\index{foldable bifunctor}
if there is a \lstinline!toList! function with this type signature:
\[
\text{toList}_{S}:S^{A,A}\rightarrow\text{List}^{A}\quad.
\]
We define $\text{toList}_{L}$ using $\text{toList}_{S}$ and the
bifunctor $S$\textsf{'}s liftings (denoted by \lstinline!mapS1! and \lstinline!mapS2!):
\begin{lstlisting}
def toList_L[A]: S[A, L[A]] => List[A] = _.mapS2(toList_L).mapS1(List(_)).toList.flatten
\end{lstlisting}
\[
\text{toList}_{L}:S^{A,L^{A}}\rightarrow\text{List}^{A}\quad,\quad\quad\text{toList}_{L}\triangleq\overline{\text{toList}_{L}}^{\uparrow S^{A,\bullet}}\bef\text{pu}_{\text{List}}^{\uparrow S^{\bullet,\text{List}^{A}}}\bef\text{toList}_{S}\bef\text{ftn}_{\text{List}}\quad.
\]
This code first converts values of type $S^{A,L^{A}}$ into type $S^{A,\text{List}^{A}}$
by lifting \lstinline!toList! recursively to the second type parameter
of $S^{\bullet,\bullet}$. We then convert $S^{A,\text{List}^{A}}$
into $S^{\text{List}^{A},\text{List}^{A}}$ by creating one-element
lists via $\text{pu}_{\text{List}}$. After that, all values of type
$A$ are extracted from $S^{\text{List}^{A},\text{List}^{A}}$ using
$\text{toList}_{S}$, which returns a nested $\text{List}^{\text{List}^{A}}$.
The final \lstinline!flatten! operation reduces that to a $\text{List}^{A}$.

This implements \lstinline!toList! for the five type constructions
that build up polynomial functors. In each case, \lstinline!toList!
extracts all stored values of type $A$. One could write a code generator
to provide a \lstinline!Foldable! instance for any polynomial functor
automatically. When writing code by hand, it is more convenient to
implement folding operations directly, as we did at the beginning
of this chapter.

\subsection{Equivalence of \texttt{traverse} and \texttt{sequence}}

The standard Scala library contains the method \lstinline!Future.sequence!
in addition to \lstinline!Future.traverse!. The main use of \lstinline!Future.sequence!
is for transforming a sequence of \lstinline!Future! values that
may run in parallel. The result is a single \lstinline!Future! value
that waits until all parallel computations are finished. Omitting
some details, we may write the type signature of \lstinline!Future.sequence!
as:
\begin{lstlisting}
def sequence[A, L[X] <: TraversableOnce[X]]: L[Future[A]] => Future[L[A]]
\end{lstlisting}

The function \lstinline!Future.sequence! is limited to sequence-like
data types such as \lstinline!L[A] = List[A]!. To generalize the
\lstinline!sequence! operation to other data types, we replace a
sequence-like type by an arbitrary traversable functor \lstinline!L[A]!.
We also replace the \lstinline!Future! type constructor by an arbitrary
applicative functor (since the \lstinline!traverse! operation accepts
one). It turns out that a function called \lstinline!sequence! can
then be defined via the \lstinline!traverse! operation and vice versa:
\begin{lstlisting}
def sequence[A, L[_]: Traversable, F[_]: Applicative : Functor]: L[F[A]] => F[L[A]] = _.traverse(id)
def traverse[A, B, L[_]: Traversable, F[_]: Applicative : Functor](la: L[A])(f: A => F[B]): F[L[B]] = sequence(la.map(f))
\end{lstlisting}
\begin{align*}
 & \text{seq}_{L}:L^{F^{A}}\rightarrow F^{L^{A}}\quad,\quad\quad\text{seq}_{L}\triangleq\text{trav}_{L}(\text{id})\quad,\\
 & \text{trav}_{L}:(A\rightarrow F^{B})\rightarrow L^{A}\rightarrow F^{L^{B}}\quad,\quad\quad\text{trav}_{L}(f)\triangleq f^{\uparrow L}\bef\text{seq}_{L}\quad.
\end{align*}
An inverse transformation ($F^{L^{A}}\rightarrow L^{F^{A}}$) is not
always possible, as shown in Exercise~\ref{subsec:Exercise-traversables-laws}.
The example with $F=$ \lstinline!Future! and $L=$ \lstinline!List!
may help remember the correct type signature of \lstinline!sequence!.

The equivalence between \lstinline!traverse! and \lstinline!sequence!
holds under the assumption that \lstinline!traverse! obeys a naturality
law with respect to the type parameter $A$. This follows from Statement~\ref{subsec:Statement-tr-equivalent-to-ftr}
where we need to define $\text{tr}\triangleq\text{seq}_{L}$, $\text{ftr}\triangleq\text{trav}_{L}$,
$F\triangleq L$, $G\triangleq F$, and $H\triangleq F\circ L$. 

Because of this equivalence, we may define the \lstinline!Traversable!
typeclass via \lstinline!sequence! instead of via \lstinline!traverse!.
It is easier to study the properties and laws of the \lstinline!sequence!
operation because it has a simpler type signature. It is also easier
to verify the laws of \lstinline!sequence!. So, we will use \lstinline!sequence!
when proving the type constructions for traversable functors.

\subsection{Laws of \texttt{traverse}}

Given a value \lstinline!p: L[A]!, where \lstinline!L! is a traversable
functor, programmers expect that the code \lstinline!p.traverse(f)!
should evaluate the function \lstinline!f: A => F[B]! on each value
of type \lstinline!A! stored within the container \lstinline!p!.
The resulting applicative $F$-effects should be merged into a single
value of type \lstinline!F[L[B]]!. At the same time, the structure
of the initial value \lstinline!p: L[A]! must be preserved as much
as possible in the resulting values of type \lstinline!L[B]! wrapped
under \lstinline!F!. Also, the \lstinline!traverse! function should
work in the same way for all applicative functors \lstinline!F! and
for all types \lstinline!A! and \lstinline!B!. How can we express
these expectations as laws?\footnote{Those laws were given by M.~Jaskelioff\index{Mauro Jaskelioff} and
O.~Rypacek in the paper \textsf{``}An investigation of the laws of traversals\textsf{''},
see \texttt{\href{https://arxiv.org/abs/1202.2919}{https://arxiv.org/abs/1202.2919}}}

A naturality law formalizes the requirement that a function with a
type parameter should work in the same way regardless of the type
chosen for that type parameter. There is one naturality law per type
parameter. So, we impose the naturality laws with respect to type
parameters \lstinline!A! and \lstinline!B!. The form of the naturality
laws follows from the type signature of \lstinline!traverse!:
\[
\text{trav}_{L}^{F,A,B}:(A\rightarrow F^{B})\rightarrow L^{A}\rightarrow F^{L^{B}}\quad.
\]
If we fix \lstinline!B! and let \lstinline!A! vary, the type signature
is of the form $J^{A}\rightarrow K^{A}$, which is a natural transformation
between two contrafunctors. If we fix \lstinline!A! and let \lstinline!B!
vary, this type signature has the form $G^{A}\rightarrow H^{A}$,
which is a natural transformation between two functors. So, we can
formulate the two \index{naturality law!of traverse@of \texttt{traverse}}naturality
laws of \lstinline!traverse! according to the recipes of Section~\ref{subsec:Naturality-laws-and-natural-transformations}.
For any $f^{:X\rightarrow A}$, $g^{:A\rightarrow F^{B}}$, and $h^{:B\rightarrow C}$:
\begin{equation}
\text{trav}_{L}^{F,X,B}(f\bef g)=f^{\uparrow L}\bef\text{trav}_{L}^{F,A,B}(g)\quad,\quad\quad\text{trav}_{L}^{F,A,C}(g\bef h^{\uparrow F})=\text{trav}_{L}^{F,A,B}(g)\bef h^{\uparrow L\uparrow F}\quad.\label{eq:naturality-laws-of-traverse}
\end{equation}
\vspace{-1.2\baselineskip}
\[
\xymatrix{\xyScaleY{2.0pc}\xyScaleX{5.0pc}L^{X}\ar[rd]\sp(0.6){~~\text{trav}_{L}^{F,X,B}(f\bef g)}\ar[d]\sp(0.45){f^{\uparrow L}} &  & L^{A}\ar[rd]\sb(0.35){\text{trav}_{L}^{F,A,C}(g\bef h)~~}\ar[r]\sp(0.5){\text{trav}_{L}^{F,A,B}(g)} & F^{L^{B}}\ar[d]\sp(0.45){h^{\uparrow L\uparrow F}}\\
L^{A}\ar[r]\sb(0.5){\text{trav}_{L}^{F,A,B}(g)} & F^{L^{B}} &  & F^{L^{C}}
}
\]

We also need a naturality law with respect to the parameter \lstinline!F!,
which is a type constructor required to be an applicative functor.
This law expresses the requirement that \lstinline!traverse! may
not inspect the type of \lstinline!F! directly and make decisions
based on that type. The code of \lstinline!traverse! may only use
$F$\textsf{'}s applicative methods (\lstinline!wu! and \lstinline!zip!).
To formulate this \textbf{applicative naturality law}\index{applicative naturality law!of traverse@of \texttt{traverse}},
we write an equation similar to the second naturality law, except
that the arbitrary function $f$ will now map the applicative functor
$F$ to another arbitrary applicative functor $G$:

\begin{wrapfigure}{l}{0.26\columnwidth}%
\vspace{-1.3\baselineskip}
$\xymatrix{\xyScaleY{1.7pc}\xyScaleX{5.0pc}L^{A}\ar[rd]\sb(0.4){\text{trav}_{L}^{G,A,B}(g\bef f)~~}\ar[r]\sp(0.5){\text{trav}_{L}^{F,A,B}(g)} & F^{L^{B}}\ar[d]\sp(0.45){f}\\
 & G^{L^{B}}
}
$\vspace{0.5\baselineskip}
\end{wrapfigure}%

~\vspace{-1.5\baselineskip}

\begin{equation}
\text{trav}_{L}^{G,A,B}(g^{:A\rightarrow F^{B}}\bef f^{:F^{B}\rightarrow G^{B}})=\text{trav}_{L}^{F,A,B}(g)\bef f^{:F^{L^{B}}\rightarrow G^{L^{B}}}\quad.\label{eq:traverse-applicative-naturality-law}
\end{equation}

\noindent Here $f$ is a natural transformation with type signature
$f:\forall X.\,F^{X}\rightarrow G^{X}$. So, we are allowed to apply
the same code of $f$ to different types: $f$ has type $F^{B}\rightarrow G^{B}$
in the left-hand side of the law and $F^{L^{B}}\rightarrow G^{L^{B}}$
in the right-hand side.

In addition to being a natural transformation, the function $f$ must
preserve the applicative typeclass. In other words, $f$ should map
$F$\textsf{'}s applicative methods (\lstinline!wu! and \lstinline!zip!)
to the corresponding methods of $G$. Otherwise, the code of $\text{trav}_{L}^{F,A,B}$
that uses those methods of $F$ will not be mapped via $f$ to the
same code of $\text{trav}_{L}^{G,A,B}$ that uses $G$\textsf{'}s applicative
methods. 

Natural transformations $f:F^{X}\rightarrow G^{X}$ that preserve
the applicative methods of $F$ and $G$ are called \index{applicative morphism}\textbf{applicative
morphisms} (compare to monoid morphisms defined in Section~\ref{subsec:Equivalence-of-foldLeft,foldMap,reduce,and-toList}
and monad morphisms defined in Section~\ref{subsec:Monads-in-category-theory-monad-morphisms}).
The laws of applicative morphisms are:\index{identity laws!of applicative morphisms}\index{composition law!of applicative morphisms}
\begin{align}
{\color{greenunder}\text{identity law of applicative morphism }f:}\quad & f^{:F^{\bbnum 1}\rightarrow G^{\bbnum 1}}(\text{wu}_{F})=\text{wu}_{G}\quad,\label{eq:identity-law-of-applicative-morphism}\\
{\color{greenunder}\text{composition law of applicative morphism }f:}\quad & \text{zip}_{G}\big(f(p^{:F^{A}})\times f(q^{:F^{B}})\big)=f(\text{zip}_{F}(p\times q))\quad.\label{eq:composition-law-of-applicative-morphism}
\end{align}
The applicative naturality law is required to hold only when $f$
is an applicative morphism.

Here are some examples of applicative morphisms and applicative naturality.

\subsubsection{Example \label{subsec:Example-some-applicative-morphisms}\ref{subsec:Example-some-applicative-morphisms}\index{solved examples}}

Consider applicative functors $\text{Id}^{A}\triangleq A$ and $L^{A}\triangleq A\times A$. 

\textbf{(a)} Show that the standard function $\Delta:\text{Id}^{A}\rightarrow L^{A}$
is an applicative morphism.

\textbf{(b)} Show that the standard function $\pi_{1}:L^{A}\rightarrow\text{Id}^{A}$
is an applicative morphism. 

\subparagraph{Solution}

We need to check that these functions obey the laws of applicative
morphisms.

\textbf{(a)} To verify the identity law, write:
\[
\Delta(\text{wu}_{\text{Id}})=\Delta(1)=1\times1=\text{wu}_{P}\quad.
\]

To verify the composition law, begin with its left-hand side:
\begin{align*}
 & \text{zip}_{L}\big(\Delta(p^{:A})\times\Delta(q^{:A})\big)=\text{zip}_{L}\big((p\times p)\times(q\times q)\big)\\
{\color{greenunder}\text{definition of }\text{zip}_{L}:}\quad & =(p\times q)\times(p\times q)=\Delta(p\times q)=\Delta(\text{zip}_{\text{Id}}(p\times q))\quad.
\end{align*}

\textbf{(b)} To verify the identity law, write:
\[
\pi_{1}(\text{wu}_{L})=\pi_{1}(1\times1)=1=\text{wu}_{\text{Id}}\quad.
\]

To verify the composition law, begin with its right-hand side:
\begin{align*}
 & \pi_{1}\big(\text{zip}_{L}((p_{1}^{:A}\times p_{2}^{:A})\times(q_{1}^{:B}\times q_{2}^{:B}))\big)=\pi_{1}\big((p_{1}\times q_{1})\times(p_{2}\times q_{2})\big)=p_{1}\times q_{1}\\
 & =\text{zip}_{\text{Id}}(p_{1}\times q_{1})=\text{zip}_{\text{Id}}(\pi_{1}(p_{1}\times p_{2})\times\pi_{1}(q_{1}\times q_{2}))\quad.
\end{align*}


\subsubsection{Example \label{subsec:Example-naturality-law-of-traverse}\ref{subsec:Example-naturality-law-of-traverse}}

Verify the naturality law of \lstinline!traverse! for $L^{A}\triangleq A\times A$
using an applicative morphism between applicative functors $G^{A}\triangleq\bbnum 1+A$
and $H^{A}\triangleq\bbnum 1+A\times A$. 

\subparagraph{Solution}

The \lstinline!traverse! method for $L^{A}\triangleq A\times A$
is defined by:
\[
\text{trav}_{L}(f^{:A\rightarrow F^{B}})\triangleq p^{:A}\times q^{:A}\rightarrow\text{zip}_{F}(f(p)\times f(q))=(f\boxtimes f)\bef\text{zip}_{F}\quad.
\]
The \lstinline!zip! methods of the applicative functors $G$ and
$H$ are:
\[
\text{zip}_{G}\triangleq\,\begin{array}{|c||cc|}
 & \bbnum 1 & A\times B\\
\hline \bbnum 1\times\bbnum 1 & \_\rightarrow1 & \bbnum 0\\
A\times\bbnum 1 & \_\rightarrow1 & \bbnum 0\\
\bbnum 1\times B & \_\rightarrow1 & \bbnum 0\\
A\times B & \bbnum 0 & \text{id}
\end{array}\quad,\quad\quad\text{zip}_{H}\triangleq\,\begin{array}{|c||cc|}
 & \bbnum 1 & (A\times B)\times(A\times B)\\
\hline \bbnum 1\times\bbnum 1 & \_\rightarrow1 & \bbnum 0\\
(A\times A)\times\bbnum 1 & \_\rightarrow1 & \bbnum 0\\
\bbnum 1\times(B\times B) & \_\rightarrow1 & \bbnum 0\\
(A\times A)\times(B\times B) & \bbnum 0 & \text{zip}_{L}
\end{array}\quad.
\]
The function $\text{zip}_{L}$ was shown in the solution to Example~\ref{subsec:Example-some-applicative-morphisms}(a). 

The applicative morphism $g:G^{A}\rightarrow H^{A}$ is implemented
as:
\[
g:\bbnum 1+A\rightarrow\bbnum 1+A\times A\quad,\quad\quad g\triangleq\,\begin{array}{|c||cc|}
 & \bbnum 1 & A\times A\\
\hline \bbnum 1 & \text{id} & \bbnum 0\\
A & \bbnum 0 & \Delta
\end{array}\quad.
\]
The fully parametric implementations of $\text{trav}_{L}$, $\text{zip}_{G}$,
$\text{zip}_{H}$, and $g$ are forced by their type signatures. The
values $\text{wu}_{G}$ and $\text{wu}_{H}$ need to be chosen as
$\text{wu}_{G}\triangleq\bbnum 0^{:\bbnum 1}+1$ and $\text{wu}_{H}\triangleq\bbnum 0^{:\bbnum 1}+1\times1$
in order to obey the identity laws of applicative morphisms.

Next, we verify that $g$ is an applicative morphism. To check the
identity law of $g$:
\[
\text{wu}_{G}\triangleright g=\,\begin{array}{|cc|}
\bbnum 0 & 1\end{array}\,\triangleright\,\begin{array}{|c||cc|}
 & \bbnum 1 & A\times A\\
\hline \bbnum 1 & \text{id} & \bbnum 0\\
A & \bbnum 0 & \Delta
\end{array}\,=\,\begin{array}{|cc|}
\bbnum 0 & (1\triangleright\Delta)\end{array}\,=\bbnum 0+1\times1=\text{wu}_{H}\quad.
\]
To check the composition law of $g$, first rewrite that law in the
point-free style:\index{composition law!of applicative morphisms!in the point-free style}\index{point-free style}
\begin{equation}
(g\boxtimes g)\bef\text{zip}_{H}=\text{zip}_{G}\bef g\quad.\label{eq:composition-law-applicative-morphism-point-free}
\end{equation}
The pair product ($g\boxtimes g$) can be written in matrix form like
this:
\[
g\boxtimes g=\,\begin{array}{|c||cc|}
 & \bbnum 1 & A\times A\\
\hline \bbnum 1 & \text{id} & \bbnum 0\\
A & \bbnum 0 & \Delta
\end{array}\,\boxtimes\,\begin{array}{|c||cc|}
 & \bbnum 1 & B\times B\\
\hline \bbnum 1 & \text{id} & \bbnum 0\\
B & \bbnum 0 & \Delta
\end{array}\,=\,\begin{array}{|c||cccc|}
 & \bbnum 1 & A\times A\times\bbnum 1 & \bbnum 1\times B\times B & A\times A\times B\times B\\
\hline \bbnum 1\times\bbnum 1 & \text{id} & \bbnum 0 & \bbnum 0 & \bbnum 0\\
A\times\bbnum 1 & \bbnum 0 & \Delta\boxtimes\text{id} & \bbnum 0 & \bbnum 0\\
\bbnum 1\times B & \bbnum 0 & \bbnum 0 & \text{id}\boxtimes\Delta & \bbnum 0\\
A\times B & \bbnum 0 & \bbnum 0 & \bbnum 0 & \Delta\boxtimes\Delta
\end{array}\quad.
\]
The two sides of the composition law~(\ref{eq:composition-law-applicative-morphism-point-free})
are then simplified to:
\begin{align*}
 & (g\boxtimes g)\bef\text{zip}_{H}=\,\begin{array}{||cccc|}
\text{id} & \bbnum 0 & \bbnum 0 & \bbnum 0\\
\bbnum 0 & \Delta\boxtimes\text{id} & \bbnum 0 & \bbnum 0\\
\bbnum 0 & \bbnum 0 & \text{id}\boxtimes\Delta & \bbnum 0\\
\bbnum 0 & \bbnum 0 & \bbnum 0 & \Delta\boxtimes\Delta
\end{array}\,\bef\,\begin{array}{||cc|}
\_\rightarrow1 & \bbnum 0\\
\_\rightarrow1 & \bbnum 0\\
\_\rightarrow1 & \bbnum 0\\
\bbnum 0 & \text{zip}_{L}
\end{array}\,=\,\begin{array}{||cc|}
\_\rightarrow1 & \bbnum 0\\
\_\rightarrow1 & \bbnum 0\\
\_\rightarrow1 & \bbnum 0\\
\bbnum 0 & (\Delta\boxtimes\Delta)\bef\text{zip}_{L}
\end{array}\quad,\\
 & \text{zip}_{G}\bef g=\,\begin{array}{|c||cc|}
 & \bbnum 1 & A\times B\\
\hline \bbnum 1\times\bbnum 1 & \_\rightarrow1 & \bbnum 0\\
A\times\bbnum 1 & \_\rightarrow1 & \bbnum 0\\
\bbnum 1\times B & \_\rightarrow1 & \bbnum 0\\
A\times B & \bbnum 0 & \text{id}
\end{array}\,\bef\,\begin{array}{|c||cc|}
 & \bbnum 1 & (A\times B)\times(A\times B)\\
\hline \bbnum 1 & \text{id} & \bbnum 0\\
A\times B & \bbnum 0 & \Delta
\end{array}\,=\,\begin{array}{||cc|}
\_\rightarrow1 & \bbnum 0\\
\_\rightarrow1 & \bbnum 0\\
\_\rightarrow1 & \bbnum 0\\
\bbnum 0 & \Delta
\end{array}\quad.
\end{align*}
The remaining difference is:
\[
(\Delta\boxtimes\Delta)\bef\text{zip}_{L}\overset{?}{=}\Delta\quad.
\]
This equation can be verified by applying both sides to some $a^{:A}\times b^{:B}$:
\begin{align*}
 & (a\times b)\triangleright(\Delta\boxtimes\Delta)\bef\text{zip}_{L}=\big((a\times a)\times(b\times b)\big)\triangleright\text{zip}_{L}=(a\times b)\times(a\times b)\quad,\\
 & (a\times b)\triangleright\Delta=(a\times b)\times(a\times b)\quad.
\end{align*}

We are now ready to check that the \lstinline!traverse! method of
$L$ satisfies the naturality law~(\ref{eq:traverse-applicative-naturality-law}),
where we swap $f$ with $g$ and use applicative functors $G$, $H$
instead of $F$, $G$. Write the two sides of the law:
\begin{align*}
{\color{greenunder}\text{left-hand side}:}\quad & \text{trav}_{L}^{H,A,B}(f^{:A\rightarrow G^{B}}\bef g^{:G^{B}\rightarrow H^{B}})=\big((f\bef g)\boxtimes(f\bef g)\big)\bef\text{zip}_{H}\quad,\\
{\color{greenunder}\text{right-hand side}:}\quad & \text{trav}_{L}^{G,A,B}(f^{:A\rightarrow G^{B}})\bef g^{:G^{L^{B}}\rightarrow H^{L^{B}}}=(f\boxtimes f)\bef\gunderline{\text{zip}_{G}\bef g}\\
{\color{greenunder}\text{composition law~(\ref{eq:composition-law-applicative-morphism-point-free})}:}\quad & \quad=\gunderline{(f\boxtimes f)\bef(g\boxtimes g)}\bef\text{zip}_{H}\\
{\color{greenunder}\text{composition law~(\ref{eq:pair-product-composition-law})}:}\quad & \quad=\gunderline{\big((f\bef g)\boxtimes(f\bef g)\big)}\bef\text{zip}_{H}\quad.
\end{align*}
The two sides of the law are now equal. $\square$

In this derivation, we were able to pass from $\text{zip}_{G}$ to
$\text{zip}_{H}$ only because $g$, being an applicative morphism,
maps $G$\textsf{'}s \lstinline!zip! operation into $H$\textsf{'}s \lstinline!zip!
operation according to Eq.~(\ref{eq:composition-law-of-applicative-morphism}).

\subsubsection{Example \label{subsec:Example-pure-is-applicative-morphism}\ref{subsec:Example-pure-is-applicative-morphism}}

For any applicative functor $F$, show that the \lstinline!pure!
method ($\text{pu}_{F}:A\rightarrow F^{A}$) is an applicative morphism
between the identity functor ($\text{Id}^{A}\triangleq A$) and $F$.

\subparagraph{Solution}

The identity functor\textsf{'}s applicative methods are identity functions:
\[
\text{wu}_{\text{Id}}=1\quad,\quad\quad\text{pu}_{\text{Id}}(x)=x\quad,\quad\quad\text{zip}_{\text{Id}}(p\times q)=p\times q\quad.
\]
We need to show that $\text{pu}_{F}$ obeys the laws~(\ref{eq:identity-law-of-applicative-morphism})\textendash (\ref{eq:composition-law-of-applicative-morphism}).
To verify the identity law:
\[
\text{pu}_{F}(\text{wu}_{\text{Id}})\overset{?}{=}\text{wu}_{F}=\text{pu}_{F}(1)=\text{pu}_{F}(\text{wu}_{\text{Id}})\quad.
\]
To verify the composition law:
\begin{align*}
 & \text{zip}_{F}\big(\text{pu}_{F}(p^{:A})\times\text{pu}_{F}(q^{:B})\big)\overset{?}{=}\text{pu}_{F}(\text{zip}_{\text{Id}}(p\times q))=\text{pu}_{F}(p\times q)\\
{\color{greenunder}\text{use Exercise~\ref{subsec:Exercise-zip-pure-pure}}:}\quad & =\text{zip}_{F}(\text{pu}_{F}(p)\times\text{pu}_{F}(q))\quad.
\end{align*}

$\square$

We have analyzed the expectation that \lstinline!traverse! should
work in the same way for all types and applicative functors. Another
reasonable expectation is that \lstinline!p.traverse(f)! should visit
\emph{every} value of type \lstinline!A! stored inside \lstinline!p!
(where \lstinline!p! has type $L^{A}$). To formulate that expectation
as a law, consider for simplicity a function $f$ of type $A\rightarrow F^{A}$.
Then the final result of \lstinline!p.traverse(f)! has type $F^{L^{A}}$,
which can be visualized as some values of type $L^{A}$ wrapped under
an $F$-effect. We expect those values to be similar to \lstinline!p!
in their structure.

To formulate this condition in a simple way, we may choose functions
$f$ that always return an \emph{empty} $F$-effect (that is, a result
of applying $F$\textsf{'}s \lstinline!pure! method to some value). So, we
choose $f\triangleq\text{pu}_{F}$. We expect that traversing \lstinline!p!
with this function $f$ should reproduce the same value \lstinline!p!
wrapped in an empty $F$-effect. So, we write the law as \lstinline!p.traverse(pure) == pure(p)!,
or in the point-free style:
\begin{equation}
\text{trav}_{L}^{F,A,A}(\text{pu}_{F})=\text{pu}_{F}\quad.\label{eq:traverse-identity-law-with-pure}
\end{equation}

Alternatively, we may choose $F^{A}\triangleq\text{Id}^{A}=A$ (the
identity functor), $A=B$, and $f^{:A\rightarrow F^{B}}=\text{id}^{:A\rightarrow A}$.
Then the result of \lstinline!p.traverse(f)! will have type $F^{L^{A}}=L^{A}$
and is expected to equal the initial value \lstinline!p!, as the
transformation \lstinline!f! is an identity function. We obtain the
\textbf{identity law}\index{identity laws!of traverse@of \texttt{traverse}}
of \lstinline!traverse!:
\begin{equation}
\text{trav}_{L}^{\text{Id},A,A}(\text{id})=\text{id}\quad.\label{eq:traverse-identity-law}
\end{equation}

We can show that this law is equivalent to Eq.~(\ref{eq:traverse-identity-law-with-pure}):

\subsubsection{Statement \label{subsec:Statement-identity-law-traverse-simplified}\ref{subsec:Statement-identity-law-traverse-simplified}}

\textbf{(a)} If \lstinline!traverse! obeys Eq.~(\ref{eq:traverse-identity-law-with-pure})
then it also obeys Eq.~(\ref{eq:traverse-identity-law}).

\textbf{(b)} If \lstinline!traverse! obeys Eq.~(\ref{eq:traverse-identity-law})
and the naturality law~(\ref{eq:traverse-applicative-naturality-law})
then it also obeys Eq.~(\ref{eq:traverse-identity-law-with-pure}).

\subparagraph{Proof}

\textbf{(a)} We may set $F\triangleq\text{Id}$ in the law~(\ref{eq:traverse-identity-law-with-pure}).
Then we get Eq.~(\ref{eq:traverse-identity-law}) since $\text{pu}_{\text{Id}}=\text{id}$.

\textbf{(b)} Example~\ref{subsec:Example-pure-is-applicative-morphism}
shows that $\text{pu}_{F}$ is an applicative morphism between $\text{Id}$
and $F$. So, we may set $F\triangleq\text{Id}$, $G\triangleq F$,
$f\triangleq\text{pu}_{F}$, and $g\triangleq\text{\text{id}}$ in
the applicative naturality law~(\ref{eq:traverse-applicative-naturality-law}).
We then obtain:
\begin{align*}
{\color{greenunder}\text{applicative naturality law~(\ref{eq:traverse-applicative-naturality-law})}:}\quad & \text{trav}_{L}(\text{id}\bef f)\overset{!}{=}\gunderline{\text{trav}_{L}(\text{id})}\bef f\\
{\color{greenunder}\text{identity law~(\ref{eq:traverse-identity-law})}:}\quad & =\text{id}\bef f=f=\text{pu}_{F}\quad.
\end{align*}
At the same time, we have: $\text{trav}_{L}(\text{id}\bef f)=\text{trav}_{L}(f)=\text{trav}_{L}(\text{pu}_{F})$.
So, Eq.~(\ref{eq:traverse-identity-law-with-pure}) holds. $\square$

Apart from visiting every stored value exactly once, we expect that
all $F$-effects returned by $f$ are collected by \lstinline!traverse!
exactly once. To see an example where that expectation fails, consider
a simple traversable functor $L^{A}=A\times A$ and write the \lstinline!traverse!
method like this:
\begin{lstlisting}
def trav2[A, B, F[_]: WuZip : Functor](f: A => F[B])(la: (A, A)): F[(B, B)] = {
  val (f1, f2) = (f(la._1), f(la._2))
  (f1 zip f1 zip f2).map { case ((x, y), z) => (y, z) }
}
\end{lstlisting}
This code applies the function \lstinline!f! to both values in the
given pair. The result is a pair \lstinline!(f1, f2)! of type \lstinline!(F[B], F[B])!.
The code copies the first $F$-effect twice and uses \lstinline!zip!
to combine the three $F$-effects together. The first of the wrapped
values of type \lstinline!B! is discarded.

We expect that the function \lstinline!trav2! should violate some
law of \lstinline!traverse!. It turns out that a suitable law is
found by composing two \lstinline!traverse! operations. Consider
two different applicative functors ($F$ and $G$), two functions
$f:A\rightarrow F^{B}$ and $g:B\rightarrow G^{C}$, and compute the
following composition:\vspace{-0.15\baselineskip}
\[
p\triangleright\text{trav}_{L}^{F,A,B}(f)\triangleright\big(\text{trav}_{L}^{G,B,C}(g)\big)^{\uparrow F}:F^{G^{L^{C}}}\quad.
\]
\vspace{-0.65\baselineskip}

\noindent The second \lstinline!traverse! operation needs to be lifted
to \lstinline!F! for the types to match. The result (of type $F^{G^{L^{C}}}$)
looks like a \lstinline!traverse! operation with respect to the functor
$F\circ G$. By Statement~\ref{subsec:Statement-applicative-composition},
the functor $F\circ G$ is applicative. So, we may apply a single
\lstinline!traverse! operation using that functor and obtain:\vspace{-0.5\baselineskip}
\[
p\triangleright\text{trav}_{L}^{F\circ G,A,C}(f\bef g^{\uparrow F}):F^{G^{L^{C}}}\quad.
\]

\begin{wrapfigure}{l}{0.38\columnwidth}%
\vspace{-1.6\baselineskip}
$\xymatrix{\xyScaleY{1.5pc}\xyScaleX{3.4pc}L^{A}\ar[dr]\sb(0.35){\text{trav}_{L}^{F\circ G,A,C}(f\bef g^{\uparrow F})~~}\ar[r]\sp(0.5){\text{trav}_{L}^{F,A,B}(f)} & F^{L^{B}}\ar[d]\sp(0.4){\big(\text{trav}_{L}^{G,B,C}(g)\big)^{\uparrow F}}\\
 & F^{G^{L^{C}}}
}
$\vspace{-1\baselineskip}
\end{wrapfigure}%

\noindent The \textbf{composition law}\index{composition law!of traverse@of \texttt{traverse}}
of \lstinline!traverse! says that the two results should be equal:
\begin{equation}
\negthickspace\negthickspace\negthickspace\negthickspace\text{trav}_{L}^{F,A,B}(f)\bef\big(\text{trav}_{L}^{G,B,C}(g)\big)^{\uparrow F}=\text{trav}_{L}^{F\circ G,A,C}(f\bef g^{\uparrow F})\quad.\label{eq:composition-law-of-traverse}
\end{equation}

So far, we have not given a motivation for the law~(\ref{eq:composition-law-of-traverse}).
We just wrote that law by analogy with composition laws of other \textsf{``}lifting\textsf{''}-like
functions. It is not obvious that the law~(\ref{eq:composition-law-of-traverse})
indeed prevents \lstinline!traverse! from evaluating some effects
more than once. To get a heuristic explanation, assume that \lstinline!traverse!
evaluates some effect twice. The left-hand side of the law~(\ref{eq:composition-law-of-traverse})
will evaluate some $F$-effect twice and then evaluate some $G$-effect
twice, wrapped under the twice evaluated $F$-effect. The right-hand
side will evaluate an $(F\circ G)$-effect twice. To illustrate the
difference between the resulting effects, we may write heuristically
$(F\times F)\circ(G\times G)$ and $(F\circ G)\times(F\circ G)$.
In general, these results will not be the same when $F$-effects and
$G$-effects do not commute.

To make this heuristic explanation more rigorous, let us show an example
where the law~(\ref{eq:composition-law-of-traverse}) is violated
by the function \lstinline!trav2! shown above. We need to choose
sufficiently complicated applicative functors $F$ and $G$ whose
effects do not commute. A suitable choice is to set both $F$ and
$G$ to the \lstinline!State! monad.\footnote{The paper \texttt{\href{https://arxiv.org/abs/1202.2919}{https://arxiv.org/abs/1202.2919}}
shows another example using the \lstinline!List! monad.} For simplicity, we set all type parameters to \lstinline!Int! and
choose the functions $f$ and $g$ that add their arguments to the
internal state. The complete code (using the \lstinline!Zippable!
typeclass defined in Section~\ref{subsec:The-Zippable-and-Applicative-typeclass})
is shown in Fig.~\ref{fig:Full-code-implementing-traverse-law-counterexample}.

\begin{figure}
\begin{centering}
\begin{lstlisting}[frame=single,fillcolor={\color{black}},framesep={0.2mm},framexleftmargin=2mm,framexrightmargin=2mm,framextopmargin=2mm,framexbottommargin=2mm]
import io.chymyst.ch.implement // Automatic implementation of some methods.

type L[A] = (A, A) // A very simple but nontrivial traversable functor.

implicit val functorL: Functor[L] = new Functor[L] {
  override def map[A, B](fa: (A, A))(f: A => B): (B, B) = implement
}

def trav2[A, B, F[_] : Zippable : Functor](f: A => F[B])(la: L[A]): F[L[B]] = la.map(f) match {
  case (f1, f2) => f1 zip f1 zip f2
}

final case class S[A](run: Int => (A, Int)) // State monad with internal state of type Int.

implicit val FunctorS: Functor[S] = new Functor[S] {
  override def map[A, B](fa: S[A])(f: A => B): S[B] = implement
}

implicit val ZippableS: Zippable[S] = new Zippable[S] {
  override def zip[A, B](fa: S[A], fb: S[B]): S[(A, B)] = S { i  =>
    val (a, j) = fa.run(i)
    val (b, k) = fb.run(j)
    ((a, b), k)
  }
}

  // Define F[A] as the composition S[S[A]] and implement WuZip and Functor instances.
final case class F[A](run: S[S[A]]) {
  // A runner method to help with testing.
  def eval(i: Int, j: Int): (A, Int, Int) = {
    val (sa, k) = run.run(i)
    val (a, l) = sa.run(j)
    (a, k, l)
  }
}

implicit val FunctorF: Functor[F] = new Functor[F] {
  override def map[A, B](fa: F[A])(f: A => B): F[B] = implement
}

implicit val ZippableF: Zippable[F] = new Zippable[F] {
  override def zip[A, B](fa: F[A], fb: F[B]): F[(A, B)] =
    F((fa.run zip fb.run).map { case (sa, sb) => sa zip sb })
}

val f: Int => S[Int] = i => S(j => (i + j, i + j))
val g: Int => S[Int] = f
val ff: Int => F[Int] = i => F(f(i).map(g))
val l: L[Int] = (1, 0)

val result1: F[L[Int]] = trav2[Int, Int, F](f1f2)(l)
val result2: F[L[Int]] = {
  val x: S[(Int, Int)] = trav2[Int, Int, S](f1)(l)
  val y: S[S[(Int, Int)]] = x.map(trav2[Int, Int, S](f2))
  F(y)
}

// Show that result1 is not equal to result2:

scala> result1.eval(0, 0)
res0: ((Int, Int), Int, Int) = ((3, 5), 2, 5)

scala> result2.eval(0,0)
res1: ((Int, Int), Int, Int) = ((4, 6), 2, 6) 
\end{lstlisting}
\par\end{centering}
\caption{A counterexample violating the composition law of \lstinline!traverse!.\label{fig:Full-code-implementing-traverse-law-counterexample}}
\end{figure}

We have shown that \emph{some} incorrect implementations of \lstinline!traverse!
are excluded by the composition laws. But it remains unknown (Problem~\ref{par:Problem-traverse-law})
whether the composition law~(\ref{eq:composition-law-of-traverse})
always guarantees that the traverse function collects each $F$-effect
exactly once.

\subsection{Laws of \texttt{sequence}}

We now derive the laws of \lstinline!sequence! that correspond to
the laws of \lstinline!traverse! found previously. 

Begin with the naturality laws. Since the type signature of \lstinline!sequence!
($L^{F^{A}}\rightarrow F^{L^{A}}$) has only one type parameter, \lstinline!sequence!
has only one naturality law\index{naturality law!of sequence@of \texttt{sequence}}:%
\begin{comment}
precarious formatting
\end{comment}

\begin{wrapfigure}{i}{0.22\columnwidth}%
\vspace{-1.3\baselineskip}
$\xymatrix{\xyScaleY{1.5pc}\xyScaleX{2.0pc}L^{F^{A}}\ar[d]\sb(0.45){\text{seq}_{L}^{F,A}}\ar[r]\sb(0.5){f^{\uparrow F\uparrow L}} & L^{F^{B}}\ar[d]\sp(0.45){\text{seq}_{L}^{F,B}}\\
F^{L^{A}}\ar[r]\sp(0.5){f^{\uparrow L\uparrow F}} & F^{L^{B}}
}
$\vspace{1.5\baselineskip}
\end{wrapfigure}%

~\vspace{-1.4\baselineskip}

\begin{equation}
(f^{:A\rightarrow B})^{\uparrow F\uparrow L}\bef\text{seq}_{L}^{F,B}=\text{seq}_{L}^{F,A}\bef f^{\uparrow L\uparrow F}\quad.\label{eq:sequence-naturality-law}
\end{equation}

\noindent It will be shown in Exercise~\ref{subsec:Exercise-traversables-laws-2}
that this naturality law is equivalent to the two naturality laws
of \lstinline!traverse!.

Turning now to the applicative naturality law~(\ref{eq:traverse-applicative-naturality-law})
of \lstinline!traverse!, we derive the corresponding applicative
naturality law of \lstinline!sequence!. Express \lstinline!traverse!
via \lstinline!sequence! and substitute into Eq.~(\ref{eq:traverse-applicative-naturality-law}):
\begin{align*}
{\color{greenunder}\text{left-hand side}:}\quad & \text{trav}_{L}^{G,A,B}(g\bef f^{B})=(g\bef f^{B})^{\uparrow L}\bef\text{seq}_{L}^{G,B}=g^{\uparrow L}\bef f^{\uparrow L}\bef\text{seq}_{L}\quad,\\
{\color{greenunder}\text{right-hand side}:}\quad & \text{trav}_{L}^{F,A,B}(g)\bef f^{L^{B}}=g^{\uparrow L}\bef\text{seq}_{L}^{F,B}\bef f^{L^{B}}\quad.
\end{align*}
The function $g^{\uparrow L}$ can be omitted from both sides (see
Exercise~\ref{subsec:Exercise-simplify-law-omit-lifted-function}
for a similar property).

\begin{wrapfigure}{l}{0.22\columnwidth}%
\vspace{-1.1\baselineskip}
$\xymatrix{\xyScaleY{1.5pc}\xyScaleX{2.0pc}L^{F^{A}}\ar[d]\sb(0.45){\text{seq}_{L}^{F,A}}\ar[r]\sb(0.5){(f^{A})^{\uparrow L}} & L^{G^{A}}\ar[d]\sp(0.45){\text{seq}_{L}^{G,A}}\\
F^{L^{A}}\ar[r]\sp(0.5){f^{L^{A}}} & G^{L^{A}}
}
$\vspace{-0.5\baselineskip}
\end{wrapfigure}%

\noindent Renaming $B$ to $A$, we rewrite the applicative naturality
law\index{applicative naturality law!of sequence@of \texttt{sequence}}
of \lstinline!sequence! as:
\[
(f^{A})^{\uparrow L}\bef\text{seq}_{L}^{G,A}=\text{seq}_{L}^{F,A}\bef f^{L^{A}}\quad.
\]

\noindent Here we have written the full type parameters in $f^{X}$
and $\text{seq}_{L}^{F,A}$ for clarity.

Turning now to the identity law, we again express \lstinline!traverse!
via \lstinline!sequence! and substitute into the simpler formulation~(\ref{eq:traverse-identity-law})
of the identity law of \lstinline!traverse!:
\[
\text{id}\overset{!}{=}\text{trav}_{L}^{\text{Id},A,A}(\text{id})=\text{id}^{\uparrow L}\bef\text{seq}_{L}^{\text{Id},A}\overset{!}{=}\text{seq}_{L}^{\text{Id},A}\quad.
\]
So, the \textbf{identity law} of \lstinline!sequence!\index{identity laws!of sequence@of \texttt{sequence}}
is:
\begin{equation}
\text{seq}_{L}^{\text{Id},A}=\text{id}\quad.\label{eq:identity-law-of-sequence}
\end{equation}

The composition law~(\ref{eq:composition-law-of-traverse}) of \lstinline!traverse!
leads to the following composition law of \lstinline!sequence!:
\begin{align*}
{\color{greenunder}\text{left-hand side}:}\quad & \text{trav}_{L}^{F,A,B}(f^{:A\rightarrow F^{B}})\bef\big(\text{trav}_{L}^{G,B,C}(g^{:B\rightarrow G^{C}})\big)^{\uparrow F}=f^{\uparrow L}\bef\text{seq}_{L}^{F,B}\bef\big(g^{\uparrow L}\bef\text{seq}_{L}^{G,C}\big)^{\uparrow F}\\
 & \quad=f^{\uparrow L}\bef\gunderline{\text{seq}_{L}^{F,B}\bef g^{\uparrow L\uparrow F}}\bef(\text{seq}_{L}^{G,C})^{\uparrow F}\\
{\color{greenunder}\text{naturality law~(\ref{eq:sequence-naturality-law})}:}\quad & \quad=f^{\uparrow L}\bef g^{\uparrow F\uparrow L}\bef\text{seq}_{L}^{F,G^{C}}\bef(\text{seq}_{L}^{G,C})^{\uparrow F}\quad,\\
{\color{greenunder}\text{right-hand side}:}\quad & \text{trav}_{L}^{F\circ G,A,C}(f\bef g^{\uparrow F})=(f\bef g^{\uparrow F})^{\uparrow L}\bef\text{seq}_{L}^{F\circ G,C}\quad.
\end{align*}

\begin{wrapfigure}{l}{0.3\columnwidth}%
\vspace{-1.7\baselineskip}
$\xymatrix{\xyScaleY{1.4pc}\xyScaleX{3.0pc}L^{F^{G^{A}}}\ar[r]\sp(0.55){\text{seq}_{L}^{F,G^{A}}}\ar[rd]\sb(0.4){\text{seq}_{L}^{F\circ G,A}} & F^{L^{G^{A}}}\ar[d]\sp(0.4){(\text{seq}_{L}^{G,A})^{\uparrow F}}\\
 & F^{G^{L^{A}}}
}
$\vspace{-1.9\baselineskip}
\end{wrapfigure}%

\noindent Omitting the common function $(f\bef g^{\uparrow F})^{\uparrow L}$
and renaming $C$ to $A$, we obtain the \index{composition law!of sequence@of \texttt{sequence}}\textbf{composition
law} of \lstinline!sequence!:
\begin{equation}
\text{seq}_{L}^{F,G^{A}}\bef(\text{seq}_{L}^{G,A})^{\uparrow F}=\text{seq}_{L}^{F\circ G,A}\quad.\label{eq:composition-law-of-sequence}
\end{equation}

Since \lstinline!sequence! has a simpler type signature than \lstinline!traverse!
and its laws have a simpler form, we will use \lstinline!sequence!
for verifying the laws of traversable functors. 

\subsection{All polynomial functors are traversable\label{subsec:All-polynomial-functors-are-traversable}}

To show that all polynomial functors are traversable, we implement
a \lstinline!sequence! operation for the five type constructions
that build up polynomial functors. We will verify the laws in every
case.

\paragraph{Fixed type}

If $L^{A}\triangleq Z$ where $Z$ is a fixed type, we define \lstinline!sequence!
via $F$\textsf{'}s \lstinline!pure! method:
\[
\text{seq}_{L}^{F,A}:L^{F^{A}}\rightarrow F^{L^{A}}\cong Z\rightarrow F^{Z}\quad,\quad\quad\text{seq}_{L}^{F,A}\triangleq\text{pu}_{F}^{:Z\rightarrow F^{Z}}\quad.
\]
The identity law~(\ref{eq:identity-law-of-sequence}) holds since
$\text{seq}_{L}^{\text{Id},A}=\text{pu}_{\text{Id}}=\text{id}$. To
verify the composition law~(\ref{eq:composition-law-of-sequence}):
\begin{align*}
{\color{greenunder}\text{left-hand side}:}\quad & \text{seq}_{L}^{F,G^{A}}\bef(\text{seq}_{L}^{G,A})^{\uparrow F}=\text{pu}_{F}\bef(\text{pu}_{G})^{\uparrow F}\\
{\color{greenunder}\text{naturality law of }\text{pu}_{F}:}\quad & \quad=\text{pu}_{G}\bef\text{pu}_{F}\quad,\\
{\color{greenunder}\text{right-hand side}:}\quad & \text{seq}_{L}^{F\circ G,A}=\text{pu}_{F\circ G}\\
{\color{greenunder}\text{definition of }\text{pu}_{F\circ G}:}\quad & \quad=\text{pu}_{G}\bef\text{pu}_{F}\quad.
\end{align*}
The two sides of the law are now equal.

\paragraph{Type parameter}

If $L^{A}\triangleq A$ (the identity functor), \lstinline!sequence!
is defined as the identity function. Identity functions always satisfy
identity and composition laws.

The functor composition $L^{A}\triangleq M^{N^{A}}$ (where $M$ and
$N$ are some traversable functors) is another case where the type
parameter is used to define $L$. We assume that $\text{seq}_{M}$
and $\text{seq}_{N}$ are already available and satisfy the laws,
and define $\text{seq}_{L}$ as:
\begin{equation}
\text{seq}_{L}:M^{N^{F^{A}}}\rightarrow F^{M^{N^{A}}}\quad,\quad\quad\text{seq}_{L}\triangleq(\text{seq}_{N}^{F,A})^{\uparrow M}\bef\text{seq}_{M}^{F,N^{A}}\quad.\label{eq:def-sequence-for-functor-composition}
\end{equation}
The laws of $\text{seq}_{L}$ will be proved in Exercise~\ref{subsec:Exercise-traversables-5}.

\paragraph{Products}

Here $L^{A}\triangleq M^{A}\times N^{A}$, with some traversable functors
$M$ and $N$. We assume that $\text{seq}_{M}$ and $\text{seq}_{N}$
are already available and satisfy the laws, and define $\text{seq}_{L}$
as:
\[
\text{seq}_{L}^{F,A}:M^{F^{A}}\times N^{F^{A}}\rightarrow F^{M^{A}\times N^{A}}\quad,\quad\quad\text{seq}_{L}^{F,A}\triangleq m^{:M^{F^{A}}}\times n^{:N^{F^{A}}}\rightarrow\text{zip}_{F}\big((m\triangleright\text{seq}_{M}^{F,A})\times(n\triangleright\text{seq}_{N}^{F,A})\big)\quad.
\]
In the point-free style, we can write the same code in a shorter formula:
\[
\text{seq}_{L}^{F,A}\triangleq(\text{seq}_{M}^{F,A}\boxtimes\text{seq}_{N}^{F,A})\bef\text{zip}_{F}\quad.
\]

To verify the identity law~(\ref{eq:identity-law-of-sequence}),
use the fact that $\text{zip}_{\text{Id}}=\text{id}$:
\begin{align*}
 & \text{seq}_{L}^{\text{Id},A}=(\gunderline{\text{seq}_{M}^{\text{Id},A}}\boxtimes\gunderline{\text{seq}_{N}^{\text{Id},A}})\bef\text{zip}_{\text{Id}}\\
{\color{greenunder}\text{identity laws of }\text{seq}_{M}\text{ and }\text{seq}_{N}:}\quad & =(\text{id}\boxtimes\text{id})\bef\text{id}=\text{id}\quad.
\end{align*}

To verify the composition law~(\ref{eq:composition-law-of-sequence}),
begin with its left-hand side:
\begin{align*}
{\color{greenunder}\text{left-hand side}:}\quad & \text{seq}_{L}^{F,G^{A}}\bef(\text{seq}_{L}^{G,A})^{\uparrow F}=(\text{seq}_{M}^{F,G^{A}}\boxtimes\text{seq}_{N}^{F,G^{A}})\bef\gunderline{\text{zip}_{F}\bef(\text{seq}_{M}^{G,A}\boxtimes\text{seq}_{N}^{G,A})^{\uparrow F}}\bef\text{zip}_{G}^{\uparrow F}\\
{\color{greenunder}\text{naturality law of }\text{zip}_{F}:}\quad & =\gunderline{(\text{seq}_{M}^{F,G^{A}}\boxtimes\text{seq}_{N}^{F,G^{A}})\bef\big((\text{seq}_{M}^{G,A})^{\uparrow F}\boxtimes(\text{seq}_{N}^{G,A})^{\uparrow F}\big)}\bef\text{zip}_{F}\bef\text{zip}_{G}^{\uparrow F}\\
{\color{greenunder}\text{composition law~(\ref{eq:pair-product-composition-law})}:}\quad & =\big((\text{seq}_{M}^{F,G^{A}}\bef(\text{seq}_{M}^{G,A})^{\uparrow F})\boxtimes(\text{seq}_{N}^{F,G^{A}}\bef(\text{seq}_{N}^{G,A})^{\uparrow F})\big)\bef\text{zip}_{F}\bef\text{zip}_{G}^{\uparrow F}\quad.
\end{align*}
By assumption, $\text{seq}_{M}$ and $\text{seq}_{N}$ satisfy their
composition laws. So, we may rewrite the last line as:
\[
\big(\text{seq}_{M}^{F\circ G,A}\boxtimes\text{seq}_{N}^{F\circ G,A}\big)\bef\text{zip}_{F}\bef\text{zip}_{G}^{\uparrow F}\quad.
\]
We can now transform the right-hand side of the law to the same expression:
\begin{align*}
{\color{greenunder}\text{right-hand side}:}\quad & \text{seq}_{L}^{F\circ G,A}=\big(\text{seq}_{M}^{F\circ G,A}\boxtimes\text{seq}_{N}^{F\circ G,A}\big)\bef\text{zip}_{F\circ G}\\
{\color{greenunder}\text{definition of }\text{zip}_{F\circ G}:}\quad & =\big(\text{seq}_{M}^{F\circ G,A}\boxtimes\text{seq}_{N}^{F\circ G,A}\big)\bef\text{zip}_{F}\bef\text{zip}_{G}^{\uparrow F}\quad.
\end{align*}

The given implementation of \lstinline!sequence! will first collect
the $F$-effects stored in $M^{F^{A}}$ and then the $F$-effects
stored in $N^{F^{A}}$. An alternative implementation could first
iterate over the data in $N$ and then in $M$. That implementation
still obeys the laws (see Exercise~\ref{subsec:Exercise-traversables-3-1}
for a proof in case $M=N=\text{Id}$).

\paragraph{Co-products}

Here $L^{A}\triangleq M^{A}+N^{A}$, with some traversable functors
$M$ and $N$. We assume that $\text{seq}_{M}$ and $\text{seq}_{N}$
are already available and satisfy the laws, and define $\text{seq}_{L}$
as:
\[
\text{seq}_{L}:M^{F^{A}}+N^{F^{A}}\rightarrow F^{M^{A}+N^{A}}\quad,\quad\quad\text{seq}_{L}\triangleq\,\begin{array}{|c||c|}
 & F^{M^{A}+N^{A}}\\
\hline M^{F^{A}} & \text{seq}_{M}^{F,A}\bef(m^{:M^{A}}\rightarrow m+\bbnum 0^{:N^{A}})^{\uparrow F}\\
N^{F^{A}} & \text{seq}_{N}^{F,A}\bef(n^{:N^{A}}\rightarrow\bbnum 0^{:M^{A}}+n)^{\uparrow F}
\end{array}\quad.
\]

\begin{comment}
It helps to rewrite $\text{seq}_{L}$ as a composition of two matrices,
separating the functions lifted to $F$:
\begin{align*}
 & \text{seq}_{L}\triangleq\,\begin{array}{|c||cc|}
 & F^{M^{A}} & F^{N^{A}}\\
\hline M^{F^{A}} & \text{seq}_{M}^{F,A} & \bbnum 0\\
N^{F^{A}} & \bbnum 0 & \text{seq}_{N}^{F,A}
\end{array}\,\bef\,\begin{array}{|c||c|}
 & F^{M^{A}+N^{A}}\\
\hline F^{M^{A}} & (m^{:M^{A}}\rightarrow m+\bbnum 0^{:N^{A}})^{\uparrow F}\\
F^{N^{A}} & (n^{:N^{A}}\rightarrow\bbnum 0^{:M^{A}}+n)^{\uparrow F}
\end{array}\\
 & =\,\begin{array}{|c||cc|}
 & F^{M^{A}} & F^{N^{A}}\\
\hline M^{F^{A}} & \text{seq}_{M}^{F,A} & \bbnum 0\\
N^{F^{A}} & \bbnum 0 & \text{seq}_{N}^{F,A}
\end{array}\,\bef\,\begin{array}{|c||c|}
 & F^{M^{A}+N^{A}}\\
\hline F^{M^{A}} & (m^{:M^{A}}\rightarrow m+\bbnum 0^{:N^{A}})^{\uparrow F}\\
F^{N^{A}} & (n^{:N^{A}}\rightarrow\bbnum 0^{:M^{A}}+n)^{\uparrow F}
\end{array}\quad.
\end{align*}
\end{comment}
To verify the identity law~(\ref{eq:identity-law-of-sequence}):
\begin{align*}
 & \text{seq}_{L}^{\text{Id},A}=\,\begin{array}{|c||c|}
 & M^{A}+N^{A}\\
\hline M^{\text{Id}^{A}} & \text{seq}_{M}^{\text{Id},A}\bef(m^{:M^{A}}\rightarrow m+\bbnum 0^{:N^{A}})^{\uparrow\text{Id}}\\
N^{\text{Id}^{A}} & \text{seq}_{N}^{\text{Id},A}\bef(n^{:N^{A}}\rightarrow\bbnum 0^{:M^{A}}+n)^{\uparrow\text{Id}}
\end{array}\\
{\color{greenunder}\text{identity laws of }\text{seq}_{M}\text{ and }\text{seq}_{N}:}\quad & =\,\begin{array}{|c||c|}
 & M^{A}+N^{A}\\
\hline M^{A} & m^{:M^{A}}\rightarrow m+\bbnum 0^{:N^{A}}\\
N^{A} & n^{:N^{A}}\rightarrow\bbnum 0^{:M^{A}}+n
\end{array}\,=\,\begin{array}{|c||cc|}
 & M^{A} & N^{A}\\
\hline M^{A} & m\rightarrow m & \bbnum 0\\
N^{A} & \bbnum 0 & n\rightarrow n
\end{array}\,=\text{id}\quad.
\end{align*}

To verify the composition law~(\ref{eq:composition-law-of-sequence}),
write its two sides separately:
\begin{align*}
{\color{greenunder}\text{left-hand side}:}\quad & \text{seq}_{L}^{F,G^{A}}\bef(\text{seq}_{L}^{G,A})^{\uparrow F}=\,\begin{array}{||c|}
\text{seq}_{M}^{F,G^{A}}\bef(m^{:M^{G^{A}}}\rightarrow m+\bbnum 0^{:N^{G^{A}}})^{\uparrow F}\bef(\text{seq}_{L}^{G,A})^{\uparrow F}\\
\text{seq}_{N}^{F,G^{A}}\bef(n^{:N^{G^{A}}}\rightarrow\bbnum 0^{:M^{G^{A}}}+n)^{\uparrow F}\bef(\text{seq}_{L}^{G,A})^{\uparrow F}
\end{array}\quad,\\
{\color{greenunder}\text{right-hand side}:}\quad & \text{seq}_{L}^{F\circ G,A}=\,\begin{array}{||c|}
\text{seq}_{M}^{F\circ G,A}\bef(m^{:M^{A}}\rightarrow m+\bbnum 0^{:N^{A}})^{\uparrow G\uparrow F}\\
\text{seq}_{N}^{F\circ G,A}\bef(n^{:N^{A}}\rightarrow\bbnum 0^{:M^{A}}+n)^{\uparrow G\uparrow F}
\end{array}\quad.
\end{align*}
The left-hand side contains some function compositions lifted to $F$.
Write them separately:
\begin{align*}
 & (m^{:M^{G^{A}}}\rightarrow m+\bbnum 0^{:N^{G^{A}}})\bef\text{seq}_{L}^{G,A}=\,\begin{array}{|c||cc|}
 & M^{G^{A}} & N^{G^{A}}\\
\hline M^{G^{A}} & \text{id} & \bbnum 0
\end{array}\,\bef\,\begin{array}{|c||c|}
 & G^{M^{A}+N^{A}}\\
\hline M^{G^{A}} & \text{seq}_{M}^{G,A}\bef(m^{:M^{A}}\rightarrow m+\bbnum 0^{:N^{A}})^{\uparrow G}\\
N^{G^{A}} & \text{seq}_{N}^{G,A}\bef(n^{:N^{A}}\rightarrow\bbnum 0^{:M^{A}}+n)^{\uparrow G}
\end{array}\\
 & \quad=\text{seq}_{M}^{G,A}\bef(m^{:M^{A}}\rightarrow m+\bbnum 0^{:N^{A}})^{\uparrow G}\quad,\\
 & (n^{:N^{G^{A}}}\rightarrow\bbnum 0^{:M^{G^{A}}}+n)\bef\text{seq}_{L}^{G,A}=\,\begin{array}{|c||cc|}
 & M^{G^{A}} & N^{G^{A}}\\
\hline N^{G^{A}} & \bbnum 0 & \text{id}
\end{array}\,\bef\,\begin{array}{|c||c|}
 & G^{M^{A}+N^{A}}\\
\hline M^{G^{A}} & \text{seq}_{M}^{G,A}\bef(m^{:M^{A}}\rightarrow m+\bbnum 0^{:N^{A}})^{\uparrow G}\\
N^{G^{A}} & \text{seq}_{N}^{G,A}\bef(n^{:N^{A}}\rightarrow\bbnum 0^{:M^{A}}+n)^{\uparrow G}
\end{array}\\
 & \quad=\text{seq}_{N}^{G,A}\bef(n^{:N^{A}}\rightarrow\bbnum 0^{:M^{A}}+n)^{\uparrow G}\quad.
\end{align*}
Then simplify the left-hand side using the composition laws of $\text{seq}_{M}$
and $\text{seq}_{N}$:
\begin{align*}
 & \text{seq}_{L}^{F,G^{A}}\bef(\text{seq}_{L}^{G,A})^{\uparrow F}=\,\begin{array}{||c|}
\gunderline{\text{seq}_{M}^{F,G^{A}}\bef(\text{seq}_{M}^{G,A})^{\uparrow F}}\bef(m^{:M^{A}}\rightarrow m+\bbnum 0^{:N^{A}})^{\uparrow G\uparrow F}\\
\gunderline{\text{seq}_{N}^{F,G^{A}}\bef(\text{seq}_{N}^{G,A})^{\uparrow F}}\bef(n^{:N^{A}}\rightarrow\bbnum 0^{:M^{A}}+n)^{\uparrow G\uparrow F}
\end{array}\\
 & =\,\begin{array}{||c|}
\text{seq}_{M}^{F\circ G,A}\bef(m^{:M^{A}}\rightarrow m+\bbnum 0^{:N^{A}})^{\uparrow G\uparrow F}\\
\text{seq}_{N}^{F\circ G,A}\bef(n^{:N^{A}}\rightarrow\bbnum 0^{:M^{A}}+n)^{\uparrow G\uparrow F}
\end{array}\quad.
\end{align*}
The two sides are now equal.

\paragraph{Recursive types}

Here $L^{A}\triangleq S^{A,L^{A}}$, with a recursion scheme given
by a bifunctor\index{bifunctor} $S^{A,R}$. In order to obtain the
traversable property of $L$, we will need to assume that $S$ is
traversable with respect to both its type parameters in a special
way, which we call \textsf{``}bitraversable\textsf{''}. It is not enough if $S^{A,R}$
is traversable with respect to each type parameter separately.

A bifunctor $S^{A,B}$ is called \textbf{bitraversable}\index{bitraversable bifunctor}
if it has a \lstinline!bisequence! method (denoted by $\text{seq2}_{S}$):
\[
\text{seq2}_{S}^{F,A,B}:S^{F^{A},F^{B}}\rightarrow F^{S^{A,B}}\quad,
\]
which is parameterized by an arbitrary applicative functor $F$ and
is natural in $F$, $A$, and $B$. The traversable laws of identity\index{identity laws!of bisequence@of \texttt{bisequence}}
and composition\index{composition law!of bisequence@of \texttt{bisequence}}
must also hold for \lstinline!bisequence!:
\begin{align}
{\color{greenunder}\text{identity law of }\text{seq2}_{S}:}\quad & \text{seq2}_{S}^{\text{Id},A,B}=\text{id}^{:S^{A,B}\rightarrow S^{A,B}}\quad,\label{eq:identity-law-of-bisequence}\\
{\color{greenunder}\text{composition law of }\text{seq2}_{S}:}\quad & \text{seq2}_{S}^{F,G^{A},G^{B}}\bef\big(\text{seq2}_{S}^{G,A,B}\big)^{\uparrow F}=\text{seq2}_{S}^{F\circ G,A,B}\quad.\label{eq:composition-law-of-bisequence}
\end{align}

Section~\ref{subsec:All-polynomial-bifunctors-are-bitraversable}
will show that all polynomial bifunctors are bitraversable. So, we
are free to use any recursion scheme $S^{A,R}$ as long as $S$ is
a polynomial bifunctor. For now, we assume that a lawful $\text{seq2}_{S}$
is available and define $\text{seq}_{L}$ as:
\[
\text{seq}_{L}^{F,A}:S^{F^{A},L^{F^{A}}}\rightarrow F^{S^{A,L^{A}}}\quad,\quad\quad\text{seq}_{L}^{F,A}\triangleq\big(\overline{\text{seq}}_{L}^{F,A}\big)^{\uparrow S^{F^{A},\bullet}}\bef\text{seq2}_{S}^{F,A,L^{A}}\quad.
\]
This definition uses a recursive call to $\overline{\text{seq}}_{L}^{F,A}$
lifted with respect to the type parameter $R$ of $S^{A,R}$. The
inductive assumption is that the recursively called $\overline{\text{seq}}_{L}^{F,A}$
already obeys the laws we are proving.

To verify the identity law~(\ref{eq:identity-law-of-sequence}),
we use the assumed law~(\ref{eq:identity-law-of-bisequence}):
\[
\text{seq}_{L}^{\text{Id},A}=\big(\overline{\text{seq}}_{L}^{\text{Id},A}\big)^{\uparrow S^{\text{Id}^{A},\bullet}}\bef\text{seq2}_{S}^{\text{Id},A,L^{A}}=\text{id}^{\uparrow S^{A,\bullet}}\bef\text{id}=\text{id}\quad.
\]

To verify the composition law~(\ref{eq:composition-law-of-sequence}),
write its two sides separately:
\begin{align*}
 & \text{seq}_{L}^{F,G^{A}}\bef(\text{seq}_{L}^{G,A})^{\uparrow F}=\big(\overline{\text{seq}}_{L}^{F,G^{A}}\big)^{\uparrow S^{F^{G^{A}},\bullet}}\bef\text{seq2}_{S}^{F,G^{A},L^{G^{A}}}\bef\big(\overline{\text{seq}}_{L}^{G,A}\big)^{\uparrow S^{G^{A},\bullet}\uparrow F}\bef\big(\text{seq2}_{S}^{G,A,L^{A}}\big)^{\uparrow F}\quad,\\
 & \text{seq}_{L}^{F\circ G,A}=\big(\overline{\text{seq}}_{L}^{F\circ G,A}\big)^{\uparrow S^{F^{G^{A}},\bullet}}\bef\text{seq2}_{S}^{F\circ G,A,L^{A}}\quad.
\end{align*}
We use the naturality law of $\text{seq2}_{S}^{F,A,B}$ with respect
to the parameter $B$:
\[
\text{seq2}_{S}^{F,A,B}\bef(f^{:B\rightarrow C})^{\uparrow S^{A,\bullet}\uparrow F}=f^{\uparrow F\uparrow S^{F^{A},\bullet}}\bef\text{seq2}_{S}^{F,A,C}\quad.
\]
Then the left-hand side of the law~(\ref{eq:composition-law-of-sequence})
becomes:
\begin{align*}
 & \text{seq}_{L}^{F,G^{A}}\bef(\text{seq}_{L}^{G,A})^{\uparrow F}\\
 & =\gunderline{\big(\overline{\text{seq}}_{L}^{F,G^{A}}\big)^{\uparrow S^{F^{G^{A}},\bullet}}\bef\big(\overline{\text{seq}}_{L}^{G,A}\big)^{\uparrow F\uparrow S^{F^{G^{A}},\bullet}}}\bef\text{seq2}_{S}^{F,G^{A},G^{L^{A}}}\bef\big(\text{seq2}_{S}^{G,A,L^{A}}\big)^{\uparrow F}\\
{\color{greenunder}\text{composition law~(\ref{eq:composition-law-of-sequence})}:}\quad & =(\text{seq}_{L}^{F\circ G,A})^{\uparrow S^{F^{G^{A}},\bullet}}\bef\gunderline{\text{seq2}_{S}^{F,G^{A},G^{L^{A}}}\bef\big(\text{seq2}_{S}^{G,A,L^{A}}\big)^{\uparrow F}}\\
{\color{greenunder}\text{composition law~(\ref{eq:composition-law-of-bisequence})}:}\quad & =(\text{seq}_{L}^{F\circ G,A})^{\uparrow S^{F^{G^{A}},\bullet}}\bef\gunderline{\text{seq2}_{S}^{F\circ G,A,L^{A}}}\quad.
\end{align*}
The two sides of the law~(\ref{eq:composition-law-of-sequence})
are now equal. $\square$

These constructions allow us to implement a \lstinline!Traversable!
instance automatically for any polynomial functor. The corresponding
\lstinline!traverse! function will iterate in a naturally defined
order over all values stored in the functor. To visualize the way
\lstinline!traverse! works, write a polynomial functor $L$ as:
\[
L^{A}=Z_{0}+Z_{1}\times A+Z_{2}\times A\times A+...\quad,
\]
where $Z_{0}$, $Z_{1}$, ..., are fixed types. Any polynomial functor
may be equivalently rewritten in this way, perhaps with an infinite
number of parts in the disjunction. Each part of the disjunction is
written as $Z_{k}\times A\times...\times A$ and contains a finite
number $k$ of values of type $A$ and a value of a fixed type $Z_{k}$.
The computation \lstinline!traverse(f)!, where $f$ is a function
of type $A\rightarrow F^{B}$, iterates over the $k$ values of type
$A$ and merges the $F$-effects in the natural order using $F$\textsf{'}s
\lstinline!zip! operation. This creates a value of type $F^{B\times...\times B}$.
The constant value of type $Z_{k}$ is then lifted into $F$ to get
the final value of type $F^{Z_{k}\times B\times B\times...\times B}$.

If the traversal order needs to be chosen differently, we will need
to provide a custom implementation of a \lstinline!Traversable! instance.

Non-polynomial functors are, in general, not traversable. As an example,
consider the functors $L^{A}\triangleq E\rightarrow A$ and $F^{A}\triangleq Z+A$,
where the types $E$ and $Z$ are arbitrary but fixed. It is impossible
to implement a fully parametric function with the type signature of
\lstinline!sequence! ($L^{F^{B}}\rightarrow F^{L^{B}}$), which is
$(E\rightarrow Z+A)\rightarrow Z+(E\rightarrow A)$. To prove that
any traversable functor \emph{must} be polynomial, one needs advanced
methods beyond the scope of this book.\footnote{See the paper by R.~Bird\index{Richard Bird} et al.: \texttt{\href{http://www.cs.ox.ac.uk/jeremy.gibbons/publications/uitbaf.pdf}{http://www.cs.ox.ac.uk/jeremy.gibbons/publications/uitbaf.pdf}}}

\subsection{All polynomial bifunctors are bitraversable\label{subsec:All-polynomial-bifunctors-are-bitraversable}}

Bitraversable bifunctors are used in the recursive-type construction
for traversable functors. The properties of bitraversable bifunctors
are similar to the properties of traversable functors. We will now
prove that all polynomial bifunctors are bitraversable. The proof
verifies that each of the five type constructions of polynomial bifunctors
will produce a bitraversable bifunctor satisfying the laws~(\ref{eq:identity-law-of-bisequence})\textendash (\ref{eq:composition-law-of-bisequence}).

\paragraph{Fixed type}

If $S^{A,B}\triangleq Z$ where $Z$ is a fixed type, we define \lstinline!bisequence!
via $F$\textsf{'}s \lstinline!pure! method:
\[
\text{seq2}_{S}^{F,A,B}:S^{F^{A},F^{B}}\rightarrow F^{S^{A,B}}\cong Z\rightarrow F^{Z}\quad,\quad\quad\text{seq2}_{S}^{F,A,B}\triangleq\text{pu}_{F}^{:Z\rightarrow F^{Z}}\quad.
\]
The identity law~(\ref{eq:identity-law-of-bisequence}) holds since
$\text{seq2}_{S}^{\text{Id},A,B}=\text{pu}_{\text{Id}}=\text{id}$.
To verify the composition law~(\ref{eq:composition-law-of-bisequence}):
\begin{align*}
{\color{greenunder}\text{left-hand side}:}\quad & \text{seq2}_{S}^{F,G^{A},G^{B}}\bef(\text{seq2}_{S}^{G,A,B})^{\uparrow F}=\text{pu}_{F}\bef(\text{pu}_{G})^{\uparrow F}\\
{\color{greenunder}\text{naturality law of }\text{pu}_{F}:}\quad & \quad=\text{pu}_{G}\bef\text{pu}_{F}\quad,\\
{\color{greenunder}\text{right-hand side}:}\quad & \text{seq2}_{S}^{F\circ G,A,B}=\text{pu}_{F\circ G}\\
{\color{greenunder}\text{definition of }\text{pu}_{F\circ G}:}\quad & \quad=\text{pu}_{G}\bef\text{pu}_{F}\quad.
\end{align*}
The two sides of the law are now equal.

\paragraph{Type parameter}

If $S^{A,B}\triangleq A$ or $S^{A,B}\triangleq B$, we define \lstinline!bisequence!
as the identity function:
\begin{align*}
{\color{greenunder}\text{if }S^{A,B}\triangleq A:}\quad & \text{seq2}_{S}^{F,A,B}\triangleq\text{id}^{:F^{A}\rightarrow F^{A}}\quad,\\
{\color{greenunder}\text{if }S^{A,B}\triangleq B:}\quad & \text{seq2}_{S}^{F,A,B}\triangleq\text{id}^{:F^{B}\rightarrow F^{B}}\quad.
\end{align*}
Identity functions always satisfy identity and composition laws.

\paragraph{Products}

Here $S^{A,B}\triangleq M^{A,B}\times N^{A,B}$, with some bitraversable
bifunctors $M$ and $N$. We assume that $\text{seq2}_{M}$ and $\text{seq2}_{N}$
are already available and satisfy the laws, and define $\text{seq2}_{S}$
as:
\[
\text{seq2}_{S}^{F,A,B}:M^{F^{A},F^{B}}\times N^{F^{A},F^{B}}\rightarrow F^{M^{A,B}\times N^{A,B}}\quad,\quad\quad\text{seq2}_{S}^{F,A,B}\triangleq(\text{seq2}_{M}^{F,A,B}\boxtimes\text{seq2}_{N}^{F,A,B})\bef\text{zip}_{F}\quad.
\]

To verify the identity law~(\ref{eq:identity-law-of-bisequence}),
use the fact that $\text{zip}_{\text{Id}}=\text{id}$:
\begin{align*}
 & \text{seq2}_{S}^{\text{Id},A,B}=(\gunderline{\text{seq2}_{M}^{\text{Id},A,B}}\boxtimes\gunderline{\text{seq2}_{N}^{\text{Id},A,B}})\bef\text{zip}_{\text{Id}}\\
{\color{greenunder}\text{identity laws of }\text{seq2}_{M}\text{ and }\text{seq2}_{N}:}\quad & =(\text{id}\boxtimes\text{id})=\text{id}\quad.
\end{align*}

To verify the composition law~(\ref{eq:composition-law-of-bisequence}),
begin with its left-hand side:
\begin{align*}
{\color{greenunder}\text{left-hand side}:}\quad & \text{seq2}_{S}^{F,G^{A},G^{B}}\bef(\text{seq2}_{S}^{G,A,B})^{\uparrow F}\\
 & =(\text{seq2}_{M}^{F,G^{A},G^{B}}\boxtimes\text{seq2}_{N}^{F,G^{A},G^{B}})\bef\gunderline{\text{zip}_{F}\bef(\text{seq2}_{M}^{G,A,B}\boxtimes\text{seq2}_{N}^{G,A,B})^{\uparrow F}}\bef\text{zip}_{G}^{\uparrow F}\\
{\color{greenunder}\text{naturality of }\text{zip}_{F}:}\quad & =\gunderline{(\text{seq2}_{M}^{F,G^{A},G^{B}}\boxtimes\text{seq2}_{N}^{F,G^{A},G^{B}})\bef\big((\text{seq2}_{M}^{G,A,B})^{\uparrow F}\boxtimes(\text{seq2}_{N}^{G,A,B})^{\uparrow F}\big)}\bef\text{zip}_{F}\bef\text{zip}_{G}^{\uparrow F}\\
{\color{greenunder}\text{the law~(\ref{eq:pair-product-composition-law})}:}\quad & =\big((\text{seq2}_{M}^{F,G^{A},G^{B}}\bef(\text{seq2}_{M}^{G,A,B})^{\uparrow F})\boxtimes(\text{seq2}_{N}^{F,G^{A},G^{B}}\bef(\text{seq2}_{N}^{G,A,B})^{\uparrow F})\big)\bef\text{zip}_{F}\bef\text{zip}_{G}^{\uparrow F}\quad.
\end{align*}
By assumption, $\text{seq2}_{M}$ and $\text{seq2}_{N}$ satisfy their
composition laws. So, we may rewrite the last line as:
\[
\text{seq2}_{S}^{F,G^{A},G^{B}}\bef(\text{seq2}_{S}^{G,A,B})^{\uparrow F}=\big(\text{seq2}_{M}^{F\circ G,A,B}\boxtimes\text{seq2}_{N}^{F\circ G,A,B}\big)\bef\text{zip}_{F}\bef\text{zip}_{G}^{\uparrow F}\quad.
\]
We can now transform the right-hand side of the law to the same expression:
\begin{align*}
{\color{greenunder}\text{right-hand side}:}\quad & \text{seq2}_{S}^{F\circ G,A,B}=\big(\text{seq2}_{M}^{F\circ G,A,B}\boxtimes\text{seq2}_{N}^{F\circ G,A,B}\big)\bef\text{zip}_{F\circ G}\\
{\color{greenunder}\text{definition of }\text{zip}_{F\circ G}:}\quad & =\big(\text{seq2}_{M}^{F\circ G,A,B}\boxtimes\text{seq2}_{N}^{F\circ G,A,B}\big)\bef\text{zip}_{F}\bef\text{zip}_{G}^{\uparrow F}\quad.
\end{align*}


\paragraph{Co-products}

Here $S^{A,B}\triangleq M^{A,B}+N^{A,B}$, with some bitraversable
bifunctors $M$ and $N$. We assume that $\text{seq2}_{M}$ and $\text{seq2}_{N}$
are already available and satisfy the laws, and define $\text{seq2}_{S}$
as:
\[
\text{seq2}_{S}:M^{F^{A},F^{B}}+N^{F^{A},F^{B}}\rightarrow F^{M^{A,B}+N^{A,B}}\quad,\quad\text{seq2}_{S}\triangleq\,\begin{array}{|c||c|}
 & F^{M^{A,B}+N^{A,B}}\\
\hline M^{F^{A},F^{B}} & \text{seq2}_{M}^{F,A,B}\bef(m^{:M^{A,B}}\rightarrow m+\bbnum 0)^{\uparrow F}\\
N^{F^{A},F^{B}} & \text{seq2}_{N}^{F,A,B}\bef(n^{:N^{A,B}}\rightarrow\bbnum 0+n)^{\uparrow F}
\end{array}\quad.
\]

To verify the identity law~(\ref{eq:identity-law-of-bisequence}):
\begin{align*}
 & \text{seq2}_{S}^{\text{Id},A}=\,\begin{array}{|c||c|}
 & M^{A,B}+N^{A,B}\\
\hline M^{\text{Id}^{A},\text{Id}^{B}} & \text{seq2}_{M}^{\text{Id},A,B}\bef(m\rightarrow m+\bbnum 0)^{\uparrow\text{Id}}\\
N^{\text{Id}^{A}} & \text{seq2}_{N}^{\text{Id},A,B}\bef(n\rightarrow\bbnum 0+n)^{\uparrow\text{Id}}
\end{array}\\
{\color{greenunder}\text{identity laws of }\text{seq2}_{M}\text{ and }\text{seq2}_{N}:}\quad & =\,\begin{array}{|c||c|}
 & M^{A,B}+N^{A,B}\\
\hline M^{A,B} & m\rightarrow m+\bbnum 0\\
N^{A,B} & n\rightarrow\bbnum 0+n
\end{array}\,=\,\begin{array}{|c||cc|}
 & M^{A,B} & N^{A,B}\\
\hline M^{A,B} & m\rightarrow m & \bbnum 0\\
N^{A,B} & \bbnum 0 & n\rightarrow n
\end{array}\,=\text{id}\quad.
\end{align*}

To verify the composition law~(\ref{eq:composition-law-of-bisequence}),
write its two sides separately:
\begin{align*}
{\color{greenunder}\text{left-hand side}:}\quad & \text{seq2}_{S}^{F,G^{A},G^{B}}\bef(\text{seq2}_{S}^{G,A,B})^{\uparrow F}=\,\begin{array}{||c|}
\text{seq2}_{M}^{F,G^{A},G^{B}}\bef(m\rightarrow m+\bbnum 0)^{\uparrow F}\bef(\text{seq2}_{S}^{G,A,B})^{\uparrow F}\\
\text{seq2}_{N}^{F,G^{A},G^{B}}\bef(n\rightarrow\bbnum 0+n)^{\uparrow F}\bef(\text{seq2}_{S}^{G,A,B})^{\uparrow F}
\end{array}\quad,\\
{\color{greenunder}\text{right-hand side}:}\quad & \text{seq2}_{S}^{F\circ G,A,B}=\,\begin{array}{||c|}
\text{seq2}_{M}^{F\circ G,A,B}\bef(m\rightarrow m+\bbnum 0)^{\uparrow G\uparrow F}\\
\text{seq2}_{N}^{F\circ G,A,B}\bef(n\rightarrow\bbnum 0+n)^{\uparrow G\uparrow F}
\end{array}\quad.
\end{align*}
The left-hand side contains some function compositions lifted to $F$.
Write them separately:
\begin{align*}
 & (m\rightarrow m+\bbnum 0)\bef\text{seq2}_{S}^{G,A,B}=\,\begin{array}{|c||cc|}
 & M^{G^{A},G^{B}} & N^{G^{A},G^{B}}\\
\hline M^{G^{A},G^{B}} & \text{id} & \bbnum 0
\end{array}\,\bef\,\begin{array}{|c||c|}
 & G^{M^{A,B}+N^{A,B}}\\
\hline M^{G^{A},G^{B}} & \text{seq2}_{M}^{G,A,B}\bef(m\rightarrow m+\bbnum 0)^{\uparrow G}\\
N^{G^{A},G^{B}} & \text{seq2}_{N}^{G,A,B}\bef(n\rightarrow\bbnum 0+n)^{\uparrow G}
\end{array}\\
 & \quad=\text{seq2}_{M}^{G,A,B}\bef(m^{:M^{A,B}}\rightarrow m+\bbnum 0^{:N^{A,B}})^{\uparrow G}\quad,\\
 & (n\rightarrow\bbnum 0+n)\bef\text{seq2}_{S}^{G,A,B}=\,\begin{array}{|c||cc|}
 & M^{G^{A},G^{B}} & N^{G^{A},G^{B}}\\
\hline N^{G^{A},G^{B}} & \bbnum 0 & \text{id}
\end{array}\,\bef\,\begin{array}{|c||c|}
 & G^{M^{A,B}+N^{A,B}}\\
\hline M^{G^{A},G^{B}} & \text{seq2}_{M}^{G,A,B}\bef(m\rightarrow m+\bbnum 0)^{\uparrow G}\\
N^{G^{A},G^{B}} & \text{seq2}_{N}^{G,A,B}\bef(n\rightarrow\bbnum 0+n)^{\uparrow G}
\end{array}\\
 & \quad=\text{seq2}_{N}^{G,A,B}\bef(n^{:N^{A,B}}\rightarrow\bbnum 0^{:M^{A,B}}+n)^{\uparrow G}\quad.
\end{align*}
We then simplify the left-hand side using the composition laws of
$\text{seq2}_{M}$ and $\text{seq2}_{N}$:
\begin{align*}
 & \text{seq2}_{S}^{F,G^{A},G^{B}}\bef(\text{seq2}_{S}^{G,A,B})^{\uparrow F}=\,\begin{array}{||c|}
\gunderline{\text{seq2}_{M}^{F,G^{A},G^{B}}\bef(\text{seq2}_{M}^{G,A,B})^{\uparrow F}}\bef(m\rightarrow m+\bbnum 0)^{\uparrow G\uparrow F}\\
\gunderline{\text{seq2}_{N}^{F,G^{A},G^{B}}\bef(\text{seq2}_{N}^{G,A,B})^{\uparrow F}}\bef(n\rightarrow\bbnum 0+n)^{\uparrow G\uparrow F}
\end{array}\\
 & =\,\begin{array}{||c|}
\text{seq2}_{M}^{F\circ G,A,B}\bef(m\rightarrow m+\bbnum 0)^{\uparrow G\uparrow F}\\
\text{seq2}_{N}^{F\circ G,A,B}\bef(n\rightarrow\bbnum 0+n)^{\uparrow G\uparrow F}
\end{array}\quad.
\end{align*}
The two sides are now equal.

\paragraph{Recursive types}

Here $S^{A,B}\triangleq T^{A,B,S^{A,B}}$, with a recursion scheme
given by a \textbf{3-functor} \index{3@3-functor}$T^{A,B,R}$ (a
type constructor covariant in each of its three type parameters).
To show that $S$ is bitraversable, we need to assume that $T$ is
traversable with respect to all its type parameters at once, which
we call \textsf{``}3-traversable\textsf{''}. We say that a 3-functor $T^{A,B,R}$
is \textbf{3-traversable}\index{3@3-traversable 3-functor} if it
has a \lstinline!sequence3! method:
\[
\text{seq3}_{T}^{F,A,B,C}:T^{F^{A},F^{B},F^{C}}\rightarrow F^{T^{A,B,C}}\quad,
\]
which is parameterized by an arbitrary applicative functor $F$ and
is natural in $F$, $A$, $B$, and $C$. The traversable laws of
identity\index{identity laws!of sequence3@of \texttt{sequence3}}
and composition\index{composition law!of sequence3@of \texttt{sequence3}}
must also hold for \lstinline!sequence3!:
\begin{align}
{\color{greenunder}\text{identity law of }\text{seq3}_{T}:}\quad & \text{seq3}_{T}^{\text{Id},A,B,C}=\text{id}^{:T^{A,B,C}\rightarrow T^{A,B,C}}\quad,\label{eq:identity-law-of-trisequence}\\
{\color{greenunder}\text{composition law of }\text{seq3}_{T}:}\quad & \text{seq3}_{T}^{F,G^{A},G^{B},G^{C}}\bef\big(\text{seq3}_{T}^{G,A,B,C}\big)^{\uparrow F}=\text{seq3}_{T}^{F\circ G,A,B,C}\quad.\label{eq:composition-law-of-trisequence}
\end{align}

To prove that all polynomial 3-functors are 3-traversable, we would
need to repeat the proofs in this section with more type parameters.
The recursive type construction for 3-traversable 3-functors will
require a 4-traversable 4-functor as a recursion scheme. To prove
that all polynomial 4-functors are 4-traversable, we will need to
use 5-traversable 5-functors, and so on. Note that the proofs in this
section differ from the proofs in Section~\ref{subsec:All-polynomial-functors-are-traversable}
only in having more type parameters in the functions. So, the proofs
for 3-traversable 3-functors, 4-traversable 4-functors, etc., will
remain essentially the same as the proofs for bitraversable bifunctors
except for having more type parameters in each function. For this
reason, we omit those proofs. The conclusion is that all polynomial
$n$-functors are $n$-traversable for $n=1,2,3,...$

Assume that a lawful $\text{seq3}_{T}$ is available and define $\text{seq2}_{S}$
as:
\[
\text{seq2}_{S}^{F,A,B}:T^{F^{A},F^{B},S^{F^{A},F^{B}}}\rightarrow F^{T^{A,B,S^{A,B}}}\quad,\quad\quad\text{seq2}_{S}^{F,A,B}\triangleq\big(\overline{\text{seq2}}_{S}^{F,A,B}\big)^{\uparrow T^{F^{A},F^{B},\bullet}}\bef\text{seq3}_{T}^{F,A,B,S^{A,B}}\quad.
\]
The inductive assumption is that the recursively called $\overline{\text{seq2}}_{S}$
already obeys the laws we are proving.

To verify the identity law~(\ref{eq:identity-law-of-bisequence}),
we use the assumed law~(\ref{eq:identity-law-of-trisequence}):
\[
\text{seq2}_{S}^{\text{Id},A,B}=\big(\overline{\text{seq2}}_{S}^{\text{Id},A,B}\big)^{\uparrow T^{\text{Id}^{A},\text{Id}^{B},\bullet}}\bef\text{seq3}_{T}^{\text{Id},A,B,S^{A,B}}=\text{id}^{\uparrow T^{A,B,\bullet}}\bef\text{id}=\text{id}\quad.
\]

To verify the composition law~(\ref{eq:composition-law-of-bisequence}):
\begin{align*}
 & \text{seq2}_{S}^{F,G^{A},G^{B}}\bef(\text{seq2}_{S}^{G,A,B})^{\uparrow F}\\
 & \quad=\big(\overline{\text{seq2}}_{S}^{F,G^{A},G^{B}}\big)^{\uparrow T^{F^{G^{A}},F^{G^{B}},\bullet}}\bef\text{seq3}_{T}^{F,G^{A},G^{B},S^{G^{A},G^{B}}}\bef\big(\overline{\text{seq2}}_{S}^{G,A,B}\big)^{\uparrow T^{G^{A},G^{B},\bullet}\uparrow F}\bef\big(\text{seq3}_{T}^{G,A,B,S^{A,B}}\big)^{\uparrow F}\quad,\\
 & \text{seq2}_{S}^{F\circ G,A,B}=\big(\overline{\text{seq2}}_{S}^{F\circ G,A,B}\big)^{\uparrow T^{F^{G^{A}},F^{G^{B}},\bullet}}\bef\text{seq3}_{T}^{F\circ G,A,B,S^{A,B}}\quad.
\end{align*}
We use the naturality law of $\text{seq3}_{T}^{F,A,B,C}$ with respect
to the parameter $C$:
\[
\text{seq3}_{T}^{F,A,B,C}\bef(f^{:C\rightarrow D})^{\uparrow T^{A,B,\bullet}\uparrow F}=f^{\uparrow F\uparrow T^{F^{A},F^{B},\bullet}}\bef\text{seq3}_{T}^{F,A,B,D}\quad.
\]
Then the left-hand side of the law~(\ref{eq:composition-law-of-bisequence})
becomes:
\begin{align*}
 & \text{seq2}_{S}^{F,G^{A},G^{B}}\bef(\text{seq2}_{S}^{G,A,B})^{\uparrow F}\\
 & =\gunderline{\big(\overline{\text{seq2}}_{S}^{F,G^{A},G^{B}}\big)^{\uparrow T^{F^{G^{A}},F^{G^{B}},\bullet}}\bef\big(\overline{\text{seq2}}_{S}^{G,A,B}\big)^{\uparrow F\uparrow T^{F^{G^{A}},F^{G^{B}},\bullet}}}\bef\text{seq3}_{T}^{F,G^{A},G^{B},G^{S^{A,B}}}\bef\big(\text{seq3}_{T}^{G,A,B,S^{A,B}}\big)^{\uparrow F}\\
{\color{greenunder}\text{Eq.~(\ref{eq:identity-law-of-bisequence})}:}\quad & =(\text{seq2}_{S}^{F\circ G,A,B})^{\uparrow T^{F^{G^{A}},F^{G^{B}},\bullet}}\bef\gunderline{\text{seq3}_{T}^{F,G^{A},G^{B},G^{S^{A,B}}}\bef\big(\text{seq3}_{T}^{G,A,B,S^{A,B}}\big)^{\uparrow F}}\\
{\color{greenunder}\text{Eq.~(\ref{eq:composition-law-of-trisequence})}:}\quad & =(\text{seq2}_{S}^{F\circ G,A,B})^{\uparrow T^{F^{G^{A}},F^{G^{B}},\bullet}}\bef\gunderline{\text{seq3}_{T}^{F\circ G,A,B,S^{A,B}}}\quad.
\end{align*}
The two sides of the law~(\ref{eq:identity-law-of-bisequence}) are
now equal.

\subsection{Exercises\index{exercises}}

\subsubsection{Exercise \label{subsec:Exercise-traversables-1}\ref{subsec:Exercise-traversables-1}}

Show that any traversable functor $L$ admits a method called \lstinline!consume!:
\[
\text{consume}_{L}:(L^{A}\rightarrow B)\rightarrow L^{F^{A}}\rightarrow F^{B}\quad,
\]
defined for any applicative functor $F$. Assuming a suitable naturality
law for \lstinline!consume!, show that \lstinline!sequence! and
\lstinline!consume! are equivalent.

\subsubsection{Exercise \label{subsec:Exercise-traversables-3-1-1}\ref{subsec:Exercise-traversables-3-1-1}}

Consider the following implementation of \lstinline!traverse! for
the functor $L^{A}\triangleq\bbnum 1+A\times A$:
\begin{lstlisting}
def badtrav[A, B, F[_]: Applicative : Functor](f: A => F[B]): Option[(A, A)] => F[Option[(B, B)]] =
  Applicative[F].pure(None)
\end{lstlisting}
Show that \lstinline!badtrav! does \emph{not} satisfy the laws of
\lstinline!traverse!.

\subsubsection{Exercise \label{subsec:Exercise-traversables-3}\ref{subsec:Exercise-traversables-3}}

Show that the laws are \emph{not} satisfied by the implementation
of $\text{seq}:L^{F^{A}}\rightarrow F^{L^{A}}$ for $F^{A}\triangleq\bbnum 1+A$
where \lstinline!seq! always returns an empty option ($1+\bbnum 0$,
or \lstinline!None! in Scala).

\subsubsection{Exercise \label{subsec:Exercise-traversables-3-1}\ref{subsec:Exercise-traversables-3-1}}

For $L^{A}\triangleq A\times A$, consider $\text{seq}_{L}^{F,A}$
which collects the $F$-effects in the opposite order:
\begin{lstlisting}
def seq[F[_]: Applicative : Functor, A]: ((F[A], F[A])) => F[(A, A)] = {
  case (fa1, fa2) => (fa2 zip fa1).map(_.swap) // Use `swap` to restore the order of wrapped values.
}
\end{lstlisting}
In the point-free style, this code is written as:
\[
\text{seq}_{L}^{F,A}:F^{A}\times F^{A}\rightarrow F^{A\times A}\quad,\quad\quad\text{seq}_{L}^{F,A}\triangleq\text{swap}\bef\text{zip}_{F}\bef\text{swap}^{\uparrow F}\quad.
\]
Prove that this implementation of \lstinline!sequence! satisfies
its laws.

\subsubsection{Exercise \label{subsec:Exercise-traversables-laws}\ref{subsec:Exercise-traversables-laws}}

Find an example of an applicative functor $F$ and a traversable functor
$L$ such that one \emph{cannot} implement a natural transformation
with type signature $F^{L^{A}}\rightarrow L^{F^{A}}$ (the inverse
to the type signature of \lstinline!sequence!).

\subsubsection{Exercise \label{subsec:Exercise-traversables-laws-2}\ref{subsec:Exercise-traversables-laws-2}}

Show that the naturality law~(\ref{eq:sequence-naturality-law})
of \lstinline!sequence! is equivalent to the two naturality laws~(\ref{eq:naturality-laws-of-traverse})
of \lstinline!traverse!.

\subsubsection{Exercise \label{subsec:Exercise-traversables-laws-1}\ref{subsec:Exercise-traversables-laws-1}}

Statement~\ref{subsec:Statement-tr-equivalent-to-ftr} shows that
functions \lstinline!tr! and \lstinline!ftr! are equivalent if a
naturality law holds for \lstinline!ftr! with respect to the type
parameter $A$. Under the same assumption, show that the naturality
law of \lstinline!ftr! with respect to the type parameter $B$ is
equivalent to the naturality law of \lstinline!tr!.

\subsubsection{Exercise \label{subsec:Exercise-traversables-laws-1-1}\ref{subsec:Exercise-traversables-laws-1-1}}

Statement~\ref{subsec:Statement-foldleft-foldmap-equivalence}(a)
shows that functions \lstinline!foldMap! and \lstinline!reduceE!
are equivalent. Show that the monoidal naturality law holds for \lstinline!foldMap!
if it holds for \lstinline!reduceE!, and vice versa.

\subsubsection{Exercise \label{subsec:Exercise-traversables-5}\ref{subsec:Exercise-traversables-5}}

Prove that $L^{A}\triangleq M^{N^{A}}$ is a lawful traversable if
$M^{\bullet}$ and $N^{\bullet}$ are traversable functors.

\subsubsection{Exercise \label{subsec:Exercise-traversables-4}\ref{subsec:Exercise-traversables-4}}

Prove directly that all the bitraversable laws hold for the bifunctor
$S^{A,B}\triangleq A\times B$.

\subsubsection{Exercise \label{subsec:Exercise-traversables-6}\ref{subsec:Exercise-traversables-6}}

For the tree-like type defined as $T^{A}\triangleq\bbnum 1+A\times T^{A}\times T^{A}$,
define a \lstinline!Traversable! instance. Verify that the laws hold
by using a suitable recursion scheme $S^{A,R}$.

\subsubsection{Exercise \label{subsec:Exercise-traversables-9}\ref{subsec:Exercise-traversables-9}}

Is the recursive type constructor $L^{A}\triangleq A+L^{\text{List}^{A}}$
traversable? Explain via examples what sort of data container it is.

\subsubsection{Exercise \label{subsec:Exercise-traversables-10-2}\ref{subsec:Exercise-traversables-10-2}}

Prove that for any two monoid morphisms $\phi:M\rightarrow N$ and
$\psi:N\rightarrow P$ (where $M$, $N$, and $P$ are monoids), the
composition $\phi\bef\psi:M\rightarrow P$ is again a lawful monoid
morphism.

\subsubsection{Exercise \label{subsec:Exercise-traversables-10}\ref{subsec:Exercise-traversables-10}}

Given a \emph{monad} $M^{\bullet}$ and a monoid morphism $\phi:R\rightarrow S$
between some monoid types $R$ and $S$, prove that $\phi^{\uparrow M}:M^{R}\rightarrow M^{S}$
is also a monoid morphism. (The types $M^{R}$ and $M^{S}$ are monoids
due to Exercise~\ref{subsec:Exercise-monad-of-monoid-is-monoid}).

\subsubsection{Exercise \label{subsec:Exercise-traversables-10-1}\ref{subsec:Exercise-traversables-10-1}}

Given a monoid type $R$ and a \emph{monad} morphism $\phi:M\leadsto N$
between some monads $M$ and $N$, prove that $\phi:M^{R}\rightarrow N^{R}$
is a monoid morphism between $M^{R}$ and $N^{R}$. (The types $M^{R}$
and $N^{R}$ are monoids due to Exercise~\ref{subsec:Exercise-monad-of-monoid-is-monoid}).

\subsubsection{Exercise \label{subsec:Exercise-traversables-10-1-1-1}\ref{subsec:Exercise-traversables-10-1-1-1}}

Each monad is at the same time an applicative functor. Given a monad
morphism $\phi:M\leadsto N$ between two monads, show that $\phi$
is also an applicative morphism between applicative functors $M$
and $N$.

\subsubsection{Exercise \label{subsec:Exercise-traversables-10-3}\ref{subsec:Exercise-traversables-10-3}}

Given an applicative functor $M$ and a monoid morphism $\phi:R\rightarrow S$
between some monoid types $R$ and $S$, prove that $\phi^{\uparrow M}:M^{R}\rightarrow M^{S}$
is also a monoid morphism. (The types $M^{R}$ and $M^{S}$ are monoids
due to Exercise~\ref{subsec:Exercise-applicative-of-monoid-is-monoid}).

\subsubsection{Exercise \label{subsec:Exercise-traversables-10-1-1}\ref{subsec:Exercise-traversables-10-1-1}}

Given a monoid type $R$ and an applicative morphism $\phi:M\leadsto N$
between some applicative functors $M$ and $N$, prove that $\phi:M^{R}\rightarrow N^{R}$
is a monoid morphism between $M^{R}$ and $N^{R}$. (The types $M^{R}$
and $N^{R}$ are monoids due to Exercise~\ref{subsec:Exercise-applicative-of-monoid-is-monoid}).

\subsubsection{Exercise \label{subsec:Exercise-traversables-10-3-1}\ref{subsec:Exercise-traversables-10-3-1}}

Given a monoid $M$, define the functions \lstinline!inMF! and \lstinline!outMF!:
\begin{align*}
 & \text{inMF}:M\rightarrow M\rightarrow M\quad,\quad\quad\text{inMF}(x^{:M})\triangleq y^{:M}\rightarrow x\oplus y\quad,\\
 & \text{outMF}:(M\rightarrow M)\rightarrow M\quad,\quad\quad\text{outMF}(p^{:M\rightarrow M})\triangleq p(e_{M})\quad.
\end{align*}
This definition of \lstinline!inMF! is similar to that used in the
proof of Statement~\ref{subsec:Statement-foldleft-foldmap-equivalence}(c).

\textbf{(a)} Prove that \lstinline!inMF! is a monoid morphism between
$M$ and the monoid $\text{MF}^{M}$ consisting of all functions of
type $M\rightarrow M$. Define the empty element and the binary operation
of $\text{MF}^{M}$ appropriately.

\textbf{(b)} Prove that $\text{inMF}\bef\text{outMF}=\text{id}$ but
$\text{outMF}\bef\text{inMF}\neq\text{id}$. (Give an example of a
monoid $M$ where the second equation does not hold.)

\textbf{(c)} Prove that \lstinline!outMF! is \emph{not} a monoid
morphism between $\text{MF}^{M}$ and $M$.

\section{Discussion and further developments}

\subsection{Traversable contrafunctors or profunctors are not useful}

In Chapter~\ref{chap:8-Applicative-functors,-contrafunctors}, we
found some uses for applicative contrafunctors and profunctors. We
will now briefly investigate whether contrafunctors or profunctors
may be traversable. To answer that question, we will try implementing
a lawful \lstinline!sequence! function for contrafunctors and profunctors.

Suppose that $L^{A}$ is a contrafunctor. A \lstinline!sequence!
function with the type signature:
\[
\text{seq}_{L}^{F,A}:L^{F^{A}}\rightarrow F^{L^{A}}\quad,
\]
is required for $L$ to be traversable. That function must obey the
applicative naturality law, which ensures that $\text{seq}_{L}^{F,A}$
works in the same way for any applicative functor $F$. We note that
$\text{seq}_{L}$ is covariant in $F$ because $L^{\bullet}$ is contravariant.
Since $\text{pu}_{F}:\text{Id}^{A}\rightarrow F^{A}$ is an applicative
morphism (Example~\ref{subsec:Example-pure-is-applicative-morphism}),
we may use this morphism to write the applicative naturality law:
\[
\text{seq}_{L}^{F,A}=\text{pu}_{F}^{\downarrow L}\bef\text{seq}_{L}^{\text{Id},A}\bef\text{pu}_{F}\quad.
\]
The identity law gives $\text{seq}_{L}^{\text{Id},A}=\text{id}$.
So, the code of $\text{seq}_{L}^{F,A}$ must be this:
\[
\text{seq}_{L}^{F,A}=\text{pu}_{F}^{\downarrow L}\bef\text{pu}_{F}\quad.
\]
The implementation of $\text{seq}_{L}$ follows unambiguously from
the applicative naturality law.

We can now verify the composition law~(\ref{eq:composition-law-of-sequence}):
\begin{align*}
{\color{greenunder}\text{left-hand side}:}\quad & \text{seq}_{L}^{F,G^{A}}\bef(\text{seq}_{L}^{G,A})^{\uparrow F}=\text{pu}_{F}^{\downarrow L}\bef\gunderline{\text{pu}_{F}\bef(\text{pu}_{G}^{\downarrow L}\bef\text{pu}_{G})^{\uparrow F}}\\
{\color{greenunder}\text{naturality law of }\text{pu}_{F}:}\quad & \quad=\gunderline{\text{pu}_{F}^{\downarrow L}\bef\text{pu}_{G}^{\downarrow L}}\bef\text{pu}_{G}\bef\text{pu}_{F}=(\text{pu}_{G}\bef\text{pu}_{F})^{\downarrow L}\bef\text{pu}_{G}\bef\text{pu}_{F}\quad,\\
{\color{greenunder}\text{right-hand side}:}\quad & \text{seq}_{L}^{F\circ G,A}=\text{pu}_{F\circ G}^{\downarrow L}\bef\text{pu}_{F\circ G}=(\text{pu}_{G}\bef\text{pu}_{F})^{\downarrow L}\bef\text{pu}_{G}\bef\text{pu}_{F}\quad.
\end{align*}
So, every contrafunctor\textsf{'}s \lstinline!sequence! method satisfies the
laws of traversables.

However, the code of $\text{seq}_{L}^{F,A}$ always produces values
with empty $F$-effects. Any information about nontrivial $F$-effects
is ignored. For example, if $L^{A}\triangleq A\rightarrow Z$ (where
$Z$ is a fixed type) then:
\[
\text{seq}_{L}^{F,A}:(F^{A}\rightarrow Z)\rightarrow F^{A\rightarrow Z}\quad,\quad\quad\text{seq}_{L}^{F,A}(f^{:F^{A}\rightarrow Z})=\text{pu}_{F}(a^{:A}\rightarrow f(\text{pu}_{F}(a)))\quad.
\]
The function $f$ will be never applied to nontrivial $F$-effects.
So, the function $\text{seq}_{L}$ will never obtain any information
that $f$ would return when applied to nontrivial $F$-effects. We
conclude that contrafunctors are traversable in a way that does not
seem to be practically useful.

Turning now to the case where $L$ is a profunctor, consider a simple
example:
\[
L^{A}\triangleq A\rightarrow A\quad,\quad\quad\text{seq}_{L}:(F^{A}\rightarrow F^{A})\rightarrow F^{A\rightarrow A}\quad.
\]
Since the applicative functor $F$ is unknown and the type signature
of $\text{seq}_{L}$ does not provide any arguments of type $F^{X}$,
we cannot produce values of type $F$ other than via $\text{pu}_{F}$.
So, the only implementation of $\text{seq}_{L}$ is:
\[
\text{seq}_{L}\triangleq p^{:F^{A}\rightarrow F^{A}}\rightarrow\text{pu}_{F}(\text{id}^{:A\rightarrow A})\quad.
\]
This implementation ignores its argument $p$, losing information
and always returning an empty $F$-effect. As with contrafunctors,
we find that profunctors are not traversable in a useful way.

\subsection{Traversals for nested recursive types}

In most of this book, recursive type constructors are defined via
equations of the form $L^{A}\triangleq S^{A,L^{A}}$, where $S$ is
a recursion scheme. We have seen one example of a recursive type constructor
(a perfect-shaped tree, Sections~\ref{subsec:Perfect-shaped-trees}
and~\ref{subsec:Example-traversal-perfect-shaped-tree}) that cannot
be defined in this way. The reason is that the recursive type equation
for a perfect-shaped binary tree\index{perfect-shaped tree} $\text{PT}^{A}$
is:
\begin{equation}
\text{PT}^{A}\triangleq A+\text{PT}^{A\times A}\quad.\label{eq:perfect-shaped-binary-tree-type-equation}
\end{equation}
This type equation is not of the form $\text{PT}^{A}\triangleq S^{A,\text{PT}^{A}}$
because the recursive use of $\text{PT}^{\bullet}$ contains a nontrivial
type expression ($A\times A$) instead of just $A$. To express the
type equation~(\ref{eq:perfect-shaped-binary-tree-type-equation})
via a recursion scheme, we may introduce an additional functor $P$
and write:
\[
\text{PT}^{A}\triangleq S^{A,\text{PT}^{P^{A}}}\quad,\quad\quad P^{A}\triangleq A\times A\quad.
\]

\textbf{Nested recursive types}\index{nested recursive types} are
defined using type equations of this more general form, for example:
\begin{equation}
T^{A}\triangleq S^{A,T^{P^{A}}}\quad,\quad\text{or}\quad\quad T^{A,B}\triangleq S^{A,B,T^{P^{A},Q^{B}}}\quad,\quad\quad\text{etc.}\label{eq:nested-recursive-type-equations}
\end{equation}
The nested functors $P$, $Q$ replace the type parameters $A$, $B$
of $T^{A,B}$ by different type expressions. 

One could write a nested recursive type equation of an even more general
form, e.g., $T^{A}\triangleq S^{A,T^{T^{A}}}$, where the nested type
expressions may use the type $T$ itself more than once. We will not
use such type equations in this book. The nested recursive types of
the form~(\ref{eq:nested-recursive-type-equations}) are sufficient
in practice.

Let us see how one can implement a \lstinline!Traversable! instance
for type constructors of the form~(\ref{eq:nested-recursive-type-equations}).

\subsubsection{Statement \label{subsec:Statement-nested-recursive-type-traversable}\ref{subsec:Statement-nested-recursive-type-traversable}}

Given a bitraversable bifunctor $S^{\bullet,\bullet}$ and a traversable
functor $P^{\bullet}$, define a nested recursive type constructor
$L^{\bullet}$ by:
\[
L^{A}\triangleq S^{A,L^{P^{A}}}\quad.
\]
Then $L$ is a lawful traversable functor with the \lstinline!sequence!
function defined as:
\[
\text{seq}_{L}^{F,A}:S^{F^{A},L^{P^{F^{A}}}}\rightarrow F^{S^{A,L^{P^{A}}}}\quad,\quad\quad\text{seq}_{L}^{F,A}\triangleq\big(\text{seq}_{P}^{F,A}\big)^{\uparrow L\uparrow S^{F^{A},\bullet}}\bef\big(\overline{\text{seq}}_{L}^{F,P^{A}}\big)^{\uparrow S^{F^{A},\bullet}}\bef\text{seq2}_{S}^{F,A,L^{P^{A}}}\quad.
\]
\vspace{-1\baselineskip}
\[
\xymatrix{\xyScaleY{1.4pc}\xyScaleX{6pc}S^{F^{A},L^{P^{F^{A}}}}\ar[r]\sp(0.5){\big(\text{seq}_{P}^{F,A}\big)^{\uparrow L\uparrow S^{F^{A},\bullet}}} & S^{F^{A},L^{F^{P^{A}}}}\ar[r]\sp(0.5){\big(\overline{\text{seq}}_{L}^{F,P^{A}}\big)^{\uparrow S^{F^{A},\bullet}}} & S^{F^{A},F^{L^{P^{A}}}}\ar[r]\sp(0.5){\text{seq2}_{S}^{F,A,L^{P^{A}}}} & F^{S^{A,L^{P^{A}}}}}
\]
\vspace{-1\baselineskip}


\subparagraph{Proof}

We assume that the methods $\text{seq}_{P}$ and $\text{seq2}_{S}$,
as well as the recursively called $\overline{\text{seq}}_{L}$, already
satisfy the laws. To verify the identity law~(\ref{eq:identity-law-of-sequence}):
\[
\text{seq}_{L}^{\text{Id},A}=\big(\text{seq}_{P}^{\text{Id},A}\big)^{\uparrow L\uparrow S^{F^{A},\bullet}}\bef\big(\overline{\text{seq}}_{L}^{\text{Id},P^{A}}\big)^{\uparrow S^{F^{A},\bullet}}\bef\text{seq2}_{S}^{\text{Id},A,L^{P^{A}}}=\text{id}^{\uparrow L\uparrow S}\bef\text{id}^{\uparrow S}\bef\text{id}=\text{id}\quad.
\]
To verify the composition law~(\ref{eq:composition-law-of-sequence}),
write its two sides, omitting some type parameters for brevity:
\begin{align*}
 & \text{seq}_{L}^{F,G^{A}}\bef(\text{seq}_{L}^{G,A})^{\uparrow F}\\
 & \quad=\big(\text{seq}_{P}^{F,G^{A}}\big)^{\uparrow L\uparrow S}\bef\big(\overline{\text{seq}}_{L}^{F,P^{G^{A}}}\big)^{\uparrow S}\bef\text{seq2}_{S}^{F,G^{A},L^{P^{G^{A}}}}\bef\big((\text{seq}_{P}^{G,A})^{\uparrow L\uparrow S}\bef\big(\overline{\text{seq}}_{L}^{G,P^{A}}\big)^{\uparrow S}\bef\text{seq2}_{S}^{G,A,L^{P^{A}}}\big)^{\uparrow F}\quad,\\
 & \text{seq}_{L}^{F\circ G,A}=\big(\text{seq}_{P}^{F\circ G,A}\big)^{\uparrow L\uparrow S}\bef\big(\overline{\text{seq}}_{L}^{F\circ G,P^{A}}\big)^{\uparrow S}\bef\text{seq2}_{S}^{F\circ G,A,L^{P^{A}}}\\
 & \quad=\big(\text{seq}_{P}^{F,G^{A}}\bef(\text{seq}_{P}^{G,A})^{\uparrow F}\big)^{\uparrow L\uparrow S}\bef\big(\overline{\text{seq}}_{L}^{F,G^{P^{A}}}\bef(\overline{\text{seq}}_{L}^{G,P^{A}})^{\uparrow F}\big)^{\uparrow S}\bef\text{seq2}_{S}^{F,G^{A},G^{L^{P^{A}}}}\bef\big(\text{seq2}_{S}^{G,A,L^{P^{A}}}\big)^{\uparrow F}\quad.
\end{align*}
To get the last line, we used the assumed composition laws of $\text{seq}_{P}$,
$\overline{\text{seq}}_{L}$, and $\text{seq2}_{S}$. 

It is now clear that the two sides of the law differ only in the order
of function compositions. The remaining difference is:
\begin{align*}
 & \big(\overline{\text{seq}}_{L}^{F,P^{G^{A}}}\big)^{\uparrow S}\bef\text{seq2}_{S}^{F,G^{A},L^{P^{G^{A}}}}\bef\big((\text{seq}_{P}^{G,A})^{\uparrow L}\bef\overline{\text{seq}}_{L}^{G,P^{A}}\big)^{\uparrow S\uparrow F}\\
 & \overset{?}{=}(\text{seq}_{P}^{G,A})^{\uparrow F\uparrow L\uparrow S}\bef\big(\overline{\text{seq}}_{L}^{F,G^{P^{A}}}\big)^{\uparrow S}\bef(\overline{\text{seq}}_{L}^{G,P^{A}})^{\uparrow F\uparrow S}\bef\text{seq2}_{S}^{F,G^{A},G^{L^{P^{A}}}}\quad.
\end{align*}
We will show that the two sides are equal if we rewrite the left-hand
side so that the various \lstinline!sequence! methods are composed
in the same order as in the right-hand side. This can be done using
naturality laws, which allow us to change the order of composition
of lifted functions:
\begin{align*}
{\color{greenunder}\text{left-hand side}:}\quad & \big(\overline{\text{seq}}_{L}^{F,P^{G^{A}}}\big)^{\uparrow S}\bef\gunderline{\text{seq2}_{S}^{F,G^{A},L^{P^{G^{A}}}}\bef\big((\text{seq}_{P}^{G,A})^{\uparrow L}\bef\overline{\text{seq}}_{L}^{G,P^{A}}\big)^{\uparrow S\uparrow F}}\\
{\color{greenunder}\text{naturality law of }\text{seq2}_{S}:}\quad & =\gunderline{\big(\overline{\text{seq}}_{L}^{F,P^{G^{A}}}\big)^{\uparrow S}\bef\big((\text{seq}_{P}^{G,A})^{\uparrow L}\bef\overline{\text{seq}}_{L}^{G,P^{A}}\big)^{\uparrow F\uparrow S}}\bef\text{seq2}_{S}^{F,G^{A},G^{L^{P^{A}}}}\\
{\color{greenunder}\text{composition under }^{\uparrow S}:}\quad & =\big(\gunderline{\overline{\text{seq}}_{L}^{F,P^{G^{A}}}\bef(\text{seq}_{P}^{G,A})^{\uparrow L\uparrow F}}\big)^{\uparrow S}\bef\big(\overline{\text{seq}}_{L}^{G,P^{A}}\big)^{\uparrow F\uparrow S}\bef\text{seq2}_{S}^{F,G^{A},G^{L^{P^{A}}}}\quad.\\
{\color{greenunder}\text{naturality law of }\overline{\text{seq}}_{L}:}\quad & =\big(\overline{\text{seq}}_{L}^{F,P^{G^{A}}}\bef(\text{seq}_{P}^{G,A})^{\uparrow L\uparrow F}\big)^{\uparrow S}\bef\big(\overline{\text{seq}}_{L}^{G,P^{A}}\big)^{\uparrow F\uparrow S}\bef\text{seq2}_{S}^{F,G^{A},G^{L^{P^{A}}}}
\end{align*}
It remains to show that: 
\[
\big(\overline{\text{seq}}_{L}^{F,P^{G^{A}}}\bef(\text{seq}_{P}^{G,A})^{\uparrow L\uparrow F}\big)^{\uparrow S}\overset{?}{=}(\text{seq}_{P}^{G,A})^{\uparrow F\uparrow L\uparrow S}\bef\big(\overline{\text{seq}}_{L}^{F,G^{P^{A}}}\big)^{\uparrow S}\quad.
\]
But this is just the naturality law of $\overline{\text{seq}}_{L}$
under the lifting $(\dots)^{\uparrow S}$. $\square$

As an advanced example of a nested traversable functor, we will derive
a type for square-shaped matrices\footnote{This and other advanced examples of nested recursive types are explained
in the paper \textsf{``}Manufacturing datatypes\textsf{''} (1999) by R.~Hinze\index{Ralf Hinze},
see \texttt{\href{https://www.cs.ox.ac.uk/ralf.hinze/publications/WAAAPL99a.ps.gz}{https://www.cs.ox.ac.uk/ralf.hinze/publications/WAAAPL99a.ps.gz}}} with elements of type $A$. For motivation, recall how Example~\ref{subsec:Example-matrix-products}
encoded square matrices via nested lists of type \lstinline!List[List[A]]!.
However, a value of that type is not guaranteed to represent a matrix
of a consistent shape. We would like to define a type \lstinline!Sq[A]!
whose values always contain $n$ nested lists of length $n$ (for
$n=1,2,...$). We need to disallow inconsistent nested lists such
as $\left[\left[1,2\right],\left[\right],\left[3\right]\right]$ that
do not correspond to a square matrix.

Begin by considering an example of a $2\times2$ matrix. The type
of such matrices is not \lstinline!List[List[A]]! but \lstinline!List2[List2[A]]!,
where \lstinline!List2[A]! is a list of \emph{exactly} $2$ elements
of type $A$. We can rewrite the type \lstinline!List2[A]! equivalently
as a pair $A\times A$ and also as a function $\bbnum 2\rightarrow A$
(where $\bbnum 2$ is the type containing exactly $2$ distinct values).
This technique allows us to formulate the types of lists of exactly
$3$, $4$, etc.,~elements as $\text{List}_{3}^{A}\triangleq\bbnum 3\rightarrow A$,
$\text{List}_{4}^{A}\triangleq\bbnum 4\rightarrow A$, and so on.
The type of an $n\times n$ matrix is then written symbolically as
$\bbnum n\rightarrow\bbnum n\rightarrow A$ (where $\bbnum n=\bbnum 1$,
$\bbnum 2$, etc.) or equivalently as $\bbnum n\times\bbnum n\rightarrow A$.

The type \lstinline!Sq[A]! is equivalent to an infinite disjunction
of types representing square matrices of every size ($1\times1$,
$2\times2$, and so on). To define an infinite disjunctive type in
a program of finite size, we need to use recursion at type level.
In a mathematical sense, this recursion will be induction on the size
of the matrix. So, let us introduce the size of the matrix as an extra
\emph{type parameter} $N$. It will be convenient to define \lstinline!SqSize[N, A]!
as the type of matrices of size \emph{at least} $N$. We intend $N$
to be equivalent to one of the types $\bbnum 1$, $\bbnum 2$, etc. 

The base case ($N=\bbnum 1$) and the inductive step (from $N$ to
$\bbnum 1+N$) are written as:
\[
\text{Sq}^{A}\triangleq\text{SqSize}^{\bbnum 1,A}\quad,\quad\quad\text{SqSize}^{N,A}\triangleq\left(N\times N\rightarrow A\right)+\text{SqSize}^{\bbnum 1+N,A}\quad.
\]
To implement this in Scala, we first define \lstinline!SqSize[N, A]!
as a disjunctive type and then define \lstinline!Sq[A]!:
\begin{lstlisting}
sealed trait SqSize[N, A]
final case class Matrix[N, A](byIndex: ((N, N)) => A) extends SqSize[N, A]
final case class Next[N, A](next: SqSize[Option[N], A])  extends SqSize[N, A]

type Sq[A] = SqSize[Unit, A]
\end{lstlisting}

As an example, the $2\times2$ matrix $\,\begin{array}{|cc|}
11 & 12\\
21 & 22
\end{array}\,$ is represented by a value of type \lstinline!Sq[Int]! as:\vspace{0.2\baselineskip}

\begin{lstlisting}
val matrix2x2: Sq[Int] = Next(Matrix {
  case (None, None)         => 11
  case (None, Some(_))      => 12
  case (Some(_), None)      => 21
  case (Some(_), Some(_))   => 22
})
\end{lstlisting}

A $3\times3$ matrix will have the form \lstinline!Next(Next(Matrix { ... }))!.
An $n\times n$ matrix has $\left(n-1\right)$ nested \lstinline!Next!
constructors in front of a \lstinline!Matrix! constructor. The size
of a matrix is encoded by its type in this way.

We now implement a \lstinline!sequence! function for \lstinline!Sq!.
Since \lstinline!Sq! is defined by induction using \lstinline!SqSize[N, A]!,
we must first implement \lstinline!sequence! for \lstinline!SqSize!:
\begin{lstlisting}
def sequence[A, N, F[_]: Applicative: Functor]: SqSize[N, F[A]] => F[SqSize[N, A]] = {
  case Matrix(byIndex)   => ???    // Base case.
  case Next(next)        => ???    // Inductive step.
}
\end{lstlisting}
In the base case, we have a function \lstinline!byIndex! of type
$N\times N\rightarrow F^{A}$ that can return $n\times n$ different
values of type $F^{A}$. We need to combine all those values together
by using $F$\textsf{'}s \lstinline!zip! method. The result will be a value
of type $F^{A\times A\times...\times A}$, which we will then need
to convert to the type $F^{N\times N\rightarrow A}$. The only way
of performing these computations is by enumerating all possible values
of type $N$.

\begin{figure}
\begin{centering}
\begin{lstlisting}[frame=single,fillcolor={\color{black}},framesep={0.2mm},framexleftmargin=2mm,framexrightmargin=2mm,framextopmargin=2mm,framexbottommargin=2mm]
type Finite[N] = List[N] // A list of all possible values of type N.

object Finite { def apply[N: Finite]: Finite[N] = implicitly[Finite[N]] }

sealed abstract class SqSize[N: Finite, A]

final case class Matrix[N: Finite, A](byIndex: ((N, N)) => A) extends SqSize[N, A]

final case class Next[N: Finite, A](next: SqSize[Option[N], A]) extends SqSize[N, A]

type Sq[A] = SqSize[Unit, A]

implicit val finiteUnit: Finite[Unit] = List(())

implicit def finiteOptionN[N: Finite]: Finite[Option[N]] = None +: Finite[N].map(Some(_))

// Access the matrix element at zero-based index (i, j).
def access[N: Finite, A](s: SqSize[N, A], i: Int, j: Int): A = s match {
  case Matrix(byIndex) => byIndex((Finite[N].apply(i), Finite[N].apply(j)))
  case Next(next) => access[Option[N], A](next, i, j)
}

def size[N: Finite, A](s: SqSize[N, A]): Int = s match {      
  case Matrix(_) => Finite[N].length
  case Next(next) => size[Option[N], A](next)
}

def toSeqSeq[N: Finite, A](s: SqSize[N, A]): Seq[Seq[A]] = {
  val length = size(s)
  (0 until length).map(i => (0 until length).map(j => access(s, i, j)))
} 

// Test: visualize the matrix defined previously by converting it to nested lists.
scala> toSeqSeq(matrix2x2)
res1: List[List[Int]] = List(List(11, 12), List(21, 22))

def sequenceList[F[_]: Applicative : Functor, A](l: List[F[A]]): F[List[A]] = l match {
  case Nil          => Applicative[F].pure(Nil)
  case head :: tail => (head zip sequenceList(tail)).map { case (x, y) => x +: y }
}

def sequence[N: Finite, F[_]: Applicative : Functor, A](sq: SqSize[N, F[A]]): F[SqSize[N, A]] =
  sq match {
    case Matrix(byIndex)   =>
      val allValuesF: List[F[((N, N), A)]] = for {
        i <- Finite[N]
        j <- Finite[N]
      } yield byIndex((i, j)).map(a => ((i, j), a))

      val fList: F[List[((N, N), A)]] = sequenceList(allValuesF)
      fList.map { values =>
        val valuesMap: ((N, N)) => A = values.toMap.apply
        Matrix[N, A](valuesMap)
      }
    case Next(next)        => sequence[Option[N], F, A](next).map(Next(_))
  }
\end{lstlisting}
\par\end{centering}
\caption{Implementing \lstinline!sequence! for square matrices.\label{fig:Full-code-implementing-traverse-for-square-matrix}}
\end{figure}

Note that the type \lstinline!Sq! sets the type parameter \lstinline!N!
in \lstinline!SqSize[N, A]! as \lstinline!N = Unit!. This forces
the type parameters \lstinline!N! in all of the \lstinline!Next()!
constructors to be \lstinline!Unit! wrapped in a number of \lstinline!Option!
constructors. All values of a type of this form can be enumerated
explicitly. However, the type \lstinline!SqSize[N, A]! does not ensure
that \lstinline!N! will be a type with a known finite number of values.
This problem prevents us from implementing the \lstinline!sequence!
method for our current definition of \lstinline!Sq!.

A solution is to add a typeclass constraint (with a typeclass called
\textsf{``}\lstinline!Finite!\textsf{''}) on the type parameter \lstinline!N!. A
suitable typeclass instance of \lstinline!Finite[N]! contains a list
of all values of type \lstinline!N!:
\begin{lstlisting}
type Finite[N] = List[N] // A list of all possible values of type N.
\end{lstlisting}
We can implement functions that create typeclass instances automatically
for all the types we will actually use instead of the type parameter
\lstinline!N!, namely, the types \lstinline!Unit!, \lstinline!Option[Unit]!,
\lstinline!Option[Option[Unit]]! and so on. Suitable typeclass instances
are defined inductively:
\begin{lstlisting}
implicit val finiteUnit: Finite[Unit] = List(())
implicit def finiteOptionN[N: Finite]: Finite[Option[N]] = None +: Finite[N].map(Some(_))
\end{lstlisting}

Using these definitions, we can now extract all values of type $A$
from a value of type $N\times N\rightarrow A$:
\begin{lstlisting}
def access[N: Finite, A](s: SqSize[N, A], i: Int, j: Int): A = s match {
  case Matrix(byIndex) => byIndex((Finite[N].apply(i), Finite[N].apply(j)))
  case Next(next) => access[Option[N], A](next, i, j)
}
\end{lstlisting}
Let us test this code:
\begin{lstlisting}
scala> access(matrix2x2, 0, 1)
res0: Int = 12
\end{lstlisting}

Figure~\ref{fig:Full-code-implementing-traverse-for-square-matrix}
shows the complete code of \lstinline!sequence! for \lstinline!Sq!.
For testing (Figure~\ref{fig:Full-code-implementing-traverse-for-square-matrix-tests}),
we define a value \lstinline!matrix2x2List! that represents a square
matrix of lists:
\[
\text{matrix2x2List}:\text{Sq}^{\text{List}^{\text{Int}}}\quad,\quad\quad\text{matrix2x2List}\triangleq\left|\begin{array}{cc}
\left[0,10,100\right] & \left[1,11,101\right]\\
\left[2,12,102\right] & \left[3,13,103\right]
\end{array}\right|\quad.
\]
 Applying \lstinline!sequence! to that value, we obtain a list of
$3$ square matrices:
\[
\text{seq}_{\text{Sq}}\,(\text{matrix2x2List})=\big[\left|\begin{array}{cc}
0 & 1\\
2 & 3
\end{array}\right|,\left|\begin{array}{cc}
10 & 11\\
12 & 13
\end{array}\right|,\left|\begin{array}{cc}
100 & 101\\
102 & 103
\end{array}\right|\big]\quad.
\]
This represents a kind of transposition operation for tensors of dimension
$2\times2\times3$.

\begin{figure}
\begin{centering}
\begin{lstlisting}[frame=single,fillcolor={\color{black}},framesep={0.2mm},framexleftmargin=2mm,framexrightmargin=2mm,framextopmargin=2mm,framexbottommargin=2mm]
// Test: use List as an Applicative Functor.
implicit val applicativeList: Applicative[List] = new Applicative[List] {
  override def pure[A](a: A): List[A] = List(a)
  override def zip[A, B](fa: List[A], fb: List[B]): List[(A, B)] = fa zip fb     }

implicit val functorList: Functor[List] = new Functor[List] {
  override def map[A, B](fa: List[A])(f: A => B): List[B] = fa map f
}

// Test value of type Sq[List[Int]]:
val matrix2x2List: Sq[List[Int]] = Next(Matrix {
  case (None, None)         => List(0, 10, 100)
  case (None, Some(_))      => List(1, 11, 101)
  case (Some(_), None)      => List(2, 12, 102)
  case (Some(_), Some(_))   => List(3, 13, 103)
})

// Apply `sequence` to the test value. The result has type List[Sq[Int]].
val list2x2Matrix: List[Sq[Int]] = sequence(matrix2x2List)

// Visualize the result by converting it to nested lists.

scala> toSeqSeq(list2x2Matrix)
res1: List[List[List[Int]]] = List(List(List(0, 1), List(2, 3)), List(List(10, 11), List(12, 13)), List(List(100, 101), List(102, 103))) 
\end{lstlisting}
\par\end{centering}
\caption{Tests for the \lstinline!sequence! method implemented in Figure~\ref{fig:Full-code-implementing-traverse-for-square-matrix}.\label{fig:Full-code-implementing-traverse-for-square-matrix-tests}}
\end{figure}

The type \lstinline!Sq[A]! assures (at compile time) that all matrices
have consistent shapes. However, it is hard to use because of complicated
type parameters and deeply nested type constructors. To achieve good
performance, square matrices and other tensor-like quantities are
usually represented by flat arrays with an interface that recalculates
the indices. When the dimensions of the matrices are known at compile
time, one could use macros or other metaprogramming features to assure
that all matrix operations are consistent, without resorting to complicated
type constructors. Alternatively, one can use dependent types to constrain
matrix dimensions at compile time.\footnote{See, for example, the \texttt{NDScala} library for Scala 3: \href{https://github.com/SciScala/NDScala}{https://github.com/SciScala/NDScala}}
\begin{comment}
this is chapter 9 of the functional programming tutorial traversable
functors to motivate the interrupted introduction of these factors
into practice I always remember the example that you have a list of
some data items and you want to process it by using a function like
this where you have a future as a as a result of this function and
the usual way of doing this in Scala is to use a function called future
dot sequence and I have seen this I have shown this in a previous
tutorial and the idea is that you want to process each element of
this list with this function and you have to wait until the entire
list is done so we have many separate computations for each element
of the list encapsulated by the future for each one of them you want
to wait until the entire list is done and basically this is the type
signature that you want in order to implement this computation you
have a list of a you have a function from a to future B and you want
to get a list of B as a result and you can get it in the future so
you have a future of lists of it as a result and that is the type
signature that the function future that sequence will allow you to
have with some work but in order to understand what this kind of computation
does we want to generalize from the future to an arbitrary type constructor
F and we want to understand what properties these type constructors
must answer so the list we have L and instead the future we have F
and the type signature of the function is like this so this function
is called Traverse I believe there\textsf{'}s also a future of traders with
a type signature like this that works on sequences we want to generalize
to some type constructors F and L and that\textsf{'}s what we will be able
to do once we understand the properties of this operation so this
operation can be implemented for instance if L is this type constructor
then what can we do in order to implement this operation well clearly
we have an LA and we can apply map F so f is this function we can
apply map F and the result will be this now in other words we have
not what we want probably wanders F lb Inc which would be this so
how can we get F of a triple from a triple of F\textsf{'}s or clearly we need
an operation that\textsf{'}s similar to zip zip would be FB times F be going
to F of B times B now we need to apply zip twice and then we get from
here to here so once we have the zip we will be able to implement
this traverse operation that\textsf{'}s the conclusion so far so it seems that
F needs to be a quick ative in order to be able to implement this
type signature for at least for this type constructor so for this
type constructor certainly it is easy to implement if F is if it has
a zip operation then we can implement it like this so that is going
to be a fundamental assumption not for the traverse operation to make
sense the type constructor F must be implicative now the type constructor
L on the other hand doesn't we don't know what that is it could be
applicative or not maybe we'll find out but for now let\textsf{'}s what say
L is traversable if it has this operation now in Scala we have a very
limited version of this Traverse which assumes L to be a sequence
so it\textsf{'}s not based on this idea of being traversable as such it\textsf{'}s just
that one of the properties of a sequence is that you can implement
this operation and we'll see why that is so but for me the example
with lists and futures or sequences of futures is the easy to remember
example that helps me remember the requirements for the traversal
duration so always think that I have a list of some items and the
processing makes me a future so L is a list F the future and then
it is clear that I want this kind of type signature I want to have
a single future and when that future completes I want to have the
entire list of process data that\textsf{'}s the easy to remember example so
not not the other way around for example not a list of future they're
not interested in having a list of futures I want a single future
with the final list of the results and so that\textsf{'}s why this example
helps me remember this somewhat complicated type signature where I
could easily make a mistake they will a to L of B today to have fun
being LF being instead of X well being it\textsf{'}s easy to mix them up but
so remembering this example that I'm starting with that what helps
me remember the time signature of triggers so the questions that I'm
going to answer in this tutorial in this chapter are to find out what
factors L can have this triggers operation to find out if we can simplify
with somewhat complicated type signature can we express it perhaps
through a simpler operation what are the laws that is reasonable to
require for this operation and finally to look at contractors and
pro fungus do they have also some kind of analog of this operation
in previous tutorials are started right away with practical examples
of usage in this chapter I will first do more theory to understand
in more detail and more deeply the properties of this operation that
will be easier to follow the usage examples so to simplify traverse
we notice that traverse is a kind of lifting of sorts it\textsf{'}s the arguments
can be permuted so these are two curried arguments so we can take
this one was the first argument and then we have la to FLV yes the
second argument so it\textsf{'}s a complicated kind of twisted lifting and
we have seen several times already that often you can find a simpler
natural transformation that is computationally equivalent to a lifting
so let\textsf{'}s derive that natural transformation that is equivalent to
traverse to derive it I asked the question so why can't we have F
map to do the work of Traverse f map would have this type signature
it Traverse sorry this is actually yeah so so this is the type signature
investment but Traverse needs F L be here instead of lfb so see this
F needs to be outside and that\textsf{'}s what\textsf{'}s missing so we need to transform
lfb with F inside into F L being with em outside so that\textsf{'}s the transformation
that is a natural transformation we expect to be equivalent the trailer
so what\textsf{'}s called a sequence so this is maybe not a very good name
sequence kind of suggesting that we change the or the order of L and
F in the functor composition not a very good name but that\textsf{'}s a traditional
name and I don't know how else to call it indeed we find that the
functions traversing sequence are computationally equivalent so this
is why well we have defined sequence likeness so then Traverse of
a function f is computed by first doing F map of F like we get here
then we get this and then apply this sequence function that I abbreviated
to seek which performs this lets transposition of the order the type
diagram looks like this so we start with a type LA and we can do Traverse
from it directly with this trap function which takes F and director
gives you from la f OB or you can first do F map so you have a function
f under L you get instead of la l FB and then you change the order
so you reorder the functions composition ever be to FM and the results
must be equal so for any value of this type you go up or horizontal
and the result must be the same value of this type that\textsf{'}s the definition
and as we have seen before this pattern implies that natural transformation
is defined as a composition of F map sub sorry and lifting is defined
as composition of F method natural transformation and then this natural
transformation is equal to lift and if you take identity instead of
F obviously and so then that\textsf{'}s a pattern we've seen time and again
where the result is that traversing sequence are computational global
you can derive one from the other and back and it gets the same function
back so I'm going to spend time through again since it\textsf{'}s exactly the
same proof as we had many times just a different type signature and
notice also F here is an arbitrary placated factor so these functions
don't use the structure of f other than that it is applicative so
that\textsf{'}s that\textsf{'}s an example we have seen just just before we implemented
a traverse function for this type constructor by applying the zip
function of F otherwise we don't you know what F is we just use zip
from it and so we are generic in the function in the functor F as
long as it\textsf{'}s implicit if we don't look at the structure of F we do
look at the structure of L so the Traverse function depends on the
structure of L but it doesn't depend on the structure of F it\textsf{'}s generic
images future that sequence has this type signature and that\textsf{'}s an
example of a sequence natural transformation you note we cannot have
the opposite transformation I'll show the example for that but well
for future analyst you could make an opposite transformation for a
future of a list and you produce a list or individual futures that
are going to be all already copies of this future mapped to select
one copy of the lid and of the element you can do that it\textsf{'}s kind of
useless to transform in the opposite way but what I will show on examples
is that it\textsf{'}s impossible to have this transformation in general and
arbitrary it\textsf{'}s possible for future not for arbitrary yeah so examples
of traversable function functors this example we have already seen
list is another example sequence in general and also finite trees
various shapes there and also traversable an example of an entre versatile
factor is there either Malad and also the lazy list or infinite product
or stream is sometimes called lazy string so let\textsf{'}s see why that is
so let\textsf{'}s implement the sequence for this type constructor first so
I'm going to define this file constructor for convenience and the
seek function is I'm just going to define directly it\textsf{'}s going to have
this type signature and that\textsf{'}s just a zip apply it twice and then
some reordering Oh nested tuple that results from the zip and we know
this is associative because we assume that F is applicative now I'm
using my own typeclass for F which I call the zip for applicative
but you can also use cats applicative just has a slightly different
name for things and it doesn't have the zip syntax so I like the use
of zip syntax so I'm using my own type process here but it\textsf{'}s equal
to standard implicit if that was so the sequence function has this
type signature just as we have seen in the case or filter balls lowlands
and applicatives it\textsf{'}s much easier to reason about this natural transformation
rather than the reason about the lifting it\textsf{'}s also will be the case
of laws or in simpler to formulate so that\textsf{'}s why I will always always
define just a sequence function I will not define a traverse function
the Traverse is easily defined in terms of a sequence - and I'm therefore
in this tutorial concentrating entirely on the sequence function I'll
never implement Traverse directly to save time so let\textsf{'}s have another
example that either as a functor either Z where Z is a constant type
then the sequence must have this type signature so it takes an either
of Z FFA and puts F outside so it\textsf{'}s pulls the function f from the
inside of our plan into the outside that\textsf{'}s the time signature of sequence
so how do we implement that all we need to match so if it\textsf{'}s the left
we have a Z to produce F of Z so the only way to do that is to use
the pure method okay I mean interchange there were there blinds here
the left z does not have any FFA and so in order to produce an F of
something we have to use the pure method from F and then we apply
that to the left of Z and then we get the right type if we have a
right of FA then all we need to do is to put the right inside the
width so we just map that we don't change the value of a we just wrap
it to wrap it into the right type construction so that\textsf{'}s the very
clearly simple implementation so if we actually write this type signature
using a shortcut notation then maybe it\textsf{'}s even easier to understand
how the sequence function works so if we have a Z then we just put
Z inside F using the pure if you cover away then we just put a 0 plus
a into that by mapping with the right type constructor so that we
don't change this value hey let\textsf{'}s see how to implement the sequence
method for the Fortran type so here\textsf{'}s a simple binary tree it has
a value of type a and belief and it has a branch of two trees so how
the hell does sequence work on a tree like we're in the leaf where
we just wrap in the leaf like we did with this either and if we're
in the branch then we apply the sequence method which is the same
sequence we're defining is a recursive function them so we recursively
apply the same since and as a sequence method to the left and to the
right branches of the tree and then we zip them together so then zip
is the operation in the F function wickety function so we can use
that zip them together we get an F of a pair two trees and then we
wrap that fear is a branch under the map so this map is under the
tree sorry under the F function so f is an arbitrary negative function
and we're using its methods map and zip here we used its metal period
but other than that we did not use any knowledge of F so it is in
this way that we are generic in the factor f we are not using any
knowledge about the structure of f other than it has that it has a
map and zip and puree method let us see examples of non traversable
functors so here\textsf{'}s an example it\textsf{'}s a reader mode with a parameter
in its non polynomial and so it will turn now that this is not reversible
so let\textsf{'}s see why what\textsf{'}s takes on applicative function f specifically
like option a and let\textsf{'}s find all implementations of this type signature
which is the pipe signature of sequence now all implementations turn
out to be just one and this implementation always returns none so
it always returns an empty option ignoring its argument so it is not
a very interesting implementation and we will see shortly that this
would not satisfy was overprotective of over a traversable contest
so this is this satisfies the pipe signature but it does not satisfy
the walls we haven't yet seen the laws but it is reasonable to say
that this function completely ignores its argument so it loses information
and typical walls for a lifting would be identity and Composition
laws identity law would tell you that some lifting is identity but
if it\textsf{'}s losing information it cannot be identity so it cannot preserve
the data that you give it but if people lose it will always return
empty option and so that\textsf{'}s reasonable to expect when this note is
not a good implementation and so there are no good implementations
- thank you let\textsf{'}s take another example and you'll see there\textsf{'}s there\textsf{'}s
one implementation for this so we can actually implement this type
signature for this duplicative functor this this code is what you
would expect it\textsf{'}s taking this function so what translate entire signature
let\textsf{'}s just e to the pair and this is pair of eternity doing now if
you have a eat to the pair and you can produce a pair of Italy in
d2 a that\textsf{'}s very easy just duplicate your your function so we do have
implementations from this book we're supposed to produce an implementation
of sequence as a generic in the function so we cannot look at the
structure from the functor F and have a different implementation for
every effort want to be generic and so because we can we cannot implement
it for some F we're stuck in this is not going to prevent reversible
functor let\textsf{'}s take another example where we have a pair so the pair
of some type and a polynomial function and let\textsf{'}s take this as the
implicated and again we find them there is an implementation so that\textsf{'}s
fine well this is actually traversable what\textsf{'}s considered the infinite
list so the infinite list class needs to be defined because we cannot
have the cursive type as a type an alien\textsf{'}s we have to have a class
and it\textsf{'}s a pair of value of type a underlays evaluated tail choosing
again an infinite list of time from all values of tightly so let\textsf{'}s
define a sequence method well we can actually define it it\textsf{'}s quite
easy you you take the head of the list you zip it with the recursive
implication of the same function sequence to the tail from the list
which will sequence were found commute the order of factories and
then you wrap it into the infinite list again so that\textsf{'}s similar to
what our implementation for the sequence operation on a tuple except
that it\textsf{'}s recursive and it turns out this is infinite recursion let\textsf{'}s
check that the even the simplest functor have the identity function
let\textsf{'}s define it like this put some type of class instances for identity
function ages tribunal was defined let\textsf{'}s define an example value of
an infinite list which is if you find like this it\textsf{'}s a recursive definition
we could do a lazy Val instead of def I believe but it\textsf{'}s just cleaning
doing them since it\textsf{'}s a recursive function and you see the tail of
the list will turn again the same list so it\textsf{'}s going to be an infinite
list of 123 on the whole the way to infinity so if we use sequence
on this value then we get a stack overflow interaction because it\textsf{'}s
an infinite recursion so it\textsf{'}s it\textsf{'}s an infinite loop there\textsf{'}s no way
to implement the Traverse of an infinite list because basically what
you would mean is we need to sequence like this we need to have a
list of F values so it\textsf{'}s an infinite list of an infinite list about
four days and that should be mapped into an f of infinite list of
is now how can we do that we need to pull F outside to the outside
of the title which means that we need to evaluate infinitely many
of these f\textsf{'}s in order to pull off outside mean generically that\textsf{'}s
what we need to do we need to evaluate infinitely many elements of
this infinite list in order to put F outside it\textsf{'}s impossible to just
pull F outside magically out of the infinitely many elements here
and so even when F is just an identity factor s won't work it\textsf{'}s impossible
to pull f outside I mean it would be possible for identity factor
of course but we have to be generic in the factor f we cannot use
any methods on F other than zip and map and so we don't know what
F is f could be something that needs to be evaluated in order to pull
a out of it and so because of that it forces us to evaluate infinitely
many elements before we even get a single value of this type and so
that will never end and so that\textsf{'}s impossible so for this reason an
infinite list is not reversible and finally I mentioned that the opposite
type signature isn't is unworkable so why let\textsf{'}s make an example calculation
so let\textsf{'}s say L is an easier which we know is reversible F is this
reader unit which we know is applicative so let\textsf{'}s find all implementations
of this time signature and we find they're not there is no implementation
of this type signature the reason is that this type signature would
have to map this function into this data but that is impossible you
cannot extract Z out of this so you could not possibly return the
Z because you need an integer so which integer are going to give imagine
that this integer is a different data type you don't you don't have
values of it necessarily for integer you could put 0 in that integer
sure but that would not be reasonable from other types so you cannot
possibly pull Z out of this because there aren't any special values
of this type and you can also not get this function because you only
have this function and this function could sometimes fail to return
an 8 could sometimes return a Z so for some integers it could return
is e and you don't know in advance for which integers it will return
Z and for which it will return in a until you run this function on
every possible integer you won't know that and so that\textsf{'}s impossible
to know and so you can't split this into a Z and into it you couldn't
either split it into two functions into Z and into in the same for
the same reason you in order to split it you'd have to run this function
on every possible integer and see what the results are so that\textsf{'}s unworkable
and so that\textsf{'}s why you don't have that\textsf{'}s that\textsf{'}s an informal reason
why you don't have any implementations of this type and a final comment
is that there are several ways of implementing the sequence usually
so let\textsf{'}s consider this type again we have seen we can implement a
sequence by applying zip to x and here\textsf{'}s another implementation we
can arbitrarily select a different order so we instead of zipping
one two three with zip to one I'm sorry this is a mistake two three
say 1 and that\textsf{'}s valid so the type is right and the laws will hold
as well we'll see why so that shows you that there\textsf{'}s more than one
way of implementing the Traverse or in the sequence function which
is equivalent for a given type constructor L different valid ways
of doing it so let\textsf{'}s find out if other polynomial factors are traversable
now one of the central results here is that all polynomial functions
are reversible we will show this quite rigorously later so for now
let\textsf{'}s see how we can implement the Traverse or the sequence function
for an arbitrary polynomial function so we have done it so far for
this and we have also done it for either which is a simple polynomial
factor and the general polynomial factor would have this form it\textsf{'}s
got a polynomial in a with some constant coefficients which I here
denoted as Z Y Q P so we have seen how to implement for a monomial
so let\textsf{'}s first consider monomial like this so one part of this polynomial
then we can apply zip to these so first we look at lfb so RV has this
type so we can apply zip to these we get this and then we can lift
the Z into the functor F by just this standard factor map this function
is always possible for any factor f alternatively we can do F pure
of this Z and then this will become FZ and then we can zip it together
with all others but the result will be exactly the same as using this
function because of the law of identity for applicatives so the result
is going to be this and then that\textsf{'}s a sequence method for a single
monomial and then for each normal you do this and you have a disjunction
of different results of type F of a monomial and then we lift it to
F of the disjunction like we did in the either implementation so we
we have seen therefore that we can perform the traversal equivalently
the sequence operation on monomials and we can also perform them on
disjunctions and therefore we can perform them on arbitrary disjunctions
on monomials and that\textsf{'}s arbitrary polynomial factor also note we could
apply zip here in different orders wicked first zipless and then Z
put Z on the right we can change orders in different ways and so traversal
order could be application-specific it could be necessary to adjust
it for a certain application you can always implement in some order
or automatically say but it might not be necessarily correct for your
application and also we have seen that non polynomial factors at least
some of them are not to her so so this is not reversible because we
cannot have a reasonable implementation of this that does not lose
information and there\textsf{'}s this paper that I'm referencing here it\textsf{'}s
a complicated paper there in theoretical but it has a proof that only
polynomial functions are reversible that that and and also they must
be finite so infinite lists do not fit the conditions of their theorem
only finite containers polynomial factors are essentially containers
with data they can have different shape they can have many items of
data or few or none it could be a disjunction of different shapes
and they also can have extra data of some constant type in addition
to data of the type a but those are the containers that are traversable
and no other containers aren't reversible sorry you have a lazy infinite
stream that\textsf{'}s not reversible they have to be finite and they have
to be polynomial so that is proved in this paper in a complicated
way so I'm not going to try to understand how they did it I believe
that this is so because I have examples that even the simplest non
polynomial factor can't have reasonable implementation of sequence
so even though all polynomial functions are traversable they are usually
traversable in several different ways and so it\textsf{'}s useful to have a
typeclass to declare a specific instance of a typeclass expressing
a specific order of traversal so this order of zipping that we can
choose here corresponds to ordering of traversal in a sense we'll
see that when we look at specific examples of traversing but now let\textsf{'}s
take a look at the laws because we have have been talking about the
laws so far and we need to see more in more detail how they work so
I prefer to look at this type signature of Traverse and to derive
laws using the lifting intuition so it\textsf{'}s a lifting of sorts and every
time we had lifting so far we had laws of identity and Composition
in other words there was some kind of identity here and some kind
of composition of these and this has to translate into identity here
and composition of these I will mention that there is this paper which
is arguing what laws traversals must have from a different perspective
not as formally as I have argued just now because my argument is completely
formal Traverse looks like a lifting therefore it should have laws
like the laws we had before for other liftings it\textsf{'}s purely a formal
argument saying that the form of this function is similar therefore
it should have similar laws but this argument doesn't look at what
Traverse actually does well what it does is that it evaluates some
function on each element this one produces some effect maybe this
F is an applicative factor which could be a monad or it could be known
more and what it encapsulate some kind of effect some kind of computational
context or a result other than B and all these contexts need to be
somehow put together and be outside of the elbe so we need to reconstruct
our container L inside the larger effect described by F so we need
to somehow combine all these effects for individual values of a into
one big effect which will be outside and then we have to combine all
the values of B after somehow pull them out combine them again into
the same shape as the container L inside the F so all that needs to
be done by the traversal function and the authors of that paper argued
that the traversal first of all should visit each element of the container
exactly once it it should evaluate each effect exactly once and then
combine these effects into a larger effect and using this intuition
they formulate some laws that seem to fit this description in some
way well they didn't actually derive these laws from these requirements
but they showed examples where these requirements are violated and
they showed that these examples also violate the laws so this is a
little not very convincing to me that these laws are correct and therefore
I prefer the more formal approach because I have more assurance but
if I find some reasonable identity and Composition laws that\textsf{'}s a correct
set of laws so far in every example we have seen with functors with
contra factors filterable applicative and wounded every single example
had a function with type signature like this which was life like a
lifting in my terminology and in every single example there were identity
law and composition law and these laws were equivalent to all the
other set of laws that were derived from intuition and these laws
also corresponded to some category laws now in this example I don't
know how to formulate this in terms of a category and everything a
way that would be simple enough so I'm satisfied that I find a law
that looks like identity law and the law that looks like a composition
law even though I'm not satisfied that I can find it easy enough category
to reason about so that it\textsf{'}s useful so I'm not going to talk about
the category in this chapter I'm not going to describe this as a lifting
from one category to another because I don't know if that\textsf{'}s really
very useful and I don't know a good formulation of that so let\textsf{'}s look
for these laws so identity law is that we map some special function
here that is identity in some sense into a function here that plays
the role of identity now the type signatures are not a to a they are
a to f of B so what could be playing the role of identity here well
the pure method obviously and F has that method by an assumption so
the identity law is that if we put a pure method here than it should
be lifted to this which is again a pyramid and except it\textsf{'}s applied
to a different type of parameter now another way of formulating identity
is to say that if F is the identity function so then there\textsf{'}s no F
you just be then identity function here is all ordinary a to a and
that should be lifted to identity elite really so f is just identity
function and then this identity should be lifted to this so that is
another way of formulating an identity law let\textsf{'}s find out now to compose
what will be the composition if we have two of these functions like
f and G then we can compose them using F map but the result would
be this because the F G would be nested now we take F be we map G
over it and we get F of G of C now if F and G are in clique are applicative
then the composition F of G is also applicative we know that from
the properties of the platitudes and so it is again a function of
the same type except that it has a different factor instead of F so
f function G factor and the composition of such functions is going
to be of this kind which which is kind of complicated it changes the
functor each time so what should be on the right-hand side what should
be a composition of these traversals now the composition of traversals
obviously works in the same way so you have la 2 FL b and then FL
b 2 f g LC and so that is your final traversal and that should be
equal to applying the traversal right away with this function as a
purgative factor so in other words applying traversal to this if that
is true then traversal of composition is equal to composition of traversals
so that\textsf{'}s with a little twisting where we keep using F map in order
to get composition and keep pulling all the functors F and G all the
implicit is we keep pulling them outside with these twists it looks
like just a composition law traversal of composition of F and G with
some twisting is equal to the composition of the two traversals with
F and with G again with some twisting so these are the laws we're
going to examine and it will turn out these are exactly the same laws
as this paper proposes when I first looked at this it looks like we
have two identity laws are they really independent no they are it
will see that but that\textsf{'}s a question we need to answer also laws for
sequence are probably going to be simpler because in our experience
so far always we found that the laws became simpler if we consider
the natural transformation instead of a lifting so let\textsf{'}s find those
laws for the sequence and finally with the laws in in hand we can
try to answer the question of which functors satisfy these laws we
have found examples where we can implement the type signatures but
are the law is respected by those examples or not so that\textsf{'}s the questions
that we have after this point so let\textsf{'}s look at the first item the
dual so a traverse of pure needs to be Pierce and how fast if it\textsf{'}s
the first law let\textsf{'}s look at the type diagram for this so we started
from away we map it with pure where we map it with traverse of pure
and it should be the same value it was type in favilla so that\textsf{'}s the
first one now the second identity law looks like this and it\textsf{'}s clearly
a consequence of the first identity law if we just put F to be the
identity founder and the first law should hold for every applicative
functors so we really just need this one law when the second one is
a consequence so let\textsf{'}s forget it so we have the identity law which
is this and we have the composition law which we can write like this
and the short notation so for any function of this type and the function
G of this type and for any applicative factors F and G we have so
the composition of F and G is twisted in the sense that G must be
lifted by factor F so this is our twist on the composition other than
that composition of two traversals is equal to Traverse of composition
here will we twist and here we twist let\textsf{'}s look at the type diagram
we start again with LA first with Traverse with function f and we
get F lb well then we want to traverse with G but G works on B so
we have to we have to traverse just on this lb and F needs to stay
outside therefore we use F map of F and we Traverse inside of F using
the Traverse of G which is giving us this value F G LC or we could
directly Traverse on the function H with respect to the factor f of
G which is a functor composition so the factor H could be defined
like this as a functor composition of F and G and we know it\textsf{'}s applicative
so we could just directly Traverse with respect to H so this can be
traversal with respective H of a function H of this type this function
is defined like that so this is the law that we demand now which will
hold let\textsf{'}s derive the corresponding law for laws for its sequence
to do that we just Express Traverse like this your sequence and substitute
that into the laws of chillers so let\textsf{'}s look at identity law Traverse
of pure equals P R so Traverse of pure equals this conduction legal
pure so that\textsf{'}s the law of identity for pure so pure lifted to Bill
and then sequence should be the same as pure applied to the type parameter
Helle I should also mention naturality law that\textsf{'}s always the case
for all of our examples so far such as filterable moon and imitative
but all these natural transformations as well as the liftings they
have natural tea laws for each type parameter that they have there\textsf{'}s
one neutrality law which expresses the fact that the code implementing
that function does not use any information about the type it\textsf{'}s completely
generic in that type and so you can map this type to another type
before the transformation or after the transformation and the results
are going to be the same so here\textsf{'}s what the naturality law looks like
for the sequence now sequence has a more complicated type signature
and its type parameter a is all the way inside therefore in order
to transform in this a to some B we need to have a double F map so
let\textsf{'}s say there\textsf{'}s some function G that transforms a to be in order
to transform this a or transform that a we need to lift this G twice
and that\textsf{'}s how I would write it down so first we can sequence this
LF a to FL a and then we can lift G twice in this order or we can
first lift G twice in this order so that instead of LF a we get LF
B and then we sequence on that so that should be the same function
that\textsf{'}s a natural t law and the code for sequence will automatically
respect that law if it is a code that is generic type parameter a
will never use any information about the type of a so usually naturality
laws are satisfied automatically by generic code and so it\textsf{'}s not important
to check them but it\textsf{'}s important to check the other laws so now let\textsf{'}s
look at the composition law we need a bit more work about it traversal
of F followed by the traversal of G lifted so let\textsf{'}s substitute the
definition of traversal and then we get this formula now we can decompose
a lifting of composition going to factor law so we can get like this
now naturality law means we can interchange this and we get that and
finally the right hand side is written like that so again we can do
that because of the thumb tree law so in other words the composition
law for traverse says that this is equal to this they have a common
prefix that we can omit because these are arbitrary functions F and
G for which this must hold and so it\textsf{'}s sufficient that these things
are equal so that\textsf{'}s the law for sequence much simpler and here\textsf{'}s the
type diagram because the type types here are actually complicated
so we start with an L of F G C C is some type parameter I could use
a here but I just can't see now first we sequence with with respect
to F and then we just pull F out and we get F LGC then we sequence
with respect to G but we lift it to be operating inside of F so that
means we interchange this L and this G the result is f G LLC alternatively
we could sequence with respect to the factor f of G this is maybe
not a very nice known notation not very consistent but this is the
composition of functions F and G so it\textsf{'}s the page I mentioned here
I should probably for clarity I should use page instead of F G and
so if we sequence with respect to F G that means we pull out F G at
the same time to the outside and we get F G of LC so the result of
these two computations must be the same having formulated the laws
now can look at constructions and we can check that the laws hold
we can also check whether some factors can be traversable when they
are polynomial or when they recursive or they're not polynomial so
we have so far seen some examples but now let\textsf{'}s be more systematic
about it here are the constructions that I found for traversable factors
and I will go through each of them one by one now now before I do
that let me explain what is a by traversable and by functor so by
factor is a type constructor with two type parameters which is a functor
in both of them so it has a map with respect to each of them separately
and it also has a map with respect to both of them at once of course
because you can just you can transform a to some C and B to some D
separately or at the same time if you wish it doesn't doesn't it so
that\textsf{'}s a bi functor and by factors are by traversable in the following
sense they have this natural transformation where we have an F wrapping
each of the type parameters of s and this F can be pulled out by this
sequence by sequence natural transformation and it should be natural
both an A and B separately so that\textsf{'}s the assumption and so you see
this a and B remains and we have pulled F out to the outside so if
statute transformation exists which is natural in both a and B and
it has the same laws the laws of identity and the laws of composition
now the law of identity is this and it needs to be adapted obviously
to this by seek so that you have pure here and pure here and the law
of composition obviously needs to be adapted as well because we have
F G a F G B and so on but other than that it\textsf{'}s a direct generalization
so analogous laws must hold so let me now begin by deriving the laws
that must hold for these constructions rather deriving the fact that
the laws hold for these constructions first construction is constant
functor and identity factor so both of them are traversable till you
find that we need to define the sequence method and to show that the
laws hold for it that\textsf{'}s very simple so for convenience I'm just going
to put a type parameter up front the constant factor is a function
that doesn't end on its type parameter so L of F of a is the same
as I've elevated Simon is this constant times e and FMLA is f of Z
so sequence is this type signature and clearly this is f dot P R is
just a pure of F functor can't in any other way produce a type signature
like this but is generic in if we're not allowed to know what F is
except that it has a pyramid zip and map so let\textsf{'}s just define that
as a pure let\textsf{'}s check Louis so the identity law is that F map of pure
followed by sequence must be equal to F dot P R now I'm putting here
this underscore ell notation for clarity I could have said for example
instead of F dot P R I could have said pure underscore F what I mean
by this is that it\textsf{'}s the method that is defined for the type instructor
F such this method is not generic and F it is defined for each F separately
in the typeclass instance similarly F map is not generic and L it\textsf{'}s
defined for each L separately so I could I could have written like
this as well it\textsf{'}s not Scala notation necessarily but it will do for
now just for clarity to remind myself where these methods come from
now sequence is defined for L so here I could write it like this to
remind myself that the sequence is defined separately for each ill
but it is generic and F so I could write it like that so sequence
has F as a type parameter but L as type constructor or or functor
for which it is defined in the Thai class now so let\textsf{'}s see if we can
verify this law F map is identity for the constant factor so this
is identity and so this is just F here sequence is sorry F map of
F of F here is identity function f map does not transform anything
because there is nothing to transform we only have this Z cannot transform
so this is just at the energy function you can cross it out and then
we have the law that sequence equals pure at sequence equals pure
by definition so the law is satisfied let\textsf{'}s consider now the composition
law which is the composition of sequence of F and the lifted sequence
of G so now I'm using this notation with more sense in here I didn't
have to say it\textsf{'}s sequence Hamilton if I could just say sequence but
here it is important now to say which for which type parameter is
because the factor is a type parameter in sequence function so we
have the sequence applied with respect to applicative factor f and
this with respect to G and F G is the factor which is the composition
of F and G we just {[}Music{]} denote it like that for gravity so
this is now sequence is defined as pure F map is {[}Music{]} defined
in with respect to F so this is f naught F naught F naught L F map
l was identity but F map F is not and we actually don't know what
it does because it f is an arbitrary let\textsf{'}s look at your family so
you need to keep it symbolic so we have f dot P R which is this sequence
and then because the Scala operation and then responds to my composition
symbol and then we have F map F of G period because this is that the
pure in G and that will be equal to FG pure but actually what is the
definition of pure for composition of functions that\textsf{'}s exactly this
it\textsf{'}s a pure of F followed by lifted pyrimidine so it\textsf{'}s the definition
of what pure use for the function f G and so therefore the law holds
you now consider the identity function you know the identity function
is like this or sometimes can it can be denoted by it with a capital
I so this is identity factor I believe the cat\textsf{'}s library does this
maybe not maybe Scalzi so how do we define a sequence for it well
this is just an identity function because it doesn't do anything because
we we have f of a it goes to F of a because it is just wrapping it
doesn't do anything so this is identity function we could even write
this sequence function differently like this in order to underscore
that factor that\textsf{'}s just identity function now let\textsf{'}s check the laws
the identity law which is this now if F map is the identity function
sequences identity function f map L is just this F map L is the identity
function of its argument and because of this identity function and
so this is equal to f dot P so this is f dot pure followed by identity
and that should be equal to F dot penis clearly so composition is
that we have a sequence followed by F map of sequence now sequence
of anything is identity so it\textsf{'}s just composing one identity and F
map of another identity but we know that lifting F map of identity
is again identity so this is just about composing one identity or
another that\textsf{'}s always going to be identity so that is also identity
by definition and so this law holds let\textsf{'}s move on to the next construction
which is the product for any traversable factors G and H their product
is reversible now I'm going to introduce my own typeclass which reversible
just for convenience but actually the cat\textsf{'}s library has a traverse
that class but I just want to show that this is easy and I prefer
to define seek and they prefer to define Traverse in their tag class
so that\textsf{'}s why I have my own tie plus but basically this is very simple
that concept has just one method and this method is Elif a to FLI
traversal defined that method now this red is a problem with IntelliJ
and install a plug-in cannot resolve the types it compares this entire
code comparison works also we have a convenience method to get the
Traverse into instance this drove in the syntax to use seek as a as
a method rather than user so let me then show how I can define an
instance of track of travel this traversable type was for this type
constructor which is a product of l1 and l2 given that both l1 and
l2 are traversable factors so first time to mend the functor instance
well that is standard it\textsf{'}s just a map this map that just a functor
instance for product and then i implement the traverse instance using
C so what am I supposed to do I have this a1 F a times l2 FA and I
need to give F of L 1 a times Delta a so first I can say please so
sorry first I can sequence each of these separately because l1 is
reversible and all to escape most of my assumptions for a sequence
separately both of them then I can zip the results and I get what
I need so that\textsf{'}s my idea about how to implement so I just say sequence
the first one sequence the second one and zip now I could also write
the code like this but then my IntelliJ doesn't understand where these
things are coming from but the code actually still compiles if I do
that so this is just much more your boss telling me where the sequence
comes from from which factor and in principle I expect a scholar to
be able to derive this automatically and it can this is much simpler
to write about intelligent isn't like it so let\textsf{'}s check the laws here
so there is the identity law which is that we can map the pure and
apply seek and that\textsf{'}s the same as the pure so how do we verify this
law well we substitute the code of F naught which is up here and we
apply both sides of this law to some arbitrary value of this type
and one until two so we do that looking at the laws actually we have
to start from LA and the law will give me this so the value on which
I apply both sides of the law must be of type le so that\textsf{'}s why as
eleiza is a pair of L 1 and L 2 and then that\textsf{'}s why I take some arbitrary
pair and i apply the law to it so the left-hand side first i apply
this and that\textsf{'}s going to be that there\textsf{'}s a half map this is L 1 this
is no true and then we apply the seek function to this now the code
of seek is I'm going to write it like this in a shorter form and the
result will be that I have this seek zip this so now we need to assume
obviously that the law already holds for a 1 and L 2 separately and
therefore for example we have this which is the formulation of the
law for L 1 and so we can substitute that in here and then we get
that now because of the identity law of applicative zipping of two
peers is a pure of the peer so this equals F pure of the peer and
that\textsf{'}s exactly the same as applying F pure which is on the right hand
side directly to the pier so we have verified the identity law let\textsf{'}s
verify the composition law this is how we could write it I have written
out sequence seek of L seek of the hell everywhere it\textsf{'}s just for clarity
so in my notation that means a sequence made up defined for the tag
constructor L applied with respect to the type parameter F so now
let\textsf{'}s apply again both sides of this law to some suitable value let\textsf{'}s
see what type that value should have which you have the type L F G
C as the initial point in the time diagram so let\textsf{'}s have an arbitrary
value of type L F G C so sort of C values a here so L F GA naar betray
value of that is a pair of l1 f g l2 f g and applying the code for
seek to this we get a 1 FG a sikh with respect to f zip f l2 mg is
equal respective so that applying just the first step according to
the type diagram that would lead us in here which is f of LG F of
L 1 G 2 G so now we apply F naught F of this to that now what is f
naught F of CBG know seeing G is with respect to L sorry with respect
to G over L so it acts on some value of this type and it will give
this according to the definition but we need to lift this function
was F naught s the difficulty here is that f is an arbitrary function
we don't know what ethnic F does we don't have code for it we need
to keep it symbolic so clearly we need to use a definition of zip
F G somehow to find out what F naught F will do so the definition
of zip of F G is this there\textsf{'}s some FG x and f gy with zip them in
the factor F G so so have G X is a type F of G of X G Y is of type
F of G the definition of the zip for the composition is one of the
constructions for applicative it\textsf{'}s it\textsf{'}s a zip in the F factor followed
by F map F of zip in the G factor is this what we have in our formula
not quite because we have not just something zip F something after
which we we do F method but we have these things transformed we seek
so let\textsf{'}s transform using some function so let\textsf{'}s use a natural G of
zip so that we can transform like this so we transform under the factor
f and so the result is the same as if we transformed FG x and f gy
and then zip so that\textsf{'}s a natural T you can transform first and then
zip or you can first zip and then transform that\textsf{'}s up to up to us
so finally if we use this formula we can see that F map f of seek
L of G is like this it\textsf{'}s the F map F of this so this is seek L of
genii because seek L of G gives us this kind of expression which is
similar to this P of GX being sequence of this zip G and Q of G Y
is sequence of this and then we apply this to those things and so
the result is that F map of seek L applied to that is it\textsf{'}s like that
it\textsf{'}s sick F map sick G and then seek F maps engine so that\textsf{'}s what
we find this really apply natural TMZ so that\textsf{'}s the left hand side
of the composition and the right hand side is that so these must be
equal now the definition of seek F G is the code of seek that is just
applied to this type parameter F G instead of that F so the same code
and it would have been the same code if we could have this equal to
that now these are six with respect to factor l1 so these are defined
for l1 and these are defined for how to now we assume that for l1
l2 the composition law already holds so that means for example this
so this is the composition law for a 1 and similarly for L 2 so therefore
this is the same as this and that is the same as now and so the composition
law holds for hell the third construction is disjunction so for any
traversable factors G and H the disjunction is again a traversable
factor so we have seen an example of implementing this in either and
in polynomial factors but this is a general construction so let\textsf{'}s
see how to again we assume that factors l1 and l2 reversible and we
defined well as the disjunction of l1 and l2 a standard factor instance
but is Junction so now let\textsf{'}s implement the founder instance traversable
instance for the disjunction so here\textsf{'}s how it will work we have this
so we apply sequence to each of these separately we get this disjunction
and finally we depending on which one we have we lift it into the
F of disjunction using just constructors of the disjunction that left
or right so that\textsf{'}s how it works first we do a sequence on each of
them so if we are on the left and we sequence it again this and then
we map it with left apply which means that we put a left on top of
this and left it\textsf{'}s just a wrapper in a disjunction so it doesn't change
the value and the same we do for the right now the redness here is
again due to limitations of IntelliJ it\textsf{'}s unlabeled transferred types
directly but this code runs it compiles and runs let\textsf{'}s check the laws
at the entity law we do the same thing as before we just substitute
the code of the flap in here and then we apply to some LA we'll know
that both sides of the identity law applied to an arbitrary element
of this type now since the code is symmetric with respect 1 or L 2
it\textsf{'}s sufficient to apply this to some left of l1e and check the law
for that so if we apply it to the left and we are in this case we
have any left applied to to this which is well we are actually in
this case and we're looking at F Nobel first so we are you want to
lift like this now you notice we can't really simplify this because
these are map and pyramids of factors that we don't know these are
a one factor and if I look at your factor so we don't really know
what these methods do we cannot substitute any more code so that\textsf{'}s
our symbolic computation right now we apply sync to this now we're
on the left so we apply seek to to this which is I'm going to denote
that a seek l1 to that and {[}Music{]} followed by map of left block
now nothing more to simplify unless we use the fact that l1 already
has this law and so this law for l1 looks like this therefore we can
simplify here substitute that we get F P were of l1 a map left apply
which by naturally T appear and we have pure of left over one a and
that\textsf{'}s exactly what the right hand side will do when applied to left
of l1 e so identity law holds let\textsf{'}s consider the composition law the
composition law needs to be applied to a value of this type and again
it\textsf{'}s sufficient to consider the left and applied to that so first
we apply the seek and that gives us in this now we apply f map F to
belt and we notice we have F map F which is this one and another if
map F so we can combine them and the result is that we have this two
which we apply map with respect to F of this kind of function so we
first do we left apply to that and then we further apply this function
so now this is not quite Scala code because I have this map underscore
F for convenience and clarity so that\textsf{'}s just keep in mind I'm not
actually using your Scala code here but I could adjust that the types
will be less clear so now what does this do now this c g you know
seek g has to act on the left and we need to substitute the code of
seek when acting on the left it gives us this expression my definition
of our seek up here so let\textsf{'}s substitute into this and we get then
this expression so now we can use the composition law that by assumption
already holds forever so then we have this one Perelman which we can
rewrite like this now this law is not quite giving us the expression
that we want which is this expression because we have this function
here but here we have another map so we're not we have another left
apply to map and also on the right hand side of the law would have
another left apply so in order to put another map inside the seek
we use a net reality of seek we add this on the right-hand side which
gives us this seek so seekers natural so we can apply seek to a transformed
argument or we can apply transformation to the result of seek and
so then if we define what lab FG is that is f map of F map and we
put that onto the left-hand summit so that we have CK one of this
map F of this map F both of that you know we can combine the map F
this one with this one and now we have a single map F with a bigger
function but that\textsf{'}s exactly what we had for the left hand side and
kill him before and so therefore the right hand side response is equal
to the left hand side his code so the code is equal and the same would
be if we replaced a woman so this shows what laws hold for injunction
now let\textsf{'}s consider there is construction for which is a recursive
function defined as la equals some s of a and la so as s is an arbitrary
by functor so this could be any type function of two type parameters
and this equation this is a type equation it defines a type la recursively
so examples of this are lists and trees different factors s can encode
very easily different kinds of wastes and trees now this also describes
infinite lists and infinite data structures and we have seen an example
where the infinite data structure does not actually work you can implement
the function without the recursive calls never stopped so that\textsf{'}s the
problem the laws will appear to hold in our proof but actually it
will not work in practice so I will comment on this when I use the
recursive assumption in the proof so let\textsf{'}s see how it works so I introduced
the biofilter as a type parameter up front so that I don't worry about
it so much it\textsf{'}s easier in Scala to do this now I cannot just introduce
a type like I did here because reclusive type illnesses are not allowed
in school so I need to introduce your class so once that actually
makes the code a little clunkier because now and you need a name for
the data value inside and I need to wrap and unwrap but that\textsf{'}s a small
inconvenience so the class contains a value of type s of a end of
the same 11 so this is a type function that I'm considering as a parameter
and now I can implement a factor instance for this now this obviously
is going to be recursive so in order to map this l 8lb i map a to
be here and I need to map this la to L be recursively by using the
same map so I need to do this under yes so I need to map at the same
time a to b and le to l be under the type constructor s so that\textsf{'}s
by map so I'm mapping the two type parameters of s at the same time
and then my map is a method that takes two functions with two type
parameters so it goes a to C and in this case we have the by factor
with type parameters a and L of a and so it takes two functions a
to C and elevate to D gives us FCD so this is combining map with respect
to the first router and map with respect to the second parameter in
a single call which is completely equivalent to doing first the first
type parameter with a map and then the second parameter with another
map but it\textsf{'}s just easier to do it with one method probably two and
this is the recursive call to the same function so this is how we
do it the functor instance now let\textsf{'}s look at the Traverse instance
how would that work so L of a is the same as recursively defined as
s of a and elevate so now we need to transform this into this if we
want to implement seek how do we do that well we can obviously seek
here recursively so the second argument of s is a recursive instance
of the same type and so we can assume that for that recursive intense
instance we already have the implementation so that would be the recursive
call to the same function so that would be transferring into Fla so
now we have so we just do a by map where the first one doesn't do
anything it\textsf{'}s identity and the second function is a sequence recursively
calling the same and now we do by Traverse or by sequence actually
by sequence which is as I indicated before it\textsf{'}s transforming s fafb
into F si be pulling out the F at the same time out of both type parameters
to the outside so we use that and transform s of F a Fla until F of
s of any LA so this is exactly what we need so in other words seek
is just a composition of by map and by seek but on that one factor
now I can use cats library it has by functors by traversable and it
has this voice sequence so I'm using white functor and white reverse
which is canceling berry typeclasses just I could have defined them
just as easily as the trove that class and the redness again is a
some problem with types although I indicated all types explicitly
but still it doesn't like it so therefore the Seekers just by map
with identity and seek which is a recursive call followed by the by
sequence call on the by factor so that\textsf{'}s exactly what we plan to do
first we buy map so that we seek under yes with respect to the second
type parameter and then we do by sequence yes so that works and let\textsf{'}s
check the laws so the identity will need to substitute the code now
for for clarity I still write f9l here and so on but it is easier
now to distinguish sequence with respect to L and s because s is a
by function so it has basic and violent and L has seek so I'm going
to write seek L I could have done it like this but I'm not going to
write it for quality for brevity so okay substitute the code of f9
that gives us this code which we defined here the lab instance the
defunct our instance for help which maps using by map under F and
the recursive on the second so let\textsf{'}s write it down so this is a by
map of this function and the recursive FF L as a second argument of
why map so the same function as we're defining here is here so now
we need to apply secrets the result is this followed by by map actually
followed by bicycles remind myself what is my definition of sequence
for you yes just take the S out of the case class which I'm not going
to write here because it\textsf{'}s just wrapping and unwrapping we want to
pretend that this case class is just a type so first it\textsf{'}s apply map
with identity and seek and then it\textsf{'}s a buy seek on the result I'm
going to ignore this because this is just wrapping right so seek is
by map followed by by sick this is sick L so we have this expression
now we can combine the by maps just like we can combine maps because
they're by factors it\textsf{'}s just first we have a map in the first type
driver so we combine these two which gives us this and then we have
a map in the second type planner which combines these two and that
gives us that so f map L and then seek is the same law that we're
trying to prove not F not LPR and then seek is f pure so it\textsf{'}s a recursive
invocation of the same law and we're trying to prove in the second
function of Y map so we can use an inductive assumption that we already
proved that by recursion and therefore we just substitute into this
expression we substitute F P R inside of this because that\textsf{'}s a look
so then we get by map of pure pure by seek and the identity law for
s means that s by map pure pure basic is the same as s is just pure
and so that\textsf{'}s wasting holds for Escalades by traversable function
by function and so this becomes the right-hand side of the identity
law and that\textsf{'}s the proof of the identity law so now let me mark about
using the recursive calls and it is that inductive assumptions corresponds
to recursive calls in code so in mathematics the inductive assumption
is that on the previous step things were already proved in code it
means that we are going to call the same function recursively assuming
that it will return the correct results then our step also returns
the correct results but that assumes that the recursive call terminates
and actually returns the results and we have seen an example that
on the infinite list it does not return it has anything to do and
so that\textsf{'}s where it is going to break so actually the inductive assumption
can be used as long as all these functions actually terminate their
evaluation and return their results and if so then it\textsf{'}s off it\textsf{'}s all
fine but so at this level we will not see any problem with infinite
lists we are using the inductive assumption and everything appears
to be correct however we have not established that these functions
will actually return at all and for the infinite list they don't and
so for some factors defined using this construction using some by
factor s some factors will have infinite loops and others won't so
that is a separate thing we need to establish in order to check that
they are actually reversible this is usually not a problem for factors
because this map is going to be cold maybe later and this is a lazy
call and so there is no infinite loop but we have seen we do a bye
map and then we do a buy sequence and so that call will evaluate everything
and that will break for the infinite list now all I'm saying is that
this proof of laws and the same will apply to the proof of the competition
law that I'm going to talk about shortly this proof is only as good
as the fact that all the functions return and if you have an infinite
loop in one of the implementations then as proof oh well great because
the inductive assumption cannot be used because the use of inductive
assumptions is is translated into recursive calls in the code and
if those goals never terminate them you can't call them so I'm not
going to present an analysis here as to what possible functors s are
admissible because I don't know how to do that analysis in general
so and that\textsf{'}s a much more difficult topic of recursive types what
are the reclusive types for which certain methods would terminate
and that\textsf{'}s for another chapter so from now we will assume that we
check separately that all the methods will terminate and if that is
so then this proof is correct we are allowed to use inductive asymmetries
let\textsf{'}s look at the composition law so this is the composition law and
the law is an equation both sides of which need to be applied to an
arbitrary value of this type let\textsf{'}s check so L F G C so use a instead
of C I'll probably check rank it in the slides that it is a and not
C I'm using a in the code consistency so s of F G a L F G so that\textsf{'}s
lfg so let\textsf{'}s apply both sides of this law to some value of this type
and we get first sequence which is by map of identity and sequence
followed by by seek and then we do a map of seek which is a function
that takes this and does a by map and and by seek but now with respect
to G so I'm just writing it out what the code is for seek and this
should be equal to by map for the advicing but with respect to FG
so that\textsf{'}s our law let\textsf{'}s check that the schools and we certainly assume
that it already holds for for the s by function which means that this
equation holds and by seek map is by seek now this is just a law this
twisted composition of seek can seek is seek so how do we use this
in order to prove this now clearly to use this we need a value of
this type which is not the same as what we have here now this is s
FG x FG y so is very homogeneous it must be the same functors FG and
we don't have the same factors here we have F G and L F G so this
L is outside if L were inside of all of this here then we could just
say this is why this is eleve is why a is X and then we are of this
shape so that means we need to permute L over there so in order to
do this we need to sequence this with respect to F G and then L will
get inside but we need to use the sequence inside the type constructor
s which requires a buy map now this is going to be a buy map with
respect to the second type parameter of S which is the recursive invocation
and so that by map and sequence is a recursive call to sequence therefore
we can use the law of composition for that as if it\textsf{'}s already proved
and we will do that so here\textsf{'}s what we do so in this law that we are
going to use now we're substituting this value s FJ XY which is defined
like a by map or a sequence with respect to L and respect to F and
then G and then that\textsf{'}s of the right shape with X equal to a and y
equal to L of a so that\textsf{'}s what we wanted and the result is this expression
which is a by map followed by by seek followed by map now we can certainly
use natural reality and exchange map sorry no we we here we're having
basic F map by CG that\textsf{'}s what we're using here it\textsf{'}s a basic FG that\textsf{'}s
that\textsf{'}s the law we're just writing out the law we're substituting s
F G X Y into both sides of the law services here and this is here
so this holds this is an equation that holds so now by inductive assumption
a composition law for sig L already holds when we use it here and
so therefore we can rewrite this right-hand side like this this is
just sick L of F G now this right-hand side is the same as the equation
that we need to check which I noted as a start and just marked labeled
that equation by start for convenience so the right-hand side of that
equation is now the same as this therefore it\textsf{'}s also equal to that
so it remains to show that that the left-hand side is equal to the
left hand side of star let\textsf{'}s write it down so like this yeah it\textsf{'}s
really easier to compare so by map now we have this instead of that
and we have a by seek and we have a map of by seek instead of this
now if we look carefully we have a map f of by seek and here we have
a map f of something that\textsf{'}s followed by by 6 so that can be pulled
out with a map F and omitted so this is a training lab F by sake of
G in both of these so as we just a minute forget this so now the problem
that they are not equal because we have a different order of by seek
and map so have a by cyghfer by map here and here we have a map followed
by by seek when we need to interchange a map and the natural transformation
that\textsf{'}s naturality law and it\textsf{'}s the same way for by factories just
by secant by map can be interchanged let\textsf{'}s write down the naturality
law for clarity it looks like this so we have a basic follower by
map f of by milk of something and then we have a by map of map f of
that function let other functions we have two functions because by
map takes two functions both of them need to be mapped and then we
have bicycling so we have interchange the order by seek and map f
so if we use this law in this equation then we get that the first
line is equal to again we can interchange by c combining up and we
get by C cadine so now we have by map followed by by map followed
by by seek by map can be combined and then we have this by map fold
but by seek and this is exactly the left hand side which is this emitted
by C kanji sorry this one so this concludes the proof of the composition
law and therefore we find that this construction was valid now the
question is which by factors are by traversal and the answer is the
same all polynomial by factors are by traversable now we see without
recursion these constructions 1 2 3 our constructions that allow us
to get any polynomial factor with arbitrary types in it doesn't have
to be monoid like in applicative can be any any constant type any
polynomial function these constructions cover all these cases now
construction to even has two different implementations we can zip
in one order or in the opposite order it still will work and therefore
all polynomial functions at reversible now exactly the same constructions
worked for by traversable constant and a and B are quite reversible
in the same way as constant an identity function traversable and products
and disjunctions are by traversable and so-called by traversable polynomial
by factors can be used in this construction I'm not going to go through
proofs for these by functor constructions they're pretty much the
same as the proofs I went through except you have more type parameters
to worry about what implementations are exactly the same and so the
conclusion is that all polynomial factors including recursive polynomial
factors as long as the Traverse and seek methods return in finite
time and they not go into infinite loops as long as that is the case
all polynomial factors at reversible all polynomial by factors also
traversable and you can go on you can define a by factor by recursion
using a try factor in the same way you can say sa B is equal to some
T of a be s a B or T X Y Z is a try factor and as long as that try
factor is tried reversible exactly the same proof would show that
the by functor is by traversable so you can have recursion at any
level as long as it\textsf{'}s a finite level of recursion obviously at some
point you would have some n factor that is not recursive recursive
and or maybe several of them and then as long as that\textsf{'}s polynomial
is going to be 2n traversable and then you go back and have your n
minus-1 traversable recursive thumpers and so on and so it\textsf{'}s clear
that all polynomial factors with arbitrary recursion so it\textsf{'}s it\textsf{'}s
a tree for example whose branches can be themselves lists or you know
you can have a list of branches or anything like that all of that
is traversable so that\textsf{'}s a major result of this consideration so now
let me consider foldable functors now we we know the fold operation
in a standard library it\textsf{'}s a full left turns out that the fold operation
is a consequence of having a traverse operation and later we'll also
see that the scan operation the scan left is also a consequence of
traverse operation so how do we derive the fold from jurors the main
idea is that we should take a specific applicative factor which is
a constant factor fa equals a constant Z or is a Z is a monoid type
and so the zip operation on this factor is just a monoid operation
which I will denote like this and we have seen in Chapter 8 that these
are applicative now the type signatures are much simplified now in
type signature of triggers becomes this and this method is called
fold map well the does is that it takes a container of type of values
of type II and takes a function that map\textsf{'}s each value to a mono it
and then it traverses the container and combines all these monoid
values into one big monoid value and that\textsf{'}s for example aggregation
and any kind of aggregation Sun Oven integer list or some pointer
so the general method with an arbitrary monoid which is generic in
the monoid is called faulkner the type signature of seek becomes simple
like this so and that is just to concatenate all monoid values in
the container into one using the monoidal operation so this is called
M concat now there aren't any more laws because the laws are about
combining oh the identity law will be automatically satisfied and
the composition law is trivial because you can't compose these things
because there is it takes two more nodes compose them you get the
second one the first one is just just going and so all these laws
are trivially satisfied there aren't any laws anymore for this foldable
operation for the fold map there aren't any laws and all traversable
functors have these operations now nevertheless it\textsf{'}s convenient to
define the foldable typeclass that has these operations like fold
map and M concat and fold left because you could traverse the containers
in different order and that would be different implementations even
though there are no laws and it\textsf{'}s basically a consequence of traversable
so there aren't any factors that are foldable but not reversible so
all polynomial factors and only the polynomial functions are both
foldable and traverses and and so nevertheless it\textsf{'}s convenient to
define this typeclass because you can have different implementations
of fold for different order of the traverses and finally let me show
how to define the fold method and that\textsf{'}s a trick where you take this
as your manual type now this is a mono eight where B is an arbitrary
fixed type and these functions are just concatenated by composition
and the identity value is the identity function now if you substitute
into the following up into this substitute B to B you get this type
signature it\textsf{'}s a curried function with one two three arguments so
if you just rearrange these arguments you see this is exactly the
type signature fold left it has your container it has initial value
it has the update function and it returns the final accumulated value
so monoid is gone we have a specific one right here the arbitrament
already that is is gone until we have an arbitrary type B so if we
put B before this argument and it\textsf{'}s not obvious where the money went
but it is just a consequence of a signature of Traverse where we first
put them on oil in it and then you specify to this node and so for
this reason every foldable has a foolproof method for map and M concat
and every traversable also has them and so now we know which factors
are foldable and reversible so let\textsf{'}s ask our contra functors useful
or profanity was useful in the same capacity because you could imagine
that you want a conscious factor that you Traverse with respect to
a function or my servers and my answer to that is after several analysis
is that they're not very useful here\textsf{'}s why let\textsf{'}s take a contra fantasy
now let\textsf{'}s try to do a seek on it so that would be this kind of time
signature now if I have a CFA I can control map with the pure method
the pure F which is a to FA I can come up with that to get CFA to
see a so I can get down and then I can put that into the F using pure
F so I get F CA and in fact there is no other way of doing this generically
now it seems that wall control factors are automatically reversible
and I could even say F doesn't have to be a factor it\textsf{'}s not using
map well if I'm just using pure life and so it\textsf{'}s it can be arbitrary
Pro functor as long as it has pure it\textsf{'}s actually applicative proof
functor or even just appointed with a pure so it can be just just
like that so is that useful well in my view it\textsf{'}s not useful because
I completely ignore all effects of EV I'm using pure of F so whatever
F ahead here I'm ignoring its effects I'm never going to have any
effect full value here either it\textsf{'}s going to be always a pure so it\textsf{'}s
not very useful and in the other direction you can't do it f seiei
to CFA because you can't extract out of F necessary now if you have
a proof factor see anything things are not workable here\textsf{'}s why consider
this simple example a simplest growth factor just neither a functor
nor control factor because it has a in both covariant and contravariant
positions now we need a sequence function of this type signature but
that\textsf{'}s it\textsf{'}s it\textsf{'}s impossible we cannot get an F of a to a unless we
somehow get an a to a first but we can't there\textsf{'}s no way to extract
a to a out of here so I can't get an A you can't get an F of a and
so there\textsf{'}s no way to do this so so the only way to implement this
type signature is to return pure F of identity here but that ignores
its argument and functions that ignore its argument will not respect
the identity law they will not preserve information and finally let\textsf{'}s
try to try to traverse profile too with respect to proof hunters and
we find again that the only way to do that is to ignore all effects
here are two examples so consider this contra factor and this contra
factor then now if it\textsf{'}s applicative it means that s is a mono it and
so this type the only way to implement is to return an empty value
of F on the right ignoring all of Fame and the second example is you
take this contra function and this function and again you can show
that the only way to implement sequence of this type signature is
to always return empty option and again that would ignore its argument
and so it\textsf{'}s most likely you're not going to be useful because we ignore
all the effects will never return anything that\textsf{'}s not empty and so
on so we we are able to implement these type signatures let\textsf{'}s see
I have I have this test code here for a check that we can actually
implement only one type signature and that it returns an empty option
in both of these counter examples so take a look at this in more detail
if you feel like but I'm of the opinion right now that since all of
these examples show that I have to ignore all effects I have to return
a value that ignores all the input data or ignores all the effects
in the factor or per factor f that\textsf{'}s probably not a very useful implementation
but there isn't any other also note that the laws of jurors suppose
the notes actually say that effects of f cannot be ignored they don't
actually say that so you can traverse each element once and that\textsf{'}s
guaranteed but then you can just ignore the effects maybe well for
some functors that\textsf{'}s possible as we have just seen so the conclusion
is that traversable contractors and pro founders aren't really so
great and aren't very useful let\textsf{'}s look at some examples where we
use traversable factors the first example is we can convert any traversable
factor data structure to a list to do that we will actually have a
trick and that will all define a list as a monoid not as a type constructor
is a constant type but as a monoid so it will be a constant factor
so to express this i define the type constructor z which has a type
parameter b but it\textsf{'}s equal to list of c or c is a fixed type parameter
up here so it\textsf{'}s not depending on B so this will be a constant factor
which are defined like this so it doesn't change anything the type
parameter is not used and it\textsf{'}s applicative as well as a constant for
a factor that is a monoid so I define the evocative instance using
the rajab and this is an eel and a monoid composition or concatenation
of lists and just for fun I want to define the monoid composition
in the opposite order {[}Music{]} so having defined this and now have
Z as a applicative factor and I can use chillers which is this function
that I defined for convenience on the reversible value and so now
I Traverse F of a which is this L of C probably better cold now this
L of C is of type unknown type constructor hell so we Traverse it
with a function that takes a value of type C and returns a list containing
a single element C the result of traversing is that for each element
from the data container L will have a list of a single element and
then we will concatenate all these lists because that\textsf{'}s the effect
located factor Z and we need to combine all these effects for it all
values in the container health and the combining is done using the
monoid composition and so then the result will be a list of all elements
from the container l so let\textsf{'}s see how that works we define L like
this we have seen the implementation already and now we define a value
of this type the value is triple 1 2 3 and then we did recall to listing
it and the result is a list of 3 2 1 because we have defined the opposite
order here so if we define a straightforward order we would have a
list of 1 2 3 otherwise we have a list of 3 2 1 so in this way we
see that any traversable factor is a data container that can be converted
to lists and the order of list elements depends on the order of traversal
which is specified by the traversal hibiclens instance the second
example is to aggregate data from a tree by using a fold so in fact
since fold is an operation of traverse where the negative factor is
a constant factor here we have just used fold map essentially we don't
need traversable here we it\textsf{'}s efficient to have foldable we could
have defined this function by folding with with a list but let\textsf{'}s continue
using traverse just so that we understand these are very much related
to each other so here\textsf{'}s a definition of a simple tree tag it has a
leaf of type A and it has a branch with two trees now we define the
function instance in the usual way and then we define the traversal
industry which is defined also in the usual way this is a recursive
construction construction for and so I'm just writing it out this
follows from the recursive construction if we have a leaf and we just
put the leaf constructor under F and if we have a branch and we run
sequence on both branches this is a recursive call to the sequence
function that we're defining right now and then we zip the results
and put the result in to pull under the branch here are some example
values of this type so this is a tree of integers and this is a tree
of strings and I driven this picture to visualize this tree t2 has
a more complicated structure T 1 is just 1 2 \& 3 it\textsf{'}s easy to see
what T 2 is not so easy to see anymore so let\textsf{'}s fold over T 1 so T
1 is 1 2 3 in order to fold let\textsf{'}s define a 1:08 for integers which
will be a multiplication and what\textsf{'}s the fold map with a function that
squares each element so the result of fold map will be the product
of squares of all and of all the elements stored in the tree which
will be 1 times 4 times 9 which is 36 the third example is to decorate
a tree with order labels obtained from a depth-first traversal so
for instance this tree if we Traverse a depth first then first we'll
reach a and we'll reach B then C and then D so then the order of traversal
is 1 2 3 4 and so we expect these labels to be attached so how do
we attach labels like this way it\textsf{'}s not just a map map would we can
certainly attach constant or a function of each of the label each
of the element but that\textsf{'}s not we won't want to attach a value that
depends on the traversal so that\textsf{'}s the Traverse and since it depends
on the traversal we need to compute it as we go and so that\textsf{'}s I mean
we need to maintain state as we go as we traverse the tree and maintain
in the state can be done with the state monad so we use that statement
others duplicative function and that expresses the effect that we
are traversing with so to visualize this traversal I could say that
we are first so let\textsf{'}s Traverse t2 since we have it on the screen so
we first Traverse t2 by visiting a and we have some function that
takes a and returns a monad value or a purgative function value in
general now we done this being and we combine the effects so we have
four different effects in these four different a purgative functionaries
we need to combine them using zip and so the result is an F of tuple
of a B here and then it\textsf{'}s going to be this nested tuple so it\textsf{'}s f
of domestic tuple and then we have to recreate the three structure
out of that nested tuple by remembering where these a B and C and
D were in it in the tree and we have to reconstruct a tree under F
so this and I'm not going to write much more detail with basic analysis
so it\textsf{'}s to be reconstructed as the value of this type after after
zip and you would just have an F of some tuple and you could have
converted it to a list first but then you completely lose the structure
of the tree so the traverse operation must somehow keep track of the
structure of the continue and you recreate it after zipping so it
zips the effects so the effects are combined linearly in the traversal
order using the zip and then after the effects are combined effects
are mono it always of some kind so they are combined linearly there
is no tree structure only effects but there is tree structure on the
values ABCD or whatever values you get after you transform them with
the function f and so those values have to be arranged so that could
be of type B those values have to be arranged into the same tree structure
as before so now we are going to use a state monad as the positive
effect and statement others applicative and you will use if statement
values together what happens is that you update the state each state
mullet value could update the state in some way and so this is a linear
sequence of effects that you need to combine by updating the state
first using this updater than using this updater and so on in the
linear order of traversal and that\textsf{'}s what well we will use now we'll
have a monadic value make label so we define first of all a state
monad I'm using the Katz library with integer state so for convenience
I'll call this s right now just in this test so I'll I define a make
label value which is a monadic value that updates the state and returns
also this value that is the previous value before I'm dating I could
return the next value it doesn't really matter and that\textsf{'}s going to
be melodic value that we are going to produce for each of the elements
of the tree now we actually need to represent this as an applicative
factor so let me just quickly convert to my zip typeclass which is
my my version of applicative of course the cats library already has
a positive instance for this but it\textsf{'}s easier to use my my typeclasses
because they're very small they don't have a lot of methods and it\textsf{'}s
easy to see what they do so we need to define the wrapped unit which
is just a pure of state factor and we need to define zip which is
defined in a regular way for anyone that this is just combined the
two effects in this order we could have changed the order here so
this is this is a freedom but let\textsf{'}s just keep this order and this
function will allow us to produce her with zip tie plus instance from
a statement well you know what\textsf{'}s right a tree traversal so a tree
traversal we already implemented up here so I just copied this code
in I don't think I needed to duplicate nice code we can just remove
this code so how do we use that so we have t2 which was defined before
that we Traverse using the effect type yes so this is the apocatip
factor that we have now a wizard instance for we have this was a probe
s and the result type is going to be a tuple of string int so we're
adding an integer label to a string type of the tree which which t2
has 32 think and the type is tray of string so the function with which
which reverse takes a label and prefer this make label operation which
is a monadic value and then we map now this magnetic value returns
the integer and then we map that integer into a tuple with the previous
tree leaf is really called leaf action to be more clear so we map
a tree leaf to a tuple of leaf and label and we get out of the Monad
so now this is a value of type s of tree of string int so this is
a result of traversal and now we need to run it so we run the statement
out on the initial value one and we extract the value out of it so
this is that cats library API two is to extract values from statement
and the result is as expected so it\textsf{'}s a labeled with 1v we still in
soon so in this way we can decorate the tree with labels that are
computed as we go on in a traversal the next example is templum and
skin functions scan map and scan left so we have seen that fold can
be implemented if you have it reversible and scan also follows from
properties of traversable so scan map is something I made up it\textsf{'}s
a method analogous to fold map where you have a monoid and you map
your data to a monoid type and then the result is that unlike fold
map it\textsf{'}s not just one monoid value that aggregates everything you
don't just accumulate the final value you keep all the intermediate
accumulated values and put them back into your container so this is
like the scan function so how does it work we use a statement again
it\textsf{'}s state is the MU node so initially it will be empty and then we
will accumulate a monoid state so this is how we accumulate we get
the previous state and we set the new accumulated state and then we
get again in new states so that we have it on hand and then we Traverse
with a function that is f which is this given F followed by accumulate
and so the result is s of Z which is a state that accumulates Z and
also gives it as a value so after traversing we get a value of type
s of L of Z\textsf{'}s and then we run it on an initially empty value and that\textsf{'}s
what we return extracting out of the state monad after we have run
it so here\textsf{'}s the test let\textsf{'}s use a string as monoid so we imported
a cat\textsf{'}s instance just for standard string 108 and let\textsf{'}s do a scan
map on t2 with no transformation so then we're just going to accumulate
the string as we go so we get a a b a b c a b c d so that\textsf{'}s the result
of scan map and we can also implement the standard scan left function
which is very similar except we don't accumulate a monoid now we just
accumulate a value of some arbitrary type z and we do the same thing
as we did before very similar code and here\textsf{'}s a test so we let\textsf{'}s accumulate
the length of strings in the in the tree so then we'll get one two
three four because all all strings have length 1 and the last example
is the traverse a rigid tree or non monadic tree as I don't think
vegetation well it is a widely used word to describe this kind of
tree but basically imagine a tree that must have all branches fall
so it cannot be unbalanced then it must be full it\textsf{'}s a binary tree
that must have one two four eight 16 32 and so on elements and no
other number of elements can be in the tree that\textsf{'}s very rigid in its
shape this recursive type equation he finds it and you can easily
see why so it\textsf{'}s either a or it\textsf{'}s the same T of any times a so then
it\textsf{'}s either a times a or the same tree of a times a times a times
a so then it\textsf{'}s for A\textsf{'}s or eight A\textsf{'}s or sixteen is consumed so it\textsf{'}s
it\textsf{'}s at this infinite disjunction with 1 a or 2 is 4 4 8 or 8 8 and
so on so that\textsf{'}s that\textsf{'}s why this equation works not we don't have a
construction where I did not write down the construction that can
generalize this tree this kind of construction could be generalized
like this where you have a recursive equation with some buy furniture
yes and the second type 300 - 2 s is not a la as it was previously
not just LA but it\textsf{'}s L of some RA where R is another function so in
this case it will be a pair but in general it could be some other
function and that function must be both applicative and reversible
and if so then I believe if this construction will work as well so
this is a more general recursive construction let\textsf{'}s see the code for
this so the definition of the tree type is this it has a single element
in the branch which is a tree of the pair parameter instead so we
substitute to the type parameter which is kind of non-trivial so we
can define the functor instance which is a bit of work because now
we need to map this branch of type 3 of a a we need to map it recursively
with a function that map\textsf{'}s a a to be me but we don't have that function
we have a to be so we need to construct a function that map\textsf{'}s a a
to be B out of F and the traversal is again a depth-first traversal
where the leaf is handled the same way as before but the branch is
handled differently because we actually have this kind of type and
we need to get this so we need to do we need to use zip for F in order
to pull F out of the tuple and so we first zip the FS together like
this and this needs to be done under the B so it\textsf{'}s a b fa fa map of
the zip so that\textsf{'}s going to have B of F of praa and then we sequence
that so we get F of of PNA and then we wrap it into the branch so
that\textsf{'}s how traversal works and certainly we tested you know this is
how we would create this data structure so it\textsf{'}s always branch branch
branch branch and finally leaf with a very large may be nested tuple
so this is why this tree is so rigid he does not have cannot have
unbalanced shape so that works we use in mono omoide instance for
integer and do a fold map with identity to get the sum of all these
numbers so these examples indicate that we can do a lot of things
with Traverse but there are there are actually things we cannot do
using each others because the Traverse separation we remember it is
an operation that needs to be generic in the effect against which
we Traverse in this applicative factor so two things that cannot be
implemented the traversal which you find interesting but this is so
or the breadth-first traversal for a tree so imagine a tree of this
shape and you want to do breadth-first traversal so you traverse first
at this level you get one and you traverse at this level you get to
an elite reverse at this level now that can cannot be expressed as
a traversal with some effect because as I indicated here to visualize
a traversal we need to imagine that we are computing a large linear
sequence of effects into one combining a large linear sequence of
effects into one big effect and now in order to express this you would
have to have to know for example here that this two must be two and
not one now this effect cannot come from just recursive traversal
you cannot just take a usual recursive depth first traversal and run
some more nuts and clever Monat with it and get this other chamber
so now we certainly probably can implement it in some other way but
I don't see how you can simply do a recursive function on a tree that
traverses it against an applicative factor and get this effect and
especially when you call recursively on this subtree you need to know
how many nodes will be in some other levels in some other subtree
so it\textsf{'}s not clear how you could possibly get that information from
any kind of monitor or {[}Music{]} applicative function now certainly
you can implement breadth-first traversal but what I'm saying is that
you cannot take some special monad and some recursive function calls
and implement it so easily I hope you can still implement it in some
way that is not too bad as a traversal certainly the standard way
of implementing breadth-first traversal to use a queue or you in queue
each node and then you in queue it to children and then you DQ and
as you dqu in queue more so that can be certainly done in a statement
that that computation but the problem is you need the traversal which
recreates the original structure so if if all you want is twofold
then you can implement fold in a breadth-first while using a queue
like that but what you need is to recreate the original tree structure
and that\textsf{'}s very cursive structure that has a depth-first logic to
it and so that\textsf{'}s the difficulty and certainly you can do it with a
lot of extra work it\textsf{'}s not just a couple of lines of code like it
was for depth-first traversal so at this point I don't have code for
this maybe this can be done in some clever way but I haven't found
it and the second thing that I don't think you can do is to label
depth of the tree using a traversal so again it\textsf{'}s a similar problem
that effects so these aren't going to be effects and obviously you
need some kind of state to hold this one and then update it to two
and two three but effects are concatenated linearly and so it cannot
be that this three is the same as this you cannot avoid concatenating
this effect with this one when you do a traversal because of this
linear nature of combining effects and you cannot avoid combining
effects during traversal and so there\textsf{'}s no way to skip incrementing
the counter when you do choose ourselves so I don't think you can
easily express this kind of labeling and this is trivial to do ad
hoc as a function on a tree but what you want is to have a generic
traversal which is generic in the factor f and just uses the zip and
combines the effects and that\textsf{'}s why traversals are not so flexible
you cannot avoid incrementing your counter you cannot have logic that
says oh here I don't increment because they're at the same level can't
have that logic you must combine the only factors together so I keep
talking about being generic in the function f so let me talk about
it in a more rigorous manner so what we're looking for is to be generic
in the sense that the code for traverse as well as the code for the
sequence in it should not depend on a specific function f it should
only use the methods P R and zip or maybe map as well from F and {[}Music{]}
recall recall how we dealt with a similar problem when we discovered
founders so a factor is a data container that is generic in type of
the data that it holds and the map function or F map is a function
that cannot use specific type information about A or B it must manipulate
this data blindly with no change you take a container you take each
element a very replacement may be and put it back at the same place
but you don't know what types those are and so we expressed the law
of natural T for various functions by saying that F map should come
should commute with those functions so you can transform first or
you can transform later and that\textsf{'}s what it means that it\textsf{'}s generic
in the type of data so similarly generic in the function f means that
if we map a functor F to some other function G then a traverse with
respect to F will be mapped to the Traverse with respect to G so we
need this mapping somehow so this cannot be this can be formulated
mathematically I don't want to spend too much time on this but I just
want to mention this is an important mathematical development perhaps
but this is the first time where we have seen this kind of thing and
we don't have I don't have good notation right now for this natural
transformations between applicative functors and its need to be considered
so here\textsf{'}s the mathematical formulation we consider - Lickety functors
F and G and we consider a natural transformation between them such
that it Maps period of F into pure of G and it maps zip of F into
the project and then we demand that it also should map the Traverse
of F into Traverse of G so if we do that well all we do is we just
say we have a traverse with respect to F let\textsf{'}s apply it and then map
F to G in the result because the result of Traverse is F of something
F of some L of a or L of B so we can map that into G of not using
the natural transformation or we can first map f to G and then apply
Traverse and that should be the same and so this is a natural T with
respect to the functor as a type parameter so it\textsf{'}s a higher-order
type parameter itself otherwise it\textsf{'}s quite similar to not two naturality
and it\textsf{'}s just more complicated because it\textsf{'}s not just any natural transformation
natural transformation by itself has laws already and in addition
to being a natural transformation it needs to preserve the implicative
properties of f so that pure is mapped to pure zip is mapped to zip
and and the laws of course must hold for both implicated factors so
if you look at this notation which is slightly made shorter on purpose
by omitting various type parameters it really looks like lifting from
a function f to G into function of traversing you have to traverse
G but F and G are not types they're type constructors so and this
is not a function this is a natural transformation which is more complicated
object so when we used category theory to describe such liftings we
would usually say this is a morphism in one category and this is a
morphism in another category but a morphism as we define it was always
between two types now we don't have types we have type constructors
so we need basically a morphism between type constructors so that\textsf{'}s
more more precisely between implicative factors so a morphism between
applicative functions that\textsf{'}s what we need to lift into a morphism
of traversable faculties and so that requires a more general definition
of category than what I have given so far we're in them in the definition
I've given so far morphisms were just twisted function types but now
we need morphisms between type constructors or even more restricted
ones morphisms between applicative functions with extra properties
so category theory prescribes this in a general definition of category
can be given which I'm not going to give right now because I don't
see in the use for it other than to indicate the property of natural
T as being similar to previous properties I encourage you to look
at category theory at some point but I will only talk about what I
see it as being useful and so right now it is useful to think about
this as a lifting and the wall is of course need to hold such as identity
and Composition so if we map F to F then we don't change anything
that should be lifted into identity and composition F to G to H mapped
it to composition of those mappings and so on so those are the rigorous
four forms for the naturality war with respect to applicative function
I'm not writing it down in detail because just as naturality with
respect to ordinary type parameter this is this morality with respectively
funter is going to be satisfied automatically by any code that nearly
uses peer and zip and why is that it\textsf{'}s because when we perform a natural
transformation that maps FPR to GD P R and F zip to gzip the code
doesn't actually change at all the code has f as a type parameter
and it just uses pure and zip in the code to cause them as we have
been doing when we implemented the traverse instances for example
so let\textsf{'}s see here is for example we are using zip the zip is from
the function f but F is a type parameter so if we now apply a natural
transformation from F to G for some other negative function G this
code doesn't change at all it still uses the zip and we have the condition
that there\textsf{'}s natural transformation that maps F to G it Maps zip to
zip and peer to peer and and map to map of course that\textsf{'}s the natural
T of natural transformation so our code won't change and our code
therefore is invariant under this natural transformation and so it\textsf{'}s
automatically going to satisfy naturality with respect to the type
parameter F and it also will automatically satisfy not reality with
respect to time over a and that\textsf{'}s why we never need to check these
laws explicitly or even spend so much time writing them explicitly
because they are they are useful sometimes when your reason about
code and prove some other law was then learned are useful but for
coding for implementing you always have natural T automatically so
this concludes chapter 9 and here are some exercises for you to implement
let me just comment and this exercise so I don't want you to prove
laws by hand for this factor T it\textsf{'}s cumbersome but instead use constructions
so we have proved general constructions so if you express T as a recursive
equation using some by factor s and you show that the wife on dress
is by traversable again using constructions of course then you don't
need to and to do the proof yourself if the constructions already
proved once you have found them and similarly for for this construction
you don't need to prove anything by hand this exercise is a little
different but you can try to prove this extra construction that I
mentioned this one but it\textsf{'}s very similar to other constructions and
so it\textsf{'}s not necessarily part of this exercise this concludes chapter
9 
\end{comment}



\part{Advanced level}

\global\long\def\gunderline#1{\mathunderline{greenunder}{#1}}%
\global\long\def\bef{\forwardcompose}%
\global\long\def\bbnum#1{\custombb{#1}}%
\global\long\def\pplus{{\displaystyle }{+\negmedspace+}}%


\chapter{Free typeclass constructions\label{chap:Free-type-constructions}}

Free typeclass constructions (free monoids, free functors, free monads,
free applicative functors, and so on) are used to implement the DSL
(domain-specific language\index{domain-specific language (DSL)})
design pattern. As a first motivation, we will look at how the free
monad\index{free monad} emerges from elaborating a certain kind of
DSL in the functional programming paradigm.

\section{Motivation for the free monad: create a DSL in five stages}

The main point of using a DSL is to separate the description of computations
from the process of their evaluation. One creates a \textbf{DSL program}\index{DSL program}
as a data structure that fully describes what computations and side
effects need to be run but does not actually perform any of those
computations or side-effects. Helper functions are implemented for
creating the DSL program values, for combining several DSL programs
into larger ones, and for interpreting (or \textsf{``}running\textsf{''}) DSL programs.

This design pattern is known as an \textbf{embedded} \textbf{DSL}\index{embedded DSL},
meaning that it is implemented not as a separate new language but
as a library in an existing programming language (e.g., Scala). This
chapter only considers embedded DSLs and calls them just \textsf{``}DSL'\textsf{'}s
for brevity.

We will look at two examples: a DSL for file operations and a DSL
for complex numbers. Refining and refactoring these DSLs to make them
more powerful and safer to use, we will obtain a construction known
as the \textsf{``}free monad\textsf{''}.

\subsection{Stage 1: unevaluated expression trees}

The first example is a DSL for reading and writing files at given
file paths. Direct Scala code for such operations may look like this:\inputencoding{latin9}
\begin{lstlisting}
import java.nio.file.{Files, Paths}
val p = Paths.get("config_location.txt")
val configLocation = Paths.get(new String(Files.readAllBytes(p)))
val config = new String(Files.readAllBytes(configLocation))
\end{lstlisting}
\inputencoding{utf8}The DSL will represent these operations by the case classes \inputencoding{latin9}\lstinline!Val!\inputencoding{utf8},
\inputencoding{latin9}\lstinline!Path!\inputencoding{utf8}, and \inputencoding{latin9}\lstinline!Read!\inputencoding{utf8}:\inputencoding{latin9}
\begin{lstlisting}
sealed trait PrgFile
final case class Val(s: String)   extends PrgFile
final case class Path(p: PrgFile) extends PrgFile
final case class Read(p: PrgFile) extends PrgFile
\end{lstlisting}
\inputencoding{utf8}A DSL program of type \inputencoding{latin9}\lstinline!PrgFile!\inputencoding{utf8}
is a value consisting of many nested case classes:\inputencoding{latin9}
\begin{lstlisting}
val prgFile: PrgFile = Read(Path(Read(Path(Val("config_location.txt")))))
\end{lstlisting}
\inputencoding{utf8}Such values represent an \emph{unevaluated} expression tree corresponding
to the operations that need to be performed. To actually perform those
operations and extract the final \inputencoding{latin9}\lstinline!String!\inputencoding{utf8}
value, we use a \textsf{``}runner\textsf{''} function:\index{runner!for free monads}\inputencoding{latin9}
\begin{lstlisting}
def runFile: PrgFile => String = {
  case Val(s)          => s
  case Path(p)         => "path=" + runFile(p)                // For debugging.
  case Read(Path(p))   => new String(Files.readAllBytes(Paths.get(runFile(p))))
  case x               => throw new Exception(s"Illegal PrgFile operation: $x")
}
\end{lstlisting}
\inputencoding{utf8}To test this code, we prepare a file \inputencoding{latin9}\lstinline!config.txt!\inputencoding{utf8}
containing the text \inputencoding{latin9}\lstinline!version = 1!\inputencoding{utf8}
and a file \inputencoding{latin9}\lstinline!config_location.txt!\inputencoding{utf8}
containing the text \inputencoding{latin9}\lstinline!config.txt!\inputencoding{utf8}.
Then we can run the DSL program and get the result:\inputencoding{latin9}
\begin{lstlisting}
scala> runFile(prgFile)
res2: String = "version = 1"
\end{lstlisting}
\inputencoding{utf8}
The second example is a DSL for calculations with complex numbers.
Begin by implementing a data structure (\inputencoding{latin9}\lstinline!case class Complex!\inputencoding{utf8})
with some operations: \inputencoding{latin9}
\begin{lstlisting}[mathescape=true]
final case class Complex(x: Double, y: Double) {
  def +(other: Complex): Complex = Complex(x + other.x, y + other.y)
  def *(other: Complex): Complex = Complex(x * other.x - y * other.y, x * other.y + y * other.x)
  def conj: Complex = Complex(x, -y)
  def phase: Double = math.atan2(y, x)         // Obtain the phase of a complex number.
  def rotate(alpha: Double): Complex = this * Complex(math.cos(alpha), math.sin(alpha))
}

val a = Complex(1, 2)          // $\color{dkgreen}a = 1 + 2 i$
val b = a * Complex(3, -4)     // $\color{dkgreen}b = (1 + 2 i)(3 - 4 i) = 11 + 2 i$
val c = b.conj                 // $\color{dkgreen}c = 11 - 2 i$

scala> c
res0: Complex = Complex(11.0, -2.0)

scala> c.rotate(Complex(0, 1).phase) // Multiply by $\color{dkgreen}i$
res1: Complex = Complex(2.000000000000001, 11.0)
\end{lstlisting}
\inputencoding{utf8}
Instead of running a complex-number program directly, the DSL pattern
first creates a data structure that contains an unevaluated expression
tree describing what needs to be computed. Separate case classes are
used for different operations as well as for inserting a literal \inputencoding{latin9}\lstinline!Complex!\inputencoding{utf8}
value (\inputencoding{latin9}\lstinline!Val!\inputencoding{utf8}):\inputencoding{latin9}
\begin{lstlisting}
sealed trait PrgComplex
final case class Val(c: Complex)                      extends PrgComplex
final case class Add(p1: PrgComplex, p2: PrgComplex)  extends PrgComplex
final case class Mul(p1: PrgComplex, p2: PrgComplex)  extends PrgComplex
final case class Conj(p: PrgComplex)                  extends PrgComplex
final case class Phase(p: PrgComplex)                 extends PrgComplex
final case class Rotate(p: PrgComplex, a: PrgComplex) extends PrgComplex
\end{lstlisting}
\inputencoding{utf8}
Complex-number calculations are now represented by nested case classes:\inputencoding{latin9}
\begin{lstlisting}
val prgComplex1: PrgComplex = Conj(Mul(Val(Complex(1, 2)), Val(Complex(3, -4))))
val prgComplex2: PrgComplex = Rotate(prgComplex1, Phase(Val(Complex(0, 1))))
\end{lstlisting}
\inputencoding{utf8}A simple runner for the DSL programs of type \inputencoding{latin9}\lstinline!PrgComplex!\inputencoding{utf8}
can be written like this:\inputencoding{latin9}
\begin{lstlisting}
def runComplex: PrgComplex => Complex = {
  case Val(c)             => c
  case Add(p1, p2)        => runComplex(p1) + runComplex(p2)
  case Mul(p1, p2)        => runComplex(p1) * runComplex(p2)
  case Conj(p)            => runComplex(p).conj
  case Phase(p)           => Complex(runComplex(p).phase, 0) // Pretend the phase is a Complex value.
  case Rotate(p, alpha)   => runComplex(p).rotate(runComplex(alpha).x) // Here alpha must be a phase.
}
\end{lstlisting}
\inputencoding{utf8}We can now apply \inputencoding{latin9}\lstinline!runComplex!\inputencoding{utf8}
to some DSL programs and get the results as \inputencoding{latin9}\lstinline!Complex!\inputencoding{utf8}
values:\inputencoding{latin9}
\begin{lstlisting}
scala> runComplex(prgComplex1)
res2: Complex = Complex(11.0, -2.0)

scala> runComplex(prgComplex2)
res3: Complex = Complex(2.000000000000001, 11.0)
\end{lstlisting}
\inputencoding{utf8}
Although the DSLs are simple, we already get several benefits. By
representing file operations as values of type \inputencoding{latin9}\lstinline!PrgFile!\inputencoding{utf8}
and complex-number calculations as values of type \inputencoding{latin9}\lstinline!PrgComplex!\inputencoding{utf8},
we can compose, verify, or optimize our DSL programs \emph{before}
running them. We may also implement different runners that execute
the same DSL program on a remote computer, on a distributed file system,
on GPUs instead of CPUs, or in a test-only sandbox environment, with
logging or benchmarking.

\subsection{Stage 2: implementing type safety in the DSLs}

The two DSLs defined in the previous section are not fully type-checked\index{type checking}
at compile time. This becomes clear by looking at the DSL program
\inputencoding{latin9}\lstinline!prgFile!\inputencoding{utf8} shown
in the previous section. The code of \inputencoding{latin9}\lstinline!runFile!\inputencoding{utf8}
assumes that the \inputencoding{latin9}\lstinline!Read!\inputencoding{utf8}
operation is always applied to a \inputencoding{latin9}\lstinline!Path!\inputencoding{utf8}.
However, this assumption is not enforced by the DSL. When composing
a larger DSL program from separately defined parts, one could by mistake
create an invalid DSL program such as \inputencoding{latin9}\lstinline!Read(Read(Val("file")))!\inputencoding{utf8}
where the \inputencoding{latin9}\lstinline!Read!\inputencoding{utf8}
operation is used incorrectly. Running this program causes a run-time
error:\inputencoding{latin9}
\begin{lstlisting}
scala> runFile(Read(Read(Val("file"))))
java.lang.Exception: Illegal PrgFile operation: Read(Read(Val(file)))
\end{lstlisting}
\inputencoding{utf8}All DSL program values are of the same type (\inputencoding{latin9}\lstinline!PrgFile!\inputencoding{utf8})
regardless of their intended meaning. So, the Scala compiler cannot
verify that we are using the DSL correctly. It would be better if
the DSL program \inputencoding{latin9}\lstinline!Read(Read(Val("file")))!\inputencoding{utf8}
failed to compile. To achieve that, we need to use different Scala
types for DSL programs returning a \inputencoding{latin9}\lstinline!String!\inputencoding{utf8}
and a \inputencoding{latin9}\lstinline!Path!\inputencoding{utf8}.
So, let us replace the type \inputencoding{latin9}\lstinline!PrgFile!\inputencoding{utf8}
by a type constructor \inputencoding{latin9}\lstinline!PrgFile[A]!\inputencoding{utf8},
representing a DSL program that will return a value of type \inputencoding{latin9}\lstinline!A!\inputencoding{utf8}
when we run it:\inputencoding{latin9}
\begin{lstlisting}
import java.nio.file.{Path => JPath}
import java.nio.file.{Files, Paths}
sealed trait PrgFile[A]
final case class Val(s: String)           extends PrgFile[String]
final case class Path(p: PrgFile[String]) extends PrgFile[JPath]
final case class Read(p: PrgFile[JPath])  extends PrgFile[String]

def runFile[A]: PrgFile[A] => A = {
  case Val(s)   => s
  case Path(p)  => Paths.get(runFile(p))
  case Read(p)  => new String(Files.readAllBytes(runFile(p)))
}
\end{lstlisting}
\inputencoding{utf8}Our example DSL is now fully type-safe.\index{type safety} Invalid
programs are rejected at compile time:\inputencoding{latin9}
\begin{lstlisting}
val prgFile: PrgFile[String] = Read(Path(Read(Path(Val("config_location.txt")))))

scala> runFile(prgFile)
res3: String = "version = 1"

scala> runFile(Read(Read(Val("file"))))
type mismatch;
 found   : Val
 required: PrgFile[java.nio.file.Path]
val res4 = runFile(Read(Read(Val("file"))))
                                ^
Compilation Failed 
\end{lstlisting}
\inputencoding{utf8}
A similar problem exists in our DSL for complex numbers. The code
of \inputencoding{latin9}\lstinline!runComplex!\inputencoding{utf8}
assumes that the \inputencoding{latin9}\lstinline!Rotate!\inputencoding{utf8}
operation is always applied to a \inputencoding{latin9}\lstinline!Phase!\inputencoding{utf8},
but the DSL does not enforce that assumption. Both arguments of the
\inputencoding{latin9}\lstinline!Rotate!\inputencoding{utf8} operation
are values of type \inputencoding{latin9}\lstinline!PrgComplex!\inputencoding{utf8},
and so the Scala compiler is unable to verify that the DSL is being
used correctly. If the programmer uses a plain complex number (\inputencoding{latin9}\lstinline!Val!\inputencoding{utf8})
instead of a \inputencoding{latin9}\lstinline!Phase!\inputencoding{utf8}
value, the program will return an incorrect result \emph{without}
any indication of error:\inputencoding{latin9}
\begin{lstlisting}
val prgComplex3: PrgComplex = Rotate(prgComplex1, Val(Complex(0, 1)))     // Forgot Phase()!

scala> runComplex(prgComplex3)    // The result is incorrect, but there is no error message!
val res5: Complex = Complex(11.0, -2.0)
\end{lstlisting}
\inputencoding{utf8}
To achieve type safety, we add a type parameter to \inputencoding{latin9}\lstinline!PrgComplex!\inputencoding{utf8}.
A DSL program of type \inputencoding{latin9}\lstinline!PrgComplex[A]!\inputencoding{utf8}
will return a value of type \inputencoding{latin9}\lstinline!A!\inputencoding{utf8}
when run. (So, the new DSL programs will be able to compute values
of any type, not only of type \inputencoding{latin9}\lstinline!Complex!\inputencoding{utf8}.)
The new definitions of the case classes are:\inputencoding{latin9}
\begin{lstlisting}
sealed trait PrgComplex[A]
final case class Val(c: Complex)                                       extends PrgComplex[Complex]
final case class Add(p1: PrgComplex[Complex], p2: PrgComplex[Complex]) extends PrgComplex[Complex]
final case class Mul(p1: PrgComplex[Complex], p2: PrgComplex[Complex]) extends PrgComplex[Complex]
final case class Conj(p: PrgComplex[Complex])                          extends PrgComplex[Complex]
final case class Phase(p: PrgComplex[Complex])                         extends PrgComplex[Double]
final case class Rotate(p: PrgComplex[Complex], a: Phase)              extends PrgComplex[Complex]
\end{lstlisting}
\inputencoding{utf8}The revised code of \inputencoding{latin9}\lstinline!runComplex!\inputencoding{utf8}
is clearer because phases are no longer wrapped in a \inputencoding{latin9}\lstinline!Complex!\inputencoding{utf8}
type:\inputencoding{latin9}
\begin{lstlisting}
def runComplex[A]: PrgComplex[A] => A = {
  case Val(c)             => c
  case Add(p1, p2)        => runComplex(p1) + runComplex(p2)
  case Mul(p1, p2)        => runComplex(p1) * runComplex(p2)
  case Conj(p)            => runComplex(p).conj
  case Phase(p)           => runComplex(p).phase
  case Rotate(p, alpha)   => runComplex(p).rotate(runComplex(alpha))
}
\end{lstlisting}
\inputencoding{utf8}The example programs, \inputencoding{latin9}\lstinline!prgComplex1!\inputencoding{utf8}
and \inputencoding{latin9}\lstinline!prgComplex2!\inputencoding{utf8},
do not change except for their type:\inputencoding{latin9}
\begin{lstlisting}
val prgComplex1: PrgComplex[Complex] = Conj(Mul(Val(Complex(1, 2)), Val(Complex(3, -4))))

val prgComplex2: PrgComplex[Complex] = Rotate(prgComplex1, Phase(Val(Complex(0, 1))))

scala> runComplex(prgComplex1)
res6: Complex = Complex(x = 11.0, y = -2.0)

scala> runComplex(prgComplex2)
res7: Complex = Complex(x = 2.000000000000001, y = 11.0)
\end{lstlisting}
\inputencoding{utf8}Since \inputencoding{latin9}\lstinline!Rotate!\inputencoding{utf8}
now requires a \inputencoding{latin9}\lstinline!PrgComplex[Double]!\inputencoding{utf8},
forgetting to use \inputencoding{latin9}\lstinline!Phase!\inputencoding{utf8}
will cause a type error:\inputencoding{latin9}
\begin{lstlisting}
scala> val prgComplex3 = Rotate(prgComplex1, Val(Complex(0, 1)))
<console>:1: type mismatch;
 found   : Val
 required: PrgComplex[Double]                                                              
  case Phase(p)           => runComplex(p)
                                       ^ 
Compilation Failed
\end{lstlisting}
\inputencoding{utf8}

\subsection{Stage 3: implementing bound variables}

Another limitation of our DSLs is that we cannot define and use new
variables within a DSL program. We also cannot use any non-DSL code,
such as a library for file manipulation, a numerical algorithms library
for complex numbers, or any other Scala code. 

Both of these limitations would be removed if the DSL could define
\emph{new Scala variables} and set them to values computed within
the DSL. We would then be able to add arbitrary Scala code for working
with those variables. Let us see how this feature may be implemented.

As an example, consider the task of reading a file only if it exists
and reporting an error otherwise. A direct (non-DSL) code for this
computation looks like this:\inputencoding{latin9}
\begin{lstlisting}
val p = Paths.get("config_location.txt")
val result = if (Files.exists(p)) new String(Files.readAllBytes(p)) else "No file."
\end{lstlisting}
\inputencoding{utf8}Trying to translate the above code to the \inputencoding{latin9}\lstinline!PrgFile!\inputencoding{utf8}
DSL, we find that we need to set the Scala variable \inputencoding{latin9}\lstinline!p!\inputencoding{utf8}
to the path computed by a previous operation. Here is a first attempt:\inputencoding{latin9}
\begin{lstlisting}
val p: JPath = runFile(Path(Val("config_location.txt")))
val prg: PrgFile[String] = if (Files.exists(p)) Read(p) else Val("No file.")
\end{lstlisting}
\inputencoding{utf8}There are two problems with this code. The first problem is that \inputencoding{latin9}\lstinline!Read(p)!\inputencoding{utf8}
does not compile because the argument of \inputencoding{latin9}\lstinline!Read!\inputencoding{utf8}
should be a \inputencoding{latin9}\lstinline!PrgFile[JPath]!\inputencoding{utf8}
rather than just a \inputencoding{latin9}\lstinline!JPath!\inputencoding{utf8}.
To solve this problem, we generalize the \inputencoding{latin9}\lstinline!Val!\inputencoding{utf8}
class to hold values of arbitrary type:\inputencoding{latin9}
\begin{lstlisting}
final case class Val[A](a: A) extends PrgFile[A]
\end{lstlisting}
\inputencoding{utf8}We then replace \inputencoding{latin9}\lstinline!Read(p)!\inputencoding{utf8}
by \inputencoding{latin9}\lstinline!Read(Val(p))!\inputencoding{utf8}
in the code above. The code of \inputencoding{latin9}\lstinline!run!\inputencoding{utf8}
remains unchanged.

The second problem is that we are using \inputencoding{latin9}\lstinline!runFile!\inputencoding{utf8}
in the middle of a DSL program. This breaks the assumption that a
DSL program merely describes the computations without executing them.
The use of a specific runner inside a DSL program will also prevent
us from being able to apply a different runner to the entire DSL program
later.

To solve this problem, we introduce a new DSL operation that binds
Scala variables to values returned by other DSL operations. A standard
way of creating \index{bound variable}bound variables is by using
those variables as arguments of nameless functions. So, let us consider
the nameless function:\inputencoding{latin9}
\begin{lstlisting}
{ p => if (Files.exists(p)) Read(Val(p)) else Val("No file.") }
\end{lstlisting}
\inputencoding{utf8}This function creates a local variable \inputencoding{latin9}\lstinline!p!\inputencoding{utf8}
and uses it directly in an expression of type \inputencoding{latin9}\lstinline!PrgFile[String]!\inputencoding{utf8}.
We can call this function at the time when the entire DSL program
is run. So, we replace the code \inputencoding{latin9}\lstinline!val p = runFile(...)!\inputencoding{utf8}
by a new DSL operation that we will call \inputencoding{latin9}\lstinline!Bind!\inputencoding{utf8}.
The above code that computes \inputencoding{latin9}\lstinline!p!\inputencoding{utf8}
and \inputencoding{latin9}\lstinline!prg!\inputencoding{utf8} is
replaced by:\inputencoding{latin9}
\begin{lstlisting}
val prg2: PrgFile[JPath] = Path(Val("config.txt"))
val prg: PrgFile[String] = Bind(prg2){ p => if (Files.exists(p)) Read(Val(p)) else Val("No file.") }
\end{lstlisting}
\inputencoding{utf8}Here, the case class \inputencoding{latin9}\lstinline!Bind!\inputencoding{utf8}
has two curried arguments: a value of type \inputencoding{latin9}\lstinline!PrgFile[JPath]!\inputencoding{utf8}
and a function of type \inputencoding{latin9}\lstinline!JPath => PrgFile[String]!\inputencoding{utf8}.
That function will be called by \inputencoding{latin9}\lstinline!runFile!\inputencoding{utf8}
when we run the DSL program. At that time, the argument \inputencoding{latin9}\lstinline!p!\inputencoding{utf8}
will be set to the value obtained by running \inputencoding{latin9}\lstinline!prg2!\inputencoding{utf8}.
We will implement this logic in the new code for \inputencoding{latin9}\lstinline!runFile!\inputencoding{utf8}
(shown below).

In a different DSL program, we may need to use different types instead
of \inputencoding{latin9}\lstinline!String!\inputencoding{utf8} and
\inputencoding{latin9}\lstinline!JPath!\inputencoding{utf8}. So,
let us replace those types by type parameters \inputencoding{latin9}\lstinline!A!\inputencoding{utf8}
and \inputencoding{latin9}\lstinline!B!\inputencoding{utf8}. The
declaration of the class \inputencoding{latin9}\lstinline!Bind!\inputencoding{utf8}
becomes:\inputencoding{latin9}
\begin{lstlisting}
final case class Bind[A, B](pa: PrgFile[B])(f: B => PrgFile[A]) extends PrgFile[A]
\end{lstlisting}
\inputencoding{utf8}
After adding the \inputencoding{latin9}\lstinline!Bind!\inputencoding{utf8}-handling
code to the runner, the implementation of the DSL is:\inputencoding{latin9}
\begin{lstlisting}
sealed trait PrgFile[A]
final case class Val[A](a: A)                                       extends PrgFile[A]
final case class Bind[A, B](pa: PrgFile[B])(val f: B => PrgFile[A]) extends PrgFile[A]
final case class Path(p: PrgFile[String])                           extends PrgFile[JPath]
final case class Read(p: PrgFile[JPath])                            extends PrgFile[String]

def runFile[A]: PrgFile[A] => A = {
  case Val(a)          => a
  case bind@Bind(pa)   => runFile(bind.f(runFile(pa)))
  case Path(p)         => Paths.get(runFile(p))
  case Read(p)         => new String(Files.readAllBytes(runFile(p)))
}
\end{lstlisting}
\inputencoding{utf8}
Let us explain certain features of Scala used in this DSL. Since the
arguments of the constructor \inputencoding{latin9}\lstinline!Bind[A, B]!\inputencoding{utf8}
are curried, the first curried argument (of type \inputencoding{latin9}\lstinline!PrgFile[B]!\inputencoding{utf8})
will allow the Scala compiler to determine the type parameter \inputencoding{latin9}\lstinline!B!\inputencoding{utf8}
automatically. At the same time, \inputencoding{latin9}\lstinline!Bind!\inputencoding{utf8}\textsf{'}s
second curried argument may be written as a nameless function in a
separate set of braces:\inputencoding{latin9}
\begin{lstlisting}
val prg: PrgFile[String] = Bind(prg2){ p => if (Files.exists(p)) Read(Val(p)) else Val("No file.") }
\end{lstlisting}
\inputencoding{utf8}Because the type parameter \inputencoding{latin9}\lstinline!B!\inputencoding{utf8}
is already fixed, the argument \inputencoding{latin9}\lstinline!p!\inputencoding{utf8}
may be written without a type annotation (in this case, it would have
been \inputencoding{latin9}\lstinline!p:JPath!\inputencoding{utf8}).
This makes the DSL easier to use. However, the underlying implementation
becomes more complicated. Scala\textsf{'}s pattern matching syntax is limited
to the first curried argument. So, the second curried argument of
\inputencoding{latin9}\lstinline!Bind!\inputencoding{utf8} needs
a \inputencoding{latin9}\lstinline!val!\inputencoding{utf8} declaration
and is accessed using another pattern variable (here called \inputencoding{latin9}\lstinline!bind!\inputencoding{utf8}).
Finally, the type parameters \inputencoding{latin9}\lstinline!A!\inputencoding{utf8}
and \inputencoding{latin9}\lstinline!B!\inputencoding{utf8} are defined
such that \inputencoding{latin9}\lstinline!Bind[A, B]!\inputencoding{utf8}
is a subtype of \inputencoding{latin9}\lstinline!PrgFile[A]!\inputencoding{utf8}
and not of \inputencoding{latin9}\lstinline!PrgFile[B]!\inputencoding{utf8}.
This allows Scala\textsf{'}s type checking to handle correctly the unknown
type \inputencoding{latin9}\lstinline!B!\inputencoding{utf8} in the
pattern \inputencoding{latin9}\lstinline!Bind(pa)!\inputencoding{utf8}.

The new \inputencoding{latin9}\lstinline!Bind!\inputencoding{utf8}
operation allows one part of a DSL program to use values computed
by other parts. Since a \inputencoding{latin9}\lstinline!Bind!\inputencoding{utf8}
value contains a \emph{Scala function} (of type \inputencoding{latin9}\lstinline!B => PrgFile[A]!\inputencoding{utf8}),
that function may run arbitrary Scala code, not limited to DSL operations,
in order to compute a result of type \inputencoding{latin9}\lstinline!PrgFile[A]!\inputencoding{utf8}.
All non-DSL code is enclosed inside a \inputencoding{latin9}\lstinline!Bind!\inputencoding{utf8}
and will be evaluated only when the DSL program is run:\inputencoding{latin9}
\begin{lstlisting}
scala> runFile(prg)
res8: String = "version = 1"
\end{lstlisting}
\inputencoding{utf8} As before, a value of type \inputencoding{latin9}\lstinline!PrgFile[A]!\inputencoding{utf8}
describes a computation but does not execute it.

\subsection{Stage 4: refactoring to a monadic DSL}

The next observation is that \inputencoding{latin9}\lstinline!Bind!\inputencoding{utf8}
and \inputencoding{latin9}\lstinline!Val!\inputencoding{utf8} have
the same type signatures as the standard \inputencoding{latin9}\lstinline!flatMap!\inputencoding{utf8}
and \inputencoding{latin9}\lstinline!pure!\inputencoding{utf8} methods
of a monad. For convenience, let us define those methods on the \inputencoding{latin9}\lstinline!PrgFile!\inputencoding{utf8}
class:\inputencoding{latin9}
\begin{lstlisting}
sealed trait PrgFile[A] {
  def flatMap[B](f: A => PrgFile[B]): PrgFile[B] = Bind(this)(f)
  def map[B](f: A => B): PrgFile[B] = flatMap(f andThen PrgFile.pure)
}
final case class Val[A](a: A)                                       extends PrgFile[A]
final case class Bind[A, B](pa: PrgFile[B])(val f: B => PrgFile[A]) extends PrgFile[A]
final case class Path(p: PrgFile[String])                           extends PrgFile[JPath]
final case class Read(p: PrgFile[JPath])                            extends PrgFile[String]

object PrgFile {
  def pure[A](a: A): PrgFile[A] = Val(a) 
}
\end{lstlisting}
\inputencoding{utf8}Here we defined \inputencoding{latin9}\lstinline!map!\inputencoding{utf8}
via \inputencoding{latin9}\lstinline!flatMap!\inputencoding{utf8}
by using a monad\textsf{'}s right identity law, as shown in Eq.~(\ref{eq:express-map-through-flatMap}).

Like other DSL operations, the methods \inputencoding{latin9}\lstinline!map!\inputencoding{utf8},
\inputencoding{latin9}\lstinline!flatMap!\inputencoding{utf8}, and
\inputencoding{latin9}\lstinline!pure!\inputencoding{utf8} do not
run or evaluate any parts of a DSL program. They only create some
nested data structures containing not-yet-called functions. The necessary
logic will be executed when the \inputencoding{latin9}\lstinline!run!\inputencoding{utf8}
method is invoked. At that time, the entire DSL program will be converted
into a result value.

Because \inputencoding{latin9}\lstinline!map!\inputencoding{utf8}
and \inputencoding{latin9}\lstinline!flatMap!\inputencoding{utf8}
are defined, DSL programs can now be written as functor blocks:\inputencoding{latin9}
\begin{lstlisting}
val prg: PrgFile[String] = for {
  p <- Path(Val("config.txt"))
  r <- if (Files.exists(p)) Read(Val(p)) else Val("No file.")
} yield r
\end{lstlisting}
\inputencoding{utf8}Longer DSL programs may be composed from shorter ones:\inputencoding{latin9}
\begin{lstlisting}
def readFileContents(filename: String): PrgFile[String] = for {
  path <- Path(Val(filename))
  text <- if (Files.exists(path)) Read(Val(path)) else Val("No file.")
} yield text

val prg2: PrgFile[String] = for {
  str <- readFileContents("config.txt")
  result <- if (str.nonEmpty) readFileContents(str) else Val("No filename.")
} yield result

scala> runFile(prg2)
res9 : String = "version = 1"
\end{lstlisting}
\inputencoding{utf8}
Another benefit of the monadic refactoring is that \inputencoding{latin9}\lstinline!flatMap!\inputencoding{utf8}
provides a general way of applying a file operation to a value wrapped
under \inputencoding{latin9}\lstinline!PrgFile!\inputencoding{utf8}.
So, we may simplify the case classes \inputencoding{latin9}\lstinline!Path!\inputencoding{utf8}
and \inputencoding{latin9}\lstinline!Read!\inputencoding{utf8} by
removing the \inputencoding{latin9}\lstinline!PrgFile!\inputencoding{utf8}
wrapping:\inputencoding{latin9}
\begin{lstlisting}
final case class Path(p: String) extends PrgFile[JPath]
final case class Read(p: JPath)  extends PrgFile[String]
\end{lstlisting}
\inputencoding{utf8}This simplifies the usage of the DSL since we may write, e.g., just
\inputencoding{latin9}\lstinline!Read(path)!\inputencoding{utf8}
instead of \inputencoding{latin9}\lstinline!Read(Val(path))!\inputencoding{utf8}.
The code of \inputencoding{latin9}\lstinline!runFile!\inputencoding{utf8}
is also shortened:\inputencoding{latin9}
\begin{lstlisting}
def runFile[A]: PrgFile[A] => A = {
  case Val(a)          => a
  case bind@Bind(pa)   => runFile(bind.f(runFile(pa)))
  case Path(p)         => Paths.get(p)
  case Read(p)         => new String(Files.readAllBytes(p))
}
\end{lstlisting}
\inputencoding{utf8}

\subsection{Stage 5: refactoring to reuse common code\label{subsec:Stage-5:-refactoring-monadDSL}}

Let us rewrite our second example (a DSL for complex numbers) in the
same way:\inputencoding{latin9}
\begin{lstlisting}
sealed trait PrgComplex[A] {
  def flatMap[B](f: A => PrgComplex[B]): PrgComplex[B] = Bind(this)(f)
  def map[B](f: A => B): PrgComplex[B] = flatMap(f andThen PrgComplex.pure)
}
object PrgComplex {
  def pure[A](a: A): PrgComplex[A] = Val(a)
}

final case class Val[A](a: A)                                             extends PrgComplex[A]
final case class Bind[A, B](pa: PrgComplex[B])(val f: B => PrgComplex[A]) extends PrgComplex[A]
final case class Add(x: Complex, y: Complex)                              extends PrgComplex[Complex]
final case class Mul(x: Complex, y: Complex)                              extends PrgComplex[Complex]
final case class Conj(x: Complex)                                         extends PrgComplex[Complex]
final case class Phase(p: Complex)                                        extends PrgComplex[Double]
final case class Rotate(p: Complex, p: Phase)                             extends PrgComplex[Complex]

def runComplex[A]: PrgComplex[A] => A = {
  case Val(a)                => a
  case bind@Bind(pa)         => runComplex(bind.f(runComplex(pa)))
  case Add(p1, p2)           => p1 + p2
  case Mul(p1, p2)           => p1 * p2
  case Conj(p)               => p.conj
  case Phase(p)              => p.phase
  case Rotate(p, Phase(a))   => p.rotate(a.phase)
}
\end{lstlisting}
\inputencoding{utf8}
Comparing the code of \inputencoding{latin9}\lstinline!runFile!\inputencoding{utf8}
and \inputencoding{latin9}\lstinline!runComplex!\inputencoding{utf8},
we notice that the \inputencoding{latin9}\lstinline!Val!\inputencoding{utf8}
and \inputencoding{latin9}\lstinline!Bind!\inputencoding{utf8} operations
are implemented in the same way. The differences are in the code for
operations specific to file handling or to complex numbers. So, let
us refactor these data structures, separating the domain-specific
operations and reusing the common code. We can gather all the domain-specific
operations in a new type constructor that we call an \index{effect constructor}\textbf{effect
constructor}. The common code will then use the effect constructor
as a type parameter. Let us now refactor the two example DSLs in this
manner.

The effect constructor for the \inputencoding{latin9}\lstinline!PrgFile!\inputencoding{utf8}
DSL should contain the operations \inputencoding{latin9}\lstinline!Path!\inputencoding{utf8}
and \inputencoding{latin9}\lstinline!Read!\inputencoding{utf8}. Let
us copy the existing case classes for these operations but rename
their parent type to \inputencoding{latin9}\lstinline!PrgFileC!\inputencoding{utf8}.
We will also move the relevant parts of \inputencoding{latin9}\lstinline!runFile!\inputencoding{utf8}
into a new function (\inputencoding{latin9}\lstinline!runFileC!\inputencoding{utf8})
that we call an \textbf{effect runner}\index{effect runner}:\inputencoding{latin9}
\begin{lstlisting}
sealed trait PrgFileC[A]
final case class Path(p: String) extends PrgFileC[JPath]
final case class Read(p: JPath)  extends PrgFileC[String]
  
def runFileC[A]: PrgFileC[A] => A = {
  case Path(p)   => Paths.get(p)
  case Read(p)   => new String(Files.readAllBytes(p))
}
\end{lstlisting}
\inputencoding{utf8}The code of \inputencoding{latin9}\lstinline!PrgFile!\inputencoding{utf8}
becomes shorter since we moved some code out of it:\inputencoding{latin9}
\begin{lstlisting}
// The same code for sealed trait PrgFile[A] and object PrgFile.
// The case classes are changed to:
final case class Val[A](a: A) extends PrgFile[A]
final case class Bind[A, B](pa: PrgFile[B])(val f: B => PrgFile[A]) extends PrgFile[A]
final case class Op[A](op: PrgFileC[A]) extends PrgFile[A]  // Wrap custom operations.

def runFile[A]: PrgFile[A] => A = {
  case Val(a)          => a
  case bind@Bind(pa)   => runFile(bind.f(runFile(pa)))
  case Op(op)          => runFileC(op) // Run the custom operation and get the result.
}
\end{lstlisting}
\inputencoding{utf8}
The code of \inputencoding{latin9}\lstinline!PrgComplex!\inputencoding{utf8}
is refactored in a similar way:\inputencoding{latin9}
\begin{lstlisting}
// The same code for `sealed trait PrgComplex[A]` and `object PrgComplex`.
// The case classes are changed to:
final case class Val[A](a: A) extends PrgComplex[A]
final case class Bind[A, B](pa: PrgComplex[B])(val f: B => PrgComplex[A]) extends PrgComplex[A]
final case class Op[A](op: PrgComplexC[A]) extends PrgComplex[A]     // Wrap custom operations.

def runComplex[A]: PrgComplex[A] => A = {
  case Val(a)             => a
  case bind@Bind(pa)      => runComplex(bind.f(runComplex(pa)))
  case Op(op)             => runComplexC(op)    // Run the custom operation and get the result.

sealed trait PrgComplexC[A]
final case class Add(x: Complex, y: Complex)      extends PrgComplexC[Complex]
final case class Mul(x: Complex, y: Complex)      extends PrgComplexC[Complex]
final case class Conj(x: Complex)                 extends PrgComplexC[Complex]
final case class Phase(p: Complex)                extends PrgComplexC[Double]
final case class Rotate(p: Complex, alpha: Phase) extends PrgComplexC[Complex]
 
def runComplexC[A]: PrgComplexC[A] => A = {
  case Add(p1, p2)           => p1 + p2
  case Mul(p1, p2)           => p1 * p2
  case Conj(p)               => p.conj
  case Phase(p)              => p.phase
  case Rotate(p, Phase(a))   => p.rotate(a.phase)
}
\end{lstlisting}
\inputencoding{utf8}
The code of \inputencoding{latin9}\lstinline!PrgComplex!\inputencoding{utf8}
is now the same as the code of \inputencoding{latin9}\lstinline!PrgFile!\inputencoding{utf8}
except for using a different effect constructor (\inputencoding{latin9}\lstinline!PrgComplexC!\inputencoding{utf8}
instead of \inputencoding{latin9}\lstinline!PrgFileC!\inputencoding{utf8})
and the corresponding runner (\inputencoding{latin9}\lstinline!runComplexC!\inputencoding{utf8}
instead of \inputencoding{latin9}\lstinline!runFileC!\inputencoding{utf8}).
The effect constructor and its runner encapsulate the entire domain-specific
logic. The code in the classes \inputencoding{latin9}\lstinline!PrgComplex!\inputencoding{utf8}
and \inputencoding{latin9}\lstinline!PrgFile!\inputencoding{utf8}
is only concerned with providing the monadic functionality to the
DSL. We can replace those two classes by a single class (called, say,
\inputencoding{latin9}\lstinline!MonadDSL!\inputencoding{utf8}) that
takes the effect constructor as a \emph{type parameter} \inputencoding{latin9}\lstinline!F!\inputencoding{utf8}.
Since the type parameter \inputencoding{latin9}\lstinline!F!\inputencoding{utf8}
is itself a type constructor, we must declare it via the Scala syntax
\inputencoding{latin9}\lstinline!F[_]!\inputencoding{utf8}. The code
is:\inputencoding{latin9}
\begin{lstlisting}
sealed trait MonadDSL[F[_], A] {
  def flatMap[B](f: A => MonadDSL[F, B]): MonadDSL[F, B] = Bind(this)(f)
  def map[B](f: A => B): MonadDSL[F, B] = flatMap(f andThen MonadDSL.pure)
}
object MonadDSL {
  def pure[F[_], A](a: A): MonadDSL[F, A] = Val(a)
}
final case class Val[F[_], A](a: A) extends MonadDSL[F, A]
final case class Bind[F[_], A, B](pa: MonadDSL[F, B])(val f: B => MonadDSL[F, A]) extends MonadDSL[F, A]
final case class Op[F[_], A](op: F[A]) extends MonadDSL[F, A] // Wrap all domain-specific operations.
\end{lstlisting}
\inputencoding{utf8}
The runner for \inputencoding{latin9}\lstinline!MonadDSL!\inputencoding{utf8}
needs to run the effects described by the effect constructor \inputencoding{latin9}\lstinline!F!\inputencoding{utf8}.
We try passing \inputencoding{latin9}\lstinline!F!\inputencoding{utf8}\textsf{'}s
runner as an additional curried argument to \inputencoding{latin9}\lstinline!MonadDSL!\inputencoding{utf8}\textsf{'}s
runner:\inputencoding{latin9}
\begin{lstlisting}
def run[F[_], A](runner: F[A] => A): MonadDSL[F, A] => A = { // This code does not compile.
  case Val(a)          => a
  case bind@Bind(pa)   => run(runner)(bind.f(run(runner)(pa)))
  case Op(op)          => runner(op)
}
\end{lstlisting}
\inputencoding{utf8}However, this code gives a type error. The type of \inputencoding{latin9}\lstinline!bind.f!\inputencoding{utf8}
is \inputencoding{latin9}\lstinline!B => MonadDSL[F, A]!\inputencoding{utf8},
so the argument of \inputencoding{latin9}\lstinline!bind.f!\inputencoding{utf8}
must have type \inputencoding{latin9}\lstinline!B!\inputencoding{utf8}.
A value of type \inputencoding{latin9}\lstinline!B!\inputencoding{utf8}
will be produced by \inputencoding{latin9}\lstinline!run(runner)(pa)!\inputencoding{utf8}
only if we set type parameters as \inputencoding{latin9}\lstinline!run[F, B]!\inputencoding{utf8},
and only if \inputencoding{latin9}\lstinline!runner!\inputencoding{utf8}
has type \inputencoding{latin9}\lstinline!F[B] => B!\inputencoding{utf8}.
But the given argument \inputencoding{latin9}\lstinline!runner!\inputencoding{utf8}
has a fixed type \inputencoding{latin9}\lstinline!F[A] => A!\inputencoding{utf8}
and cannot be used with type \inputencoding{latin9}\lstinline!B!\inputencoding{utf8}.

The function \inputencoding{latin9}\lstinline!run(runner)!\inputencoding{utf8}
will work with arbitrary types only if \inputencoding{latin9}\lstinline!runner!\inputencoding{utf8}
has type \inputencoding{latin9}\lstinline!F[X] => X!\inputencoding{utf8}
where \inputencoding{latin9}\lstinline!X!\inputencoding{utf8} remains
free and is \emph{not} known in advance. The type signatures of the
effect runners \inputencoding{latin9}\lstinline!runComplexC!\inputencoding{utf8}
and \inputencoding{latin9}\lstinline!runFileC!\inputencoding{utf8}
already have that form:\inputencoding{latin9}
\begin{lstlisting}
def runComplexC[X]: PrgComplexC[X] => X    // PrgComplexC[X] => X with arbitrary types X
def runFileC[X]: PrgFileC[X] => X          // PrgFileC[X] => X with arbitrary types X
\end{lstlisting}
\inputencoding{utf8}We need to pass those functions as the \inputencoding{latin9}\lstinline!runner!\inputencoding{utf8}
arguments to \inputencoding{latin9}\lstinline!run(runner)!\inputencoding{utf8}
while not losing the freedom provided by the type parameter \inputencoding{latin9}\lstinline!X!\inputencoding{utf8}. 

To achieve that, we define a new type \inputencoding{latin9}\lstinline!Runner!\inputencoding{utf8}
as a Scala trait encapsulating the type signature of an effect runner.
The two effect runners can then be declared as values of type \inputencoding{latin9}\lstinline!Runner!\inputencoding{utf8}:\inputencoding{latin9}
\begin{lstlisting}
trait Runner[F[_]] { def run[X]: F[X] => X }
val runnerComplex = new Runner[PrgComplexC] { def run[X]: PrgComplexC[X] => X = runComplexC[X] }
val runnerFile = new Runner[PrgFileC] { def run[X]: PrgFileC[X] => X = runFileC[X] } 
\end{lstlisting}
\inputencoding{utf8}
Function values that encapsulate a type parameter are called \textbf{polymorphic
functions}\index{polymorphic function}. The type notation for the
type \inputencoding{latin9}\lstinline!Runner!\inputencoding{utf8}
is:
\[
\text{Runner}^{F}\triangleq\forall X.\,F^{X}\rightarrow X\quad.
\]
The universal quantifier ($\forall X$) indicates that a value of
type \inputencoding{latin9}\lstinline!Runner[F]!\inputencoding{utf8}
still has the freedom of using any type \inputencoding{latin9}\lstinline!X!\inputencoding{utf8}
when the method \inputencoding{latin9}\lstinline!run!\inputencoding{utf8}
is called.

In Scala 3, the type of an effect runner is written in a shorter syntax:\inputencoding{latin9}
\begin{lstlisting}
type Runner[F[_]] = [X] => F[X] => X
\end{lstlisting}
\inputencoding{utf8}This corresponds to the type notation $\forall X.\,F^{X}\rightarrow X$.
The programmer does not need to declare a trait \inputencoding{latin9}\lstinline!Runner!\inputencoding{utf8}
in Scala 3 because the compiler does that automatically for polymorphic
functions.

Using the \inputencoding{latin9}\lstinline!Runner!\inputencoding{utf8}
type, we rewrite the code for the function \inputencoding{latin9}\lstinline!run!\inputencoding{utf8}
as:\inputencoding{latin9}
\begin{lstlisting}
def run[F[_], A](runner: Runner[F]): MonadDSL[F, A] => A = {
  case Val(a)          => a
  case bind@Bind(pa)   => run(runner)(bind.f(run(runner)(pa)))
  case Op(op)          => runner.run(op)
}
\end{lstlisting}
\inputencoding{utf8}We can now define \inputencoding{latin9}\lstinline!PrgComplex!\inputencoding{utf8}
through \inputencoding{latin9}\lstinline!MonadDSL!\inputencoding{utf8}
and test the new code:\inputencoding{latin9}
\begin{lstlisting}
type PrgComplex[A] = MonadDSL[PrgComplexC, A]
val prgComplex: PrgComplex[Complex] = for {
  x <- Op(Add(Complex(1, 1), Complex(0, 1)))
  y <- Op(Mul(x, Complex(3, -4)))
  z <- Op(Conj(y))
  r <- Op(Rotate(x, Phase(Complex(0, 1))))
} yield r

scala> runComplex(runnerComplex)(prgComplex2)
res0: Complex = Complex(x = 2.000000000000001, y = 11.0)
\end{lstlisting}
\inputencoding{utf8}

\subsection{A first recipe for monadic DSLs\label{subsec:A-first-recipe-monadic-dsl}}

The previous section gave motivation for defining the type constructor
\inputencoding{latin9}\lstinline!MonadDSL[F[_], A]!\inputencoding{utf8}.
That type constructor is called the \textsf{``}free monad on \inputencoding{latin9}\lstinline!F!\inputencoding{utf8}\textsf{''},
for reasons explained later in this chapter. Note that \inputencoding{latin9}\lstinline!MonadDSL[F[_], A]!\inputencoding{utf8}
supports the monad\textsf{'}s methods \inputencoding{latin9}\lstinline!map!\inputencoding{utf8},
\inputencoding{latin9}\lstinline!flatMap!\inputencoding{utf8}, and
\inputencoding{latin9}\lstinline!pure!\inputencoding{utf8}, which
\inputencoding{latin9}\lstinline!F!\inputencoding{utf8} does not
necessarily have. So, we may view \inputencoding{latin9}\lstinline!MonadDSL[F[_], A]!\inputencoding{utf8}
as a special wrapper that adds the monadic variable-binding functionality
to any given DSL whose operations are described by an effect constructor
\inputencoding{latin9}\lstinline!F!\inputencoding{utf8}.

Below we will show that monadic DSL programs will satisfy the monad
laws \emph{after} the effects are run (but not necessarily before
that!). Let us now summarize a recipe for using \inputencoding{latin9}\lstinline!MonadDSL!\inputencoding{utf8}
in practice.

Begin by writing down the types of the available operations in the
DSL. The operations may be functions between fixed known types, such
as $\text{Int}\rightarrow\text{Int}\rightarrow\text{String}$ or $\text{String}\rightarrow\bbnum 1$.
Operations may be also functions with type parameters, for example,
$A\times\text{Int}\rightarrow\text{String}$ where $A$ is a type
parameter. In every case, we need to write down the types in the form
of a simple (non-curried) function, such as $P\rightarrow Q$, where
$P$ and $Q$ are type expressions possibly depending on some type
parameters.

The next step is to define a new type constructor \inputencoding{latin9}\lstinline!F[_]!\inputencoding{utf8}
that will serve as the DSL\textsf{'}s effect constructor\index{effect constructor}:\inputencoding{latin9}
\begin{lstlisting}
sealed trait F[_]
\end{lstlisting}
\inputencoding{utf8} The effect constructor \inputencoding{latin9}\lstinline!F!\inputencoding{utf8}
will be a disjunctive type whose parts correspond to each of the domain-specific
operations. For an operation with type $P\rightarrow Q$, define the
corresponding case class as:\inputencoding{latin9}
\begin{lstlisting}
final case class ReadPWriteQ(x: P) extends F[Q]
\end{lstlisting}
\inputencoding{utf8}The names of case classes (such as \inputencoding{latin9}\lstinline!ReadPWriteQ!\inputencoding{utf8})
may be chosen for clarity.

It is an important part of the recipe that the result type (\inputencoding{latin9}\lstinline!Q!\inputencoding{utf8})
is wrapped in the effect constructor \inputencoding{latin9}\lstinline!F!\inputencoding{utf8}
while the argument type (\inputencoding{latin9}\lstinline!P!\inputencoding{utf8})
is \emph{not}. For example, the \inputencoding{latin9}\lstinline!PrgFile!\inputencoding{utf8}
DSL described in the previous section has two operations with types
\inputencoding{latin9}\lstinline!String => JPath!\inputencoding{utf8}
and \inputencoding{latin9}\lstinline!JPath => String!\inputencoding{utf8}.
The corresponding effect constructor (\inputencoding{latin9}\lstinline!PrgFileC!\inputencoding{utf8})
is defined by:\inputencoding{latin9}
\begin{lstlisting}
sealed trait PrgFileC[_]
final case class Path(p: String) extends PrgFileC[JPath]
final case class Read(p: JPath)  extends PrgFileC[String]
\end{lstlisting}
\inputencoding{utf8}
Typically, a type constructor \inputencoding{latin9}\lstinline!F!\inputencoding{utf8}
defined in this way will be an \textsf{``}unfunctor\textsf{''}\index{unfunctor}.
In the example shown above, \inputencoding{latin9}\lstinline!PrgFileC!\inputencoding{utf8}
cannot be a functor because it is impossible to create values of type
\inputencoding{latin9}\lstinline!PrgFileC[A]!\inputencoding{utf8}
with an arbitrary type \inputencoding{latin9}\lstinline!A!\inputencoding{utf8}
(the type \inputencoding{latin9}\lstinline!A!\inputencoding{utf8}
must be either \inputencoding{latin9}\lstinline!JPath!\inputencoding{utf8}
or \inputencoding{latin9}\lstinline!String!\inputencoding{utf8}).
However, in some cases the effect constructor \inputencoding{latin9}\lstinline!F!\inputencoding{utf8}
could be itself a lawful functor.

Next, we implement an \textsf{``}effect runner\textsf{''} \index{effect runner}for
\inputencoding{latin9}\lstinline!F!\inputencoding{utf8}. An \textbf{effect
runner} for an effect constructor \inputencoding{latin9}\lstinline!F!\inputencoding{utf8}
is a polymorphic function\index{polymorphic function} with the type
signature $\forall X.\,F^{X}\rightarrow X$:\inputencoding{latin9}
\begin{lstlisting}
val runF: Runner[F] = new Runner[F] { def run[X]: F[X] => X = ??? }
\end{lstlisting}
\inputencoding{utf8}If the definition of \inputencoding{latin9}\lstinline!F!\inputencoding{utf8}
allows only certain types \inputencoding{latin9}\lstinline!A!\inputencoding{utf8}
in values of type \inputencoding{latin9}\lstinline!F[A]!\inputencoding{utf8}
then the runner only needs to work for those types. In the \inputencoding{latin9}\lstinline!PrgFile!\inputencoding{utf8}
DSL, the runner only needs to work with arguments of types \inputencoding{latin9}\lstinline!PrgFileC[JPath]!\inputencoding{utf8}
and \inputencoding{latin9}\lstinline!PrgFileC[String]!\inputencoding{utf8}
because those are the only allowed types of values that can be wrapped
by the unfunctor \inputencoding{latin9}\lstinline!PrgFileC!\inputencoding{utf8}.
The runner will never be given an argument of type \inputencoding{latin9}\lstinline!PrgFile[X]!\inputencoding{utf8}
with an arbitrary unknown \inputencoding{latin9}\lstinline!X!\inputencoding{utf8}.
However, different effect constructors may have different allowed
types, and \inputencoding{latin9}\lstinline!MonadDSL!\inputencoding{utf8}
needs to be able to work with any effect constructor. So, the runner\textsf{'}s
type signature still needs a general type parameter \inputencoding{latin9}\lstinline!X!\inputencoding{utf8}:\inputencoding{latin9}
\begin{lstlisting}
val runnerFile = new Runner[PrgFileC] { def run[X]: (PrgFileC[X] => X) = runFileC[X] }
\end{lstlisting}
\inputencoding{utf8}
We now define the monadic DSL type (\inputencoding{latin9}\lstinline!MyDSL!\inputencoding{utf8})
and its runner function (\inputencoding{latin9}\lstinline!runMyDSL!\inputencoding{utf8})
as:\inputencoding{latin9}
\begin{lstlisting}
type MyDSL[A] = MonadDSL[F, A]
def runMyDSL[A]: MyDSL[A] => A = MonadDSL.run(runF)
\end{lstlisting}
\inputencoding{utf8}DSL programs may be written using functor blocks:\inputencoding{latin9}
\begin{lstlisting}
type FileDSL[A] = MonadDSL[PrgFileC, A]
def runFileDSL[A]: FileDSL[A] => A = MonadDSL.run(runnerFile)
val prgFile1: FileDSL[String] = for {
  x <- Op(Path("config.txt"))
  y <- Op(Read(x))
} yield y

scala> runFileDSL(prgFile1)
res0: String = "version = 1"
\end{lstlisting}
\inputencoding{utf8}
Note that each DSL operation needs to be wrapped in an \inputencoding{latin9}\lstinline!Op!\inputencoding{utf8}
case class. To make the code shorter, we may define an implicit conversion
from \inputencoding{latin9}\lstinline!PrgFileC[A]!\inputencoding{utf8}
to \inputencoding{latin9}\lstinline!Op[A]!\inputencoding{utf8}:\inputencoding{latin9}
\begin{lstlisting}
implicit def toOp[A](p: PrgFileC[A]): Op[A] = Op(p) // *** check if this code actually works
val prgFile2: FileDSL[String] = for { // Does it help if we begin the functor block with a Pure()?
  x <- Path("config.txt")
  y <- Read(x)
} yield y

scala> runFileDSL(prgFile1)
res1: String = "version = 1"
\end{lstlisting}
\inputencoding{utf8}
Does\index{free monad!monad laws} the type constructor \inputencoding{latin9}\lstinline!MonadDSL[F, A]!\inputencoding{utf8}
satisfy the laws of the monad? We can quickly find out that it does
\emph{not}. The left identity law of \inputencoding{latin9}\lstinline!flatMap!\inputencoding{utf8}
fails:\inputencoding{latin9}
\begin{lstlisting}
MonadDSL.pure(a).flatMap(f) == Bind(Val(a))(f) // This should equal f(a) by the left identity law.
\end{lstlisting}
\inputencoding{utf8}However, the law will hold \emph{after} we apply a runner function
to both sides. For brevity, let us denote \inputencoding{latin9}\lstinline!MonadDSL!\inputencoding{utf8}\textsf{'}s
runner function by \inputencoding{latin9}\lstinline!r(x)!\inputencoding{utf8}
instead of \inputencoding{latin9}\lstinline!run(runner)(x)!\inputencoding{utf8}.
The code of \inputencoding{latin9}\lstinline!run!\inputencoding{utf8}
shows that \inputencoding{latin9}\lstinline!r(Bind(p)(f)) == r(f(r(p)))!\inputencoding{utf8}
and \inputencoding{latin9}\lstinline!r(Val(a)) == a!\inputencoding{utf8}.
Then we find that both sides of the law give \inputencoding{latin9}\lstinline!r(f(a))!\inputencoding{utf8}
when run:\inputencoding{latin9}
\begin{lstlisting}
r(MonadDSL.pure(a).flatMap(f)) == r(Bind(Val(a))(f)) == r(f(a))
\end{lstlisting}
\inputencoding{utf8}
The associativity law~(\ref{eq:associativity-law-flatMap}) of \inputencoding{latin9}\lstinline!flatMap!\inputencoding{utf8}
also does not hold directly for \inputencoding{latin9}\lstinline!MonadDSL!\inputencoding{utf8}
values but will hold after a \inputencoding{latin9}\lstinline!MonadDSL!\inputencoding{utf8}
program has been run. To see that, first compare the two sides of
the law~(\ref{eq:associativity-law-flatMap}):\inputencoding{latin9}
\begin{lstlisting}
/* left-hand side:  */   p.flatMap(x => f(x).flatMap(g)) == Bind(p)(x => Bind(f(x))(g))
/* right-hand side: */   p.flatMap(f).flatMap(g) == Bind(Bind(p)(f))(g)
\end{lstlisting}
\inputencoding{utf8}The law fails since the two data structures are not the same. When
we apply a runner \inputencoding{latin9}\lstinline!r!\inputencoding{utf8}
to both sides, we get:\inputencoding{latin9}
\begin{lstlisting}
/* left-hand side:  */   r(Bind(p)(x => Bind(f(x))(g))) == r(Bind(f(r(p)))(g)) == r(g(r(f(r(p)))))  
/* right-hand side: */   r(Bind(Bind(p)(f))(g)) == r(g(r(Bind(p)(f)))) == r(g(r(f(r(p)))))
\end{lstlisting}
\inputencoding{utf8}
We may say that failures of the monad laws are \textsf{``}not observable\textsf{''}
since the laws will hold after running the DSL programs.

Could we implement \inputencoding{latin9}\lstinline!MonadDSL!\inputencoding{utf8}\textsf{'}s
\inputencoding{latin9}\lstinline!flatMap!\inputencoding{utf8} method
differently so that the monad laws already hold before applying a
runner? To satisfy the left identity law, we need to avoid creating
a nested structure of the form \inputencoding{latin9}\lstinline!Bind(Val(a))!\inputencoding{utf8}
when applying \inputencoding{latin9}\lstinline!flatMap!\inputencoding{utf8}
to a value of the form \inputencoding{latin9}\lstinline!Val(a)!\inputencoding{utf8}.
Instead, \inputencoding{latin9}\lstinline!Val(a).flatMap(f)!\inputencoding{utf8}
should return \inputencoding{latin9}\lstinline!f(a)!\inputencoding{utf8}
as required by the left identity law. To satisfy the associativity
law, we need to avoid creating a nested structure of the form \inputencoding{latin9}\lstinline!Bind(Bind(...))!\inputencoding{utf8}
when applying \inputencoding{latin9}\lstinline!flatMap!\inputencoding{utf8}
to a \inputencoding{latin9}\lstinline!Bind!\inputencoding{utf8} value.
Instead, applying \inputencoding{latin9}\lstinline!flatMap!\inputencoding{utf8}
to \inputencoding{latin9}\lstinline!Bind!\inputencoding{utf8} should
directly return the value required by the associativity law. The new
code of \inputencoding{latin9}\lstinline!flatMap!\inputencoding{utf8}
is:\inputencoding{latin9}
\begin{lstlisting}
sealed trait MonadDSL[F[_], A] { *** check that this code works
  def flatMap[B](f: A => MonadDSL[F, B]): MonadDSL[F, B] = this match {
    case Val(a)          => f(a)
    case bind@Bind(pa)   => Bind(pa)(x => Bind(bind.f(x))(f))
    case _               => Bind(this)(f)
  }
  ... // Other code remains unchanged.
}
\end{lstlisting}
\inputencoding{utf8}The resulting code creates fewer nested case classes in memory.

Unlike other laws, the right identity law already holds for any value
\inputencoding{latin9}\lstinline!p!\inputencoding{utf8} of type \inputencoding{latin9}\lstinline!MonadDSL[F, A]!\inputencoding{utf8}:\inputencoding{latin9}
\begin{lstlisting}
p.flatMap(f andThen MonadDSL.pure) == p.map(f)
\end{lstlisting}
\inputencoding{utf8}The reason is that our current code implements \inputencoding{latin9}\lstinline!MonadDSL!\inputencoding{utf8}\textsf{'}s
\inputencoding{latin9}\lstinline!map!\inputencoding{utf8} method
via \inputencoding{latin9}\lstinline!flatMap!\inputencoding{utf8}
in exactly this way. 

\subsection{Running a DSL program into another monad}

In the code from the previous section, one may use different effect
runners with \emph{the same} function \inputencoding{latin9}\lstinline!MonadDSL.run!\inputencoding{utf8}.
We may apply \inputencoding{latin9}\lstinline!MonadDSL.run!\inputencoding{utf8}
to any effect runner of type $\forall X.\,F^{X}\rightarrow X$ and
obtain a monad runner of type \inputencoding{latin9}\lstinline!MonadDSL[F, A] => A!\inputencoding{utf8}.

More generally, we may need to run the $F$-effects into \emph{another
monad}. Given an effect runner of type $\forall X.\,F^{X}\rightarrow M^{X}$,
where $M$ is some monad, we hope to obtain a runner with type signature
$\forall A.\,\text{MonadDSL}^{F,A}\rightarrow M^{A}$.

To see an example of using such general runners in practice, assume
that the execution of $F$-effects may produce errors.\footnote{\index{jokes}\textsf{``}\emph{A function that launches real-world missiles
can run out of missiles.}\textsf{''} (A quote attributed to \index{Simon Peyton Jones}Simon
Peyton Jones, see \texttt{\href{https://stackoverflow.com/questions/2773004}{https://stackoverflow.com/questions/2773004}}
for discussion.)} For instance, the code of \inputencoding{latin9}\lstinline!runFileC!\inputencoding{utf8}
(the runner for the \inputencoding{latin9}\lstinline!PrgFile!\inputencoding{utf8}
DSL) will throw exceptions when files are not found or not readable.
We may catch those exceptions with Scala\textsf{'}s standard \inputencoding{latin9}\lstinline!Try!\inputencoding{utf8}
class. We will then need to replace a runner of type \inputencoding{latin9}\lstinline!PrgFileC[X] => X!\inputencoding{utf8}
by a polymorphic function of type \inputencoding{latin9}\lstinline!PrgFileC[X] => Try[X]!\inputencoding{utf8}.
The new code looks like this:\inputencoding{latin9}
\begin{lstlisting}
trait RunnerTry[F[_]] { def run[X]: F[X] => Try[X] }
val runnerFileTry = new RunnerTry[PrgFileC] { def run[X]: PrgFileC[X] => Try[X] = p => Try(runFileC(p)) }
\end{lstlisting}
\inputencoding{utf8} The code of the \inputencoding{latin9}\lstinline!MonadDSL!\inputencoding{utf8}\textsf{'}s
runner needs to be modified to accommodate the new type. Since a \inputencoding{latin9}\lstinline!Val(a)!\inputencoding{utf8}
cannot generate errors (only $F$-effects can), we transform a \inputencoding{latin9}\lstinline!Val(a)!\inputencoding{utf8}
into \inputencoding{latin9}\lstinline!Success(a)!\inputencoding{utf8}.
A \inputencoding{latin9}\lstinline!Bind!\inputencoding{utf8} value
can only result from using \inputencoding{latin9}\lstinline!MonadDSL!\inputencoding{utf8}\textsf{'}s
\inputencoding{latin9}\lstinline!flatMap!\inputencoding{utf8}, and
so it is reasonable to transform \inputencoding{latin9}\lstinline!Bind!\inputencoding{utf8}
into a call to \inputencoding{latin9}\lstinline!Try!\inputencoding{utf8}\textsf{'}s
\inputencoding{latin9}\lstinline!flatMap!\inputencoding{utf8} method.
The new runner\textsf{'}s code becomes:\inputencoding{latin9}
\begin{lstlisting}
def runTry[F[_], A](runnerTry: RunnerTry[F]): MonadDSL[F, A] => Try[A] = {
  case Val(a)          => Success(a)
  case bind@Bind(pa)   => runTry(runner)(pa).flatMap(bind.f andThen runTry(runner)) // Use flatMap from Try.
  case Op(op)          => runner.run(op)
} *** check that this code works
\end{lstlisting}
\inputencoding{utf8}
Let us generalize the code of \inputencoding{latin9}\lstinline!runTry!\inputencoding{utf8}
to an arbitrary monad $M$ instead of \inputencoding{latin9}\lstinline!Try!\inputencoding{utf8}.
We note that \inputencoding{latin9}\lstinline!runTry!\inputencoding{utf8}
uses only two values specific to \inputencoding{latin9}\lstinline!Try!\inputencoding{utf8}:
the \inputencoding{latin9}\lstinline!flatMap!\inputencoding{utf8}
method of the \inputencoding{latin9}\lstinline!Try!\inputencoding{utf8}
monad and the \inputencoding{latin9}\lstinline!Success!\inputencoding{utf8}
case class, which corresponds to the \inputencoding{latin9}\lstinline!Try!\inputencoding{utf8}
monad\textsf{'}s \inputencoding{latin9}\lstinline!pure!\inputencoding{utf8}
method. To adapt the code to another monad $M$ instead of \inputencoding{latin9}\lstinline!Try!\inputencoding{utf8},
we need to use $M$\textsf{'}s \inputencoding{latin9}\lstinline!flatMap!\inputencoding{utf8}
and to replace \inputencoding{latin9}\lstinline!Success!\inputencoding{utf8}
by $M$\textsf{'}s \inputencoding{latin9}\lstinline!pure!\inputencoding{utf8}.
The result is a \textsf{``}universal runner\textsf{''}\index{free monad!universal runner}
for the free monad:\inputencoding{latin9}
\begin{lstlisting}
trait RunnerM[F[_], M[_]] { def run[X]: F[X] => M[X] }
def runM[F[_], M[_]: Monad, A](runnerM: RunnerM[F, M]): MonadDSL[F, A] => M[A] = {
  case Val(a)          => Monad[M].pure(a)
  case bind@Bind(pa)   => runM(runnerM)(pa).flatMap(bind.f andThen runM(runnerM)) // Use flatMap from M.
  case Op(op)          => runnerM.run(op)
} *** check that this code works
\end{lstlisting}
\inputencoding{utf8}
We have implemented the universal runner by mapping \inputencoding{latin9}\lstinline!MonadDSL!\inputencoding{utf8}\textsf{'}s
monadic methods (\inputencoding{latin9}\lstinline!pure!\inputencoding{utf8}
and \inputencoding{latin9}\lstinline!flatMap!\inputencoding{utf8})
into the corresponding \inputencoding{latin9}\lstinline!pure!\inputencoding{utf8}
and \inputencoding{latin9}\lstinline!flatMap!\inputencoding{utf8}
methods of the monad $M$. More precisely, if we fix an effect runner
\inputencoding{latin9}\lstinline!runnerM!\inputencoding{utf8} then
the corresponding runner \inputencoding{latin9}\lstinline!run = runM(runnerM)!\inputencoding{utf8}
satisfies:\inputencoding{latin9}
\begin{lstlisting}
run(MonadDSL.pure(a)) == run(Val(a))  == Monad[M].pure(a)
run(p.flatMap(f)) == run(Bind(p)(f)) == run(p.flatMap(f andThen run))
\end{lstlisting}
\inputencoding{utf8}These equations also show that there is \emph{only one} implementation
of a runner function of type \inputencoding{latin9}\lstinline!run[A]: MonadDSL[A] => M[A]!\inputencoding{utf8}
that preserves the \inputencoding{latin9}\lstinline!pure!\inputencoding{utf8}
and \inputencoding{latin9}\lstinline!flatMap!\inputencoding{utf8}
operations. The \inputencoding{latin9}\lstinline!Op(op)!\inputencoding{utf8}
values must be transformed into \inputencoding{latin9}\lstinline!runnerM.run(op)!\inputencoding{utf8}
as there is no other way of converting a value of type \inputencoding{latin9}\lstinline!F[A]!\inputencoding{utf8}
into a value of type \inputencoding{latin9}\lstinline!M[A]!\inputencoding{utf8}
with an arbitrary monad \inputencoding{latin9}\lstinline!M!\inputencoding{utf8}. 

The use of universal runners (with an arbitrary monad $M$) allows
the programmer to better separate the business logic of an application
from its execution environment. For instance, $M$ may itself be a
monadic DSL with a different set of operations and its own effect
runner that works more efficiently or supports a different run-time
environment.

\section{Different encodings of the free monad}

\subsection{Motivation\label{subsec:Motivation-free-monad-different-encodings}}

The type \inputencoding{latin9}\lstinline!MonadDSL[F, A]!\inputencoding{utf8}
is called a \textbf{free monad} on\index{free monad} the type constructor
\inputencoding{latin9}\lstinline!F!\inputencoding{utf8}. The word
\textsf{``}free\textsf{''} indicates that \inputencoding{latin9}\lstinline!MonadDSL!\inputencoding{utf8}
is free of any domain-specific code. We may view \inputencoding{latin9}\lstinline!MonadDSL[F, A]!\inputencoding{utf8}
as a wrapper that adds the monad\textsf{'}s functionality to any type constructor
\inputencoding{latin9}\lstinline!F!\inputencoding{utf8} and creates
a new monadic DSL based on \inputencoding{latin9}\lstinline!F!\inputencoding{utf8}\textsf{'}s
operations. The construction uses \inputencoding{latin9}\lstinline!F!\inputencoding{utf8}
as a type parameter, and so it works equally well with every \inputencoding{latin9}\lstinline!F!\inputencoding{utf8}.

A monad must support the methods \inputencoding{latin9}\lstinline!pure!\inputencoding{utf8},
\inputencoding{latin9}\lstinline!flatMap!\inputencoding{utf8}, and
\inputencoding{latin9}\lstinline!map!\inputencoding{utf8}. The effect
constructor \inputencoding{latin9}\lstinline!F!\inputencoding{utf8}
often does not have these methods. The \inputencoding{latin9}\lstinline!MonadDSL!\inputencoding{utf8}
constructor is able to convert an arbitrary \inputencoding{latin9}\lstinline!F!\inputencoding{utf8}
into a monad because \inputencoding{latin9}\lstinline!MonadDSL!\inputencoding{utf8}
defines the methods \inputencoding{latin9}\lstinline!pure!\inputencoding{utf8},
\inputencoding{latin9}\lstinline!flatMap!\inputencoding{utf8}, and
\inputencoding{latin9}\lstinline!map!\inputencoding{utf8} in a special
way. Those methods create some data structures (nested case classes)
in memory but do not perform any domain-specific effects or actions.
The effects will be performed at a later time by an effect runner
(\inputencoding{latin9}\lstinline!runF!\inputencoding{utf8} of type
\inputencoding{latin9}\lstinline!F[A] => A!\inputencoding{utf8})
when we apply \inputencoding{latin9}\lstinline!MonadDSL.run!\inputencoding{utf8}
to a DSL program. Effect runners are not part of the code of \inputencoding{latin9}\lstinline!MonadDSL!\inputencoding{utf8}
and will need to be supplied separately. So, we may run the same \inputencoding{latin9}\lstinline!MonadDSL!\inputencoding{utf8}
program using different runners.

The code of \inputencoding{latin9}\lstinline!MonadDSL!\inputencoding{utf8}
satisfies our goal of producing a type-safe DSL out of a given set
of domain-specific operations. But it turns out that \inputencoding{latin9}\lstinline!MonadDSL!\inputencoding{utf8}
is not the only way of turning a given type constructor \inputencoding{latin9}\lstinline!F!\inputencoding{utf8}
into a monad. Various presentations and blog posts about the free
monad have used different implementations that are not obviously equivalent
to each other. Here are three examples that we will call \inputencoding{latin9}\lstinline!Free1!\inputencoding{utf8},
\inputencoding{latin9}\lstinline!Free2!\inputencoding{utf8}, and
\inputencoding{latin9}\lstinline!Free3!\inputencoding{utf8}. 

In a 2012 blog post,\footnote{See \texttt{\href{http://www.haskellforall.com/2012/06/you-could-have-invented-free-monads.html}{http://www.haskellforall.com/2012/06/you-could-have-invented-free-monads.html}}}
\index{Gabriella Gonzalez (known as Gabriel Gonzalez prior to 2021)}G.~Gonzalez
showed a free monad corresponding to this Scala code:\inputencoding{latin9}
\begin{lstlisting}
abstract class Free1[F[_]: Functor, T] {
  def flatMap[A](f: T => Free1[F, A]): Free1[F, A] = this match {
    case Pure(t)      => f(a)
    case Flatten(p)   => Flatten(p.map(g => g.flatMap(f))) // Use F\textsf{'}s Functor instance.
  }
}
final case class Pure[F[_], T](t: T)                 extends Free1[F, T]
final case class Flatten[F[_], T](p: F[Free1[F, T]]) extends Free1[F, T] 
\end{lstlisting}
\inputencoding{utf8}
In 2014,\footnote{See \texttt{\href{http://functionaltalks.org/2014/11/23/runar-oli-bjarnason-free-monad/}{http://functionaltalks.org/2014/11/23/runar-oli-bjarnason-free-monad/}}}
R.~Bjarnason\index{Runar@R\'unar Bjarnason} presented the following
implementation of a free monad:\inputencoding{latin9}
\begin{lstlisting}
sealed trait Free2[F[_], T] {
  def flatMap[A](f: T => Free2[F, A]): Free2[F, A] = this match {
    case Return(t)    => f(t)
    case Bind(p, g)   => Bind(p, (b => g(b) flatMap f))
  }
}
final case class Return[F[_], T](t: T)                          extends Free2[F, T]
final case class Bind[F[_], T, A](p: F[A], g: A => Free2[F, T]) extends Free2[F, T]
\end{lstlisting}
\inputencoding{utf8}
In 2016, K.~Robinson\index{Kelley Robinson} gave a talk\footnote{See \texttt{\href{https://www.slideshare.net/KelleyRobinson1/why-the-free-monad-isnt-free-61836547}{https://www.slideshare.net/KelleyRobinson1/why-the-free-monad-isnt-free-61836547}}}
where the code for the free monad looked like this:\inputencoding{latin9}
\begin{lstlisting}
sealed trait Free3[F[_], T] {
  def flatMap[A](f: T => Free3[F, A]): Free3[F, A] = FlatMap(this, f) 
}
final case class Pure[F[_], T](t: T)                                      extends Free3[F, T]
final case class Suspend[F[_], T](f: F[T])                                extends Free3[F, T]
final case class FlatMap[F[_], T, A](p: Free3[F, A], g: A => Free3[F, T]) extends Free3[F, T]
\end{lstlisting}
\inputencoding{utf8}
The main differences between \inputencoding{latin9}\lstinline!Free1!\inputencoding{utf8},
\inputencoding{latin9}\lstinline!Free2!\inputencoding{utf8}, and
\inputencoding{latin9}\lstinline!Free3!\inputencoding{utf8} are in
the definitions of the case classes and of the \inputencoding{latin9}\lstinline!flatMap!\inputencoding{utf8}
functions, so we omitted all other code. The type of \inputencoding{latin9}\lstinline!Free3!\inputencoding{utf8}
is the same as \inputencoding{latin9}\lstinline!MonadDSL!\inputencoding{utf8}
except for renaming \inputencoding{latin9}\lstinline!Pure!\inputencoding{utf8}
to \inputencoding{latin9}\lstinline!Val!\inputencoding{utf8}, \inputencoding{latin9}\lstinline!Suspend!\inputencoding{utf8}
to \inputencoding{latin9}\lstinline!Op!\inputencoding{utf8}, and
\inputencoding{latin9}\lstinline!FlatMap(f, g)!\inputencoding{utf8}
to \inputencoding{latin9}\lstinline!Bind(f)(g)!\inputencoding{utf8}.
However, \inputencoding{latin9}\lstinline!Free1!\inputencoding{utf8}
and \inputencoding{latin9}\lstinline!Free2!\inputencoding{utf8} have
different data in their case classes and are not obviously equivalent
to each other (or to \inputencoding{latin9}\lstinline!Free3!\inputencoding{utf8}).
We call these implementations \textbf{encodings}\index{free monad!encodings}
of the free monad. The different codes implement the same idea (turning
a type constructor \inputencoding{latin9}\lstinline!F!\inputencoding{utf8}
into a monad) in different ways.

To figure out how to use those codes in practice, a programmer might
ask the following questions: Are the three encodings equivalent? What
laws does their code need to satisfy in order to be considered \textsf{``}correct\textsf{''}?
The \inputencoding{latin9}\lstinline!Free2!\inputencoding{utf8} type
has 2 case classes and \inputencoding{latin9}\lstinline!Free3!\inputencoding{utf8}
has 3; are there any other encodings of the free monad? Is there a
systematic way of finding constructions that convert any type into
a \textsf{``}free functor\textsf{''}, a \textsf{``}free monoid\textsf{''}, or into another \textsf{``}free\textsf{''}
typeclass? The rest of this chapter will develop the required theory
and show the derivations needed to answer these questions. 

\subsection{The raw tree encoding}

To understand the relationships between the encodings of the free
monad, let us first examine how the code of \inputencoding{latin9}\lstinline!MonadDSL!\inputencoding{utf8}
(or equivalently \inputencoding{latin9}\lstinline!Free3!\inputencoding{utf8})
implements the \inputencoding{latin9}\lstinline!map!\inputencoding{utf8}
method. In \inputencoding{latin9}\lstinline!MonadDSL!\inputencoding{utf8},
we implemented \inputencoding{latin9}\lstinline!map!\inputencoding{utf8}
through \inputencoding{latin9}\lstinline!flatMap!\inputencoding{utf8}.
As a result, our code for the \inputencoding{latin9}\lstinline!map!\inputencoding{utf8}
method is not similar to the code of the two other monadic methods
(\inputencoding{latin9}\lstinline!pure!\inputencoding{utf8} and \inputencoding{latin9}\lstinline!flatMap!\inputencoding{utf8})
that merely create new values of the case classes \inputencoding{latin9}\lstinline!Val!\inputencoding{utf8}
and \inputencoding{latin9}\lstinline!Bind!\inputencoding{utf8} without
performing any computations.

We could implement \inputencoding{latin9}\lstinline!map!\inputencoding{utf8}
in a similar way if we added a new case class, say \inputencoding{latin9}\lstinline!FMap!\inputencoding{utf8},
to \inputencoding{latin9}\lstinline!MonadDSL!\inputencoding{utf8}.
The result is a new encoding of the free monad that we will call \inputencoding{latin9}\lstinline!Free4!\inputencoding{utf8}
(since it uses $4$ case classes):\inputencoding{latin9}
\begin{lstlisting}
sealed trait Free4[F[_], A] {
  def flatMap[B](f: A => Free4[F, B]): Free4[F, B] = Bind(this)(f)
  def map[B](f: A => B): Free4[F, B] = FMap(this)(f)
}
object Free4 {
  def pure[F[_], A](a: A): Free4[F, A] = Val(a)
}
final case class Val[F[_], A](a: A)                                            extends Free4[F, A]
final case class Bind[F[_], A, B](pa: Free4[F, B])(val f: B => Free4[F, A])    extends Free4[F, A]
final case class FMap[F[_], A, B](pa: Free4[F, B])(val f: B => A)              extends Free4[F, A]
final case class Op[F[_], A](op: F[A]) extends Free4[F, A] // Wrap all domain-specific operations.
\end{lstlisting}
\inputencoding{utf8}The code of the runner needs to be revised accordingly:\inputencoding{latin9}
\begin{lstlisting}
def run[F[_], A](runner: Runner[F]): Free4[F, A] => A = {
  case Val(a)            => a
  case bind @ Bind(pa)   => run(runner)(bind.f(run(runner)(pa)))
  case fmap @ FMap(pa)   => fmap.f(run(runner)(pa))
  case Op(op)            => runner.run(op)
} *** check that this code works
\end{lstlisting}
\inputencoding{utf8}
The code of \inputencoding{latin9}\lstinline!Free4!\inputencoding{utf8}\textsf{'}s
methods \inputencoding{latin9}\lstinline!map!\inputencoding{utf8},
\inputencoding{latin9}\lstinline!flatMap!\inputencoding{utf8}, and
\inputencoding{latin9}\lstinline!pure!\inputencoding{utf8} merely
wraps the arguments of each method into a case class without performing
any computations with those arguments. We call this implementation
of the free monad the \textbf{raw tree encoding}\index{free monad!raw tree encoding}.
The name means that a DSL program is a fully unevaluated (or \textsf{``}raw\textsf{''})
expression tree that creates a new nested case class for each step
of the required computations.

\subsection{Deriving reduced encodings of the free monad}

In the raw tree encoding (\inputencoding{latin9}\lstinline!Free4!\inputencoding{utf8}),
\emph{none} of the monad laws hold. This is not a problem in practice
because the monad laws will hold after running a DSL program. However,
we notice that the \inputencoding{latin9}\lstinline!Free3!\inputencoding{utf8}
encoding (which is the same as \inputencoding{latin9}\lstinline!MonadDSL!\inputencoding{utf8}
from Section~\ref{subsec:Stage-5:-refactoring-monadDSL}) satisfies
the monad\textsf{'}s right identity law by design. A byproduct of that design
is that \inputencoding{latin9}\lstinline!Free3!\inputencoding{utf8}
has fewer case classes than \inputencoding{latin9}\lstinline!Free4!\inputencoding{utf8}.
We also notice that the code of \inputencoding{latin9}\lstinline!Free2!\inputencoding{utf8}
uses just $2$ case classes and satisfies all monad laws.

These observations suggest that we might find an encoding with fewer
case classes (which we call a \textbf{reduced encoding}) if we use
the monad laws when designing the free monad\textsf{'}s type constructor. To
see how this works, we will first reduce \inputencoding{latin9}\lstinline!Free4!\inputencoding{utf8}
to \inputencoding{latin9}\lstinline!Free3!\inputencoding{utf8} and
then \inputencoding{latin9}\lstinline!Free3!\inputencoding{utf8}
to \inputencoding{latin9}\lstinline!Free2!\inputencoding{utf8} by
imposing certain monad laws directly on the DSL program values. These
derivations will also show that \inputencoding{latin9}\lstinline!Free2!\inputencoding{utf8},
\inputencoding{latin9}\lstinline!Free3!\inputencoding{utf8}, and
\inputencoding{latin9}\lstinline!Free4!\inputencoding{utf8} are equivalent
to each other in their expressive power.\footnote{The \inputencoding{latin9}\lstinline!Free1!\inputencoding{utf8} encoding
cannot be derived from the others because \inputencoding{latin9}\lstinline!Free1!\inputencoding{utf8}
\emph{requires} that the effect constructor \inputencoding{latin9}\lstinline!F!\inputencoding{utf8}
be a functor with a \inputencoding{latin9}\lstinline!map!\inputencoding{utf8}
method, while the other encodings work for any \inputencoding{latin9}\lstinline!F!\inputencoding{utf8}.
Later in this chapter, we will prove that \inputencoding{latin9}\lstinline!Free1!\inputencoding{utf8}
is equivalent to \inputencoding{latin9}\lstinline!Free2!\inputencoding{utf8}
when \inputencoding{latin9}\lstinline!F!\inputencoding{utf8} is a
functor. We will also show more examples of free typeclass constructions
that require another typeclass.}

To derive \inputencoding{latin9}\lstinline!Free3!\inputencoding{utf8}
from \inputencoding{latin9}\lstinline!Free4!\inputencoding{utf8},
we use the right identity law of \inputencoding{latin9}\lstinline!flatMap!\inputencoding{utf8}
to express \inputencoding{latin9}\lstinline!map!\inputencoding{utf8}
through \inputencoding{latin9}\lstinline!flatMap!\inputencoding{utf8}:\inputencoding{latin9}
\begin{lstlisting}
p.map(f) == p.flatMap (f andThen pure)
\end{lstlisting}
\inputencoding{utf8}As we have already seen, this definition makes the right identity
law hold for \inputencoding{latin9}\lstinline!Free3!\inputencoding{utf8}.
Following the form of that law, we write a function for transforming
a \inputencoding{latin9}\lstinline!Free4!\inputencoding{utf8} program
into a \inputencoding{latin9}\lstinline!Free3!\inputencoding{utf8}
program:\inputencoding{latin9}
\begin{lstlisting}
def free4toFree3[F[_], A]: Free4[F, A] => Free3[F, A] = {
  case fmap @ FMap(p)   => Bind(free4toFree3(p))(fmap.f andThen Free3.pure) // Replace FMap by Bind.
  case Val(a)           => Val(a)                                  // All other cases are unchanged.
  case bind@Bind(p)     => Bind(free4toFree3(p))(bind.f)
  case Op(op)           => Op(op)
}
\end{lstlisting}
\inputencoding{utf8}An inverse transformation (from \inputencoding{latin9}\lstinline!Free3!\inputencoding{utf8}
to \inputencoding{latin9}\lstinline!Free4!\inputencoding{utf8}) is
implemented by translating all case classes identically:\inputencoding{latin9}
\begin{lstlisting}
def free3toFree4[F[_], A]: Free3[F, A] => Free4[F, A] = {
  case Val(a)           => Val(a)
  case bind@Bind(p)     => Bind(free3toFree4(p))(bind.f)
  case Op(op)           => Op(op)
}
\end{lstlisting}
\inputencoding{utf8}The composition \inputencoding{latin9}\lstinline!free3toFree4 andThen free4toFree3!\inputencoding{utf8}
is an identity transformation because each case class is simply copied
over to the other type. However, the composition in the other order,
\inputencoding{latin9}\lstinline!free4toFree3 andThen free3toFree4!\inputencoding{utf8},
is not an identity since \inputencoding{latin9}\lstinline!free3toFree4!\inputencoding{utf8}
never creates \inputencoding{latin9}\lstinline!Free4!\inputencoding{utf8}
values of type \inputencoding{latin9}\lstinline!FMap!\inputencoding{utf8}.
It is impossible to convert any values of type \inputencoding{latin9}\lstinline!Bind!\inputencoding{utf8}
to \inputencoding{latin9}\lstinline!FMap!\inputencoding{utf8} because
it is impossible to determine whether a given function of type \inputencoding{latin9}\lstinline!A => Free3[B]!\inputencoding{utf8}
is expressible in the form \inputencoding{latin9}\lstinline!(f andThen pure)!\inputencoding{utf8}
with some \inputencoding{latin9}\lstinline!f: A => B!\inputencoding{utf8}.

It follows that \inputencoding{latin9}\lstinline!free3toFree4!\inputencoding{utf8}
is injective while \inputencoding{latin9}\lstinline!free4toFree3!\inputencoding{utf8}
is surjective. In this sense, the encoding \inputencoding{latin9}\lstinline!Free4!\inputencoding{utf8}
is \textsf{``}larger\textsf{''} than \inputencoding{latin9}\lstinline!Free3!\inputencoding{utf8}.
However, \inputencoding{latin9}\lstinline!Free4!\inputencoding{utf8}
does not express any more functionality than \inputencoding{latin9}\lstinline!Free3!\inputencoding{utf8}.
After running the corresponding \inputencoding{latin9}\lstinline!Free3!\inputencoding{utf8}
and \inputencoding{latin9}\lstinline!Free4!\inputencoding{utf8} DSL
programs, the results will be the same:\inputencoding{latin9}
\begin{lstlisting}
runFree4(runner)(free4program) == runFree3(runner)(free4toFree3(free4program))
runFree3(runner)(free3program) == runFree4(runner)(free3toFree4(free3program))
\end{lstlisting}
\inputencoding{utf8}To verify this, we first note that the case classes \inputencoding{latin9}\lstinline!Val!\inputencoding{utf8},
\inputencoding{latin9}\lstinline!Bind!\inputencoding{utf8}, and \inputencoding{latin9}\lstinline!Op!\inputencoding{utf8}
within the types \inputencoding{latin9}\lstinline!Free3!\inputencoding{utf8}
and \inputencoding{latin9}\lstinline!Free4!\inputencoding{utf8} will
be evaluated in exactly the same way by both runners. So, DSL programs
in \inputencoding{latin9}\lstinline!Free3!\inputencoding{utf8} and
\inputencoding{latin9}\lstinline!Free4!\inputencoding{utf8} containing
those case classes will give the same results when run.

It remains to verify that the runners will produce the same results
for a \inputencoding{latin9}\lstinline!Free4!\inputencoding{utf8}
program containing an \inputencoding{latin9}\lstinline!FMap!\inputencoding{utf8}
case class and for the corresponding \inputencoding{latin9}\lstinline!Free3!\inputencoding{utf8}
program. Running a \inputencoding{latin9}\lstinline!Free4!\inputencoding{utf8}
program of the form \inputencoding{latin9}\lstinline!p = FMap(q)(f)!\inputencoding{utf8},
we will get:\inputencoding{latin9}
\begin{lstlisting}
runFree4(runner)(p) == runFree4(runner)(FMap(q)(f)) == f(runFree4(runner)(q))
\end{lstlisting}
\inputencoding{utf8}The \inputencoding{latin9}\lstinline!Free3!\inputencoding{utf8} program
corresponding to \inputencoding{latin9}\lstinline!p!\inputencoding{utf8}
is:\inputencoding{latin9}
\begin{lstlisting}
free4toFree3(p) == Bind(q)(f andThen Free3.pure)
\end{lstlisting}
\inputencoding{utf8}Running the program \inputencoding{latin9}\lstinline!free4toFree3(p)!\inputencoding{utf8}
with \inputencoding{latin9}\lstinline!Free3!\inputencoding{utf8}\textsf{'}s
runner will give:\inputencoding{latin9}
\begin{lstlisting}
runFree3(runner)(free4toFree3(p)) == runFree3(runner)(Bind(free4toFree3(q))(f andThen Free3.pure))
   == runFree3(runner)(Free3.pure(f(runFree3(runner)(free4toFree3(q)))))
   == runFree3(runner)(Pure(f(runFree3(runner)(free4toFree3(q)))))
   == f(runFree3(runner)(free4toFree3(q))))
\end{lstlisting}
\inputencoding{utf8}The last expression differs from the result of evaluating \inputencoding{latin9}\lstinline!runFree4(runner)(p)!\inputencoding{utf8}
only in replacing \inputencoding{latin9}\lstinline!runFree4!\inputencoding{utf8}
by \inputencoding{latin9}\lstinline!runFree3!\inputencoding{utf8}.
So, it remains to show that:\inputencoding{latin9}
\begin{lstlisting}
f(runFree4(runner)(q)) == f(runFree3(runner)(free4toFree3(q))))
\end{lstlisting}
\inputencoding{utf8}Note that \inputencoding{latin9}\lstinline!q!\inputencoding{utf8}
is a \emph{smaller} monadic program (contains fewer case classes)
than \inputencoding{latin9}\lstinline!p!\inputencoding{utf8}. If
\inputencoding{latin9}\lstinline!q!\inputencoding{utf8} contains
case classes other than \inputencoding{latin9}\lstinline!FMap!\inputencoding{utf8},
we already showed that \inputencoding{latin9}\lstinline!runFree4!\inputencoding{utf8}
and \inputencoding{latin9}\lstinline!runFree3!\inputencoding{utf8}
give the same results. If \inputencoding{latin9}\lstinline!q!\inputencoding{utf8}
again contains the \inputencoding{latin9}\lstinline!FMap!\inputencoding{utf8}
case class, we will use the inductive assumption that \inputencoding{latin9}\lstinline!runFree4!\inputencoding{utf8}
and \inputencoding{latin9}\lstinline!runFree3!\inputencoding{utf8}
give the same results when evaluating smaller monadic programs.

Let us now derive \inputencoding{latin9}\lstinline!Free2!\inputencoding{utf8}
from \inputencoding{latin9}\lstinline!Free3!\inputencoding{utf8}.
A conversion function \inputencoding{latin9}\lstinline!free2toFree3!\inputencoding{utf8}
needs to replace \inputencoding{latin9}\lstinline!Free2!\inputencoding{utf8}\textsf{'}s
case classes (\inputencoding{latin9}\lstinline!Return!\inputencoding{utf8}
and \inputencoding{latin9}\lstinline!Bind!\inputencoding{utf8}) by
\inputencoding{latin9}\lstinline!Free3!\inputencoding{utf8}\textsf{'}s case
classes (\inputencoding{latin9}\lstinline!Pure!\inputencoding{utf8},
\inputencoding{latin9}\lstinline!Suspend!\inputencoding{utf8}, and
\inputencoding{latin9}\lstinline!FlatMap!\inputencoding{utf8}). The
\inputencoding{latin9}\lstinline!Return!\inputencoding{utf8} case
class corresponds to \inputencoding{latin9}\lstinline!Pure!\inputencoding{utf8}.
Trying to replace \inputencoding{latin9}\lstinline!Bind!\inputencoding{utf8}
by \inputencoding{latin9}\lstinline!FlatMap!\inputencoding{utf8},
we find that the type of their data are not the same: the first part
of \inputencoding{latin9}\lstinline!Bind!\inputencoding{utf8} has
type \inputencoding{latin9}\lstinline!F[A]!\inputencoding{utf8} while
the first part of \inputencoding{latin9}\lstinline!FlatMap!\inputencoding{utf8}
has type \inputencoding{latin9}\lstinline!Free3[F, A]!\inputencoding{utf8}.
We note that \inputencoding{latin9}\lstinline!Suspend!\inputencoding{utf8}
converts \inputencoding{latin9}\lstinline!F[A]!\inputencoding{utf8}
into \inputencoding{latin9}\lstinline!Free3[F, A]!\inputencoding{utf8}.
So, we write code like this:\inputencoding{latin9}
\begin{lstlisting}
def free2toFree3[F[_], A]: Free2[F, A] => Free3[F, A] = {
  case Return(a)    => Pure(a)
  case Bind(p, g)   => FlatMap(Suspend(p), g andThen free2toFree3)
}
\end{lstlisting}
\inputencoding{utf8}
The inverse conversion function (\inputencoding{latin9}\lstinline!free3toFree2!\inputencoding{utf8})
is more complicated because \inputencoding{latin9}\lstinline!Suspend!\inputencoding{utf8}
and \inputencoding{latin9}\lstinline!FlatMap!\inputencoding{utf8}
are not straightforwardly mapped into \inputencoding{latin9}\lstinline!Free2!\inputencoding{utf8}\textsf{'}s
case classes.{*}{*}{*}\inputencoding{latin9}
\begin{lstlisting}
def free3toFree2[F[_], A]: Free3[F, A] => Free2[F, A] = {
  case Pure(a)         => Return(a)
  case Suspend(f)      => Bind(f, a => Return(a))
  case FlatMap(p, g)   => ???
}
\end{lstlisting}
\inputencoding{utf8}{*}{*}{*}Some of these conversions are injective.

\subsection{Types with existential quantifiers}

The previous section showed all derivations in the Scala code syntax
rather than in the code notation. The reason is that the constructions
\inputencoding{latin9}\lstinline!Free2!\inputencoding{utf8}, \inputencoding{latin9}\lstinline!Free3!\inputencoding{utf8},
and \inputencoding{latin9}\lstinline!Free4!\inputencoding{utf8} involve
a special use of type parameters that is not supported by the type
and code notations shown in this book so far. The missing feature
is types with an \textbf{existential quantifier}\index{existential quantifier (exists)@existential quantifier ($\exists$)},
which is denoted by the symbol $\exists$ (pronounced \textsf{``}exists\textsf{''}).

To clarify the usage of the symbol $\exists$, we begin by recalling
the definition of the type \inputencoding{latin9}\lstinline!MonadDSL!\inputencoding{utf8}:\inputencoding{latin9}
\begin{lstlisting}
sealed trait MonadDSL[F[_], T] {
  def flatMap[A](...) = ...
  def map[A](f: T => A): MonadDSL[F, A] = Bind(this)(f andThen MonadDSL.pure)
}
final case class Val[F[_], T](t: T)                                          extends MonadDSL[F, T]
final case class Op[F[_], T](f: F[T])                                        extends MonadDSL[F, T]
final case class Bind[F[_], T, A](p: MonadDSL[F, A])(g: A => MonadDSL[F, T]) extends MonadDSL[F, T]
\end{lstlisting}
\inputencoding{utf8}The case class \inputencoding{latin9}\lstinline!Bind[F, T, A]!\inputencoding{utf8}
uses the type parameter \inputencoding{latin9}\lstinline!A!\inputencoding{utf8}
in a special way. Whenever a value of type \inputencoding{latin9}\lstinline!Bind[F, T, A]!\inputencoding{utf8}
is constructed, the type parameter \inputencoding{latin9}\lstinline!A!\inputencoding{utf8}
must be set to a specific type. However, the result will be a value
of type \inputencoding{latin9}\lstinline!MonadDSL[F, T]!\inputencoding{utf8},
which is \emph{not} parameterized by \inputencoding{latin9}\lstinline!A!\inputencoding{utf8}.
The extra type parameter (\inputencoding{latin9}\lstinline!A!\inputencoding{utf8})
is hidden inside the constructed value of type \inputencoding{latin9}\lstinline!Bind[F, T, A]!\inputencoding{utf8}.

To see this in a code example, let us apply \inputencoding{latin9}\lstinline!map!\inputencoding{utf8}
to a given value \inputencoding{latin9}\lstinline!p!\inputencoding{utf8}
of type \inputencoding{latin9}\lstinline!MonadDSL[F, Int]!\inputencoding{utf8}:\inputencoding{latin9}
\begin{lstlisting}
val p: MonadDSL[F, Int] = ...
val q: MonadDSL[F, String] = p.map { i: Int => s"value = $i" }     // Creates a Bind[F, String, Int].
\end{lstlisting}
\inputencoding{utf8}The value \inputencoding{latin9}\lstinline!q!\inputencoding{utf8}
is declared to be of type \inputencoding{latin9}\lstinline!MonadDSL[F, T]!\inputencoding{utf8}
with \inputencoding{latin9}\lstinline!T = String!\inputencoding{utf8}.
However, the actual data structure in \inputencoding{latin9}\lstinline!q!\inputencoding{utf8}
contains a value of type \inputencoding{latin9}\lstinline!Bind[F, T, A]!\inputencoding{utf8}
with a type parameter \inputencoding{latin9}\lstinline!A!\inputencoding{utf8}
chosen as \inputencoding{latin9}\lstinline!A = Int!\inputencoding{utf8}.
The type declaration \inputencoding{latin9}\lstinline!Bind[F, T, A](...) extends MonadDSL[F, T]!\inputencoding{utf8}
hides the type parameter \inputencoding{latin9}\lstinline!A!\inputencoding{utf8}
from the rest of the code, although that type parameter will be needed
when constructing a value \inputencoding{latin9}\lstinline!q!\inputencoding{utf8}
of type \inputencoding{latin9}\lstinline!Bind!\inputencoding{utf8}.
This usage of a type parameter is called an \textbf{existentially
quantified} type and is denoted by $\exists A$.

The other type parameters (\inputencoding{latin9}\lstinline!F!\inputencoding{utf8},
\inputencoding{latin9}\lstinline!T!\inputencoding{utf8}) in \inputencoding{latin9}\lstinline!Bind[F, T, A]!\inputencoding{utf8}
are not existentially quantified because they appear both in the case
class and in the parent trait. So, we write the notation for \inputencoding{latin9}\lstinline!Bind[F, T, A]!\inputencoding{utf8}
as: 
\[
\exists A.\,\text{Bind}^{F,T,A}\triangleq\exists A.\,\text{MonadDSL}^{F,A}\times(A\rightarrow\text{MonadDSL}^{F,T})\quad.
\]
Adding the type notation for the case classes \inputencoding{latin9}\lstinline!Val!\inputencoding{utf8}
and \inputencoding{latin9}\lstinline!Op!\inputencoding{utf8}, we
get a definition of the type \inputencoding{latin9}\lstinline!MonadDSL!\inputencoding{utf8}:
\begin{align*}
\text{MonadDSL}^{F,T} & \triangleq\text{Val}^{F,T}+\text{Op}^{F,T}+\exists A.\,\text{Bind}^{F,T,A}\\
 & \quad\quad=T+F^{T}+\exists A.\,\text{MonadDSL}^{F,A}\times(A\rightarrow\text{MonadDSL}^{F,T})\quad.
\end{align*}

To interpret the name \textsf{``}existential\textsf{''}, we may imagine that a type
parameter \inputencoding{latin9}\lstinline!A!\inputencoding{utf8}
\textsf{``}still exists\textsf{''} inside the value \inputencoding{latin9}\lstinline!q: MonadDSL[F, T]!\inputencoding{utf8},
since \inputencoding{latin9}\lstinline!q!\inputencoding{utf8} must
have been created as \inputencoding{latin9}\lstinline!Bind[F, T, A](...)(...)!\inputencoding{utf8}
with a specific (somehow chosen) type \inputencoding{latin9}\lstinline!A!\inputencoding{utf8}.
However, we cannot inspect or obtain the type \inputencoding{latin9}\lstinline!A!\inputencoding{utf8}
if we only have the value \inputencoding{latin9}\lstinline!q!\inputencoding{utf8}.
No pattern-matching or other operations on \inputencoding{latin9}\lstinline!q!\inputencoding{utf8}
can determine whether the type \inputencoding{latin9}\lstinline!A!\inputencoding{utf8}
is, say, \inputencoding{latin9}\lstinline!Int!\inputencoding{utf8}
or \inputencoding{latin9}\lstinline!String!\inputencoding{utf8}.
We can use the value \inputencoding{latin9}\lstinline!q!\inputencoding{utf8}
in our code as long as we treat \inputencoding{latin9}\lstinline!A!\inputencoding{utf8}
as an unknown, arbitrary, but fixed type.

Scala gives two ways of defining types with existential quantifiers:
via case classes with an extra type parameter, or via a class with
a type member. To illustrate these techniques, consider a type expression
$\exists C.\,L^{A,B,C}$ where $L$ is a type constructor with type
parameters $A$, $B$, $C$ and an existential quantifier on $C$.
Note that the type parameter\index{bound type parameter} $C$ is
\textbf{bound} by the quantifier and will not be visible outside the
type expression $\exists C.\,L^{A,B,C}$. The type parameters $A$
and $B$ remain free in $\exists C.\,L^{A,B,C}$.

An example of such a type expression is $\exists C.\,A\times C\times\left(C\rightarrow B\right)$.
Let us implement this type via case classes and via type members.
The code using case classes may look like this:\inputencoding{latin9}
\begin{lstlisting}
sealed trait L[A, B]
final case class L1[A, B, C](a: A, c: C, p: C => B) extends L[A, B]
\end{lstlisting}
\inputencoding{utf8}
The type notation for \inputencoding{latin9}\lstinline!L[A, B]!\inputencoding{utf8}
is:
\[
L^{A,B}\triangleq\exists C.\,\text{L1}^{A,B,C}=\exists C.\,A\times C\times(C\rightarrow B)\quad.
\]

When we create a value of type \inputencoding{latin9}\lstinline!L[A, B]!\inputencoding{utf8},
we need to assign the type $C$:\inputencoding{latin9}
\begin{lstlisting}
val q1: L[Int, Int] = L1[Int, Int, String](1, "abc", _.length)
\end{lstlisting}
\inputencoding{utf8}External code may use data of type \inputencoding{latin9}\lstinline!L[A, B]!\inputencoding{utf8}
as long as it does not need to know the type assigned to $C$:\inputencoding{latin9}
\begin{lstlisting}
def toInt: L[Int, Int] => Int = {
  case L1(a, c, p) => a + p(c) // May apply p: C => B to c: C without knowing what C is.
}

scala> toInt(q1)
res0: Int = 4
\end{lstlisting}
\inputencoding{utf8}In this code, the expression \inputencoding{latin9}\lstinline!p(c)!\inputencoding{utf8}
uses data of unknown type \inputencoding{latin9}\lstinline!C!\inputencoding{utf8}
but does not try to inspect that type.

To implement the type $\exists C.\,A\times C\times\left(C\rightarrow B\right)$
via a trait with a type member, we write:\inputencoding{latin9}
\begin{lstlisting}
trait L2[A, B] {
  type C
  val a: A
  val c: C
  val p: C => B
}
val q2: L2[Int, Int] = new L2[Int, Int] {
  type C = String
  val a: Int = 1
  val c: String = "abc"
  val p: String => Int = _.length
}
def toInt(q: L2[Int, Int]): Int = q.a + q.p(q.c)

scala> toInt(q2)
res1: Int = 4
\end{lstlisting}
\inputencoding{utf8}The code is more verbose but has equivalent functionality to the code
that uses a case class. Each time we create a value of type \inputencoding{latin9}\lstinline!L[A, B]!\inputencoding{utf8}
or \inputencoding{latin9}\lstinline!L2[A, B]!\inputencoding{utf8},
we will have to specify some type for \inputencoding{latin9}\lstinline!C!\inputencoding{utf8}.
Code external to the trait \inputencoding{latin9}\lstinline!L2!\inputencoding{utf8}
will no longer know what type was chosen as $C$. It will be a type
error if external code depends on the actual type assigned to \inputencoding{latin9}\lstinline!C!\inputencoding{utf8}:\inputencoding{latin9}
\begin{lstlisting}
scala> val x: q2.C = "abc"        // The type q2.C is String but the code may not use that knowledge.
                    ^
error: type mismatch;
found   : String("abc")
required: q2.C
\end{lstlisting}
\inputencoding{utf8}External code must treat the type \inputencoding{latin9}\lstinline!q2.C!\inputencoding{utf8}
as an unknown, arbitrary type.

We will use the existential quantifier notation for proofs of properties
of free monads and other constructions. With the new notation for
existential types, the free monad constructions are written as:
\begin{align*}
{\color{greenunder}\text{free monad on }F\text{ in a reduced encoding}:}\quad & \text{Free}_{2}^{F,T}\triangleq T+\exists A.\,F^{A}\times(A\rightarrow\text{Free}_{2}^{F,T})\quad,\\
{\color{greenunder}\text{free monad on }F\text{ in a reduced encoding}:}\quad & \text{Free}_{3}^{F,T}\triangleq T+F^{T}+\exists A.\,\text{Free}_{3}^{F,A}\times(A\rightarrow\text{Free}_{3}^{F,T})\quad,\\
{\color{greenunder}\text{free monad on }F\text{ in a raw tree encoding}:}\quad & \text{Free}_{4}^{F,A}\triangleq A+F^{A}+\exists B.\,\text{Free}_{4}^{F,B}\times(B\rightarrow A)\\
 & \quad\quad+\exists B.\,\text{Free}_{4}^{F,B}\times(B\rightarrow\text{Free}_{4}^{F,A})\quad.
\end{align*}


\subsection{Expressing types with existential quantifiers via universal quantifiers}

We will now motivate and define the connection between existential
and universal quantifiers.

The previous section shows that a type expression with an existentially
quantified type, such as $\exists C.\,A\times C\times\left(C\rightarrow B\right)$,
hides the actual type under the parameter $C$ and only exposes its
existence. Fully parametric\index{fully parametric!code} external
code may work with values of type $C$ as long as $C$ is treated
as an unknown and arbitrary type. This agrees with the restriction
that fully parametric code may not make decisions based on the actual
type assigned to a type parameter.

Let us formulate this property in more precise terms. An example of
\textsf{``}external code\textsf{''} is the function \inputencoding{latin9}\lstinline!toInt!\inputencoding{utf8}
shown in the previous section. The type signature and the code of
\inputencoding{latin9}\lstinline!toInt!\inputencoding{utf8} is written
in the short notation as:
\[
\text{toInt}:\left(\exists C.\,\text{Int}\times C\times(C\rightarrow\text{Int})\right)\rightarrow\text{Int}\quad,\quad\quad\text{toInt}\triangleq(\exists C.\,a^{:\text{Int}}\times c^{:C}\times p^{:C\rightarrow\text{Int}})\rightarrow a+p(c)\quad.
\]
Generalizing from this example, we say that \textsf{``}code external to the
existential quantifier\textsf{''} means any function having an argument of
an existentially quantified type. For instance, this could be a function
of type $(\exists C.\,F^{C})\rightarrow D$ where $F^{C}$ is any
type expression and $D$ is any type. The existential quantifier allows
us to use values of the type $C$ as long as the code does not depend
on a specific type that is hidden under $C$. This means treating
$C$ as a \emph{type parameter}. Functions with type parameters have
type signatures with the \emph{universal} quantifier, such as $\forall C.\:F^{C}\rightarrow G^{C}$,
showing that the same code works with any type $C$. So, we a fully
parametric function of type $(\exists C.\,F^{C})\rightarrow D$ must
be a function of type $F^{C}\rightarrow D$ that works in the same
way for all types $C$. We usually write the types of such functions
as $\forall C.\,F^{C}\rightarrow D$. We conclude that the following
types are equivalent:
\begin{equation}
\text{for any fixed type }D:\quad(\exists C.\,F^{C})\rightarrow D\cong\forall C.\,(F^{C}\rightarrow D)\quad.\label{eq:existential-via-universal}
\end{equation}

It is important that the universal quantifier in $\forall C.\,F^{C}\rightarrow D$
is \emph{outside} the entire type expression $F^{C}\rightarrow D$.
(The type notations $\forall C.\,(F^{C}\rightarrow D)$ and $\forall C.\,F^{C}\rightarrow D$
are equivalent.) In contrast, the type $(\exists C.\,F^{C})\rightarrow D$
contains the existential quantifier ($\exists C$) \emph{inside} a
function argument.

To illustrate the type equivalence~(\ref{eq:existential-via-universal}),
we may rewrite the function \inputencoding{latin9}\lstinline!toInt!\inputencoding{utf8}
as:\inputencoding{latin9}
\begin{lstlisting}
def toIntL1[C]: L1[Int, Int, C] => Int = { case L1(a, c, p) => a + p(c) }
\end{lstlisting}
\inputencoding{utf8}The function body of \inputencoding{latin9}\lstinline!toIntL1!\inputencoding{utf8}
is the same as that of \inputencoding{latin9}\lstinline!toInt!\inputencoding{utf8},
although the type signature changed.

To get more intuition for Eq.~(\ref{eq:existential-via-universal}),
first consider a value with a universally quantified type:\inputencoding{latin9}
\begin{lstlisting}
def f[A]: F[A] = ...
\end{lstlisting}
\inputencoding{utf8}The code of the function \inputencoding{latin9}\lstinline!f!\inputencoding{utf8}
can work with any type \inputencoding{latin9}\lstinline!A!\inputencoding{utf8}.
If Scala did not support type parameters, we would need to define
the function \inputencoding{latin9}\lstinline!f!\inputencoding{utf8}
separately for each possible type, as if in a giant infinite tuple:
\inputencoding{latin9}\lstinline!(f[Int], f[String], f[List[Int]], ...)!\inputencoding{utf8}.
In this sense, the definition of \inputencoding{latin9}\lstinline!f[A]!\inputencoding{utf8}
replaces an \textsf{``}infinite product\textsf{''} ranging over all possible types:
\[
f:\forall A.\,F^{A}\quad,\quad\quad\forall A.\,F^{A}=F^{\text{Int}}\times F^{\text{String}}\times F^{\text{List}^{\text{Int}}}\times...
\]
This formula is not rigorous (the \textsf{``}infinite product\textsf{''} is not clearly
defined). Importantly, this formula does not express the requirement
that the code of the function \inputencoding{latin9}\lstinline!f!\inputencoding{utf8}
must work in the same way for all types \inputencoding{latin9}\lstinline!A!\inputencoding{utf8}.
We will use this formula only as a heuristic illustration.

A value of an existentially quantified type, such as $\exists C.\,F^{C}$,
must be created by using a specific type for $C$. So, we may view
the type $\exists C.\,F^{C}$ as an \textsf{``}infinite disjunction\textsf{''} ranging
over all possible types that could have been used for $C$:
\[
\exists C.\,F^{C}=F^{\text{Int}}+F^{\text{String}}+F^{\text{List}^{\text{Int}}}+...
\]
Now look at the following type equivalence where a disjunctive type
is in the function argument:
\[
\left(A+B\right)\rightarrow D\cong\left(A\rightarrow D\right)\times\left(B\rightarrow D\right)\quad.
\]
Write the same type equivalence using the infinite product and the
infinite disjunction:
\[
(F^{\text{Int}}+F^{\text{String}}+F^{\text{List}^{\text{Int}}}+...)\rightarrow D\cong(F^{\text{Int}}\rightarrow D)\times(F^{\text{String}}\rightarrow D)\times(F^{\text{List}^{\text{Int}}}\rightarrow D)\times...
\]
This (non-rigorous) formula can be rewritten in terms of type quantifiers
as:
\[
(\exists C.\,F^{C})\rightarrow D\cong\forall C.\,(F^{C}\rightarrow D)\quad.
\]
This is the same as Eq.~(\ref{eq:existential-via-universal}).

Another way to see why Eq.~(\ref{eq:existential-via-universal})
works is to view the case class constructor \inputencoding{latin9}\lstinline!L1[A, B, C]()!\inputencoding{utf8}
as a function from a triple of type \inputencoding{latin9}\lstinline!(A, C, C => B)!\inputencoding{utf8}
to a value of type \inputencoding{latin9}\lstinline!L[A, B]!\inputencoding{utf8}.
For clarity, let us denote that function temporary by \inputencoding{latin9}\lstinline!f!\inputencoding{utf8}
and write:\inputencoding{latin9}
\begin{lstlisting}
def f[A, B, C](a: A, c: C, p: C => B): L[A, B] = L1(a, c, p) 
\end{lstlisting}
\inputencoding{utf8}This code is just a function with some type parameters; that is, a
function that works in the same way for all types. So, the type notation
for the function \inputencoding{latin9}\lstinline!f!\inputencoding{utf8}
must be written via the \emph{universal} quantifiers for each type
parameter:
\[
f:\forall(A,B,C).\,A\times C\times(C\rightarrow B)\rightarrow L^{A,B}\quad.
\]
The equivalent notation with the existential quantifier for $C$ looks
like this:
\[
f:\forall(A,B).\,\big(\exists C.\,A\times C\times(C\rightarrow B)\big)\rightarrow L^{A,B}=\forall(A,B).\,\big(\exists C.\,\text{L1}^{A,B,C})\rightarrow L^{A,B}\quad.
\]
This explains why the existential quantifier is represented in Scala
by a case class with an extra type parameter, such as \inputencoding{latin9}\lstinline!L1[A, B, C]!\inputencoding{utf8}.
The type parameter \inputencoding{latin9}\lstinline!C!\inputencoding{utf8}
is existentially quantified because it does not appear in \inputencoding{latin9}\lstinline!L[A, B]!\inputencoding{utf8},
while the two other parameters (\inputencoding{latin9}\lstinline!A!\inputencoding{utf8}
and \inputencoding{latin9}\lstinline!B!\inputencoding{utf8}) do appear
in \inputencoding{latin9}\lstinline!L[A, B]!\inputencoding{utf8}
and are therefore universally quantified.

Motivated by these considerations, we may \emph{define} the meaning
of the symbol $\exists$ through Eq.~(\ref{eq:existential-via-universal}).
It is more convenient to use another formula that expresses $\exists C$
via $\forall C$:

\subsubsection{Statement \label{subsec:Statement-type-equivalence-existential-universal-Yoneda}\ref{subsec:Statement-type-equivalence-existential-universal-Yoneda}}

For any type constructor $F^{\bullet}$ (not necessarily covariant
or contravariant):
\begin{equation}
\exists C.\,F^{C}\cong\forall D.\,(\forall C.\,(F^{C}\rightarrow D))\rightarrow D\quad.\label{eq:existential-via-universal-Yoneda}
\end{equation}


\subparagraph{Proof}

The covariant Yoneda identity (Statement \ref{subsec:Statement-covariant-yoneda-identity-for-types})
shows that $Z\cong\forall D.\,(Z\rightarrow D)\rightarrow D$, where
$Z$ is any fixed type (independent of $D$). Set $Z\triangleq\exists C.\,F^{C}$
and find:
\begin{align*}
 & \gunderline{\exists C.\,F^{C}}\\
{\color{greenunder}\text{by covariant Yoneda identity}:}\quad & \cong\forall D.\,(\gunderline{(\exists C.\,F^{C})\rightarrow D})\rightarrow D\\
{\color{greenunder}\text{by Eq.~(\ref{eq:existential-via-universal})}:}\quad & \cong\forall D.\,(\gunderline{\forall C.\,(F^{C}\rightarrow D)})\rightarrow D\quad.
\end{align*}
$\square$

\section{Free constructions for other typeclasses}

The free monad was motivated by implementing a type-safe DSL with
variable binding. However, we may view the free monad as just a trick
that wraps an arbitrary type constructor in some case classes to obtain
the features of a monad. Are there similar tricks for adding the features
of an applicative functor, a filterable functor, etc., to an arbitrary
type constructor? 

Looking at the raw tree encoding and other encodings of the free monad,
we find a recipe that we can generalize to many other typeclasses.
Let us now formulate this recipe more formally.

The first step is to define the raw tree encoding of the free typeclass.
For the free monad, we wrote a case class for each of the monad\textsf{'}s
standard methods (\inputencoding{latin9}\lstinline!map!\inputencoding{utf8},
\inputencoding{latin9}\lstinline!flatMap!\inputencoding{utf8}, \inputencoding{latin9}\lstinline!pure!\inputencoding{utf8})
and another case class for wrapping the given set of DSL operations.
The same recipe gives a raw tree encoding for many other free typeclasses,
as long as the typeclass operations are of the form of functions whose
last return type is again of the same typeclass. For instance, a monad
\inputencoding{latin9}\lstinline!M!\inputencoding{utf8}\textsf{'}s \inputencoding{latin9}\lstinline!flatMap!\inputencoding{utf8}
has the type signature of the form \inputencoding{latin9}\lstinline!... => M[A]!\inputencoding{utf8},
where the last return type uses the same monad \inputencoding{latin9}\lstinline!M!\inputencoding{utf8}.

The second step is to implement the \textsf{``}universal runner\textsf{''} that transforms
a free typeclass value into a value of another type belonging to the
same typeclass. For instance, the universal runner for the free monad
on \inputencoding{latin9}\lstinline!F!\inputencoding{utf8} takes
a polymorphic function of type $\forall A.\,F^{A}\rightarrow M^{A}$,
where $M$ is any other monad, and runs the free monad\textsf{'}s effects into
$M$\textsf{'}s effects.

Typically, the raw tree encoding will not satisfy the laws of the
typeclass. (However, the laws will hold after applying the runner.)
The third step is to impose the typeclass laws and to obtain a reduced
encoding that uses a smaller number of case classes and satisfies
the laws automatically. The code of the universal runner needs to
be revised to work with the reduced encoding.

The following sections will show several examples of these constructions.

\subsection{Free pointed types}

One of the simplest typeclasses is the \textbf{pointed}\index{pointed type}
type, that is, a type that has a designated default value. This is
the \inputencoding{latin9}\lstinline!HasDefault!\inputencoding{utf8}
typeclass from Example~\ref{subsec:tc-Example-Pointed-type}. Let
us now apply the free typeclass construction to this example. We will
obtain a type constructor \inputencoding{latin9}\lstinline!FreeDefault[T]!\inputencoding{utf8}
that wraps an arbitrary type \inputencoding{latin9}\lstinline!T!\inputencoding{utf8}
and has a \inputencoding{latin9}\lstinline!HasDefault!\inputencoding{utf8}
type instance.

The first step is to formulate the required methods of the typeclass
as functions with known type signatures. The \inputencoding{latin9}\lstinline!HasDefault!\inputencoding{utf8}
typeclass for a pointed type \inputencoding{latin9}\lstinline!P!\inputencoding{utf8}
has only one operation: obtaining a value \inputencoding{latin9}\lstinline!default!\inputencoding{utf8}
of type \inputencoding{latin9}\lstinline!P!\inputencoding{utf8}.
This operation is equivalent to a function of type $\bbnum 1\rightarrow P$.
The free typeclass will model this operation via a case class containing
a value of unit type (equivalently, a named unit). We may write the
code like this:\inputencoding{latin9}
\begin{lstlisting}
sealed trait FreeDefault[T]
final case class Default[T](unit: Unit) extends FreeDefault[T]
final case class Op[T](op: T) extends FreeDefault[T]
\end{lstlisting}
\inputencoding{utf8}In the type notation, this is written as:
\[
\text{FreeDefault}^{T}\triangleq\bbnum 1+T\quad.
\]
 We find that \inputencoding{latin9}\lstinline!FreeDefault[T]!\inputencoding{utf8}
is equivalent to \inputencoding{latin9}\lstinline!Option[T]!\inputencoding{utf8},
so we will define it that way:\inputencoding{latin9}
\begin{lstlisting}
type FreeDefault[T] = Option[T]
\end{lstlisting}
\inputencoding{utf8}

\subsubsection{Definition \label{subsec:Definition-free-pointed-type}\ref{subsec:Definition-free-pointed-type}}

The \textbf{free pointed type} on\index{free pointed type} a given
type $T$ is the type $\bbnum 1+T$.

The \inputencoding{latin9}\lstinline!HasDefault!\inputencoding{utf8}
typeclass has no laws, so the raw tree encoding ($\bbnum 1+T$) cannot
be reduced by imposing any typeclass laws. 

The free pointed typeclass comes with a \textsf{``}universal runner\textsf{''}, which
is a function from \inputencoding{latin9}\lstinline!FreeDefault[T]!\inputencoding{utf8}
to a chosen pointed type $P$. The runner needs to know how to translate
$T$ into $P$, and the code can be written as:\inputencoding{latin9}
\begin{lstlisting}
def runner[T, P: HasDefault](run: T => P): Option[T] => P = {
  case Some(t)   => run(t)
  case None      => implicitly[HasDefault[P]].value          // The default value of type P.
}
\end{lstlisting}
\inputencoding{utf8}This logic repeats the Scala standard library\textsf{'}s \inputencoding{latin9}\lstinline!getOrElse!\inputencoding{utf8}
method for \inputencoding{latin9}\lstinline!Option!\inputencoding{utf8}.
For any value \inputencoding{latin9}\lstinline!x: Option[T]!\inputencoding{utf8}:\inputencoding{latin9}
\begin{lstlisting}
runner(run)(x) == x.map(run).getOrElse(implicitly[HasDefault[P]].value)
\end{lstlisting}
\inputencoding{utf8}So, we can view \inputencoding{latin9}\lstinline!getOrElse!\inputencoding{utf8}
as a general form of the universal runner for the free pointed typeclass.

What if the type $T$ is already pointed? The free pointed type on
$T$ is $\bbnum 1+T$, which is never equivalent to $T$. However,
the universal runner can be applied to the identity function of type
$T\rightarrow T$ and gives a function of type $\bbnum 1+T\rightarrow T$.
When $T$ is itself of the form $\bbnum 1+U$ then that function (of
type $\bbnum 1+\bbnum 1+U\rightarrow\bbnum 1+U$) is the standard
method \inputencoding{latin9}\lstinline!flatten!\inputencoding{utf8}
defined on \inputencoding{latin9}\lstinline!Option!\inputencoding{utf8}
types.

In this way, \inputencoding{latin9}\lstinline!Option!\inputencoding{utf8}\textsf{'}s
standard methods \inputencoding{latin9}\lstinline!getOrElse!\inputencoding{utf8}
and \inputencoding{latin9}\lstinline!flatten!\inputencoding{utf8}
are seen as regular properties of the free pointed type construction.

\subsection{Free semigroups}

A semigroup (see Example~\ref{subsec:tc-Example-Semigroups}) is
a typeclass with a single method called \inputencoding{latin9}\lstinline!combine!\inputencoding{utf8}.
A common notation for \inputencoding{latin9}\lstinline!combine!\inputencoding{utf8}
is $\oplus$, used as an infix binary operation with type signature
$\oplus:T\times T\rightarrow T$. The semigroup\textsf{'}s law is the associativity
law for the operation $\oplus$.

How can we convert an arbitrary type $T$ into a semigroup? Following
the general recipe, we first construct the raw tree encoding of a
free semigroup on a given type $T$. There is one case class for the
binary operation and another for wrapping a value of type $T$.\inputencoding{latin9}
\begin{lstlisting}
sealed trait FSR[T]
final case class Combine[T](left: FSR[T], right: FSR[T]) extends FSR[T]
final case class Wrap[T](value: T) extends FSR[T]
\end{lstlisting}
\inputencoding{utf8}The short notation for this type constructor is:
\[
\text{FSR}^{T}\triangleq T+\text{FSR}^{T}\times\text{FSR}^{T}\quad.
\]
We can now see that \inputencoding{latin9}\lstinline!FSR[T]!\inputencoding{utf8}
is a binary tree with leaf values of type $T$ (see Section~\subsecref{Binary-trees}).

We can implement a \inputencoding{latin9}\lstinline!Semigroup!\inputencoding{utf8}
typeclass instance for \inputencoding{latin9}\lstinline!FSR[T]!\inputencoding{utf8}
like this:\inputencoding{latin9}
\begin{lstlisting}
implicit def semigroupFSR[T]: Semigroup[FSR[T]] = Semigroup((l, r) => Combine(l, r)) *** verify code
\end{lstlisting}
\inputencoding{utf8}
For convenience, we will define a syntax extension so that we can
use the infix binary operation \inputencoding{latin9}\lstinline!|+|!\inputencoding{utf8}:\inputencoding{latin9}
\begin{lstlisting}
implicit class SemigroupOp[S: Semigroup](s: S) {
  def |+|(other: S): S = implicitly[Semigroup[S]].combine(s, other)
}*** check if it works
\end{lstlisting}
\inputencoding{utf8}
As usual with raw tree encodings, the free semigroup\textsf{'}s binary operation
\inputencoding{latin9}\lstinline!|+|!\inputencoding{utf8} does not
perform any computations but simply wraps its arguments into the case
class \inputencoding{latin9}\lstinline!Combine!\inputencoding{utf8}.
A \textsf{``}free semigroup program\textsf{''} (specifically, an \inputencoding{latin9}\lstinline!FSR!\inputencoding{utf8}-program)
is an unevaluated expression tree containing a number of nested \inputencoding{latin9}\lstinline!Combine!\inputencoding{utf8}
case classes as well as some values of type \inputencoding{latin9}\lstinline!T!\inputencoding{utf8}
wrapped in \inputencoding{latin9}\lstinline!Wrap!\inputencoding{utf8}. 

The next step is to implement a universal runner that will evaluate
that expression tree. Given any semigroup \inputencoding{latin9}\lstinline!S!\inputencoding{utf8}
and a function \inputencoding{latin9}\lstinline!T => S!\inputencoding{utf8},
we can run \inputencoding{latin9}\lstinline!FSR[T]!\inputencoding{utf8}
into \inputencoding{latin9}\lstinline!S!\inputencoding{utf8} like
this:\inputencoding{latin9}
\begin{lstlisting}
def runner[S: Semigroup, T](runT: T => S): FSR[T] => S = {
  case Combine(left, right)   => runner(runT)(left) |+| runner(runT)(right)
  case Wrap(value)            => runT(value)
}
\end{lstlisting}
\inputencoding{utf8}
This code is analogous to the code of \inputencoding{latin9}\lstinline!foldMap!\inputencoding{utf8}
\index{foldMap function@\texttt{foldMap} function}for aggregating
tree-like data (Section~\subsecref{Aggregating-tree-like-data-bfs}).

Let us see an example of an \inputencoding{latin9}\lstinline!FSR!\inputencoding{utf8}-program
for a free semigroup on \inputencoding{latin9}\lstinline!String!\inputencoding{utf8}:\inputencoding{latin9}
\begin{lstlisting}
val fsfProgram: FSR[String] = Wrap("abc") |+| (Wrap("xyz") |+| Wrap(""))
\end{lstlisting}
\inputencoding{utf8} To run this program, we need to choose another semigroup (\inputencoding{latin9}\lstinline!S!\inputencoding{utf8})
and provide a function that maps \inputencoding{latin9}\lstinline!String!\inputencoding{utf8}
to \inputencoding{latin9}\lstinline!S!\inputencoding{utf8}. We choose
\inputencoding{latin9}\lstinline!S = Int!\inputencoding{utf8}, a
semigroup with respect to integer addition. The function \inputencoding{latin9}\lstinline!runT: String => Int!\inputencoding{utf8}
will return the length of the string. In this way, we can compute
the total length of all strings within the \inputencoding{latin9}\lstinline!FSR!\inputencoding{utf8}-program:\inputencoding{latin9}
\begin{lstlisting}
*** test that this works
implicit semigroupInt: Semigroup[Int] = Semigroup(_ + _)

scala> runner(_.length)(fsfProgram)
res0: Int = 6
\end{lstlisting}
\inputencoding{utf8}
The associativity law of \inputencoding{latin9}\lstinline!|+|!\inputencoding{utf8}
will hold after applying the runner to an \inputencoding{latin9}\lstinline!FSR!\inputencoding{utf8}-program
because associativity is assumed to hold for the operation \inputencoding{latin9}\lstinline!|+|!\inputencoding{utf8}
of the semigroup \inputencoding{latin9}\lstinline!S!\inputencoding{utf8}.
However, the raw tree encoding itself does not obey the associativity
law because the nested data structure \inputencoding{latin9}\lstinline!Combine(Combine(x, y), z)!\inputencoding{utf8}
is not equal to \inputencoding{latin9}\lstinline!Combine(x, Combine(y, z))!\inputencoding{utf8}.
The next step is to look for a reduced encoding of the free semigroup
that obeys the associativity law.

The associativity law would hold if the result of \inputencoding{latin9}\lstinline!Combine(x, y) |+| z!\inputencoding{utf8}
were not \inputencoding{latin9}\lstinline!Combine(Combine(x, y), z)!\inputencoding{utf8}
but \inputencoding{latin9}\lstinline!Combine(x, Combine(y, z))!\inputencoding{utf8}.
To achieve this, we could redefine the \inputencoding{latin9}\lstinline!|+|!\inputencoding{utf8}
operation by modifying the \inputencoding{latin9}\lstinline!Semigroup!\inputencoding{utf8}
typeclass instance:\inputencoding{latin9}
\begin{lstlisting}
implicit def semigroupFSR[T]: Semigroup[FSR[T]] = Semigroup((l, r) => l match {
  case Wrap(value)     => Combine(l, r)
  case Combine(p, q)   => p |+| (q |+| r) // Recursive call of |+|.
}) *** verify that this code works
\end{lstlisting}
\inputencoding{utf8}The data structure from the new \inputencoding{latin9}\lstinline!|+|!\inputencoding{utf8}
operation will be of the form \inputencoding{latin9}\lstinline!Combine(Wrap(x), Combine(Wrap(y), ...))!\inputencoding{utf8},
and the associativity law will always hold. 

The new data structure is equivalent to a non-empty list containing
values of type \inputencoding{latin9}\lstinline!Wrap[T]!\inputencoding{utf8}.
The resulting type (\inputencoding{latin9}\lstinline!NEL[Wrap[T]]!\inputencoding{utf8})
can be simplified to just \inputencoding{latin9}\lstinline!NEL[T]!\inputencoding{utf8}
since a nested \inputencoding{latin9}\lstinline!Wrap!\inputencoding{utf8}
carries no functionality by itself. So, we can reuse the non-empty
list type \inputencoding{latin9}\lstinline!NEL!\inputencoding{utf8}
and its \inputencoding{latin9}\lstinline!concat!\inputencoding{utf8}
function (see Example~\ref{subsec:Disjunctive-Example-non-empty-list-foldLeft}
and Exercise~\ref{subsec:Disjunctive-Exercise-non-empty-list-2}).
The universal runner is similar to the code of the \inputencoding{latin9}\lstinline!foldLeft!\inputencoding{utf8}
function shown in Example~\ref{subsec:Disjunctive-Example-non-empty-list-foldLeft}:\inputencoding{latin9}
\begin{lstlisting}
implicit def semigroupNEL[T]: Semigroup[NEL[T]] = Semigroup((l, r) => concat(l, r))
 
@tailrec def runner[S: Semigroup, T](runT: T => S)(n: NEL[T]): S = n match {
  case Last(x)        => runT(x)
  case More(x, tail)  => runner(runT)(tail)
}
\end{lstlisting}
\inputencoding{utf8}
This defines \inputencoding{latin9}\lstinline!NEL[T]!\inputencoding{utf8}
as a reduced encoding of the free semigroup on \inputencoding{latin9}\lstinline!T!\inputencoding{utf8}. 

Comparing the raw tree encoding \inputencoding{latin9}\lstinline!FSR[T]!\inputencoding{utf8}
and the reduced encoding \inputencoding{latin9}\lstinline!NEL[T]!\inputencoding{utf8},
we find some differences in run-time performance. For \inputencoding{latin9}\lstinline!NEL[T]!\inputencoding{utf8},
the binary operation \inputencoding{latin9}\lstinline!|+|!\inputencoding{utf8}
involves concatenating two lists, which may require traversing the
lists. The raw tree encoding\textsf{'}s binary operation takes constant time
as it only wraps the data in a new case class. However, \inputencoding{latin9}\lstinline!NEL[T]!\inputencoding{utf8}\textsf{'}s
run operation may be implemented with tail recursion while \inputencoding{latin9}\lstinline!FSR[T]!\inputencoding{utf8}
is a binary tree whose traversal cannot be tail-recursive.

\subsection{Free monoid and its partially lawful encodings\label{subsec:Free-monoids}}

A monoid (see Example~\ref{subsec:tc-Example-Monoids}) is a typeclass
with a two methods: \inputencoding{latin9}\lstinline!combine!\inputencoding{utf8}
and \inputencoding{latin9}\lstinline!empty!\inputencoding{utf8}.
The monoid\textsf{'}s laws are the associativity law and two identity laws.

To convert an arbitrary type $T$ into a free monoid on $T$, we follow
the general recipe and write the raw tree encoding. There is one case
class for the binary operation, one case class for \inputencoding{latin9}\lstinline!empty!\inputencoding{utf8},
and one for wrapping a value of type $T$.\inputencoding{latin9}
\begin{lstlisting}
sealed trait FMR[T]
final case class Combine[T](left: FMR[T], right: FMR[T]) extends FMR[T]
final case class Empty[T]() extends FMR[T]
final case class Wrap[T](value: T) extends FMR[T]
\end{lstlisting}
\inputencoding{utf8}The short notation for this type constructor is:
\[
\text{FMR}^{T}\triangleq\bbnum 1+T+\text{FMR}^{T}\times\text{FMR}^{T}\quad.
\]
We can see that \inputencoding{latin9}\lstinline!FMR[T]!\inputencoding{utf8}
is just a binary tree with leaf values of type $\bbnum 1+T$:
\[
\text{FMR}^{T}=\text{Tree2}^{\bbnum 1+T}\quad.
\]
The binary tree type \inputencoding{latin9}\lstinline!Tree2!\inputencoding{utf8}
was defined in Section~\ref{subsec:Binary-trees}. The data type
$\text{FMR}^{T}$ represents an \emph{unevaluated} expression tree
built from the monoid operations and from values of type $T$, for
example:

\begin{wrapfigure}{l}{0.6\columnwidth}%
\vspace{-0.4\baselineskip}
\inputencoding{latin9}\begin{lstlisting}
val exampleFMR: FMR[Int] = Combine(Empty(), Combine( Combine(Wrap(456), Empty()), Wrap(123)))
\end{lstlisting}
\inputencoding{utf8}\vspace{0.2\baselineskip}
\end{wrapfigure}%

\noindent {\tiny{}\hspace{0.1\columnwidth}}{\tiny{} \Tree[ [ $e$ ] [ [ [ $456$ ] [ $e$ ] ] [ $123$ ] ] ] }{\tiny\par}

We can implement a \inputencoding{latin9}\lstinline!Monoid!\inputencoding{utf8}
typeclass instance for \inputencoding{latin9}\lstinline!FMR[T]!\inputencoding{utf8}
like this:\inputencoding{latin9}
\begin{lstlisting}
implicit def monoidFMR[T]: Monoid[FMR[T]] = Monoid((l, r) => Combine(l, r), Empty())
\end{lstlisting}
\inputencoding{utf8}We define a syntax extension for the infix binary operation \inputencoding{latin9}\lstinline!|+|!\inputencoding{utf8}
as in the previous section.

As usual with raw tree encodings, this free monoid\textsf{'}s operations not
perform any computations but only create nested case classes. Those
nested case classes represent an unevaluated expression tree of an
\textsf{``}\inputencoding{latin9}\lstinline!FMR!\inputencoding{utf8}-program\textsf{''}.
The next step is to implement a universal runner that will evaluate
that expression tree using the operations of any chosen monoid \inputencoding{latin9}\lstinline!M!\inputencoding{utf8}.
Given a function \inputencoding{latin9}\lstinline!T => M!\inputencoding{utf8},
we can convert any \inputencoding{latin9}\lstinline!FMR!\inputencoding{utf8}-program
(a value of type \inputencoding{latin9}\lstinline!FMR[T]!\inputencoding{utf8})
into a value of type \inputencoding{latin9}\lstinline!M!\inputencoding{utf8}
like this:\inputencoding{latin9}
\begin{lstlisting}
def runnerFMR[M: Monoid, T](runT: T => M)(fmr: FMR[T]): M = fmr match {
  case Combine(left, right)   => runnerFMR(runT)(left) |+| runnerFMR(runT)(right)
  case Empty()                => Monoid[M].empty
  case Wrap(value)            => runT(value)
}
\end{lstlisting}
\inputencoding{utf8}
The raw tree encoding \inputencoding{latin9}\lstinline!FMR[T]!\inputencoding{utf8}
does not satisfy any of the monoid laws (the associativity law and
the two identity laws). This is not a problem in practice, since the
laws will hold after running an \inputencoding{latin9}\lstinline!FMR!\inputencoding{utf8}-program
into any lawful monoid \inputencoding{latin9}\lstinline!M!\inputencoding{utf8}.
Imposing the monoid laws will help us find reduced encodings of the
free monoid.

As with the free semigroup, the associativity law will hold if we
replace a binary tree by a non-empty list. The result is a non-empty
list of values of type $\bbnum 1+T$. The type \inputencoding{latin9}\lstinline!NEL[Option[T]]!\inputencoding{utf8}
is a possible encoding of the free monoid; it satisfies the associativity
law but fails the identity laws. The empty value of the free monoid
is represented by a single-element list containing \inputencoding{latin9}\lstinline!None!\inputencoding{utf8}
(in the code notation, $[1+\bbnum 0^{:T}]$). The identity laws of
the monoid would hold if lists of the form: 
\[
\left[\bbnum 0+x_{1},...,\bbnum 0+x_{m},1+\bbnum 0,\bbnum 0+y_{1},...,\bbnum 0+y_{n}\right]
\]
were replaced by lists $\left[x_{1},...,x_{m},y_{1},...,y_{n}\right]$
that do not contain any \inputencoding{latin9}\lstinline!None!\inputencoding{utf8}
values. This would mean that it is enough to use \inputencoding{latin9}\lstinline!NEL[T]!\inputencoding{utf8}
instead of \inputencoding{latin9}\lstinline!NEL[Option[T]]!\inputencoding{utf8},
except for the possibility that a non-empty list contains only \inputencoding{latin9}\lstinline!None!\inputencoding{utf8}
values. For simplicity, we can represent that situation by an \emph{empty}
list. So, we may replace \inputencoding{latin9}\lstinline!NEL[Option[T]]!\inputencoding{utf8}
by simply \inputencoding{latin9}\lstinline!List[T]!\inputencoding{utf8}.
The binary operation for \inputencoding{latin9}\lstinline!List[T]!\inputencoding{utf8}
is just the \inputencoding{latin9}\lstinline!concat!\inputencoding{utf8}
function for lists; the empty list is the empty monoid value.

The type \inputencoding{latin9}\lstinline!List[T]!\inputencoding{utf8}
is the shortest reduced encoding for the free monoid on $T$, and
it satisfies all monoid laws. The runner is the same as the \inputencoding{latin9}\lstinline!foldMap!\inputencoding{utf8}
function for lists (see Section~\ref{subsec:From-reduce-and-foldleft-to-foldmap}):\index{foldMap function@\texttt{foldMap} function}\inputencoding{latin9}
\begin{lstlisting}
def runnerList[M: Monoid, T](runT: T => M): List[T] => M = foldMap[M, T](runT)
\end{lstlisting}
\inputencoding{utf8}
There exist other reduced encodings that satisfy only a subset of
the monoid laws. As an example, let us define a reduced encoding of
the free monoid that satisfies the identity laws but not the associativity
law. To impose the identity laws on $\text{Tree2}^{\bbnum 1+T}$,
we note that the identity laws reduce expression trees creating \textsf{``}empty\textsf{''}
leaf values ($1+\bbnum 0^{:T}$) to expression trees that do not contain
any \textsf{``}empty\textsf{''} values in the leaves. For instance, the value \inputencoding{latin9}\lstinline!exampleFMR!\inputencoding{utf8}
defined above will be reduced to just \inputencoding{latin9}\lstinline!Combine(Wrap(456), Wrap(123))!\inputencoding{utf8}.
If all leaves of the expression tree are \inputencoding{latin9}\lstinline!Empty!\inputencoding{utf8},
the tree must be reduced to just a single empty value. This means
we can simplify the encoding from $\text{Tree2}^{\bbnum 1+T}$ to
$\bbnum 1+\text{Tree2}^{T}$.

To summarize, we have obtained four different \textsf{``}partially lawful\textsf{''}
free monoid encodings:\\
\textbf{1)} The encoding $F_{1}^{T}\triangleq\text{Tree2}^{\bbnum 1+T}$
(called \inputencoding{latin9}\lstinline!FMR[T]!\inputencoding{utf8}
above) satisfies none of the monoid laws. The code is:\inputencoding{latin9}
\begin{lstlisting}
type F1[T] = Tree2[Option[T]]
def wrapF1[T](t: T): F1[T] = Leaf(Some(t))
implicit def monoidF1[T]: Monoid[F1[T]] = Monoid((l, r) => Branch(l, r), Leaf(None))
def runnerF1[M: Monoid, T](runT: T => M)(fmr: F1[T]): M = fmr match {
  case Branch(left, right)   => runnerF1(runT)(left) |+| runnerF1(runT)(right)
  case Leaf(None)            => Monoid[M].empty
  case Leaf(Some(value))     => runT(value)
}
\end{lstlisting}
\inputencoding{utf8}
\textbf{2)} The encoding $F_{2}^{T}\triangleq\text{NEL}^{\bbnum 1+T}$
satisfies only the associativity law. The code is:\inputencoding{latin9}
\begin{lstlisting}
type F2[T] = NEL[Option[T]]
def wrapF2[T](t: T): F2[T] = (Some(t), Nil)
implicit def monoidF2[T]: Monoid[F2[T]] = Monoid(NEL.concat, (None, Nil))
def runnerF2[M: Monoid, T](runT: T => M)(fmr: F2[T]): M = foldMap(runT).apply(fmr.toList.flatten) 
\end{lstlisting}
\inputencoding{utf8}For completeness, here is a simple implementation of non-empty lists
and some methods on them:\inputencoding{latin9}
\begin{lstlisting}
type NEL[T] = (T, List[T])
object NEL {
  def concat[T]: (NEL[T], NEL[T]) => NEL[T] = {
    case ((head1, tail1), (head2, tail2)) => (head1, tail1 ++ List(head2) ++ tail2)
  }
  implicit class ToList[T](nel: NEL[T]) {
    def toList: List[T] = nel._1 +: nel._2
  }
}
\end{lstlisting}
\inputencoding{utf8}
\textbf{3)} The encoding $F_{3}^{T}\triangleq\bbnum 1+\text{Tree2}^{T}$
satisfies only the identity laws. The code is:\inputencoding{latin9}
\begin{lstlisting}
type F3[T] = Option[Tree2[T]]
def wrapF3[T](t: T): F3[T] = Some(Leaf(t))
def concatF3[T]: (F3[T], F3[T]) => F3[T] = {
  case (None, x)            => x
  case (x, None)            => x
  case (Some(a), Some(b))   => Some(Branch(a, b))
}
def wrapF3[T](t: T): F3[T] = Some(Leaf(t))
implicit def monoidF3[T]: Monoid[F3[T]] = Monoid(concatF3, None)
def runnerTree2[M: Monoid, T](runT: T => M)(tree2: Tree2[T]): M = tree2 match {
  case Leaf(a)               => runT(a)
  case Branch(left, right)   => runnerTree2(runT)(left) |+| runnerTree2(runT)(right)
}
def runnerF3[M: Monoid, T](runT: T => M)(fmr: F3[T]): M = fmr match {
  case Some(t)   => runnerTree2(runT)(t)
  case None      => Monoid[M].empty
}
\end{lstlisting}
\inputencoding{utf8}
\textbf{4)} The encoding $F_{4}^{T}\triangleq\text{List}^{T}$ satisfies
all the monoid laws. The code is:\inputencoding{latin9}
\begin{lstlisting}
type F4[T] = List[T]
def wrapF4[T](t: T): F4[T] = List(t)
implicit def monoidF4[T]: Monoid[F4[T]] = Monoid(_ ++ _, Nil)
def runnerF4[M: Monoid, T](runT: T => M)(fmr: F4[T]): M = foldMap(runT).apply(fmr)
\end{lstlisting}
\inputencoding{utf8}
To get more intuition about using these encodings, let us explore
whether the types \inputencoding{latin9}\lstinline!F1!\inputencoding{utf8},
\inputencoding{latin9}\lstinline!F2!\inputencoding{utf8}, \inputencoding{latin9}\lstinline!F3!\inputencoding{utf8},
\inputencoding{latin9}\lstinline!F4!\inputencoding{utf8} can be mapped
into each other. All of those encodings have a \inputencoding{latin9}\lstinline!Monoid!\inputencoding{utf8}
instance, so we can use their runner functions to map any encoding
into any other:\inputencoding{latin9}
\begin{lstlisting}
def f1_to_f2[T]: F1[T] => F2[T] = runnerF1(wrapF2[T](_))
def f3_to_f2[T]: F3[T] => F2[T] = runnerF3(wrapF2[T](_))
def f4_to_f3[T]: F4[T] => F3[T] = runnerF4(wrapF3[T](_)) // And so on.
\end{lstlisting}
\inputencoding{utf8}But it turns out that some of those mappings fail to preserve the
monoid operations. For instance, converting a list $\left[1,2,3\right]$
of type \inputencoding{latin9}\lstinline!F4[Int]!\inputencoding{utf8}
to the type \inputencoding{latin9}\lstinline!F3[Int]!\inputencoding{utf8}
will create the following tree:\inputencoding{latin9}
\begin{lstlisting}
scala> f4_to_f3(List(1, 2, 3))
res0: F3[Int] = Some(Branch(Branch(Leaf(1), Leaf(2)), Leaf(3)))
\end{lstlisting}
\inputencoding{utf8}The same value \inputencoding{latin9}\lstinline!List(1, 2, 3)!\inputencoding{utf8}
can be computed via \inputencoding{latin9}\lstinline!F4!\inputencoding{utf8}\textsf{'}s
monoid operation as \inputencoding{latin9}\lstinline!List(1) |+| List(2, 3)!\inputencoding{utf8}.
However, converting the two shorter lists to \inputencoding{latin9}\lstinline!F3[Int]!\inputencoding{utf8}
and combining them via \inputencoding{latin9}\lstinline!F3!\inputencoding{utf8}\textsf{'}s
monoid operation will give a different tree:\inputencoding{latin9}
\begin{lstlisting}
scala> f4_to_f3(List(1)) |+| f4_to_f3(List(2, 3))
res1: F3[Int] = Some(Branch(Leaf(1), Branch(Leaf(2), Leaf(3))))
\end{lstlisting}
\inputencoding{utf8}The trees are not the same because \inputencoding{latin9}\lstinline!F3!\inputencoding{utf8}
does not obey the monoid associativity law while \inputencoding{latin9}\lstinline!F4!\inputencoding{utf8}
does.

{*}{*}{*}Show injectivity of those transformations

Looking at these examples, we find that whenever a free monoid encoding
\inputencoding{latin9}\lstinline!P!\inputencoding{utf8} obeys \emph{more
laws} than another encoding \inputencoding{latin9}\lstinline!Q!\inputencoding{utf8},
the transformation of type \inputencoding{latin9}\lstinline!P => Q!\inputencoding{utf8}
does not preserve the monoid operations (while the opposite one, \inputencoding{latin9}\lstinline!Q => P!\inputencoding{utf8},
does). At the same time, the transformation of type \inputencoding{latin9}\lstinline!P => Q!\inputencoding{utf8}
is injective, suggesting that the encoding is \textsf{``}smaller\textsf{''} when it
satisfies more laws. These observations will be generalized in Section~\ref{subsec:Free--typeclasses-that-satisfy-laws}
to a rigorous theory of free typeclass encodings that satisfy only
a subset of the laws of a given typeclass.

\subsection{Free functors}

Consider the \inputencoding{latin9}\lstinline!Functor!\inputencoding{utf8}
typeclass whose only method is \inputencoding{latin9}\lstinline!fmap!\inputencoding{utf8}:
\[
\text{fmap}:\left(A\rightarrow B\right)\rightarrow F^{A}\rightarrow F^{B}\quad.
\]
For some type constructors $F$, the \inputencoding{latin9}\lstinline!fmap!\inputencoding{utf8}
method is not available. An example is when $F$ is a contrafunctor
or an unfunctor. (See Section~\ref{subsec:Examples-of-non-functors}
for more examples.) Let us now apply the raw tree encoding recipe
to the \inputencoding{latin9}\lstinline!Functor!\inputencoding{utf8}
typeclass. The result will be a new type constructor that we call
a \textbf{free functor on} $F$.\index{free functor} Here $F$ is
the effect constructor and does not need to be covariant with respect
to its type parameter.

The recipe tells us to define a case class for the \inputencoding{latin9}\lstinline!fmap!\inputencoding{utf8}
method and another case class to wrap the given type constructor $F$.
So, the raw tree encoding of a free functor on $F$ looks like this:\inputencoding{latin9}
\begin{lstlisting}
sealed trait FFR[F[_], A] {
  def map[B](f: A => B): FFR[F, B] = FMap(f, this)
}
final case class FMap[F[_], X, Y](f: X => Y, p: FFR[F, X]) extends FFR[F, Y]
final case class Op[F[_], A](op: F[A]) extends FFR[F, A]
\end{lstlisting}
\inputencoding{utf8}
This code corresponds to the following notation for the lifting $f^{\uparrow\text{FFR}}$:
\begin{align}
 & \text{FFR}^{F^{\bullet},A}\triangleq F^{A}+\exists X.\,(X\rightarrow A)\times\text{FFR}^{F^{\bullet},X}\quad,\label{eq:definition-FFR-existential-type}\\
 & p^{:\text{FFR}^{F^{\bullet},A}}\triangleright(f^{:A\rightarrow B})^{\uparrow\text{FFR}^{F^{\bullet},\bullet}}\triangleq\bbnum 0^{:F^{B}}+\exists^{A}.\,f^{:A\rightarrow B}\times p^{:\text{FFR}^{F^{\bullet},A}}\quad.\nonumber 
\end{align}
The notation $\exists^{A}$ means that we bind the type $A$ to the
existentially quantified type $X$ in the definition~(\ref{eq:definition-FFR-existential-type})
of $\text{FFR}^{F^{\bullet},B}$. This notation expresses the requirement
that the existentially quantified type must be assigned to a specific
type every time we create a specific value.

A \textsf{``}free functor program\textsf{''} is a value of type \inputencoding{latin9}\lstinline!FFR[F, A]!\inputencoding{utf8}.
To construct such values, we need to begin with a wrapped \inputencoding{latin9}\lstinline!F!\inputencoding{utf8}-operation
followed by some \inputencoding{latin9}\lstinline!map!\inputencoding{utf8}
methods. To use a free functor program in practice, we need to apply
a runner to it. A simple runner is a function of type $\forall A.\,\text{FFR}^{F,A}\rightarrow A$
that extracts a value of type $A$ from a free functor program. To
obtain a runner, we need to know how to extract values from the effect
constructor $F$. That information is given by a function of type
$\forall C.\,F^{C}\rightarrow C$ (an \index{effect runner}effect
runner for $F$). To implement such functions, we use the trait \inputencoding{latin9}\lstinline!Runner!\inputencoding{utf8}
defined in Section~\ref{subsec:A-first-recipe-monadic-dsl}. Now
we can write the code of the runner for $\text{FFR}^{F,A}$:\inputencoding{latin9}
\begin{lstlisting}
def runFFR[F[_], A](runner: Runner[F]): FFR[F, A] => A = {
  case FMap(f, p)   => f(runFFR(runner)(p))
  case Op(op)       => runner.run(op)
}*** check if this works
\end{lstlisting}
\inputencoding{utf8}The code notation for this function is:
\begin{align*}
 & \text{runFFR}^{F^{\bullet},A}:(\forall C.\,F^{C}\rightarrow C)\rightarrow\text{FFR}^{F^{\bullet},A}\rightarrow A\quad,\\
 & \text{runFFR}(\text{run}:\forall C.\,F^{C}\rightarrow C)\triangleq\forall B.\,\,\begin{array}{|c||c|}
 & A\\
\hline F^{A} & \text{run}\\
(B\rightarrow A)\times\text{FFR}^{F^{\bullet},B} & f^{:B\rightarrow A}\times p^{:FFR^{F^{\bullet},B}}\rightarrow p\triangleright\big(\overline{\text{runFFR}}(\text{run})\big)\triangleright f
\end{array}\quad.
\end{align*}
The outside universal quantifier $\forall B$ replaces $\exists B$
in the function argument, according to Eq.~(\ref{eq:existential-via-universal}).

More generally, we may want to run the effects of $F$ into the effects
of a given functor $G$. (The functor $G$ could, for example, describe
errors or asynchronous execution.) The corresponding runner will have
type $\forall C.\,F^{C}\rightarrow G^{C}$, and the code is: 
\begin{align*}
 & \text{runFFR}^{F^{\bullet},G^{\bullet},A}:(\forall C.\,F^{C}\rightarrow G^{C})\rightarrow\text{FFR}^{F^{\bullet},A}\rightarrow G^{A}\quad,\\
 & \text{runFFR}\,(\text{run})\triangleq\forall B.\,\,\begin{array}{|c||c|}
 & G^{A}\\
\hline F^{A} & \text{run}\\
(B\rightarrow A)\times\text{FFR}^{F^{\bullet},B} & f^{:B\rightarrow A}\times p^{:FFR^{F^{\bullet},B}}\rightarrow p\triangleright\big(\overline{\text{runFFR}}\,(\text{run})\big)\triangleright f^{\uparrow G}
\end{array}\quad.
\end{align*}

The type constructor \inputencoding{latin9}\lstinline!FFR!\inputencoding{utf8}
is a \index{free functor!raw tree encoding}raw tree encoding of the
free functor and does not satisfy the functor laws. This is not a
problem in practice, because the functor laws will be satisfied after
we run an \inputencoding{latin9}\lstinline!FFR!\inputencoding{utf8}-program
into any lawful functor $G$:

\subsubsection{Statement \label{subsec:Statement-free-functor-raw-tree-encoding-satisfies-laws}\ref{subsec:Statement-free-functor-raw-tree-encoding-satisfies-laws}}

Take any free functor ($\text{FFR}^{F^{\bullet},A}$) and any runner
($\text{run}:\forall C.\,F^{C}\rightarrow G^{C}$), where $G$ is
a lawful functor. Denote for brevity $f^{\uparrow\text{FFR}}\triangleq f^{\uparrow\text{FFR}^{F^{\bullet},\bullet}}$.
The functor laws will hold if we apply the runner function, denoted
for brevity by $\rho\triangleq\text{runFFR}\,(\text{run})$, to both
sides of the laws:
\begin{align*}
{\color{greenunder}\text{identity law}:}\quad & \text{id}^{\uparrow\text{FFR}}\bef\text{runFFR}\,(\text{run})=\text{runFFR}\,(\text{run})\quad,\\
{\color{greenunder}\text{composition law}:}\quad & f^{\uparrow\text{FFR}}\bef g^{\uparrow\text{FFR}}\bef\text{runFFR}\,(\text{run})=(f\bef g)^{\uparrow\text{FFR}}\bef\text{runFFR}\,(\text{run})\quad.
\end{align*}
These equations hold even though $\text{id}^{\uparrow\text{FFR}}\neq\text{id}$
and $f^{\uparrow\text{FFR}}\bef g^{\uparrow\text{FFR}}\neq(f\bef g)^{\uparrow\text{FFR}}$.

\subparagraph{Proof}

Apply both sides of the laws to an arbitrary \inputencoding{latin9}\lstinline!FFR!\inputencoding{utf8}-program
$p^{:\text{FFR}^{F^{\bullet},A}}$:
\begin{align*}
{\color{greenunder}\text{identity law}:}\quad & p\triangleright\text{id}^{\uparrow\text{FFR}}\bef\rho=p\triangleright\rho\quad,\\
{\color{greenunder}\text{composition law}:}\quad & p\triangleright f^{\uparrow\text{FFR}}\triangleright g^{\uparrow\text{FFR}}\triangleright\rho=p\triangleright(f\bef g)^{\uparrow\text{FFR}}\triangleright\rho\quad.
\end{align*}

To verify the identity law, we write the code matrices of the functions:
\begin{align*}
{\color{greenunder}\text{expect to equal }p\triangleright\rho:}\quad & \gunderline{p\triangleright\text{id}^{\uparrow\text{FFR}}}\triangleright\rho\\
 & =\big(\bbnum 0+\text{id}\times p\big)\triangleright\,\begin{array}{||c|}
\text{run}\\
k\times p\rightarrow p\triangleright\rho\triangleright k^{\uparrow G}
\end{array}\,=p\triangleright\rho\triangleright\gunderline{\text{id}^{\uparrow G}}\\
{\color{greenunder}\text{identity law of }G:}\quad & =p\triangleright\rho\quad.
\end{align*}
To verify the composition law, begin by computing the left-hand side:
\begin{align*}
 & p\triangleright f^{\uparrow\text{FFR}}\triangleright g^{\uparrow\text{FFR}}\triangleright\rho=(\bbnum 0+f\times p)\triangleright g^{\uparrow\text{FFR}}\triangleright\rho=(\bbnum 0+g\times(\bbnum 0+f\times p))\triangleright\rho\\
 & =\big(\bbnum 0+g\times(\bbnum 0+f\times p)\big)\triangleright\,\begin{array}{||c|}
\text{run}\\
k\times p\rightarrow p\triangleright\rho\triangleright k^{\uparrow G}
\end{array}\,=\gunderline{(\bbnum 0+f\times p)\triangleright\rho}\triangleright g^{\uparrow G}\\
{\color{greenunder}\text{definition of }\rho:}\quad & =p\triangleright\rho\triangleright f^{\uparrow G}\triangleright g^{\uparrow G}\\
{\color{greenunder}\text{composition law of }G:}\quad & =p\triangleright\rho\triangleright(f\bef g)^{\uparrow G}\quad.
\end{align*}
The right-hand side is then simplified to the same code:
\begin{align*}
 & p\triangleright(f\bef g)^{\uparrow\text{FFR}}\triangleright\rho=(\bbnum 0+(f\bef g)\times p)\triangleright\rho\\
{\color{greenunder}\text{definition of }\rho:}\quad & =p\triangleright\rho\triangleright(f\bef g)^{\uparrow G}\quad.
\end{align*}
$\square$

To transform the raw tree encoding of the free functor into a reduced
encoding, we require that all functor laws should hold even before
applying a runner. We begin by finding out in detail why the functor
laws fail to hold for \inputencoding{latin9}\lstinline!FFR[F, A]!\inputencoding{utf8}.

The type \inputencoding{latin9}\lstinline!FFR[F, A]!\inputencoding{utf8}
is a disjunction of two case classes, \inputencoding{latin9}\lstinline!FMap!\inputencoding{utf8}
and \inputencoding{latin9}\lstinline!Op!\inputencoding{utf8}. The
functor\textsf{'}s composition law says that the composition of lifted functions,
$f^{\uparrow\text{FFR}}\bef g^{\uparrow\text{FFR}}$, must be equal
to the lifted composition: $(f\bef g)^{\uparrow\text{FFR}}$. If we
apply $(f\bef g)^{\uparrow\text{FFR}}$ to a value of the form \inputencoding{latin9}\lstinline!Op(op)!\inputencoding{utf8},
we will get \inputencoding{latin9}\lstinline!FMap(f andThen g, Op(op))!\inputencoding{utf8}.
However, applying the composition $f^{\uparrow\text{FFR}}\bef g^{\uparrow\text{FFR}}$
to \inputencoding{latin9}\lstinline!Op(op)!\inputencoding{utf8},
we obtain a different value: \inputencoding{latin9}\lstinline!FMap(g, FMap(f, Op(op)))!\inputencoding{utf8}.
That value has nested \inputencoding{latin9}\lstinline!FMap!\inputencoding{utf8}
classes instead of the composition of \inputencoding{latin9}\lstinline!f!\inputencoding{utf8}
and \inputencoding{latin9}\lstinline!g!\inputencoding{utf8}. The
composition law would hold if the \inputencoding{latin9}\lstinline!map!\inputencoding{utf8}
method created a non-nested \inputencoding{latin9}\lstinline!FMap!\inputencoding{utf8}
class containing the composition of \inputencoding{latin9}\lstinline!f!\inputencoding{utf8}
and \inputencoding{latin9}\lstinline!g!\inputencoding{utf8}. To achieve
that, let us define \inputencoding{latin9}\lstinline!map!\inputencoding{utf8}
on the \inputencoding{latin9}\lstinline!FMap!\inputencoding{utf8}
case class like this:\inputencoding{latin9}
\begin{lstlisting}
final case class FMap[F[_], X, Y](f: X => Y, p: FFR[F, X]) extends FFR[F, Y] {
  def map[Z](g: Y => Z): FFR[F, Z] = FMap[F, X, Z](f andThen g, p)
}
\end{lstlisting}
\inputencoding{utf8}
Turn now to the functor\textsf{'}s identity law, which says that the lifted
identity function, $\text{id}^{\uparrow\text{FFR}}$, must be again
an identity function. The new definition of \inputencoding{latin9}\lstinline!map!\inputencoding{utf8}
for the \inputencoding{latin9}\lstinline!FMap!\inputencoding{utf8}
case class satisfies that law. However, applying $\text{id}^{\uparrow\text{FFR}}$
to a value \inputencoding{latin9}\lstinline!Op(op)!\inputencoding{utf8}
does not return \inputencoding{latin9}\lstinline!Op(op)!\inputencoding{utf8}
but instead gives \inputencoding{latin9}\lstinline!FMap(identity, Op(op))!\inputencoding{utf8}.
Values of the form \inputencoding{latin9}\lstinline!Op(op)!\inputencoding{utf8}
represent \inputencoding{latin9}\lstinline!F!\inputencoding{utf8}-effects.
The functor identity law would hold if we instead represented $F$-effects
by the \inputencoding{latin9}\lstinline!FMap!\inputencoding{utf8}
case class as \inputencoding{latin9}\lstinline!FMap(identity, Op(op))!\inputencoding{utf8}.

Any \inputencoding{latin9}\lstinline!FFR!\inputencoding{utf8}-program
must have the form \inputencoding{latin9}\lstinline!Op(x).map(y).map(z)...!\inputencoding{utf8},
having zero or more \inputencoding{latin9}\lstinline!map!\inputencoding{utf8}
methods. If we represent \inputencoding{latin9}\lstinline!F!\inputencoding{utf8}-effects
by \inputencoding{latin9}\lstinline!FMap(identity, Op(op))!\inputencoding{utf8}
and implement the \inputencoding{latin9}\lstinline!map!\inputencoding{utf8}
methods for \inputencoding{latin9}\lstinline!FMap!\inputencoding{utf8}
as shown above, it will follow that \emph{all} \inputencoding{latin9}\lstinline!FFR!\inputencoding{utf8}-programs
always have the form \inputencoding{latin9}\lstinline!FMap(f, Op(op))!\inputencoding{utf8}
for some function \inputencoding{latin9}\lstinline!f!\inputencoding{utf8}.
So, we may remove the \inputencoding{latin9}\lstinline!Op!\inputencoding{utf8}
case class altogether.

The result is a simplified definition of the free functor. The complete
code is:\inputencoding{latin9}
\begin{lstlisting}
sealed trait FF[F[_], A] {
  def map[B](f: A => B): FF[F, B]
}
final case class FMap[F[_], X, Y](f: X => Y, p: F[X]) extends FF[F, Y] {
  def map[Z](g: Y => Z): FF[F, Z] = FMap[F, X, Z](f andThen g, p)
}
def runFF[F[_], A](runner: Runner[F]): FF[F, A] => A = {
  case FMap(f, p)   => f(runner.run(p))
}*** check if this works
\end{lstlisting}
\inputencoding{utf8}
The code notation for this code and a general runner is:
\begin{align*}
 & \text{FF}^{F^{\bullet},A}\triangleq\exists C.\,\left(C\rightarrow A\right)\times F^{C}\quad,\quad\quad(\exists C.\,f^{:C\rightarrow A}\times p^{:F^{C}})\triangleright(g^{:A\rightarrow B})^{\uparrow\text{FR}}\triangleq\exists^{C}.\,(f\bef g)\times p\quad,\\
 & \text{runFF}:(\forall C.\,F^{C}\rightarrow G^{C})\rightarrow\text{FF}^{F^{\bullet},A}\rightarrow G^{A}\quad,\\
 & \text{runFF}\,(\text{run})\triangleq(\exists C.\,f^{:C\rightarrow A}\times p^{:F^{C}})\rightarrow p\triangleright\text{run}\triangleright f^{\uparrow G}\quad.
\end{align*}
This is the \textbf{reduced encoding}\index{free functor!reduced encoding}
of the free functor on $F$. This encoding was derived by imposing
the functor laws, so those laws will hold for values of type \inputencoding{latin9}\lstinline!FF[F, A]!\inputencoding{utf8}
even before applying a runner.

The free functor construction \inputencoding{latin9}\lstinline!FF[F, A]!\inputencoding{utf8}
converts \emph{any} type constructor \inputencoding{latin9}\lstinline!F[_]!\inputencoding{utf8}
into a lawful functor. What if \inputencoding{latin9}\lstinline!F[_]!\inputencoding{utf8}
is already a functor? It turns out that the type \inputencoding{latin9}\lstinline!FF[F, A]!\inputencoding{utf8}
will then be \emph{equivalent} to \inputencoding{latin9}\lstinline!F[A]!\inputencoding{utf8},
assuming that all code that uses \inputencoding{latin9}\lstinline!FF[F, A]!\inputencoding{utf8}
is fully parametric. This property of the reduced encoding is called
the \textbf{co-Yoneda identity}\index{co-Yoneda identity}:
\begin{align*}
{\color{greenunder}\text{covariant co-Yoneda identity}:}\quad & \exists C.\,\left(C\rightarrow A\right)\times F^{C}\cong F^{A}\quad\text{for any functor }F\quad.
\end{align*}
We will prove this type equivalence in Statement~\ref{subsec:Statement-co-Yoneda-two-identities}
below. 

We conclude that the reduced encoding of the free functor (\inputencoding{latin9}\lstinline!FF[F, A]!\inputencoding{utf8})
has advantages over the raw tree encoding (\inputencoding{latin9}\lstinline!FFR[F, A]!\inputencoding{utf8}).
The reduced encoding contains only one case class, and the runner
code is not recursive because an \inputencoding{latin9}\lstinline!FF!\inputencoding{utf8}-program
does not contain any nested case classes. If $F$ is already a functor,
the reduced encoding of a free functor on $F$ is equivalent to $F$.

A disadvantage of the reduced encoding is that the function composition
is done in the code \inputencoding{latin9}\lstinline!FMap(f andThen g, p)!\inputencoding{utf8}.
The Scala compiler cannot directly handle the composition of a large
number of functions without causing a stack overflow. This problem
can be resolved if we postpone the function composition and instead
create a list of functions that need to be composed. The runner can
evaluate the list of functions without running into a stack overflow.

This gives us an idea for another encoding of the stack-safe free
functor, which we will call \inputencoding{latin9}\lstinline!FFS!\inputencoding{utf8}.
First, we implement a data structure called \inputencoding{latin9}\lstinline!FuncSeq!\inputencoding{utf8}
for storing a list of functions with matching types. A value of type
\inputencoding{latin9}\lstinline!FuncSeq[A, B]!\inputencoding{utf8}
holds a list of functions with types $A\rightarrow C_{1}$, $C_{1}\rightarrow C_{2}$,
..., $C_{n-1}\rightarrow C_{n}$, $C_{n}\rightarrow B$, where $C_{i}$
are some chosen types. A list of with those types can be composed
to yield a function of type $A\rightarrow B$. To simplify code, we
will cast all intermediate types to \inputencoding{latin9}\lstinline!Any!\inputencoding{utf8}
and back. Our code will take care to construct \inputencoding{latin9}\lstinline!FuncSeq!\inputencoding{utf8}
values with correct types.\inputencoding{latin9}\lstinline!final case class FuncSeq[X, Y](first: X => Any, funcs: Vector[Any => Any]) {  def append[Z](g: Y => Z): FuncSeq[X, Z] = FuncSeq(first, funcs :+ g.asInstanceOf[Any => Any])}!\inputencoding{utf8}

To ensure stack safety when working with \inputencoding{latin9}\lstinline!FuncSeq!\inputencoding{utf8},
we implement a tail-recursive function \inputencoding{latin9}\lstinline!runSeq!\inputencoding{utf8}
that composes all functions stored in the sequence and applies them
to a given value.

\inputencoding{latin9}\begin{lstlisting}
@tailrec def runSeq[X, Y](x: X, p: FuncSeq[X, Y]): Y = p.funcs.headOption match {
  case None => p.first(x).asInstanceOf[Y]
  case Some(second) => runSeq(p.first(x), FuncSeq(second, p.funcs.tail))
}
\end{lstlisting}
\inputencoding{utf8}Now we can write the code of the free functor \inputencoding{latin9}\lstinline!FFS!\inputencoding{utf8}:\inputencoding{latin9}
\begin{lstlisting}
sealed trait FFS[F[_], A] {
  def map[B](f: A => B): FFS[F, B]
}
final case class FMap[F[_], X, Y](f: FuncSeq[X, Y], p: F[X]) extends FFS[F, Y] {
  def map[Z](g: Y => Z): FFS[F, Z] = FMap[F, X, Z](f append g, p)
}
def runFF[F[_], A](runner: Runner[F]): FFS[F, A] => A = {
 case FMap(f, p) => runSeq(runner.run(p), f)
}*** check if this works
\end{lstlisting}
\inputencoding{utf8}

\subsection{Free contrafunctors}

Method $\text{contramap}:C^{A}\times\left(B\rightarrow A\right)\rightarrow C^{B}$ 

Tree encoding: $\text{FreeCF}^{F^{\bullet},B}\triangleq F^{B}+\exists A.\text{FreeCF}^{F^{\bullet},A}\times\left(B\rightarrow A\right)$

Reduced encoding: $\text{FreeCF}^{F^{\bullet},B}\triangleq\exists A.F^{A}\times\left(B\rightarrow A\right)$ 

A value of type $\text{FreeCF}^{F^{\bullet},B}$ must be of the form
{\footnotesize{}
\[
\exists Z_{1}.\exists Z_{2}...\exists Z_{n}.F^{Z_{1}}\times\left(B\rightarrow Z_{n}\right)\times\left(Z_{n}\rightarrow Z_{n-1}\right)\times...\times\left(Z_{2}\rightarrow Z_{1}\right)
\]
}{\footnotesize\par}

The functions $B\rightarrow Z_{n}$, $Z_{n}\rightarrow Z_{n-1}$,
etc., are composed associatively

The equivalent type is $\exists Z_{1}.F^{Z_{1}}\times\left(B\rightarrow Z_{1}\right)$

The reduced encoding is non-recursive

Example: $F^{A}\triangleq A$, \textsf{``}interpret\textsf{''} into the contrafunctor
$C^{A}\triangleq A\rightarrow\text{String}$

\texttt{\textcolor{blue}{\footnotesize{}def prefixLog{[}A{]}(p: A): A
$\rightarrow$ String = a $\rightarrow$ p.toString + a.toString}}{\footnotesize\par}

If $F^{\bullet}$ is already a contrafunctor then $\text{FreeCF}^{F^{\bullet},A}\cong F^{A}$

\subsection{Free constructions that assume other typeclasses}

It is sometimes possible to find a simpler encoding of a free typeclass
on a type $T$ if we assume that $T$ already has another typeclass
instance.

The first example is the free monoid on $T$ if $T$ is already a
semigroup. The only thing missing in a semigroup compared with a monoid
is the empty element: a semigroup does not necessarily have one. We
also notice that the difference between the free monoid (\inputencoding{latin9}\lstinline!List[T]!\inputencoding{utf8})
and the free semigroup (\inputencoding{latin9}\lstinline!NEL[T]!\inputencoding{utf8})
is just the empty element: $\text{List}^{T}=\bbnum 1+\text{NEL}^{T}$.
Motivated by these considerations, we define the free monoid on $T$
as $\bbnum 1+T$ when $T$ is already a semigroup. Indeed, we may
implement the monoid instance:\inputencoding{latin9}
\begin{lstlisting}
def monoidOnSemi[T: Semigroup]: Monoid[Option[T]] = Monoid(
  (l, r) => (l zip r).map { case (x, y) => x |+| y },
  None) *** verify code
\end{lstlisting}
\inputencoding{utf8}
The second example is found by considering the free monad encoding
\inputencoding{latin9}\lstinline!Free2!\inputencoding{utf8} (see
Section~\ref{subsec:Motivation-free-monad-different-encodings}).
The encoding \inputencoding{latin9}\lstinline!Free2!\inputencoding{utf8}
can be seen as \inputencoding{latin9}\lstinline!Free1!\inputencoding{utf8}
applied to the free functor \inputencoding{latin9}\lstinline!FF[F, A]!\inputencoding{utf8}.
If the effect constructor \inputencoding{latin9}\lstinline!F!\inputencoding{utf8}
is already functor, the free functor \inputencoding{latin9}\lstinline!FF[F, A]!\inputencoding{utf8}
is equivalent to just \inputencoding{latin9}\lstinline!F[A]!\inputencoding{utf8}.
So, the free monad on a functor \inputencoding{latin9}\lstinline!F!\inputencoding{utf8}
is the encoding \inputencoding{latin9}\lstinline!Free1!\inputencoding{utf8}.
It is a simpler data structure than \inputencoding{latin9}\lstinline!Free2!\inputencoding{utf8},
the free monad on an arbitrary type constructor \inputencoding{latin9}\lstinline!F!\inputencoding{utf8}.

It also follows that the free monad encodings \inputencoding{latin9}\lstinline!Free1!\inputencoding{utf8}
and \inputencoding{latin9}\lstinline!Free2!\inputencoding{utf8} are
equivalent when the effect constructor \inputencoding{latin9}\lstinline!F!\inputencoding{utf8}
is a functor.

The free functor construction can be viewed as a building block for
other free typeclasses. It is often simpler to construct a free typeclass
assuming that the effect constructor is already a functor. For type
constructors that are not functors, we can first apply the free functor
construction and then build the free typeclass based on a functor.
So, in the following sections we will restrict our attention to free
typeclasses on functors.

To gain some intuition about how to build typeclasses based on functors,
let us examine the definition of the free monad encoding \inputencoding{latin9}\lstinline!Free1[F, A]!\inputencoding{utf8}.
(That definition assumes that \inputencoding{latin9}\lstinline!F!\inputencoding{utf8}
is a functor.) A monad may be defined as a functor that additionally
has the \inputencoding{latin9}\lstinline!pure!\inputencoding{utf8}
and \inputencoding{latin9}\lstinline!flatten!\inputencoding{utf8}
methods satisfying suitable laws (the equivalence of \inputencoding{latin9}\lstinline!flatten!\inputencoding{utf8}
and \inputencoding{latin9}\lstinline!flatMap!\inputencoding{utf8}
was proved in Statement~\ref{subsec:Statement-flatten-equivalent-to-flatMap}).
The raw tree encoding for a monad\textsf{'}s definition via \inputencoding{latin9}\lstinline!pure!\inputencoding{utf8}
and \inputencoding{latin9}\lstinline!flatten!\inputencoding{utf8}
gives:\inputencoding{latin9}
\begin{lstlisting}
sealed trait Free5[F[_], A]
final case class Pure[F[_], A](a: A) extends Free5[F, A]
final case class Wrap[F[_], A](fa: F[A]) extends Free5[F, A]
final case class Flatten[F[_], A](ff: Free5[F, Free5[F, A]]) extends Free5[F, A]
\end{lstlisting}
\inputencoding{utf8}This code differs from \inputencoding{latin9}\lstinline!Free1!\inputencoding{utf8}
in two ways. First, \inputencoding{latin9}\lstinline!Free1!\inputencoding{utf8}
does not use a \inputencoding{latin9}\lstinline!Wrap!\inputencoding{utf8}
case class. Second, \inputencoding{latin9}\lstinline!Free5!\inputencoding{utf8}
has a recursive type definition of the form \inputencoding{latin9}\lstinline!Free5[F, Free5[F, A]]!\inputencoding{utf8}
that uses \inputencoding{latin9}\lstinline!Free5!\inputencoding{utf8}
twice, while \inputencoding{latin9}\lstinline!Free1!\inputencoding{utf8}
instead uses a simpler type: \inputencoding{latin9}\lstinline!F[Free1[F, A]]!\inputencoding{utf8}.
{*}{*}{*} We avoid the simplification Free5{[}F, F{[}A{]}{]} because
this creates a recursive definition of Free5{[}F, A{]} whose recursive
use modifies the type parameter A of Free5

\subsection{Free pointed functors}

\subsection{Free filterable functors}

\subsection{Free applicative functors}

\section{Advanced applications}

\subsection{Church encodings of free typeclasses}

\textsf{``}\textbf{Final} \textbf{Tagless} style\textsf{''} means \textsf{``}Church encoding
of free monad over $F^{\bullet}$\textsf{''}

Free monad over a functor $F^{\bullet}$ is $\text{FreeM}^{F^{\bullet},A}\triangleq A+F^{\text{FreeM}^{F^{\bullet},A}}$

Free monad $\text{FreeM}^{M^{\bullet},\bullet}$ over a monad $M^{\bullet}$
is not equivalent to $M^{\bullet}$

Free monad over a pointed functor $F^{\bullet}$ is {\footnotesize{}$\text{FreeM}^{F^{\bullet},A}\triangleq F^{A}+F^{\text{FreeM}^{F^{\bullet},A}}$}{\footnotesize\par}

start from half-reduced encoding $F^{A}+\exists Z.F^{Z}\times\big(Z\rightarrow\text{FreeM}^{F^{\bullet},A}\big)$ 

replace the existential type by an equivalent type $F^{\text{FreeM}^{F^{\bullet},A}}$

\paragraph{Another encoding: (to be studied in more detail):}

We have:
\[
\forall X^{:\text{MyTypeclass}}.\,(A\rightarrow X)\rightarrow X
\]
is the free \inputencoding{latin9}\lstinline!MyTypeclass!\inputencoding{utf8}
in the lawful Church encoding. The laws hold! This is more economical
than the raw tree encoding, see the \inputencoding{latin9}\lstinline!Semigroup!\inputencoding{utf8}
example.

\section{Laws of free constructions}

This chapter developed the free monad via the implementation of a
simple type-safe DSL. We found different encodings of the free monad
(the raw tree encoding and the various reduced encodings) that have
different performance trade-offs and satisfy different subsets of
the monad laws. Free constructions of other typeclasses are motivated
by analogy with the free monad and its various encodings. Can we formulate
any properties or laws that validate the correctness of those constructions?
Are all the different encodings equally safe to use? We will now develop
the necessary theory for answering these questions.

\subsection{Free constructions for $P$-typeclasses\label{subsec:Free-constructions-for-inductive-typeclasses}}

Some features are common to all the free typeclass constructions we
have seen. We begin by generalizing the features of the free monoid
construction to other similar typeclasses. We will then extend the
results to typeclasses for type constructors (such as \inputencoding{latin9}\lstinline!Functor!\inputencoding{utf8}
and \inputencoding{latin9}\lstinline!Monad!\inputencoding{utf8}).

The free monoid is a type constructor ($\text{FM}^{\bullet}$) that
transforms an arbitrary type $T$ into a new type ($\text{FM}^{T}$)
having a \inputencoding{latin9}\lstinline!Monoid!\inputencoding{utf8}
typeclass instance. Values of type $T$ can be wrapped into values
of type $\text{FM}^{T}$. For any given monoid $M$ and a given function
of type $T\rightarrow M$, a \textsf{``}free monoid program\textsf{''} (i.e., a value
of type $\text{FM}^{T}$) can be \textsf{``}run into $M$\textsf{''}. The resulting
runner (of type $\text{FM}^{T}\rightarrow M$) will preserve the monoid
operations between $\text{FM}^{T}$ and $M$. So, the monoid laws
will hold after running a \textsf{``}free monoid program\textsf{''}, even when the
chosen encoding $\text{FM}^{T}$ violates some of the monoid laws.

To generalize from monoids to other typeclasses, it helps to use the
notion of a \textsf{``}$P$-typeclass\textsf{''} introduced in Section~\ref{subsec:P-algebraic-typeclasses}.
For a given (covariant) functor $P$, a $P$\textbf{-typeclass} \index{$P$-typeclass}
has the evidence data equivalent to a value of type $P^{A}\rightarrow A$.
For the \inputencoding{latin9}\lstinline!Monoid!\inputencoding{utf8}
typeclass, the structure functor is $P^{A}\triangleq\bbnum 1+A\times A$,
and the evidence data has type $A\times\left(A\times A\rightarrow A\right)$,
which is equivalent to $P^{A}\rightarrow A$. The two parts of the
disjunctive type $\bbnum 1+A\times A$ correspond to the \emph{arguments}
of the monoid\textsf{'}s two operations: the empty value ($e_{A}$) and the
binary operation ($\oplus_{A}$). 

The next step is to generalize the property of \textsf{``}preserving the monoid\textsf{'}s
operations\textsf{''} to an arbitrary $P$-typeclass. Given two monoids $M$
and $N$, a function $f:M\rightarrow N$ that preserves the monoid\textsf{'}s
operations is a monoid morphism\index{monoid morphism} according
to Definition~\ref{subsec:Definition-monoid-morphism}. Can we describe
the laws in Definition~\ref{subsec:Definition-monoid-morphism} purely
in terms of the monoid\textsf{'}s structure functor $P^{A}=\bbnum 1+A\times A$?
As $M$ and $N$ are monoids, their typeclass instances must be available
as values $p_{M}$ and $p_{N}$: 
\begin{align*}
 & p_{M}:P^{M}\rightarrow M=\bbnum 1+M\times M\rightarrow M\quad,\quad\quad p_{M}\triangleq\,\begin{array}{|c||c|}
 & M\\
\hline \bbnum 1 & 1\rightarrow e_{M}\\
M\times M & a\times b\rightarrow a\oplus_{M}b
\end{array}\quad;\\
 & p_{N}:P^{N}\rightarrow N=\bbnum 1+N\times N\rightarrow N\quad,\quad\quad p_{N}\triangleq\,\begin{array}{|c||c|}
 & N\\
\hline \bbnum 1 & 1\rightarrow e_{N}\\
N\times N & c\times d\rightarrow c\oplus_{N}d
\end{array}\quad.
\end{align*}
So far, we have functions of types $P^{M}\rightarrow M$, $M\rightarrow N$,
and $P^{N}\rightarrow N$. It appears promising to arrange these functions
in a type diagram. The missing edge of the diagram is a function of
type $P^{M}\rightarrow P^{N}$.

\begin{wrapfigure}{l}{0.2\columnwidth}%
\vspace{-1.15\baselineskip}
\[
\xymatrix{\xyScaleY{1.4pc}\xyScaleX{3.0pc}P^{M}\ar[r]\sp(0.5){\ p_{M}}\ar[d]\sp(0.45){\,f^{\uparrow P}} & M\ar[d]\sp(0.45){\,f}\\
P^{N}\ar[r]\sp(0.5){~p_{N}} & N
}
\]
\vspace{-0.6\baselineskip}
\end{wrapfigure}%

\noindent Let us see what happens if we use $f^{\uparrow P}$ as that
function and require that the resulting diagram should commute:\vspace{-0.4\baselineskip}
\begin{equation}
p_{M}\bef f=f^{\uparrow P}\bef p_{N}\quad.\label{eq:p-algebra-morphism-law}
\end{equation}
We simplify both sides of Eq.~(\ref{eq:p-algebra-morphism-law})
by using the definition of $f^{\uparrow P}$:
\begin{align*}
 & f^{\uparrow P}=\,\begin{array}{|c||cc|}
 & \bbnum 1 & N\times N\\
\hline \bbnum 1 & \text{id} & \bbnum 0\\
M\times M & \bbnum 0 & a\times b\rightarrow f(a)\times f(b)
\end{array}\quad,\\
 & p_{M}\bef f=\,\begin{array}{||c|}
1\rightarrow e_{M}\\
a\times b\rightarrow a\oplus_{M}b
\end{array}\,\bef f=\,\begin{array}{||c|}
1\rightarrow f(e_{M})\\
a\times b\rightarrow f(a\oplus_{M}b)
\end{array}\quad,\\
 & f^{\uparrow P}\bef p_{N}=\,\begin{array}{||cc|}
\text{id} & \bbnum 0\\
\bbnum 0 & a\times b\rightarrow f(a)\times f(b)
\end{array}\,\bef\,\begin{array}{||c|}
1\rightarrow e_{N}\\
c\times d\rightarrow c\oplus_{N}d
\end{array}\,=\,\begin{array}{||c|}
1\rightarrow e_{N}\\
a\times b\rightarrow f(a)\oplus_{N}f(b)
\end{array}\quad.
\end{align*}
Then Eq.~(\ref{eq:p-algebra-morphism-law}) is rewritten as:
\[
\begin{array}{||c|}
1\rightarrow f(e_{M})\\
a\times b\rightarrow f(a\oplus_{M}b)
\end{array}\,\overset{!}{=}\,\begin{array}{||c|}
1\rightarrow e_{N}\\
a\times b\rightarrow f(a)\oplus_{N}f(b)
\end{array}\quad.
\]
This equation is the same as the identity and composition laws in
Definition~\ref{subsec:Definition-monoid-morphism}. It is now clear
how to define the property of \textsf{``}preserving the typeclass operations\textsf{''}
for an arbitrary $P$-typeclass: we just need to impose Eq.~(\ref{eq:p-algebra-morphism-law}).
In this way, we have reformulated the typeclass preservation property
in terms of the functor $P$.

At this point it is helpful to borrow some definitions from \index{category theory}category
theory. 

\subsubsection{Definition \label{subsec:Definition-f-algebra}\ref{subsec:Definition-f-algebra}}

Given a functor $F$, a type $M$ is called an $F$\textbf{-algebra}
if there exists a morphism $p_{M}:F^{M}\rightarrow M$. The type $M$
is the \textbf{carrier} and the morphism $p_{M}$ is the \index{structure map of $F$-algebra}\textbf{structure
map} of the $F$-algebra. \index{$F$-algebra!structure map} All $F$-algebras
form a category whose objects are pairs $\left(M,p_{M}\right)$ and
morphisms are defined as functions $f^{:M\rightarrow N}$ satisfying
Eq.~(\ref{eq:p-algebra-morphism-law}) with $P=F$. Such functions
$f$ are called $F$-\textbf{algebra morphisms}.\index{$F$-algebra!morphism}
$\square$

To show that the category laws hold for $F$-algebras, we use the
following properties:

\subsubsection{Statement \label{subsec:Statement-category-of-P-algebras}\ref{subsec:Statement-category-of-P-algebras}}

Assume that $K$, $L$, $M$ are some $F$-algebras with structure
maps $p_{K}$, $p_{L}$, $p_{M}$.

\textbf{(a)} The identity function $\text{id}^{:M\rightarrow M}$
is an $F$-algebra morphism.

\textbf{(b)} If $g^{:K\rightarrow L}$ and $h^{:L\rightarrow M}$
are $F$-algebra morphisms then so is the composition $g\bef h$.

\subparagraph{Proof}

\textbf{(a)} The law~(\ref{eq:p-algebra-morphism-law}) holds with
$P=F$, $M=N$, and $f=\text{id}$ since both sides of the law are
equal to $p_{M}$:
\[
p_{M}\bef f=p_{M}\bef\text{id}=p_{M}\quad,\quad\quad f^{\uparrow F}\bef p_{N}=\text{id}^{\uparrow F}\bef p_{M}=p_{M}\quad.
\]

\textbf{(b)} To verify that the law~(\ref{eq:p-algebra-morphism-law})
holds for $f\triangleq g\bef h$ with $P=F$, we write:
\begin{align*}
{\color{greenunder}\text{expect to equal }f^{\uparrow F}\bef p_{M}:}\quad & p_{K}\bef\gunderline f=\gunderline{p_{K}\bef g}\bef h\\
{\color{greenunder}F\text{-algebra morphism law for }g^{:K\rightarrow L}:}\quad & =g^{\uparrow F}\bef\gunderline{p_{L}\bef h}\\
{\color{greenunder}F\text{-algebra morphism law for }h^{:L\rightarrow M}:}\quad & =\gunderline{g^{\uparrow F}\bef h^{\uparrow F}}\bef p_{M}=(\gunderline{g\bef h})^{\uparrow F}\bef p_{M}=f^{\uparrow F}\bef p_{M}\quad.
\end{align*}
$\square$

We can now formulate our findings about $P$-typeclasses in the language
of $F$-algebras. Using the functor $P$ instead of $F$, we say that
a $P$\textbf{-typeclass with laws} is\index{$P$-typeclass!with laws}
a $P$-algebra whose structure map needs to satisfy given laws. The
structure map ($p_{M}:P^{M}\rightarrow M$) describes at once all
the typeclass methods and so plays the role of an \emph{evidence value}
showing that a type $M$ belongs to the $P$-typeclass. If $M$ and
$N$ are two $P$-algebras and a function $f^{:M\rightarrow N}$ is
a $P$-algebra morphism then we say that $f$ \textsf{``}preserves the $P$-typeclass
operations\textsf{''}. 

The next step is to generalize the free monoid construction to a \textsf{``}free
$P$-typeclass\textsf{''} construction. We have seen that there are several
versions, or \textsf{``}encodings\textsf{''}, of the free monoid. Different encodings
satisfy different subsets of the monoid laws. By analogy with the
free monoid, we list the expected properties of an \textsf{``}encoding\textsf{''}
of a free $P$-typeclass. We should have a type constructor $E^{T}$
that wraps an arbitrary type $T$ and produces a $P$-typeclass instance
automatically. Values of type $E^{T}$ represent \textsf{``}$P$-typeclass
programs\textsf{''}, that is, unevaluated expression trees with primitive
values of type $T$. A \textsf{``}runner\textsf{''} can evaluate those expression
trees into values of a specific type $C$ as long as $C$ already
belongs to the $P$-typeclass and a function of type $T\rightarrow C$
is given:
\[
\text{run}_{E}^{T,C}:(T\rightarrow C)\rightarrow E^{T}\rightarrow C\quad.
\]

It turns out that we may simplify the definition of the runner so
that we no longer need to use a function $r:T\rightarrow C$. Since
$E$ is a functor, we may apply the lifted function $r^{\uparrow E}:E^{T}\rightarrow E^{C}$.
It remains to find a transformation of type $E^{C}\rightarrow C$.
That transformation (as we will show) is equivalent to $\text{run}_{E}^{T,C}$.

With these goals in mind, we define a free $P$-typeclass encoding:

\subsubsection{Definition \label{subsec:Definition-free-P-typeclass-encoding}\ref{subsec:Definition-free-P-typeclass-encoding}}

Given a functor $P$, a \textbf{free} $P$\textbf{-typeclass} \textbf{encoding}
is a functor $E$ such that:

\textbf{(a)} For any type $T$, the type $E^{T}$ has a $P$-typeclass
instance $p_{E}^{T}:P^{E^{T}}\rightarrow E^{T}$ natural in $T$:
\begin{equation}
\forall f^{:T\rightarrow U}:\quad f^{\uparrow E\uparrow P}\bef p_{E}^{U}=p_{E}^{T}\bef f^{\uparrow E}\quad.\label{eq:free-typeclass-encoding-P-naturality-law}
\end{equation}
 As a consequence of Eq.~(\ref{eq:free-typeclass-encoding-P-naturality-law}),
for any $f^{:T\rightarrow U}$ the function $f^{\uparrow E}:E^{T}\rightarrow E^{U}$
is a $P$-algebra morphism. The function $p_{E}^{T}$ may obey a subset
of the laws of the $P$-typeclass; that subset must be the same for
all $T$.

\textbf{(b)} The functor $E$ is pointed\index{pointed functor}:
there exists a natural transformation $\text{pu}_{E}^{T}:T\rightarrow E^{T}$.

\textbf{(c)} There exists a \textsf{``}universal evaluator\textsf{''} function ($\text{eval}_{E}^{C}:E^{C}\rightarrow C$)
whose type parameter $C$ is restricted to $P$-algebras $C$ whose
structure map $p_{C}:P^{C}\rightarrow C$ obeys all those $P$-typeclass
laws that $E^{T}$ obeys. The evaluator function must satisfy:
\begin{align}
{\color{greenunder}\text{left identity law}:}\quad & \text{pu}_{E}^{C}\bef\text{eval}_{E}^{C}=\text{id}^{:C\rightarrow C}\quad,\label{eq:free-typeclass-encoding-left-identity-law}\\
{\color{greenunder}\text{right identity law}:}\quad & (\text{pu}_{E}^{T})^{\uparrow E}\bef\text{eval}_{E}^{E^{T}}=\text{id}^{:E^{T}\rightarrow E^{T}}\quad,\label{eq:free-typeclass-encoding-right-identity-law}\\
{\color{greenunder}P\text{-algebra morphism law}:}\quad & p_{E}^{C}\bef\text{eval}_{E}^{C}=\big(\text{eval}_{E}^{C}\big)^{\uparrow P}\bef p_{C}\quad,\nonumber \\
{\color{greenunder}P\text{-algebra naturality law}:}\quad & \text{for any }P\text{-algebra morphism }g^{:C\rightarrow D}:\quad\text{eval}_{E}^{C}\bef g=g^{\uparrow E}\bef\text{eval}_{E}^{D}\quad.\label{eq:free-typeclass-encoding-P-algebra-naturality-law}
\end{align}
The requirement that $C$ should obey \textsf{``}all\textsf{''} the laws of $E^{T}$
means that, for any type $T$, if $E^{T}$ obeys any $P$-typeclass
law (such as an identity law, an associativity law, etc.) then $C$
must also obey the same law. The type $C$ might obey \emph{more}
laws than $E^{T}$, but any law that holds for $E^{T}$ must also
hold for $C$. (Section~\ref{subsec:Describing-laws-of-P-typeclasses-as-values}
will describe $P$-typeclass laws in more detail, but we do not yet
need the the techniques developed there.)

\textbf{(d)} The function $\text{eval}_{E}^{C}$ has a uniqueness
property: there is only one $P$-algebra morphism $f:E^{C}\rightarrow C$
obeying the left identity law ($\text{pu}_{E}\bef f=\text{id}$),
namely $f=\text{eval}_{E}^{C}$. $\square$

The uniqueness property reflects a programmer\textsf{'}s expectation that there
is only one correct way of running (i.e., evaluating) a free $P$-typeclass
program while preserving the typeclass\textsf{'}s operations. The \textsf{``}universal
runner\textsf{''} function, $\text{run}_{E}^{T,C}$, can be defined via $\text{eval}_{E}^{C}$
as:
\[
\text{run}_{E}^{T,C}:(T\rightarrow C)\rightarrow E^{T}\rightarrow C\quad,\quad\quad\text{run}_{E}^{T,C}(r^{:T\rightarrow C})\triangleq r^{\uparrow E}\bef\text{eval}_{E}^{C}\quad.
\]
The uniqueness property of the runner is proved in the next statement.

\subsubsection{Statement \label{subsec:Statement-some-properties-of-free-P-typeclass-encoding}\ref{subsec:Statement-some-properties-of-free-P-typeclass-encoding}}

Suppose $E$ is a free $P$-typeclass encoding and $T$ is any type
(not necessarily of the $P$-typeclass).

\textbf{(a)} If $C$ is a $P$-algebra that obeys all the laws of
$E$ then any $P$-algebra morphism $g:E^{T}\rightarrow C$ can be
expressed as: 
\[
g=\text{run}_{E}^{T,C}(r)\triangleq r^{\uparrow E}\bef\text{eval}_{E}^{C}\quad\quad,\quad\text{where we defined}\quad r^{:T\rightarrow C}\triangleq\text{pu}_{E}^{T}\bef g\quad.
\]
This is the universal runner\textsf{'}s uniqueness property: for any chosen
function $r:T\rightarrow C$, there is only one $P$-algebra morphism
$g:E^{T}\rightarrow C$ satisfying $\text{pu}_{E}^{T}\bef g=r$, namely
$g=r^{\uparrow E}\bef\text{eval}_{E}^{C}$.

\textbf{(b)} If we assume that \textbf{(a)} holds then we can \emph{derive}
the right identity law~(\ref{eq:free-typeclass-encoding-right-identity-law}).

\subparagraph{Proof}

\textbf{(a)} Use the $P$-algebra naturality law~(\ref{eq:free-typeclass-encoding-P-algebra-naturality-law})
with $E^{T}$ and $C$ instead of $C$ and $D$:
\[
g^{\uparrow E}\bef\text{eval}_{E}^{C}=\text{eval}_{E}^{E^{A}}\bef g\quad.
\]
Now we can verify the uniqueness property:
\begin{align*}
{\color{greenunder}\text{expect to equal }g:}\quad & \text{run}_{E}^{T,C}(\text{pu}_{E}^{T}\bef g)=(\text{pu}_{E}^{T}\bef g)^{\uparrow E}\bef\text{eval}_{E}^{C}=(\text{pu}_{E}^{T})^{\uparrow E}\bef\gunderline{g^{\uparrow E}\bef\text{eval}_{E}^{C}}\\
{\color{greenunder}\text{use the }P\text{-algebra naturality law}:}\quad & =\gunderline{(\text{pu}_{E}^{T})^{\uparrow E}\bef\text{eval}_{E}^{E^{A}}}\bef g\\
{\color{greenunder}\text{use the right identity law (\ref{eq:free-typeclass-encoding-right-identity-law})}:}\quad & =\text{id}\bef g=g\quad.
\end{align*}

\textbf{(b)} Setting $C=E^{T}$ and $g=\text{id}$ in part \textbf{(a)},
we get:
\[
g\overset{!}{=}(\text{pu}_{E}^{T}\bef g)^{\uparrow E}\bef\text{eval}_{E}^{E^{T}}\quad.
\]
Substituting $g=\text{id}$, we obtain $\text{id}=(\text{pu}_{E}^{T})^{\uparrow E}\bef\text{eval}_{E}^{E^{T}}$,
which is the right identity law. $\square$

It is important that the definition of a free $P$-typeclass encoding
$E$ does not require that $E$ should obey \emph{all} of the $P$-typeclass
laws. For instance, we have seen that the raw tree encoding of free
monads, free monoids, and other typeclasses does not obey any of the
relevant laws. Because this does not lead to problems in practical
programming, we design our definition of free $P$-typeclasses to
admit encodings that obey only a subset of typeclass laws (possibly,
no laws at all).

The next step is to generalize the raw tree encoding from \inputencoding{latin9}\lstinline!Monoid!\inputencoding{utf8}
to an arbitrary $P$-typeclass and to show that it satisfies the definition
of a \textsf{``}free $P$-typeclass encoding\textsf{''}. Although the raw tree encoding
does not satisfy any typeclass laws, it gives us a valid and usable
encoding that we can use as a starting point to develop other, more
concise encodings for a free $P$-typeclass. 

The raw tree encoding of a free monoid (see Section~\ref{subsec:Free-monoids})
has the type: 
\[
\text{FMR}^{T}\triangleq\bbnum 1+T+\text{FMR}^{T}\times\text{FMR}^{T}\quad.
\]
The parts of the disjunctive type $\bbnum 1+T+\text{FMR}^{T}\times\text{FMR}^{T}$
correspond to three ways in which a free monoid value may be created:
from the empty value, from a value of type $T$, and from two existing
values of the free monoid type. The structure functor $P$ describes
two of these three ways. This suggests to rewrite the definition of
$\text{FMR}^{T}$ as $\text{FMR}^{T}\triangleq T+P^{\text{FMR}^{T}}$.
We can now generalize to arbitrary $P$-typeclasses and define the
raw tree encoding of a free $P$-typeclass (denoted by $\text{FPR}^{T}$)
as:
\[
\text{FPR}^{T}\triangleq T+P^{\text{FPR}^{T}}\quad.
\]
We notice that $\text{FPR}^{T}$ is the same as the free monad\index{free monad}
on $P$ (as defined in Statement~\ref{subsec:Statement-monad-construction-4-free-monad}).
The free monad on $P$ is a tree-like data structure whose branch
shape is described by the functor $P$. This data structure represents
an unevaluated expression tree built up from operations of type $P^{A}\rightarrow A$
and values of type $T$. 

\subsubsection{Statement \label{subsec:Statement-free-P-typeclass-raw-tree-encoding}\ref{subsec:Statement-free-P-typeclass-raw-tree-encoding}}

The free monad on $P$, denoted by $\text{FPR}^{T}\triangleq T+P^{\text{FPR}^{T}}$,
is a free $P$-typeclass encoding. The monad\textsf{'}s \inputencoding{latin9}\lstinline!flatten!\inputencoding{utf8}
method is the same as \inputencoding{latin9}\lstinline!eval!\inputencoding{utf8}
applied to an argument of type \inputencoding{latin9}\lstinline!FPR[FPR[T]]!\inputencoding{utf8}.

\subparagraph{Proof}

We need to verify the properties in Definition~\ref{subsec:Definition-free-P-typeclass-encoding}.

\textbf{(a)} To define $p_{\text{FPR}}^{T}$, we note the type equivalence
$T+P^{\text{FPR}^{T}}\cong\text{FPR}^{T}$ and rewrite the type signature
$p_{\text{FPR}}^{T}:P^{\text{FPR}^{T}}\rightarrow\text{FPR}^{T}$
equivalently as $p_{\text{FPR}}^{T}:P^{\text{FPR}^{T}}\rightarrow T+P^{\text{FPR}^{T}}$.
Then we define $p_{\text{FPR}}^{T}$ as:
\[
p_{\text{FPR}}^{T}:P^{\text{FPR}^{T}}\rightarrow T+P^{\text{FPR}^{T}}\quad,\quad\quad p_{\text{FPR}}(x)\triangleq\bbnum 0^{:T}+x^{:P^{\text{FPR}^{T}}}\quad.
\]
This code is fully parametric in $T$, so the naturality law holds:
for any $f:T\rightarrow U$, we have:
\[
p_{\text{FPR}}^{T}\bef f^{\uparrow\text{FPR}}=f^{\uparrow\text{FPR}\uparrow P}\bef p_{\text{FPR}}^{U}\quad.
\]
This equation is the same as the $P$-algebra morphism law for the
function $f^{\uparrow\text{FPR}}$.

\textbf{(b)} The function $\text{pu}_{\text{FPR}}^{T}:T\rightarrow\text{FPR}^{T}$
is defined as $\text{pu}_{\text{FPR}}^{T}(t^{:T})\triangleq t+\bbnum 0$.

\textbf{(c)} Given $p_{C}:P^{C}\rightarrow C$, the function $\text{eval}_{\text{FPR}}^{C}$
is defined recursively by:
\[
\text{eval}_{\text{FPR}}^{C}:\text{FPR}^{C}\rightarrow C\quad,\quad\quad\text{eval}_{\text{FPR}}^{C}\triangleq\,\begin{array}{|c||c|}
 & C\\
\hline C & \text{id}\\
P^{\text{FPR}^{C}} & \overline{\text{eval}_{\text{FPR}}^{C}}^{\uparrow P}\bef p_{C}
\end{array}\quad.
\]

To verify the left identity law~(\ref{eq:free-typeclass-encoding-left-identity-law}):
\[
\text{pu}_{\text{FPR}}^{C}\bef\text{eval}_{\text{FPR}}^{C}=\,\begin{array}{|c||cc|}
 & C & P^{\text{FPR}^{C}}\\
\hline C & \text{id} & \bbnum 0
\end{array}\,\bef\,\begin{array}{|c||c|}
 & C\\
\hline C & \text{id}\\
P^{\text{FPR}^{C}} & \overline{\text{eval}_{\text{FPR}}^{C}}^{\uparrow P}\bef p_{C}
\end{array}\,=\text{id}\quad.
\]

To verify the right identity law~(\ref{eq:free-typeclass-encoding-right-identity-law}),
we first show that the function $\text{eval}_{\text{FPR}}^{\text{FPR}^{T}}:\text{FPR}^{\text{FPR}^{T}}\rightarrow\text{FPR}^{T}$
is the same as the monad\textsf{'}s \inputencoding{latin9}\lstinline!flatten!\inputencoding{utf8}
method for \inputencoding{latin9}\lstinline!FPR!\inputencoding{utf8}
given by the following code matrix (see Statement~\ref{subsec:Statement-monad-construction-4-free-monad}):
\[
\text{ftn}_{\text{FPR}}=\,\begin{array}{|c||cc|}
 & T & P^{\text{FPR}^{T}}\\
\hline T & \text{id} & \bbnum 0\\
P^{\text{FPR}^{T}} & \bbnum 0 & \text{id}\\
P^{\text{FPR}^{\text{FPR}^{T}}} & \bbnum 0 & \overline{\text{ftn}}_{\text{FPR}}^{\uparrow P}
\end{array}\quad.
\]
We use the definition of \inputencoding{latin9}\lstinline!eval!\inputencoding{utf8}
given in \textbf{(a)} with the type $\text{FPR}^{T}$ instead of $T$:
\[
\text{eval}_{\text{FPR}}^{\text{FPR}^{T}}=\,\begin{array}{|c||c|}
 & \text{FPR}^{T}\\
\hline \text{FPR}^{T} & \text{id}\\
P^{\text{FPR}^{\text{FPR}^{T}}} & \overline{\text{eval}_{\text{FPR}}^{\text{FPR}^{T}}}^{\uparrow P}\bef p_{\text{FPR}}
\end{array}\,=\,\begin{array}{|c||cc|}
 & T & P^{\text{FPR}^{T}}\\
\hline T & \text{id} & \bbnum 0\\
P^{\text{FPR}^{T}} & \bbnum 0 & \text{id}\\
P^{\text{FPR}^{\text{FPR}^{T}}} & \bbnum 0 & \overline{\text{eval}_{\text{FPR}}^{\text{FPR}^{T}}}^{\uparrow P}
\end{array}\,=\text{ftn}_{\text{FPR}}\quad.
\]
Now the required property follows from the monad\textsf{'}s right identity
law:
\[
(\text{pu}_{\text{FPR}})^{\uparrow\text{FPR}}\bef\text{eval}_{\text{FPR}}^{\text{FPR}^{T}}=(\text{pu}_{\text{FPR}})^{\uparrow\text{FPR}}\bef\text{ftn}_{\text{FPR}}=\text{id}\quad.
\]

To verify the $P$-algebra morphism law~(\ref{eq:p-algebra-morphism-law})
for the function $\text{eval}_{\text{FPR}}^{C}$:
\[
p_{\text{FPR}}\bef\text{eval}_{\text{FPR}}^{C}=\,\begin{array}{|c||cc|}
 & C & P^{\text{FPR}^{C}}\\
\hline P^{\text{FPR}^{C}} & \bbnum 0 & \text{id}
\end{array}\,\bef\,\begin{array}{|c||c|}
 & C\\
\hline C & \text{id}\\
P^{\text{FPR}^{C}} & \overline{\text{eval}_{\text{FPR}}^{C}}^{\uparrow P}\bef p_{C}
\end{array}\,=\overline{\text{eval}_{\text{FPR}}^{C}}^{\uparrow P}\bef p_{C}\quad.
\]

To verify the $P$-algebra naturality law~(\ref{eq:free-typeclass-encoding-P-algebra-naturality-law}),
assume $g^{:C\rightarrow D}$ is a $P$-algebra morphism ($p_{C}\bef g=g^{\uparrow P}\bef p_{D}$)
and use the inductive assumption that $g^{\uparrow\text{FPR}}\bef\overline{\text{eval}_{\text{FPR}}^{D}}=\overline{\text{eval}_{\text{FPR}}^{C}}\bef g$
holds for recursive calls to $\text{eval}_{\text{FPR}}$:
\begin{align*}
{\color{greenunder}\text{left-hand side}:}\quad & g^{\uparrow\text{FPR}}\bef\text{eval}_{\text{FPR}}^{D}=\,\begin{array}{|c||cc|}
 & D & P^{\text{FPR}^{D}}\\
\hline C & g & \bbnum 0\\
P^{\text{FPR}^{C}} & \bbnum 0 & g^{\uparrow\text{FPR}\uparrow P}
\end{array}\,\bef\,\begin{array}{|c||c|}
 & D\\
\hline D & \text{id}\\
P^{\text{FPR}^{D}} & \overline{\text{eval}_{\text{FPR}}^{D}}^{\uparrow P}\bef p_{D}
\end{array}\\
{\color{greenunder}\text{matrix composition}:}\quad & \quad\quad=\,\begin{array}{|c||c|}
 & D\\
\hline C & g\\
P^{\text{FPR}^{C}} & \overline{g^{\uparrow\text{FPR}}\bef\text{eval}_{\text{FPR}}^{D}}^{\uparrow P}\bef p_{D}
\end{array}\\
{\color{greenunder}\text{inductive assumption}:}\quad & \quad\quad=\,\begin{array}{||c|}
g\\
\big(\overline{\text{eval}_{\text{FPR}}^{C}}\bef g\big)^{\uparrow P}\bef p_{D}
\end{array}\,=\,\begin{array}{||c|}
g\\
\overline{\text{eval}_{\text{FPR}}^{C}}^{\uparrow P}\bef g^{\uparrow P}\bef p_{D}
\end{array}\quad;\\
{\color{greenunder}\text{right-hand side}:}\quad & \text{eval}_{E}^{C}\bef g=\,\begin{array}{|c||c|}
 & C\\
\hline C & \text{id}\\
P^{\text{FPR}^{C}} & \overline{\text{eval}_{\text{FPR}}^{C}}^{\uparrow P}\bef p_{C}
\end{array}\,\bef g=\,\begin{array}{|c||c|}
 & D\\
\hline C & g\\
P^{\text{FPR}^{C}} & \overline{\text{eval}_{\text{FPR}}^{C}}^{\uparrow P}\bef p_{C}\bef g
\end{array}\quad.
\end{align*}
The two code matrices are now equal due to the assumed law $p_{C}\bef g=g^{\uparrow P}\bef p_{D}$.

\textbf{(d)} For a given type $C$ with $P$-typeclass instance $p_{C}:P^{C}\rightarrow C$,
suppose a function $f:\text{FPR}^{C}\rightarrow C$ satisfies the
left identity law and is a $P$-algebra morphism:
\[
\text{pu}_{\text{FPR}}^{C}\bef f=\text{id}\quad,\quad\quad p_{\text{FPR}}\bef f=f^{\uparrow P}\bef p_{C}\quad.
\]
We need to show that $f=\text{eval}_{\text{FPR}}^{C}$. Since we have
the function $\text{eval}_{\text{FPR}}^{C}$ in matrix form, it is
convenient to rewrite $f$ also in that form:
\[
f:C+P^{\text{FPR}^{C}}\rightarrow C\quad,\quad\quad f\triangleq\,\begin{array}{|c||c|}
 & C\\
\hline C & g\\
P^{\text{FPR}^{C}} & h
\end{array}\quad,
\]
where $g$ and $h$ are new arbitrary functions of suitable types.
The identity law ($\text{pu}_{\text{FPR}}\bef f=\text{id}$) then
gives simply $g=\text{id}$, while the $P$-algebra morphism law gives:
\[
p_{\text{FPR}}\bef f=h\overset{!}{=}f^{\uparrow P}\bef p_{C}\quad.
\]
It remains to show that the following code matrices are equal:
\[
\text{eval}_{\text{FPR}}^{C}=\,\begin{array}{|c||c|}
 & C\\
\hline C & \text{id}\\
P^{\text{FPR}^{C}} & \overline{\text{eval}_{\text{FPR}}^{C}}^{\uparrow P}\bef p_{C}
\end{array}\,\overset{?}{=}f=\,\begin{array}{|c||c|}
 & C\\
\hline C & r\\
P^{\text{FPR}^{C}} & f^{\uparrow P}\bef p_{C}
\end{array}\quad.
\]
By the inductive assumption, the recursive call $\overline{\text{eval}_{\text{FPR}}^{C}}$
already satisfies the equation we need to prove: $f=\overline{\text{eval}_{\text{FPR}}^{C}}$.
So, the code matrices for $f$ and for $\text{eval}_{\text{FPR}}^{C}$
are equal. $\square$ 

As we will now show, \inputencoding{latin9}\lstinline!FPR!\inputencoding{utf8}\textsf{'}s
\inputencoding{latin9}\lstinline!flatten!\inputencoding{utf8} function
is a $P$-algebra morphism; in other words, it preserves the $P$-typeclass
operations. So, it is not useful to apply the free $P$-typeclass
construction twice, as the result can be reduced to a single layer
of \inputencoding{latin9}\lstinline!FPR[T]!\inputencoding{utf8} while
preserving the operations.

\subsubsection{Statement \label{subsec:Statement-free-P-typeclass-monad}\ref{subsec:Statement-free-P-typeclass-monad}}

The free monad\textsf{'}s \inputencoding{latin9}\lstinline!flatten!\inputencoding{utf8}
function, $\text{ftn}_{\text{FPR}}:\text{FPR}^{\text{FPR}^{T}}\rightarrow\text{FPR}^{T}$,
is a $P$-algebra morphism.

\subparagraph{Proof}

By Statement~\ref{subsec:Statement-free-P-typeclass-raw-tree-encoding},
the \inputencoding{latin9}\lstinline!flatten!\inputencoding{utf8}
function can be expressed as $\text{eval}_{\text{FPR}}^{\text{FPR}^{T}}$.
By the same statement, $\text{eval}_{\text{FPR}}^{C}$ is always a
$P$-algebra morphism as long as $C$ is a $P$-algebra obeying all
the laws of $\text{FPR}^{T}$. So, we may use $C=\text{FPR}^{T}$
and obtain the result that $\text{ftn}_{\text{FPR}}$ is also a $P$-algebra
morphism.

It is not accidental that the raw tree encoding ($\text{FPR}^{T}$)
is a monad. It turns out that \emph{all} free $P$-typeclass encodings
are monads (although not necessarily free monads):

\subsubsection{Statement \label{subsec:Statement-free-typeclass-encoding-is-a-monad}\ref{subsec:Statement-free-typeclass-encoding-is-a-monad}}

Any free $P$-typeclass encoding $E$ satisfying Definition~\ref{subsec:Definition-free-P-typeclass-encoding}
is a monad. The monad $E$\textsf{'}s \inputencoding{latin9}\lstinline!flatten!\inputencoding{utf8}
function ($\text{ftn}_{E}$) is equal to the unique $P$-algebra morphism
$\text{eval}_{E}^{E^{A}}$ between $P$-algebras $E^{E^{A}}$ and
$E^{A}$. The $P$-algebra morphism law of $\text{ftn}_{E}$ is:

\begin{wrapfigure}{l}{0.2\columnwidth}%
\vspace{-2.15\baselineskip}
\[
\xymatrix{\xyScaleY{1.4pc}\xyScaleX{3.0pc}P^{E^{E^{A}}}\ar[r]\sp(0.5){\text{ftn}_{E}^{\uparrow P}}\ar[d]\sp(0.4){p_{E}^{E^{A}}} & P^{E^{A}}\ar[d]\sp(0.4){p_{E}^{A}}\\
E^{E^{A}}\ar[r]\sp(0.5){\text{ftn}_{E}} & E^{A}
}
\]
\vspace{-0.6\baselineskip}
\end{wrapfigure}%

\noindent ~\vspace{-1\baselineskip}
\begin{equation}
\text{ftn}_{E}^{\uparrow P}\bef p_{E}^{A}=p_{E}^{E^{A}}\bef\text{ftn}_{E}\quad.\label{eq:P-algebra-morphism-law-of-flatten}
\end{equation}


\subparagraph{Proof}

We need to implement the monad\textsf{'}s methods $\text{pu}_{E}$ and $\text{ftn}_{E}$
and show that the monad laws hold. The method $\text{pu}_{E}$ exists
by Definition~\ref{subsec:Definition-free-P-typeclass-encoding}(b).
The \inputencoding{latin9}\lstinline!flatten!\inputencoding{utf8}
method ($\text{ftn}_{E}$) is defined as:
\[
\text{ftn}_{E}:E^{E^{A}}\rightarrow E^{A}\quad,\quad\quad\text{ftn}_{E}\triangleq\text{eval}_{E}^{E^{A}}\quad.
\]
The code for $\text{ftn}_{E}$ uses the evaluator function $\text{eval}_{E}^{C}$
with $C=E^{A}$. This is justified because $E^{A}$ is a $P$-algebra.
The condition that $C$ should satisfy all the laws of $E^{A}$ holds
trivially, as since $C=E^{A}$.

By Definition~\ref{subsec:Definition-free-P-typeclass-encoding}(c),
we find that $\text{ftn}_{E}$ is a $P$-algebra morphism. 

The monad\textsf{'}s two identity laws follow from the identity laws in Definition~\ref{subsec:Definition-free-P-typeclass-encoding}(c):
\begin{align*}
 & \text{pu}_{E}\bef\text{ftn}_{E}=\text{pu}_{E}\bef\text{eval}_{E}^{E^{A}}=\text{id}\quad.\\
 & \text{pu}_{E}^{\uparrow E}\bef\text{ftn}_{E}=\text{pu}_{E}^{\uparrow E}\bef\text{eval}_{E}^{E^{A}}=\text{id}\quad.
\end{align*}

The remaining monad law is written as:
\begin{align}
{\color{greenunder}\text{associativity law of }E:}\quad & (\text{ftn}_{E}^{A})^{\uparrow E}\bef\text{ftn}_{E}^{A}\overset{?}{=}\text{ftn}_{E}^{E^{A}}\bef\text{ftn}_{E}^{A}\quad.\label{eq:associativity-law-of-E-derivation1}
\end{align}
Let us denote by $g$ and $h$ the two sides of Eq.~(\ref{eq:associativity-law-of-E-derivation1}):
\[
g\triangleq(\text{ftn}_{E}^{A})^{\uparrow E}\bef\text{ftn}_{E}^{A}=(\text{ftn}_{E}^{A})^{\uparrow E}\bef\text{eval}_{E}^{E^{A}}\quad,\quad\quad h\triangleq\text{ftn}_{E}^{E^{A}}\bef\text{ftn}_{E}^{A}=\text{eval}_{E}^{E^{E^{A}}}\bef\text{eval}_{E}^{E^{A}}\quad.
\]
 We notice that $g$ has the form $f^{\uparrow E}\bef\text{eval}_{E}^{E^{A}}$
with some function $f$. By Definition~\ref{subsec:Definition-free-P-typeclass-encoding}(a),
the function $f^{\uparrow E}$ is a $P$-algebra morphism. So, the
left-hand is a composition of two $P$-algebra morphisms. The right-hand
side is also such a composition. It follows that both $g$ and $h$
are $P$-algebra morphisms of type $E^{E^{E^{A}}}\rightarrow E^{A}$. 

It remains to show that $g=h$. For that, it is convenient to use
the uniqueness property in Statement~\ref{subsec:Statement-some-properties-of-free-P-typeclass-encoding}(a).
In order to be able to use that property, we need to show that there
exists a function $r:E^{E^{A}}\rightarrow E^{A}$ such that $\text{pu}_{E}\bef g=\text{pu}_{E}\bef h=r$.
It will then follow that both $g$ and $h$ are equal to $r^{\uparrow E}\bef\text{eval}_{E}^{E^{A}}$.
A suitable $r$ is just $r\triangleq\text{ftn}_{E}^{A}=\text{eval}_{E}^{E^{A}}$.
So, we write:
\begin{align*}
{\color{greenunder}\text{expect to equal }r:}\quad & \text{pu}_{E}\bef g=\gunderline{\text{pu}_{E}\bef(\text{ftn}_{E}^{A})^{\uparrow E}}\bef\text{eval}_{E}^{E^{A}}\\
{\color{greenunder}\text{naturality of }\text{pu}_{E}:}\quad & \quad\quad=\text{ftn}_{E}^{A}\bef\gunderline{\text{pu}_{E}\bef\text{eval}_{E}^{E^{A}}}\\
{\color{greenunder}\text{left identity law~(\ref{eq:free-typeclass-encoding-left-identity-law})}:}\quad & \quad\quad=\text{ftn}_{E}^{A}=r\quad;\\
{\color{greenunder}\text{expect to equal }r:}\quad & \text{pu}_{E}\bef h=\gunderline{\text{pu}_{E}\bef\text{eval}_{E}^{E^{A}}}\bef\text{eval}_{E}^{E^{A}}=\text{eval}_{E}^{E^{A}}=r\quad.
\end{align*}
$\square$

\subsection{Describing laws of $P$-typeclasses as values\label{subsec:Describing-laws-of-P-typeclasses-as-values}}

Usually, typeclasses impose some laws on their methods. We will now
develop a rigorous description of $P$-typeclass laws where all laws
are represented by values of a certain type.

For motivation, look at the laws for a monoid type $A$:
\begin{align*}
{\color{greenunder}\text{left identity law}:}\quad & \text{for all }x^{:A}\quad:\quad e_{_{A}}\oplus_{_{A}}x=x\quad,\\
{\color{greenunder}\text{right identity law}:}\quad & \text{for all }x^{:A}\quad:\quad x\oplus_{_{A}}e_{_{A}}=x\quad,\\
{\color{greenunder}\text{associativity law}:}\quad & \text{for all }x^{:A},y^{:A},z^{:A}\quad:\quad(x\oplus_{_{A}}y)\oplus_{_{A}}z=x\oplus_{_{A}}(y\oplus_{_{A}}z)\quad,\\
{\color{greenunder}\text{commutativity law}:}\quad & \text{for all }x^{:A},y^{:A}\quad:\quad x\oplus_{_{A}}y=y\oplus_{_{A}}x\quad.
\end{align*}
Each law is an equation between some expressions of the same type
$A$. These expressions are computed via the monoid\textsf{'}s operations using
arbitrary values ($x$, $y$, $z$), which are also of type $A$.
To generalize this situation, we write the monoid laws as equations
of the form $f_{1}(...)=f_{2}(...)$:
\[
\text{for all }x^{:A},y^{:A},z^{:A}\quad:\quad f_{1}(\oplus_{_{A}},e_{_{A}},x,y,z)=f_{2}(\oplus_{_{A}},e_{_{A}},x,y,z)\quad.
\]
For the associativity law, we need to choose the functions $f_{1}$
and $f_{2}$ as:
\begin{align*}
{\color{greenunder}\text{for the associativity law}:}\quad & f_{1}(\oplus,e,x,y,z)\triangleq(x\oplus y)\oplus z\quad,\quad\quad f_{2}(\oplus,e,x,y,z)\triangleq x\oplus(y\oplus z)\quad.
\end{align*}
The functions $f_{1}$ and $f_{2}$ for the identity laws do not depend
on $y$ and $z$, for instance:
\begin{align*}
{\color{greenunder}\text{for the left identity law}:}\quad & f_{1}(\oplus,e,x,y,z)\triangleq e\oplus x\quad,\quad\quad f_{2}(\oplus,e,x,y,z)\triangleq x\quad.
\end{align*}
The functions $f_{1}$ and $f_{2}$ for the commutativity law will
not depend on $z$.

The data in the pair $(\oplus_{_{A}},e_{_{A}})$ is a value of type
$P^{A}\rightarrow A$, where $P^{A}\triangleq\bbnum 1+A\times A$
for the \inputencoding{latin9}\lstinline!Monoid!\inputencoding{utf8}
typeclass (but $P$ will be different for other typeclasses). So,
the functions $f_{1}$ and $f_{2}$ have type $(P^{A}\rightarrow A)\times A\times A\times A\rightarrow A$.
The pair $(f_{1},f_{2})$ is equivalent to a single value of type
$(P^{A}\rightarrow A)\times A\times A\times A\rightarrow A\times A$.
Each law corresponds to a specific value of that type, which we will
call a \index{law function}\textbf{law function}. A law function
computes the two sides of the law at once as a single value of type
$A\times A$. The law then requires that the two values in that pair
should be equal. If $p:A\times A$ is a pair of values of type $A$
then we will write the condition: 
\[
p\triangleright\pi_{1}=p\triangleright\pi_{2}
\]
to indicate that the two values in the pair are equal.

The monoid laws use up to three arbitrary values of type $A$, but
laws for other typeclasses might require more than three arbitrary
values. To make the definition of a law function more general, we
replace an arbitrary value of type $A\times A\times A$ by an arbitrary
function $f$ of type $\text{Int}\rightarrow A$. Instead of values
$x$, $y$, $z$ we will then use the values $f(1)$, $f(2)$, $f(3)$,
etc., which are arbitrary values of type $A$ since $f$ is an arbitrary
function. This allows us to generalize the definition of the \inputencoding{latin9}\lstinline!Monoid!\inputencoding{utf8}
laws to other typeclasses of similar form:

\subsubsection{Definition \label{subsec:Definition-law-function-P-typeclass}\ref{subsec:Definition-law-function-P-typeclass}}

A \textbf{law function} for a $P$-typeclass \index{$P$-typeclass!law function}
\index{law function of a $P$-typeclass} is a value $l$ of type:
\begin{equation}
\text{LawF}_{P}\triangleq\forall A.\,(P^{A}\rightarrow A)\times(\text{Int}\rightarrow A)\rightarrow A\times A\quad.\label{eq:P-typeclass-law-type}
\end{equation}
Given a function $l$ of type \inputencoding{latin9}\lstinline!LawF!\inputencoding{utf8},
we say that an evidence value $p_{T}:P^{T}\rightarrow T$ \textbf{satisfies
the $l$-law} if:
\begin{align}
{\color{greenunder}p_{T}\text{ satisfies the }l\text{-law}:}\quad & \text{for all }f^{:\text{Int}\rightarrow T}\quad:\quad\big(l^{T}(p_{T},f)\big)\triangleright\pi_{1}=\big(l^{T}(p_{T},f)\big)\triangleright\pi_{2}\quad.\label{eq:P-typeclass-law}
\end{align}
Here $l^{T}$ is the law function $l$ used with the type parameter
$T$.

A $P$\textbf{-typeclass with laws} \index{$P$-typeclass!with laws}has
specific chosen law functions $l_{1}$, $l_{2}$, ..., $l_{k}$ and
imposes each of the $l_{i}$-laws (for $i=1$, $2$, ..., $k$) on
the evidence values. A type $T$ belongs to that $P$-typeclass if
there exists an evidence value $p_{T}:P^{T}\rightarrow T$ that satisfies
the $l_{i}$-law~(\ref{eq:P-typeclass-law}) for each of the specified
law functions $l_{i}$. For brevity, we also say that the $l_{i}$-laws
\textsf{``}\textbf{hold for} $T$\textsf{''} as long as it is clear how what evidence
value $p_{T}$ is implied. $\square$

Another approach to typeclass laws is by describing the two sides
of a law via expression trees. The raw tree encoding of a free $P$-typeclass
($\text{FPR}^{T}$) is a type that represents arbitrary expression
trees with leaf values of type $T$. We can use a pair of values of
type $\text{FPR}^{T}$ as two sides of the law. The two sides can
be evaluated as expression trees using the $P$-operations of a specific
type $T$. The result will be two values of type $T$; the law holds
if these two values are equal.

To see in detail how this approach works, consider again the \inputencoding{latin9}\lstinline!Monoid!\inputencoding{utf8}
typeclass with its three laws. Suppose the type $T$ already has the
monoid methods ($\oplus_{_{T}}$ and $e_{_{T}}$), and we would like
to check whether the monoid laws hold for $T$. We will use the raw
tree encoding of the free monoid on $T$, that is, the type \inputencoding{latin9}\lstinline!FMR[T]!\inputencoding{utf8}
defined in Section~\ref{subsec:Free-monoids}. Begin with the left
identity law of monoids: it says that for any value $t^{:T}$ we must
have $e_{_{T}}\oplus_{_{T}}t=t$. The two sides of that law are two
expressions involving the monoid $T$\textsf{'}s operations as well as an arbitrary
value $t$ of type $T$. We first consider those two expressions as
unevaluated expression trees. Those \textsf{``}law expression trees\textsf{''} are
two values of the type \inputencoding{latin9}\lstinline!FMR[T]!\inputencoding{utf8}
that we may implement in Scala code like this:\inputencoding{latin9}
\begin{lstlisting}
def lhs1[T](t: T): FMR[T] = Combine(Empty(), Wrap(t))
def rhs1[T](t: T): FMR[T] = Wrap(t)
\end{lstlisting}
\inputencoding{utf8}The two sides of the law must be defined as \emph{functions} of type
$T\rightarrow\text{FMR}^{T}$ because the law should hold for arbitrary
values $t^{:T}$. In order to verify that the law holds, we now need
to apply the \inputencoding{latin9}\lstinline!runner!\inputencoding{utf8}
function to the values \inputencoding{latin9}\lstinline!lhs1!\inputencoding{utf8}
and \inputencoding{latin9}\lstinline!rhs1!\inputencoding{utf8}. To
evaluate the expression trees of type $\text{FMR}^{T}$ into values
of type $T$, we specify an identity function (of type $T\rightarrow T$)
as the first argument of the \inputencoding{latin9}\lstinline!runner!\inputencoding{utf8}
function:\inputencoding{latin9}
\begin{lstlisting}
def lhs1Evaluated[T: Monoid](t: T): T = runner[T, T](identity)(lhs1(t))
def rhs1Evaluated[T: Monoid](t: T): T = runner[T, T](identity)(rhs1(t))
\end{lstlisting}
\inputencoding{utf8}The law says that both sides should be equal as values of type $T$.
We could test that:\inputencoding{latin9}
\begin{lstlisting}
forAll { t: T =>               // Using the scalacheck library.
  lhs1Evaluated(t) shouldEqual rhs1Evaluated(t)
}
\end{lstlisting}
\inputencoding{utf8}
The right identity law is treated similarly. Turning now to the associativity
law, we find that we need \emph{three} arbitrary values of type $T$,
so the \textsf{``}law expression trees\textsf{''} are:\inputencoding{latin9}
\begin{lstlisting}
def lhs3[T](t1: T, t2: T, t3: T): FMR[T] = Combine(Combine(Wrap(t1), Wrap(t2)), Wrap(t3))
def rhs3[T](t1: T, t2: T, t3: T): FMR[T] = Combine(Wrap(t1), Combine(Wrap(t2), Wrap(t3)))
\end{lstlisting}
\inputencoding{utf8}To verify that the law holds for $T$, we again evaluate both expression
trees into values of type $T$ using the \inputencoding{latin9}\lstinline!runner!\inputencoding{utf8}
function:\inputencoding{latin9}
\begin{lstlisting}
def lhs3Evaluated[T: Monoid](t1: T, t2: T, t3: T): T = runner[T, T](identity)(lhs3(t1, t2, t3))
def rhs3Evaluated[T: Monoid](t1: T, t2: T, t3: T): T = runner[T, T](identity)(rhs3(t1, t2, t3))
\end{lstlisting}
\inputencoding{utf8}To test that the law holds:\inputencoding{latin9}
\begin{lstlisting}
forAll { (t1: T, t2: T, t3: T) =>               // Using the scalacheck library.
  lhs3Evaluated(t1, t2, t3) shouldEqual rhs3Evaluated(t1, t2, t3)
}
\end{lstlisting}
\inputencoding{utf8}
We can now generalize the type of \textsf{``}law expression trees\textsf{''} from
\inputencoding{latin9}\lstinline!Monoid!\inputencoding{utf8} to an
arbitrary $P$-typeclass. The two sides of the law are functions that
take as arguments one or more arbitrary values of type $T$ and will
return a value of type $\text{FPR}^{T}$. In order to be able to describe
any number of arbitrary values of type $T$, we use an arbitrary function
of type $\text{Int}\rightarrow T$. So, the law is expressed by a
function of type $(\text{Int}\rightarrow T)\rightarrow\text{FPR}^{T}\times\text{FPR}^{T}$.
Since the law is supposed to work in the same way for all types $T$,
we add a universal quantifier ($\forall T$). The type of \textsf{``}pairs
of law expression trees\textsf{''} (\inputencoding{latin9}\lstinline!LawET!\inputencoding{utf8})
becomes:
\begin{equation}
\text{LawET}_{P}\triangleq\forall T.\,(\text{Int}\rightarrow T)\rightarrow\text{FPR}^{T}\times\text{FPR}^{T}\quad.\label{eq:law-expression-tree-type}
\end{equation}

How does a given value \inputencoding{latin9}\lstinline!et: LawET!\inputencoding{utf8}
specify a law of a $P$-typeclass (the \textsf{``}\inputencoding{latin9}\lstinline!et!\inputencoding{utf8}-law\textsf{''})?
Suppose we need to verify whether a type $T$ with a $P$-algebra
structure map $p_{T}$ obeys the \inputencoding{latin9}\lstinline!et!\inputencoding{utf8}-law.
First, we apply \inputencoding{latin9}\lstinline!et!\inputencoding{utf8}
to an arbitrary function \inputencoding{latin9}\lstinline!f: Int => T!\inputencoding{utf8}
and obtain a pair \inputencoding{latin9}\lstinline!(lhs, rhs)!\inputencoding{utf8}
of values of type \inputencoding{latin9}\lstinline!FPR[T]!\inputencoding{utf8}:\inputencoding{latin9}
\begin{lstlisting}
val f: Int => T = ...
val (lhs: FPR[T], rhs: FPR[T]) = et(f)
\end{lstlisting}
\inputencoding{utf8}Then we evaluate both expression trees using $T$\textsf{'}s operations (that
is, data from the structure map $p_{T}$):\inputencoding{latin9}
\begin{lstlisting}
val (lhsT: T, rhsT: T) = (runner(id)(lhs), runner(id)(rhs))
\end{lstlisting}
\inputencoding{utf8}The \inputencoding{latin9}\lstinline!et!\inputencoding{utf8}-law
is then the condition \inputencoding{latin9}\lstinline!lhsT == rhsT!\inputencoding{utf8}.

The types \inputencoding{latin9}\lstinline!LawF!\inputencoding{utf8}
and \inputencoding{latin9}\lstinline!LawET!\inputencoding{utf8} appear
to be different but turn out to be \emph{equivalent}:

\subsubsection{Statement \label{subsec:Statement-equivalence-law-function-law-expression-tree-P-typeclass}\ref{subsec:Statement-equivalence-law-function-law-expression-tree-P-typeclass}}

The types of fully parametric functions \inputencoding{latin9}\lstinline!LawF!\inputencoding{utf8}
and \inputencoding{latin9}\lstinline!LawET!\inputencoding{utf8} (defined
by Eq.~(\ref{eq:P-typeclass-law-type}) and Eq.~(\ref{eq:law-expression-tree-type})
respectively) are equivalent to each other and to the type $\text{FPR}^{\text{Int}}\times\text{FPR}^{\text{Int}}$.

\subparagraph{Proof}

To verify the first type equivalence ($\text{LawF}\cong\text{FPR}^{\text{Int}}\times\text{FPR}^{\text{Int}}$):
\begin{align*}
 & \forall A.\,(P^{A}\rightarrow A)\gunderline{\times}\,(\text{Int}\rightarrow A)\rightarrow A\times A\\
 & \cong\forall A.\,(P^{A}+\text{Int}\rightarrow A)\rightarrow\gunderline{A\times A}\\
 & \cong\forall A.\,(P^{A}+\text{Int}\rightarrow A)\rightarrow\bbnum 2\rightarrow A\\
 & \cong\bbnum 2\rightarrow\forall A.\,(P^{A}+\text{Int}\rightarrow A)\rightarrow A\quad.
\end{align*}
The type $\forall A.\,(P^{A}+\text{Int}\rightarrow A)\rightarrow A$
is a Church encoding (see Section~\ref{subsec:The-Church-encoding-of-recursive-types})
of the recursive type $U$ defined as the least fixpoint of the type
equation $U\cong P^{U}+\text{Int}$. The least fixpoint $U$ is the
same as the type we denote by $\text{FPR}^{\text{Int}}$ according
to the definition of $\text{FPR}^{T}$.

To verify the second type equivalence ($\text{LawET}\cong\text{FPR}^{\text{Int}}\times\text{FPR}^{\text{Int}}$),
use the Yoneda lemma:
\[
\forall T.\,(\gunderline{\text{Int}}\rightarrow T)\rightarrow\gunderline{\text{FPR}^{T}\times\text{FPR}^{T}}\cong\text{FPR}^{\text{Int}}\times\text{FPR}^{\text{Int}}\quad.
\]
\vspace{-1\baselineskip}

The types \inputencoding{latin9}\lstinline!LawF!\inputencoding{utf8}
and \inputencoding{latin9}\lstinline!LawET!\inputencoding{utf8} are
equivalent because they are both equivalent to the same type. $\square$

Since the types \inputencoding{latin9}\lstinline!LawF!\inputencoding{utf8},
\inputencoding{latin9}\lstinline!LawET!\inputencoding{utf8}, and
$\text{FPR}^{\text{Int}}\times\text{FPR}^{\text{Int}}$ are equivalent,
we may use either of them according to convenience within a particular
derivation or proof. As an illustration, let us see how a value of
type $\text{FPR}^{\text{Int}}\times\text{FPR}^{\text{Int}}$ gives
rise to a $P$-typeclass law.

Denote for brevity $\text{LawE}^{T}\triangleq\text{FPR}^{T}\times\text{FPR}^{T}$.
Given a value $e:\text{LawE}^{\text{Int}}$ and a $P$-algebra $M$
with a structure map $p_{M}:P^{M}\rightarrow M$, we take an arbitrary
function $f:\text{Int}\rightarrow M$ and compute a value $c$ of
type $M\times M$:
\[
c^{:M\times M}\triangleq e\triangleright f^{\uparrow\text{LawE}}\triangleright\big(\text{run}_{\text{FPR}^{M}}^{M,M}(\text{id})\boxtimes\text{run}_{\text{FPR}^{M}}^{M,M}(\text{id})\big)\quad.
\]
Now, the $e$-law consists of the requirement that $c$ must be a
pair of equal values, for any $f$. In this way, a law expression
$e$ of type $\text{FPR}^{\text{Int}}\times\text{FPR}^{\text{Int}}$
is a pair of unevaluated expression trees where arbitrary values of
type $M$ are labeled by integers (the same integers must correspond
to the same values of type $M$). When evaluated to a final value
of type $M$, the two expression trees must give the same value. The
$e$-law holds if this is true for arbitrary choices of values of
type $M$.

As an explicit example of a law expression, consider the associativity
law of the \inputencoding{latin9}\lstinline!Monoid!\inputencoding{utf8}
typeclass :

{*}{*}{*} add an explicit code example with monoid laws{*}{*}{*}

It turns out that laws of a $P$-typeclass are closely connected with
$P$-algebra morphisms. In particular, $P$-algebra morphisms preserve
the typeclass laws and hide the law violations, as the following statements
shows.

\subsubsection{Statement \label{subsec:Statement-P-algebra-morphisms-and-laws}\ref{subsec:Statement-P-algebra-morphisms-and-laws}}

For any $P$-algebras $M$ and $N$ with structure maps $p_{M}$ and
$p_{N}$; for any $P$-algebra morphism $\phi:M\rightarrow N$; and
for any fully parametric law function $l:\text{LawF}_{P}$ that defines
an $l$-law:

\textbf{(a)} Suppose that the $l$-law holds for $M$ and $\phi$
is \emph{surjective}. Then the $l$-law also holds for $N$. (In this
sense, a $P$-algebra morphism \textsf{``}preserves\textsf{''} the $P$-typeclass
laws.)

\textbf{(b)} Suppose the $l$-law holds for $N$ but \emph{not} for
$M$. The $l$-law for $M$ has the form of Eq.~(\ref{eq:P-typeclass-law})
with $T=M$. Suppose a violation of the $l$-law for $M$ is found
using a specific $f:\text{Int}\rightarrow M$ as:
\begin{equation}
\big(l^{M}(p_{M},f)\big)\triangleright\pi_{1}\neq\big(l^{M}(p_{M},f)\big)\triangleright\pi_{2}\quad.\label{eq:p-typeclass-law-violation}
\end{equation}
Then this law violation will disappear after applying $\phi$:
\begin{equation}
\big(l^{M}(p_{M},f)\big)\triangleright\pi_{1}\bef\phi=\big(l^{M}(p_{M},f)\big)\triangleright\pi_{2}\bef\phi\quad.\label{eq:p-typeclass-law-violation-hidden}
\end{equation}
(In this sense, a $P$-algebra morphism \textsf{``}hides\textsf{''} any violations
of the $P$-typeclass laws.)

\subparagraph{Proof}

First, we prove that any fully parametric law function $l$ obeys
the strong dinaturality law (Section~\ref{subsec:Strong-dinaturality.-General-properties}).
By Statement~\ref{subsec:Statement-post-wedge-entails-strong-dinaturality},
the function $l$ is strongly dinatural if the argument type of $l$
can be expressed through a profunctor with the post-wedge property.
The type \inputencoding{latin9}\lstinline!LawF!\inputencoding{utf8}
is:
\[
\text{LawF}_{P}=\forall A.\,(P^{A}\rightarrow A)\times(\text{Int}\rightarrow A)\rightarrow A\times A=\forall A.\,R^{A,A}\rightarrow S^{A}\quad,
\]
where we defined the profunctor $R^{X,Y}\triangleq(P^{X}\rightarrow Y)\times(\text{Int}\rightarrow Y)$
and the functor $S^{A}\triangleq A\times A$. The argument type of
$l$ is $R^{A,A}$, and indeed we find, via Statement~~\ref{subsec:Statement-post-wedge}
and Example~\ref{subsec:Example-strong-dinaturality-for-some-type-signatures}(b),
that $R$ has the post-wedge property. To write the strong dinaturality
law of $l$, we use Eq.~(\ref{eq:strong-dinaturality-law}) with
types $A=M$, $B=N$, and $t=l$, $f=\phi$. For all $x^{:P^{M}\rightarrow M}$,
$u^{:\text{Int}\rightarrow M}$, $y^{:P^{N}\rightarrow N}$, $v^{:\text{Int}\rightarrow N}$
we have:
\begin{equation}
\text{when}\quad x\bef\phi=\phi^{\uparrow P}\bef y\quad\text{and}\quad u\bef\phi=v\quad\text{then}\quad l^{M}(x,u)\triangleright\phi^{\uparrow S}=l^{N}(y,v)\quad.\label{eq:strong-dinaturality-of-l-general}
\end{equation}
When we write the equation for the $l$-law, we will apply the function
$l$ as, say, $l^{M}(p_{M},u)$. Indeed, we may use $x=p_{M}$ and
$y=p_{N}$ in Eq.~(\ref{eq:strong-dinaturality-of-l-general}) because
the condition $x\bef\phi=\phi^{\uparrow P}\bef y$ is the same as
the $P$-algebra morphism law~(\ref{eq:p-algebra-morphism-law})
of $\phi$, which is assumed to hold. Then Eq.~(\ref{eq:strong-dinaturality-of-l-general})
is simplified to:
\begin{equation}
\text{for all }u^{:\text{Int}\rightarrow M}:\quad l^{M}(p_{M},u)\triangleright\phi^{\uparrow S}=l^{N}(p_{N},u\bef\phi)\quad.\label{eq:strong-dinaturality-of-typeclass-law-function-l}
\end{equation}

\textbf{(a)} The two sides of the $l$-law for any $u^{:\text{Int}\rightarrow M}$
are contained in the value $l^{M}(p_{M},u)$ of type $M\times M$.
That value is a pair of \emph{equal} values of type $M$ because it
is given that the $l$-law holds for $M$:
\[
l^{M}(p_{M},u)\triangleright\pi_{1}=l^{M}(p_{M},u)\triangleright\pi_{2}.
\]
Applying $\phi^{\uparrow S}$ to $l^{M}(p_{M},u)$, we obtain again
a pair of equal numbers. To see this more formally, we use the naturality
laws of $\pi_{1}$ and $\pi_{2}$:
\[
\phi^{\uparrow S}\bef\pi_{1}=\pi_{1}\bef\phi\quad,\quad\quad\phi^{\uparrow S}\bef\pi_{2}=\pi_{2}\bef\phi\quad,
\]
to find:
\[
l^{M}(p_{M},u)\triangleright\phi^{\uparrow S}\triangleright\pi_{1}=l^{M}(p_{M},u)\triangleright\pi_{1}\triangleright\phi=l^{M}(p_{M},u)\triangleright\pi_{2}\triangleright\phi=l^{M}(p_{M},u)\triangleright\phi^{\uparrow S}\triangleright\pi_{2}\quad.
\]
Using Eq.~(\ref{eq:strong-dinaturality-of-typeclass-law-function-l}),
we obtain:
\[
l^{N}(p_{N},u\bef\phi)\triangleright\pi_{1}=l^{N}(p_{N},u\bef\phi)\triangleright\pi_{2}\quad.
\]
This is not yet enough to prove that the $l$-law holds for $N$.
We need to prove that $l^{N}(p_{N},v)$ is a pair of equal values
for \emph{arbitrary} $v^{:\text{Int}\rightarrow N}$, but so far we
have only proved it for $v$ of the form $u\bef\phi$. So, it remains
to show that for an arbitrary $v^{:\text{Int}\rightarrow N}$ there
exists some $u_{v}^{:\text{Int}\rightarrow M}$ such that $v=u_{v}\bef\phi$.
By assumption, $\phi$ is surjective, which means that there exists
a function $\chi:N\rightarrow M$ such that $\chi\bef\phi=\text{id}$.
Then we define $u_{v}\triangleq\chi(v)$ and obtain $u_{v}\bef\phi=v\triangleright\chi\bef\phi=v$
as required. This definition of $u_{v}$ gives: 
\[
l^{N}(p_{N},v)=l^{N}(p_{N},u_{v}\bef\phi)\quad,\quad\text{and so}:\quad l^{N}(p_{N},v)\triangleright\pi_{1}=l^{N}(p_{N},v)\triangleright\pi_{2}\quad.
\]
 This holds for all $v^{:\text{Int}\rightarrow N}$, which shows that
the $l$-law holds for the type $N$.

\textbf{(b)} The $l$-law violation is witnessed by the inequality~(\ref{eq:p-typeclass-law-violation}).
To show that Eq.~(\ref{eq:p-typeclass-law-violation-hidden}) holds,
we write:
\begin{align*}
{\color{greenunder}\text{expect to equal }\big(l^{M}(p_{M},f)\big)\triangleright\pi_{2}\bef\phi:}\quad & \big(l^{M}(p_{M},f)\big)\triangleright\gunderline{\pi_{1}\bef\phi}\\
{\color{greenunder}\text{naturality law of }\pi_{1}:}\quad & =\gunderline{\big(l^{M}(p_{M},f)\big)\triangleright\phi^{\uparrow S}}\bef\pi_{1}\\
{\color{greenunder}\text{use Eq.~(\ref{eq:strong-dinaturality-of-typeclass-law-function-l})}:}\quad & =l^{N}(p_{N},f\bef\phi)\bef\gunderline{\pi_{1}}\\
{\color{greenunder}\text{the }l\text{-law holds for }N:}\quad & =l^{N}(p_{N},f\bef\phi)\bef\pi_{2}\\
{\color{greenunder}\text{use Eq.~(\ref{eq:strong-dinaturality-of-typeclass-law-function-l})}:}\quad & =\big(l^{M}(p_{M},f)\big)\triangleright\gunderline{\phi^{\uparrow S}\bef\pi_{2}}\\
{\color{greenunder}\text{naturality law of }\pi_{2}:}\quad & =\big(l^{M}(p_{M},f)\big)\triangleright\pi_{2}\bef\phi\quad.
\end{align*}
$\square$

The property of \textsf{``}hiding\textsf{''} the law violations justifies the practical
use of free typeclass encodings (e.g., the raw tree encoding) that
do not satisfy laws. In practice, a free monad program will be interpreted
(or \textsf{``}run\textsf{''}) into a lawful monad such as \inputencoding{latin9}\lstinline!Try!\inputencoding{utf8},
and the runner will preserve the monad operations. Statement~\ref{subsec:Statement-P-algebra-morphisms-and-laws}(b)
then shows that no law violations will be observable after the runner
is applied. We have proved this for our free monad DSL in Section~\ref{subsec:A-first-recipe-monadic-dsl}.
Now we see that this is a general property that applies to all $P$-typeclasses.
It is safe to use a free typeclass encoding (even if it violates some
laws) as long as its runner preserves the typeclass operations and
the target typeclass is lawful. In the next section we will study
such \textsf{``}partially lawful\textsf{''} free typeclass encodings.

\subsection{Free $P$-typeclasses that satisfy a subset of the laws\label{subsec:Free--typeclasses-that-satisfy-laws}}

In Section~\ref{subsec:Free-monoids}, we have seen four different
encodings of the free monoid (denoted by \inputencoding{latin9}\lstinline!F1!\inputencoding{utf8},
\inputencoding{latin9}\lstinline!F2!\inputencoding{utf8}, \inputencoding{latin9}\lstinline!F3!\inputencoding{utf8},
and \inputencoding{latin9}\lstinline!F4!\inputencoding{utf8}). It
turns out that all those encodings satisfy the requirements of Definition~\ref{subsec:Definition-free-P-typeclass-encoding},
although they obey different subsets of the monoid laws. We will now
generalize that situation to $P$-typeclasses and study the properties
of free $P$-typeclass constructions that satisfy only a subset of
the typeclass\textsf{'}s laws. The resulting theory will make it simpler to
prove that a given type constructor $E^{\bullet}$ is indeed a valid
free $P$-typeclass encoding as specified by Definition~\ref{subsec:Definition-free-P-typeclass-encoding}.

Namely, the next statement shows that a functor $E$ is a free $P$-typeclass
encoding if it is a strict subset of the raw tree encoding (\inputencoding{latin9}\lstinline!FPR!\inputencoding{utf8})
and has certain additional properties. In this way, we can use \inputencoding{latin9}\lstinline!FPR!\inputencoding{utf8}
to validate other free $P$-typeclass encodings with less work than
if we were applying Definition~\ref{subsec:Definition-free-P-typeclass-encoding}
directly.

\subsubsection{Statement \label{subsec:Statement-compatible-retract-of-free-monad}\ref{subsec:Statement-compatible-retract-of-free-monad}}

Let $E$ be a functor such that:

1. For any type $T$, the type $E^{T}$ is a $P$-algebra with a structure
map $p_{E}^{T}:P^{E^{T}}\rightarrow E^{T}$, natural in $T$. 

2. There exist natural transformations $\text{pu}_{E}^{T}:T\rightarrow E^{T}$
and $\text{in}_{E}^{T}:E^{T}\rightarrow\text{FPR}^{T}$ (which is
\emph{not} required to be a $P$-algebra morphism).

3. The following properties hold:
\begin{equation}
\text{pu}_{E}^{T}\bef\text{in}_{E}^{T}=\text{pu}_{\text{FPR}}^{T}\quad,\quad\quad\text{in}_{E}^{T}\bef(\text{pu}_{E}^{T})^{\uparrow\text{FPR}}\bef\text{eval}_{\text{FPR}}^{E^{T}}=\text{id}^{:E^{T}\rightarrow E^{T}}\quad.\label{eq:laws-of-in-E-free-P-typeclass-encoding}
\end{equation}
The last requirement says that the (unique) $P$-algebra morphism
$(\text{pu}_{E}^{T})^{\uparrow\text{FPR}}\bef\text{eval}_{\text{FPR}}^{E^{T}}$
of type $\text{FPR}^{T}\rightarrow E^{T}$ is the left inverse to
the function $\text{in}_{E}^{T}$. This indicates that $\text{in}_{E}^{T}:E^{T}\rightarrow\text{FPR}^{T}$
is injective\index{injective function}. So, $E^{T}$ is a subset
of $\text{FPR}^{T}$ in this sense.

4. The function $\text{ftn}_{E}:E^{E^{T}}\rightarrow E^{T}$ defined
as $\text{ftn}_{E}\triangleq\text{in}_{E}^{E^{T}}\bef\text{eval}_{\text{FPR}}^{E^{T}}$
is a $P$-algebra morphism.

5. The $P$-algebra $E^{T}$ satisfies some (or all) of the $P$-typeclass
laws but \emph{no other} laws. To formulate this requirement precisely,
we use law expressions $e:\text{FPR}^{\text{Int}}\times\text{FPR}^{\text{Int}}$
for describing $P$-typeclass laws (see Section~\ref{subsec:Describing-laws-of-P-typeclasses-as-values}).
We say that $E^{T}$ obeys \textsf{``}no other laws than those of the $P$-typeclass\textsf{''}
if for any $e:\text{FPR}^{\text{Int}}\times\text{FPR}^{\text{Int}}$
whose corresponding $e$-law holds for $E^{T}$, the same $e$-law
will also hold for any $P$-algebra $C$ that satisfies all the $P$-typeclass
laws. (A counterexample is an $E^{T}$ that satisfies all the standard
monoid laws and in addition the commutativity law. By Statement~\ref{subsec:Statement-P-algebra-morphisms-and-laws},
there cannot be a surjective monoid morphism $E^{M}\rightarrow M$
when $M$ is a non-commutative monoid. Yet, that morphism would have
existed if $E$ were a valid encoding of a free monoid.)

If assumptions 1-5 hold then:

\textbf{(a)} The functor $E$ is a free $P$-typeclass encoding. When
the function $\text{in}_{E}$ is itself a $P$-algebra morphism then
the encoding $E$ is equivalent to \inputencoding{latin9}\lstinline!FPR!\inputencoding{utf8}.
(This statement does \emph{not} require $\text{in}_{E}^{T}$ to be
a $P$-algebra morphism, in order to be able to describe free $P$-typeclass
encodings $E^{T}$ that are different from $\text{FPR}^{T}$.)

\textbf{(b)} The universal evaluator $\text{eval}_{E}^{C}$ does not
depend on the choice of the function $\text{in}_{E}$ as long as the
same assumptions hold for that function.

\subparagraph{Proof}

\textbf{(a)} If $\text{in}_{E}^{T}$ is a $P$-algebra morphism then
the function $g_{E}^{T}$ defined by:
\[
g_{E}^{T}:\text{FPR}^{T}\rightarrow\text{FPR}^{T}\quad,\quad\quad g_{E}^{T}\triangleq\text{run}_{\text{FPR}}^{T,E^{T}}(\text{pu}_{E}^{T})\bef\text{in}_{E}^{T}=(\text{pu}_{E}^{T})^{\uparrow\text{FPR}}\bef\text{eval}_{\text{FPR}}^{E^{T}}\bef\text{in}_{E}^{T}\quad,
\]
will be also a $P$-algebra morphism (as a composition of several
$P$-algebra morphisms). However, $P$-algebra morphisms of type $\text{FPR}^{T}\rightarrow C$
are unique when a function $r:T\rightarrow C$ is fixed (Statement~\ref{subsec:Statement-some-properties-of-free-P-typeclass-encoding}).
In the present case, we have $C=\text{FPR}^{T}$ and the function
of type $T\rightarrow C$ is $\text{pu}_{\text{FPR}}$. Let us apply
Statement~\ref{subsec:Statement-some-properties-of-free-P-typeclass-encoding}(a).
We first need to compute $r$ as: 
\begin{align*}
 & r\triangleq\text{pu}_{\text{FPR}}^{T}\bef g_{E}^{T}=\gunderline{\text{pu}_{\text{FPR}}^{T}\bef(\text{pu}_{E}^{T})^{\uparrow\text{FPR}}}\bef\text{eval}_{\text{FPR}}^{E^{T}}\bef\text{in}_{E}^{T}\\
{\color{greenunder}\text{naturality of }\text{pu}_{\text{FPR}}:}\quad & =\text{pu}_{E}^{T}\bef\gunderline{\text{pu}_{\text{FPR}}^{E^{T}}\bef\text{eval}_{\text{FPR}}^{E^{T}}}\bef\text{in}_{E}^{T}\\
{\color{greenunder}\text{left identity law~(\ref{eq:free-typeclass-encoding-left-identity-law}) for }\text{FPR}:}\quad & =\text{pu}_{E}^{T}\bef\text{in}_{E}^{T}\\
{\color{greenunder}\text{assumption~(\ref{eq:laws-of-in-E-free-P-typeclass-encoding})}:}\quad & =\text{pu}_{\text{FPR}}^{T}\quad.
\end{align*}
By Statement~\ref{subsec:Statement-some-properties-of-free-P-typeclass-encoding}(a),
there exists only one $P$-algebra morphism with this $r$, namely
$r^{\uparrow\text{FPR}}\bef\text{eval}_{\text{FPR}}$. That morphism
is, however, equal to the identity function due to Eq.~(\ref{eq:free-typeclass-encoding-right-identity-law}):
\[
r^{\uparrow\text{FPR}}\bef\text{eval}_{\text{FPR}}=\text{pu}_{\text{FPR}}^{\uparrow\text{FPR}}\bef\text{eval}_{\text{FPR}}=\text{id}^{:\text{FPR}^{T}\rightarrow\text{FPR}^{T}}\quad.
\]
We find that we must have $g_{E}^{T}=\text{id}$. It means that the
$P$-algebra morphisms $\text{in}_{E}^{T}$ and $g$ are inverses
of each other and give an \emph{isomorphism} (a type equivalence that
preserves the $P$-typeclass operations) between $P$-algebras $E^{T}$
and $\text{FPR}^{T}$. 

To show that $E$ is a $P$-typeclass encoding, we look at the conditions
in Definition~\ref{subsec:Definition-free-P-typeclass-encoding}.

Conditions (a) and (b) of Definition~\ref{subsec:Definition-free-P-typeclass-encoding}
are satisfied by our assumptions 1 and 2.

To verify condition (c), we assume that $C$ is any $P$-algebra that
obeys all the $P$-typeclass laws that $E$ obeys. We then \emph{define}
$\text{eval}_{E}^{C}$ via $\text{eval}_{\text{FPR}}^{C}$:
\[
\text{eval}_{E}^{C}:E^{C}\rightarrow C\quad,\quad\quad\text{eval}_{E}^{C}\triangleq\text{in}_{E}^{C}\bef\text{eval}_{\text{FPR}}^{C}\quad.
\]
Now we need to show that $\text{eval}_{E}^{C}$ satisfies the required
properties.

To verify the left identity law:
\begin{align*}
{\color{greenunder}\text{expect to equal }\text{id}^{:C\rightarrow C}:}\quad & \text{pu}_{E}^{C}\bef\text{eval}_{E}^{C}=\gunderline{\text{pu}_{E}^{C}\bef\text{in}_{E}^{C}}\bef\text{eval}_{\text{FPR}}^{C}\\
{\color{greenunder}\text{use Eq.~(\ref{eq:laws-of-in-E-free-P-typeclass-encoding})}:}\quad & =\gunderline{\text{pu}_{\text{FPR}}^{C}\bef\text{eval}_{\text{FPR}}^{C}}\\
{\color{greenunder}\text{left identity law~(\ref{eq:free-typeclass-encoding-left-identity-law}) of }\text{FPR}:}\quad & =\text{id}^{:C\rightarrow C}\quad.
\end{align*}

To verify the right identity law:
\begin{align*}
{\color{greenunder}\text{expect to equal }\text{id}^{:E^{T}\rightarrow E^{T}}:}\quad & (\text{pu}_{E}^{T})^{\uparrow E}\bef\text{eval}_{E}^{E^{T}}=\gunderline{(\text{pu}_{E}^{T})^{\uparrow E}\bef\text{in}_{E}^{E^{T}}}\bef\text{eval}_{\text{FPR}}^{E^{T}}\\
{\color{greenunder}\text{naturality law of }\text{in}_{E}^{E^{T}}:}\quad & =\text{in}_{E}^{T}\bef(\text{pu}_{E}^{T})^{\uparrow\text{FPR}}\bef\text{eval}_{\text{FPR}}^{E^{T}}\\
{\color{greenunder}\text{assumption in Eq.~(\ref{eq:laws-of-in-E-free-P-typeclass-encoding})}:}\quad & =\text{id}^{:E^{T}\rightarrow E^{T}}\quad.
\end{align*}

To verify the $P$-algebra naturality law~(\ref{eq:free-typeclass-encoding-P-algebra-naturality-law}),
write its left-hand side:
\begin{align*}
{\color{greenunder}\text{expect to equal }g^{\uparrow E}\bef\text{eval}_{E}:}\quad & \text{eval}_{E}^{C}\bef g=\text{in}_{E}\bef\gunderline{\text{eval}_{\text{FPR}}\bef g}\\
{\color{greenunder}P\text{-algebra naturality law of }\text{eval}_{\text{FPR}}:}\quad & =\gunderline{\text{in}_{E}\bef g^{\uparrow\text{FPR}}}\bef\text{eval}_{\text{FPR}}\\
\text{naturality law of }\text{in}_{E}:is & =g^{\uparrow E}\bef\gunderline{\text{in}_{E}\bef\text{eval}_{\text{FPR}}}=g^{\uparrow E}\bef\text{eval}_{E}\quad.
\end{align*}

It remains to prove the $P$-algebra morphism law of $\text{eval}_{E}$:
\begin{align}
p_{E^{C}}\bef\text{in}_{E}^{C}\bef\text{eval}_{\text{FPR}}^{C}\overset{?}{=}\big(\text{in}_{E}^{C}\bef\text{eval}_{\text{FPR}}^{C}\big)^{\uparrow P}\bef p_{C}\quad.\label{eq:P-algebra-morphism-law-for-in-r-derivation1}
\end{align}

{*}{*}{*}Verify Eq.~(\ref{eq:P-algebra-morphism-law-for-in-r-derivation1})
takes much more work. We first note that for any $C$ satisfying the
given assumptions, the $P$-algebra morphism law already holds for
$r_{\text{FPR}}^{T,C}$:
\[
p_{\text{FPR}^{T}}\bef r_{\text{FPR}}^{T,C}=\big(r_{\text{FPR}}^{T,C}\big)^{\uparrow P}\bef p_{C}\quad.
\]
Then the $P$-algebra morphism law for $r_{E}^{T,C}$ becomes:
\[
p_{E^{T}}\bef\text{in}_{E}^{T}\bef r_{\text{FPR}}^{T,C}\overset{?}{=}\big(\text{in}_{E}^{T}\big)^{\uparrow P}\bef p_{\text{FPR}^{T}}\bef r_{\text{FPR}}^{T,C}\quad.
\]
Apply both sides of the last equation to an arbitrary fixed value
$x:P^{E^{T}}$ and rewrite the equation as: 
\[
r_{\text{FPR}}^{T,C}(a)\overset{?}{=}r_{\text{FPR}}^{T,C}(b)\quad,\quad\quad a^{:\text{FPR}^{T}}\triangleq x\triangleright p_{E^{T}}\bef\text{in}_{E}^{T}\quad,\quad\quad b^{:\text{FPR}^{T}}\triangleq x\triangleright\big(\text{in}_{E}^{T}\big)^{\uparrow P}\bef p_{\text{FPR}^{T}}\quad.
\]

The values $a$ and $b$ play a similar role to the two parts of a
law expression $e:\text{FPR}^{\text{Int}}\times\text{FPR}^{\text{Int}}$.
We note that Eq.~(\ref{eq:P-algebra-morphism-law-for-in-r-derivation1})
holds for $C=E^{T}$ due to the assumption $\text{in}_{E}^{T}\bef r_{\text{FPR}}^{T,E^{T}}=\text{id}$:
\begin{align*}
{\color{greenunder}\text{left-hand side of Eq.~(\ref{eq:P-algebra-morphism-law-for-in-r-derivation1}) with }C=E^{T}:}\quad & p_{E^{T}}\bef\gunderline{\text{in}_{E}^{T}\bef r_{\text{FPR}}^{T,E^{T}}}=p_{E^{T}}\bef\text{id}=p_{E^{T}}\quad,\\
{\color{greenunder}\text{right-hand side of Eq.~(\ref{eq:P-algebra-morphism-law-for-in-r-derivation1}) with }C=E^{T}:}\quad & \big(\gunderline{\text{in}_{E}^{T}\bef r_{\text{FPR}}^{T,E^{T}}}\big)^{\uparrow P}\bef p_{E^{T}}=\text{id}^{\uparrow P}\bef p_{E^{T}}=p_{E^{T}}\quad.
\end{align*}
The fact that Eq.~(\ref{eq:P-algebra-morphism-law-for-in-r-derivation1})
holds for $C=E^{T}$ can be expressed via $a$, $b$ as:
\[
r_{\text{FPR}}^{T,E^{T}}(a)\overset{!}{=}r_{\text{FPR}}^{T,E^{T}}(b)\quad.
\]
This resembles the statement of a typeclass law that holds in $E^{T}$,
except that $a$ and $b$ are specific values of type $\text{FPR}^{T}$.
We have the assumption that all laws of $E$ also hold for $C$. This
assumption will allow us to complete the proof of Eq.~(\ref{eq:P-algebra-morphism-law-for-in-r-derivation1})
once we represent the relevant laws as values of type $\text{FPR}^{\text{Int}}\times\text{FPR}^{\text{Int}}$.

We first set $T=\text{Int}$. For any $x:P^{E^{\text{Int}}}$ we define
a law expression $e$ as:
\[
e:\text{FPR}^{\text{Int}}\times\text{FPR}^{\text{Int}}\quad,\quad\quad e\triangleq a\times b\quad,\quad\quad a\triangleq(x\triangleright p_{E^{\text{Int}}}\bef\text{in}_{E}^{\text{Int}})\quad,\quad\quad b\triangleq x\triangleright\big(\text{in}_{E}^{\text{Int}}\big)^{\uparrow P}\bef p_{\text{FPR}^{\text{Int}}}\quad.
\]
Since $r_{\text{FPR}}^{\text{Int},E^{\text{Int}}}(a)=r_{\text{FPR}}^{\text{Int},E^{\text{Int}}}(b)$,
the $e$-law holds in $E^{\text{Int}}$.{*}{*}{*}Clarify that in the
previous section! 

We need to write explicitly a law that holds in $E^{T}$.

first set $T=\text{Int}$

{*}{*}{*}

We have proved that $r_{E}^{T,C}$ is a $Q_{T}$-algebra morphism.
It remains to show that any other $Q_{T}$-algebra morphism $g:E^{T}\rightarrow C$
is equal to $r_{E}^{T,C}$. Consider the function $h\triangleq r_{\text{FPR}}^{T,E^{T}}\bef g$
of type $\text{FPR}^{T}\rightarrow C$. The function $h$ is a composition
of $Q_{T}$-algebra morphisms, and so $h$ is itself one. By the uniqueness
property of $r_{\text{FPR}}$, any $Q_{T}$-algebra morphism of type
$\text{FPR}^{T}\rightarrow C$ is equal to $r_{\text{FPR}}^{T,C}$:
\begin{equation}
r_{\text{FPR}}^{T,E^{T}}\bef g=r_{\text{FPR}}^{T,C}\quad.\label{eq:qt-morphism-derivation1}
\end{equation}
To prove that $g=r_{E}^{T,C}$, we use the definition $r_{E}^{T,C}\triangleq\text{in}_{E}^{T}\bef r_{\text{FPR}}^{T,C}$
and write:
\begin{align*}
{\color{greenunder}\text{expect to equal }g:}\quad & r_{E}^{T,C}=\text{in}_{E}^{T}\bef\gunderline{r_{\text{FPR}}^{T,C}}\\
{\color{greenunder}\text{by Eq.~(\ref{eq:qt-morphism-derivation1})}:}\quad & =\gunderline{\text{in}_{E}^{T}\bef r_{\text{FPR}}^{T,E^{T}}}\bef g\\
{\color{greenunder}\text{by Eq.~(\ref{eq:laws-of-in-E-free-P-typeclass-encoding})}:}\quad & =g\quad.
\end{align*}

\textbf{(b)} Given a new function $\text{in}_{E}^{\prime}$ that also
satisfies Eq.~(\ref{eq:laws-of-in-E-free-P-typeclass-encoding}),
we define the corresponding new evaluator function, $\text{eval}_{E}^{\prime C}\triangleq\text{in}_{E}^{\prime C}\bef\text{eval}_{\text{FPR}}^{C}$.
Now we will show that $\text{eval}_{E}^{\prime}=\text{eval}_{E}$.
Part \textbf{(a)} of the proof shows that $\text{eval}_{E}^{\prime}$
has the same properties as $\text{eval}_{E}$; in particular, $\text{eval}_{E}^{\prime C}$
is a $P$-algebra morphism of type $E^{C}\rightarrow C$ and satisfies
the equation:
\begin{align*}
 & \text{pu}_{E}^{C}\bef\text{eval}_{E}^{\prime C}=\gunderline{\text{pu}_{E}^{C}\bef\text{in}_{E}^{\prime C}}\bef\text{eval}_{\text{FPR}}^{C}\\
{\color{greenunder}\text{use Eq.~(\ref{eq:laws-of-in-E-free-P-typeclass-encoding})}:}\quad & =\text{pu}_{\text{FPR}}^{C}\bef\text{eval}_{\text{FPR}}^{C}\\
{\color{greenunder}\text{left identity law~(\ref{eq:free-typeclass-encoding-left-identity-law}) for }\text{FPR}:}\quad & =\text{id}^{:C\rightarrow C}\quad.
\end{align*}
By Definition~\ref{subsec:Definition-free-P-typeclass-encoding}(d),
there is only one $P$-algebra morphism of type $E^{C}\rightarrow C$
that satisfies that equation, namely $\text{eval}_{E}^{C}$. So, $\text{eval}_{E}^{\prime C}=\text{eval}_{E}^{C}$.
$\square$

\subsection{Imposing laws of $P$-typeclasses via monad algebras}

We have shown in Statement~\ref{subsec:Statement-free-typeclass-encoding-is-a-monad}
that any free $P$-typeclass encoding $E$ is always a monad. It turns
out that we may may use $E$\textsf{'}s monad methods to impose $P$-typeclass
laws on a given $P$-algebra. The required technique is based on the
notion of a \textsf{``}\index{monad algebra}monad algebra\textsf{''}:

\subsubsection{Definition \label{subsec:Definition-monad-algebra}\ref{subsec:Definition-monad-algebra}}

Given a monad $E$, a type $C$ is an $E$-\textbf{monad algebra}
if $C$ is an $E$-algebra with a structure map $s:E^{C}\rightarrow C$
that additionally obeys the following laws:

\begin{wrapfigure}{l}{0.2\columnwidth}%
\vspace{-2.15\baselineskip}
\[
\xymatrix{\xyScaleY{1.4pc}\xyScaleX{2.5pc}C\ar[r]\sp(0.5){\text{pu}_{E}}\ar[dr]\sb(0.5){\,\text{id}} & E^{C}\ar[d]\sp(0.45){s} & E^{E^{C}}\ar[d]\sb(0.45){s^{\uparrow E}}\ar[r]\sp(0.55){\text{ftn}_{E}} & E^{C}\ar[d]\sb(0.45){s}\\
 & C & E^{C}\ar[r]\sp(0.5){s} & C
}
\]
\vspace{-0.6\baselineskip}
\end{wrapfigure}%

\noindent ~\vspace{-2\baselineskip}
\begin{align*}
{\color{greenunder}\text{identity law of monad algebras}:}\quad & \text{pu}_{E}^{C}\bef s=\text{id}^{:C\rightarrow C}\quad,\\
{\color{greenunder}\text{composition law of monad algebras}:}\quad & \text{ftn}_{E}\bef s=s^{\uparrow E}\bef s\quad.
\end{align*}
$\square$

\subsubsection{Statement \label{subsec:Statement-Monad-algebra-is-P-typeclass}\ref{subsec:Statement-Monad-algebra-is-P-typeclass}}

Suppose $E$ is a free $P$-typeclass encoding according to Definition~\ref{subsec:Definition-free-P-typeclass-encoding}.
A monad structure for $E$ is given by Statement~\ref{subsec:Statement-free-typeclass-encoding-is-a-monad}.
Then the type of $E$-monad algebras is equivalent to the type of
$P$-algebras that obey all the laws of $E$. In more detail:

\textbf{(a)} Any $E$-monad algebra $C$ is also a $P$-algebra. The
structure map $s:E^{C}\rightarrow C$ is a surjective $P$-algebra
morphism.

\textbf{(b)} Any $P$-algebra $C$ that satisfies all the laws of
$E^{\bullet}$ is also an $E$-monad algebra.

\textbf{(c)} The $P$-algebra structures and the $E$-monad algebra
structures are in a 1-to-1 correspondence.

\subparagraph{Proof}

\textbf{(a)} To show that $C$ is a $P$-algebra, we define the structure
map $p_{C}:P^{C}\rightarrow C$ by:
\[
p_{C}:P^{C}\rightarrow C\quad,\quad\quad p_{C}\triangleq\text{pu}_{E}^{\uparrow P}\bef p_{E}^{C}\bef s\quad.
\]

To show that $s$ is a $P$-algebra morphism, we write the law~(\ref{eq:p-algebra-morphism-law})
with $f=s$:
\begin{align*}
{\color{greenunder}\text{expect to equal }p_{E}^{C}\bef s:}\quad & s^{\uparrow P}\bef\gunderline{p_{C}}=\gunderline{s^{\uparrow P}\bef\text{pu}_{E}^{\uparrow P}}\bef p_{E}^{C}\bef s=(\gunderline{s\bef\text{pu}_{E}})^{\uparrow P}\bef p_{E}^{C}\bef s\\
{\color{greenunder}\text{naturality law of }\text{pu}_{E}:}\quad & =\gunderline{(\text{pu}_{E}\bef s^{\uparrow E})^{\uparrow P}}\bef p_{E}^{C}\bef s=\text{pu}_{E}^{\uparrow P}\bef\gunderline{s^{\uparrow E\uparrow P}\bef p_{E}^{C}}\bef s\\
{\color{greenunder}\text{naturality law~(\ref{eq:free-typeclass-encoding-P-naturality-law}) of }p_{E}^{C}:}\quad & =\text{pu}_{E}^{\uparrow P}\bef p_{E}^{E^{C}}\bef\gunderline{s^{\uparrow E}\bef s}\\
{\color{greenunder}E\text{-monad algebra composition law of }s:}\quad & =\text{pu}_{E}^{\uparrow P}\bef\gunderline{p_{E}^{E^{C}}\bef\text{ftn}_{E}}\bef s\\
{\color{greenunder}P\text{-algebra morphism law~(\ref{eq:P-algebra-morphism-law-of-flatten}) of }\text{ftn}_{E}:}\quad & =\gunderline{\text{pu}_{E}^{\uparrow P}\bef\text{ftn}_{E}^{\uparrow P}}\bef p_{E}^{C}\bef s=(\gunderline{\text{pu}_{E}\bef\text{ftn}_{E}})^{\uparrow P}\bef p_{E}^{C}\bef s\\
{\color{greenunder}\text{left identity law of }E:}\quad & =p_{E}^{C}\bef s\quad.
\end{align*}
The $E$-monad algebra identity law ($\text{pu}_{E}\bef s=\text{id}$)
means that $s$ is surjective.

We note that this part of the proof does not use the uniqueness property
of $E$. So, $C$ is also a $P$-algebra under the weaker assumption
that $E$ is any monad with a $P$-algebra structure, appropriate
naturality laws, and the property that $\text{ftn}_{E}$ is a $P$-algebra
morphism. 

\textbf{(b)} If $C$ is a $P$-algebra, we define an $E$-algebra
structure map $s:E^{C}\rightarrow C$ by:
\[
s:E^{C}\rightarrow C\quad,\quad\quad s\triangleq\text{run}_{E}(\text{id}^{:C\rightarrow C})\quad.
\]
As $C$ satisfies all the laws of $E$, we find from Definition~\ref{subsec:Definition-free-P-typeclass-encoding}(c)
that $s=\text{run}_{E}(\text{id})$ is a $P$-algebra morphism. It
remains to verify that $s$ obeys the monad algebra\textsf{'}s laws:
\begin{align*}
{\color{greenunder}\text{identity law}:}\quad & \text{pu}_{E}^{C}\bef s\overset{?}{=}\text{id}^{:C\rightarrow C}\quad,\\
{\color{greenunder}\text{composition law}:}\quad & \text{ftn}_{E}\bef s\overset{?}{=}s^{\uparrow E}\bef s\quad.
\end{align*}

The identity law is verified by using Eq.~(\ref{eq:free-typeclass-encoding-left-identity-law})
with $T=C$ and $r=\text{id}$:
\[
\text{pu}_{E}^{C}\bef s=\text{pu}_{E}^{C}\bef\text{run}_{E}^{C,C}(\text{id}^{:C\rightarrow C})=\text{id}^{:C\rightarrow C}\quad.
\]

To verify the composition law, we rewrite it in terms of $\text{run}_{E}$
using the fact that $\text{ftn}_{E}=\text{run}_{E}(\text{id})$:
\[
\gunderline{\text{ftn}_{E}}\bef s=\text{run}_{E}^{E^{C},E^{C}}(\text{id}^{:E^{C}\rightarrow E^{C}})\bef s\overset{?}{=}s^{\uparrow E}\bef\gunderline s=s^{\uparrow E}\bef\text{run}_{E}^{C,C}(\text{id}^{:C\rightarrow C})\quad.
\]
The $P$-algebraic naturality law~(\ref{eq:free-typeclass-encoding-P-algebraic-naturality-law})
with $g=s$ and $r=\text{id}^{:E^{C}\rightarrow E^{C}}$ gives for
the left-hand side:
\[
\text{run}_{E}^{E^{C},E^{C}}(\text{id}^{:E^{C}\rightarrow E^{C}})\bef s=\text{run}_{E}^{E^{C},C}(\gunderline{\text{id}^{:E^{C}\rightarrow E^{C}}\bef s})=\text{run}_{E}^{E^{C},C}(s)\quad.
\]
The right-hand side is simplified using the naturality law~(\ref{eq:free-typeclass-encoding-naturality-law})
with $f=s$ and $r=\text{id}^{:E^{C}\rightarrow E^{C}}$:
\[
s^{\uparrow E}\bef\text{run}_{E}^{C,C}(\text{id}^{:E^{C}\rightarrow E^{C}})=\text{run}_{E}^{E^{C},C}(s\bef\text{id}^{:E^{C}\rightarrow E^{C}})=\text{run}_{E}^{E^{C},C}(s)\quad.
\]
Both sides of the composition law are now equal to $\text{run}_{E}^{E^{C},C}(s)$.

\textbf{(c)} We have mapped an $E$-monad algebra to a $P$-algebra
and back. Now we will show that this mapping is an isomorphism in
both directions.

For a given $E$-monad algebra $C$ with a structure map $s:E^{C}\rightarrow C$,
we defined a structure map $p_{C}$:
\[
p_{C}:P^{C}\rightarrow C\quad,\quad\quad p_{C}\triangleq\text{pu}_{E}^{\uparrow P}\bef p_{E}^{C}\bef s\quad.
\]
Since $C$ is now a $P$-algebra, we can define a new $E$-monad algebra
structure map, $s\textsf{'}\triangleq\text{run}_{E}^{C,C}(\text{id})$. To
show that $s\textsf{'}=s$, note that $s$ is a $P$-algebra morphism, as we
proved in part \textbf{(a)}. Additionally, $s$ satisfies $\text{pu}_{E}\bef s=\text{id}$.
Then we may apply the uniqueness property of $\text{run}_{E}$ from
Definition~\ref{subsec:Definition-free-P-typeclass-encoding}(d)
and get $s=\text{run}_{E}^{C,C}(\text{id})=s\textsf{'}$.

For a given $P$-algebra $C$ that satisfies all the laws of $E$,
we define the $E$-monad algebra structure map by $s\triangleq\text{run}_{E}(\text{id})$.
Now we can define a new $P$-algebra structure map $p_{C}^{\prime}\triangleq\text{pu}_{E}^{\uparrow P}\bef p_{E}^{C}\bef s$.
To prove that $p_{C}^{\prime}=p_{C}$, we use the $P$-algebra morphism
law of $s$:
\[
p_{E}^{C}\bef s=s^{\uparrow P}\bef p_{C}\quad,
\]
as well as the identity law $\text{pu}_{E}\bef s=\text{id}$ to obtain:
\[
p_{C}^{\prime}=\text{pu}_{E}^{\uparrow P}\bef\gunderline{p_{E}^{C}\bef s}=\gunderline{\text{pu}_{E}^{\uparrow P}\bef s^{\uparrow P}}\bef p_{C}=(\gunderline{\text{pu}_{E}\bef s})^{\uparrow P}\bef p_{C}=\gunderline{\text{id}^{\uparrow P}}\bef p_{C}=p_{C}\quad.
\]
$\square$

\subsubsection{Exercise \label{subsec:Exercise-P-algebras-monad-algebras}\ref{subsec:Exercise-P-algebras-monad-algebras}}

Suppose $E$ is a free $P$-typeclass encoding according to Definition~\ref{subsec:Definition-free-P-typeclass-encoding}.
If $C$ and $D$ are any given $E$-algebras, the proof of Statement~\ref{subsec:Statement-Monad-algebra-is-P-typeclass}
defines functions $p_{C}$, $p_{D}$ that make $C$ and $D$ into
$P$-algebras. If $f:C\rightarrow D$ is any $E$-algebra morphism,
show that $f$ is at the same time a $P$-algebra morphism between
the $P$-algebras $C$ and $D$. $\square$

We can now use monad algebras to prove that certain typeclass constructions
always give new and lawful typeclass instances. {*}{*}{*}

\subsection{Church encodings of free $P$-typeclasses\label{subsec:Church-encodings-for-free-P-typeclasses}}

\subsection{Free constructions on more than one generator}

Can combine two or more DSLs in a disjunction: $\text{DSL}^{F+G+H,A}$ 

- combine semigroup and pointed to get a monoid

\section{Slides }


\paragraph{Mapping a free semigroup to different targets}

What if we interpret $\text{FS}^{X}$ into \emph{another} free semigroup?

Given $Y\rightarrow Z$, can we map $\text{FS}^{Y}\rightarrow\text{FS}^{Z}$?

Need to map $\text{FS}^{Y}\triangleq Y+\text{FS}^{Y}\times\text{FS}^{Y}\rightarrow Z+\text{FS}^{Z}\times\text{FS}^{Z}$

This is straightforward since $\text{FS}^{X}$ is a functor in $X$:

\texttt{\textcolor{blue}{\footnotesize{}def fmap{[}Y, Z{]}(f: Y $\rightarrow$
Z): FS{[}Y{]} $\rightarrow$ FS{[}Z{]} = \{}}{\footnotesize\par}

\texttt{\textcolor{blue}{\footnotesize{}  case Wrap(y) $\rightarrow$
Wrap(f(y))}}{\footnotesize\par}

\texttt{\textcolor{blue}{\footnotesize{}  case Comb(a, b) $\rightarrow$
Comb(fmap(f)(a), fmap(f)(b))}}{\footnotesize\par}

\texttt{\textcolor{blue}{\footnotesize{}\}}}{\footnotesize\par}

Now we can use \texttt{\textcolor{blue}{\footnotesize{}run}} to interpret
$\text{FS}^{X}\rightarrow\text{FS}^{Y}\rightarrow\text{FS}^{Z}\rightarrow S$,
etc.

Functor laws hold for $\text{FS}^{X}$, so \texttt{\textcolor{blue}{\footnotesize{}fmap}}
is composable as usual

The \textsf{``}interpreter\textsf{''} commutes with \texttt{\textcolor{blue}{\footnotesize{}fmap}}
as well (naturality law):{\footnotesize{}}{\footnotesize{}
\[
\xymatrix{\xyScaleY{0.2pc}\xyScaleX{3pc} & \text{FS}^{Y}\ar[rd]\sp(0.6){\ \text{run}^{S}g^{:Y\rightarrow S}}\\
\text{FS}^{X}\ar[ru]\sp(0.45){\text{fmap}\,f^{:X\rightarrow Y}}\ar[rr]\sb(0.5){\text{run}^{S}(f\bef g)^{:X\rightarrow S}} &  & S
}
\]
}{\footnotesize\par}

Combine two free semigroups: $\text{FS}^{X+Y}$; inject parts: $\text{FS}^{X}\rightarrow\text{FS}^{X+Y}$ 


\paragraph{Church encoding I: Motivation}

Multiple target semigroups $S_{i}$ require many \textsf{``}extractors\textsf{''}
$\text{ex}_{i}:Z\rightarrow S_{i}$

Refactor extractors $\text{ex}_{i}$ into evidence of a typeclass
constraint on $S_{i}$

\textcolor{darkgray}{\footnotesize{}// Typeclass ExZ{[}S{]} has a
single method, extract: Z $\rightarrow$ S.}{\footnotesize\par}

\texttt{\textcolor{blue}{\footnotesize{}implicit val exZ: ExZ{[}MySemigroup{]}
= \{ z $\rightarrow$ ... \}}}{\footnotesize\par}

\texttt{\textcolor{blue}{\footnotesize{}def run{[}S: ExZ : Semigroup{]}(fs: FS{[}Z{]}): S
= fs match \{}}{\footnotesize\par}

\texttt{\textcolor{blue}{\footnotesize{}  case Wrap(z) $\rightarrow$
implicitly{[}ExZ{[}S{]}{]}.extract(z)}}{\footnotesize\par}

\texttt{\textcolor{blue}{\footnotesize{}  case Comb(x, y) $\rightarrow$
run(x) |+| run(y)}}{\footnotesize\par}

\texttt{\textcolor{blue}{\footnotesize{}\}}}{\footnotesize\par}

\texttt{\textcolor{blue}{\footnotesize{}run()}} replaces case classes
by fixed functions parameterized by \texttt{\textcolor{blue}{\footnotesize{}S:~ExZ}};
instead we can represent \texttt{\textcolor{blue}{\footnotesize{}FS{[}Z{]}}}
directly by such functions, for example:

\texttt{\textcolor{blue}{\footnotesize{}def wrap{[}S: ExZ{]}(z: Z): S
= implicitly{[}ExZ{[}S{]}{]}.extract(z)}}{\footnotesize\par}

\texttt{\textcolor{blue}{\footnotesize{}def x{[}S: ExZ : Semigroup{]}: S
= wrap(1) |+| wrap(2)}}{\footnotesize\par}

The type of \texttt{\textcolor{blue}{\footnotesize{}x}} is {\footnotesize{}$\forall S.\left(Z\rightarrow S\right)\times\left(S\times S\rightarrow S\right)\rightarrow S$};
an equivalent type is{\footnotesize{}
\[
\forall S.\left(\left(Z+S\times S\right)\rightarrow S\right)\rightarrow S
\]
}{\footnotesize\par}

This is the \textsf{``}\textbf{Church encoding}\textsf{''} (of the free semigroup
over $Z$)

The Church encoding is based on the theorem {\footnotesize{}$A\cong\forall X.\left(A\rightarrow X\right)\rightarrow X$} 

this \emph{resembles} the type of the continuation monad, $\left(A\rightarrow R\right)\rightarrow R$ 

but $\forall X$ makes the function fully generic, like a natural
transformation


\paragraph{Church encoding II: Disjunction types}

Consider the Church encoding for the disjunction type $P+Q$ 

The encoding is {\footnotesize{}$\forall X.\left(P+Q\rightarrow X\right)\rightarrow X\cong\forall X.\left(P\rightarrow X\right)\rightarrow\left(Q\rightarrow X\right)\rightarrow X$}{\footnotesize\par}

\texttt{\textcolor{blue}{\footnotesize{}trait Disj{[}P, Q{]} \{ def
run{[}X{]}(cp: P $\rightarrow$ X)(cq: Q $\rightarrow$ X): X \}}}{\footnotesize\par}

Define some values of this type:

\texttt{\textcolor{blue}{\footnotesize{}def left{[}P, Q{]}(p: P) =
new Disj{[}P, Q{]} \{}}{\footnotesize\par}

\texttt{\textcolor{blue}{\footnotesize{} def run{[}X{]}(cp: P $\rightarrow$
X)(cq: Q $\rightarrow$ X): X = cp(p) }}{\footnotesize\par}

\texttt{\textcolor{blue}{\footnotesize{}\}}}{\footnotesize\par}

Now we can implement the analog of the \texttt{\textcolor{blue}{\footnotesize{}case}}
expression simply as

\texttt{\textcolor{blue}{\footnotesize{}val result = disj.run \{p
$\rightarrow$ ...\} \{q $\rightarrow$ ...\}}}{\footnotesize\par}

This works in programming languages that have no disjunction types

General recipe for implementing the Church encoding: 

\texttt{\textcolor{blue}{\footnotesize{}trait Blah \{ def run{[}X{]}(cont: ... $\rightarrow$
X): X \}}}{\footnotesize\par}

For convenience, define a type class \texttt{\textcolor{blue}{\footnotesize{}Ex}}
describing the inner function:

\texttt{\textcolor{blue}{\footnotesize{}trait Ex{[}X{]} \{ def cp: P
$\rightarrow$ X; def cq: Q $\rightarrow$ X \}}}{\footnotesize\par}

Different methods of this class return \texttt{\textcolor{blue}{\footnotesize{}X}};
convenient with disjunctions

Church-encoded types have to be \textsf{``}run\textsf{''} for pattern matching to
work on the results


\paragraph{Church encoding III: How it works}

Why is the type $\text{Ch}^{A}\triangleq\forall X.\left(A\rightarrow X\right)\rightarrow X$
equivalent to the type $A$?

\texttt{\textcolor{blue}{\footnotesize{}trait Ch{[}A{]} \{ def run{[}X{]}(cont: A
$\rightarrow$ X): X \}}}{\footnotesize\par}

\texttt{\textcolor{blue}{\footnotesize{}}}%
\begin{minipage}[t]{0.65\textwidth}%
\begin{itemize}
\item If we have a value of $A$, we can get a $\text{Ch}^{A}$
\end{itemize}
\begin{lyxcode}
\textcolor{blue}{\footnotesize{}def~a2c{[}A{]}(a:~A):~Ch{[}A{]}~=~new~Ch{[}A{]}~\{~}{\footnotesize\par}

\textcolor{blue}{\footnotesize{}~~def~run{[}X{]}(cont:~A~$\rightarrow$~X):~X~=~cont(a)}{\footnotesize\par}

\textcolor{blue}{\footnotesize{}\}}{\footnotesize\par}
\end{lyxcode}
\begin{itemize}
\item If we have a $\text{ch}:\text{Ch}^{A}$, we can get an $a:A$ 
\end{itemize}
\begin{lyxcode}
\textcolor{blue}{\footnotesize{}def~c2a{[}A{]}(ch:~Ch{[}A{]}):~A~=~ch.run{[}A{]}(a$\rightarrow$a)}{\footnotesize\par}
\end{lyxcode}
%
\end{minipage}\texttt{\textcolor{blue}{\footnotesize{}\hfill{}}}%
\begin{minipage}[t]{0.3\columnwidth}%
{\footnotesize{}
\[
\xymatrix{\xyScaleY{1pc}\xyScaleX{3pc}\text{id}:\left(A\rightarrow A\right)\ar[r]\sp(0.65){\text{ch}.\text{run}^{A}}\ar[d]\sp(0.5){\text{fmap}_{\text{Reader}_{A}}\left(f\right)} & A\ar[d]\sp(0.45){f}\\
f:\left(A\rightarrow X\right)\ar[r]\sb(0.65){\text{ch}.\text{run}^{X}} & X
}
\]
}%
\end{minipage}\texttt{\textcolor{blue}{\footnotesize{}\hfill{}}}{\footnotesize\par}

The functions \texttt{\textcolor{blue}{\footnotesize{}a2c}} and \texttt{\textcolor{blue}{\footnotesize{}c2a}}
are inverses of each other

To implement a value $\text{ch}^{:\text{Ch}^{A}}$, we must compute
an $x^{:X}$ given $f^{:A\rightarrow X}$, for \emph{any} $X$, which
\emph{requires} having a value $a^{:A}$ available

To show that \texttt{\textcolor{blue}{\footnotesize{}ch = a2c(c2a(ch))}},
apply both sides to an \texttt{\textcolor{blue}{\footnotesize{}f:~A$\rightarrow$X}}
and get \texttt{\textcolor{blue}{\footnotesize{}ch.run(f) = a2c(c2a(ch)).run(f)
= f(c2a(ch)) = f(ch.run(a$\rightarrow$a))}} 

This is naturality of \texttt{\textcolor{blue}{\footnotesize{}ch.run}}
as a transformation between \texttt{\textcolor{blue}{\footnotesize{}Reader}}
and \texttt{\textcolor{blue}{\footnotesize{}Id}} 

Naturality of \texttt{\textcolor{blue}{\footnotesize{}ch.run}} follows
from parametricity of its code

It is straightforward to compute \texttt{\textcolor{blue}{\footnotesize{}c2a(a2c(a))
= identity(a) = a}} 

Church encoding satisfies laws: it is built up from parts of \texttt{\textcolor{blue}{\footnotesize{}run}}
method

\paragraph{Worked example III: Free functor I}

The \texttt{\textcolor{blue}{\footnotesize{}Functor}} type class has
one method, \texttt{\textcolor{blue}{\footnotesize{}fmap}}: $\left(Z\rightarrow A\right)\rightarrow F^{Z}\rightarrow F^{A}$ 

The tree encoding of a free functor over $F^{\bullet}$ needs two
case classes:

\texttt{\textcolor{blue}{\footnotesize{}sealed trait FF{[}F{[}\_{]},
A{]}}}{\footnotesize\par}

\texttt{\textcolor{blue}{\footnotesize{}case class Wrap{[}F{[}\_{]},
A{]}(fa: F{[}A{]}) extends FF{[}F, A{]}}}{\footnotesize\par}

\texttt{\textcolor{blue}{\footnotesize{}case class Fmap{[}F{[}\_{]},
A, Z{]}(f: Z => A)(ffz: FF{[}F, Z{]}) extends FF{[}F, A{]}}}{\footnotesize\par}

The constructor \texttt{\textcolor{blue}{\footnotesize{}Fmap}} has
an extra type parameter $Z$, which is \textsf{``}hidden\textsf{''}

Consider a simple example of this:

\texttt{\textcolor{blue}{\footnotesize{}sealed trait Q{[}A{]}; case
class QZ{[}A, Z{]}(a: A, z: Z) extends Q{[}A{]}}}{\footnotesize\par}

Need to use specific type $Z$ when constructing a value of \texttt{\textcolor{blue}{\footnotesize{}Q{[}A{]}}},
e.g.,

\texttt{\textcolor{blue}{\footnotesize{}val q: Q{[}Int{]} = QZ{[}Int,
String{]}(123, \textquotedbl abc\textquotedbl )}}{\footnotesize\par}

The type $Z$ is hidden inside $q:Q^{\text{Int}}$; all we know is
that $Z$ \textsf{``}exists\textsf{''}

Type notation for this: $Q^{A}\triangleq\exists Z.A\times Z$

The existential quantifier applies to the \textsf{``}hidden\textsf{''} type parameter

The constructor \texttt{\textcolor{blue}{\footnotesize{}QZ}} has type
$\exists Z.\left(A\times Z\rightarrow Q^{A}\right)$

It is not $\forall Z$ because a specific $Z$ is used when building
up a value

The code does not show $\exists Z$ explicitly! We need to keep track
of that


\paragraph{Encoding with an existential type: How it works}

Show that $P^{A}\triangleq\exists Z.Z\times\left(Z\rightarrow A\right)\cong A$

\texttt{\textcolor{blue}{\footnotesize{}sealed trait P{[}A{]}; case
class PZ{[}A, Z{]}(z: Z, f: Z $\rightarrow$ A) extends P{[}A{]}}}{\footnotesize\par}

How to construct a value of type $P^{A}$ for a given $A$?

Have a function $Z\rightarrow A$ and a $Z$, construct $Z\times\left(Z\rightarrow A\right)$

Particular case: $Z\triangleq A$, have $a:A$ and build $a\times\text{id}^{:A\rightarrow A}$

\texttt{\textcolor{blue}{\footnotesize{}def a2p{[}A{]}(a: A): P{[}A{]}
= PZ{[}A, A{]}(a, identity)}}{\footnotesize\par}

Cannot extract $Z$ out of $P^{A}$ – the type $Z$ is hidden

\emph{Can} extract $A$ out of $P^{A}$ – do not need to know $Z$

\texttt{\textcolor{blue}{\footnotesize{}def p2a{[}A{]}: P{[}A{]} $\rightarrow$
A = \{ case PZ(z, f) $\rightarrow$ f(z) \}}}{\footnotesize\par}

Cannot transform $P^{A}$ into anything else other than $A$

A value of type $P^{A}$ is observable only via \texttt{\textcolor{blue}{\footnotesize{}p2a}} 

Therefore the functions \texttt{\textcolor{blue}{\footnotesize{}a2p}}
and \texttt{\textcolor{blue}{\footnotesize{}p2a}} are \textsf{``}observational\textsf{''}
inverses (i.e.~we need to use \texttt{\textcolor{blue}{\footnotesize{}p2a}}
in order to compare values of type $P^{A}$)

If $F^{\bullet}$ is a functor then $Q^{A}\triangleq\exists Z.F^{Z}\times\left(Z\rightarrow A\right)\cong F^{A}$

A value of $Q^{A}$ can be observed only by extracting an $F^{A}$
from it

Can define \texttt{\textcolor{blue}{\footnotesize{}f2q}} and \texttt{\textcolor{blue}{\footnotesize{}q2f}}
and show that they are observational inverses


\paragraph{Worked example III: Free functor II}

Tree encoding of \texttt{\textcolor{blue}{\footnotesize{}FF}} has
type $\text{FF}^{F^{\bullet},A}\triangleq F^{A}+\exists Z.\text{FF}^{F^{\bullet},Z}\times\left(Z\rightarrow A\right)$

Derivation of the reduced encoding:

A value of type $\text{FF}^{F^{\bullet},A}$ must be of the form {\footnotesize{}
\[
\exists Z_{1}.\exists Z_{2}...\exists Z_{n}.F^{Z_{n}}\times\left(Z_{n}\rightarrow Z_{n-1}\right)\times...\times\left(Z_{2}\rightarrow Z_{1}\right)\times\left(Z_{1}\rightarrow A\right)
\]
}{\footnotesize\par}

The functions $Z_{1}\rightarrow A$, $Z_{2}\rightarrow Z_{1}$, etc.,
must be composed associatively

The equivalent type is $\exists Z_{n}.F^{Z_{n}}\times\left(Z_{n}\rightarrow A\right)$

Reduced encoding: $\text{FreeF}^{F^{\bullet},A}\triangleq\exists Z.F^{Z}\times\left(Z\rightarrow A\right)$

Substituted $F^{Z}$ instead of $\text{FreeF}^{F^{\bullet},Z}$ and
eliminated the case $F^{A}$

The reduced encoding is non-recursive

Requires a proof that this encoding is equivalent to the tree encoding

If $F^{\bullet}$ is already a functor, can show $F^{A}\cong\exists Z.F^{Z}\times\left(Z\rightarrow A\right)$

Church encoding (starting from the tree encoding): $\text{FreeF}^{F^{\bullet},A}\triangleq\forall P^{\bullet}.\left(\forall C.\big(F^{C}+\exists Z.P^{Z}\times\left(Z\rightarrow C\right)\big)\leadsto P^{C}\right)\rightarrow P^{A}$

The structure of the type expression: $\forall P^{\bullet}.\left(\forall C.(...)^{C}\leadsto P^{C}\right)\rightarrow P^{A}$

Cannot move $\forall C$ or $\exists Z$ to the outside of the type
expression!


\paragraph{Church encoding IV: Recursive types and type constructors}

Consider the recursive type {\footnotesize{}$P\triangleq Z+P\times P$}
(tree with $Z$-valued leaves)

The Church encoding is {\footnotesize{}$\forall X.\left(\left(Z+X\times X\right)\rightarrow X\right)\rightarrow X$}{\footnotesize\par}

This is \emph{non-recursive}: the inductive use of $P$ is replaced
by $X$

Generalize to recursive type $P\triangleq S^{P}$ where $S^{\bullet}$
is a \textsf{``}induction functor\textsf{''}:

The Church encoding of $P$ is {\footnotesize{}$\forall X.\left(S^{X}\rightarrow X\right)\rightarrow X$}{\footnotesize\par}

Church encoding of recursive types is non-recursive

Example: Church encoding of \texttt{\textcolor{blue}{\footnotesize{}List{[}Int{]}}} 

Church encoding of a type constructor $P^{\bullet}$:

Notation: $P^{\bullet}$ is a type function; Scala syntax is \texttt{\textcolor{blue}{\footnotesize{}P{[}\_{]}}} 

The Church encoding is {\footnotesize{}$\text{Ch}^{P^{\bullet},A}=\forall F^{\bullet}.\left(\forall X.P^{X}\rightarrow F^{X}\right)\rightarrow F^{A}$}{\footnotesize\par}

Note: $\forall X.P^{X}\rightarrow F^{X}$ or $P^{\bullet}\leadsto F^{\bullet}$
resembles a natural transformation

Except that $P^{\bullet}$ and $F^{\bullet}$ are not necessarily
functors, so no naturality law

Example: Church encoding of \texttt{\textcolor{blue}{\footnotesize{}Option{[}\_{]}}} 

Church encoding of a \emph{recursively} defined type constructor $P^{\bullet}$:

Definition: $P^{A}\triangleq S^{P^{\bullet},A}$ where $S^{P^{\bullet},A}$
describes the \textsf{``}induction principle\textsf{''}

Notation: {\footnotesize{}$S^{\bullet^{\bullet},A}$} is a higher-order
type function; Scala syntax: \texttt{\textcolor{blue}{\footnotesize{}S{[}\_{[}\_{]},A{]}}} 

Example: $\text{List}^{A}\triangleq1+A\times\text{List}^{A}\triangleq S^{\text{List}^{\bullet},A}$
where $S^{P^{\bullet},A}\triangleq1+A\times P^{A}$ 

The Church encoding of $P^{A}$ is {\footnotesize{}$\text{Ch}^{P^{\bullet},A}=\forall F^{\bullet}.\big(S^{F^{\bullet}}\leadsto F^{\bullet}\big)\rightarrow F^{A}$}{\footnotesize\par}

The Church encoding of \texttt{\textcolor{blue}{\footnotesize{}List{[}\_{]}}}
is non-recursive

\paragraph{Details: Why Church encoding of a free semigroup is a semigroup}

- it\textsf{'}s not obvious

FS = forall S. (Z => S) \texttimes{} (S \texttimes{} S => S) => S
is a semigroup.  We need to define the binary operation |+| on values
of type FS. A value f of type FS is a function with a type parameter,
that we can use as f{[}S{]}(e, c) to compute a value of any given
type S from arguments e : Z => S and c: S \texttimes{} S => S. Scala
code for f will be

def f{[}S{]}(empty: Z => S, combine: (S, S) => S): S = ???

So, given f and g of this type, we need to somehow define a new function
h = f |+| g also of the same type. Begin to write code for that function:

def h{[}S{]}(empty: Z => S, combine: (S, S) => S): S = ???

The free semigroup in the tree encoding defines the binary operation
as a formal operation that does not compute anything. In the Church
encoding, however, we have the binary operation as the argument \textquotedbl combine\textquotedbl{}
of h, and so we should call that function. So we use it:

def h{[}S{]}(empty: Z => S, combine: (S, S) => S): S = combine(???,
???)

We need to fill the typed holes ??? of type S. It is clear that we
should use f and g somehow. We can use f and g simply by calling those
functions on the arguments `empty` and `combine`. Since f and g both
have a universally quantified type parameter, we can just use the
given type S for them.

def h{[}S{]}(empty: Z => S, combine: (S, S) => S): S = combine(f{[}S{]}(empty,
combine), g{[}S{]}(empty, combine))

The types match, and we have used both functions f and g in a way
that is intuitively correct. We have preserved information. So, this
is likely the correct implementation. It remains to verify the associativity
law. To do that, we need to assume that `combine` is associative for
the actual type $S$ on which we use the Church encoding (i.e. a non-free,
lawful semigroup)

To show equivalence between FSZ and Chz, write code for the type ChZ
and the two directions of the isomorphism,\inputencoding{latin9}
\begin{lstlisting}
trait FSCh[Z] { def run[S](empty: Z => S, combine: (S, S) => S): S }
def fsz2ch[Z](fsz: NEList[Z]): FSCh[Z] = ???
def ch2fsz[Z](ch: FSCh[Z]): NEList[Z] = ???
\end{lstlisting}
\inputencoding{utf8}

\paragraph{Church encoding V: Type classes}

Look at the Church encoding of the free semigroup:{\footnotesize{}
\[
\text{ChFS}^{Z}\triangleq\forall X.\left(Z\rightarrow X\right)\times\left(X\times X\rightarrow X\right)\rightarrow X
\]
}{\footnotesize\par}

If $X$ is constrained to the \texttt{\textcolor{blue}{\footnotesize{}Semigroup}}
typeclass, we will already have a value {\footnotesize{}$X\times X\rightarrow X$},
so we can omit it: {\footnotesize{}$\text{ChFS}^{Z}=\forall X^{:\text{Semigroup}}.\left(Z\rightarrow X\right)\rightarrow X$}{\footnotesize\par}

The \textsf{``}induction functor\textsf{''} for \textsf{``}semigroup over $Z$\textsf{''} is {\footnotesize{}$\text{SemiG}^{X}\triangleq Z+X\times X$}{\footnotesize\par}

So the Church encoding is $\forall X.\big(\text{SemiG}^{X}\rightarrow X\big)\rightarrow X$

Generalize to arbitrary type classes:

Type class $C$ is defined by its operations{\footnotesize{} $C^{X}\rightarrow X$}
(with a suitable $C^{\bullet}$)

call $C^{\bullet}$ the \textbf{structure functor} of the inductive
typeclass $C$

Tree encoding of \textsf{``}free $C$ over $Z$\textsf{''} is recursive, $\text{FreeC}^{Z}\triangleq Z+C^{\text{FreeC}^{Z}}$

Church encoding is $\text{FreeC}^{Z}\triangleq\forall X.\left(Z+C^{X}\rightarrow X\right)\rightarrow X$

Equivalently, $\text{FreeC}^{Z}\triangleq\forall X^{:C}.\left(Z\rightarrow X\right)\rightarrow X$

Laws of the typeclass are satisfied automatically after \textsf{``}running\textsf{''}

Works similarly for type constructors: operations $C^{P^{\bullet},A}\rightarrow P^{A}$

Free typeclass $C$ over $F^{\bullet}$ is $\text{FreeC}^{F^{\bullet},A}\triangleq\forall P^{\bullet:C}.\left(F^{\bullet}\leadsto P^{\bullet}\right)\rightarrow P^{A}$

\paragraph{Properties of free type constructions}

Generalizing from our examples so far:

We \textsf{``}enriched\textsf{''} $Z$ to a monoid $\text{FM}^{Z}$, and $F^{A}$
to a monad $\text{DSL}^{F,A}$ 

The \textsf{``}enrichment\textsf{''} adds case classes representing the needed operations

Works for a generating type $Z$ and for a generating type constructor
$F^{A}$

Obtain a \textbf{free type construction}, which performs no computations

$\text{FM}^{Z}$ wraps $Z$ in \textsf{``}just enough\textsf{''} stuff to make it
look like a monoid

$\text{FreeF}^{F^{\bullet},A}$ wraps $F^{A}$ in \textsf{``}just enough\textsf{''}
stuff to make it look like a functor

A value of a free construction can be \textsf{``}run\textsf{''} to yield non-free
values 

Questions:

Can we construct a free typeclass $C$ over any type constructor $F^{A}$?

Yes, with typeclasses: (contra)functor, filterable, monad, applicative

Which of the possible encodings to use?

Tree encoding, reduced encodings, Church encoding

What are the laws for the{\footnotesize{} $\text{FreeC}^{F,A}$} –
\textsf{``}free instance of $C$ over $F$\textsf{''}?

For all $F^{\bullet}$, must have \texttt{\textcolor{blue}{\footnotesize{}wrap{[}A{]}}}
$:F^{A}\rightarrow\text{FreeC}^{F,A}$ or $F^{\bullet}\leadsto\text{FreeC}^{F,\bullet}$

For all $M^{\bullet}:C$, must have \texttt{\textcolor{blue}{\footnotesize{}run}}
$:\left(F^{\bullet}\leadsto M^{\bullet}\right)\rightarrow\text{FreeC}^{F,\bullet}\leadsto M^{\bullet}$

The laws of typeclass $C$ must hold after interpreting into an $M^{\bullet}:C$

Given any \texttt{\textcolor{blue}{\footnotesize{}t}}$:F^{\bullet}\leadsto G^{\bullet}$,
must have \texttt{\textcolor{blue}{\footnotesize{}fmap(t)}}$:\text{FreeC}^{F,\bullet}\leadsto\text{FreeC}^{G,\bullet}$


\paragraph{Recipes for encoding free typeclass instances}

Build a free instance of typeclass $C$ over $F^{\bullet}$, as a
type constructor $P^{\bullet}$ 

The typeclass $C$ can be functor, contrafunctor, monad, etc.

Assume that $C$ has methods $m_{1}$, $m_{2}$, ..., with type signatures
{\footnotesize{}$m_{1}:Q_{1}^{P^{\bullet},A}\rightarrow P^{A}$},
{\footnotesize{}$m_{2}:Q_{2}^{P^{\bullet},A}\rightarrow P^{A}$},
etc., where $Q_{i}^{P^{\bullet},A}$ are covariant in $P^{\bullet}$ 

\textbf{Inductive typeclass} is defined via a methods functor, $S^{P^{\bullet}}\leadsto P^{\bullet}$

The tree encoded $\text{FC}^{A}$ is a disjunction defined recursively
by{\footnotesize{}
\[
\text{FC}^{A}\triangleq F^{A}+Q_{1}^{\text{FC}^{\bullet},A}+Q_{2}^{\text{FC}^{\bullet},A}+...
\]
}{\footnotesize\par}

\texttt{\textcolor{blue}{\footnotesize{}sealed trait FC{[}A{]}; case
class Wrap{[}A{]}(fa: F{[}A{]}) extends FC{[}A{]}}}{\footnotesize\par}

\texttt{\textcolor{blue}{\footnotesize{}case class Q1{[}A{]}(...)
extends FC{[}A{]}}}{\footnotesize\par}

\texttt{\textcolor{blue}{\footnotesize{}case class Q2{[}A{]}(...)
extends FC{[}A{]}; ...}}{\footnotesize\par}

Any type parameters within $Q_{i}$ are then existentially quantified

\texttt{\textcolor{blue}{\footnotesize{}run()}} maps $F^{\bullet}\leadsto M^{\bullet}$
in the disjunction and recursively for other parts

Derive a reduced encoding via reasoning about possible values of $\text{FC}^{A}$
and by taking into account the laws of the typeclass $C$

A Church encoding can use the tree encoding or the reduced encoding

Church encoding is \textsf{``}automatically reduced\textsf{''}, but performance may
differ


\paragraph{Properties of inductive typeclasses}

If a typeclass $C$ is inductive with methods $C^{X}\rightarrow X$
then:

A free instance of $C$ over $Z$ can be tree-encoded as {\footnotesize{}$\text{FreeC}^{Z}\triangleq Z+C^{\text{FreeC}^{Z}}$} 

All inductive typeclasses have free instances, $\text{FreeC}^{Z}$

If $P^{:C}$ and $Q^{:C}$ then $P\times Q$ and $Z\rightarrow P$
also belong to typeclass $C$

but not necessarily $P+Q$ or $Z\times P$

Proof: can implement $(C^{P}\rightarrow P)\times(C^{Q}\rightarrow Q)\rightarrow C^{P\times Q}\rightarrow P\times Q$
and $\left(C^{P}\rightarrow P\right)\rightarrow C^{Z\rightarrow P}\rightarrow Z\rightarrow P$,
but cannot implement $\left(...\right)\rightarrow P+Q$

Analogous properties hold for type constructor typeclasses

Methods described as $C^{F^{\bullet},A}\rightarrow F^{A}$ with type
constructor parameter $F^{\bullet}$

What typeclasses \emph{cannot} be tree-encoded (or have no \textsf{``}free\textsf{''}
instances)?

Any typeclass with a method \emph{not ultimately returning} a value
of $P^{A}$

Example: a typeclass with methods $\text{pt}:A\rightarrow P^{A}$
and $\text{ex}:P^{A}\rightarrow A$

Such typeclasses are not inductive nor co-inductive

Typeclasses with methods of the form $P^{A}\rightarrow...$ are \textbf{co-inductive}


\paragraph{Worked example V: Free pointed functor}

Over an arbitrary type constructor $F^{\bullet}$:

Pointed functor methods {\footnotesize{}$\text{pt}:A\rightarrow P^{A}$}
and {\footnotesize{}$\text{map}:P^{A}\times\left(A\rightarrow B\right)\rightarrow P^{B}$}{\footnotesize\par}

Tree encoding: {\footnotesize{}$\text{FreeP}^{F^{\bullet},A}\triangleq A+F^{A}+\exists Z.\text{FreeP}^{F^{\bullet},Z}\times\left(Z\rightarrow A\right)$}{\footnotesize\par}

Derivation of the reduced encoding:

The tree encoding of a value $\text{FreeP}^{F^{\bullet},A}$ is either{\footnotesize{}
\[
\exists Z_{1}.\exists Z_{2}...\exists Z_{n}.F^{Z_{n}}\times\left(Z_{n}\rightarrow Z_{n-1}\right)\times...\times\left(Z_{2}\rightarrow Z_{1}\right)\times\left(Z_{1}\rightarrow A\right)
\]
}or{\footnotesize{}
\[
\exists Z_{1}.\exists Z_{2}...\exists Z_{n}.Z_{n}\times\left(Z_{n}\rightarrow Z_{n-1}\right)\times...\times\left(Z_{2}\rightarrow Z_{1}\right)\times\left(Z_{1}\rightarrow A\right)
\]
}{\footnotesize\par}

Compose all functions by associativity; one function $Z_{n}\rightarrow A$
remains

The case $\exists Z_{n}.Z_{n}\times\left(Z_{n}\rightarrow A\right)$
is equivalent to just $A$

Reduced encoding: {\footnotesize{}$\text{FreeP}^{F^{\bullet},A}\triangleq A+\exists Z.F^{Z}\times\left(Z\rightarrow A\right)$,
}non-recursive

This reuses the free functor as $\text{FreeP}^{F^{\bullet},A}=A+\text{FreeF}^{F^{\bullet},A}$

If the type constructor $F^{\bullet}$ is \emph{already} a functor,
$\text{FreeF}^{F^{\bullet},A}\cong F^{A}$ and so:

Free pointed functor over a functor $F^{\bullet}$ is simplified:
$A+F^{A}$

If $F^{\bullet}$ is already a pointed functor, need not use the free
construction

If we do, we will have $\text{FreeP}^{F^{\bullet},A}\not\cong F^{A}$ 

only functors and contrafunctors do not change under \textsf{``}free\textsf{''}


\paragraph{Worked example VI: Free filterable functor}

(See Chapter 6.) Methods:
\begin{align*}
\text{map} & :F^{A}\rightarrow\left(A\rightarrow B\right)\rightarrow F^{B}\\
\text{mapOpt} & :F^{A}\rightarrow\left(A\rightarrow1+B\right)\rightarrow F^{B}
\end{align*}

We can recover \texttt{\textcolor{blue}{\footnotesize{}map}} from
\texttt{\textcolor{blue}{\footnotesize{}mapOpt}}, so we keep only
\texttt{\textcolor{blue}{\footnotesize{}mapOpt}} 

Tree encoding: $\text{FreeFi}^{F^{\bullet},A}\triangleq F^{A}+\exists Z.\text{FreeFi}^{F^{\bullet},Z}\times\left(Z\rightarrow1+A\right)$

If $F^{\bullet}$ is already a functor, can simplify the tree encoding
using the identity $\exists Z.P^{Z}\times\left(Z\rightarrow1+A\right)\cong P^{A}$
and obtain $\text{FreeFi}^{F^{\bullet},A}\triangleq F^{A}+\text{FreeFi}^{F^{\bullet},1+A}$,
which is equivalent to $\text{FreeFi}^{F^{\bullet},A}=F^{A}+F^{1+A}+F^{1+1+A}+...$

Reduced encoding: $\text{FreeFi}^{F^{\bullet},A}\triangleq\exists Z.F^{Z}\times\left(Z\rightarrow1+A\right)$,
non-recursive

Derivation: $\exists Z_{1}...\exists Z_{n}.F^{Z_{n}}\times\left(Z_{n}\rightarrow1+Z_{n-1}\right)\times...\times\left(Z_{1}\rightarrow1+A\right)$
is simplified using the laws of \texttt{\textcolor{blue}{\footnotesize{}mapOpt}}
and Kleisli composition, and yields $\exists Z_{n}.F^{Z_{n}}\times\left(Z_{n}\rightarrow1+A\right)$.
Encode $F^{A}$ as $\exists Z.F^{Z}\times\left(Z\rightarrow0+Z\right)$.

If $F^{\bullet}$ is already a functor, the reduced encoding is $\text{FreeFi}^{F^{\bullet},A}=F^{1+A}$

Free filterable over a filterable functor $F^{\bullet}$ is not equivalent
to $F^{\bullet}$

Free filterable contrafunctor is constructed in a similar way


\paragraph{Worked example VII: Free monad}


\paragraph{Worked example VIII: Free applicative functor}

Methods:
\begin{align*}
\text{pure} & :A\rightarrow F^{A}\\
\text{ap} & :F^{A}\rightarrow F^{A\rightarrow B}\rightarrow F^{B}
\end{align*}

We can recover \texttt{\textcolor{blue}{\footnotesize{}map}} from
\texttt{\textcolor{blue}{\footnotesize{}ap}} and \texttt{\textcolor{blue}{\footnotesize{}pure}},
so we omit \texttt{\textcolor{blue}{\footnotesize{}map}} 

Tree encoding: {\footnotesize{}$\text{FreeAp}^{F^{\bullet},A}\triangleq F^{A}+A+\exists Z.\text{FreeAp}^{F^{\bullet},Z}\times\text{FreeAp}^{F^{\bullet},Z\rightarrow A}$}{\footnotesize\par}

Reduced encoding:{\footnotesize{} $\text{FreeAp}^{F^{\bullet},A}\triangleq A+\exists Z.F^{Z}\times\text{FreeAp}^{F^{\bullet},Z\rightarrow A}$}{\footnotesize\par}

Derivation: a $\text{FreeAp}^{A}$ is either $\exists Z_{1}...\exists Z_{n}.Z_{1}\times\text{FreeAp}^{Z_{1}\rightarrow Z_{2}}\times...$
or $\exists Z_{1}...\exists Z_{n}.F^{Z_{1}}\times\text{FreeAp}^{Z_{1}\rightarrow Z_{2}}\times...$;
encode $Z_{1}\times\text{FreeAp}^{Z_{1}\rightarrow Z_{2}}$ equivalently
as $\text{FreeAp}^{Z_{1}\rightarrow Z_{2}}\times\left(\left(Z_{1}\rightarrow Z_{2}\right)\rightarrow Z_{2}\right)$
using the identity law; so the first $\text{FreeAp}^{Z}$ is always
$F^{A}$, or we have a pure value 

Free applicative over a functor $F^{\bullet}$: 
\begin{align*}
\text{FreeAp}^{F^{\bullet},A} & \triangleq A+\text{FreeZ}^{F^{\bullet},A}\\
\text{FreeZ}^{F^{\bullet},A} & \triangleq F^{A}+\exists Z.F^{Z}\times\text{FreeZ}^{F^{\bullet},Z\rightarrow A}
\end{align*}

$\text{FreeZ}^{F^{\bullet},\bullet}$ is the reduced encoding of \textsf{``}free
zippable\textsf{''} (no \texttt{\textcolor{blue}{\footnotesize{}pure}})

$\text{FreeAp}^{F^{\bullet},\bullet}$ over an applicative functor
$F^{\bullet}$ is not equivalent to $F^{\bullet}$


\paragraph{Laws for free typeclass constructions}

Consider an inductive typeclass $C$ with methods $C^{A}\rightarrow A$

Define a free instance of $C$ over $Z$ recursively, {\footnotesize{}$\text{FreeC}^{Z}\triangleq Z+C^{\text{FreeC}^{Z}}$}{\footnotesize\par}

$\text{FreeC}^{Z}$ has an instance of $C$, i.e.~we can implement
$C^{\text{FreeC}^{Z}}\rightarrow\text{FreeC}^{Z}$

$\text{FreeC}^{Z}$ is a functor in $Z$; {\footnotesize{}$\text{fmap}_{\text{FreeC}}:\left(Y\rightarrow Z\right)\rightarrow\text{FreeC}^{Y}\rightarrow\text{FreeC}^{Z}$}{\footnotesize\par}

{\footnotesize{}\vspace{-0.45cm}}%
\begin{minipage}[t]{0.64\columnwidth}%
\begin{itemize}
\item For a $P^{:C}$ we can implement the functions {\footnotesize{}
\begin{align*}
\text{run}^{P} & :\left(Z\rightarrow P\right)\rightarrow\text{FreeC}^{Z}\rightarrow P\\
\text{wrap} & :Z\rightarrow\text{FreeC}^{Z}
\end{align*}
}
\end{itemize}
%
\end{minipage}{\footnotesize{}}%
\begin{minipage}[t]{0.36\columnwidth}%
{\footnotesize{}}{\footnotesize{}
\[
\xymatrix{\xyScaleY{1.5pc}\xyScaleX{5pc}\text{FreeC}^{Y}\ar[d]\sb(0.45){\text{fmap}\,f^{:Y\rightarrow Z}}\ar[rd]\sp(0.65){\ \text{run}\left(f\bef g\right)}\\
\text{FreeC}^{Z}\ar[r]\sp(0.5){\text{run}(g^{:Z\rightarrow P})} & P
}
\]
}%
\end{minipage}\hfill{}

Law 1: {\footnotesize{}$\text{run}\left(\text{wrap}\right)=\text{id}$};
law 2: {\footnotesize{}$\text{fmap}\,f\bef\text{run}\,g=\text{run}\left(f\bef g\right)$}
(naturality of \texttt{\textcolor{blue}{\footnotesize{}run}})

For any $P^{:C},Q^{:C},g^{:Z\rightarrow P}$, and a typeclass-preserving
$f^{:P\rightarrow Q}$, we have{\footnotesize{}
\[
\text{run}^{P}(g)\bef f=\text{run}^{Q}\left(g\bef f\right)\quad\quad\text{– \textsf{``}universal property\textsf{''} of }\text{run}
\]
}{\footnotesize{}
\[
\xymatrix{\xyScaleY{2.0pc}\xyScaleX{3pc}\text{FreeC}^{Z}\ar[d]\sb(0.4){\text{run}^{P}(g^{:Z\rightarrow P})}\ar[rd]\sp(0.55){\quad\text{run}^{Q}(g\bef f)} &  &  & C^{P}\ar[d]\sb(0.4){\text{fmap}_{S}f}\ar[r]\sp(0.5){\text{ops}_{P}} & P\ar[d]\sb(0.4){f}\\
P\ar[r]\sp(0.5){f^{:P\rightarrow Q}} & Q &  & C^{Q}\ar[r]\sp(0.5){\text{ops}_{Q}} & Q
}
\]
}{\footnotesize\par}

$f^{:P\rightarrow Q}$ \textbf{preserves typeclass} $C$ if the diagram
on the right commutes


\paragraph{Combining the generating constructors in a free typeclass}

Consider $\text{FreeC}^{Z}$ for an inductive typeclass $C$ with
methods $C^{X}\rightarrow X$

We would like to combine generating constructors $Z_{1}$, $Z_{2}$,
etc.

In a monadic DSL – combine different operations defined separately

Note: monads do not compose in general

To combine generators, use $\text{FreeC}^{Z_{1}+Z_{2}}$; an \textsf{``}instance
over $Z_{1}$ and $Z_{2}$\textsf{''}

but need to inject parts into disjunction, which is cumbersome

Church encoding makes this easier to manage:

{\footnotesize{}$\text{FreeC}^{Z}\triangleq\forall X.\left(Z\rightarrow X\right)\times\big(C^{X}\rightarrow X\big)\rightarrow X$}
and then {\footnotesize{}
\[
\text{FreeC}^{Z_{1}+Z_{2}}\triangleq\forall X.\left(Z_{1}\rightarrow X\right)\times\left(Z_{2}\rightarrow X\right)\times\big(C^{X}\rightarrow X\big)\rightarrow X
\]
}{\footnotesize\par}

Encode the functions $Z_{i}\rightarrow X$ via typeclasses \texttt{\textcolor{blue}{\footnotesize{}ExZ1}},
\texttt{\textcolor{blue}{\footnotesize{}ExZ2}}, etc., where typeclass
\texttt{\textcolor{blue}{\footnotesize{}ExZ1}} has method $Z_{1}\rightarrow X$,
etc.

Then {\footnotesize{}
\[
\text{FreeC}^{Z_{1}+Z_{2}}=\forall X^{:E_{Z_{1}}:E_{Z_{2}}}.\big(C^{X}\rightarrow X\big)\rightarrow X
\]
}or equivalently{\footnotesize{}
\[
\text{FreeC}^{Z_{1}+Z_{2}}=\forall X^{:C~:E_{Z_{1}}:E_{Z_{2}}}.X
\]
}{\footnotesize\par}

The code is easier to maintain

This works for all typeclasses $C$ and any number of generators $Z_{i}$

\paragraph{Combining different free typeclasses}

To combine free instances of different typeclasses $C_{1}$ and $C_{2}$:

Option 1: use functor composition, $\text{FreeC}_{12}^{Z}\triangleq\text{FreeC}_{1}^{\text{FreeC}_{2}^{Z}}$

Order of composition matters!

Operations of $C_{2}$ need to be lifted into $C_{1}$

Works only for inductive typeclasses

Encodes $C_{1}^{C_{2}}$ but not $C_{2}^{C_{1}}$

Option 2: use disjunction of method functors, $C^{X}\triangleq C_{1}^{X}+C_{2}^{X}$,
and build the free typeclass instance using $C^{X}$

Church encoding: $\text{FreeC}_{12}^{Z}\triangleq\forall X.\left(Z\rightarrow X\right)\times\big(C_{1}^{X}+C_{2}^{X}\rightarrow X\big)\rightarrow X$

Example 1: $C_{1}$ is functor, $C_{2}$ is contrafunctor

Interpret a free functor/contrafunctor into a profunctor

Example 2: $C_{1}$ is monad, $C_{2}$ is applicative functor

Interpret into a monad that has a non-standard \texttt{\textcolor{blue}{\footnotesize{}zip}}
implementation

Example: interpret into \texttt{\textcolor{blue}{\footnotesize{}Future}}
and convert \texttt{\textcolor{blue}{\footnotesize{}zip}} into parallel
execution

Each \texttt{\textcolor{blue}{\footnotesize{}zip}} creates parallel
branch, each \texttt{\textcolor{blue}{\footnotesize{}flatMap}} creates
sequential chain


\subsection{Exercises}

\subsubsection{Exercise \label{subsec:Exercise-free-monad-example}\ref{subsec:Exercise-free-monad-example}\index{exercises}}

The \textbf{interactive input-output}\index{monads!interactive input-output monad}
monad is defined recursively by:\inputencoding{latin9}
\begin{lstlisting}
sealed trait TIO[A]
final case class Pure[A](a: A) extends TIO[A]
final case class Read[A](read: P => TIO[A]) extends TIO[A]
final case class Write[A](output: Q, next: TIO[A]) extends TIO[A]
\end{lstlisting}
\inputencoding{utf8}In the type notation, this is written as:
\[
\text{TIO}^{A}\triangleq A+(P\rightarrow\text{TIO}^{A})+Q\times\text{TIO}^{A}\quad,
\]
Here $P$ and $Q$ are fixed types. The monad \inputencoding{latin9}\lstinline!TIO!\inputencoding{utf8}
represents computations that may consume an input value of type $P$
or produce an output value of type $Q$. Use the free monad construction
to show that \inputencoding{latin9}\lstinline!TIO!\inputencoding{utf8}
is a lawful monad. Implement a monad instance for \inputencoding{latin9}\lstinline!TIO!\inputencoding{utf8}.

\subsubsection{Exercise \label{subsec:Exercise-free-type-1}\ref{subsec:Exercise-free-type-1}}

Implement a free semigroup on a type $Z$ in the tree encoding and
in the reduced encoding. Show that the semigroup laws hold for the
reduced encoding but \emph{not} for the tree encoding. Show that the
laws hold for the tree encoding after interpreting into a lawful semigroup
$S$.

\subsubsection{Exercise \label{subsec:Exercise-free-type-2}\ref{subsec:Exercise-free-type-2}}

For a fixed monoid $L$, define a typeclass $\text{Mod}_{L}$ (called
\textsf{``}$L$-module\textsf{''}). Type $P$ is an $L$-module if the monoid $L$
\textsf{``}acts\textsf{''} on $P$ via a function act$:L\rightarrow P\rightarrow P$,
with laws $\text{act}\,x\bef\text{act}\,y=\text{act}\left(x\bef y\right)$
and $\text{act}\left(e_{L}\right)=\text{id}$. - Monoid morphism between
$L$ and $\text{MF}^{P}$. Show that $\text{Mod}_{L}$ is an inductive
typeclass. Implement a free $L$-module on a type $Z$. 

\subsubsection{Exercise \label{subsec:Exercise-free-type-3}\ref{subsec:Exercise-free-type-3}}

\textbf{(a)} Implement a monadic DSL with operations \inputencoding{latin9}\lstinline!put: A => Unit!\inputencoding{utf8}
and \inputencoding{latin9}\lstinline!get: Unit => A!\inputencoding{utf8}.
These operations should store and retrieve a state value of type \inputencoding{latin9}\lstinline!A!\inputencoding{utf8}.
Test on some example programs written in that DSL. 

\textbf{(b)} Implement a monadic DSL with operations \inputencoding{latin9}\lstinline!put: A => Unit!\inputencoding{utf8},
\inputencoding{latin9}\lstinline!get: Unit => Option[A]!\inputencoding{utf8},
and \inputencoding{latin9}\lstinline!clear: Unit => Unit!\inputencoding{utf8}.
These operations should store and retrieve a state value of type \inputencoding{latin9}\lstinline!A!\inputencoding{utf8}.
Running \inputencoding{latin9}\lstinline!clear!\inputencoding{utf8}
should delete the state value. When there is no state value, \inputencoding{latin9}\lstinline!get!\inputencoding{utf8}
should return \inputencoding{latin9}\lstinline!None!\inputencoding{utf8}.
Test on some example programs.

\subsubsection{Exercise \label{subsec:Exercise-free-type-4}\ref{subsec:Exercise-free-type-4}}

Implement the Church encoding of the type constructor $P^{A}\triangleq\text{Int}+A\times A$.
For the resulting type constructor, implement a \inputencoding{latin9}\lstinline!Functor!\inputencoding{utf8}
instance.

\subsubsection{Exercise \label{subsec:Exercise-free-type-5}\ref{subsec:Exercise-free-type-5}}

Describe the monoid type class via a method functor $C^{\bullet}$
(such that the monoid\textsf{'}s operations are combined into the type $S^{M}\rightarrow M$).
Using $S^{\bullet}$, implement the free monoid on a type $Z$ in
the Church encoding.

\subsubsection{Exercise \label{subsec:Exercise-free-type-6}\ref{subsec:Exercise-free-type-6}}

Assuming that $F^{\bullet}$ is a functor, define $Q^{A}\triangleq\exists Z.F^{Z}\times\left(Z\rightarrow A\right)$
and implement \inputencoding{latin9}\lstinline!f2q!\inputencoding{utf8}$:F^{A}\rightarrow Q^{A}$
and \inputencoding{latin9}\lstinline!q2f!\inputencoding{utf8}$:Q^{A}\rightarrow F^{A}$.
Show that these functions are natural transformations, and that they
are inverses of each other \textsf{``}observationally\textsf{''}, i.e., after applying
\inputencoding{latin9}\lstinline!q2f!\inputencoding{utf8} in order
to compare values of $Q^{A}$.

\subsubsection{Exercise \label{subsec:Exercise-free-type-7}\ref{subsec:Exercise-free-type-7}}

Prove the following type equivalences involving quantified types:

\textbf{(a)} $\forall A.\,A\times A\cong\bbnum 0\quad.$

\textbf{(b)} $\forall A.\,\left(A\times A\times A\rightarrow A\right)\cong\bbnum 1+\bbnum 1+\bbnum 1\quad.$

\textbf{(c)} $\exists Z.\,Z\cong\bbnum 1\quad.$

\textbf{(d)} $\exists Z.\,Z\times\left(A\rightarrow Z\right)\times\left(Z\rightarrow B\right)\cong B\times\left(A\rightarrow B\right)\quad.$

\subsubsection{Exercise \label{subsec:Exercise-free-type-9-1}\ref{subsec:Exercise-free-type-9-1}{*}}

Prove the following type equivalences involving quantified type constructors
(the types $A$, $B$, $C$, $D$ are fixed):

\textbf{(a)} $\forall F^{\bullet}.\,F^{A}\cong\bbnum 0\quad.$

\textbf{(b)} $\forall F^{\bullet}.\,F^{A}\rightarrow B\cong B\quad.$

\textbf{(c)} $\forall F^{\bullet}.\,F^{A}\rightarrow F^{B}\cong A\rightarrow B\quad.$

\textbf{(d)} $\forall F^{\bullet}.\,\left(A\rightarrow F^{B}\right)\rightarrow C+F^{D}\cong C+A\times\left(B\rightarrow D\right)\quad.$

\textbf{(e)} $\forall F^{\bullet}.\,\left(A\rightarrow F^{B}\right)\rightarrow C\rightarrow F^{D}\cong\left(C\rightarrow A\right)\times\left(C\times B\rightarrow D\right)\quad.$

\subsubsection{Exercise \label{subsec:Exercise-free-type-8}\ref{subsec:Exercise-free-type-8}}

Derive a reduced encoding for a free applicative functor on a pointed
functor.

\subsubsection{Exercise \label{subsec:Exercise-free-type-9}\ref{subsec:Exercise-free-type-9}}

Implement a \textsf{``}free pointed filterable\textsf{''} typeclass (combining pointed
and filterable) on a type constructor $F^{\bullet}$ in the tree encoding.
Derive a reduced encoding. Simplify these encodings when $F^{\bullet}$
is already a functor.

\paragraph{Corrections}

The slides say that the \textsf{``}universal property\textsf{''} of the runner is
$\text{run}^{P}g\bef f=\text{run}^{Q}\left(g\bef f\right)$, however,
this is not true; it is the right naturality property of $\text{run}^{P}:\left(Z\rightarrow P\right)\rightarrow\text{FreeC}^{Z}\rightarrow P$
with respect to the type parameter $P$. The universal property is
$f=\text{wrap}\bef\text{run}^{P}f$ for any $f:Z\rightarrow P$ and
any type $P$ that belongs to the typeclass $C$.

The \textsf{``}logarithm\textsf{''} $\text{Lg}\,(F^{\bullet})\triangleq\forall A.\,F^{A}\rightarrow A$
is an operation with bizarre properties. Examples: $\forall A.\,\left(Z\rightarrow A\right)\rightarrow A\cong Z$,
so $\text{Lg}\,(Z\rightarrow\bullet)=Z$. This might motivate the
name \textsf{``}logarithm\textsf{''}. But $\text{Lg}\,(F^{\bullet}+G^{\bullet})=\text{Lg}\,(F)\times\text{Lg}\,(G)$,
which resembles the distributive law for the \emph{exponential} function
rather than for the logarithm. Also, $\forall A.\,(Z\times A\times A)\rightarrow A\cong Z\times\bbnum 2$,
so $\text{Lg}\,(Z\times(\bbnum 2\rightarrow\bullet))=Z\times\bbnum 2$.
However, for a constant functor, $\text{Lg}\,(Z)=\bbnum 0$. This
shows that $\text{Lg}\,(F^{\bullet}\times G^{\bullet})\not\cong\text{Lg}\,(F)\times\text{Lg}\,(G)$.
We also have $\text{Lg}\,(\text{Opt})=\bbnum 0$.

\section{Properties of free constructions}

\subsection{Free monad}

The free monad on a functor $F$ is defined by
\[
\text{Free}^{F,A}\triangleq A+F^{\text{Free}^{F,A}}\quad.
\]
It was shown in Statement~\ref{subsec:Statement-monad-construction-4-free-monad}
that $\text{Free}^{F,A}$ is a lawful monad for any functor $F$.
We will now derive some further properties of the free monad construction.

The next statement shows that one can change the underlying functor
$F$ while preserving the free monad operations.

\subsubsection{Statement \label{subsec:Statement-free-monad-monadic-naturality}\ref{subsec:Statement-free-monad-monadic-naturality}}

For any functor $G$ and any natural transformation $\phi:F^{A}\rightarrow G^{A}$,
the corresponding transformation $\psi(\phi):\text{Free}^{F,A}\rightarrow\text{Free}^{G,A}$
defined by:
\[
\psi(\phi):\text{Free}^{F,A}\rightarrow\text{Free}^{G,A}\quad,\quad\quad\psi\triangleq\,\begin{array}{|c||cc|}
 & A & G^{\text{Free}^{G,A}}\\
\hline A & \text{id} & \bbnum 0\\
F^{\text{Free}^{F,A}} & \bbnum 0 & \overline{\psi}^{\uparrow F}\bef\phi
\end{array}
\]
is a monad morphism. In other words, the free monad $\text{Free}^{F}$
is natural in the functor $F$. 

\subparagraph{Proof}

Since $\phi$ is fixed, we can write $\psi(\phi)$ as simply $\psi$
for brevity. Denote $P\triangleq\text{Free}^{F}$ and $Q\triangleq\text{Free}^{G}$;
we need to show that $\psi:P\leadsto Q$ is a monad morphism.

To verify the identity law:
\begin{align*}
{\color{greenunder}\text{expect to equal }\text{pu}_{Q}:}\quad & \text{pu}_{P}\bef\psi=\,\begin{array}{|c||cc|}
 & A & F^{P^{A}}\\
\hline A & \text{id} & \bbnum 0
\end{array}\,\bef\,\begin{array}{|c||cc|}
 & A & G^{Q^{A}}\\
\hline A & \text{id} & \bbnum 0\\
F^{P^{A}} & \bbnum 0 & \overline{\psi}^{\uparrow F}\bef\phi
\end{array}\\
 & =\,\begin{array}{|c||cc|}
 & A & G^{Q^{A}}\\
\hline A & \text{id} & \bbnum 0
\end{array}\,=\text{pu}_{Q}\quad.
\end{align*}

To verify the composition law, write the two sides separately:
\begin{align*}
{\color{greenunder}\text{left-hand side}:}\quad & \text{ftn}_{P}\bef\psi=\,\begin{array}{|c||cc|}
 & A & F^{P^{A}}\\
\hline A & \text{id} & \bbnum 0\\
F^{P^{A}} & \bbnum 0 & \text{id}\\
F^{P^{P^{A}}} & \bbnum 0 & \overline{\text{ftn}}_{P}^{\uparrow F}
\end{array}\,\bef\,\begin{array}{|c||cc|}
 & A & G^{Q^{A}}\\
\hline A & \text{id} & \bbnum 0\\
F^{P^{A}} & \bbnum 0 & \overline{\psi}^{\uparrow F}\bef\phi
\end{array}\\
 & \quad=\,\begin{array}{|c||cc|}
 & A & G^{Q^{A}}\\
\hline A & \text{id} & \bbnum 0\\
F^{P^{A}} & \bbnum 0 & \overline{\psi}^{\uparrow F}\bef\phi\\
F^{P^{P^{A}}} & \bbnum 0 & \overline{\text{ftn}}_{P}^{\uparrow F}\bef\overline{\psi}^{\uparrow F}\bef\phi
\end{array}\quad,\\
{\color{greenunder}\text{right-hand side}:}\quad & \psi^{\uparrow P}\bef\psi\bef\text{ftn}_{Q}=\,\begin{array}{|c||ccc|}
 & A & G^{Q^{A}} & F^{P^{Q^{A}}}\\
\hline A & \text{id} & \bbnum 0 & \bbnum 0\\
F^{P^{A}} & \bbnum 0 & \psi^{\uparrow F}\bef\phi & \bbnum 0\\
F^{P^{P^{A}}} & \bbnum 0 & \bbnum 0 & \overline{\psi}^{\uparrow P\uparrow F}
\end{array}\,\bef\,\begin{array}{|c||ccc|}
 & A & G^{Q^{A}} & G^{Q^{Q^{A}}}\\
\hline A & \text{id} & \bbnum 0 & \bbnum 0\\
G^{Q^{A}} & \bbnum 0 & \text{id} & \bbnum 0\\
F^{P^{Q^{A}}} & \bbnum 0 & \bbnum 0 & \overline{\psi}^{\uparrow F}\bef\phi
\end{array}\,\bef\text{ftn}_{Q}\\
 & =\,\begin{array}{|c||ccc|}
 & A & G^{Q^{A}} & G^{Q^{Q^{A}}}\\
\hline A & \text{id} & \bbnum 0 & \bbnum 0\\
F^{P^{A}} & \bbnum 0 & \psi^{\uparrow F}\bef\phi & \bbnum 0\\
F^{P^{P^{A}}} & \bbnum 0 & \bbnum 0 & \overline{\psi}^{\uparrow P\uparrow F}\bef\overline{\psi}^{\uparrow F}\bef\phi
\end{array}\,\bef\,\begin{array}{|c||cc|}
 & A & G^{Q^{A}}\\
\hline A & \text{id} & \bbnum 0\\
G^{Q^{A}} & \bbnum 0 & \text{id}\\
G^{Q^{Q^{A}}} & \bbnum 0 & \overline{\text{ftn}}_{Q}^{\uparrow G}
\end{array}\\
 & =\,\,\begin{array}{|c||cc|}
 & A & G^{Q^{A}}\\
\hline A & \text{id} & \bbnum 0\\
F^{P^{A}} & \bbnum 0 & \psi^{\uparrow F}\bef\phi\\
F^{P^{P^{A}}} & \bbnum 0 & \overline{\psi}^{\uparrow P\uparrow F}\bef\overline{\psi}^{\uparrow F}\bef\phi\bef\overline{\text{ftn}}_{Q}^{\uparrow G}
\end{array}\quad.
\end{align*}
The remaining difference is between the last rows of the matrices:
\[
\overline{\text{ftn}}_{P}^{\uparrow F}\bef\overline{\psi}^{\uparrow F}\bef\phi\overset{?}{=}\overline{\psi}^{\uparrow P\uparrow F}\bef\overline{\psi}^{\uparrow F}\bef\phi\bef\overline{\text{ftn}}_{Q}^{\uparrow G}\quad.
\]
By the inductive assumption, the law already holds for recursive calls
of $\overline{\psi}$:
\[
\text{ftn}_{P}\bef\overline{\psi}=\overline{\psi}^{\uparrow P}\bef\overline{\psi}\bef\text{ftn}_{Q}\quad.
\]
So, it remains to show that
\[
(\overline{\psi}^{\uparrow P}\bef\overline{\psi}\bef\text{ftn}_{Q}\big)^{\uparrow F}\bef\phi\overset{?}{=}\overline{\psi}^{\uparrow P\uparrow F}\bef\overline{\psi}^{\uparrow F}\bef\phi\bef\overline{\text{ftn}}_{Q}^{\uparrow G}\quad.
\]
This holds due to the naturality law of $\phi$, in the form $\phi\bef f^{\uparrow G}=f^{\uparrow F}\bef\phi$.
$\square$

Heuristically, a free monad on a functor $F$ will wrap $F$ in a
more complicated type constructor such that the resulting type has
the required monad operations. If $F$ is already a monad, constructing
the free monad on $F$ is unnecessary. Indeed, a value of type $\text{Free}^{F,A}$
can be always mapped back to $F^{A}$ while preserving the monad operations:

\subsubsection{Statement \label{subsec:Statement-free-monad-on-a-monad-mapped}\ref{subsec:Statement-free-monad-on-a-monad-mapped}}

Assume that $F$ is itself a monad, and denote $T\triangleq\text{Free}^{F}$
for brevity.

\textbf{(a)} There is a monad morphism $p:T^{A}\rightarrow F^{A}$
defined by
\[
p\triangleq\,\begin{array}{|c||c|}
 & F^{A}\\
\hline A & \text{pu}_{F}\\
F^{T^{A}} & \overline{p}^{\uparrow F}\bef\text{ftn}_{F}
\end{array}\quad.
\]

\textbf{(b)} The function $q:F^{A}\rightarrow T^{A}$ defined by $q(f)\triangleq\bbnum 0+f\triangleright(a^{:A}\rightarrow a+\bbnum 0)^{\uparrow F}$
is \emph{not} a monad morphism.

{*}{*}{*} but $q\bef f=\text{id}$?

\subparagraph{Proof}

\textbf{(a)} To verify the identity law of $p$:
\begin{align*}
{\color{greenunder}\text{expect to equal }\text{pu}_{F}:}\quad & \text{pu}_{T}\bef p=\,\begin{array}{|c||cc|}
 & A & F^{T^{A}}\\
\hline A & \text{id} & \bbnum 0
\end{array}\,\bef\,\begin{array}{|c||c|}
 & F^{A}\\
\hline A & \text{pu}_{F}\\
F^{T^{A}} & \overline{p}^{\uparrow F}\bef\text{ftn}_{F}
\end{array}\,=\text{pu}_{F}\quad.
\end{align*}

To verify the composition law, write its two sides separately:
\begin{align*}
{\color{greenunder}\text{left-hand side}:}\quad & \text{ftn}_{T}\bef p=\,\begin{array}{|c||cc|}
 & A & F^{T^{A}}\\
\hline A & \text{id} & \bbnum 0\\
F^{T^{A}} & \bbnum 0 & \text{id}\\
F^{T^{T^{A}}} & \bbnum 0 & \overline{\text{ftn}}_{T}^{\uparrow F}
\end{array}\,\bef\,\begin{array}{|c||c|}
 & F^{A}\\
\hline A & \text{pu}_{F}\\
F^{T^{A}} & \overline{p}^{\uparrow F}\bef\text{ftn}_{F}
\end{array}\,=\,\begin{array}{|c||c|}
 & F^{A}\\
\hline A & \text{pu}_{F}\\
F^{T^{A}} & \overline{p}^{\uparrow F}\bef\text{ftn}_{F}\\
F^{T^{T^{A}}} & \overline{\text{ftn}}_{T}^{\uparrow F}\bef\overline{p}^{\uparrow F}\bef\text{ftn}_{F}
\end{array}\quad,\\
{\color{greenunder}\text{right-hand side}:}\quad & p^{\uparrow T}\bef p\bef\text{ftn}_{F}=\,\begin{array}{|c||cc|}
 & F^{A} & F^{T^{F^{A}}}\\
\hline A & \text{pu}_{F} & \bbnum 0\\
F^{T^{A}} & \overline{p}^{\uparrow F}\bef\text{ftn}_{F} & \bbnum 0\\
F^{T^{T^{A}}} & \bbnum 0 & \overline{p}^{\uparrow T\uparrow F}
\end{array}\,\bef\,\begin{array}{|c||c|}
 & F^{F^{A}}\\
\hline F^{A} & \text{pu}_{F}\\
F^{T^{F^{A}}} & \overline{p}^{\uparrow F}\bef\text{ftn}_{F}
\end{array}\,\bef\text{ftn}_{F}\\
 & \quad=\,\begin{array}{|c||c|}
 & F^{A}\\
\hline A & \text{pu}_{F}\bef\gunderline{\text{pu}_{F}\bef\text{ftn}_{F}}\\
F^{T^{A}} & \overline{p}^{\uparrow F}\bef\text{ftn}_{F}\bef\gunderline{\text{pu}_{F}\bef\text{ftn}_{F}}\\
F^{T^{T^{A}}} & \overline{p}^{\uparrow T\uparrow F}\bef\overline{p}^{\uparrow F}\bef\gunderline{\text{ftn}_{F}\bef\text{ftn}_{F}}
\end{array}\,=\,\begin{array}{|c||c|}
 & F^{A}\\
\hline A & \text{pu}_{F}\\
F^{T^{A}} & \overline{p}^{\uparrow F}\bef\text{ftn}_{F}\\
F^{T^{T^{A}}} & \overline{p}^{\uparrow T\uparrow F}\bef\overline{p}^{\uparrow F}\bef\text{ftn}_{F}^{\uparrow F}\bef\text{ftn}_{F}
\end{array}\quad.
\end{align*}
The last two matrices differ only in the last rows, and the difference
is
\[
\overline{\text{ftn}}_{T}^{\uparrow F}\bef\overline{p}^{\uparrow F}\overset{?}{=}\overline{p}^{\uparrow T\uparrow F}\bef\overline{p}^{\uparrow F}\bef\text{ftn}_{F}^{\uparrow F}\quad.
\]
Omitting the lifting to $F$, we get:
\[
\overline{\text{ftn}}_{T}\bef\overline{p}\overset{?}{=}\overline{p}^{\uparrow T}\bef\overline{p}\bef\text{ftn}_{F}\quad.
\]
This holds by the inductive assumption that the recursive calls to
$\overline{p}$ already obey the composition law.

\textbf{(b)} The function $q\triangleq f\rightarrow\bbnum 0+f\triangleright(x\rightarrow x+\bbnum 0)^{\uparrow F}$
fails the monad morphism identity law. Given $f\triangleq\text{pu}_{F}(a)$,
we compute: 
\[
q(f)=\bbnum 0+a\triangleright\text{pu}_{F}\triangleright(x\rightarrow x+\bbnum 0)^{\uparrow F}=\bbnum 0+a\triangleright(x\rightarrow x+\bbnum 0)\triangleright\text{pu}_{F}=\bbnum 0+\text{pu}_{F}(a+\bbnum 0)\quad.
\]
However, the expected value is $\text{pu}_{T}(a)=a+\bbnum 0$, which
cannot equal  $\bbnum 0+\text{pu}_{F}(a+\bbnum 0)$.

\section{Working with quantified types}

{*}{*}{*}Move all this to an appendix? {*}{*}{*} also discuss existentials?

In the notation used in this book, there is a key difference between
the quantified type $\forall X.\,F^{X}$ and the type expression $F^{X}$
that contains the type parameter $X$. In both cases, $X$ is a completely
unknown type parameter, and an example of Scala code implementing
those types would be:\inputencoding{latin9}
\begin{lstlisting}
def f[X]: F[X] = ???
\end{lstlisting}
\inputencoding{utf8}However, the type quantifier $\forall X$ implies that all values
of type $\forall X.\,F^{X}$ must be implemented via fully parametric
code. The code of the function \inputencoding{latin9}\lstinline!f[X]!\inputencoding{utf8}
may not make decisions based on the actual type passed at run time
as the type parameter \inputencoding{latin9}\lstinline!X!\inputencoding{utf8}
into the function. The assumption of full parametricity enables us
to reason about quantified types in a special way. This section explores
the techniques of this reasoning.

\subsection{The Yoneda identities for type constructors}

The Yoneda identities (see Section~\ref{subsec:Yoneda-identities})
can be extended to many other contexts. For instance, a Yoneda identity
holds for types parameterized by a type constructor:

\subsubsection{Statement \label{subsec:Statement-covariant-yoneda-identity-for-type-constructors}\ref{subsec:Statement-covariant-yoneda-identity-for-type-constructors}
(covariant Yoneda identity for functors)\index{Yoneda identity!for functors}}

Assume that $P^{\bullet}$ is any type constructor and $S^{F^{\bullet}}$
is a higher-order functor\index{functor!higher-order}\index{higher-order functor},
i.e., $S$ depends covariantly on an arbitrary type constructor $F^{\bullet}$.
(An example of such $S$ is $S^{F^{\bullet}}\triangleq F^{\text{Int}}\times F^{\text{String}}$.)
Then the type $S^{P^{\bullet}}$ is equivalent to the function type
$\forall F^{\bullet}.\,(P^{\bullet}\leadsto F^{\bullet})\rightarrow S^{F}$,
where the function is required to be natural in the parameter $F^{\bullet}$.
The corresponding naturality law for functions $\sigma$ of type $\forall F^{\bullet}.\,(P^{\bullet}\leadsto F^{\bullet})\rightarrow S^{F}$
involves arbitrary type constructors $Q^{\bullet}$, $R^{\bullet}$,
and arbitrary functions $f:P^{\bullet}\leadsto Q^{\bullet}$ and $g:Q^{\bullet}\leadsto R^{\bullet}$,
and may be written as
\begin{equation}
\sigma^{Q}(f)\bef g^{\uparrow S}=\sigma^{R^{\bullet}}(f\bef g)\quad.\label{eq:assumed-naturality-of-argument-sigma}
\end{equation}
Here, $g^{\uparrow S}$ has type $S^{Q^{\bullet}}\rightarrow S^{R^{\bullet}}$
and is a lifting of the function $g:Q^{\bullet}\leadsto R^{\bullet}$
to the higher-order functor $S$. At the same time, the functions
$f$ and $g$ do \emph{not} have to be natural transformations, and
the type constructors $F^{\bullet}$, $P^{\bullet}$, $Q^{\bullet}$,
$R^{\bullet}$ are \emph{not} required to be functors.

\subparagraph{Proof}

{*}{*}{*}rewrite the proof in a simpler way like in chapter 10{*}{*}{*}For
brevity, we will write just $F$ and $P$ instead of $F^{\bullet}$
and $P^{\bullet}$.

The isomorphism is implemented via two functions \inputencoding{latin9}\lstinline!toC!\inputencoding{utf8}
and \inputencoding{latin9}\lstinline!fromC!\inputencoding{utf8},
\begin{align*}
\text{toC}:S^{P}\rightarrow\forall F.\,(P\leadsto F)\rightarrow S^{F}\quad, & \quad\quad\text{toC}\triangleq s^{:S^{P}}\rightarrow\forall F.\,g^{:P\leadsto F}\rightarrow s\triangleright g^{\uparrow S}\quad,\\
\text{fromC}:(\forall F.\,(P\leadsto F)\rightarrow S^{F})\rightarrow S^{P}\quad, & \quad\quad\text{fromC}\triangleq\sigma^{:\forall F.\,(P\leadsto F)\rightarrow S^{F}}\rightarrow\sigma^{P}(\text{id}^{:P\leadsto P})\quad.
\end{align*}
In the last line, the function $\sigma$ is required to be natural
in its type parameter $Q$.

We need to show that $\text{fromC}\bef\text{toC}=\text{id}$ and $\text{toC}\bef\text{fromC}=\text{id}$.
To verify that $\text{fromC}\bef\text{toC}=\text{id}$, apply both
sides to an arbitrary function $\sigma^{:\forall F.\,(P\leadsto F)\rightarrow S^{F}}$:
\begin{align*}
{\color{greenunder}\text{expect to equal }\sigma:}\quad & \sigma^{:\forall F.\,(P\leadsto F)\rightarrow S^{F}}\triangleright\text{fromC}\bef\text{toC}=\sigma^{P}(\text{id})\triangleright\text{toC}=\forall F.\,g^{:P\leadsto F}\rightarrow\sigma^{P}(\text{id})\triangleright g^{\uparrow S}\quad.
\end{align*}
Since by assumption $\sigma$ satisfies the naturality law~(\ref{eq:assumed-naturality-of-argument-sigma}),
we may apply that law with $Q=P$, $R=F$, and $f=\text{id}$:
\[
\sigma^{P}(\text{id})\bef g^{\uparrow S}=\sigma^{F}(\text{id}\bef g)=F^{F}(g)\quad.
\]
It follows that the function $\sigma\triangleright\text{fromC}\bef\text{toC}$
is the same as $\sigma$:
\[
\sigma\triangleright\text{fromC}\bef\text{toC}=\forall F.\,g^{:P\leadsto F}\rightarrow\sigma^{F}(g)=\forall F.\,\sigma^{F}=\sigma\quad.
\]

To verify that $\text{toC}\bef\text{fromC}=\text{id}$, apply both
sides to an arbitrary $s^{:S^{P}}$:
\begin{align*}
{\color{greenunder}\text{expect to equal }s:}\quad & s^{:S^{P}}\triangleright\text{toC}\bef\text{fromC}=s\triangleright\text{toC}\triangleright\text{fromC}=(\forall F.\,g^{:P\leadsto F}\rightarrow s\triangleright g^{\uparrow S})\triangleright\text{fromC}\\
 & =(g^{:P\leadsto P}\rightarrow s\triangleright g^{\uparrow S})(\text{id}^{:P\leadsto P})=s\triangleright\text{id}^{\uparrow S}=s\quad.
\end{align*}
It remains to check that the function $\sigma^{F}\triangleq g^{:P\leadsto F}\rightarrow s\triangleright g^{\uparrow S}$,
used as an argument of \inputencoding{latin9}\lstinline!fromC!\inputencoding{utf8},
is natural in $F$. To verify the naturality law~(\ref{eq:assumed-naturality-of-argument-sigma}):
\begin{align*}
{\color{greenunder}\text{expect to equal }\sigma^{R}(f\bef g):}\quad & \sigma^{Q}(f)\bef g^{\uparrow S}=s\triangleright f^{\uparrow S}\bef g^{\uparrow S}=s\triangleright(f\bef g)^{\uparrow S}=\sigma^{R}(f\bef g)\quad.
\end{align*}


\subsection{Recursive type equations with different fixpoints}

A recursive type may be defined as a \textbf{fixpoint} of\index{fixpoint of a functor}
a functor; that is, a solution of a type equation\index{recursive type equation}
of the form $T\cong F^{T}$, where $F$ is a \textsf{``}structure functor\textsf{''}
that specifies the details of the type recursion. A solution of the
type equation $T\cong F^{T}$ is a type $T$ that is equivalent to
$F^{T}$ via two isomorphisms:
\[
\text{fix}:F^{T}\rightarrow T\quad,\quad\quad\text{unfix}:T\rightarrow F^{T}\quad,\quad\quad\text{fix}\bef\text{unfix}=\text{id}\quad,\quad\text{unfix}\bef\text{fix}=\text{id}\quad.
\]
Section~\ref{subsec:Recursive-types-and-the-existence-of-their-values}
gave a condition for implementability of such types $T$. We will
now consider the question of whether there can be several fixpoints
$T$.

A functor $F$ may have several \emph{inequivalent} fixpoints $T_{1}$,
$T_{2}$, etc. It means that each $T_{i}$ separately satisfies the
fixpoint equation $T\cong F^{T}$. An example is the fixpoint equation
for the \textsf{``}lazy list\textsf{''}:
\begin{equation}
L^{A}\cong\bbnum 1+(\bbnum 1\rightarrow A\times L^{A})\quad.\label{eq:fixpoint-type-equation-for-oncall-list}
\end{equation}
We may also write the same equation using a suitable structure functor
$F$ like this:
\[
L^{A}\cong F^{A,L^{A}}\quad,\quad\quad F^{A,R}\triangleq\bbnum 1+(\bbnum 1\rightarrow A\times R)\quad.
\]
A solution of this fixpoint equation can be visualized (non-rigorously)
as the type:
\[
L^{A}=F^{A,F^{A,F^{A,...}}}=\bbnum 1+(\bbnum 1\rightarrow A\times(\bbnum 1+(\bbnum 1\rightarrow A\times(...))))\quad,
\]
representing an \textsf{``}on-call\textsf{''} list of values of type $A$. To get
the next value of type $A$, one must evaluate a function call. In
this way, the elements of the on-call list are computed only when
needed. It could happen that a value of type $L^{A}$ will \emph{never}
stop yielding new values of type $A$ if we keep requesting the next
elements of the list.

To show rigorously that the type $L^{A}$ is a solution of the type
equation $L^{A}\cong F^{A,L^{A}}$, {*}{*}{*}

To see that the fixpoint equation~(\ref{eq:fixpoint-type-equation-for-oncall-list})
has (at least) three inequivalent solutions, consider a function \inputencoding{latin9}\lstinline!toList!\inputencoding{utf8}
that converts a value of type $L^{A}$ into a sequence of type \inputencoding{latin9}\lstinline!List[A]!\inputencoding{utf8},
whose elements are eagerly evaluated. The function \inputencoding{latin9}\lstinline!toList!\inputencoding{utf8}
keeps recursively requesting new elements of the on-call list and
accumulates the resulting values:
\[
\text{toList}:L^{A}\rightarrow\text{List}^{A}\quad,\quad\quad\text{toList}\triangleq\begin{array}{|c||cc|}
 & \bbnum 1 & \bbnum 1+A\times\text{List}^{A}\\
\hline \bbnum 1 & \text{id} & \bbnum 0\\
\bbnum 1\rightarrow A\times L^{A} & \bbnum 0 & p\rightarrow p(1)\triangleright(\text{id}\boxtimes\overline{\text{toList}})
\end{array}\quad.
\]
Does this function terminate? It is clear that \inputencoding{latin9}\lstinline!toList!\inputencoding{utf8}
will terminate only if the on-call list eventually stops yielding
new values of type $A$. On the other hand, if the on-call list never
stops yielding new values of type $A$, the function \inputencoding{latin9}\lstinline!toList!\inputencoding{utf8}
will not terminate. So, \inputencoding{latin9}\lstinline!toList!\inputencoding{utf8}
must be a partial function. 

Let us denote by $L_{\text{fin}}^{A}$ the subtype of $L^{A}$ consisting
of finite lists, i.e., on-call lists that eventually stop yielding
new values of type $A$. Then the function:
\[
\text{toList}:L_{\text{fin}}^{A}\rightarrow\text{List}^{A}
\]
is total. On-call lists of type $L_{\text{fin}}^{A}$ contain a finite
number of values and thus are equivalent to eager lists. To make this
equivalence formal, we may define isomorphism functions, $\text{toList}:L_{\text{fin}}^{A}\rightarrow\text{List}^{A}$
and $\text{fromList}:\text{List}^{A}\rightarrow L_{\text{fin}}^{A}$. 

Let us also denote by $L_{\text{inf}}^{A}$ the subtype corresponding
to \textsf{``}always infinite\textsf{''} on-call lists, i.e., those that \emph{never}
stop yielding new values of type $A$. Then the function \inputencoding{latin9}\lstinline!toList!\inputencoding{utf8}
does not terminate for any argument of type $L_{\text{inf}}^{A}$.
So, the types $L_{\text{fin}}^{A}$ and $L_{\text{inf}}^{A}$ are
\emph{not} equivalent. Were they equivalent, we would have an isomorphism
$q:L_{\text{inf}}^{A}\rightarrow L_{\text{fin}}^{A}$, and then we
could compose $q$ with \inputencoding{latin9}\lstinline!toList!\inputencoding{utf8}
to obtain a function $\text{toList}:L_{\text{inf}}^{A}\rightarrow\text{List}^{A}$
that terminates, which is impossible.

To show that both types ($L_{\text{fin}}^{A}$ and $L_{\text{inf}}^{A}$)
satisfy the fixpoint type equation~(\ref{eq:fixpoint-type-equation-for-oncall-list}),
we can implement the corresponding isomorphisms \inputencoding{latin9}\lstinline!fix!\inputencoding{utf8}
and \inputencoding{latin9}\lstinline!unfix!\inputencoding{utf8}.
Each of these functions will either add or remove one element at the
beginning of the list. These operations keep finite lists finite and
infinite lists infinite. So, a composition (such as, $\text{unfix}\bef\text{fix}$)
of these isomorphisms will act as an identity function on $L_{\text{fin}}^{A}$
or on $L_{\text{inf}}^{A}$.

The type $L^{A}$ is an on-call list that may or may not terminate.
So, $L^{A}$ is equivalent to a disjunction $L_{\text{fin}}^{A}+L_{\text{inf}}^{A}$.
We see that the type equation~(\ref{eq:fixpoint-type-equation-for-oncall-list})
has three inequivalent solutions: $L_{\text{fin}}^{A}$, $L_{\text{inf}}^{A}$,
and $L^{A}$.

Some fixpoints represent \textsf{``}larger\textsf{''} types than other fixpoints.
For instance, $L^{A}$ is \textsf{``}larger\textsf{''} than either of $L_{\text{fin}}^{A}$
and $L_{\text{inf}}^{A}$. To see this formally, we consider the functions
$f_{1}:L_{\text{fin}}^{A}\rightarrow L^{A}$ and $f_{2}:L_{\text{inf}}^{A}\rightarrow L^{A}$.
The function $f_{1}$ embeds values of type $L_{\text{fin}}^{A}$
(finite lists) in the type $L^{A}$ that includes both finite and
infinite lists. The functions $f_{1}$ and $f_{2}$ are injective
because they are functions of type $P\rightarrow P+Q$ for some $P$
and $Q$.

The functions $f_{1}$ and $f_{2}$ are in a sense \textsf{``}well-adapted\textsf{''}
to the fixpoint structure of the types. The following definition makes
this property precise:

\subsubsection{Definition \label{subsec:Definition-fixpoint-preserving-function}\ref{subsec:Definition-fixpoint-preserving-function}}

Suppose a functor $F$ has two fixpoint types $T_{1}$ and $T_{2}$
with corresponding functions $\text{fix}_{1}:F^{T_{1}}\rightarrow T_{1}$,
$\text{unfix}_{1}:T_{1}\rightarrow F^{T_{1}}$, $\text{fix}_{2}:F^{T_{2}}\rightarrow T_{2}$,
and $\text{unfix}_{2}:T_{2}\rightarrow F^{T_{2}}$. A function $f:T_{1}\rightarrow T_{2}$
is \index{fixpoint-preserving function}\textbf{fixpoint-preserving}
if the compatibility law holds:

\begin{wrapfigure}{l}{0.32\columnwidth}%
\vspace{-2\baselineskip}
\[
\xymatrix{\xyScaleY{1.0pc}\xyScaleX{3pc}T_{1}\ar[d]\sp(0.45){f}\ar[r]\sp(0.5){\text{unfix}_{1}} & F^{T_{1}}\ar[d]\sp(0.4){f^{\uparrow F}}\ar[r]\sp(0.5){\text{fix}_{1}} & T_{1}\ar[d]\sp(0.45){f}\\
T_{2}\ar[r]\sp(0.5){\text{unfix}_{2}} & F^{T_{2}}\ar[r]\sp(0.5){\text{fix}_{2}} & T_{2}
}
\]
\vspace{-1\baselineskip}
\end{wrapfigure}%

~\vspace{-1\baselineskip}
\begin{align*}
 & \text{fix}_{1}\bef f=f^{\uparrow F}\bef\text{fix}_{2}\quad,\\
 & \text{unfix}_{1}\bef f^{\uparrow F}=f\bef\text{unfix}_{2}\quad.
\end{align*}
\vspace{-0.5\baselineskip}

To show that both $f_{1}$ and $f_{2}$ are fixpoint-preserving, we
note that {*}{*}{*}

As an example of a function that is \emph{not} fixpoint-preserving,
consider truncating an infinite list at a fixed length of, say, $100$
elements. {*}{*}{*}

\subsection{The Church encoding of recursive types\label{subsec:The-Church-encoding-of-recursive-types}}

Any given type can be represented in a \textbf{Church encoding}\index{Church encoding},
which is a function type with a universally quantified type parameter.
A simple Church encoding is given by the type equivalence
\[
T\cong\forall X.\,\left(T\rightarrow X\right)\rightarrow X\quad,
\]
which follows from the covariant Yoneda identity (Statement~\ref{subsec:Statement-covariant-yoneda-identity-for-types}
with the functor $F\triangleq\text{Id}$). 

There is rarely an advantage in replacing a simple type $T$ by a
more complicated function type, $\forall X.\,(T\rightarrow X)\rightarrow X$.
However, the Church encoding has a different form when $T$ is a \emph{recursive}
type.\footnote{This \textsf{``}Church encoding\textsf{''} is known more precisely as \textsf{``}Boehm-Berarducci
encoding\textsf{''}. For the purposes of this book, they are the same. See
\texttt{\href{http://okmij.org/ftp/tagless-final/course/Boehm-Berarducci.html}{http://okmij.org/ftp/tagless-final/course/Boehm-Berarducci.html}}
for discussion.}

Consider a recursive type $T$ defined by a fixpoint equation $T\triangleq F^{T}$
with a given structure functor $F$. We could write a Church encoding
for $T$ as $\forall X.\,(T\rightarrow X)\rightarrow X$ or as $\forall X.\,(F^{T}\rightarrow X)\rightarrow X$,
but these encodings give no advantages. It turns out that another,
more useful Church encoding for $T$ is:
\begin{equation}
T\cong\forall X.\,(F^{X}\rightarrow X)\rightarrow X\quad\text{if the type }T\text{ is defined by }T\triangleq F^{T}\quad.\label{eq:Church-encoding-recursive-type}
\end{equation}
The Scala code for the type $\forall X.\,(F^{X}\rightarrow X)\rightarrow X$
is:\inputencoding{latin9}
\begin{lstlisting}
trait TC[F[_]] { def run[X](fold: F[X] => X): X }
\end{lstlisting}
\inputencoding{utf8}In this section, we will study the type equivalence~(\ref{eq:Church-encoding-recursive-type}).

Note that the Yoneda lemma cannot be used to prove Eq.~(\ref{eq:Church-encoding-recursive-type}).
The Yoneda lemma only applies to types of the form $\forall X.\,(A\rightarrow X)\rightarrow F^{X}$,
where the type $A$ cannot depend on the quantified type $X$. 

The following statement\footnote{See also the papers \textsf{``}A note on strong dinaturality\textsf{''} (\texttt{\href{https://web.archive.org/web/20110601105059/http://www.cs.ioc.ee/~tarmo/papers/fics10.pdf}{http://www.cs.ioc.ee/$\sim$tarmo/papers/fics10.pdf}})
and \textsf{``}Build, augment, and destroy universally\textsf{''} (\texttt{\href{https://kodu.ut.ee/~varmo/papers/aplas04.ps.gz}{https://kodu.ut.ee/$\sim$varmo/papers/aplas04.ps.gz}}).} shows that the Church encoding~(\ref{eq:Church-encoding-recursive-type})
is a fixpoint:

\subsubsection{Statement \label{subsec:Statement-Church-encoding-recursive-type-covariant}\ref{subsec:Statement-Church-encoding-recursive-type-covariant}}

For a given structure functor $F$, define the type $T$ as
\[
T\triangleq\forall X.\,(F^{X}\rightarrow X)\rightarrow X\quad.
\]
Additionally, we require all values $t$ of type $T$ to satisfy the
strong dinaturality law,\index{strong dinaturality law!of Church encoding}
which for the given type $T$ has the following form: for any $r^{:F^{A}\rightarrow A}$,
$s^{:F^{B}\rightarrow B}$, and $f^{:A\rightarrow B}$: 

\begin{equation}
\text{if }\quad r\bef f=f^{\uparrow F}\bef s\quad\text{ then }\quad r\triangleright t^{A}\triangleright f=s\triangleright t^{B}\quad.\label{eq:strong-dinaturality-for-church-encoded-fix-unfix}
\end{equation}
Defined in this way, the type $T$ is a solution of the fixpoint equation
$T\cong F^{T}$. (Strong dinaturality holds for all fully parametric
functions of type $T$. This follows from Example~\ref{subsec:Example-strong-dinaturality-for-some-type-signatures}(b)
with $G^{A}\triangleq A$ and $L^{X,Y}\triangleq Y$.)

\subparagraph{Proof}

First, we use typed holes to derive the code for the isomorphisms:
\[
\text{fix}:F^{T}\rightarrow T\quad,\quad\quad\text{unfix}:T\rightarrow F^{T}\quad.
\]
Begin by writing out the type signature of \inputencoding{latin9}\lstinline!fix!\inputencoding{utf8}
and an implementation with a typed hole:
\begin{align*}
 & \text{fix}:F^{\forall X.\,(F^{X}\rightarrow X)\rightarrow X}\rightarrow\forall Y.\,(F^{Y}\rightarrow Y)\rightarrow Y\quad,\\
 & \text{fix}\triangleq f^{:F^{\forall X.\,(F^{X}\rightarrow X)\rightarrow X}}\rightarrow\forall Y.\,q^{:F^{Y}\rightarrow Y}\rightarrow\text{???}^{:Y}\quad.
\end{align*}
The only way of computing a value of type $Y$ is to apply $q$ to
an argument of type $F^{Y}$:
\[
\text{fix}=f^{:F^{\forall X.\,(F^{X}\rightarrow X)\rightarrow X}}\rightarrow\forall Y.\,q^{:F^{Y}\rightarrow Y}\rightarrow\big(\text{???}^{:F^{Y}}\big)\triangleright q\quad.
\]
The functor $F$ is arbitrary, so the only way of computing a value
of type $F^{Y}$ is to use the given value $f$ with a \inputencoding{latin9}\lstinline!map!\inputencoding{utf8}
method (the only method we can use with any functor $F$):
\[
\text{???}^{:F^{Y}}=f\triangleright(p^{:\forall X.\,(F^{X}\rightarrow X)\rightarrow X}\rightarrow\text{???}^{:Y})^{\uparrow F}\quad.
\]
Since the type parameter $X$ is universally quantified inside $p$,
we may set $X$ to any type as needed. So, we set $X=Y$ and use the
given value $q^{:F^{Y}\rightarrow Y}$ to fill the typed hole:
\[
\text{???}^{:F^{Y}}=f\triangleright(p^{:\forall X.\,(F^{X}\rightarrow X)\rightarrow X}\rightarrow q\triangleright p^{Y})^{\uparrow F}\quad.
\]
This allows us to complete the code of \inputencoding{latin9}\lstinline!fix!\inputencoding{utf8}:
\[
\text{fix}\triangleq f^{:F^{\forall X.\,(F^{X}\rightarrow X)\rightarrow X}}\rightarrow\forall Y.\,q^{:F^{Y}\rightarrow Y}\rightarrow f\triangleright\big(p^{:\forall X.\,(F^{X}\rightarrow X)\rightarrow X}\rightarrow q\triangleright p^{Y}\big)^{\uparrow F}\triangleright q\quad.
\]

We turn to implementing \inputencoding{latin9}\lstinline!unfix!\inputencoding{utf8}:
\begin{align*}
 & \text{unfix}:\big(\forall Y.\,(F^{Y}\rightarrow Y)\rightarrow Y\big)\rightarrow F^{\forall X.\,(F^{X}\rightarrow X)\rightarrow X\big)}\quad,\\
 & \text{unfix}\triangleq t^{:\forall Y.\,(F^{Y}\rightarrow Y)\rightarrow Y}\rightarrow\text{???}^{:F^{T}}\quad.
\end{align*}
The only way of filling the typed hole $\text{???}^{:F^{T}}$ is to
apply $t^{Y}$ while setting the type parameter $Y$ to $F^{T}$:
\[
\text{unfix}=t^{:\forall Y.\,(F^{Y}\rightarrow Y)\rightarrow Y}\rightarrow t^{F^{T}}\big(\text{???}^{:F^{F^{T}}\rightarrow F^{T}}\big)\quad.
\]
We can now fill the typed hole $\text{???}^{:F^{F^{T}}\rightarrow F^{T}}$
by lifting \inputencoding{latin9}\lstinline!fix!\inputencoding{utf8}
to the functor $F$:
\[
\text{unfix}\triangleq t^{:\forall Y.\,(F^{Y}\rightarrow Y)\rightarrow Y}\rightarrow t^{F^{T}}\big(\text{fix}^{\uparrow F}\big)\quad.
\]

It remains to show that \inputencoding{latin9}\lstinline!fix!\inputencoding{utf8}
and \inputencoding{latin9}\lstinline!unfix!\inputencoding{utf8} are
inverses. Start with one direction:
\begin{align*}
{\color{greenunder}\text{expect to equal }t:}\quad & t^{:T}\triangleright\text{unfix}\bef\text{fix}=t^{F^{T}}(\text{fix}^{\uparrow F})\triangleright\text{fix}\\
 & =\forall Y.\,q^{:F^{Y}\rightarrow Y}\rightarrow\text{fix}^{\uparrow F}\triangleright t^{F^{T}}\triangleright\big(p^{:\forall X.\,(F^{X}\rightarrow X)\rightarrow X}\rightarrow q\triangleright p^{Y}\big)^{\uparrow F}\triangleright q\quad.
\end{align*}
The resulting function will be equal to $t$ if we show that it gives
the same result when applied to an arbitrary argument $q^{:F^{Y}\rightarrow Y}$
(where the type parameter $Y$ is free):
\begin{equation}
\text{fix}^{\uparrow F}\triangleright t^{F^{T}}\triangleright\big(p^{:\forall X.\,(F^{X}\rightarrow X)\rightarrow X}\rightarrow p^{Y}(q)\big)^{\uparrow F}\triangleright q\overset{?}{=}t^{Y}(q^{:F^{Y}\rightarrow Y})\quad.\label{eq:unfix-fix-identity-derivation1}
\end{equation}
To proceed, we need to use the strong dinaturality law of $t$ with
suitable $r$, $s$, and $f$:
\begin{equation}
r^{:F^{A}\rightarrow A}\triangleright t^{A}\triangleright f^{:A\rightarrow B}=s^{:F^{B}\rightarrow B}\triangleright t^{B}\quad.\label{eq:unfix-fix-strong-dinaturality-derivation1}
\end{equation}
This law will reproduce Eq.~(\ref{eq:unfix-fix-identity-derivation1})
if we choose $r$, $s$, and $f$ such that:
\[
r=\text{fix}^{\uparrow F}\quad,\quad\quad f=\big(p^{:\forall X.\,(F^{X}\rightarrow X)\rightarrow X}\rightarrow q\triangleright p^{Y}\big)^{\uparrow F}\bef q\quad,\quad\quad s=q\quad.
\]
For these equations to hold, the type parameters $A$ and $B$ must
be set appropriately in Eq.~(\ref{eq:unfix-fix-strong-dinaturality-derivation1}).
The type of $\text{fix}^{\uparrow F}$ is $F^{F^{T}}\rightarrow F^{T}$,
which means that $A=F^{T}$. The type of $q$ is $F^{Y}\rightarrow Y$,
so $B=Y$. This agrees with the type of $f^{:A\rightarrow B}$. So,
the strong dinaturality law~(\ref{eq:unfix-fix-strong-dinaturality-derivation1})
will yield Eq.~(\ref{eq:unfix-fix-identity-derivation1}) as long
as the chosen variables satisfy the condition
\[
r\bef f\overset{?}{=}f^{\uparrow F}\bef s\quad.
\]
It remains to verify that the last line holds. Substituting the values
of the variables, we get:
\begin{align*}
 & \text{fix}^{\uparrow F}\bef f\overset{?}{=}f^{\uparrow F}\bef q\quad,\\
{\color{greenunder}\text{or equivalently}:}\quad & \text{fix}^{\uparrow F}\bef\big(p^{:\forall X.\,(F^{X}\rightarrow X)\rightarrow X}\rightarrow q\triangleright p^{Y}\big)^{\uparrow F}\bef q\overset{?}{=}\big(p^{:\forall X.\,(F^{X}\rightarrow X)\rightarrow X}\rightarrow q\triangleright p^{Y}\big)^{\uparrow F\uparrow F}\bef q^{\uparrow F}\bef q\quad.
\end{align*}
Evaluate the function composition in the left-hand side, and make
both sides equal:
\begin{align*}
 & \gunderline{\text{fix}}^{\uparrow F}\bef\big(p^{:\forall X.\,(F^{X}\rightarrow X)\rightarrow X}\rightarrow q\triangleright p^{Y}\big)^{\uparrow F}\bef q\\
{\color{greenunder}\text{expand function}:}\quad & =\big((f^{:F^{T}}\rightarrow f\triangleright\text{fix})\bef(p\rightarrow q\triangleright p^{Y})\big)^{\uparrow F}\bef q=\big(f^{:F^{T}}\rightarrow q\triangleright(f\triangleright\gunderline{\text{fix}})^{Y}\big)^{\uparrow F}\bef q\\
{\color{greenunder}\text{definition of }\text{fix}:}\quad & =\big(\gunderline{f^{:F^{T}}\rightarrow f\,\triangleright}\,(p\rightarrow q\triangleright p^{Y})^{\uparrow F}\bef q\big)^{\uparrow F}\bef q=\big((p\rightarrow q\triangleright p^{Y})^{\uparrow F}\bef q\big)^{\uparrow F}\bef q\quad.
\end{align*}
The last expression is equal to the right-hand side.

This proves one direction of the isomorphism between $T$ and $F^{T}$,
namely: $\text{unfix}\bef\text{fix}=\text{id}$.

To verify the opposite direction of the isomorphism, we write:
\begin{align*}
{\color{greenunder}\text{expect to equal }f:}\quad & f^{:F^{T}}\triangleright\text{fix}\triangleright\text{unfix}=\big(\forall Y.\,q^{:F^{Y}\rightarrow Y}\rightarrow f\triangleright\big(p^{:\forall X.\,(F^{X}\rightarrow X)\rightarrow X}\rightarrow q\triangleright p^{Y}\big)^{\uparrow F}\triangleright q\big)\triangleright\text{unfix}\\
 & =f\triangleright\big(p^{:\forall X.\,(F^{X}\rightarrow X)\rightarrow X}\rightarrow p^{F^{T}}(\text{fix}^{\uparrow F})\big)^{\uparrow F}\triangleright\text{fix}^{\uparrow F}\\
 & =f\triangleright\big(p^{:\forall X.\,(F^{X}\rightarrow X)\rightarrow X}\rightarrow p^{F^{T}}(\text{fix}^{\uparrow F})\triangleright\text{fix}\big)^{\uparrow F}\quad.
\end{align*}
The last value will be equal to $f$ if the function under $(...)^{\uparrow F}$
is an identity function:
\[
p^{F^{T}}(\text{fix}^{\uparrow F})\triangleright\text{fix}\overset{?}{=}p\quad.
\]
Since $p$ is of function type, both sides must be equal when applied
to an arbitrary $s^{:F^{B}\rightarrow B}$:
\begin{equation}
s\triangleright(p^{F^{T}}(\text{fix}^{\uparrow F})\triangleright\text{fix})\overset{?}{=}s\triangleright p\quad.\label{eq:fix-unfix-derivation2}
\end{equation}
To prove the last equation, we use the assumption that all values
of type $T$ satisfy the strong dinaturality law. So, the law must
apply to the value $p$:
\[
r\triangleright p\triangleright f=s\triangleright p\quad.
\]
This law reproduces Eq.~(\ref{eq:fix-unfix-derivation2}) if we define
$r$ and $f$ by
\[
r\triangleq\text{fix}^{\uparrow F}\quad,\quad\quad f\triangleq u^{:F^{T}}\rightarrow s\triangleright(u\triangleright\text{fix})\quad.
\]
It remains to verify that the assumption of the strong dinaturality
law holds:
\begin{align*}
 & r\bef f\overset{?}{=}f^{\uparrow F}\bef s\quad,\\
{\color{greenunder}\text{or equivalently}:}\quad & \text{fix}^{\uparrow F}\bef(u\rightarrow s\triangleright(u\triangleright\text{fix}))\overset{?}{=}\big(u^{:F^{T}}\rightarrow s\triangleright(u\triangleright\text{fix})\big)^{\uparrow F}\bef s\quad.
\end{align*}
Rewrite the left-hand side above until it becomes equal to the right-hand
side:
\begin{align*}
 & \text{fix}^{\uparrow F}\bef(u\rightarrow s\triangleright(u\triangleright\text{fix}))=(u^{:F^{F^{T}}}\rightarrow u\triangleright\text{fix}^{\uparrow F})\bef(u\rightarrow s\triangleright(u\triangleright\text{fix}))\\
{\color{greenunder}\text{compute composition}:}\quad & =u\rightarrow s\triangleright(u\triangleright\text{fix}^{\uparrow F}\triangleright\gunderline{\text{fix}})=\gunderline{u\rightarrow u\,\triangleright}\,\text{fix}^{\uparrow F}\triangleright(q\rightarrow s\triangleright q)^{\uparrow F}\bef s\\
{\color{greenunder}\text{unexpand function}:}\quad & =\big(\text{fix}\bef(q\rightarrow s\triangleright q)\big)^{\uparrow F}\bef s=\big((u^{:F^{T}}\rightarrow u\triangleright\text{fix})\bef(q\rightarrow s\triangleright q)\big)^{\uparrow F}\bef s\\
{\color{greenunder}\text{compute composition}:}\quad & =\big(u^{:F^{T}}\rightarrow s\triangleright(u\triangleright\text{fix})\big)^{\uparrow F}\bef s\quad.
\end{align*}
The two sides are now equal. $\square$

A curious property of the type $T$ is that it is a function with
argument of type $F^{X}\rightarrow X$, where $X$ can be any type,
including $T$ itself. But we already have a function of type $F^{T}\rightarrow T$;
it is the function \inputencoding{latin9}\lstinline!fix!\inputencoding{utf8}.
So, applying a value $t$ of type $T$ to the function \inputencoding{latin9}\lstinline!fix!\inputencoding{utf8}
yields again a value of type $T$. As the next statement shows, that
value is the same as $t$:

\subsubsection{Statement \label{subsec:Statement-strong-dinaturality-property-of-fix}\ref{subsec:Statement-strong-dinaturality-property-of-fix}}

Consider the Church-encoded type $T$ and the function \inputencoding{latin9}\lstinline!fix!\inputencoding{utf8}
defined in Statement~\ref{subsec:Statement-Church-encoding-recursive-type-covariant}.
It follows from the strong dinaturality law\footnote{\index{Dan Doel}Dan Doel gave a proof using relational parametricity:
see \texttt{\href{https://cs.stackexchange.com/questions/131901/}{https://cs.stackexchange.com/questions/131901/}}} for any value $t^{:T}$ that:
\begin{equation}
t^{T}(\text{fix})=t\quad.\label{eq:fix-unfix-property-of-T}
\end{equation}


\subparagraph{Proof}

We use the strong dinaturality law~(\ref{eq:strong-dinaturality-for-church-encoded-fix-unfix}):
\[
\text{if }\quad r\bef f=f^{\uparrow F}\bef s\quad\text{ then }\quad r\triangleright t^{A}\triangleright f=s\triangleright t^{B}\quad.
\]
It remains to choose suitable values $r^{:F^{A}\rightarrow A}$, $s^{:F^{B}\rightarrow B}$,
and $f^{:A\rightarrow B}$ so that the law~(\ref{eq:strong-dinaturality-for-church-encoded-fix-unfix})
reproduces Eq.~(\ref{eq:fix-unfix-property-of-T}). Since the law
always involves applying the function $t$ to some arguments, while
the right-hand side of Eq.~(\ref{eq:fix-unfix-property-of-T}) contains
just $t$, let us apply both sides of Eq.~(\ref{eq:fix-unfix-property-of-T})
to an arbitrary value $s^{:F^{B}\rightarrow B}$, where the type $B$
is also arbitrary:
\[
s\triangleright(\text{fix}\triangleright t^{T})\overset{?}{=}s\triangleright t^{B}\quad.
\]
The left-hand side will have the form $r\triangleright t^{A}\triangleright f$
if we set $A=T$, $r=\text{fix}$, and $f$ a function that applies
its argument to $s$:
\[
f^{:T\rightarrow B}\triangleq u^{:T}\rightarrow s\triangleright u^{B}\quad.
\]
It remains to verify the assumption of the strong dinaturality law~(\ref{eq:strong-dinaturality-for-church-encoded-fix-unfix}):
\begin{align*}
 & r\bef f\overset{?}{=}f^{\uparrow F}\bef s\quad,\\
{\color{greenunder}\text{or equivalently}:}\quad & \text{fix}\bef(u^{:T}\rightarrow s\triangleright u^{B})\overset{?}{=}(u^{:T}\rightarrow s\triangleright u^{B})^{\uparrow F}\bef s\quad.
\end{align*}
Rewrite the left-hand side of the last line above:
\begin{align*}
 & \text{fix}\bef(u^{:T}\rightarrow s\triangleright u^{B})=f^{:F^{T}}\rightarrow s\triangleright(f\triangleright\text{fix})^{B}\\
{\color{greenunder}\text{definition of }\text{fix}:}\quad & =\gunderline{f^{:F^{T}}\rightarrow f\,\triangleright}\,(q\rightarrow s\triangleright q)^{\uparrow F}\bef s\\
{\color{greenunder}\text{unexpand function}:}\quad & =(q\rightarrow s\triangleright q)^{\uparrow F}\bef s\quad.
\end{align*}
This is equal to the left-hand side after renaming $s$ to $u$. $\square$

The Church encoding $T\triangleq\forall X.\,(F^{X}\rightarrow X)\rightarrow X$
of the fixpoint has a special property: for any other fixpoint $R$,
there is a unique fixpoint-preserving function of type $T\rightarrow R$
called a \textbf{catamorphism}\index{catamorphism}. To define the
catamorphism, assume that the fixpoint $R$ has a known function $\text{fix}_{R}:F^{R}\rightarrow R$
and write:
\[
\text{cata}:T\rightarrow R\quad,\quad\quad\text{cata}\triangleq t^{:\forall X.\,(F^{X}\rightarrow X)\rightarrow X}\rightarrow t^{R}(\text{fix}_{R})\quad.
\]


\subsubsection{Statement \label{subsec:Statement-catamorphism-church-encoding}\ref{subsec:Statement-catamorphism-church-encoding}}

\textbf{(a)} The function \inputencoding{latin9}\lstinline!cata!\inputencoding{utf8}
(defined above) is a fixpoint-preserving function.

\textbf{(b)} Any other fixpoint-preserving function of type $T\rightarrow R$
is equal to \inputencoding{latin9}\lstinline!cata!\inputencoding{utf8}.

\subparagraph{Proof}

{*}{*}{*}

\subsection{The co-Yoneda identities}

The Yoneda identities allow us in many cases to simplify type expressions
with universal quantifiers. Similar identities hold for existentially
quantified types. 

\subsubsection{Statement \label{subsec:Statement-co-Yoneda-two-identities}\ref{subsec:Statement-co-Yoneda-two-identities}}

For any functor $F$ and any contrafunctor $H$, the following identities
hold:
\begin{align*}
{\color{greenunder}\text{\textbf{(a)} }\text{covariant co-Yoneda identity}:}\quad & \exists A.\,(A\rightarrow R)\times F^{A}\cong F^{R}\quad,\\
{\color{greenunder}\text{\textbf{(b)} }\text{contravariant co-Yoneda identity}:}\quad & \exists A.\,(R\rightarrow A)\times H^{A}\cong H^{R}\quad.
\end{align*}


\subparagraph{Proof}

We use Eq.~(\ref{eq:existential-via-universal-Yoneda}) to express
$\exists A$ via $\forall A$. 

\textbf{(a)} We write:
\begin{align*}
{\color{greenunder}\text{expect to equal }F^{R}:}\quad & \exists A.\,(A\rightarrow R)\times F^{A}\\
{\color{greenunder}\text{definition of }\exists\text{ in Eq.~}(???):}\quad & \cong\forall B.\,\big(\forall A.\,\gunderline{(A\rightarrow R)\times F^{A}}\rightarrow B\big)\rightarrow B\\
{\color{greenunder}\text{uncurry arguments}:}\quad & \cong\forall B.\,\big(\gunderline{\forall A.\,(A\rightarrow R)\rightarrow F^{A}\rightarrow B}\big)\rightarrow B\\
{\color{greenunder}\text{contravariant Yoneda identity}:}\quad & \cong\gunderline{\forall B.\,\big(F^{R}\rightarrow B\big)\rightarrow B}\\
{\color{greenunder}\text{covariant Yoneda identity}:}\quad & \cong F^{R}\quad.
\end{align*}

\textbf{(b)} We write:
\begin{align*}
{\color{greenunder}\text{expect to equal }H^{R}:}\quad & \exists A.\,(R\rightarrow A)\times H^{A}\\
{\color{greenunder}\text{definition of }\exists\text{ in Eq.~}(???):}\quad & \cong\forall B.\,\big(\forall A.\,\gunderline{(R\rightarrow A)\times H^{A}}\rightarrow B\big)\rightarrow B\\
{\color{greenunder}\text{uncurry arguments}:}\quad & \cong\forall B.\,\big(\gunderline{\forall A.\,(R\rightarrow A)\rightarrow H^{A}\rightarrow B}\big)\rightarrow B\\
{\color{greenunder}\text{covariant Yoneda identity}:}\quad & \cong\gunderline{\forall B.\,\big(H^{R}\rightarrow B\big)\rightarrow B}\\
{\color{greenunder}\text{covariant Yoneda identity}:}\quad & \cong H^{R}\quad.
\end{align*}
$\square$

The Scala type \inputencoding{latin9}\lstinline!Any!\inputencoding{utf8}
closely corresponds to the type $\exists X.\,X$, which is observationally
equivalent to \inputencoding{latin9}\lstinline!Unit!\inputencoding{utf8}.
The advantage of using \inputencoding{latin9}\lstinline!Any!\inputencoding{utf8}
instead of \inputencoding{latin9}\lstinline!Unit!\inputencoding{utf8}
is that \inputencoding{latin9}\lstinline!Any!\inputencoding{utf8}
is understood by the Scala compiler as a supertype of \emph{all} Scala
types (while \inputencoding{latin9}\lstinline!Unit!\inputencoding{utf8}
is not). Indeed, for any type $T$ there is an injective function
$T\rightarrow\exists X.\,X$, This function corresponds to a function
of type \inputencoding{latin9}\lstinline!T => Any!\inputencoding{utf8}
in Scala:\inputencoding{latin9}
\begin{lstlisting}
def toAny[T](t: T): Any = t
\end{lstlisting}
\inputencoding{utf8} This is just an identity function that relabels the types; so, this
function establishes the subtyping relation \inputencoding{latin9}\lstinline!T <: Any!\inputencoding{utf8}.

\subsection{Exercises\index{exercises}}

\subsubsection{Exercise \label{subsec:Exercise-Yoneda}\ref{subsec:Exercise-Yoneda}}

Use a Yoneda identity to prove that there are no fully parametric
functions with this type:

\inputencoding{latin9}\begin{lstlisting}
def f[A]: Option[A] => A
\end{lstlisting}
\inputencoding{utf8}

\section{Discussion}

\subsection{Universally quantified function types cover all other types}

It turns out that all fully parametric type expressions are equivalent
to some type expressions that use only two type constructions: the
function type ($A\rightarrow B$) and the universal quantifier ($\forall A.\,F^{A}$).
If a programming language only supports these two type constructions,
one can write a library that implements all other type constructions.

We will now show the required type expressions and prove their equivalence.
The main tools in those proofs are the Church encoding and the Yoneda
lemma.

\paragraph{Void type}

The void type ($\bbnum 0$) is equivalent to the type expression $\forall A.\,A$.
To prove that $\forall A.\,A\cong\bbnum 0$, we may use the Yoneda
lemma:
\[
\forall A.\,A\cong\forall A.\,\bbnum 1\rightarrow A\cong\forall A.\,(\bbnum 0\rightarrow A)\rightarrow A\cong\bbnum 0\quad.
\]


\paragraph{Unit type}

The unit type ($\bbnum 1$) is equivalent to $\forall A.\,A\rightarrow A$.
To prove that, we may use the Yoneda lemma:
\[
\forall A.\,A\rightarrow A\cong\forall A.\,(\bbnum 1\rightarrow A)\rightarrow A\cong\bbnum 1\quad.
\]


\paragraph{Products}

The type $A\times B$ is equivalent to $\forall T.\,(A\rightarrow B\rightarrow T)\rightarrow T$.
To prove that, we use uncurrying and the Yoneda lemma:
\[
\forall T.\,(A\rightarrow B\rightarrow T)\rightarrow T\cong\forall T.\,(A\times B\rightarrow T)\rightarrow T\cong A\times B\quad.
\]


\paragraph{Co-products}

The type $A+B$ is equivalent to $\forall T.\,(A\rightarrow T)\rightarrow(B\rightarrow T)\rightarrow T$.
To prove that, we use the Yoneda lemma:
\begin{align*}
 & \forall T.\,(A\rightarrow T)\rightarrow(B\rightarrow T)\rightarrow T\cong\forall T.\,(A\rightarrow T)\times(B\rightarrow T)\rightarrow T\\
 & \quad\cong\forall T.\,(A+B\rightarrow T)\rightarrow T\cong A+B\quad.
\end{align*}


\paragraph{Recursive types}

A recursive type $T$ defined via the type equation $T\triangleq F^{T}$,
where $F$ is a covariant functor, is equivalent to the Church encoding
of $T$:
\[
T\cong\forall A.\,(F^{A}\rightarrow A)\rightarrow A\quad.
\]
This is proved in Statement~\ref{subsec:Statement-Church-encoding-recursive-type-covariant}.
By the inductive assumption, the type $F^{A}$ is equivalent to some
type expression containing only function types and universally quantified
type parameters.

\paragraph{Existential types}

We have the following equivalence:
\[
\exists A.\,F^{A}\cong\forall T.\,(\forall A.\,F^{A}\rightarrow T)\rightarrow T\quad.
\]
To prove this equivalence, begin with the Yoneda identity:
\[
\exists A.\,F^{A}\cong\forall T.\,\big((\exists A.\,F^{A})\rightarrow T\big)\rightarrow T\quad.
\]
It remains to show the type equivalence: 
\[
(\exists A.\,F^{A})\rightarrow T\cong\forall A.\,F^{A}\rightarrow T\quad.
\]
But this is just the definition of observational equivalence for existential
types.{*}{*}{*}

It is important that the type expression $\forall A.\,F^{A}\rightarrow T$
puts the universal quantifier \emph{inside} the function argument.
The type with both quantifiers outside, $\forall T.\,\forall A.\,(F^{A}\rightarrow T)\rightarrow T$
is \emph{not} equivalent to $\exists A.\,F^{A}$.

\paragraph{Free constructions in mathematics: Example I}

Consider the Cyrillic letter \foreignlanguage{russian}{ц} (ts\`{e})
and the Chinese word \shui~(shu\textipa{\v i})

We want to \emph{multiply} \foreignlanguage{russian}{ц} by \shui.
Multiply how?

Say, we want an associative (but noncommutative) product of them

So we want to define a \emph{semigroup} that \emph{contains} \foreignlanguage{russian}{ц}
and \shui~as elements

while we still know nothing about \foreignlanguage{russian}{ц} and
\shui

Consider the set of all \emph{unevaluated expressions} such as \foreignlanguage{russian}{ц}$\cdot$\shui$\cdot$\shui$\cdot$\foreignlanguage{russian}{ц}$\cdot$\shui

Here \foreignlanguage{russian}{ц}$\cdot$\shui~is different from
\shui$\cdot$\foreignlanguage{russian}{ц} but $\left(a\cdot b\right)\cdot c=a\cdot\left(b\cdot c\right)$

All these expressions form a \textbf{free semigroup} generated by
\foreignlanguage{russian}{ц} and \shui

This is the most unrestricted semigroup that contains \foreignlanguage{russian}{ц}
and \shui

Example calculation: (\shui$\cdot$\shui)$\cdot$(\foreignlanguage{russian}{ц}$\cdot$\shui)$\cdot$\foreignlanguage{russian}{ц}
$=$ \shui$\cdot$\shui$\cdot$\foreignlanguage{russian}{ц}$\cdot$\shui$\cdot$\foreignlanguage{russian}{ц}

How to represent this as a data type:

\textbf{Tree encoding}: the full expression tree: (((\shui,\shui),(\foreignlanguage{russian}{ц},\shui)),\foreignlanguage{russian}{ц})

Implement the operation $a\cdot b$ as pair constructor (easy)

\textbf{Reduced encoding}, as a \textsf{``}smart\textsf{''} structure: List(\shui,\shui,\foreignlanguage{russian}{ц},\shui,\foreignlanguage{russian}{ц})

Implement $a\cdot b$ by concatenating the lists (more expensive)


\paragraph{Free constructions in mathematics: Example II}

Want to define a product operation for $n$-dimensional vectors: $\mathbf{v}_{1}\otimes\mathbf{v}_{2}$

The $\otimes$ must be linear and distributive (but not commutative):
\begin{align*}
\mathbf{u}_{1}\otimes\mathbf{v}_{1}+\left(\mathbf{u}_{2}\otimes\mathbf{v}_{2}+\mathbf{u}_{3}\otimes\mathbf{v}_{3}\right) & =\left(\mathbf{u}_{1}\otimes\mathbf{v}_{1}+\mathbf{u}_{2}\otimes\mathbf{v}_{2}\right)+\mathbf{u}_{3}\otimes\mathbf{v}_{3}\\
\mathbf{u}\otimes\left(a_{1}\mathbf{v}_{1}+a_{2}\mathbf{v}_{2}\right) & =a_{1}\left(\mathbf{u}\otimes\mathbf{v}_{1}\right)+a_{2}\left(\mathbf{u}\otimes\mathbf{v}_{2}\right)\\
\left(a_{1}\mathbf{v}_{1}+a_{2}\mathbf{v}_{2}\right) & \otimes\mathbf{u}=a_{1}\left(\mathbf{v}_{1}\otimes\mathbf{u}\right)+a_{2}\left(\mathbf{v}_{2}\otimes\mathbf{u}\right)
\end{align*}

We have such a product for 3-dimensional vectors; but it cannot be
made to work for 2 or 4-dimensional vectors

Consider \emph{unevaluated} \emph{expressions} of the form $\mathbf{u}_{1}\otimes\mathbf{v}_{1}+\mathbf{u}_{2}\otimes\mathbf{v}_{2}+...$

A free vector space generated by pairs of vectors

Impose the equivalence relationships shown above

The result is known as the \textbf{tensor product}

Tree encoding: full unevaluated expression tree

A list of any number of vector pairs $\sum_{i}\mathbf{u}_{i}\otimes\mathbf{v}_{i}$

Reduced encoding: an $n\times n$ matrix

Reduced encoding requires proofs and more complex operations

\subsection{Beyond Yoneda: using parametricity to simplify quantified types}

The covariant Yoneda identity,
\[
\forall R.\,(A\rightarrow R)\rightarrow F^{R}\cong F^{A}\quad,
\]
and other similar identities hold for fully parametric code. If we
assume parametricity, we can also simplify certain expressions containing
quantified types where the Yoneda identities cannot be applied.

\subsubsection{Statement \label{subsec:Statement-quantifier-across-functor}\ref{subsec:Statement-quantifier-across-functor}}

For any exponential-polynomial functor $F$ and for any profunctor
$P^{X,Y}$, the types $\forall A.\,F^{P^{A,A}}$ and $F^{\forall A.\,P^{A,A}}$
are equivalent when restricted to fully parametric implementations.
The same holds when $F$ is an exponential-polynomial contrafunctor.

\subparagraph{Proof}

{*}{*}{*}

\subsubsection{Statement \label{subsec:Statement-existential-quandifier-via-Church-encoding}\ref{subsec:Statement-existential-quandifier-via-Church-encoding}}

\textbf{(a)} For any profunctor $P^{X,Y}$, the types $\exists A.\,P^{A,A}$
and $\forall B.\,(\forall A.\,(P^{A,A}\rightarrow B))\rightarrow B$
are equivalent when restricted to fully parametric implementations.

\textbf{(b)} Without the outer quantifier ($\forall B$), this property
does not hold: the types $(\exists A.\,P^{A,A})\rightarrow B$ and
$\forall A.\,(P^{A,A}\rightarrow B)$ are \emph{not} equivalent, at
least for some profunctors $P$.

\subparagraph{Proof}

{*}{*}{*}

\begin{comment}
jatin or the functional programming tutorial the focus of this chapter
is on three type constructions to begin let us consider the interpreter
pattern this is a design pattern where you present your program as
a data structure and you program an interpreter to run your data structure
so as an example consider this domain specific language for complex
numbers it\textsf{'}s a very simple language it has three operations to create
a complex number out of string to multiply complex numbers and to
compute the complex conjugate number if I want to represent this computation
as data in other words not to run it yet but to write down the operations
as data then I could imagine implementing it like this I can implement
some case classes in a disjunction like this so I have a program type
it has three parts of the disjunction which is either a string which
will represent this operation parsing a string into a complex number
multiplication of two complex numbers and computing the complex conjugate
number and then I can imagine that instead of this program I will
have a data structure with nested case classes like this in order
to be able to define such a data structure in my case classrooms should
have these types so for instance multiplication case class will contain
two parts and each can be itself another program so that\textsf{'}s why the
types of the parts of these schemes classes are again the type program
itself so in this way having defined these type parts of the case
class as programs I enable myself to write down arbitrary nested case
classes so this has type program I can use this as part of another
case class like MO or conjugate so in this way I have created a domain-specific
language that expresses computations with complex numbers as data
structures in order to actually compute anything with any complex
numbers I would need to run this program this dsl program as I would
say the interpreter will be a function of the type signature it may
be like this it will take an argument of type program and it will
return a pair of double numbers which would represent a complex number
that is the result of computing this program so why would you use
the interpreter pattern because it has certain benefits in certain
cases one main main benefit is that you represent a certain domain
specific language that is a number of operations that are specific
to a certain task set or a domain such as complex number of computations
you encapsulate all these operations in a data type that fully describes
what needs to be done without actually doing it so you present as
data what otherwise you would write as executable code data is much
more easily composable it can be manipulated transformed before running
it so before you run here DSO program you can store it in some data
structure you can put it on disk in a file read it back send it over
the internet and compose it with other DSL programs in the larger
DSL program all that is data manipulation that has nothing running
yet nothing has computed yet when you're ready you call the run function
and actually compute the results so this very simple DSL domain-specific
language has shortcomings specifically it works only with simple expressions
it represents expressions as unev a lated expression trees so every
operation needs to be some vertex of the tree but that\textsf{'}s that\textsf{'}s okay
but we don't have enough different operations defined so that for
instance you cannot express variable binding and conditions here for
instance we can imagine that this a could be used somehow in these
operations but I cannot express it here all I can express is multiplying
two complex numbers converting strings to complex numbers and computing
a complex conjugate number there is no way to express that I have
a variable in my language in the DSL not color variable Scala variables
I can of of course have I can say Val x equals this but that is not
at the level of the domain-specific language the language itself doesn't
know as so far anything about defining variables and because of this
I cannot use any code that is not expressed in this DSL so for example
I could imagine calling a numerical algorithms library to compute
some special function of the complex number and that could be a complicated
algorithm but I cannot put it into a DSL I would have to express the
entire algorithm using DSL operations if I wanted to do that so let\textsf{'}s
try to overcome these shortcomings these are certainly not due to
interpreter pattern itself it\textsf{'}s just that our DSL is too simple so
let\textsf{'}s see how we can do variable binding in a DSL like this now let\textsf{'}s
consider another example for this which is a DSL for reading and writing
files but let\textsf{'}s just look at reading files for now so the DSL will
have two operations first so this on the left is a non DSL program
is a program that we write in Scala we want to replace this with a
DSL program within data structure so the functionality we want to
implement is to create a path for a given file name so this could
check that it exists or whatever translate this into some URL if necessary
we don't know so right now we just say there is some operation that
creates Bath\textsf{'}s out of strings and there is another operation that
reads a file at the given path and the result is a string so you read
the contents of a file so then suppose we have this logic we read
one file and if its contents is not empty and then we interpret its
contents as another file name and we read that and then we want to
return the string that is in the second file if it\textsf{'}s an empty file
then we return an error string like this so how can we implement this
logic in the DSL well we need to bind a variable such as here STR
to a value that is computed by the DSL at runtime and we need to evaluate
some condition or generally we need to use the value of this variable
while constructing further DSL expressions so to understand how we
can implement this consider that in the DSL everything must be some
kind of expression tree and this part of the program needs to be also
represented by an expression tree and this expression tree is actually
a function of the variable STR so the variable STR will be assigned
when we run this DSL program and actually read the files but before
we do that the DSL already needs to specify that this entire rest
of the program is a function of this variable so in order to represent
that we need a special construction in the expression tree and I call
this construction bind which is just the name of a case class and
this case class will have an argument which is a function actually
a scholar function from a Scala variable STR to another tree and so
this is how I implement this domain-specific language again we I have
a seal trade program or probe which now has four case classes and
the three case classes here are the ones that I would need to implement
functionality so for example I need Val strings so I'll represent
that with this Val case class I need paths whose contents are maybe
programs again because I don't know path can be computed and I need
to read again I need to read something which could be another DSL
program so that\textsf{'}s what I do in these three case classes and I also
had a case class bind which represents binding enum DSL variable to
a value which is computed when you run the DSL and then so I have
the first part of this case class is a DSL program which when run
will give me a value of type string and the second part of the case
class is a function from string to another DSL programmer so this
function is a scalar function is not a DSL function is a scholar function
which is now part of my data structure in this way I can inject arbitrary
Scala code in principle in the code of this function including conditions
or creating another dsl program by using the values of these variables
in an arbitrary way so this variable will be the argument of F so
here is an example I make a bind so this entire thing becomes a blind
of this which is read path Val file and this which is a scholar function
that executes my conditional computation and then returns a value
of type program again so it returns a DSL program so this is a function
that takes a string and returns a program so that is how I can easily
implement the requirement that the DSL should express variable binding
conditional computations arbitrary Scala code in those calculations
and using the scala variables such as this one in creating expression
trees so I still have an expression tree this entire thing is still
expression tree and still undervalued but now I have a lot more flexibility
in what sort of computations I can implement with the DSL the interpreter
for the DSL will still have the same type signature it will be perhaps
slightly more complicated so let\textsf{'}s look at a code examples so first
the DSL for complex numbers which is what we saw before so the only
interesting code here is in running a DSL program and here\textsf{'}s how we
run we basically take the value of the program which is going to be
one of these three case classes and we match it in each case we run
what\textsf{'}s inside so in this strain case the inside is a definition of
a complex number by string such as this one we need to parse it so
I have some regular expression that I parse this with and the result
is going to be one value than a sign and another value and then I
create a complex number out of that a multiplication is a standard
formula for complex multiplication but notice that both of these are
programs so the mall case class contains two programs that first have
to be run in order to get a complex numbers out of them and then I
execute a complex multiplication similarly the conjugate operation
first I have to run the program that is the argument here and then
I execute the operation so here\textsf{'}s a test conjugate of multiply of
this which is equal to this complex number so in order to get it I
do run of program so when I do this nothing is wrong yet it\textsf{'}s a data
structure and I could have code that for example simplifies this in
some way maybe or prints it or whatever it\textsf{'}s a data structure that
is available for me to work with I could type set this in latex if
I wanted to before running it so then I also can run it so this is
the power of the interpreter pattern let\textsf{'}s look at implementing the
DSL for file operations that I described in order to run this I will
have a mock file system which we just a map from string to string
so that the filename is mapped to the text inside the farm so that
this is just so that my tests are easy and I don't need to write a
lot of code actually reading and writing files so I declare my probe
type as a disjunction like this like shown in the slide and now I
need to define the run now run is similar to what we had in the complex
number case and that for example I need to always run the arguments
first and then I do something with them so for simplicity path will
just evaluate your string and read will look up the file contents
in the dictionary so note we cannot guarantee that three is a path
here he is just a program it could it is evaluated to a string but
so maybe it surpassed maybe maybe not we have to be careful what right
right writing this program the runner cannot check what the program
makes sense and finally let\textsf{'}s look at how we implement the bind so
the bind is actually easy to implement so key as a type program we
need to run it to get a string out of it F has a private string to
program so we run the P then we apply F to that result which is a
straining so then F of string is another program which we again run
so that\textsf{'}s how mind works and that\textsf{'}s the entire implementation of variable
binding for our four in claim which for the DSL domain-specific language
here is an example program this is what is shown in the slide and
we can run it and see that it is equal to text this is equal to text
because first we read the file 1 which gives us the string and we
will read the file at this path which gives us the string so that\textsf{'}s
the text now notice that as DSL is not typesafe it allows us to read
to write nonsensical programs like this when you read read read and
that is nonsensical because you can only read the path and the result
of a read is not a path as a string and the program doesn't know about
it and just it gives us an exception key not found text which is a
kind of a runtime exception since we don't have a file named text
in our file system but this should not be the error the error should
be you cannot read a Val string you must read a class on the file
system which would have been a type error if this were a type safe
language so that\textsf{'}s the next concern our DSL so far has no type safety
every value in it is a program and it\textsf{'}s evaluated per string so what
can we do if we wanted to avoid errors such that for example read
of raid shouldn't even compile it should be impossible to write programs
like this and compile them and run so the way to solve this problem
is to change the type of the program data type to a type constructor
so let us denote by this program of a a DSL program that when run
will return a value of type a now in our case right now is going to
be string but let\textsf{'}s make sure it\textsf{'}s string and not some other type
such as a file fast so here\textsf{'}s how we do it we define a disjunction
type characterised by type a type parameter in and everything else
remains the same except now we explicitly say that for example the
argument of bind is a program that has a string result and a function
will take that result and output another program with string result
whereas previously bind had a program and this function returned the
program now we explicitly demand the result must be of type string
and then we can apply F to that result so Val will also give us a
program returning a string and path will take a program that returns
a string but it will heal the program that has the past in Iowa file
path type in other words it\textsf{'}s not a program returning string and the
read will take that kind of program and return a program that evaluates
the string so in this way we can achieve type safety so the program
remains mostly the same except for the type the interpreter remains
mostly the same except now it has type safety let\textsf{'}s see how that works
now so let\textsf{'}s implement instead of niño Java file types let\textsf{'}s just
have a mock type that represents a file path so now how do we implement
run you know it\textsf{'}s the same except now it\textsf{'}s impossible to have pass
in a program of type string so the program of type strain can only
be by and Val or read it cannot be a path because that\textsf{'}s a program
of type the are G of F path so we don't need this case here and instead
we just implement directly this case where we have a read of the text
because there is nothing else we can have so now the code is type
safe it still works the same code works but a program like this doesn't
compile this is a compile time error so we will not be able to even
create data structures that represent incorrect expressions that\textsf{'}s
the advantage of making the DSL file save our types a so here is our
DSL so far there is a problem with it which is it basically only binds
variables of type string it cannot bind variables of other types or
it cannot also return variables of other values of other types because
our runner returns string and requires a program of strings so still
the string type is very special and limiting us so for example we
cannot do this we must have a program that returns at rest so we cannot
directly read the past and if we wanted to compute this path using
a scholar program then we cannot convert this into a program of path
because there is no way to do that Val can only take a string and
extends program of string so let\textsf{'}s fix these problems now so first
of all let\textsf{'}s make Val a fully parameterize declaration so that it\textsf{'}s
Val of a and it returns program of a for any a and secondly let\textsf{'}s
replace this string by a parameter a as well so that would be a parameter
even will be an aid to program of B so we want to now have arbitrary
types instead of string here so we introduce two type parameters in
Bound and we will have this generality everything else stays the same
and except we get rid of this program of string and program of path
because now we have a Val and the Val can always convert a string
into a program of strain in the path into a program of path with no
problem so we don't need to have programs as types here if we need
this kind of thing we just bind a variable and we'll get get what
we need so now this is an interesting type because the signatures
of bind and Val are very similar to signatures of flat map and pure
if you look at this carefully so Val is of type a to program of a
bind is of this type program of a a to program of B and it returns
program will be so if we imagine that this is a function from here
to here then this is going to be just type signature flatmap and actually
it is in some sense a function binding dot apply is a function that
takes these as arguments and returns a value of this type except it\textsf{'}s
not just any function so type constructor so this function is defined
in a special way so essentially this type has methods of type signature
flat map and pure it looks like this type is a moment so let\textsf{'}s actually
define these methods flat map map and pure and it\textsf{'}s very easy to do
that flat map will just create a data structure with the case class
bind map we defined automatically by a flat map on theorem as we know
that in a wallet you can define Maps through flat map and pure the
pure is defined as just Val case closed so these methods don't actually
compute anything they don't run the DSL they create further unevaluated
data structures in other words these methods create DSL programs out
of previously defined DSL programs these are combinators innocence
but usually says then the combinators data are functions that take
values of some type and circle those and create new values of the
same type the advantage of Khalid affined in flat map map and pure
is that we can write the assault programs as functor blocks and we
can compose them very easily so for example if you look at the previous
program we have a bit of a repetition to have this read past Val which
we are using twice so how can we reuse that well they are easily make
a function that returns a string value program like this so we take
a path we read that files contents and this is a standard Scala syntax
for the Thunderer block as I call it for yield block and we can use
this syntax now because we have defined flat map and map in the program
trade and now we can easily combine and reuse the mimetic values in
another factor block so we can write this code as functor blocks as
we would do with any other moaner let\textsf{'}s see what the interpreter looks
like for this one I think DSL now I have full dramatization of types
defined viewer and map and flatmap and before as a shown in the slide
now there is a bit of ugliness in the runner because of the problems
with type pattern matching scholar has this problem where you want
to have image at least closet has type parameters it\textsf{'}s not easy to
do that so I have to do a bit of dancing around first I match the
bind and then I imagine a result I cannot put I parameters here that
won't compile I think maybe it will but I wasn't able to do it right
let me see if I can do it in one go Ashley you can see this was entirely
wrong a great simplification maybe my idea is not the Val and other
things are the same except now I have to do typecast so again Scala
is not great when you have to do type parameters on a case class and
the dual match expression so that\textsf{'}s a bit of ugliness but that\textsf{'}s not
so bad perhaps so let\textsf{'}s see that all of this actually works and so
now we are pretty happy we can a monadic DSL now it\textsf{'}s perhaps a little
too cumbersome because you need to define all these things every time
so if I wanted to define a DSL for complex numbers in the same way
as this DSL then I would have to add the bind and the Val and these
definitions every time so I would have to repeat this code every time
note that there are no code changes between this DSL for the file
operations and this DSL for complex numbers up to here so this is
completely the same the custom code starts later when we define some
more case classes so let\textsf{'}s refactor the DSL so that the common code
is separated and the custom code is just wrapped in some type constructor
called F so here\textsf{'}s how we do that we say there\textsf{'}s a DSL type constructor
that is paralyzed by the type a is minus 1 and it type constructor
F that will encapsulate always custom code so the type constructor
F will have the definition like this just like our first first try
at DSL so this could be once we add the type parameter this could
be the type constructor F and then we define just the typeclass the
case classes that are necessary to implement the moon add functionality
to bind and Val case classes and then we have this case class ops
for operations which contain a value of type F of n and so this is
a wrapper over whatever custom operations we have in our DSL and notice
here DSL does not have programmers parameters here has the complex
as parameters it\textsf{'}s up to us we could have programmers parameters it\textsf{'}s
a matter of convenience what important in it but whatever that is
it\textsf{'}s going to be encapsulated now in the case class ops so this entire
code is going to be generic in operations of your DSL so the type
constructor F represents the operations of the DSL and power the our
tech constructor DSL is a melodic DSL is permit rised by that tightness
factor f so this car is now engineered in the operations of your Union
now the cost of this is that the interpreter now needs to know how
to interpret your operations so you have to write extra code as opposed
to hard coding is you just write extra code that converts your operations
to values so you evaluate your operation so for example this operation
would be the domain specific part inside this F type constructor it
needs to be evaluated to yield this path so let\textsf{'}s see how that works
so the DSL will be general so all this part of code is generic it
does not depend on the domain all the domain-specific operations are
encapsulated by the type constructor S which is defined later in a
different place of your code so in this way we penalize by this type
constructor in this code is fully generic in any domain so now we
need to have the extractor the value extractor were evaluated for
your domain so this needs to be a function as I showed of this type
I would like to emphasize of this type is actually parameterize by
a typewriter a inside the expression it is not business for all a
I quantifier needs to be inside here it cannot be over there when
the reason is that when we run the DSL it takes a program that evaluates
a value of type a but intermediate steps could have different types
it could be that in order to compute a value of type a you first need
to compute some value of pi b r c and so on and when you run those
programs you need to extract a value of type b from some domain-specific
operation or a type c so you actually need to have a function in the
extractor that is parametrized by an arbitrary other type it\textsf{'}s not
going to be of the same type a as the runner and that\textsf{'}s why we cannot
use in scala just a type parameter you need to have an extra trait
that encapsulate inside but another type parameter so the extractor
is not parameterize by eight here this parameter is just by F and
it\textsf{'}s function applied or extract or whatever you want to call it just
has a single method and this method is paralyzed itself by the type
aid so in this way if the runner has a parameter which is the extractor
the runner is able to call the function extract on arbitrary types
here not necessarily the same type as this one so this I could rename
for clarity that\textsf{'}s necessary for the correct operation so that\textsf{'}s why
this parameter of the runner is not characterized by a it\textsf{'}s only parameterize
by F and inside that drag that that value there is a function that
works for every X for every type X so that is a little clunky in Scala
Scala does not have right now a good syntax that expresses such a
function but the cats library has a case class but essentially does
this so you can use that it\textsf{'}s called the natural transformation however
so in the cats library this will be like that it\textsf{'}s a natural transformation
which has code something like that now in this case F does not have
to be a factor you see if you look at our code for our domain-specific
language this program is not a factor because we have specific types
here now we do have a map function but only on the entire DSL the
F will only encapsulate these two case classes because we are now
separating the custom code from the generic monadic wrapper the binding
Val and the F type constructor will only encapsulate the domain-specific
code which has specific types here and does not have a map method
so it cannot be a functor it\textsf{'}s a partial type two type function that\textsf{'}s
only defined for specific type parameters and that cannot be a factor
and so it\textsf{'}s not really a natural transformation in the usual sense
because natural transformations are defined between factors but it\textsf{'}s
very similar it\textsf{'}s kind of a generic or maybe more general case of
a natural transformation which you don't need a name for it\textsf{'}s basically
this generic mapping from f of X to G of X for any given X so that\textsf{'}s
the Scala code that expresses this and this now needs to be the argument
of run so the first argument of run is this extractor and the second
argument of run is this DSL program so how do we implement run very
similarly to what we have before except now we have this extra argument
extract if you compare this with the previous Runner code same code
except we have run of extract here every time run extract run extract
and the ops case is slightly different very similar to those other
things but here we already take care of any custom operations because
the extract function knows how to evaluate them so extract of F is
the apply method which will give you an X out of f of X whatever X
might might be and so this ops doesn't know what type it is doesn't
it\textsf{'}s parameterize by a type and that\textsf{'}s fine so the result of this
extract is is an f of a and so some kind of f of a for unknown type
really not there\textsf{'}s necessarily the same as this a so let\textsf{'}s see now
how we use this so we now define a type constructor just for the custom
file operations so this is going to be the F here so we called File
ops and now this is just the domain-specific operations there are
no bind case constants were Vally classes we don't need those those
are going to be provided generically extractor needs to be defined
only for these so this is the domain specific code how to read files
how to create file paths or verify them or whatever this needs to
be so this is going to be actual domain specific code and then how
do we write programs the same way we just need to wrap our domain
specific operations in ops case class that\textsf{'}s all we could define helper
functions to have less boilerplate in this code but that doesn't really
matter right now what matters is that we are able to simply write
monadic code with their little boilerplate so imagine that all this
up here with a runner it\textsf{'}s completely generic it\textsf{'}s in the library
our code is just this it\textsf{'}s only the domain-specific operations and
then we just use the DSL type constructor from the and we're done
we use the oops from the library so cats library provides this it\textsf{'}s
called Freeman art and it works let\textsf{'}s see how we can use now this
is interesting yes L know we don't just want to rewrite code in a
fancy way we want to have value out of this generality so one example
of how we extract great value from this code is that now we can easily
handle errors so previously we evaluated a DSL of FA to a now we can
evaluate it to either of error and a all we need to do is to provide
a different extractor an extractor would be of this type so instead
of going FA to a and goes to fheo some error type was in it and the
code of the interpreter is almost unchanged except so this is the
same except the pure needs to put the or the Val case cause it\textsf{'}s a
pure function in the Monad it needs to put this into the right or
the either and the bind needs to use the flat map on the either other
than that it\textsf{'}s exactly the same so how does this code work well it
goes through the expression tree when it finds the bind expression
it will now use the flat map of the either when Al it will first run
the same it\textsf{'}s the same around functions recursive will run on the
P so we have a bind of P and F run the PETA gives you an either when
you use the flat map on that either with a function that runs on the
result and then applies the run to the result of the run ISM is a
curried function so that I can write this more easily F and then run
so the flat map here is from the ether moment and this is the pure
function of the ether moment so it\textsf{'}s very interesting to see that
the code of runner only uses flat map and pure from the ether moment
it\textsf{'}s not otherwise aware of the fact that we are running to evaluate
things into the ether moment and that\textsf{'}s very good because it means
we can very easily generalize to any other model except instead of
this one so let\textsf{'}s look at the code for the either one and the way
to do that so see the program remains the same we do not change the
program at all we just evaluate it into a different unit into the
ether moment instead of evaluating it to just the value a which is
actually the identity moment so previously we evaluated the code into
the identity monad now we are going to evaluate it into the either
movement so all we need to do when define a new extractor which are
called e\textasciicircum x just for brevity a new Runner which is
aware of an arbitrary well it\textsf{'}s aware of the either one I'm actually
not arbitrable not yet and I just rewrite things a little bit so that
I compile as : Scala and I find that these type parameters are required
but that\textsf{'}s all right so the code works in the same way as in the slides
and it applies this functional F which is of type that it doesn't
know it\textsf{'}s not really of type in E it\textsf{'}s of type type parameter that
I have here but it doesn't know that so this function f computes a
DSL program which I then run so I applied a runner to that program
so that\textsf{'}s exactly the same code as I had before except I'm extremely
inserting a flight map from the either Monat and here is the implementation
of the extractor so need a new extractor which will run domain-specific
code and catch exceptions so I would very easily do it like this now
I have an extractor that takes my file operations and from file operations
of a gives me an either of throneworld name so that\textsf{'}s all I need to
run my program now you see I running exactly the same program as before
I did not have to change that code but domain-specific language and
I wrote here in order to add error extraction arrogantly that is a
great power so I can just replace the extractor here and I run exactly
the same program so this program could be computed by one part of
the code and the extractor could be prepared by another part of the
code completely independently and here I have shown how we can interpret
the program with the result being an either so I call this to interpret
the DSL into a monad and so here we interpreted this DSL into the
ether moment we can just as easily interpret it into any other modern
by adding the moolaade here is a type parameter and getting rid of
either here and that\textsf{'}s it the changes will be minimal because we're
not actually using a specifics of either here we'll have to replace
this by pure that\textsf{'}s all so let us see what the resulting construction
actually is we start with an Operations type constructor denoted by
F such as this one v oops so this is a type constructor that needs
to have a type parameter and it needs to encapsulate your domain specific
operations in a very special way namely it takes the arguments of
the operations as parts of the case class and the return type of the
operation becomes the type parameter of this type constructor so that\textsf{'}s
the encoding this is because I remind you that this means a program
that when run will compute a value of type a so this is a program
that when run will compute a value of type path and that\textsf{'}s how we
encode domain-specific operations so this operation could be a function
from string to file path and this is a function from file path to
string so that\textsf{'}s what we need to encompassed and often this type constructor
will be not a factor maybe it will be a partial type to type function
not a factor that needs to be a total type to type function always
then we use this DSL which is a library construction that is written
once for all F the interpreter again has written once for all F and
then we run that program once we prepare a program value actually
which we can do using a functor block or in any other way or we can
do it directly using helper functions for instance we do ops of something
and so on flat map we can just write by hand map flatmap and so on
so in this way we can prepare a value of this type by combining value
so this is very composable it\textsf{'}s pure value it doesn't yet run anything
can be stored in variables and arrays whatever you want then you prepare
an extractor value that will run just your operations or your custom
operations and represents their values in some monad so this mu naught
can be identity mu naught if you already want just the final results
it could be an error gathering monad it could be some other moment
for instance could be a state monad if you want to represent your
operations purely as a state updates or someone base it can be a combination
of monads it can be anything that is a monad it could be another DSL
with a different F it could be anything so once you have this extractor
you run the program like this and this computes a value of that moment
which could be just a or it could be error something or and so on
so to summarize so far we begin with a number of operations and these
operations could have these types we define a type constructor then
like this typically well this could be more arguments and I'll have
more parts in the case class if I have no arguments then I would have
a case class with zero arguments but I need to have a type here so
usually a domain-specific language would have functions like this
with some return types so you just put them into your type constructor
like this and then you do what I just described now there are some
other things you can do which I will not discuss in a lot of detail
in this chapter for instance you can choose a different monad and
then you can interpret this value that you have into another moment
so this transformation you can define separately and if this itself
is a different DSL created in the same way then this will be the runner
for that DSL that evaluated into yet another will not in so this could
be very useful if you want to say test your your program so you have
exactly the same program and you run a test interpreter into someone
and that catches all the calls to something and prints Diagnostics
or whatever or you could give design a different kinetic DSL that
is more optimal let\textsf{'}s say more low-level and then you can have a sophisticated
optimizer that translates one DSL into another and the second DSL
will be run later in a yet another runner you can use monad transformers
since this is a monad API and you can combine these cells very easily
using disjunction so you have several factors or type constructors
not necessarily factors you can define a disjunction factor and the
DSL of that contains all the operations from each of these factors
in a single DSL so in this way you can define separately several dia
cells using these different F\textsf{'}s you could have F G H and so on and
then you put all of them at once into a monadic wrapper so this is
what I call DSL of something is really a monadic wrapper over type
constructors so these are all the benefits that you get by modeling
operations one article so let\textsf{'}s see whether this dsl program respond
which i and keep calling it an attic DSL is it really omona does it
satisfy Monad laws it turns out that no it does not satisfy mana flows
but it actually does satisfy them once you evaluate the program once
you interpret it so after you run the program that\textsf{'}s when the Monad
laws are satisfied and that is a very interesting property let\textsf{'}s see
why that is so so consider one more nut law this this is one of the
identity laws so flatmap applied to pure must be identity let\textsf{'}s see
if this is so now both sides of this law are functions on the moon'll
so Munna is this so it\textsf{'}s a function from this to this so we need to
apply both sides to some arbitrary program of this type and we need
to get the new value and see if that value is the same because that
should be identity so let\textsf{'}s see so what happens if we take a program
and we execute dot flatmap of pure on it now flatmap according to
our definition just makes bind data structure since nothing is really
evaluated we'll just put more these classes on the data structure
so that that is going to be the result now this value is a new data
structure it\textsf{'}s not equal to program PRG it cannot be equal because
it contains that thing inside a case class it cannot be itself equal
to a PRG so it means that this monitor law fails and we find that
other laws also fail because those laws usually say that something
is equal to something but all our operations if you look at the implementation
of flatmap and mount all they do is put more case classes on top of
things they don't actually simplify anything ever so for this reason
it cannot simplify this to PRT it will create a new bind and all the
other monad operations will create new case classes and never reduce
anything so basically the laws fail if you demand that they hold Vally
like this so our data structure DSL is not a lawful munna it does
not satisfy the laws but once you interpret this data structure into
a target monad and assuming that this monad satisfies the Lord then
the resulting values will satisfy the Lord and that\textsf{'}s a very interesting
property let\textsf{'}s see how that works so let\textsf{'}s run this value so how would
you run this value if you apply a run to this and by definition of
the code it needs to first run this and then apply flatmap with this
function and then run the results of this function so that is the
code and if we now symbolically evaluate this code will find that
the runner of the Val it will just give you a write of a let\textsf{'}s say
in the ether moment it will be really pure of a in general but I'm
just substituting the code from the previous line and because this
is a pure for the either moment the either moment has the wall satisfied
and so flat map of error is identity and so the result will be equal
to running the program PRG so in this way assuming that the laws will
hold for the monel m this both sides when we run them will won't be
the same so all other laws also hold I will show that next but think
about what it means it means that the violations of the Monad laws
that this data structure has are not observable once you run the computation
so the data structure and they have some extra information inside
that gets computed away it gets reduced or simplified when you run
or when you evaluate this into or interpret this into some target
unit so in this sense I would say that the moral law violations are
not observable when you actually observe or run or interpret this
program there are no violations so these violations are hidden somewhere
in this data structure and they don't change the results they don't
make the results invalid and so it\textsf{'}s okay to have those violations
so let me show you now in the code why the moon at law was called
after evaluating entire law faloona so we will reason by taking an
arbitrary DSL program and just denote by M the result of running this
program for brevity and let\textsf{'}s see what happens when we run monadic
operations on this program so for example let\textsf{'}s say that program is
a pure of something when we run that then we execute the code of the
runner and that code is a pure in the case of the either mona this
was the right of X but in the case of a general one other will be
pure of X so therefore running the pure of the DSL gives you the pure
of the target monad let\textsf{'}s now run the map in the DSL and get some
other ESL program with some arbitrary function f and by definition
is going to be translated into this and we run this we have to translate
that into flatmap because that\textsf{'}s how buying is translated and then
we get this combination now we know that when we do run dot flat map
this is a flat map in them monad M now if we look at this this is
a run of the pure so that is already as we know Emma dot pure so now
we have a flat map in the moon at M of F followed by pure so that
is the definition in the monad M of map so now this is equal to map
in limited M in other words running the results of map in the DSL
gives you the result of melt in the target monad and the same happens
with flat map if you run the result of flat mapping in the DSL which
is another DSL program and F is a function from some type a to a different
ESL program now we still need to interpret the result of this F in
the mana dem so this will give us a function G of this type instead
of a going to DSL of B it\textsf{'}s going to be a going to M of B this function
is like this is f and then run so now if we interpret the bind it
is going to be the flat map in the model M of F and then run and if
you just look at what that is that\textsf{'}s the function G that we defined
which is the evaluating of the result of the function f so in this
sense evaluating flatmap first in the DSL and then running the results
is the same as evaluating in the Monad M with a function G which is
obtained from F by running its results so in this sense all the Monad
operations in the DSL are directly translated by the interpreter into
the corresponding one at operations in the target one of them now
if we consider the laws it\textsf{'}s very easy to see that they hold after
interpreting now we already saw that in the slides for this right
identity long let\textsf{'}s look at the left identity law this it needs to
be verified we have apply run to both sides and we have to show that
run of this is equal to run of that so let\textsf{'}s evaluate the run if we
do the pure flat map then this is translated into that we run that
get run of Val which is just M of pure so you have a pure followed
by flat map of this but pure followed by flat map is going to be in
the Monon m and that is equivalent to just this function which is
G so that\textsf{'}s why the run of the two sides is the same because the run
of this is G of X the natural T law for pure is like this so the DSL
peer of X of f of X is the slf map of f of dsl P of X so now if we
evaluate run on both sides then this becomes ampere this becomes MF
map this becomes ampere so now obviously this hold because M has this
law too and finally associative 84 flat map it is this one so that
lets apply both sides to some program PRG and then apply run to both
sides so we have the run of this should be equal to the run of that
so if we now simplify this into the Monad m operations then we get
this now this flat map G is still a bit complicated because G is not
yet run in the moon and heaven into the moon and M so let\textsf{'}s use the
law and let\textsf{'}s rewrite somehow this expression so that we get associative
et law for the moon at M now the left hand side is this and it should
be equal to run of this which is flat map of F and the run flat map
of G and then run now notice these flat maps are in the m1 had this
flat map is in the M walnut but the argument of that flat map is complicated
so we do have the same law for the moon at em but we just need to
rewrite this a little bit so because that this is going to be M flat
map something flat map something needs to be simplified into M flat
map this and then that so how do we figure that out we rewrite this
complicated expression as an explicit function from a to - what well
first we apply F to a size F of a then we apply map of G which is
this and then we apply run so let\textsf{'}s run over all this so let\textsf{'}s simplify
now so run of F of a now if we run a flat map that\textsf{'}s the same as running
this flat mapping of running that so what\textsf{'}s this and equivalently
we can say this is just F and then run applied to a and then this
is flat map genome then run so if we get rid of this a now then we
get just a function f and then run and then M\textsf{'}s flat map of G and
then run so that\textsf{'}s exactly what we have in the associativity wall
for EM it\textsf{'}s M flat map of this is equal to that so now FM and GM are
just these FM then run is FM G and the run is G M so we get the associativity
law the naturality was for flat map could be verified as well we don't
need to do that since our code is purely type parametric and naturality
is automatic for that code so I mentioned that this construction is
called a free Monat and in the cat\textsf{'}s library is called free why this
word free what does it mean free why do we call it a free construction
well this terminology comes from mathematics in mathematics usually
free construction is a group or mono end or vector space or some other
kind of right construction that is generated by certain data with
no constraints so free means no constraints so let me illustrate this
is a bit vague so let me illustrate in two by two examples consider
two things and I will choose things that mean very little by themselves
the Russian lettered said and the Chinese word way the water say it
doesn't really mean anything by itself it\textsf{'}s just a letter of acrylic
alphabet and the Chinese word sway it means water but it doesn't matter
for now so now suppose what I wanted to multiply them I wanted to
multiply say by Chui so what does it mean to multiply how would I
multiply them so mathematicians first asked what kind of product do
you want do you want associative commutative distributive product
so let\textsf{'}s say we want an associative product not necessarily commutative
so mathematicians would then say very very well what you want is to
define some kind of semi group in other words a structure that has
an associative but not necessarily too negative product and you want
a semi group that contains say and Shui as elements that\textsf{'}s what you
want you don't and and you would say well but I have no idea what
these are would say in Shui is I've no idea no no worries I'll get
you a semi group that contains them and if you have a semi group that
contains on a semi group is a set and these will be elements of that
set and if you have a semi group that contains them then you can take
a product of them so here\textsf{'}s how the mathematicians would do it they
would consider the set of all unevaluated expressions of this kind
any onion valued expression with the multiplication sign or a product
symbol dot which I have here and one of these symbols say say or Shui
so this would be an unrelated expression this will be another undervalued
expression but we will have the law that this product is associative
so see this expression isn't equal to another letter of the Russian
alphabet or another Chinese word it\textsf{'}s not equal to any of those things
it\textsf{'}s just an expression that\textsf{'}s not evaluative it\textsf{'}s a new thing so
we have a set of a lot of new things and say and Troy is our one of
those things but there are a lot more of those things in the set because
we are considering the set of all unevaluated expressions of this
kind so the set of all these expressions is called a free semi group
generated by of the elements say and Shui and in some sense it\textsf{'}s the
most unrestricted semi group that contains these two things you could
have a lot of semi groups that contain these two things as elements
but this one is the least restrictive it\textsf{'}s the most free of all arbitrary
restrictions as long as of course you have associative 'ti of multiplication
so you can calculate in this semi group for example this is a calculation
that I can do I take these two expressions I take their product and
then I multiplied by this expression and I get this expression as
a result these are calculations that I would do in this free seminar
and what would I do with that well I could interpret the semigroup
value into another semi group for example integers imagine integers
as a semi group with multiplication as a semi group operation I say
that say is 17 and Troy is 3 so then these are just going to be 3
370 370 will take a product of all of those and I have a number so
I have evaluated this so in other words this is going to be some kind
of symbolic program that will later be evaluated in some way and that\textsf{'}s
very similar to what we have been doing with our DSL was a symbolic
program that was interpreted at the end into a specific values but
we can do calculations like this before evaluation and this is a similar
to combining parts of a DSL into a larger DSL program and while we're
doing this we still have the illusion we are performing these operations
so how do we represent this as a data type now the easiest thing and
what we have been doing so far is what I call the tree encoding in
other words we represent the free semi group as a full expression
tree so here\textsf{'}s an example each operation of product is just a pair
in the data structure so I have a tuple of this and this and I'm missing
one parentheses on the left I will insert that in the slides and after
the recording yeah so I have a tuple and this tuple represents the
free product of the tube Shui then I have this tuple which is a free
product then I have a free product of these two and finally a free
product of the result and it\textsf{'}s a and so that in this way I represent
my expressions it\textsf{'}s very easy and operations are very easy to implement
because in order to do for example multiplication I just put the two
parts into pop and I'm done so this is exactly equivalent to adding
one more case class on top and having a nested structure and in this
way I implement all my required operations but there is a another
encoding which I call reduced encoding and this encoding is smarter
it is less redundant and in this case it\textsf{'}s going to be a list of all
these things taken in this order this list is equivalent to what you
would write on paper because the associativity law means that it doesn't
matter where the parentheses are you can omit all parentheses and
they will still get the correct result and so since we know about
that we are clever and smart and we realize that the list of these
things in this order is sufficient it is sufficient information to
represent a value in the threesome Anoop now if we want to implement
the multiplication operation you cannot just put the two lists in
a tuple you need to actually concatenate the two lists and that could
be more expensive depending on your implementation of Lists it could
be a very quick Big O of one operation or it could be a more expensive
operation but this structure has no redundancy whereas this structure
has redundancy you could put parentheses in different order and it
will be a different expression tree although the final value is supposedly
the same let\textsf{'}s consider another example which is a product of n dimensional
vectors so what if I wanted to define a product of two n dimensional
vectors or we have such a product for three dimensional vectors this
is the well-known vector product in the usual euclidean three-dimensional
space but let\textsf{'}s ignore that and in any case I want product for n dimensional
vectors with any n and that doesn't seem to be generalizable from
three dimensional vectors so how do I do that all a mathematician
again will ask me what kind of product do I want I say well it\textsf{'}s a
product of vectors so I expect it to be linear and distributive not
necessarily commutative but I want a product that has these properties
for example I want to be able to add so linear means I supposed to
be able to add different products together and that should be associative
and I'm supposed to do this so if I have a linear combination of vectors
under a product I should be able to pull this thing out and expand
the parentheses and that\textsf{'}s a distributive law and the distributive
law should hold for left and for right as well all right says the
mathematician you need a free vector space generated by all kinds
of pairs of vectors from your own dimensional space so let\textsf{'}s do it
in this way we consider all unevaluated expressions of this form where
u and v are arbitrary vectors from your n-dimensional space so this
is a the first step the second step is to impose the equivalence relationship
so before this you gather just a free vector space you have all all
possible linear combinations of all possible products that\textsf{'}s the first
step the second step is to impose equivalence relations so you will
consider certain pairs of expressions to be equivalent according to
these laws the result is usually called the tensor product of vectors
and again we can have two in codings for the tensor product the first
encoding is the full onion valuated expression tree and that will
be just a list of these vector pairs and that could be a very inefficient
representation if you have a lot of those pairs but it could also
be a very efficient representation if you have a very sparse tensor
product the reduced encoding that is the encoding that has no redundancy
is to represent tensor product as an N by n matrix of vector coordinates
in some basis now reducing this expression to the matrix form requires
computation and it could be well first we need to prove that you're
encoding is adequate that for example this expression and this expression
always corresponds to the same encoding and then your laws would be
satisfied your preferences will be satisfied and any component operations
so we'll translate this into matrix and add matrices and so on but
do that so that\textsf{'}s a choice so this is why we use the word free construction
so basically we can use the mathematician the mathematics intuition
to implement data structures with properties generated by things that
don't have these properties you see the the common topic here is that
I wanted to define an operation for things that don't have this operation
like I wanted to multiply a Chinese and Russian together its word
and the latter it\textsf{'}s it\textsf{'}s not defined but I wanted to define it in
some way and I can in a free way so in the programming language we
just saw an example where I was able to define a monad out of a type
constructor that isn't even a function let\textsf{'}s look at some other examples
and here would be an example of a semigroup that\textsf{'}s generated by two
types so that\textsf{'}s kind of similar to my chinese and russian example
so how do we define that so let\textsf{'}s see how that works so let\textsf{'}s call
it FS is which is free semigroup from integer and string so a value
of FS is could be an integer or it could be a string also or if x
and y are already of type of a silenced and so is this combination
of ex-wife co-come the case class so i straightforwardly translate
this specification into the datatype and this will be the three encoding
it\textsf{'}s a full expression tree unevaluated and but that\textsf{'}s okay it\textsf{'}s a
good encoding for some usages the short type notation for this is
going to be this is recursive type that is defined by this type equation
so let\textsf{'}s think about how we can use it now if we have an actual semigroup
as a specific 7u and we know how to map integers and strings into
that same group then we can map this FS is interested in you that\textsf{'}s
our interpretation so let\textsf{'}s see how that works it\textsf{'}s a little too specific
with integers and strings let\textsf{'}s just put all of these domain types
into a type Z and make that type of parameter so then the three encoding
would look like this it\textsf{'}s a recursive type that\textsf{'}s defined like this
so I omit the Scour definition let me just write the definitions of
the methods so the method of semigroup operation is very easy I just
put the two arguments into a case class and the run method takes a
semigroup and an extractor function which Maps my Z into a semigroup
and that\textsf{'}s equivalent to the two functions that I assumed here before
just a single function from Z to s so then I get a function from my
free semigroup generated by Z to us how would that work I match on
the free Simon group it has two cases the case of F well I call it
rap here let\textsf{'}s call it f then I just oops I just extract I have a
value of Z and I call this function extract and to extract the value
of semigroup s from it and if I have a combination then I first run
these two and then I get two values of type s and I just combined
them in the seven group operation of s quite similarly the semigroup
laws will hold after I try this run they did not hold before applying
rather why is that it\textsf{'}s well it\textsf{'}s very easy to see that social Timothy
does not hold because I would have a comp nested in different order
and that\textsf{'}s not equal so it\textsf{'}s only after applying the interpreter that
laws will start holding and the reduced encoding is a non-empty list
of Z\textsf{'}s so that\textsf{'}s a reduced encoding actually I should have said here
it\textsf{'}s non empty list I didn't make that that remark MFG lists cannot
be constructed because you have to start with either sell or Shui
and apply the semigroup operation there is no empty value possible
so that\textsf{'}s why it\textsf{'}s a non empty list and then the combination operation
will require when you run this you'll have to concatenate the lists
but maybe the run operation will become faster because then you have
fewer structures to traverse as another example let\textsf{'}s implement implement
the feel annoyed the Fremen are generated by type Z it\textsf{'}s very similar
to a free summer the value of free monoid of Z can be empty because
it\textsf{'}s a monoid or it can be a Z and then you have a multiplication
so I should have called it comm not law so therefore the female noid
of Z in the tree encoding has these case classes the empty the wrap
which has the inside and the chemical combination which has two values
of F M of Z inside the short type notation for this is just like that
so here\textsf{'}s an implementation of there brother the plus operation simply
puts to the occasion top and the runner just does the same thing as
before and it puts the MS empty and Emma being Illinois it has an
empty element instead of this so when we interpret this tree structure
we just substitute specific operations of the monoi M except for the
wrapped case when we use the extractor and Malloy Clause will hold
after we apply this function so this was the tree encoding and the
reduced encoding is just a simple list where this operation is concatenate
in the lists the empty is the empty list and the wrap is a list over
one element and so it\textsf{'}s interesting actually to notice that after
running the trillion coding and the reduced encoding would give you
the same result there are just different in coulombs of the same value
there are not equivalent in terms of their performance perhaps and
memory requirements are different our equivalent in terms of the resulting
value let\textsf{'}s look at the code so here is an implementation of the free
moderate generated by type Z so Z is some domain specific type and
we have this combination and we just implement what I said in the
slides and here\textsf{'}s an example of using this definition so first I define
an annoyed of integers in the standard way and then I want to do a
free monoid over this this was my example in the slides so I define
Z to be that then I if I an extractor extractor is a function from
z to integer so how do I do that well if I have an integer I just
leave it there if I have a string I have length of the string it\textsf{'}s
just for this illustration so now I construct a free monoid value
so how do I do that well I use the wrap constructor to do specific
values of Z so either left of interest right of strength so I wrap
them and then I combine them with the plus operation so this is a
free monoid value which I can then run with my extractor and the result
is 16 because it\textsf{'}s 12 and then 3 the length of this and then 0 because
it\textsf{'}s empty and then one so all this must be added so that\textsf{'}s why it\textsf{'}s
16 so let me also verify that the monoid laws would hold after running
so let\textsf{'}s just maybe make extract into an implicit argument and not
not right every time or something just I'll just run of excellent
yes miss oh shit a beauty law so I run this and I should get the same
result as when I'm running it with people other order of parentheses
when I run this I run over this structure now you see this structure
still has the information about the order of parenthesis but when
they run it each comb is translated into the monoid operation plus
in the target memory M and so when I run it the second time I get
this result which is in the target monoid m and it has now no more
information about the order of parentheses and so when I run the other
order of parentheses I get the same result let\textsf{'}s check the identity
law this must be equal to the result of running X now this is not
actually equal to X because it\textsf{'}s this combination this class so as
usual the laws do not hold before you run because you are piling up
case classes but when you run that iran identity that becomes m empty
then you're on of X and that\textsf{'}s a monoid law in humanoid m that this
should be equal to run of X and so running of empty + X gives you
the same result as running X and the same will be for the other order
now in the reduced encoding it\textsf{'}s obvious that all of this works because
it\textsf{'}s just a list we know that list as I will know it so there\textsf{'}s not
much to implement and the runner however needs to go over the entire
list so the runner I'm implementing it using a fold over list and
I'm folding with the monoid operation in the target one with and I'm
running exactly the same code as before with pretty much the same
code except here I'm using a helper function to wrap my values I get
again exactly the same result so what if we interpret this free semi
group that we had before into another free semi group well that would
be an interesting thing to do in general we can interpret if we have
so for example free semi group generated by Y into a free semi group
generated by Z we can interpret if we have an embedding from Y into
the free seminar of Z that is certainly what we can do but we know
it\textsf{'}s a free semi group so what if we just haven't been emitting from
Y to Z not from Y to the free Simon group of Z there\textsf{'}s a free semi
group is a big thing it\textsf{'}s not maybe it\textsf{'}s much easier to do this indeed
that\textsf{'}s very easy because we just need to map this into that and it\textsf{'}s
straightforward because this is a fun trip so this type constructor
is a factor as you see it has the type parameter always in a covariant
position for positive position so this is a standard code that you
would write with your eyes closed to implement the map for this function
so now we can use that and have a chain like this we first map map
and then run let\textsf{'}s think about how we can simplify this well first
of all this is a functor so functor laws hold for its of' map is composable
we can compose these two functions from X to Y and from Y to Z into
a single function from X to Z and just F map once instead of F mapping
twice what\textsf{'}s interesting is that the interpreter also composes with
F map in a way and this is done by this diagram so if you first so
I'm killing the Z here so I have just FS x FS y and s if the first
F map X to Y and then run through some function G that is the extractor
from Y to s we should get the same result as when we are running with
the composition of these two functions indeed that is a law that the
interpreter satisfies and we can combine the semigroups in this way
and we can also combine them in disjunctive way why is that well consider
this semigroup we have obviously an injection from X to the disjunction
X plus y so then we can F map it and we automatically get this injection
which means that a free semigroup generated by a disjunction of some
types contains a free seminar generated by one of these types so in
this way we can combine semigroups in easily if we know the types
of free semigroups to combine free semigroups if we know the types
from which they were generated so next we will consider what we can
do further to simplify mapping free semigroups to different targets 
\end{comment}

\begin{comment}
if we need to map a free semigroup into multiple targets in groups
say s1 s2 and so on then it would require many extractor functions
with this type signature each extractor function will have to convert
the generating element Z over the free semi group into a specific
segment Rufus 1 as 2 and so on we can refactor these extractors it
into evidence of a typeclass constraint so instead of saying we have
a semi group s and we have this function for that semi group it would
say we have a semi group s that additionally has a typeclass constraint
and so we define a new typeclass let\textsf{'}s call it X Z for extracting
from Z and it has a single method of this type signature then we can
refactor the run function into this form it will be now parameterize
by a semi group s that additionally to the semi group typeclass also
has an extract Z typeclass instance and that would mean would have
an evidence on value of this type which would contain this function
so that\textsf{'}s very similar to what we had before when we had the run method
it had an argument containing the extract now we will have no such
argument we'll just have an argument specifying the free semi group
value and additionally we'll have a typeclass constraint which in
Scala is translated into an implicit argument of the type exe of Seminole
which will just contain this extract function so far this is a refactoring
that doesn't seem to bring a lot of benefit except that now this code
is going to be completely the same for all extractors and we just
need to define different extractor typeclasses for different semi
groups so another refactoring that will follow from this is found
if we look at the structure of this run function so what does it do
it translates the free semi group value into a value of the specific
semi group s by pattern matching on the case classes from the free
semi group and the free semi group has two case classes the rapp and
the combined case class what the run does is that it replaces these
case classes by some fixed functions and these fixed functions are
permit rised by a semi group having this extractor constraint so all
we have done is first we have created a value of free semigroup which
will be some case classes and then we just translate these case classes
mechanically into these fixed functions so the main idea of what is
called the church encoding is to represent the free semi group directly
by these functions just skip the case classes all these cases class
case classes do is to denote what needs to be done what these functions
will have to do when we run the free value so instead of representing
a free semigroup through these case classes represented directly through
these pieces of the run function in other words instead of saying
that the free symmetric value is of type rap we say it is equal to
this function which will be terrorized by this semi group s with two
typeclass constraints so here\textsf{'}s what will happen if we do that we
will have two functions so one would be the combining function and
one would be the wrapping function but the combining function actually
is defined in the semi group s it is not something we define so really
we just need to define the wrapping function so this wrapping function
will be this part of the room and the combining function is already
defined because the plus operation is part of the semi group typeclass
so the definition of the free semi group just becomes the definition
of a wrap function which is parameter I'll begin by by this we don't
need the semi group constraint right now for this function we could
have written it but we wouldn't have used it and then suppose we want
to define the value X of type free semigroup which would be say combination
of wrapping one and wrap in two instead of doing that we just write
down this so you know what these are values of the free semi group
and these values are now deaf because they're they're not vowels their
deaths because they're actually functions parametrized by a type parameter
and having implicit arguments so they cannot really be valid anymore
because they're parameterize by a type parameter that\textsf{'}s another difference
so now we have encoded this X so this X is basically a function that
already runs it\textsf{'}s waiting for you to give it a semigroup yes but once
you give it then it will run and all the implicit arguments will be
substituted and you will have a value in your signature but until
then you have defined it and it\textsf{'}s waiting for you to run it so this
is then the encoding of the free semigroup using functions using directly
pieces of the run function so we don't need a run function anymore
we already encode values on the free semigroup through the pieces
of the run function that would be run in the previous encoding so
the previous including is a tree encoding or the expression tree encoding
this encoding is called the Trojan Colin let\textsf{'}s look at the type of
X explicitly let\textsf{'}s drop all this syntax what is the type of X well
it\textsf{'}s first of all is parameterize by a semi group type s and so this
is a function that will work for any type s so let\textsf{'}s write it down
explicitly as a universal quantifier which will be read for all s
so for all s we have a function that takes the extract Z typeclass
evidence which is a function type z2 s it takes the semigroup class
Evelyn\textsf{'}s likewise evidence which is this method and it produces the
value of s so in other words it is this function which is parameterized
by type s so this should work for every s and we want to write those
explicitly using the universal quantifier now we can simplify this
type using an identity but the product of these two functions is equivalent
to a single function from disjunction of Z and s times s to s so now
this type which is equivalent to this type which is equivalent to
this this is the church encoding of the free semi group over Z or
free semi group generated by a type Z I call this charge encoding
for reasons that I will explain but look at this type signature this
type signature looks a little bit like a continuation monad continuation
monad would have this type now we have this and then these two nested
functions they are very similar to a continuation monad but it isn't
really that it\textsf{'}s it\textsf{'}s not really a continuation monad because of this
quantified type the continuation monad has a fixed result type R it
is not quantified over that type does not permit rised by that type
and does not have for all our in front of this but we do have for
all s because this is our type parameter in the function so each value
over the free Simon group is a function parameterize by an arbitrary
s s being a concrete non free semi group or github perhaps another
free semi group but eventually it must have must be a non free Simon
group in order to get any values onto this non free actual useful
values so that\textsf{'}s the type now there is a theorem in type theory which
is that this type expression is equivalent to just a type a I will
present a derivation of this somewhat informally but this is the basic
fact that is at the basis of the entire idea of the church encoding
what I call the church encoding of a type a is this type expression
so whatever type a is you can just say I have a type a or you can
say I have this function parameterize by an arbitrary X with this
type signature that\textsf{'}s equivalent to having a type a so I call this
the church encoding of other type a and so unlike the continuation
monad the the presence of the universal quantifier makes this function
fully generic in X and it becomes like a natural transformation between
this factor and this identity factor so this is a reader factor with
type a being read X is the parameter and this is the identity function
with X as the parameter so this is resembling a natural transformation
between these two functions and we know that if this is a a function
with fully parametric code in other words code that does not use any
type information about X then this will actually be a natural transformation
so there is however a bit of difficulty in understanding how to work
with Church encoded types there are complicated there is this function
whose argument is again a function and it\textsf{'}s parameterize over arbitrary
X it\textsf{'}s actually not easy to reason about such types so in order to
develop intuition let us consider a simpler example where we take
a disjunction type just an ordinary disjunction type not a functor
nothing like that just an ordinary disjunction of types P and Q and
let\textsf{'}s work with its Church in Korea so by definition the church encoding
of this type is this type expression now we can simplify this because
it\textsf{'}s a disjunction in a function argument and this is equivalent to
a product of two functions from P to X and from Q to X and this is
equivalent to a curried function with this type so so far I have done
nothing but I have equivalently transformed this type into this for
convenience now in Scala in order to implement such things I have
to hide this type parameter somehow I cannot have a type so I need
to have a type that has inside it inside of it a def with type parameter
so in order to hide it I have this Scala code which is the usual pattern
for putting a universally quantified value into star distance car
does not have the universally quantified values it must be a def as
I did before now this is not very convenient you want to have a Val
with the universal quantification inside so in order to do that you
define a trait let\textsf{'}s call this trait disjunction it\textsf{'}s going to be
church and call it disjunction P and Q are going to be just parameters
for it I didn't necessarily have to do it this way importantly the
street has a method inside that is parameterize by X and this X is
not one of these type parameters so this type parameter X is hidden
inside the trait in this way as a method of the trait and when that
happens when a type parameterize function is a method of the trait
it means that you can call this method with any type parameter X so
in this way it implements the universally quantified type X and it\textsf{'}s
very easy to just write down this function signature like this so
how can we define values of this type so for example we define left
given some ULP we want to define a left part of the disjunction we
need to create a value of type disjunction so in Scala this would
be creating a new anonymous class by extending this trait and implementing
the method round so this is Scala\textsf{'}s boilerplate for hiding the universal
quantifier but then we just need to implement this function which
is easy we need to return X we have the two functions P to X and Q
to X and we have a P so how do we return X we'll just call this function
on the P so in this way we implement the left we'll also implement
the right in this way we can create values of this type now quite
easily so suppose that this this this G is a value of this type how
can we implement a case expression well we can just call the run method
on two functions like this and that would be actually the case expression
so the result would be of type X because that\textsf{'}s the result of the
run so in this way we program with disjunctions in the church including
and note that this would work in any programming language that has
nameless functions it does not the programming language does not need
to have disjunction types built-in so all we need to do is we need
to create this construction which does not have any disjunctions inside
now this does so this I would not be able to implement in a programming
language that doesn't have disjunctions but this I can implement in
such a programming language and so actually I have heard that people
have used this trick the church encoding for implementing disjunctions
in JavaScript the GU language also comes to mind as a very primitive
type system and I'm not sure how but with generics it would certainly
be able to implement disjunction types Java could do this too so general
recipe for church encoding is that you need to hide your universal
quantifier so you create a trait with method which adi will always
call run in this tutorial this method has an argument which is this
continuation like function or this function maybe several of them
may be a product of them if you have a disjunction and then you can
also think about making it more convenient so if you have a lot of
things here and not just one function but a lot of parts of the disjunction
this could be cumbersome so you could split it into products product
of functions and then you could say this is a value of some type so
you could even do a trait or a case class and parameterize by X containing
just this argument of the Church encoding just for convenience and
this is specifically very convenient with disjunctions because you
could just define like this instead of defining a run method with
this type signature define it as a function from exo X to X and so
this is actually much easier to use with languages such as Java or
JavaScript where you have objects with methods but you do not have
disjunctions now notice that case expression which replaces pattern
matching for these junctions is actually consisting of running this
function so the church encoding of the type is a function and calling
that function means running so just like in the free type constructions
when you interpret the free value or DSL or your interpreter runs
you get some final value that\textsf{'}s in the Church in Korean means you
call this function and get your final result the church encoding in
some sense encodes your DSL or your operations or your program your
declarative program encodes in terms of pieces of the interpreter
that are necessary to run it and so pattern matching is impossible
on functions you cannot determine whether this function uses its argument
or not for example by any kind of pattern matching on this function
value you cannot do that the only thing you can do is to run this
function so one deficiency of church encoded types is that they have
to be run in order to a pattern match they cannot pattern match say
on disjunction without actually having some kind of result type some
kind of target X and putting that X in there putting the extractors
in there and running this function now certainly you could be clever
and your ex could be another Church encoded something else so you
or non Church encoded something else you could very easily convert
this back into the ordinary disjunction type and then you could pattern
match on that but in order to convert this to anything you have to
run it so Church encoding has certain advantages it is easier to work
with if you have many targets and we will see other advantages of
the church and queen it does have also disadvantages and I will talk
about them but one disadvantage we see right away is that pattern
matching is impossible until you run or unless you actually run your
church encoded value so let us see how the church encoding works so
why is this type equivalent to the type a so let\textsf{'}s just consider this
very simple church encoding of a fixed type a which will be implemented
like this so in order to show equivalence between the church encoding
and the type a we need to present isomorphism between the types which
is a pair of functions from a to the church encoding and from the
church encoding vector a and we need to show that these functions
are inverses of each other so that a composition of these two functions
in every order is identity so if we have a value of a how do we get
the value of church encoded a well if we have a value of a then in
order to produce this we take this argument which is a function of
a and apply that function to the value of a that we have the result
will be a value of x which we return so that is the code I just applied
this given function or the continuation argument if you will to the
given value of a so that is in one direction in the other direction
in order to extract a out of this we can call well the only thing
we can do obviously on this value which is a function is the call
that function on which argument and with which type X that is our
choice so we call this function by calling run with type X equal to
a so like this and in an argument which is identity function from
a to a the result would be some a so that\textsf{'}s our second converter c28
Church included two direct type so it remains to show that these functions
are inverses of each other so how do we do that let\textsf{'}s think about
how could it be that we have a value of this type for any X given
this function we're able to produce a value of x now if I I'm able
to produce a value of an arbitrary type X and I don't know anything
about that type the only way I can do that is by using this function
somehow and this function needs to be called to produce an X on some
value of a so unless I have a value away I can't possibly have this
so this is the intuition that explains why this type is equivalent
to a the only way of having a value of this type is to have some value
of a now this value what if I have two different values of a well
the problem is I could only use one of these two values because I'm
supposed to produce this which has a universally quantified X and
I'm not supposed to look at X so this is supposed to be generic in
X so I could not for example check whether X is integer then I use
one value of a if X is not integer then I use another value of a that
is not allowed by by this type this type is fully generic in X and
so I am not allowed to use any specific information about what the
type X might be I could not write code like that I mean I can write
code like this and scholar of course but that is not what this type
is this type says this is a fully generic function which is a natural
transformation from this function to this functor I'm not allowed
to look at the code of X at the type of X I'm not allowed to use reflection
for instance or any other information about the type of X or or the
value of anything I'm supposed to be completely generic so if I had
many values of a at my disposal I be forced to choose one of them
for all X and use that one value of a in order to create this thing
for all X in other words the only way to have a value of this type
is to have a fixed value of a and then this is how I'm forced to implement
a value of this type so that\textsf{'}s intuition now I would like to be more
formal and show that for any Church encoded value CH if I first convert
it to a using this converter and then I converted back to CH then
I have the same stage as I started with but what does it mean I have
the same CHCH as a function so this function must be equal to that
function now equality of functions means if I substitute some argument
into that function I get the same value as a result by applying this
side and this side so let\textsf{'}s apply both sides to some function f of
type a 2x and then we can simplify this so what does the CH run of
f is on the left hand side and the right right hand side is this run
of F now this run of F we can see what that is it is a continuation
of a which means it is an F because the continuation is going to be
F the argument of run is going to be f so f of this now substitute
the definition of c2a it is this and F of that so see H dot run of
a to a now we cannot really simplify this anymore because we don't
know what CH that run does it is a arbitrary given value of this type
so we don't really know what it does when we call it on a to it but
if we look carefully at this equation so we are now required to prove
that this is equal to that looking careful in this equation we find
that this is the condition of naturality of the function G H run as
a transformation between the reader factor and the identity function
applied at type X this is a natural allottee condition here is how
I can illustrate this using a type diagram naturality condition means
that if we do an F map so we have one factor on the left another factor
on the right we have a natural transformation between them if we now
F map with some function on the left and we F map with the same function
on the right diagram should commute and this is precisely the equation
what we have written here run over F is precisely that F sorry I'm
I'm confused it is this direction first the left hand side is this
direction this is the run of F and the right hand side is this direction
first Iran of identity and then you apply F to that so the left-hand
side corresponds to this direction on the diagram and the right-hand
side responds to that direction on the diagram commutativity of the
diagram is therefore exactly the same as this equation so in other
words we have shown that this function will be equal to that function
as long as we demand that this is a natural transformation so this
code must be fully generic should not use any type information about
X and the counter example would be looking at the type of X and using
different values of a to create a to see here so calling this on different
values of a depending on what X is so that we could write this code
in Scala but this would not be an actual transformation the other
direction is easy very fun if we just substitute the code C to a of
a to C of a C to a is this and then a to C away as a run function
and then you have identity applied to a and that\textsf{'}s 8 so in this way
we can show more formally that this type the church encoded a is actually
equivalent to the type a as long as we understand that this must be
fully generic code and in other words a natural transformation that
means these two parameters and another property of the church encoding
is that since it is built up from parts of the run method of some
typeclass usually it will automatically satisfy the laws of that typeclass
now this example as well as this example were not examples of typeclasses
that are and what I was going to church in code this example was a
typeclass the tie church encoded the free semi group and the property
of the church encoding is that it will automatically satisfy the laws
and the reason is we know that laws will be satisfied after you run
the typeclass a free time class instance this we already saw and therefore
since our church encoding is basically functions that run and the
only way you can use them is to call these functions then Church including
will satisfy laws automatically in the same ways this function is
equal to that function which we verify by calling these functions
applying them to specific arbitrary argument a war for a typeclass
means that you need to run the church encoding and then compare the
results so since we know that the run method for free typeclasses
satisfies the laws it follows that the church enrolling over free
typeclass will automatically satisfy the laws of the type course so
this is a very nice property of the church encoding let us look at
the code of church encoding a free similar here we define the extraction
as a typeclass and then in order to define the free semi group we
have very low to work left to do unlike previous implementations where
we have to first define case classes and so on we don't define any
case classes here the Reb constructor is like this and then we are
ready after this we have defined the extraction typeclass we are ready
to start working with seven group values don't need any anymore preparation
so here is x and y these are already values of the three seven groups
so here\textsf{'}s a computation we wrap one wrap to wrap three add them that\textsf{'}s
it we have now defined x and y these are three similar values no more
ceremony so this is another good thing about the church including
perhaps in order to interpret we don't need to define interpreters
these are already interpreters let\textsf{'}s see how that works we will interpret
this threesome in two string which would have a standard seven group
instance so let\textsf{'}s define are those standard semi group instance for
string now in order to extract into string we need to have an extractor
so let\textsf{'}s make it available so now string has a typeclass instance
of the extractor typeclass and we can run that\textsf{'}s it that\textsf{'}s how we
run we do need to specify the type parameter but that\textsf{'}s it so we don't
say or run this with that extractor all of that is in place and so
we have a lot of computations in our DSL those computations are going
to be more concise here is the code for implementing disjunction with
some testing here is for example how we have a case expression so
X is a disjunction which is left of ABC and here we want to match
on X and we have the two possible cases and that\textsf{'}s how it works so
now we have seen the encoding coatings of three typeclasses let\textsf{'}s
now look at examples and have a more have more intuition about how
these including actually work and what are the trade-offs in each
of these in committees the simplest typeclass that has type constructors
is factor now until now we are looking at the semi group or Minh mono
it now these are typeclasses for types functor is a typeclass for
type constructors and for type constructors things are a little more
difficult and there will be more syntax and more type notations however
they are quite similar to non-constructive typeclasses in very important
ways they are very similar so keep in mind that free semigroup Freeman
or your free filter are basically applications of the same construction
to different typeclasses in order to construct a free functor the
first question we need to ask is what methods is typeclass requires
so there is one method let\textsf{'}s look at this method so the tree encoding
of a free factor would have directly encoded this as a case class
let\textsf{'}s call it f map and it will also have a case class for wrapping
a type constructor that we base that we generate from so to remind
the free functor typeclass needs to be generated by a type constructor
so we don't just have a free semi group we have a seat free semi group
generated by type Z so we don't have a free functor we have a free
functor generated by a type constructor if we need to start with some
type constructor which doesn't have to be a functor it can be but
it doesn't have to be so we I call this free funder over F now I introduce
this notation this bullet in order to emphasize that F is a type constructor
it has a type argument here which I'm not writing I could write F
a but a is not known is an argument so it\textsf{'}s a type function really
so I'm trying to find notation for type function so in Scala it'll
be like this and I don't like this notation so much but it\textsf{'}s okay
but in my short notation I right now found this to be a little better
more visual so this is a type constructor F that waits to be given
a type arguments so just like this in Scala and so this is the tree
encoding of a free factor so I call this a tree encoding because this
encodes an expression tree unevaluated expression tree for a functor
valley now what did I do in order to write this code are basically
the wrapping now a trait must be there this is a scala syntax for
disjunction so I need a disjunction so one case class raps a value
of F of N and the other case class raps this so it\textsf{'}s it\textsf{'}s going to
denote the result of applying F map to a free constructor note that
this F map has an extra type parameter because we extend the F of
a so I have chosen the name Z here the result of F map is F of a so
we extend f of a but the arguments of this case class or the parts
of the case class have a parameter Z it\textsf{'}s an arbitrary FZ which is
going to be this free functor of Z which we map with a function of
Z to it and the result is a free factor of a so this type parameter
Z is hidden inside the type constructor we extend F of F F F F a naught
of Z so the Z is not visible outside so we outside we will think this
is a value of this type but actually inside it has a Z now this is
a very interesting situation that we have a case class permit rise
by an extra type parameter which is hidden from the outside type and
let us look a little in more detail about what it means to have such
a type parameter let\textsf{'}s consider a simpler example simpler than all
this and write this code so I declare a sealed trait with a type parameter
a and inside it I have a case class that is permit rised by another
type parameter Z and it has values but depend on Z but it extends
key of it so it hides the Z from the outside type let\textsf{'}s look at how
this works what if I wanted to construct a value of type K of F how
would I do that well here\textsf{'}s here\textsf{'}s how there\textsf{'}s only this case class
so I have to use this case class and I have to specify some other
type for Z let\textsf{'}s specify string and then I would have a value which
has visible type Q int but actually inside it\textsf{'}s hiding a string type
and it knows that it\textsf{'}s hiding experiment right so it could have been
another type so when we have a value Q of this type we know that it
is integer in this parameter but we don't know what is this other
parameter Z we know that it exists inside Q hidden inside Q so it
is called the existential quantified type so this is a tie Plantation
that I would use to denote this this definition this definition is
a type constructor with parameter a witch inside hides a type Z which
must exists also to build a value this type we need to find some type
or select some type Z put it in put a value of this type in there
but we hide it so other outside we don't see that Z it\textsf{'}s exists inside
so this is a notation and this is called the existential quantifier
so this existential quantifier basically says that this I constructor
it has this type so the function qz construct s-{}- a value of key
of a so it hides zi some very interesting thing so the syntax says
that qz is parameterize by both NZ but rho is very different for a
and precede the role for Z is existentially quantified because it\textsf{'}s
hidden from the outside role of a is a type parameter visible from
outside and the functor ends in a so this is always a factor in a
it is not universally quantified so even though it\textsf{'}s a type parameter
here it\textsf{'}s not universally quantified with respect to Z and this is
so because when you build up a value of this type you must use a specific
Z it will not work for another see later it would have that specific
Z baked in the value Q once you construct it so that\textsf{'}s why it is an
existential quantifier and not a universal quantifier but the code
does not show this explicitly the code is a bit confusing we have
just seen a universal quantifier in the code here and here and the
way to implement this universal quantifier was to have a method insider
threat a trait and the method was paralyzed by this X the way to have
an existential quantifier is to have a case class inside the trait
and the case class experiment rised by the Z the method inside the
trade hides the X because the X is not a parameter here the case class
inside the trade hides the Z because the Z is not a parameter here
so until now it\textsf{'}s very similar but case class is not method quite
rate so this is the crucial difference so if it\textsf{'}s a method of a trait
then this would be a universal quantifier a method that has an extra
parameter hidden from the outside a case class with an extra parameter
hidden from the outside that represents the existence of quantifier
so we have to keep track of this ourselves the syntax of Scala does
not help so much to keep track of this but this is a very significant
difference between the types 
\end{comment}

\begin{comment}
so we need to keep in mind that the encoding of the three-factor uses
here Z as the existence of quantified title to get a little more intuition
about how the existence of modified type works let us consider a simple
example similar to this one where you have existential quantified
type Z with a function mapping it to something and another piece of
data containing that type so consider this type expression just temporarily
I denoted this by P a and we will now show that P a is actually equivalent
to the type a a scholar implementation of PA would look like this
we would have a sealed trait and a single case class that hides the
type Z now imagine we would like to construct a value of P a where
a would be some fixed type say integer in order to construct it it
would have to use the case class easy and we would have to give some
value of the type Z and the function from Z to a so imagine that we
have the type Z equal to a well it\textsf{'}s our choice we can choose that
we give a value of type a and here instead of this function will give
identity we can always do that for any type a and so this means we
can always build a value of this type if we have a value of type a
so that gives us a function that converts from a value of type a to
a value of this type just inserting identity function here and inserting
the value a here and setting Z equal to a so we are free to choose
what Z is when we construct the value of this type so this gives us
the equivalence function in one direction from a to P now how about
extracting a from P if we have a value of this type we actually cannot
extract Z out of it so a value of this type contains Z as part of
it but we don't know what the type of Z is and because the type Z
is hidden we cannot extract it out of the function PA we cannot have
a function whose type is unknown whose whose type signature contains
an unknown type however what we can do is we can extract a out of
PA in order to do that we need to apply this function to this value
this is the code and we don't need to know what the type disease this
would be some unknown type the function f has the right type signature
so that we can apply it to that Valley so this is a well defined value
and so in this way we can extract a value V out of PA no actually
we cannot transform PA into anything else other than into a value
of type a because this data only allows us to get a Z or to get this
function or to apply this function to this now we cannot get a Z out
because we don't know the type of Z we cannot get this out because
we don't know the type of this function so we cannot write code that
says take a pee and output some unknown type that doesn't work in
Scala the result type of the function must be given must be fixed
before you can write the code of the function so this means this value
is observable only via this function so the only way of doing any
computations with this PA is to apply is extracting function and to
get an A out of it and so if you wanted for example to compare two
different values of type PA then you cannot directly do that because
you don't know what what Z is you cannot look into it it\textsf{'}s hidden
so the only thing you have to you are then forced to do is to extract
an a out of this and compare the resulting values of type a and so
for this reason the functions a to P and P to a are inverse to each
other when we use P to a in order to come compare any values of type
P a so this can be shown relatively easily thank you for example take
a composition of a to P and P to a in one or another direction and
you can substitute the code in one direction this will be identity
of Z so that\textsf{'}s clearly going to be identity in the other direction
you have this and this should be equal to the results on so this a
equal to a PTA of sum P and that should be equal to that P in order
to show the isomorphism in the opposite direction so that requires
us to compare two different values of type P of a and we have to do
that by applying P to a to both sides and that\textsf{'}s what will again give
us identity so I skipped this calculation but this is very similar
to what we did for proving the identity of types and a couple of slides
before when we used the universally quantified type so this proves
this equivalence and actually there is a stronger version of the equivalence
which is this if you have a functor if so this is not a free construction
this is just a given factor then this is equivalent to that factor
and this is proved in a very similar way the only way to observe a
value of QA in other words to compute anything out of it is to extract
an FA out of it you can extract an FA by taking FZ and doing F map
with this function and you can't extract anything else cannot extract
an FZ out of it because you don't know Z you can't extract this function
out of it you don't know what this function is what would the type
of this function is so the only thing you can get out of this QA is
some value of type of a so that\textsf{'}s the transform transformed q2 f f2
q is similar to this one you take an affiliate take identity function
and very similarly we can show that these two are observational inverse
is now they are not directly inverses in a sense because you cannot
directly compare values of this type because that contains some unknown
type Z inside and what if this type is different however this type
Z is not observable even if it\textsf{'}s different so you have some value
of Q with one Z and another value of Q with another Z you cannot see
that this type z are different in these two values you have to first
extract the observable value out of this which is a value of this
type once you have extracted it you compare those so this is what
I mean by a traditional equality and so you can show that these two
functions are inverse of each other when the Equality is understood
as observational equality so whenever you compare values of this type
instead you extract FA by using this function and compare the resulting
values so this is how the existing shop type works now in the free
functor construction we use the extensional type and we can rewrite
a construction using this type expression so I just taken this code
I have rewritten it using the type notation that I'm using so that
is the definition of the free function so this is a recursive definition
because we're reusing the type FF itself as we are doing here so if
F is reused as part of one of the case classes so this is the tree
encoding in other words this encodes the young unevaluated expression
tree of an expression obtained from with a free factor of values and
operations so there are operations which are insert an FA into the
free function and apply and a map to free frontier so using these
operations in any order we gets arbitrary values of the free functor
so let us derive reduced in cooking to derive the reduced important
we start from the tree including we try to see how it could simplify
values of the tree encoded type using the laws of the typeclass so
the furniture typeclass has two laws there are the identity and the
composition law and composition is also associative so that\textsf{'}s another
property so we need to see if we can simplify values of this type
so let\textsf{'}s consider values of this type any value of this type you must
be by construction either wrapping of this or it will be a previously
constructed value of this type multiplied by a function like this
so essentially we have to start with some wrapping and then we multiply
a few times every time we multiply so we use a map function every
time we do that we add another existing type parameter so then we
have all these existential type parameters then we have the first
wrapped value of this type constructor F which is not necessarily
itself a functor and we have a bunch of functions of different types
now all these functions must be composed associatively in other words
the law of composition is that the result of mapping with this function
and then later mapping with another function and then later again
mapping with another function must be the same as a result of a single
mapping with the composition of all these functions and the composition
is associative so in other words we should be able to simplify this
value into a product of this and the single function here which is
composed out of all of these and the result of the composition doesn't
depend on on the order in which we evaluate the composition because
of associativity so therefore by using these laws we can simplify
this expression into this expression where there is only one quantified
type all of these other types are not visible anymore because we're
not using them and there\textsf{'}s only one function here or also there is
a possibility that we just have this no functions and this possibility
should be equivalent to having this value mapped with an identity
function because of the identity law for function so therefore we
can say let\textsf{'}s just always have this type and if necessary put identity
function here and we always have some value of type constructive F
and we always have a single function but we no longer have disjunction
because we can represent this case by putting an identity function
in here due to the identity law of the factor so this concludes the
derivation of the reduced encoding so the result is this formula which
means we have successfully simplified this expression we got rid of
a disjunction and we got rid of the recursion this is non recursive
we don't use the recursive instance anymore we just found that it\textsf{'}s
equivalent to have just type constructor F and that\textsf{'}s the reduced
encoding of the free factor so the only important remark here is that
it requires a proof that actually this is a reduced in Korean so by
definition a reduced encoding is such that it respects the laws so
if you apply for example map to a value of this type with identity
the result must be equal to this it must so it is not true for the
tree encoding the result will have an extra function here an extra
existential type and so on and that\textsf{'}s kind of not good enough for
reduced encoding it must not be there so the reduced income must satisfy
the laws and that\textsf{'}s what it does so it requires nevertheless some
ingenuity we have to derive it doesn't follow automatically what it
should be and the proof that it is equivalent to the tree in holding
and satisfies a lot another good result from reduced encoding is that
we can see what happens when the type constructor from which we generate
the free function is already itself a functor so if we are taking
a free factor of a tied over a type constructor that is already a
factor then as I already already said this expression which is that
is equivalent to the factor itself so so this type is going to be
equivalent which means that while there\textsf{'}s no harm done including this
free function except maybe performance will suffer you have some extra
stuff will have some identity function here\textsf{'}s a mother function basically
you are just postponing the map there might be some advantages in
doing that which I will show because you can make this stack safe
but this will certainly be a performance hit so don't do this if the
type constructor is already a function if you can avoid it but there\textsf{'}s
no harm done it\textsf{'}s the same equivalent type so you won't have more
information so proud so this is an interesting property because usually
what happens with free constructions is that they wrap you're generating
type in some stuff and so they add information to it so the resulting
type is usually not equivalent to the type constructor that your racket
but in this case a physical one so for functors functors are special
and free functor all over a functor is equivalent to that function
so that\textsf{'}s a special property finally let\textsf{'}s look at the church and
according now this is a more challenging task because we are dealing
with the type constructor so we let\textsf{'}s start with this and what\textsf{'}s children
coldness now church encoding means we need to add a universal quantifier
but since our result is a type constructor the universal quantifier
must be for a type constructor so the church encoding that I have
shown before was for a ordinary type or a free semigroup or something
like that for a free semi group the type is not a type constructor
for a free factor it is a type constructor so therefore the church
encoding must have a universally quantified type constructor in it
and things are just going to be more complicated because here\textsf{'}s the
structure of the type expression in the church encoding of a type
constructor we have a universally quantified different type constructor
then you have a function from your type to that type constructor and
again from that to your type constructor so that\textsf{'}s the general structure
of the church encoding but because the insides are type constructors
then this must be natural transformation so this must have another
Universal quantified type inside so I use this squiggly arrow to indicate
universally quantified functions such as natural transformations just
it\textsf{'}s the same error it\textsf{'}s just suggestive so that like keep track of
where I have universally quantified where I don't so if you follow
this structure then this is going to be the entire expression for
the church encoded free function now this is starting from the tree
encoding we have a choice what do we charge encode which version code
this or do we charge encode this and they're going to be two different
Church encoders so starting from the tree encoding that\textsf{'}s what we
need to do now there\textsf{'}s one interesting side of the church including
that I'm going to explore in more detail now which is that the recursive
use of the type is not seen in the church including in instead he
replaced that with this universally quantified type constructor that
is present so to speak so instead of the recursive use of free factor
here I have this universally quantified P that is a very important
part how church including works with recursive types and just before
we go through that I want to remark that in this expression the quantifiers
cannot be moved and you cannot move this quantifier to the outside
it is really inside these parentheses that the C is dis quantified
so to all see this function is given so for all C this function is
given and that function is the argument of the outside function so
this is important for the church included so you have several layers
where types are 25 this type quantifier is specific to the free functor
this wouldn't appear if we had no type quantifier here so this is
specific but this would be always true for any Church including of
a type constructor have a type quantifier inside that cannot be moved
to the outside and of course also you cannot move this existential
quantifier to the outside so for this reason when we write code we
have to take care to hide these quantified types at the right place
inside the data structure so let\textsf{'}s look at in more detail on at how
church encoding deals with recursive types and with type constructors
so let\textsf{'}s consider an example here\textsf{'}s a recursive type not a type constructor
it\textsf{'}s just a type with fixed type Z it\textsf{'}s a tree with leaves carrying
values of type C the church encoding of this type looks like this
so I'm looking directly at our encoding of the fee-free mono which
was very similar and this was the encoding the church including of
the free memory so we know this is correct now let\textsf{'}s look at how it
works we take this expression which uses the type P recursively twice
and we write it here but instead of the recursive type P we replace
that with X where X is the universal quantified type given outside
so in this way the church encoding replaces the inductive use or recursive
use of P by using this parameter X so the result is a non recursive
type expression or at least it doesn't look recursive it does not
use itself somehow to define it it\textsf{'}s type but it is equivalent to
this recursive type so it\textsf{'}s very interesting that just by using a
type quantifier you can remove type recursion well at least on the
surface you don't of course actually remove it because the type is
equivalent it\textsf{'}s still encodes a tree with devalued leaves so it\textsf{'}s
still a recursive type or your recursive data structure but it\textsf{'}s encoding
does not show recursion so that\textsf{'}s that\textsf{'}s interesting so how shall
we understand the way it works this is a run method of a declarative
way encoded DSL and so this method tells us that in order to extract
a value what you need is to be able to extract value from this and
here you have again these values how would you ever get the value
X in practice in practice it would have to call this function on a
Z several times to get some X\textsf{'}s and then you would put these X\textsf{'}s in
here call this function again to get some more X\textsf{'}s and so on so in
practice it is a recursive process it can encode recursion but all
of this is already encoded in this function so the type does not show
recursion so in other words how can we produce a value of type X and
we don't know what that type is it\textsf{'}s we're required to write code
that produces an value of type X whatever the type might be well the
only ways to use this function somehow and this function requires
us to give this as an argument so how can we give this as an argument
either we give a value of Z as an argument and then we have our X
we can return it or we give two excess as arguments and then we have
a new value of x and we can return it but how where do we get the
two x\textsf{'}s well we still have this function so again either we give some
Z to this function or we give a two-axis to this function so this
is where the recursion comes in in order to produce a value of this
type we need to have a tree with Z valued leaves and once we have
that tree we can write this function so these functions are equivalent
to trees with Z value leaves in this way now we can generalize this
construction to a recursive type defined arbitrarily like this now
here s is an arbitrary function that is fixed and this functor determines
the structure of the step of recursion so for example here this functor
would be as P equals Z plus P times P so this factor I call this induction
factor because it describes one step of the induction when we derived
values of the type so what we have seen right now suggests that the
church encoding of this recursive type looks like this so this is
a general way of encoding recursive types by church encoding and it\textsf{'}s
not recursive at least on the surface and I will show an example of
church encoding of lists of integers {[}Music{]} so here\textsf{'}s an example
of church encoding of list of integers first let\textsf{'}s do the recursive
encoding just for reference it would have a be a shield trait with
two case classes one representing the end of the list or an empty
list and another representing non empty link in other words a value
of integer type and and next value now Scala does not allow us to
do this this would be the short type notation but this cannot be done
in Scala because Scala does not allow you to do universal quantifiers
while is explicitly so instead we denote first this as some helper
case class CP just to make it easier for us and well we could actually
probably define this as a type rather than as a case class but let\textsf{'}s
keep it like this for clarity so the CP of a is just a helper case
class that represents this type it represents a product because we
can simplify this function as a product of two functions one to a
and this to and then we can simplify further 1/2 is just a and this
is like that just for convenience later now it\textsf{'}s very easy to encode
the church encoding of the rest of it just a function from C POA to
it now if you look at what that type signature is it\textsf{'}s very similar
to the type signature we'll fold it\textsf{'}s a function from a and this function
which looks like an updated updated function for fall gives you an
A and a is arbitrary so here a needs to be hidden inside the trade
as a universal quantifier so as I said before that\textsf{'}s how we need to
keep in mind that this is universal quantifier and therefore we do
a method in a trade so we don't do a case class parameterize by a
hidden parameter that would be an existential qualifier we do a method
in a trade because it means that this method can be called with any
parameter unknown at the time of defining this method and that\textsf{'}s what
the universal quantifier does so ok we are done we define this type
let\textsf{'}s define values of this type so to create an empty list we need
to write this boilerplate now how do we implement an empty list well
we need to implement a function that takes this and returns that you
know this has two functions inside we need to think about what these
functions mean in order to be able to implement anything here so what
do they mean all these functions mean what to do we'll look at this
for example what to do when the list is empty how to run the list
how to fold the list what is the result value when the list is empty
and this function tells you what is the result value when the list
is not empty it has a head value act some type integer and it has
some additional arrest values which have been already evaluated or
folded that your value a is given so what do you do then how do you
update your fold down now an empty list would never get into this
case it will always just give you the eighth so therefore this function
for an empty list ignores the link function and just returns in whatever
the end is the empty list when folded always gives you gives you that
justice this value which is denoted by hint in the fold signature
this is denoted by in it but we're trying to imagine the list being
created so there\textsf{'}s a empty list or end of the list that\textsf{'}s just the
name of the variable so for this reason that\textsf{'}s the implementation
of an empty list in the church encoding now let\textsf{'}s do them list with
one element how do we do that so we need to fold and in order to fold
with a list with one element we need to use this updater function
on this element and on the rest of the list which is empty which is
going to always evaluate to the end value so that\textsf{'}s there for the
implementation of a one element list so this is now more suggestive
but we are having an and one element list linking X to end so this
could be this class like this in your case class in coin but you know
this word we don't have any case classes here representing lists all
the lists are our functions the CP is just a convenience type where
pattern matching here just for convenience we don't need to do that
if we for example encoded this just as a tuple of two values then
we would not need the case expression we would just take this under
square one on your store two and so on but it will be less readable
so that\textsf{'}s why I I write it like this let\textsf{'}s implement appending so
we have an element X and a previous list we want to add this to the
list what we need to implement is how to run the resulting list or
running a list means folding it we are given the initial value of
length function now we need to use the link function on the X and
on the rest of the list but the rest of the list is this so we need
to run this using the same fold information so the CP case class encapsulate
all the fold information we need in order to run the list so when
you run it with the same old information we get a value and then we
update with the X so this is how we append now folding is just the
same as run so we can implement the fold function with this type signature
and it\textsf{'}s just calling run with these arguments notice that fold is
non recursive the fold function is non recursive actually none of
these functions are recursive we can implement convert into ordinary
lists just as a fold with a list constructor we can implement math
again this is going to be non recursive because we're just going to
pass some modified food so the lists are how do we run how do we fold
a list after mapping we just fold it with modified function so instead
of X\textsf{'}s we substitute f of X that\textsf{'}s all this is not recursive so the
map is non recursive on these lists the fold is non recursive in these
lists all the recursion has hidden inside these functions that they
run functions those run functions will call other run functions and
critically what we don't see that our our code here is not you closer
here\textsf{'}s how do we-{}-how implement has option we run it on an especially
crafted folder folding information and {[}Music{]} that actually is
an interesting observation that I would like to make is that pattern
matching such as head and tail is not directly available on this data
structure this is a function now it is not a bunch of case classes
we cannot directly pattern mention it and determine if it\textsf{'}s empty
or not for example we have to run this function on some arguments
and this run could take a while so for example tail cannot be implemented
efficiently as a Big O of one operation it has to run lowest to the
end and build the tails as a second list so that\textsf{'}s a deficiency of
the church encoding but if you need pattern matching operations you
need to run the structure or the entire function which might take
a while so let\textsf{'}s run some tests here implement just some function
so here\textsf{'}s how we create it\textsf{'}s just a folding with the Sun here\textsf{'}s how
we create some lists in the church encoding so it\textsf{'}s a pure of ten
which is one element list we append five to this we get a two element
lists with five and ten in it so then we check the sum of these elements
is correct and converting it to list gives you what you expect in
the map gives you what you expect now the map operation here is perhaps
stock safe we can check that some is stack safe so our fold implementation
is stack safe creating a list of any elements is stack safe when implemented
in this way so what we need to do is we need to compose many links
together but this needs to be done in a stack safe way which is why
we do it by hand here we don't just do link compose link composing
writing this would actually not be stack safe you'll see that later
in more detail also appending many elements is not stack safe we obtained
a large number of elements then trying to run that list do anything
with it would be a stack overflow so you see the sum the sum function
itself is a stack safe as long as you can run the list inside it but
it\textsf{'}s the list itself that needs to be stack safe now the function
that constructs the list is a function that builds other functions
and that function needs to be stack safe I should for example avoid
composing many functions but you can't avoid that if you do attending
one by one so you need some more clever implementation which is possible
but I will not discuss it right here let us see that the church encoding
of a type constructor so I'm using this notation with a bullet to
denote type functions so the scows index for that will be this so
the church encoding let\textsf{'}s begin with the church encoding of a type
function P just the type constructor P so what is the church encoding
of that this is the church encoding of the type constructor P you
have to have two quantified types and one of them is a new type construction
which is quantified in other words this is a function others parameterize
by an arbitrary type constructor F and it\textsf{'}s argument is a function
that\textsf{'}s parameterize by an arbitrary X which has this type signature
so this function world alternatively can denote it like that with
the squiggly arrow that I'm using just it\textsf{'}s the same I just want to
have a different notation for this it resembles a natural transformation
however these P and F are not necessarily filters so we don't necessarily
have a naturality law it\textsf{'}s just a generic function parameterize by
a very mature X with fully generic code but if these are not factors
then there are no materiality laws imposed on us so this is not a
natural transformation but a type signature is exactly the same so
this is somewhat complicated and for this reason I'm going to show
you an example of how to encode the option type yeah the option type
constructor in the church encoding so that you see how all this is
translated into code so the direct encoding option would be a polynomial
data type like this and with these classes that you could call like
this so we're going to implement this type expression now where this
is going to be the first type parameter and this is going to be a
second type parameter which is inside this argument so let\textsf{'}s first
encode this argument has a separate type for convenience so this argument
is a function that extracts a P from an option so it\textsf{'}s again looks
like a natural transformation from option to P except P is not necessarily
a function so let\textsf{'}s denote this X option which is this extractor from
option now this X option just for convenience we define this type
separately parameterize by P but it is not parameterize by X because
X is the universal quantifier type which needs to be hidden inside
this X option so therefore we have methods in the X option what are
characterized by X now I could have just had one method here such
as apply with parameter X and then I would have this function as the
type of that method but it is actually more convenient especially
in Scala to have separated methods so if this is a typical pattern
of a function from a disjunction to something is equivalent to product
of functions from each part of the disjunction to that something so
then the equivalent type is less and we can just denote each of these
as a separate method in the trait so that\textsf{'}s just convenience we haven't
done anything really we just equivalently transformed this type for
convenience it\textsf{'}s a little easier to read and we can give these trade
methods suggestive names so having defined this type it\textsf{'}s now easy
to define the church encoding on the option which is parameterize
by a notice listen thing has only one type parameter which is a that
is visible outside the type parameters P and X should not be visible
outside they're hidden inside his type expression so therefore we
put the type parameter a outside and the run method I just call this
run for convenience to suggest what the church encoding does is that
if it runs a DSL program with an interpreter so this is an interpreter
for the the operations of the DSL and this entire thing is the runner
of the DSL program into an arbitrary target type so that\textsf{'}s why I always
call these methods run but this is just been named doesn't do anything
by itself it is the type that do all the word no types so the run
method needs to be permit rised by the parameter P which is itself
a type constructor and this is the type of the run so that\textsf{'}s it we
have finished implementing the church encoding as a type now we need
some helper functions so that we can easily create values of the Church
encoded option so how do we implement for instance constructors the
Sun and none now these are not these trade methods these are our so
these are our methods mean we could you could make these methods private
if we wanted to this entire type could be made private the users should
not have access to it so to define some we need to put an X of type
a into the auction so how do we do that we define this church and
call it option with around method and we need to implement this so
how do we run a non empty option well clearly we use the sum method
on the x value that we have to get a P of X so that\textsf{'}s what we do here
how do we run an empty option we use the non method which has no arguments
and gives us a P of X that\textsf{'}s it a lot of boilerplate as you can notice
all of this is boilerplate all this is boilerplate this is the actual
code implementing the Constructors for direction there are some libraries
that make it easier to use but it doesn't matter enough so we can
also show that option the ordinary option is equivalent to the church
encoded option to do that we do a wrap and unwrap methods let\textsf{'}s say
so first we take an ordinary option and we implement the church encoded
option that\textsf{'}s very easy we just do one of these two constructors unwrapping
from a see option into an ordinary option requires running let\textsf{'}s see
option again we have the same {[}Music{]} same phenomenon but if you
want to pattern match for instance you want to detect whether this
option is empty or not you have to run it there is nothing we can
pattern match directly on this value because this values a function
you cannot pattern match on code of functions so how do we do that
so we need to run it but to run it on what we need to provide an interpreter
so the interpreter will take our church encoded option and produce
an ordinary option so that\textsf{'}s what we need to prove produce and these
are just the standard methods of the standard Scala option and that\textsf{'}s
that\textsf{'}s it so here\textsf{'}s how we can use it so we can create some values
of option type now pattern matching does not work cannot directly
implement that imagine so which if we try it there\textsf{'}s a type of problem
so for example we wanted to pattern match directly like this by running
the option on something but I can't really do that because we need
to provide an interpreter that interprets arbitrary type X under the
option but we only have a specific types here type a so if we write
this code which will be kind of what we want we gathered a pair so
the only way of doing a case expression would be first to run this
like this to unwrap it convert it to an actual option with case buttons
and then we can pattern match on those on the other hand natural transformations
work fine they don't require running on some first on on a real option
you can just run on a constructed interpreter and interpret this into
another function so that\textsf{'}s that works fine and here\textsf{'}s the test code
so now finally let\textsf{'}s look at how the church encoding works for a recursively
defined type constructor so this is very similar to how it works for
a recursive type in that all the recursive usages of the type constructor
are replaced by this type that is universally quantified and since
now we are dealing with a type constructor we need to adjust our notation
so that we define first of all the reclusive type constructor like
this where s is now a factor that describes at the induction principle
but it\textsf{'}s now paralyzed by this type constructor so this notation that
I'm using for a higher-order type function in other words it\textsf{'}s a it\textsf{'}s
a functional of types that are themselves function of types and Escalus
index won't be like this so an example of that would be a list cursor
we define like this and if we define s like this then you see the
P parameter P denotes the recursive use of the type constructor in
its recursive definition so this is how we could denote this construction
and then the church encoding of this looks like that so there is similar
to a church including for cursor types and it\textsf{'}s non-recursive it\textsf{'}s
a type expression that does not require recursion so let\textsf{'}s see how
the list constructor is defined in the church encoding this is the
type expression for the church encoding of the list because this is
the structure or induction factor for a list we just saw and I'm just
adding all the type quantifiers explicitly I have a B which works
inside these parentheses only it\textsf{'}s hidden I'm gonna have a tree which
gives me this is this it\textsf{'}s also hidden from the outside I can equivalently
transform this type signature into this where I {[}Music{]} again
replace a function from disjunction to B or B by a function from just
part of the disjunction to P of B and then from this part of the disjunction
to P of B which I simplify to just beyond P so that is how I would
seem to fight now I will deliberately write code similarly to an on
parameterize list that I did first so the end needs to be paralyzed
by this B now I used X instead of B here and now finally I search
encode the list of a just as a function from this to P of a very similar
code that I had before in order to define empty list a list of one
element and appending there was a difference that link and the run
I'm getting them as methods of a trait whereas before I was getting
them as parts of a case class I could have done a trait before as
well because really it\textsf{'}s just a convenience but here I could not do
case class because I need this type parameter oh I didn't know that
this be in English used for consistency this type parameter being
what I have here needs to be hidden inside the type CL which is this
type which is the argument of this function and case classes will
not do this frame so I need a little trade with methods in Scala other
than that the code is very similar fold is non recursive in order
to implement fold I need to have a bit of typecasting because the
only way to get anything out of the list is to run but I need to run
on an interpreter if I'm folding are not interpreting into another
type constructor I am interpreting into a single type so however need
to pretend I'm interpreting into a type constructor because that\textsf{'}s
the type signature of the church encoding it\textsf{'}s run into an arbitrary
type constructor so I can choose that type constructor to be the identity
factor and in this way I can get ordinary types out so I define or
constant factor be another possibility so I define a constant factor
and then I run into that so I give that C is a type parameter and
then I encode the running just like I did in the fold implementation
above in this hold implementation here except that now I need to specify
this as methods over trait rather than as parts of the case class
other than that it\textsf{'}s very similar and here I typecast X X has type
X but I know this will only be called on values of type a so I know
that even though I'm supposed to provide {[}Music{]} this CL of C
with arbitrary X actually this will only be called on values of type
a so I can cast this safe way to satisfy the type checker now this
is a little ugly but that\textsf{'}s what I found to be necessary with the
constant factor being used so using this I define a sum I define two
lists and I run exactly the same tests as I ran before network of
the lists so now I would like to generalize the constructions we have
seen two arbitrary typeclasses so this is something that the church
encoding makes it particularly easy to understand but it does not
have anything to do with the church encoding so let\textsf{'}s look at first
of the church encoding of a three-cylinder looked like this now here
X is an arbitrary type but this is a signature of the semi group method
which is combined so if X were constrained to the semi group typeclass
and this would be given already as an implicit argument let\textsf{'}s omit
that argument and I would denote it like this so here the typeclass
constraint is denoted here like this so now it is an arbitrary type
X which must be of this typeclass and then I have this so it\textsf{'}s a much
simpler presentation and notice that the induction factor for semi
group over Z is this so basically what happens is that a church encoding
of a cylinder which is this one which we can generally write down
as the semi group functor or induction functor as I call it before
but actually this is always going to be of the form Z plus something
when we do a free side quest inspection because we have to wrap the
Z pipe and then we have these methods this tells us how to generalize
the country construction to arbitrary typeclasses so first of all
we define a functor that describes the operations of the typeclass
so for example for the semi group we have a single operation and writing
it in this form means that CX is just a pair of XX but in more general
typeclasses ships would be more more general and so that would be
what I call the method factor this type is in a category theory called
an algebra or sea algebra so if C is a functor when this type is called
a sea algebra which is a type parameterize by X but I just mentioned
this because the fact that it is called sea algebra in category theory
it doesn't really help us so much it is suggestive typeclass has some
operations these operations maybe have some laws some algebraic laws
and so for this reason it is suggestive to call this an algebra but
it\textsf{'}s perhaps more confusing than suggested so let\textsf{'}s just not use that
terminology instead let\textsf{'}s concentrate on what these things actually
do so this I would call the method factor of the typeclass C I would
call the typeclass C inductive if such a functor exists why because
it appears that we are defining values of x by induction so if this
factor is given so this is some kind of container of some shape containing
X then we can derive new values of X using these operations so the
operations the value of this type that tells us how we can derive
new values of x given some previous values and how we're supposed
to have the previous values is described by the shape of the method
functor so for instance this method FUNKER could have a disjunction
of several parts and each part would have zero one or more X\textsf{'}s in
it and so that would correspond to operations with zero one or more
arguments in the operations of the typeclass so for the semigroup
the C is just a pair for others it will be more complicated but given
this C we can write down a general formula for the free typeclass
C over a type Z so the three encoding would be like this so the free
c generated by z or free c over z is a recursive type defined like
this the church encoding would be like this because we replace the
recursive use of the type through the type parameter X now quite equivalently
we could say the church encoding is like this it\textsf{'}s for every X of
the typeclass C in other words for which we have this value we have
this it is then obvious that also the laws of the typeclass will be
automatically satisfied by the church encoding after running and this
value and the reason is X must be of typeclass C and so after you
run this you would have a value of typeclass C so whatever operations
you apply to this are actually implemented in the typeclass C and
they therefore already satisfied law type constructors used in the
same way here I have shown what to do with new type constructors with
ordinary types it\textsf{'}s just that there\textsf{'}s more notation and more parameters
so for example the freetypeclass C over a type constructor F in the
church encoding looks like this where you have an arbitrary P from
typeclass C and then you have this generic transformation or natural
transformation which might be mapped into PA now it\textsf{'}s very important
to notice where the tag trailers are here so this a is the outside
a this is the only outside type parameter that is visible and this
type rather is hidden and also this is a hidden type parameter inside
so we have generalized from our examples to an arbitrary typeclass
let us remind ourselves what we have done first we start with some
arbitrary type z and we enriched it to a monoid which was a free monoid
we have started with an arbitrary that constructor and we enriched
it to a unit which is DSL that was motivated by the interpreter pattern
but this was actually a free monad so this enrichment was done in
the tree encoding by adding case classes that simply represented the
operations but there are also other encoded in cuttings that are more
sophisticated and so this works for any type Zi and any type constructor
and the result is a free type construction and this type construction
performs no computations it just accumulates all the data and it needs
to be run in order to actually perform computations and so intuitively
the free Mona and mono it over the type Zi adds some wrapping to Z
just enough to make it look like a monoid to satisfy the type signatures
of the moon your head doesn't actually perform computations inside
it just adds some stuff so that the result looks like a mono it similarly
the free functor it wraps a type constructor and just enough stuff
to make it look like a functor and we can interpret these free values
into non free values into specific concrete functors monomers and
so on by running please notice so we have seen several coatings and
running is done differently for these inquiries but all of these including
do the same thing they provide you a free type construction which
performs no computations it delays all the computations records all
the data that you to perform these computations later and later happens
when you run so you create a DSL program you can combine different
programs very easily and you can then run so what are the questions
that are but remain to us so what are the five classes we can construct
in this way so can we construct for a given typeclass C can we construct
a free instance on the typeclass over and given say FA the answer
is in turns out to be yes with some typeclasses no with others so
I will show examples I really started with functor I will show examples
of these typeclasses and I will show why you cannot sometimes have
a free typeclass which encoding is to use this is an important question
for reference some recordings perform better than others these encoders
are not the only ones available but I don't want to go too far into
other possible encodings and if you're interested look up church encoding
and you would immediately see other related components such as course
encoding every go encoding and some other info base and another set
of questions related to each other are about the laws so what are
the properties of this free instance can we define the free instance
by its properties formulated in some way in the turns out there are
four main properties that are important first of all if we have a
free instance of a typeclass over a type construct if it means that
we need to be able to wrap a value of F into a value of this free
type lasso this free C is this free instance of a typeclass so we
should have a function with this type signature now for second property
is that for all specific instances of this typeclass we should be
able to run our type free instance into that specific instance given
this function so this function the extractor as I was used calling
it before this extractor only shows how to map the generating element
or to generating type the F into m and once we know how to do that
we should be able to wrap the entire tree instance into him so in
other words is generating type the Z here and the F here has been
wrapped into some stuff to make it look like typeclass C and we should
be able to unwrap it into a specific M only knowing how to transform
to generating type into him so this extra stuff should be transferable
automatically into the correct typeclass the laws of the typeclass
must hold after running into that ami and the last property is interesting
is that if we transform the generating type into another type then
we should be able to automatically transform the free instance as
well so in other words the free instance should be so to speak a factor
in F except of course F is a type constructor so we need to generalize
the notion of laughter it should be covariant in the type parameter
F and so we should have some kind of map function that map\textsf{'}s atlandis
indeed if we look at this definition we had right here it satisfies
all these properties so for example this is covariant in F because
F is behind two arrows this is covariant in Z obviously so these are
the properties that we will show that how we already know that some
of these properties hold but we will show that more more formally
so what is the recipe for encoding a freetypeclass the recipe is this
first the typeclass needs to be understood as having methods that
is functions with some type signatures like this and all these Q\textsf{'}s
must be covariant complete there are some functors or some type constructors
that must be covariant in the parameter P and that is required that
all these methods should have a type signature of this form in other
words the final return type must be the type constructor not here
that were given if that is so we can put all of these Q\textsf{'}s together
in a disjunction like this call it s and then this generic function
will be a single value that represents all the methods of the typeclass
at once and then we call s the methods factor so once we do that we
define the tree encoding which would look like this it will be recursive
because these queues will have FC inside them perhaps but that that\textsf{'}s
fine now the queues might contain existential type primer so if if
these methods contain more type parameters on the left hand side it\textsf{'}s
fine they will become type parameters here and as we have seen type
parameters in case class inside a trait that are hidden from the trait
become exist tential type parameters and so those are fine those might
be present and then finally we can implement the run method for this
there are easier so if we are in the wrap case which is this part
of the disjunction then we just map it to him and we're done and for
all other parts these methods are run recursively so we run recursively
all the keys that are inside the queues and then we just use the method
of the typeclass M which will have the same signature except that
instead of peas will have specific values already of type M and so
we can just use those methods and that\textsf{'}s how the run function works
so this is very easy to encode the tree with coding it\textsf{'}s very straightforward
almost mechanical so you have a definition of a typeclass you can
mechanically generate this and it\textsf{'}s run method there\textsf{'}s no problem
at all to generate instance of the typeclass and the run method mechanical
in order to get a reduced encoding however you need to perform reasoning
about what are the possible values of this free typeclass instance
and what are the laws of a typeclass and how you can simplify if possible
values you would start with the tree encoding which is going to give
you some nested case classes and then you try to simplify them and
that is non trivial you don't have a generic procedure for doing that
so that has to be done separately in each case or factor for contra
contrary and so on so the factor we have done this but we will do
that for all the other typeclasses finally the church encoding can
be defined either using the tree encoding which again is completely
mechanical so we just do this you do the S and then you have SP TP
all in parentheses going to P put a type quantifier and P and you're
done or you can do a church encoding of the reduced encoding so that\textsf{'}s
a choice and there might be different performance in all these occurrences
so this in other words well we will show the more formally if this
is all true but if you have an inductive typeclass which is characterized
by a methods fantasy and so it\textsf{'}s methods are this algebra the C algebra
then you have a free instance always it has all the properties and
we have further properties for instance if P and Q are instances of
this class then the product of P and Q and the function from z2 p
where Z is a fixed type are also instances of pi plus C not necessarily
these sum or a disjunction and the product with a constant type are
not necessary parts of it last class but those are and this is relatively
easy to prove for instance if you have this and this means you have
a P is an instance of C and you have a Q as an instance of C and you
can derive this easily just project this to this project out the Q
C is a factor so you can do that project out the P to get sick you
get your P get your Q get their product so that\textsf{'}s very easy and it\textsf{'}s
similar will you come to this but you cannot implement the disjunction
for example because you would need to decide which part of the disjunction
you have but you can decide that because we have a C of P plus Q and
it\textsf{'}s not necessarily that you can decide which part of the disjunction
it must be so that function cannot be implemented without losing information
so that won't satisfy lowest similarly you cannot get Z P because
if we need to create values of type Z but you don't have them necessarily
plus Z so it\textsf{'}s also an instance of the same of course and indeed we
have seen looking at all the previous typeclasses that we analyzed
factored contra functor applicative Minard and so on they all have
this property the product of two typeclass instances and this construction
always again if you have a new type cross sixties so that is one because
they are inducted all inductive typeclasses have this property and
type constructor typeclasses have the same property just that the
methods function needs to be in coded language what type lasses cannot
be trained included and they're not inductive well one typeclass that\textsf{'}s
not inductive is reversible and indeed this is not true for traversable
this construction does not work for traversable if P is favorable
than a function from some types need P is not reversible in general
any typeclass that has a method that\textsf{'}s not returning a value of that
type that\textsf{'}s not inductive the reason is methods must be of this sort
so if you have methods not of the sort that did not return values
of this type then it\textsf{'}s not an inductive typeclass so here\textsf{'}s an example
imagine a typeclass for a type constructor PA it has two methods point
which inserts a value of a into P and extract which extracts a value
of a out of P now this would be of this form because it returns the
type construction but this is not it returns just bear type a and
so this typeclass that has both of these methods is not inductive
it does not have a representation of methods through a methods function
like this or like this and therefore we cannot do a free construction
of it using trees and we don't know how to encode free instances of
this class traversable functor is another example of non-inductive
now just to mention that if all methods of the typeclass have the
opposite form that it consumes a value of this type constructor and
then gives you something like this but if all methods are on this
for not just some so this is still not non good enough there\textsf{'}s some
methods did not consume values over this high class of this type constructor
but if all methods are of this form they consume PA and return something
then there is another way of doing a free instance which is called
Co free and these typeclasses are called Co inductive I'm not going
to describe them in this tutorial but perhaps in another chapter so
if we're going to free control function let me go to the code over
free factor which I have not yet shown but actually here\textsf{'}s the code
what I was just talking about I can define free instance of any inductive
typeclass I can write code for this generically so it\textsf{'}s a free instance
of an arbitrary inductive typeclass now this is not for type constructors
for simplicity this is just for ordinary types this before semigroup
monoid and such typeclasses and here goes the rap and the operation
so the operations contains the sea of free so this is the definition
but I just showed of the free instructor so that is the hopes and
basically that\textsf{'}s it we can show that it has methods of the typeclass
and we can run we can write the run method so you see the run method
is for a generic oh it means to assume is that C is a functor C not
not everything I see is this methods function and and I'm done so
the P method C is the evidence that the type P into which I am running
is an instance of typeclass C so this is a value that encapsulates
all the required methods so that\textsf{'}s why I called P methods C so P has
all the methods of C and this is a very easy code that just checks
whether it\textsf{'}s wrapping then it runs the extractor or if it\textsf{'}s the operation
then it runs the methods of T on the result of running under map so
I'm required to run under map because I'm under the sea and coming
here I need to run this first into a value of P under map of C so
then I get C of P and then I can use methods to convert that to pee
that\textsf{'}s how it works and there\textsf{'}s an example I use this for a generic
construction to define some free semigroup so these are this is the
methods functor for the semigroup I'll show it as a factor this is
a free semi-group that\textsf{'}s it generate it by screen now this is just
to be clear this is my own definition here it\textsf{'}s not the cats library
definition or any other or scalzi definition I believe in those libraries
what is called free is actually the free Mona it\textsf{'}s a free Mona not
just freaking structure any type cons so in my definition here it\textsf{'}s
of generic free instance of a given typeclass for inductive typeclasses
and here\textsf{'}s code that creates some values and runs them so I create
ABC XYZ wrap them then I use operations to add them and I run this
into a string and I'm done and I can do the same with the church encoding
it\textsf{'}s slightly more type definitions but it\textsf{'}s exactly similar the church
encoding is this type equivalently it is this type and so I just define
the trait for the argument and then I have this and then I can show
that it\textsf{'}s a five class now the showing that it is a typeclass is actually
non-trivial this church encoding must have methods of the typeclass
C in other words it must have a function of this type C of the charging
: must be mapped into the church encoding itself and the types here
are different with an X inside here and all Y inside here because
are different so this actually is not very trivial to derive you just
need to be very careful and following the types and here\textsf{'}s how this
works so how can we get this value now we would get this value we
have Z or we don't have a Z obviously all we have is this we have
C of something so it looks like our only hope is to get this C of
Y and then we would have a function from C of Y to Y and we can call
on that C of one and that give us the result now we are given this
so we are required to be able to work with an arbitrary given Y so
imagine we are given someone but then we have this function which
has an arbitrary X inside that we can set so this function accepts
arbitrary axis so let\textsf{'}s set this X equal to Y in that function and
then we would have a value of C with Y in here Z plus C of Y going
to 1 but we have AZ plus C of Y going to 1 we are given that value
so we can substitute that value in here mapping undersea so that we
can get a sea of why as a result so then we get a sea of why we can
put it in here and get the white house so this is a little convoluted
but that\textsf{'}s the code we map under sea given a type train with your
Y in here to run the free instance and then we use the ops and the
resulting sea wide so this is this gives us a CL what we run the ops
on it we get away and that\textsf{'}s what we returned so if you want to understand
exactly how this code works you would have to write it yourself it\textsf{'}s
a lot of manipulation with types all these types are permit rised
so this is kind of technical and not so easy to see looking at the
code but this is the only way to implement the required type signature
which is this and now we implement the rap which is straightforward
and the Run which is straightforward by using these methods so this
again we run the semigroup example is exactly the same code up to
some syntax changes give us exactly the same results so let us now
look at the functor so I already showed how the functor here the free
functor is implemented and I derived the tree encoding and the reduced
income and then I started to talk about the church encoding and that
was so complicated that we have to go through much slower and through
all the parts that are required so let\textsf{'}s now look at the code for
the free functor so the free functor actually starts from a given
type constructor so it\textsf{'}s always a free funder generated by a given
type constructor or a free funder over a card constructor and that
that constructor doesn't have to be a function and actually in many
applications it is a node is not a factor and it cannot be possibly
a functor because it has index types with specific type parameters
and it tries types with non-free specific sign type parameters like
this so I call these things unfocus it\textsf{'}s a funny term that I think
is appropriate here because this is it is like a factor because it\textsf{'}s
characterized by a type but it couldn't possibly be a furniture because
of the way it\textsf{'}s defined so this unfactored could be seen as describing
two operations in some business logic where you add the name to some
database say and you get an ID back don't you get named by ID and
this name may may not exist in a database and so the result is an
option of string now we would like to transform this unfilter into
a factor run some program with it and then transform that into an
ordinary option of some result so say sometimes you would have an
option of strange sometimes not so let\textsf{'}s be safe and run this into
an option so in order to do this we need to define an extractor or
interpreter for this and type constructor into options so this is
this type which is natural transformation but actually doesn't have
to be natural transformation it\textsf{'}s just a generic transformation from
one type constructor to another I'm using the cats library type it\textsf{'}s
defined there so how do I transform other two cases if it\textsf{'}s alone
then I just transform it into a one and if it\textsf{'}s an option string then
I transform it into none so doesn't matter it\textsf{'}s just an interpreter
of some kind it doesn't all it does is gives me an empty option or
non empty option it doesn't really do a lot of good so that\textsf{'}s what
I'm going to be using so I'm going to right now a free factor based
on this free function over this on Fronter I'm going to write some
maps on it add some transformations and then run into an option that\textsf{'}s
going to be the example so here\textsf{'}s where the code starts this is the
three encoding of the three factor and have the wrap case class I
have the map case class map case class has an existential type B I
define the function instance for the typeclass functor and the map
method does nothing just creates a new wrapping with map typeclass
case classes then I have my function here which creates a free program
300 programs all it does it applies map to some given value many times
so this number of iterations is given so that many times I apply map
with the same function to that value so this is just a test I'm going
to start with some value wrap it and then map many times the runner
I'm trying to make it stack safe and that\textsf{'}s a little bit of a problem
actually and here is why I will have a lot of nested case classes
I'll have to and go through all of them so the unfold function and
I'm defining here is going to call itself and so one thing I could
have done is just applied in the map we see that F of Z is a free
functor value I could just have run it through the same function and
then apply the map with F to it but that would have been not like
safe so I did another thing so I didn't take a recursive unfold where
I first accumulate so what I did here actually is I did and then George
that was my first implementation every time I have a map I have a
new function so the result is this accumulator that I have in the
unfold and so I just accumulate all the functions inside the map and
I run them only at the end so I I run this only at the end so that\textsf{'}s
the idea so that is tail recursive and then I hope to be stack safe
but I'm not stuck safe this will actually give me a stack overflow
and the reason is that and then is problematic in Scala and that was
quite surprising to me because it\textsf{'}s not so in Haskell and I didn't
expect it but and then is actually not stack safe in this car here\textsf{'}s
why here\textsf{'}s my sample test code that shows why this is so let\textsf{'}s compose
a large number of functions and call the resulting function so here\textsf{'}s
a code it does this let\textsf{'}s just compose a lot of functions all these
functions are just adding one to their argument and so that\textsf{'}s very
simple but if we do do this with a large number of functions we'll
have a stack overflow so whatever we do we'll have a stack work for
we we can compose these functions or we can directly compute still
we have a stack overflow so the reason is that composition of functions
introduces another stack frame and that\textsf{'}s unfortunate and it cannot
be removed apparently in job in JVM for technical reasons so you could
not have a compiler that automatically removes this extra stack frame
so what do we do well cats library includes this and then structure
which you can use and if you use that so if you start with this and
then you compose with more functions then this is stack safe how does
it work it doesn't actually compose these functions until much later
until you need to run this function and then it actually accumulates
all the functions you give it in the list and then it runs that so
I implemented the same kind of thing which I call safe compose using
a data structure called chain which is from the cat\textsf{'}s library which
is a very high-performance list and so here\textsf{'}s what I implemented based
on some of these suggestions actually Michael Gilchrist suggested
this so I just did good massaging so now the idea is that instead
of composing you wrap the function into this chain F data structure
and then composing this with others things doesn't actually compose
functions it adds to the to the chain of functions of these functions
are all stored in lists and only after you run them so the apply function
is called then you fold over the chain and apply these functions so
that\textsf{'}s how it works and I made it so that you can compose on the left
or on the right with the chain and you get a chain again so this what
I did and the result is good so it\textsf{'}s actually faster then using cats
library and attend so that\textsf{'}s what I had to do introduced here I have
to use before before is my replacement of anything which is easier
to use than cats library and it is faster so this is what I would
do if I didn't have this I would accumulate results functions in the
list one at the end I would fold over that list so this is kind of
uglier and it\textsf{'}s better to put this code into a library and here\textsf{'}s
a benchmark free functor in the reduced encoding it has fewer at least
classes but it has a more complicated map because now it needs to
do this in the map notice that in the tree encoding the map didn't
do anything it just wrapped the data into a nucleus class but in the
reduced encoding the map method of the factor actually performs a
computation it already composes the functions although were revealing
this smart composition but that\textsf{'}s what we do before and after are
the smart methods of I implemented here in place and then compose
in the sky and then I do the same testing and it\textsf{'}s slightly slower
than three encoding 
\end{comment}

\begin{comment}
now let us consider the church including of the three functor the
church including is our equivalent as types but they have different
performance characteristics and they are more complicated to implement
the three including of the three factor is like this and in order
to implement it we would have to first implement the trait that hides
this inside type parameter you will have to implement the trait that
hides this type parameter and the outside parameter as well so that\textsf{'}s
why there is quite a lot of boilerplate involved so first I define
this auxiliary type and then i encode the free factor using a trick
that if if I look at the type expression here then this is the same
as the magnitude of the function so I can transform this type expression
into this which is an equivalent type and this is the same as the
magnitude of the factor and so a shorter way of implementing a free
typeclass in the church tree encoding is to impose a type constraint
typeclass constraint on G and to declare this as an argument so that\textsf{'}s
why I'm saving a lot of typing if I do that but it\textsf{'}s not necessary
it\textsf{'}s equivalent to doing it in a straightforward way now the result
is that in life refactor looks like this it has a single method trade
parameterize by an arbitrary type constructor which needs to have
a functor constraint and the type of the method is this now we know
that the three-factor has a punctured typeclass instance and in the
tree encoding it was trivial to define it but in the church including
it is much less trivial because of the complicated type signatures
of these functions so these are functions whose arguments or functions
and type parameters are hidden inside so here is the definition of
the factor instance for this type I need to define a map function
that takes a previous ffs away in function a to be and return the
new F F of B where F is kept as a type parameter throughout now in
order to return a new fffb the only thing i can do is to create a
new anonymous instance of the trait and overwrite the run method so
now I'm here in the run method I'm supposed to do this so let me write
a function so now I have the following data I have this church encoded
fffe I have a function A to B by heaven FF C which is simply a natural
transformation F to G or if you wish generic transformation after
Jesus if F is not necessarily a function so given this data I need
to produce G of B how do I do that well the only way to produce any
kind of G is to call this run method from the previous F of a which
will give me a G of some X where X is up to this F of a so is going
to give me G of a out of here there\textsf{'}s no other way I can get any G
here I need to run therefore this method but I'm free to specify a
type parameter for this run method so I specify the G which is given
to me here at that type parameter and then I apply that to an F FC
which I have so that gives me a G of a now I need to write this syntax
and not just run away for see because run has an implicit argument
and it will be confused if I do that this will not work so having
gotten G of a I just map it through the function f into a G of B since
G has a functor instance so I do it like this just very explicitly
so that\textsf{'}s my implementation now in order to use this I need a helper
function that lifts values of F of a into the free functor so that
is like that we need to implement the run method so the only way to
create a value of the free founder is to do this too to have a new
FF with the run method so then I have an FF C I can apply this to
FA and I get G of a out of it since F FC is directly a transformation
from F to G so that\textsf{'}s very easy and another helper function is to
interpret free factor into a given specific function G so that is
also very easy just to run the function the run function itself is
already the interpreter that is a defining characteristic of the church
encoding it is encoding which is assembled out of pieces of the run
function so the value of a church included type itself is already
its own interpreter now performing tests performance tests given some
reasonable results however some paper claims that church and coatings
are always slower it\textsf{'}s not necessarily always the case you need to
benchmark your code if you want to be sure now the problem with this
code is that there there\textsf{'}s a stack overflow because this run method
is actually not stack safe again as before I already said stack safety
in the Trojan Queen it is up to the implementation of the run method
if the run method calls functions too many times then it\textsf{'}s not going
to I'm not going to be able to make it stack safe in the other parts
of my implementation and can't really made stock safe the reason being
that I have to do so the map function needs to do this I have to apply
to this FFC and you have to run in the map function so since map function
has to run there\textsf{'}s no way to guarantee stack safety every time if
you have a million maps when I would have a million nested calls here
so that\textsf{'}s a stack overflow now let\textsf{'}s consider the church encoding
of the reduced encoding of the free function {[}Music{]} reduced in
Korean is slightly simpler for the free function and so let\textsf{'}s search
encode that we can search encode anything any any type can be Church
encode the question it doesn't bring us any advantage it turns out
it does because the church encoding of the reduced encoding of the
free factor can be made stack safe and here is how so first of all
we cannot do the same trick as before with the typeclasses we have
to encode directly this type expression so that\textsf{'}s encoded directly
so first we called the exists tential type so we named a straight
and it will have a single case class representing this value and here
instead of B and is a Z so that\textsf{'}s the implementation of an existential
type expression then I implement this function for all a from that
to G of a so that\textsf{'}s the straight with apply method parameterize by
a and finally I'll code the type by having another universal quantifier
outside which is paralyzed here so in this way I have encoded this
type expression so this is the boilerplate in Scala that is required
in order to encode this type expression now to define the function
instance I'm going to be more careful in order to make its taxi so
I'm going to pull things out of the run method they have to be if
they don't have to be reevaluated and so one thing I can pull outside
is to reevaluate the run it can be done once so it can be done outside
of this room if I do that and I actually can achieve stage stack safety
another way but I use the before method which is my own implementation
my IntelliJ is a bit confused right now but before it\textsf{'}s my own implementation
of a function composition which is tag safe so how do i implement
this well it\textsf{'}s kind of cumbersome because of all these boilerplate
and types I have a very complicated type expression it is still a
factor the church encoding doesn't change the properties of the type
that will church encoding it just adds a lot of functions and have
a choice types and quantified types but it doesn't change the properties
of the type it\textsf{'}s just a different encoding of the same type so clearly
we should be able to define a furniture instance if we are able to
define a factor instance before church including but defining the
file that would just be more work but this has work done only once
so here\textsf{'}s what we need to do we need to define a lab method and that
method needs to return a new FF of FB so we return that and overwrite
the run method in it run method takes an FF C and now we have the
situation at work that we have this data and we need to produce G
OD now the only way of getting G of B anywhere is to use this run
to run this but running this will give us a value of death commit
which will produce us G of B so where do we get that value we need
to get that value out of our church included free function that we
have we have before so now let\textsf{'}s remember that the church encoding
is equivalent to the type that is being encoded so you can extract
that type outer that encoded value so since the church encoding of
the reduced encoding of the free factor is basically encoding of this
we can retrieve this back from a church encoded type no need to do
this is where it is done so basically we get this free F of fa out
of C F of a which is basically running it with identity so random
with identity is the way to retrieve the underlying type out of a
church and call it that so that\textsf{'}s what we do in these two lines and
we put these lines outside of the run because this needs to be done
it doesn't depend on these arguments needs to be done outside a memory
also achieve star safety when we economize on the stack doing this
one side of the run function so haven't gotten this value we imagine
it or we imagine it just because we need to extract the parts the
F Z and Z to be parts of this 3 F 3 F is just the case one side you
find right here with this map C so I'm going to extract those things
in it to match and this is a key part of the implementation I'm composing
the functions inside the map case class I'm not actually running the
map anywhere yet I'm only going to be running it here so this together
with putting this outside achieves tag safety if I remove this replace
Ness with and then or if I put this inside into the run function I
will have a stack waterfall everything else in this encoding is very
similar to what we had before the wrapping the run is trivial the
interpreter is stack safe it just runs them up and we guarantee that
run is stuck safe so performance test shows us that this is actually
significantly slower then especially creating this method and all
these things that come with creating nested Maps method is significantly
slower than other in puddings but as we will see later Church encoding
has certain advantages so if performance is not a great concern but
flexibility of design is a great concern as well in charge encoding
has significant advantages so this concludes our implementation of
the free function let us look at other three type questions how they
are implemented the next typeclass is the free culture hunter in order
to implement it let us follow the general procedure first be right
down the methods of the typeclass it has only one method which has
this type signature we realize that it\textsf{'}s inductive because this method
returns again the value of the type we're constraining and also we
realize that it has an existential type inside because it\textsf{'}s parameterize
with this be there for the three encoding looks like this it\textsf{'}s again
we are just following the general recipe what tree encoding is a recursive
type what is made out of a disjunction the first is the wrapping of
the generating type constructor F and the second part of the disjunction
is the method the method factor now in this case this is what we need
to do this is the accuracy of instance of that type now reduced encoding
quite similarly to deriving the reduced including for the free filter
we derived for the free control enter a value of a free country func
you type in the tree encoding will always be of the form that either
we have a NFB or we apply several times map to FB which will add here
I forgot the direct product sign we add a few terms of this kind with
products of functions and existence or quantifiers the difference
between this and a fee factor is the opposite directions of the function
arrows so this starts with Z 1 and then we have a function Z - 2 Z
1 and so on the N to the N minus 1 B to Z and the result is a contra
factor parameterize by B however the property of contra funky are
still such that we need to compose these functions associatively and
we can compose them before doing map or other contra motive or we
can compose them after doing Countryman that\textsf{'}s a composition law there
for the reduced encoding can simply compose all these functions and
put them inside a single function of this type and so the equivalent
type is just this with all these other and residential types simply
dropped since they are not used so that\textsf{'}s how we derive the reduced
encoding we need to figure out what are the possible expressions and
how they can be simplified using the laws of the typeclass there\textsf{'}s
some simpler type expression notice that the reduced encoding is non
recursive just as it is for the free functor and I'm going to show
code now this free country function might be a little difficult to
understand or to see where it is to be used but it is just a general
scan you know I don't take any type constructor and wrap it into some
stuff can make it into a pet typeclass instance of an arbitrary typeclass
so for instance I can take a function such as this one you know identity
factor and I can wrap it into a freaking tree function so then this
will become a contractor after wrapping and I can create a control
factor program by replying confirm I have a few times to this then
I can interpret the results interspecific control function such as
this one and the interpreter the only thing the interpreter needs
is a function from here to here this function example the something
like this where we take a value and return a contractor and contractors
are usually consumers of values so imagine you're logging something
that can see so the logger is a typical consumer of values I'm just
going to simplify this very much and consider this function as a contractor
and then this would be a transformation from identity factor to this
control factor which is prepending prefix to the log message and that\textsf{'}s
I'm going to show the code in a few minutes that\textsf{'}s how we would use
a free contractor anything that another important property is that
if the type constructor F is already a contractor then this wrapping
does not produce a new in equivalent type it\textsf{'}s the result is equivalent
the free country function of over F is equivalent to F you just like
it is the case for the pre factor or the look at example code so here\textsf{'}s
how I encode the free country function tree encoding I encode the
wrapping case - and then code the culture map case using an existing
show a quantified type when I create a helper function to wrap things
it\textsf{'}s just putting it into the wrap is constantly the country funky
turquoise instance which does nothing but wrap into the typeclass
sorry into the in our case cost so there is no complete computation
done here other than memory allocations finally I write an interpreter
which is trivial you just run you do a controller and I implement
it producing coding reduced encoding is shorter he just has a single
case class with existential 25 type a wrapper for than a reduced encoding
now that wrapper is less trivial because we don't have the wrap case
anymore so wrapping a value of F means we have this reduced case when
we have we put this F here and we supply an identity function in this
place the control factor instance is stuck safe because we reduce
every function we don't apply map and you are we and I don't run anything
we just collect all these functions and we collect them in a stack
safely using this before method which is a stack safe alternative
to and then and here is a rather so we just run by extracting a value
of C out of the value of F and then running a contra map with the
single function that is left so that\textsf{'}s the example I just described
we have a logger with the prefix now the writer factor is going to
be wrapped in two star to make it into a country funky so we have
a free country founder over the writer function so the fact that it
is a factor is just I chosen us to show that I can take anything but
including the furniture that certainly is not a country function and
I can wrap it into this construction 3 CFR and the result is a contractor
so that\textsf{'}s an that\textsf{'}s an interesting property so the result is is a
contractor but if you look at the country factories type signature
it is say this it\textsf{'}s a contractor in being but it is no longer a functor
so even if the constructor that we used was a functor it is no longer
function and indeed so it\textsf{'}s a contractor indeed so we have we if we
do this we take a function like the right here function we wrap it
into American structure and we lose the factor for it\textsf{'}s a little bit
you know it is a country function just to make it clear what\textsf{'}s going
on and here is some example code where and have some prefix logger
and I wrap my writer function which could be computed after something
sound function computation that I wrap it in their country factor
then I do some contour map on it and the result can be used I interpret
that I run this thing and the result is as expected the next example
is a free pointed front kick a pointed funder is a flat class it has
a symbol method other than function so if we say that this is an arbitrary
type construction done that pointer factor pointed functor class has
two methods to the point which is this or it is just the same type
signature as the pure method and moanin an applicative but since this
is not going to be a monitor a negative this method is called point
it just takes a and inject stuff into p8 and map so this is familiar
so what\textsf{'}s the tree encoding of this well it\textsf{'}s just a so we follow
the recipe we have the wrapping yes we have the first method which
takes a few turns PA I have a second method which takes this in turns
so that\textsf{'}s how we encode so we have a disjunction with three parts
always going to be like this one part is going to be the wrapping
and the other parts correspond each to a metal in inductive typeclasses
are all going to be minus three encoded to derive the reduced encoding
we're going to have a bit more work need to do a bit more work we
see what kind of expressions can be found by using this definition
so we took we take the tree encode it and reason about it so either
we take an A and apply a bunch of maps to it so we first apply points
to some value you get an a value type a and then we apply some map
state so that would be one possibility another possibility that would
take some F a wrap it and then apply some maps to it so therefore
we only have two cases one is like this a general value of this type
will be either like this or like this well it also could be a single
affair or a single a but none of those so if we do have those things
we can compose all of them and just as we did before so we just have
one function one function is sufficient now consider the second case
we have a value and then a function we can just apply this function
to a value and we just have a single pure value so we can encode the
single pure value therefore and we can encode a function x sorry a
functor wrapped x a single function and if if we just have this we
put identity function in there as we did before there for the reduced
encoding has only two cases one is a pure value and the second is
this wrapped constructor times the function and it\textsf{'}s not recursive
so that\textsf{'}s very nice note that this is exactly the same as a free function
over F Z so basically this is what we have if the type constructor
F is already a functor then this is equivalent to F itself therefore
a three-pointed function over a functor is just this it is a very
simple expression so just adding the type a to a factor makes it into
a three-pointed so that becomes appointed and it\textsf{'}s actually free pointed
and of course if it is already pointed factor we should not use this
construction because then this would be not the same as a factor itself
so unlike other cases if we just saw I should not use a free construction
if a factor already has the pointed method only it functor and contra
fun to have the property that applying the free construction doesn't
change them all other typeclasses will change usually when you do
free type construction so for example free wound at over a moment
is not the same moment free pointed over a point that is not the same
function only factors and country functions pure factors pure country
funky typeclasses do not change under applying the free construction
let\textsf{'}s look at the code it\textsf{'}s very easy to implement this we need three
case classes that encapsulate these three parts and to implement this
we only need two his courses and in implementing the factor is very
similar except now we have a point case so in a point case we need
to implement the function by him so applying a map to the value of
this type will just need to apply that function to that type so consider
a pretty filterable now the filterable typeclass was explained in
chapter 6 it is not a class that is widely known so look at chapter
6 for more details it has two methods map and map ok so these are
the methods of of inductive kind or inductive type signature when
they return the type of know actually it\textsf{'}s sufficient to keep just
map upped because we can restore map from it so let\textsf{'}s not overdo things
and let\textsf{'}s just implement one method in the free construction since
we can easily get this out of this in other words if we have this
function and we can implement this function by substituting a going
to zero plus B yeah so the tree encoding has two cases very similar
to tree encoding on a three-factor you start here we have this type
signature if the F type constructor is already a function we can simplify
the tree encoding by using the identity the basic identity of existential
types and then we just obtained this recursive definition so we'll
get rid of the essential type and this recursive definition can be
visualized as an infinite disjunction like this so it\textsf{'}s F a F 1 plus
a F 1 plus 1 plus a and so on so clearly applying filter function
to this will give us this applying filter to this will give us this
so on so it\textsf{'}s this in this way it\textsf{'}s implementing the free filterable
in the green color now this is not the most economical encoding and
it reduced including actually is like this it is non recursive and
you can size in order to derive it we do the same procedure as we
did before these are we stay this is this should be essential like
25 not universal level correctly since lights now an arbitrary value
of this type in the tree encoding would be FA to which a bunch of
map opt have been applied so that will give you a product like this
now using the laws remember that these are composed using the class
like composition because these are of type a to option B so these
can be composed since option of the moment so that composition needs
to be done and it gives us a single function so that can be done and
if we just have a single affair with no function that will encoded
like this so we can still encode it so that\textsf{'}s going to be reduced
encoding and the most interesting simplification is when F is already
itself a factor then we'll use basic identity and we get F of 1 plus
a so the free filterable over F factor is just this you can just implement
a filter for any factor applied to an option so for any function f
f co-option of a is filterable and that\textsf{'}s a free filterable over a
functor and this is a free filterable over an arbitrary type constructor
so we see again that free filterable over f is not the same as f in
a very similar way we can construct filterable contra factors free
filter whole country hunters will not go into details about free filter
will control factors because that is completely analogous here\textsf{'}s the
code for the free filter will factor I'm just sure this is nothing
new in terms of how to implement existential types and recursive types
in three encoding introduced encode consider now Freeman and the moolaade
has two methods pure and flat map the map method can be recovered
and in this way we formulate inductive teleclass now just a comment
and we have seen in previous chapters that typeclasses can be formulated
in different ways you can for example cumulative monad as having methods
pure and flat map or you can do flatten instead of flat map but flatten
does not have the same power as pure and flat map because you cannot
restore a flat map from flatten so you would have to have map here
as a third method if you wanted to if you wanted to do a free one
over an arbitrary type constructor that is not itself already a factor
for the filterable there could be different ways of doing the definition
as well and for the implicit if there\textsf{'}s the different ways but what
we need is a set of methods that return the type that we are constraining
type itself and not something else and we need a set of methods that
are sufficient so without assuming that the function instance is already
given for example so that\textsf{'}s why would she was pure and flat map here
and we can recover map from that so now the tree encoding is very
similar so what we have before except now it has two places in which
we use the same type recursively free m and freedom so the reduced
encoding needs to be derived let\textsf{'}s derive it so first of all let\textsf{'}s
see what happens come on we use the tree encoding and create some
values of the free monotype first we can take this cut type constructor
and apply a few flat maps to it second we can take this constructor
which will appear when the playa few flatmap start but these are the
two possibilities if we take a pure value and apply a flat map to
it that can be simplified due to the laws of the minute so if you
take something else and apply two flat maps they also can be simplified
to a single flat map with a more complicated function here again this
is what social tivity load of the Bonett so therefore it is not necessary
to have many flat maps here they can all be collapsed to a single
flat map maybe with a more complicated function inside so so so then
clearly the first element in the product does not have to be a pure
that can also always be replaced so the first element in the product
is going to be F a or F C for some for some Z and then we have a single
flat map so we don't need more than one flat map however this does
not let us encode the pure value in without any flat maps applied
to it so that means we cannot just have one part of a disjunction
we need to we still need to keep this part of the disjunction that
we have here but we can eliminate a fee and we can eliminate one of
the recursive usages but not the second one so the reduced in chlorine
is still recursive it is somewhat shorter but it\textsf{'}s still recursive
now one comment is that recently the so-called final tagless style
of programming has become more known in a scale community has become
also known and has gotten into a few years before in my terminology
what is called final Tablas style is nothing more than the church
encoding of a free moment so you can do Church encoding of any type
and you can do free going out without a church encoding and if you
want you can do Georgian going over three mooner and you have a choice
you can charge encode this or you can Church encode this and that
could have different performance implications however just keep in
mind I'm not going to talk about final tagless because it\textsf{'}s not really
something specific or or special to jamuna\textsf{'}s or put the portal DSL
it\textsf{'}s just the church encoding of a free movement and you can choose
it for certain reasons or anything not choose it for other it is stack
safety is important I have just found that the church encoding of
a free factor is not stack safe and unless you use reduced encoding
first so you first reduce the improvement using the frontier laws
and the nutrition code the results and that can be made stack safe
most likely it\textsf{'}s similar with monads and because the three encoding
is twice recursive use of the type so that was probably going to prevent
you from being stuck safe this is difficult enough to make a stack
safe but may be possible certainly I would if I were to make a library
I would use reduced encoding and Church encode that as an option but
also provide non-church encoded reduced encoding of the free moment
as an option there\textsf{'}s almost never advantages in using a non reduced
encoding but there might be an advantage in a church encoding so again
let us consider what happens if you do a free Mon and over a functor
so you can actually save yourself a lot of trouble because if F is
a factor then we can use the identity which says that this expression
is equal to F of this which is this so now we get recursive definition
which is much simple which is a free monad over a factor so I would
also provide this as an option in the library because it\textsf{'}s so much
simpler and more efficient perhaps definition and also it shows you
that a free monad is different from the mana of itself if you just
substitute it into here warranty here so the free mode of Ramon odd
is not equivalent to that normal so don't do it as an exercise we
can ask what is a 3-1 out over a pointed functor so again all we need
to do is we need to start with 3 encoding and try to reduce it so
how do we reduce it well first we start with this clearly the pointed
factor doesn't mean this part of the disjunction and clearly we can
start with this encoding collapsing all those flat map functions into
one and the result is going to be this and that\textsf{'}s it we cannot really
simplify this because we don't have a pure encoded so we cannot say
oh let\textsf{'}s only have this case because we cannot encode a fade so we
don't have a permitted for free we have a pure method for F a itself
but not for free we have eliminated that so we cannot encode a failure
we cannot save us as part of the disjunction therefore this is the
reduced encoding we use again the identity for the existential type
to get F of this so therefore the reduced encoding is FA plus F of
this still record so going to be recursive free if free M of F is
FA plus this so that\textsf{'}s reduced encoding here is a code for the free
moment so you spend a lot of time on this let\textsf{'}s consider the free
plug ative which is an interesting tie class it has two methods we
choose pure and app because the other choice would be for example
wrapped unit and zip they do not return the type that we're constraining
so wrapped unit returns F of one not F of a and zip returns F of pair
a B but we cannot have that as a return targetnode type must be the
simple F a with no changes if it is not that and our typeclass is
not inductive so if the typeclass allows us to have a formulation
equivalent to the previous formulation of the typeclass such that
the new formulation contains only methods of inductive type signatures
that is methods that return the typeclass the instructor constraint
then it\textsf{'}s an inductive type cons so as before we think about which
methods to choose and which is pure and half because map can be recovered
from these so the tree encoding is straightforward and in order you
can have two usages of the recursive type let\textsf{'}s derive the reduced
encoding that\textsf{'}s going to be this so how do we derive that so we reason
about what structure of the values we will have either we take F a
and we apply a bunch of apps to it or we take a and apply bunch of
apps to it so we can encode if we have an a to which we apply we can
encode that as something like like this so we can do an app with a
pure here with a different type signature because that\textsf{'}s basically
it\textsf{'}s going basically going to be a Mac due to the laws of the zip
so it\textsf{'}s going to be equivalent to if free app on the Left just the
one that is on the right we'll put it on the left and on the right
we put some other thing so that means if the first one does not have
to be a pure well the first value in this product it does not have
to be a pure value it can always be wrapped F so but then we cannot
encode the pure value so we still need in that case so therefore we
need only two cases the pure value and the wrapped constructor with
a free app so that\textsf{'}s there for the reduced encoding a free applicative
over a factor is taken from here we use this representation and what
we find is that we cannot really reduce this using the identity for
the existential type because this is not of the form exists Z and
then F Z and Z goes to something it is not Z goes to something it
is a type constructor so therefore free applicative over a factor
it looks like this we can still reduce this but we do need to have
a rap constructor but the a is outside of the recursion so the only
game that we have is that if the type constructor F is already a functor
then the pure value is outside the recursion and other than that it\textsf{'}s
a very similar construction so first we construct the recursive case
which is which can be seen as a reduced encoding of freezy bubble
that is a type coins were only the ab method is given the no pure
so this is a freezie bubble and then we do any three pointed out of
that it\textsf{'}s a free pointed over a freezie bubble over a functor if and
we see that again free applicative over applicative is not the same
as that functor so here\textsf{'}s the code of the free applicative in the
trie encoding and introduced including so having gone through all
these examples let us generalize what our laws of the freetypeclass
constructions so we will consider a general inductive typeclass and
for simplicity we will not consider type constructors here only ordinary
types so the typeclass will have methods of this type signature but
you see all the examples we have seen have been inductive so it\textsf{'}s
going to be equivalent for them except is going to be much more syntax
for all these types of constructors and type parameters so I'm not
going to go through laws for type construction typeclasses they're
going to be my analogous up to a much more complicated syntax I'm
going to consider in inductive typeclasses that have just ordinary
types as elements of the or instances of that class so a typeclass
with a method functor c has these methods and this is the definition
of the three instance of c over c over a fixed type z so that\textsf{'}s we
have seen so there are several general properties that this construction
has i'm not going to consider church encoding because the church encoding
is equivalent to this it has exactly the same properties but it\textsf{'}s
much more complicated to reason about it\textsf{'}s just a much more complicated
type I'm going to use a recursive algebraic or polynomial definition
of the three type instance the first property is that this type is
actually an instance of typeclass C so in other words we can implement
these methods all the methods are summarized in a single value of
this type and so once we show that we can implement a value of this
type we're done we have implemented methods another property is that
it is a function in Z so it has an F map which works by changing the
generating type another property is that if we have a specific type
of that see we can implement these functions around and wrap so the
wrap will lift a value of Z into the freetypeclass instance and the
run will take a free typeclass instance it will also take this extractor
function which translates Z into P and then it can translate the entire
free instance entity these functions have certain laws the first law
is that run of the rap is identity so what does it mean um if we wrap
a value Z and then you run it then it\textsf{'}s the same as if you did not
wrap so he transforms e to the same P if you first just take Z to
P or you first wrap it and then you run it you get the same P so that\textsf{'}s
this law the second law is the natural allottee of run so the run
was this z type argument and in this type argument it is natural so
the naturality law as usual it is a lot of how to put F map outside
or inside of your function so this is typically an equation of this
sort you left map either the right hand side has no F mat board has
an F map on the other side so in this case it has no worth map here
is a type diagram so you start with a crease free instance of C over
Y you can transform it into a free C over Z using the F not function
f map of F f is an arbitrary transformation Y to Z and then you can
run this into P or you can run this into P and that shouldn't be different
so you can first run directly with the combine function or you can
run first by transforming into Z and then you can run from Z and it
should be exactly the same so that\textsf{'}s what this law specifies another
important property is the so called Universal property and this property
says that the Runner is universal you only have one runner so if you
have for example to run into two different types and the same runner
works for them in a way that is compatible so suppose you have two
types a and Q and you want to run into P and run into Q and there
is a function that transforms from P to Q well this function preserves
the typeclass so it\textsf{'}s not just an arbitrary transformation from P
to Q it\textsf{'}s a transformation that preserves the time class the property
of preserving the typeclass is this diagram which is that here is
the methods frontier of the typeclass and here\textsf{'}s the method for P
and here are the methods for Q and if you map P the Q and the methyls
are also mapped to each other so in other words for instance if the
class is a monoid then the function f must transform the unit element
of the mono would P into the unit element of the monomyth Q and it
must transform product in the P monoid into the product in the QM
owner so that\textsf{'}s automatically guaranteed by this diagram because the
ops value is already all the methods but all the metals put together
into a single so if the function f from P to Q satisfies this commutativity
condition then what I say it preserves the typeclass and for these
functions f you can see this diagram that you can run first into P
and then you transform into Q or you directly transform into Q and
that\textsf{'}s the same function run does it so that is another kind of property
that is it is quite important so let\textsf{'}s see how these properties can
be proved so first I will repeat the code for the universal construction
of a tree encoded typeclass now you noticed I have been using a tree
encoding here again this is the simplest encoding it is mechanically
produced does not require any reasoning in order to be implemented
whereas the reduced encoding requires some reasoning improves to show
that it is actually reduced and adequate for encoding all possible
values and so it\textsf{'}s much more difficult to reason about although it
can give you advantages in performance so that\textsf{'}s why I'm reasoning
about the tree in going here so I implemented the function ops that
basically says that the free typeclass has the methods that are required
for that typeclass C so the opposite of function from C all free to
free so that\textsf{'}s this function and that that\textsf{'}s the first one that I
need and put in this the second is that I can wrap that is trivial
I just put the wrap constructor then of that I can interpret interpreter
it\textsf{'}s just the general interpreter which we have seen I sure that it
is a functor well this is a simple exercise in making a functor instance
for a recursive type where I use the recursively the map function
in the hopes because obviously the recursive part of the type so the
first law is that um if we run on wrap let\textsf{'}s our identity on both
sides from the first law are functions of this type so therefore we
need to apply both functions on arbitrary value of this type so let\textsf{'}s
call this value free seasonally and also instead of this because we're
working in this type the runner needs an even argument of this obstacle
need to have the hopes for the free type constructor so that\textsf{'}s our
hopes you find above so if we do that and we substitute the code then
this is what we get now if we look at this this function must be identity
which means that it\textsf{'}s a free CZ match something in this match should
be just identity cases cases like this indeed they are identity the
rap case is just rap which is defined like this that\textsf{'}s the definition
of herbs and wrap so oops requires us to execute a recursive call
to the run and so we can use the induction assumption but run of ramp
is identity and that assumption can be applied to recursive calls
of the function that we are proving more for so therefore this is
identity so that\textsf{'}s a map on identity well that\textsf{'}s just observe C F
which is equal to that so that is again identity the second law is
a little more involved it\textsf{'}s a natural T now I could say well much
reality is obvious because our code doesn't look at an edit type but
we can prove this informally so again we look at two sides of this
equation the functions of this type so we apply both sides to an arbitrary
value of that type when we just substitute and compute the left-hand
side which is going to be this and run of that is going to be that
then we compute the right-hand side so we already see that the wrap
case is the same and water is what remains is to demonstrate that
the ops case is the same you know there is a slight difference between
these two expressions and the difference can be resolved because we
can use the induction assumption for this so we are proving the law
of this kind and we can use this law for the recursive call here so
if we do that we can assume this is true and then that\textsf{'}s exactly what
remains to be demonstrated that we can do the map so this is a functor
whose map for United we're using so we can use a composition law for
the function and that\textsf{'}s exactly the expression that we have here what
remains not equal that\textsf{'}s the composition of two maps so we simplify
that we get this so the universal property is slightly more involved
yet but it is proved in a similar way so both sides are functions
of this type so we apply both sides to an arbitrary value of type
faces E or compute the left hand side will contain the right hand
side and we use the typeclass preserving property which is this equation
or it will only written in a scholar corner to space equation so when
we have F of ops P we replace that with upscale of something so oops
P and obscure are assumed to be available now these are typeclass
evidence for P and Q and simplifying the code we put the code until
identical shape so that leaves us home proof from the laws now what
she would call that I have here that code is for the next slide so
another general thing we can do the free typeclass is that we can
combine different generating instructors so far we have been only
looking at a free instance of a typeclass C generated by a single
type Z but we can also consider several constructors at the same time
several generating types of constructors and this would correspond
in a monadic DSL that we have different sets of operations that are
defined separately we would like to combine them and recall that monads
did not compose in general so it is in general not easy to compose
different sets of operations but it is actually easy in the free typeclass
because all you need to do is to take a disjunction since the definition
of the free typeclass is this if you have several generating types
all you need to do is to have several parts of the disjunction here
Z 1 plus Z 2 plus Z 3 and so on which means that the free instance
of C over several constructor is the same as the free instance of
C over the disjunction of these constructors so it is sufficient to
take the disjunction and generate the free type cause using a disjunction
now the only inconvenience is that you would need to inject parts
into the disjunction that it can become cumbersome I will show called
in a minute and the church encoding actually makes it easier to manage
this situation so the reason is that the church encoding for an inductive
typeclass it looks like yes if you have several constructors and it
will be like this you can take the junction and transform it into
this sort of expression now you can do the trick of type questions
so you can encode each of these as extractor typeclass and then the
church encoding would look like this and you can even simplify it
further by saying that even this is a typeclass constraint on X which
means that X must be of type C and so the church encoding would look
like this so in this in this form it is an easier code to maintain
at the same time we find that this definition actually works for any
number of generators and for any typeclass C it\textsf{'}s a general formula
for Church encoded free instance of class typeclass C and generated
by any number of given types so in this very concise form it\textsf{'}s very
easy to implement as well let\textsf{'}s see how this works in order to test
this I have three type constructors that are not functions in other
words unfactored and they describe different kinds of operations in
some kind of imaginary business project the first one factor adds
a name care database and returns a database ID and it gets named by
second and funky logs a message returns unit a third on factor creates
a new ID so these are just for example and I defined generic transformations
from each of these unfactored to the option factor these are just
defined in arbitrary ways for testing purposes only this is not useful
for any kind of application this kind of transformation which are
none a lot nicer just testin I testing that all the types fit together
so the first way of combining the three operation constructors I'm
going to put them into a three factor and as we have seen all I need
to do is I need to take a disjunction of all these constructors so
let me define this unfold as a disjunction of the three type constructors
now I also need to define a generic transformation from the new function
to option let me put the syntax a little easier like this now notice
that I need to write this column in this code this code is pure boilerplate
but it depends on how I defined my unfunny here if I add another one
then this entire core don't have to be reversed and this is kinda
this is a bit of a burden now defining a free function in the reduced
encoding is straightforward in order to use it I need to lift values
from each of the unfactored into the free function so I define these
lifting functions and again I have boilerplate that depends on how
the order of disjunction is chosen now this is an example computation
where I require this type annotation in order to lift I could have
called these functions directly : over keys I could have called them
directly where I can do this but I have to do it so let\textsf{'}s see how
the same works in the church encoding so first I define an extractor
typeclasses and then I define the church encoding this is the entire
church including it has a type parameter G and then there are three
type constraints for each of the unfactored extractors and then a
function type constraint and I've done this is the entire encoding
the furniture instance is trivial but it is not stock safe unfortunately
because we know that the church including of the three encoding over
three frontier is not stack safe now the boilerplate for lifting does
not depend on the order of the heart functions there\textsf{'}s no notion of
order anywhere if I need the first one I can just add it here here
here and here there\textsf{'}s no possibility of making a mistake there and
finally I run the computation the extractors are need to declare and
see there is the run is very simple I don't need an extra brother
I have less code {[}Music{]} another important thing that we can do
is combining different trade type process so suppose you have two
different hypotheses C 1 and C 2 and you want to combine them several
ways of doing that and one is to use factor composition so for example
I can do 3 C 1 over a 3 C 2 over Z I can do that there are disadvantages
in doing this one big disadvantage is that the order of composition
actually matters in terms of what semantics I get if these type assets
return effects and the effects are combined and nested in this order
and not in the opposite order so I would not be able to encode this
order of nesting I must encode always miss order of nesting and of
course all the operations that I want to execute in situ need to be
lifted through the factor C 1 I can do it because it\textsf{'}s a factor as
we have seen and this only works for inductive typeclasses of course
well that is not a big limitation perhaps since most of our typeclasses
are inductive not all for example traversable is not inductive so
there is no free traversable that can be encoded in this way but these
are significant disadvantages most importantly we are not free to
encode arbitrary nesting of effects the second option is to use the
disjunction of method factors and then you build the free typeclass
instance using this C in other words you make a new typeclass that
has all the methods of the previous hypothesis disjunction of method
functors is equivalent to conjunction of the f-typeclass evidence
families because type costs evidence values are of type function CX
to X and so this disjunction is equivalent to taking a conjunction
of those functions so that\textsf{'}s the same as building one big type ones
with all the methods put together now this is of course not ideal
because you would have to change the code if you wanted to combine
different typeclasses but church encoding can give you this combination
for free because all you need to do is to write this kind of type
and then obviously this was just a product of c1 c2 so you have this
formula where you can put different typeclasses here and different
in your use here and so the church encoding automatically gives you
a way to have a way of having a free typeclass instance for any combination
of typeclasses and any combination of generators it\textsf{'}s a very powerful
mechanism let\textsf{'}s consider just two examples one is for curiosity what
if we combine the filter in the control function we would have a typeclass
that is at once a function and the control factor and that\textsf{'}s possible
we can do it the question is what do you do with this class because
this class is going to be free encoding so you need to interpret it
into some specific class typeclass so you could probably interpret
it in a pro factor although I don't know what is the use case for
this a better example is to combine munna and applicative this is
actually useful in practice because ma not an applicative encode slightly
different kind of effects monad encodes sequential effects look at
them encodes effects that are partially parallel so not necessarily
completely independent but the effect parts can be run in Kerala so
computation with applicative factor can be paralyzed very easily but
computations with monads cannot be paralyzed because they are sequentially
the next step depends on the value of the previous step and the effect
also depends on the value of the previous step so you cannot start
running the effect before you know the value but with a quick edit
of you can now ordinarily a monad also has an applicative instance
however in Sun walnuts in other words you can implement zip if you
have flat map but in some walnuts there is a non-standard implementation
of zip that has specific advantages in other words an implementation
of zip that is not equal to that which you get out of a flat map and
you do that for specific reasons either for performance so it could
it could give you the same results but it will run effects differently
and will have different performance for example imagine that you have
a future monad and the zip can be implemented as parallel execution
of futures and flatmap is implemented as sequential execution of futures
so if you combined monad and duplicative in the freetypeclass and
interpreted that pi plus into a future such that the monadic methods
the flat map are converted into flat map in the future but applicative
methods are converted into parallel execution of futures using a special
code then you get an advantage because each zip or create fellow branches
in each flat mark sequential branches so let\textsf{'}s see how that works
in some example code so here\textsf{'}s the encoding of the free monad and
applicative at the same time so I put them into one typeclass for
simplicity I could have done it differently you could have done it
using the church including the real is right before but this is more
interesting because you can see how you would run a combined frame
on a duplicative over a type constructor G that has both applicative
and one on instances the way you would do that is that all you would
translate of course wrapping to wrap here into pure flatmap you would
translate like this you would run and then you run this you so you
run the flat map in the target unit but the applicative you know that
you translate like this you run in parallel and then you combine them
using an applicative method in the model so you see this run is not
in peril because this is inside the function which is this function
which will be called only inside the flat map after the first effect
is finished but these two effects are going to be run in parallel
so if I implement the interpreter like this and I will automatically
translate all the applicative operations into parallel executions
potentially parallel executions let\textsf{'}s see for example for future I
just translate the same into futures and this is actually going to
be pair execution of futures because we know what\textsf{'}s calendars once
you have a future it already is scheduled and run I give it an execution
context and it\textsf{'}s already run so in this way I can have so I don't
have the typeclass instances for functor monad it\textsf{'}s more or less boilerplate
I just do a little bit of simplification it\textsf{'}s not important most so
let me just skip this and then here is my DSL I have some operations
I make a free monad out of it which is at the same time a free applicative
and then I run it through the future so I can run my code melodically
so this is nomadic code the lift is a just method that I defined to
make it quicker to lift things it\textsf{'}s a wrapper so basically I generate
an ID when I generate three new IDs and I validate them in parallel
and I wait until all of this validation is done and I close the session
so this is a combination of melodic and wicked methods at the same
time the interpreter is just some translating these into specific
business logic that I imagined and this is all for a less boilerplate
so basically when I run this computations computations of course just
a pure value doesn't do anything it\textsf{'}s a pure data structure that describes
what needs to be done but when I run this computation and automatically
all these zipped parts become executed in parallel but all the mimetic
parts are executed sequentially and I can I don't need to worry about
the order of these things I can nest them in any way want for example
I could define first validate using monadic the line I combine different
validates using applicative and I again put the results into a monadic
context it\textsf{'}s fine it\textsf{'}s all working it all works in arbitrary order
this is what I was indicating here and this is an advantage over this
method the frankly composition where I would have to choose whether
I have moon add outside applicative inside or more not inside applicative
outside and whatever I choose I can only then encode one of those
combinations so this concludes this chapter and here are some exercises
for making your dream code introduced in coding and working with inductive
typeclasses implement an idea cells implementing the church encoding
simplifying quantified types and deriving a reduced encoding color
of a tree encoding for different cases this concludes chapter tune 
\end{comment}

\include{sofp-transformers}

\part{Discussions}

\include{sofp-summary}

\include{sofp-essays}

\part{Appendixes}


\appendix

\appendix

\chapter{Notations\label{chap:Appendix-Notations}}

\global\long\def\gunderline#1{\mathunderline{greenunder}{#1}}%
\global\long\def\bef{\forwardcompose}%
\global\long\def\bbnum#1{\custombb{#1}}%
\global\long\def\pplus{{\displaystyle }{+\negmedspace+}}%
This book chooses certain notations differently from what the functional
programming community currently uses. The proposed notation is well
adapted to reasoning about types and code, and especially for designing
data types and proving the laws of various typeclasses.

\section{Summary of notations}
\begin{description}
\item [{$A$}] \textemdash{} type parameter. Names of type parameters are
always capitalized.
\item [{$F^{A}$}] \textemdash{} type constructor $F$ with type argument
$A$. In Scala, \lstinline!F[A]!
\item [{$F^{\bullet}$}] \textemdash{} the type constructor $F$ understood
as a type-level function. In Scala, \lstinline!F[_]! 
\item [{$x^{:A}$}] \textemdash{} value $x$ has type $A$; in Scala, \lstinline!x:A!.
Value names are always in lowercase.
\item [{$\bbnum 1,\,1$}] \textemdash{} the unit type and its value; in
Scala, \lstinline!Unit! and \lstinline!()!
\item [{$\bbnum 0$}] \textemdash{} the void type. In Scala, \lstinline!Nothing!
\item [{$A+B$}] \textemdash{} a disjunctive type (co-product). In Scala,
this type is \lstinline!Either[A, B]! 
\item [{$x^{:A}+\bbnum 0^{:B}$}] \textemdash{} a value of a disjunctive
type $A+B$. In Scala, \lstinline!Left(x)!
\item [{$A\times B$}] \textemdash{} a product (tuple) type. In Scala,
this type is \lstinline!(A,B)!
\item [{$a^{:A}\times b^{:B}$}] value of a tuple type $A\times B$. In
Scala, \lstinline!(a, b)!
\item [{$A\rightarrow B$}] \textemdash{} the function type, mapping from
$A$ to $B$
\item [{$x^{:A}\rightarrow f$}] \textemdash{} a nameless function (as
a value). In Scala, \lstinline!{ x:A => f }!
\item [{$\text{id}$}] \textemdash{} an identity function; in Scala, \lstinline!identity[A]!
\item [{$\triangleq$}] \textemdash{} \textsf{``}is defined to be\textsf{''} or \textsf{``}is equal
by definition\textsf{''}
\item [{$\overset{!}{=}$}] \textemdash{} \textsf{``}must be equal according to
what we know\textsf{''}
\item [{$\overset{?}{=}$}] \textemdash{} \textsf{``}we ask \textemdash{} is it
equal? \textemdash{} because we still need to prove that\textsf{''}
\item [{$\square$}] \textemdash{} \textsf{``}this proof or this derivation or
this definition or this example is finished\textsf{''}
\item [{$\cong$}] \textemdash{} for types, a natural isomorphism between
types; for values, \textsf{``}equivalent\textsf{''} values according to an already
established isomorphism
\item [{$A^{:F^{B}}$}] \textemdash{} special type annotation, used for
defining unfunctors (GADTs)
\item [{$\wedge$}] \textemdash{} logical conjunction; $\alpha\wedge\beta$
means \textsf{``}both $\alpha$ and $\beta$ are true\textsf{''}
\item [{$\vee$}] \textemdash{} logical disjunction; $\alpha\vee\beta$
means \textsf{``}either $\alpha$ or $\beta$ or both are true\textsf{''}
\item [{$\Rightarrow$}] \textemdash{} logical implication; $\alpha\Rightarrow\beta$
means \textsf{``}if $\alpha$ is true then $\beta$ is true\textsf{''}
\item [{$\text{fmap}_{F}$}] \textemdash{} the standard method \lstinline!fmap!
of a functor $F$. In Scala, \lstinline!Functor[F].fmap!
\item [{$\text{flm}_{F},\text{ftn}_{F},\text{pu}_{F}$}] \textemdash{}
the standard methods \lstinline!flatMap!, \lstinline!flatten!, and
\lstinline!pure! of a monad $F$
\item [{$F^{\bullet}$}] \textemdash{} the type constructor $F$ understood
as a type-level function. In Scala, \lstinline!F[_]! 
\item [{$F^{\bullet}\leadsto G^{\bullet}$}] \textemdash{} or $F\leadsto G$
a natural transformation between functors $F$ and $G$. In Scala,
\lstinline!F ~> G!
\item [{$\forall A.\,P^{A}$}] \textemdash{} a universally quantified type
expression. In Scala 3, \lstinline![A] => P[A]!
\item [{$\exists A.\,P^{A}$}] \textemdash{} an existentially quantified
type expression. In Scala, \lstinline!{ type A; val x: P[A] }! 
\item [{$f\bef g$}] \textemdash{} the forward composition of functions:
$f\bef g$ is $x\rightarrow g(f(x))$. In Scala, \lstinline!f andThen g!
\item [{$f\circ g$}] \textemdash{} the backward composition of functions:
$f\circ g$ is $x\rightarrow f(g(x))$. In Scala, \lstinline!f compose g!
\item [{$F\circ G$}] \textemdash{} the backward composition of type constructors:
$F\circ G$ is $F^{G^{\bullet}}$. In Scala, \lstinline!F[G[A]]! 
\item [{$\triangleright$}] \textemdash{} use a value as the argument of
a function: $x\triangleright f$ is $f(x)$. In Scala, \lstinline!x.pipe(f)!
\item [{$f^{\uparrow G}$}] \textemdash{} a function $f$ lifted to a functor
$G$; same as $\text{fmap}_{G}(f)$
\item [{$f^{\uparrow G\uparrow H}$}] \textemdash{} a function lifted first
to $G$ and then to $H$. In Scala, \lstinline!h.map(_.map(f))! 
\item [{$f^{\downarrow H}$}] \textemdash{} a function $f$ lifted to a
contrafunctor $H$ 
\item [{$\diamond_{M}$}] \textemdash{} the Kleisli product operation for
the monad $M$
\item [{$L\varangle M$}] or equivalently $T_{L}^{M}$ \textemdash{} the
monad $L$\textsf{'}s transformer applied to a monad $M$
\item [{$\oplus$}] \textemdash{} the binary operation of a monoid. In
Scala, \lstinline!x |+| y!
\item [{$\Delta$}] \textemdash{} the \textsf{``}diagonal\textsf{''} function of type $\forall A.\,A\rightarrow A\times A$
\item [{$\pi_{1},\pi_{2},...$}] \textemdash{} the projections from a tuple
to its first, second, ..., parts
\item [{$\boxtimes$}] \textemdash{} pair product of functions: $(f\boxtimes g)(a\times b)\triangleq f(a)\times g(b)$
\item [{$\boxplus$}] \textemdash{} pair co-product of functions
\item [{$\ogreaterthan$}] \textemdash{} pair mapper of relations
\item [{$\left[a,b,c\right]$}] \textemdash{} an ordered sequence of values.
In Scala, \lstinline!Seq(a, b, c)!
\item [{$\begin{array}{||cc|}
x\rightarrow x & \bbnum 0\\
\bbnum 0 & a\rightarrow a\times a
\end{array}$}] ~ \textemdash{} a function that works with disjunctive types
(a \textsf{``}\index{disjunctive functions}disjunctive function\textsf{''})
\end{description}

\section{Detailed explanations}

$F^{A}$ means a type constructor $F$ with a type parameter $A$.
In Scala, this is \lstinline!F[A]!. Type parameters are uppercase
($A$, $B$, ...). Type constructors with multiple type parameters
are denoted by $F^{A,B,C}$. Nested type constructors such as Scala\textsf{'}s
\lstinline!F[G[A]]! are denoted by $F^{G^{A}}$, meaning $F^{(G^{A})}$.

$x^{:A}$ means a value $x$ that has type $A$; this is a \textbf{\index{type annotation}type
annotation}. In Scala, a type annotation is \lstinline!x:A!. The
colon symbol, $:$, in the superscript shows that $A$ is not a type
argument (as it would be in a type constructor, $F^{A}$). A less
concise notation for $x^{:A}$ is $x:A$.

$\bbnum 1$ means the unit type\index{unit type}, and $1$ means
the value of the unit type. In Scala, the unit type is \lstinline!Unit!,
and its value is \lstinline!()!. An example of this notation is $\bbnum 1+A$,
which corresponds to \lstinline!Option[A]! in Scala.

$\bbnum 0$ means the void\index{void type} type (the type with no
values). In Scala, this is the type \lstinline!Nothing!. The notation
$\bbnum 0$ is often used to denote an empty part of disjunctive types
or values. E.g., the disjunctive type \lstinline!Option[A]! has two
parts: \lstinline!Some[A]! and \lstinline!None!. These types are
denoted by $\bbnum 0+A$ and $\bbnum 1+\bbnum 0$ respectively. Similarly,
$A+\bbnum 0$ denotes the first part of the type $A+B$ (in Scala,
\lstinline!Left[A]!), while $\bbnum 0+B$ denotes its second part
(in Scala, \lstinline!Right[A]!). Values of disjunctive types are
denoted similarly. E.g., $x^{:A}+\bbnum 0^{:B}$ denotes a value of
the left part of the type $A+B$. In Scala, this value is written
with fully annotated types as \lstinline!Left[A,B](x)!.

$A+B$ means the disjunctive type made from types $A$ and $B$. In
Scala, this is the type \texttt{}\lstinline!Either[A, B]!.

$x^{:A}+\bbnum 0^{:B}$ denotes a value of a disjunctive type $A+B$,
where $x$ is the value of type $A$, which is the chosen case, and
$\bbnum 0$ stands for other possible cases. For example, $x^{:A}+\bbnum 0^{B}$
is \lstinline!Left[A,B](x)! in Scala. Type annotations $^{:A}$ and
$^{:B}$ may be omitted if the types are unambiguous from the context.

$A\times B$ means the product type made from types $A$ and $B$.
In Scala, this is the tuple type \lstinline!(A,B)!.

$a^{:A}\times b^{:B}$ means a value of a tuple type $A\times B$;
in Scala, this is the tuple value \lstinline!(a, b)!. Type annotations
$^{:A}$ and $^{:B}$ may be omitted if the types are unambiguous
from the context.

$A\rightarrow B$ means the type of functions $A$ to $B$. In Scala,
this is the type \lstinline!A => B!. The function type\textsf{'}s arrow binds
weaker than $+$, which binds weaker than $\times$. So, $A+B\rightarrow C\times D$
means $(A+B)\rightarrow(C\times D)$.

$x^{:A}\rightarrow y$ means a nameless function with argument $x$
of type $A$ and function body $y$. (Usually, the body $y$ will
be an expression that uses $x$. In Scala, this is \lstinline!{ x: A => y }!.
Type annotation $^{:A}$ may be omitted if the type is unambiguous
from the context.

$\text{id}$ means the identity function. The type of its argument
should be either specified as $\text{id}^{A}$ or $\text{id}^{:A\rightarrow A}$,
or else should be unambiguous from the context. In Scala,  \lstinline!identity[A]!
corresponds to $\text{id}^{A}$.

$\triangleq$ means \textsf{``}equal by definition\textsf{''}. A definition of a function
$f$ is written as $f\triangleq(x^{:\text{Int}}\rightarrow x+10)$;
in Scala, this is \lstinline!val f = { x: Int => x + 10 }!. A definition
of a type constructor $F$ is written as $F^{A}\triangleq\bbnum 1+A$;
in Scala, this is \lstinline!type F[A] = Option[A]!.

$\cong$ for types means an equivalence (an isomorphism) of types.
For example, $A+A\times B\cong A\times\left(\bbnum 1+B\right)$. The
same symbol $\cong$ for \emph{values} means \textsf{``}equivalent\textsf{''} according
to an equivalence relation that needs to be established in the text.
For example, if we have established an equivalence that allows nested
tuples to be reordered whenever needed, we can write $\left(a\times b\right)\times c\cong a\times\left(b\times c\right)$,
meaning that these values are mapped to each other by the established
isomorphism functions. 

$A^{:F^{B}}$ in type definitions means that the definition assigns
the type $F^{B}$ to the type expression $A$. This notation is used
for defining unfunctors (GADTs). For example, the Scala code:

\begin{lstlisting}
sealed trait F[A]
case class F1() extends F[Int]
case class F2[A](a: A) extends F[(A, String)]
\end{lstlisting}
defines an unfunctor\index{unfunctor} denoted by $F^{A}\triangleq\bbnum 1^{:F^{\text{Int}}}+A^{:F^{A\times\text{String}}}$.

$\wedge$ (conjunction), $\vee$ (disjunction), and $\Rightarrow$
(implication) are used in formulas of Boolean as well as constructive
logic in Chapter~\ref{chap:5-Curry-Howard}, e.g., $\alpha\wedge\beta$,
where Greek letters stand for logical propositions.

$\text{fmap}_{F}$ is a functor $F$\textsf{'}s the standard method \lstinline!fmap!
of the \lstinline!Functor! typeclass. In Scala, this may be written
as \texttt{}\lstinline!Functor[F].fmap!. Since each functor $F$
has its own specific implementation of $\text{fmap}_{F}$, the subscript
\textsf{``}$F$\textsf{''} is \emph{not} a type parameter of $\text{fmap}_{F}$. The
function $\text{fmap}_{F}$ has two type parameters, which may be
written as $\text{fmap}_{F}^{A,B}$, and we may write its type signature
as $\text{fmap}_{F}^{A,B}:\left(A\rightarrow B\right)\rightarrow F^{A}\rightarrow F^{B}$.
In most cases, the type parameters $A$, $B$ can be omitted without
loss of clarity.

$\text{pu}_{F}$ denotes a monad $F$\textsf{'}s method \lstinline!pure!.
This function has type signature $A\rightarrow F^{A}$ and has a type
parameter $A$. In the code notation, the type parameter may be either
omitted or denoted as $\text{pu}_{F}^{A}$. If we are using \lstinline!pure!
with a complicated type, e.g., $\bbnum 1+P^{B}$, as of the type parameter
$A$, we may write the type parameter for clarity as $\text{pu}_{F}^{\bbnum 1+P^{B}}$.
The type signature of that function then becomes: 
\[
\text{pu}_{F}^{1+P^{B}}:\bbnum 1+P^{B}\rightarrow F^{\bbnum 1+P^{B}}\quad.
\]
But in most cases we will not need to write out the type parameters.

$\text{flm}_{F}$ denotes the curried version of a monad $F$\textsf{'}s method
\lstinline!flatMap!. The type signature of $\text{flm}_{F}$ is $\text{flm}_{F}:(A\rightarrow F^{B})\rightarrow F^{A}\rightarrow F^{B}$.
Note that Scala\textsf{'}s standard \lstinline!flatMap! type signature is
not curried. The curried method $\text{flm}_{F}$ is easier to use
in calculations involving the monad laws.

$\text{ftn}_{F}$ denotes a monad $F$\textsf{'}s method \lstinline!flatten!
with the type signature $\text{ftn}_{F}:F^{F^{A}}\rightarrow F^{A}$.

$F^{\bullet}$ means the type constructor $F$ understood as a type-level
function, \textemdash{} that is, with a type parameter unspecified.
In Scala, this is \lstinline!F[_]!. The bullet symbol, $\bullet$,
is used as a placeholder for the missing type parameter. When no type
parameter is needed, $F$ means the same as $F^{\bullet}$. (For example,
\textsf{``}a functor $F$\textsf{''} and \textsf{``}a functor $F^{\bullet}$\textsf{''} mean the same
thing.) However, it is useful for clarity to be able to indicate the
place where the type parameter would appear. For instance, functor
composition is denoted as $F^{G^{\bullet}}$; in Scala 2, this is
\texttt{}\lstinline!Lambda[X => F[G[X]]]! when using the \textsf{``}kind
projector\textsf{''}\index{kind@\textsf{``}kind projector\textsf{''} plugin} plugin.\footnote{\texttt{\href{https://github.com/typelevel/kind-projector}{https://github.com/typelevel/kind-projector}}}
When the type parameter $B$ of a bifunctor $P^{A,B}$ is fixed to
$Z$, we get a functor (with respect to $A$) denoted by $P^{\bullet,Z}$.
Another example: $T_{L}^{M,\bullet}$ denotes a monad transformer
for the base monad $L$ and the foreign monad $M$. The foreign monad
$M$ is a type parameter in $T_{L}^{M,\bullet}$. The symbol $\bullet$
stands for the transformer\textsf{'}s second type parameter. (The base monad
$L$ is not a type parameter in $T_{L}^{M,\bullet}$ because the construction
of the monad transformer depends on the internal details of $L$.)

$F^{\bullet}\leadsto G^{\bullet}$ or $F\leadsto G$ means a natural
transformation between two functors $F$ and $G$. In some Scala libraries,
this is denoted by \lstinline!F ~> G!.

$\forall A.\,P^{A}$ is a universally quantified type expression,
in which $A$ is a bound type parameter.

$\exists A.\,P^{A}$ is an existentially quantified type expression,
in which $A$ is a bound type parameter.

$\bef$ means the forward composition\index{forward composition}
of functions: $f\bef g$ (reads \textsf{``}$f$ before $g$\textsf{''}) is the function
defined as $x\rightarrow g(f(x))$.

$\circ$ means the backward composition\index{backward composition}
of functions: $f\circ g$ (reads \textsf{``}$f$ after $g$\textsf{''}) is the function
defined as $x\rightarrow f(g(x))$.

$\circ$ with type constructors means their (backward) composition,
for example $F\circ G$ denotes the type constructor $F^{G^{\bullet}}$.
In Scala, this is \lstinline!F[G[A]]!. 

$x\triangleright f$ (the \textbf{pipe notation})\index{pipe notation}\index{\$@$\triangleright$-notation!see \textsf{``}pipe notation\textsf{''}}
is a different syntax for $f(x)$. The value $x$ is passed as the
argument to the function $f$. In Scala, the expression $x\triangleright f$
is written as \lstinline!x.pipe(f)! or, if \lstinline!f! is a method,
\lstinline!x.f!. This syntax is used with many standard methods such
as \lstinline!size! or \lstinline!toSeq!. Because the argument $x$
is to the left of the function $f$ in this notation, forward compositions
of functions such as $x\triangleright f\triangleright g$ are naturally
grouped to the left as it is done in Scala code, for example \lstinline!x.toSeq.sorted!.
The operation $\triangleright$ (pronounced \textsf{``}pipe\textsf{''}) groups weaker
than the forward composition ($\bef$), and so we have $x\triangleright f\bef g=x\triangleright f\triangleright g$
in this notation. Reasoning about code in the pipe notation uses the
identities:
\begin{align*}
x\triangleright f=f(x)\quad,\quad\quad & \left(x\triangleright f\right)\triangleright g=x\triangleright f\triangleright g\quad,\\
x\triangleright f\bef g=x\triangleright\left(f\bef g\right)\quad,\quad\quad & x\triangleright f\triangleright g=x\triangleright f\bef g\quad.
\end{align*}
The pipe symbol groups stronger than the function arrow, so $x\rightarrow y\triangleright f$
is the same as $x\rightarrow(y\triangleright f)$. Here are some examples
of reasoning with functions in the pipe notation:
\begin{align*}
 & \left(a\rightarrow a\triangleright f\right)=\left(a\rightarrow f(a)\right)=f\quad,\\
 & f\triangleright\left(y\rightarrow a\triangleright y\right)=a\triangleright f=f(a)\quad,\\
 & f(y(x))=x\triangleright y\triangleright f\neq x\triangleright\left(y\triangleright f\right)=f(y)(x)\quad.
\end{align*}
The correspondence between the forward composition and the backward
composition:
\begin{align*}
 & f\bef g=g\circ f\quad,\\
 & x\triangleright(f\bef g)=x\triangleright f\bef g=x\triangleright f\triangleright g=g(f(x))=(g\circ f)(x)\quad.
\end{align*}

$f^{\uparrow G}$ means a function $f$ lifted to a functor $G$.
For a function $f^{:A\rightarrow B}$, the application of $f^{\uparrow G}$
to a value $g^{:G^{A}}$ is written as $f^{\uparrow G}(g)$ or as
$g\triangleright f^{\uparrow G}$. In Scala, this is \lstinline!g.map(f)!.
Nested lifting (i.e., lifting to the functor composition $H\circ G$)
can be written as $f^{\uparrow G\uparrow H}$, which means $\left(f^{\uparrow G}\right)^{\uparrow H}$,
and produces a function of type $H^{G^{A}}\rightarrow H^{G^{B}}$.
Applying a nested lifting to a value $h$ of type $H^{G^{A}}$ is
written as $h\triangleright f^{\uparrow G\uparrow H}$. In Scala,
this is \lstinline!h.map(_.map(f))!. The functor composition law
is written as:
\[
p^{\uparrow G}\bef q^{\uparrow G}=\left(p\bef q\right)^{\uparrow G}\quad.
\]
The notation $x\triangleright p^{\uparrow G}\triangleright q^{\uparrow G}$
is intended to be similar to the Scala code \lstinline!x.map(p).map(q)!.

$f^{\downarrow H}$ means a function $f$ lifted to a contrafunctor
$H$. For a function $f^{:A\rightarrow B}$, the application of $f^{\downarrow H}$
to a value $h:H^{B}$ is written as $h\triangleright f^{\downarrow H}$
and yields a value of type $H^{A}$. In Scala, this may be written
as \lstinline!h.contramap(f)!. Nested lifting is denoted as, e.g.,
$f^{\downarrow H\uparrow G}\triangleq(f^{\downarrow H})^{\uparrow G}$.

$\diamond_{M}$ means the Kleisli product operation for a given monad
$M$. This is a binary operation working on two Kleisli functions
of types $A\rightarrow M^{B}$ and $B\rightarrow M^{C}$ and yields
a new function of type $A\rightarrow M^{C}$.

$L\varangle M$ denotes the monad $L$\textsf{'}s transformer applied to a
foreign monad $M$. We define $(K\varangle L)\varangle M\triangleq K\varangle(L\varangle M)$,
which makes the monad transformer application into an associative
operation.

$\oplus$ means the binary operation of a monoid, e.g., $x\oplus y$.
For this expression to make sense, a specific monoid type should be
defined . In Scala libraries, $x\oplus y$ is often denoted as \lstinline!x |+| y!.

$\Delta$ means the standard \textsf{``}diagonal\textsf{''} function of type $\forall A.\,A\rightarrow A\times A$,
i.e., $\Delta\triangleq a^{:A}\rightarrow a\times a$. In Scala:
\begin{lstlisting}
def delta[A](a: A): (A, A) = (a, a)
\end{lstlisting}

$\pi_{1},\pi_{2},...$ denote the functions extracting the first,
second, ..., parts in a tuple. In Scala, $\pi_{1}$ is \lstinline!(_._1)!.

$\boxtimes$ means the pair product\index{pair product of functions}
of functions, defined by $(f\boxtimes g)(a\times b)=f(a)\times g(b)$.
In Scala, the pair product can be implemented as a higher-order function:
\begin{lstlisting}
def pair_product[A,B,P,Q](f: A => P, g: B => Q): ((A, B)) => (P, Q) = {
  case (a, b) => (f(a), g(b))
}
\end{lstlisting}
The operations $\Delta$, $\pi_{i}$ (where $i=1,2,...$), and $\boxtimes$
allow us to express any function operating on tuples. Useful properties
for reasoning about code of such functions: 
\begin{align*}
{\color{greenunder}\text{identity law}:}\quad & \Delta\bef\pi_{i}=\text{id}\quad,\\
{\color{greenunder}\text{naturality law}:}\quad & f\bef\Delta=\Delta\bef(f\boxtimes f)\quad,\\
{\color{greenunder}\text{left and right projection laws}:}\quad & (f\boxtimes g)\bef\pi_{1}=\pi_{1}\bef f\quad,\quad\quad(f\boxtimes g)\bef\pi_{2}=\pi_{2}\bef g\quad,\\
{\color{greenunder}\text{composition law}:}\quad & (f\boxtimes g)\bef(p\boxtimes q)=(f\bef p)\boxtimes(g\bef q)\quad,
\end{align*}
as well as the functor lifting laws for $\Delta$ and $\pi_{i}$:
\begin{align*}
 & f^{\uparrow F}\bef\Delta=\Delta\bef f^{\uparrow(F\times F)}=\Delta\bef(f^{\uparrow F}\boxtimes f^{\uparrow F})\quad,\\
 & (f^{\uparrow F}\boxtimes f^{\uparrow G})\bef\pi_{1}=f^{\uparrow(F\times G)}\bef\pi_{1}=\pi_{1}\bef f^{\uparrow F}\quad.
\end{align*}

$\left[a,b,c\right]$ means an ordered sequence of values, such as
a list or an array. In Scala, this can be \lstinline!List(a, b, c)!,
\lstinline!Vector(a, b, c)!, \lstinline!Array(a, b, c)!, or another
collection type.

$f^{:Z+A\rightarrow Z+A\times A}\triangleq\,\begin{array}{||cc|}
z\rightarrow z & \bbnum 0\\
\bbnum 0 & a\rightarrow a\times a
\end{array}\,\,$ is the \textbf{matrix notation}\index{matrix notation} for a function
whose input and/or output type is a disjunctive type (\index{disjunctive functions}a
\textbf{disjunctive function}). In Scala, the function $f$ is written
as:
\begin{lstlisting}
def f[Z, A]: Either[Z, A] => Either[Z, (A, A)] = {
  case Left(z)   => Left(z)       // Identity function on Z.
  case Right(a)  => Right((a, a)) // Delta on A.
}
\end{lstlisting}
The rows of the matrix indicate the different \lstinline!case!s in
the function\textsf{'}s code, corresponding to the different parts of the input
disjunctive type. If the input type is not disjunctive, there will
be only one row. The columns of the matrix indicate the parts of the
output disjunctive type. If the output type is not disjunctive, there
will be only one column.

A matrix may show all parts of the disjunctive types in separate \textsf{``}type
row\textsf{''} and \textsf{``}type column\textsf{''}:
\begin{equation}
f^{:Z+A\rightarrow Z+A\times A}\triangleq\,\begin{array}{|c||cc|}
 & Z & A\times A\\
\hline Z & \text{id} & \bbnum 0\\
A & \bbnum 0 & a\rightarrow a\times a
\end{array}\quad.
\end{equation}
This notation clearly indicates the input and the output types of
the function and is useful for reasoning about the code. The vertical
double line separates the \emph{input} types from the function code
(output types have a single line). In the code above, the \textsf{``}type
column\textsf{''} shows the parts of the input disjunctive type $Z+A$. The
\textsf{``}type row\textsf{''} shows the parts of the output disjunctive type $Z+A\times A$.

The matrix notation is adapted to \emph{forward} function composition
($f\bef g$). Assume that $A$ is a monoid type, and consider the
composition of the function $f$ shown above and the function $g$
defined as:
\begin{lstlisting}
def g[Z, A: Monoid]: Either[Z, (A, A)] => A = {
  case Left(_)          => Monoid[A].empty
  case Right((a1, a2))  => a1 |+| a2
}
\end{lstlisting}
In the matrix notation, the function $g$ is written (with and without
types) as:
\[
g\triangleq\,\begin{array}{|c||c|}
 & A\\
\hline Z & \_\rightarrow e^{:A}\\
A\times A & a_{1}\times a_{2}\rightarrow a_{1}\oplus a_{2}
\end{array}\quad,\quad\quad g\triangleq\,\begin{array}{||c|}
\_\rightarrow e^{:A}\\
a_{1}\times a_{2}\rightarrow a_{1}\oplus a_{2}
\end{array}\quad.
\]
The forward composition $f\bef g$ is computed by forward-composing
the matrix elements using the rules of the ordinary matrix multiplication,
omitting any terms containing $\bbnum 0$:
\begin{align*}
f\bef g & =\,\begin{array}{||cc|}
\text{id} & \bbnum 0\\
\bbnum 0 & a\rightarrow a\times a
\end{array}\,\bef\,\begin{array}{||c|}
\_\rightarrow e^{:A}\\
a_{1}\times a_{2}\rightarrow a_{1}\oplus a_{2}
\end{array}\\
 & =\,\,\begin{array}{||c|}
\text{id}\bef(\_\rightarrow e^{:A})\\
\left(a\rightarrow a\times a\right)\bef\left(a_{1}\times a_{2}\rightarrow a_{1}\oplus a_{2}\right)
\end{array}\,=\,\begin{array}{||c|}
\_\rightarrow e^{:A}\\
a\rightarrow a\oplus a
\end{array}\quad.
\end{align*}
Applying a function to a disjunctive value such as $x^{:Z+A}$ is
computed by writing $x$ as a row vector:
\[
x=z^{:Z}+\bbnum 0^{:A}=\,\begin{array}{|cc|}
z^{:Z} & \bbnum 0\end{array}\quad.
\]
Then the computation $x\triangleright f\bef g$ again follows the
rules of matrix multiplication:
\[
x\triangleright f\bef g=\,\begin{array}{|cc|}
z^{:Z} & \bbnum 0\end{array}\,\triangleright\,\begin{array}{||c|}
\_\rightarrow e^{:A}\\
a\rightarrow a\oplus a
\end{array}\,=z\triangleright(\_\rightarrow e)=e\quad.
\]
Since the standard rules of matrix multiplication are associative,
the properties of the $\triangleright$-notation such as $x\triangleright(f\bef g)=(x\triangleright f)\triangleright g$
are guaranteed to hold with matrices.

To use the matrix notation with \emph{backward} compositions ($f\circ g$),
all code matrices need to be transposed. (A standard identity of matrix
calculus is that the transposition reverses the order of composition:
$\left(AB\right)^{T}=B^{T}A^{T}$.) The input types will then appear
in the \emph{top} \emph{row} and the output types in the left column.
The double line is at the top of a code matrix since that is where
the function inputs come from. The above calculations are then rewritten
as:
\begin{align*}
g\circ f & =\,\begin{array}{|c|cc|}
 & Z & A\times A\\
\hline\hline A & \_\rightarrow e^{:A} & a_{1}\times a_{2}\rightarrow a_{1}\oplus a_{2}
\end{array}\,\circ\,\begin{array}{|c|cc|}
 & Z & A\\
\hline\hline Z & \text{id} & \bbnum 0\\
A\times A & \bbnum 0 & a\rightarrow a\times a
\end{array}\\
 & =\,\,\begin{array}{|cc|}
\hline\hline \text{id}\bef(\_\rightarrow e^{:A}) & \left(a\rightarrow a\times a\right)\bef\left(a_{1}\times a_{2}\rightarrow a_{1}\oplus a_{2}\right)\end{array}\,=\,\begin{array}{|cc|}
\hline\hline \_\rightarrow e^{:A} & a\rightarrow a\oplus a\end{array}\quad.\\
(g\circ f)(x) & =\,\begin{array}{|cc|}
\hline\hline \_\rightarrow e^{:A} & a\rightarrow a\oplus a\end{array}\,\,\begin{array}{|c|}
z^{:Z}\\
\bbnum 0
\end{array}\,=(\_\rightarrow e^{:A})(z)=e\quad.
\end{align*}
The \emph{forward} composition ($\bef$) may be easier to read and
to reason about in the matrix notation.

\chapter{Glossary of terms\label{chap:Appendix-Glossary-of-terms}}
\begin{description}
\item [{Code~notation}] \index{code notation}A mathematical notation
developed in this book for deriving properties of code in functional
programs. Variables have optional type annotations, such as $x^{:A}$
or $f^{:A\rightarrow B}$. Nameless functions are denoted by$x^{:A}\rightarrow f$,
products by $a\times b$, and values of a disjunctive type $A+B$
are written as $x^{:A}+\bbnum 0^{:B}$ or $\bbnum 0^{:A}+y^{:B}$.
Functions working with disjunctive types are denoted by matrices.
Lifting of functions to functors, such as $\text{fmap}_{L}(f)$, is
denoted by $f^{\uparrow L}$; function compositions are denoted by
$f\bef g$ (forward composition) and $f\circ g$ (backward composition);
and function applications by $f(x)$ or equivalently $x\triangleright f$.
See Appendix~\ref{chap:Appendix-Notations} for details.
\item [{\index{contrafunctor}Contrafunctor}] A type constructor having
the properties of a contravariant functor\index{contrafunctor} with
respect to a type parameter. Instead of \textsf{``}contravariant functor\textsf{''},
this book uses the shorter name \textsf{``}contrafunctor\textsf{''}.
\item [{Disjunctive~type}] \index{disjunctive type}A type representing
one of several distinct possibilities. In Scala, this is usually implemented
as a sealed trait extended by several case classes. The standard Scala
disjunction types are \lstinline!Option[A]! and \lstinline!Either[A, B]!.
Also known as \index{sum type!see \textsf{``}disjunctive type\textsf{''}}\textbf{sum
}type, \textbf{tagged union}\index{tagged union type!see \textsf{``}disjunctive type\textsf{''}}
type, \textbf{co-product}\index{co-product type!see \textsf{``}disjunctive type\textsf{''}}
type, and variant type (in Object Pascal and in OCaml). The shortest
name is \textsf{``}sum type,\textsf{''} but the English word \textsf{``}disjunctive\textsf{''} is
less ambiguous to the ear than \textsf{``}sum\textsf{''}.
\item [{Exponential-polynomial~type}] \index{exponential-polynomial type}A
type constructor built using products, disjunctions (sums or co-products),
and function types (\textsf{``}exponentials\textsf{''}), as well as type parameters
and fixed types. For example,  \lstinline!type F[A] = Either[(A,A), Int=>A]!
is an exponential-polynomial type constructor. Such type constructors
are always profunctors and can also be functors or contrafunctors.
\item [{\index{functor block}Functor~block}] A short syntax for composing
several \lstinline!map!, \lstinline!flatMap!, and \lstinline!filter!
operations applied to a functor-typed value. The type constructor
corresponding to that value must be a functor and is fixed throughout
the entire functor block. For example, the Scala code
\begin{lstlisting}
for { x <- List(1,2,3); y <- List(10, x); if y > 2 }
  yield 2 * y
\end{lstlisting}
is equivalent to the code
\begin{lstlisting}
List(1, 2, 3).flatMap(x => List(10, x))
  .filter(y => y > 1).map(y => 2 * y)
\end{lstlisting}
and computes the value \lstinline!List(20, 20, 20, 6)!. This is a
functor block that \textsf{``}raises\textsf{''} computations to the \lstinline!List!
functor. Similar syntax exists in a number of languages and is called
a \textbf{\textsf{``}for-comprehension\textsf{''}}\index{for-comprehensions (Python)@\texttt{for}-comprehensions (Python)}
or a \textsf{``}list comprehension\textsf{''} in Python, \textbf{\textsf{``}do-notation\textsf{''}}\index{do-notation (Haskell)@\texttt{do}-notation (Haskell)}
in Haskell, and \textbf{\textsf{``}computation expressions\textsf{''}}\index{computation expressions (F#)@computation expressions (F\#)}
in F\#. I use the name \textsf{``}functor block\textsf{''} in this book because it
is shorter and more descriptive. (The type constructor used in a functor
block needs to be at least a functor but does not have to be a monad.)
\item [{Kleisli~function}] \index{Kleisli!functions} A function with
type signature $A\rightarrow M^{B}$ (in Scala, \lstinline!A => M[B]!)
for some fixed monad $M$. Also called a Kleisli morphism\index{Kleisli!morphisms}
(a morphism in the Kleisli category corresponding to the monad $M$).
The monadic method $\text{pure}_{M}:A\rightarrow M^{A}$ has the type
signature of a Kleisli function. The Kleisli composition operation,
$\diamond_{M}$, is a binary operation that combines two Kleisli functions
(of types $A\rightarrow M^{B}$ and $B\rightarrow M^{C}$) into a
new Kleisli function (of type $A\rightarrow M^{C}$).
\item [{\index{method}Method}] This word is used in two ways: 1) A method$_{1}$
is a Scala function defined as a member of a typeclass. For example,
\lstinline!flatMap! is a method defined in the \lstinline!Monad!
typeclass. 2) A method$_{2}$ is a Scala function defined as a member
of a data type declared as a Java-compatible \lstinline!class! or
\lstinline!trait!. Trait methods$_{2}$ are necessary in Scala when
implementing functions whose arguments have type parameters (because
Scala function values defined via \lstinline!val! cannot have type
parameters). So, many typeclasses such as \lstinline!Functor! or
\lstinline!Monad!, whose methods$_{1}$ require type parameters,
will use Scala \lstinline!traits! with methods$_{2}$ for their implementation.
The same applies to type constructions with quantified types, such
as the Church encoding. 
\item [{Nameless~function}] \index{nameless function}An expression of
function type, representing a function. For example, \lstinline!(x: Int) => x * 2!.
Also known as function expression, function literal, anonymous function,\index{anonymous function!see \textsf{``}nameless functions\textsf{''}}
closure, \index{lambda-function!see \textsf{``}nameless function\textsf{''}}lambda-function,
lambda-expression, or simply a \textsf{``}lambda\textsf{''}.
\item [{Partial~type-to-value~function~(PTVF)}] A function with a type
parameter but defined only for a certain subset of types.\index{partial type-to-value function}
In Scala, PTVFs are implemented via a typeclass constraint:
\begin{lstlisting}
def double[T: Semigroup](t: T): T = implicitly[Semigroup[T]].combine(t, t)
\end{lstlisting}
This PTVF is defined only for types \lstinline!T! for which a \lstinline!Semigroup!
typeclass instance is available.
\item [{Polynomial~functor}] \index{polynomial functor}A type constructor
built using disjunctions (sums), products (tuples), type parameters
and fixed types. For example, in Scala, \lstinline!type F[A] = Either[(Int, A), A]!
is a polynomial functor with respect to the type parameter \lstinline!A!,
while \lstinline!Int! is a fixed type (not a type parameter). Polynomial
functors are also known as \textbf{algebraic data types}\index{algebraic data types}.
\item [{Product~type}] \index{product type}A type representing several
values given at once. In Scala, product types are the tuple types,
for example \lstinline!(Int, String)!, and case classes. Also known
as \index{tuples}\textbf{tuple} type, \textbf{struct} (in C and C++),
and \textbf{record}.
\item [{\index{profunctor}Profunctor}] A type constructor whose type parameter
occurs in both covariant and contravariant positions and satisfying
the appropriate laws; see Section~\ref{subsec:f-Profunctors}.
\item [{Type~notation}] \index{type notation}A mathematical notation
for type expressions developed in this book for easier reasoning about
types in functional programs. Disjunctive types are denoted by $+$,
product types by $\times$, and function types by $\rightarrow$.
The unit type is denoted by $\bbnum 1$, and the void type by $\bbnum 0$.
The function arrow $\rightarrow$ groups weaker than $+$, which in
turn groups weaker than $\times$. This means
\[
Z+A\rightarrow Z+A\times A\quad\text{is the same as}\quad\left(Z+A\right)\rightarrow\left(Z+\left(A\times A\right)\right)\quad.
\]
 Type parameters are denoted by superscripts. As an example, the Scala
definition\texttt{}
\begin{lstlisting}
type F[A] = Either[(A, A => Option[Int]), String => List[A]]
\end{lstlisting}
is written in the type notation as 
\[
F^{A}\triangleq A\times\left(A\rightarrow\bbnum 1+\text{Int}\right)+(\text{String}\rightarrow\text{List}^{A})\quad.
\]
\item [{\index{unfunctor}Unfunctor}] A type constructor that cannot possibly
be a functor, nor a contrafunctor, nor a profunctor. An example is
a type constructor with explicitly indexed type parameters, such as
$F^{A}\triangleq\left(A\times A\right)^{:F^{\text{Int}}}+\left(\text{Int}\times A\right)^{:F^{\bbnum 1}}$.
The Scala code for this type constructor is
\begin{lstlisting}
sealed trait F[A]
final case class F1[A](x: A, y: A)   extends F[Int]
final case class F2[A](s: Int, t: A) extends F[Unit]
\end{lstlisting}
This can be seen as a \index{GADT!see \textsf{``}unfunctor\textsf{''}}\textbf{GADT}
(generalized algebraic data type) that uses specific values of type
parameters instead of the type parameter $A$ in at least some of
its case classes.
\end{description}

\section{How the term \textquotedblleft algebra\textquotedblright{} is used
in functional programming}

This book avoids using the terms \textsf{``}algebra\index{algebra}\textsf{''} or
\textsf{``}algebraic\index{algebraic}\textsf{''} because these terms are too ambiguous.
As we will see, the FP community uses the word \textsf{``}algebra\textsf{''} in at
least \emph{three} incompatible ways.

\paragraph{Definition 0.}

In mathematics, an \textquotedblleft algebra\textquotedblright{} is
a vector space with a multiplication operation ($*$) satisfying certain
properties, such as distributivity: $z*(x+y)=z*x+z*y$. For example,
the set of all $10\times10$ matrices with real coefficients is a
$100$-dimensional vector space that satisfies the required properties.
Such matrices form an \textquotedblleft algebra\textquotedblright{}
in the mathematical sense. (This meaning of the word \textsf{``}algebra\textsf{''}
is \emph{not} used in FP.)

\paragraph{Definition 1.}

For a given functor $F$, an \textsf{``}$F$-algebra\textsf{''}\index{$F$-algebra}
is a type $T$ and a function with type signature $F^{T}\rightarrow T$.
This definition comes from category theory. The name \textsf{``}$F$-algebra\textsf{''}
emphasizes the dependence on a chosen functor $F$. There is no direct
connection between \textsf{``}$F$-algebra\textsf{''} and Definition~0, except when
the functor $F$ is defined by $F^{A}\triangleq A\times A$: a function
of type $T\times T\rightarrow T$ may then be interpreted as a \textsf{``}multiplication\textsf{''}
operation for values of type $T$. However, even in that case $T$
will not necessarily satisfy the laws of a vector space.

A recursive type $T$ defined by $F^{T}\cong T$ has an isomorphism
$F^{T}\rightarrow T$ and so is an $F$-algebra. Recursive polynomial
types are known as \textsf{``}algebraic\textsf{''} in this sense: they are $F$-algebras
with a polynomial recursion scheme $F^{\bullet}$ and an added requirement
of isomorphism for the map $F^{T}\rightarrow T$. In terms of category
theory, least fixpoints of a type equation $F^{T}\cong T$ is the
\textsf{``}initial $F$-algebra\textsf{''}.\footnote{See \texttt{\href{https://homepages.inf.ed.ac.uk/wadler/papers/free-rectypes/free-rectypes.txt}{https://homepages.inf.ed.ac.uk/wadler/papers/free-rectypes/free-rectypes.txt}} }

Another use of $F$-algebras is in formulating properties of inductive
typeclasses.\index{inductive typeclass}\index{typeclass!inductive}
An inductive $P$-typeclass is a $P$-algebra with (usually) additional
laws imposed. This book prefers the term \textsf{``}inductive typeclass\textsf{''}
or \textsf{``}$P$-typeclass\textsf{''} instead of calling those typeclasses \textsf{``}$P$-algebraic\textsf{''}.

The \textsf{``}algebra\textsf{''} defined in this sense is also relevant to the Church
encoding of a free monad (also known as the \textsf{``}tagless final\index{tagless final}\textsf{''}
pattern), or more generally to Church encodings of free inductive
typeclasses that involve functions of type $\forall E^{\bullet}.\,(S^{E}\leadsto E)\leadsto E$.
That type uses a higher-order type $S^{E}$ parameterized by a \emph{type
constructor} parameter $E^{\bullet}$. In this context, a value of
type $S^{E}\leadsto E$ (or, more verbosely, $\forall A.\,S^{A,E^{\bullet}}\rightarrow E^{A}$)
suggests the existence of an $S$-algebra in the category of type
constructors. However, knowing about the existence of that $S$-algebra
does not provide any help or additional insights for practical work
with the Church encodings of free typeclasses.

\paragraph{Definition 2.}

Polynomial data types (for example, \lstinline!type F[A] = Option[(A, A)]!)
are often called \textquotedblleft algebraic data types\index{algebraic data types}\textquotedblright{}
(ADTs). An ADT may admit some binary or unary operations, but this
does not turn it into an algebra in the sense of Definition~0. The
types of ADTs are also not of the form $F^{A}\rightarrow A$ and are
not necessarily defined recursively via a type equation $F^{T}\cong T$,
so Definition~1 does not apply directly. The word \textsf{``}algebraic\textsf{''}
in \textsf{``}ADT\textsf{''} may be better understood as referring to \textsf{``}school-level
algebra\textsf{''} dealing with polynomials, as the ADTs are built from \textsf{``}sums\textsf{''}
and \textsf{``}products\textsf{''} of types. 

Instead of calling types \textsf{``}algebraic\textsf{''}, this book uses more precise
terms \textquotedblleft polynomial types\textquotedblright{} and \textquotedblleft exponential-polynomial
types\textquotedblright . Data types containing functions, such as
\lstinline!Option[Int => (A, A)]!, are exponential-polynomial (but
not polynomial).

\paragraph{Definition 3.}

One talks about the \textquotedblleft algebra\textquotedblright{}
of properties of functions such as \lstinline!map! or \lstinline!flatMap!,
meaning that these functions satisfy certain \textsf{``}algebraic\textsf{''} laws
(e.g., the composition, naturality, or associativity laws). But these
laws do not make functions \lstinline!map! or \lstinline!flatMap!
into an algebra in the sense of Definitions~0 or~1. There is also
no relation to the ADTs of Definition~2. So, this is a different
usage of the word \textsf{``}algebra\textsf{''}. However, there is no general \textsf{``}algebra
of laws\textsf{''} that we can use. Every derivation proceeds in a different
way, specific to the laws being proved. In mathematics, \textsf{``}algebraic\textsf{''}
equations are distinguished from differential or integral equations.
In that sense, the laws used in FP are \emph{always} algebraic: they
are just equations with compositions and applications of functions.
So, there is no advantage in calling FP laws \textsf{``}algebraic\textsf{''}. 

We find that the current usage of the word \textsf{``}algebra\textsf{''} in FP is
both inconsistent and unhelpful to software practitioners. In this
book, the word \textsf{``}algebra\textsf{''} always means a branch of mathematics,
as in \textsf{``}high-school algebra\textsf{''}. Instead of \textsf{``}algebras\textsf{''} as in Definitions~1
to~3, this book talks about \textsf{``}polynomial types\textsf{''} or \textsf{``}recursive
polynomial types\textsf{''}, \textsf{``}equations\textsf{''} or \textsf{``}laws\textsf{''}, and $P$-typeclasses.\index{$P$-typeclass}

\chapter{Parametricity theorem and naturality laws\label{app:Proofs-of-naturality-parametricity}}

Functional programming focuses on a small set of language features
\textemdash{} the six type constructions and the nine code constructions\index{nine code constructions},
introduced in Sections~\ref{subsec:Type-notation-and-standard-type-constructions}
and~\ref{subsec:The-rules-of-proof}. These constructions, summarized
again in Tables~\ref{tab:Mathematical-notation-for-basic-code-constructions}
and \ref{tab:six-pure-type-constructions}\textendash \ref{tab:nine-pure-code-constructions},
create \textbf{fully parametric} programs\index{fully parametric!code}
and are sufficient to implement all design patterns of functional
programming. At the same time, restricting programs to be fully parametric
will enable several ways of mathematical reasoning about code. Examples
of such reasoning are treating programs as mathematical values (referential
transparency\index{referential transparency}); deriving the most
general type from code (type inference\index{type inference}); and
deriving code from type (code inference\index{code inference}).

Additionally, all fully parametric programs will automatically satisfy
certain laws derived from the so-called \textsf{``}parametricity theorem\textsf{''}.
The most often used laws of this kind are naturality laws for natural
transformations, i.e., for functions of type $\forall A.\,F^{A}\rightarrow G^{A}$
where both $F^{\bullet}$ and $G^{\bullet}$ are functors (or both
contrafunctors). Not having to verify naturality laws by hand saves
a lot of time.

Other laws that follow automatically from parametricity are composition
laws for functors, commutativity laws for bifunctors, and dinaturality
laws for profunctors. This Appendix presents sufficient theoretical
material to be able to derive all those laws. 

\begin{wraptable}{l}{0.545\columnwidth}%
\begin{centering}
\vspace{-0.2\baselineskip}
\begin{tabular}{|c|c|c|}
\hline 
\textbf{\small{}Type construction} & \textbf{\small{}Scala example} & \textbf{\small{}Type notation}\tabularnewline
\hline 
\hline 
{\small{}unit or \textsf{``}named unit\textsf{''}} & {\small{}}\lstinline!Unit!{\small{} or }\lstinline!None!{\small{} } & {\small{}$\bbnum 1$}\tabularnewline
\hline 
{\small{}type parameter} & {\small{}}\lstinline!A!{\small{} as in }\lstinline!F[A]!{\small{} } & {\small{}$A$ as in $F^{A}$}\tabularnewline
\hline 
{\small{}product type} & {\small{}}\lstinline!(A, B)! & {\small{}$A\times B$}\tabularnewline
\hline 
{\small{}co-product type} & {\small{}}\lstinline!Either[A, B]! & {\small{}$A+B$}\tabularnewline
\hline 
{\small{}function type} & {\small{}}\lstinline!A => B! & {\small{}$A\rightarrow B$}\tabularnewline
\hline 
{\small{}recursive type} & {\small{}}\lstinline!Fix[F[_]]! & {\small{}$\text{Fix}^{F^{\bullet}}$}\tabularnewline
\hline 
\end{tabular}
\par\end{centering}
\caption{\index{fully parametric!type constructions}The six type constructions
that may be used in fully parametric programs.\label{tab:six-pure-type-constructions}}
\vspace{-0.2\baselineskip}
\end{wraptable}%

\begin{table}
\begin{centering}
\begin{tabular}{|c|c|c|}
\hline 
\textbf{\small{}Code construction} & \textbf{\small{}Scala example} & \textbf{\small{}Code notation}\tabularnewline
\hline 
\hline 
{\small{}1. Use unit value} & {\small{}}\lstinline!()! & {\small{}$1$}\tabularnewline
\hline 
{\small{}2. Use given argument} & {\small{}}\lstinline!x! & {\small{}$x$}\tabularnewline
\hline 
{\small{}3. Create function} & {\small{}}\lstinline!x => expression! & {\small{}$x\rightarrow expression$}\tabularnewline
\hline 
{\small{}4. Use function} & {\small{}}\lstinline!f(x)! & $f(x)$ ~or~ $x\triangleright f$\tabularnewline
\hline 
{\small{}5. Create tuple} & {\small{}}\lstinline!(a, b)! & {\small{}$a\times b$}\tabularnewline
\hline 
{\small{}6. Use tuple} & {\small{}}\lstinline!{ case (a, b) => ... }!{\small{} or }\lstinline!p._1!{\small{}
or }\lstinline!p._2!{\small{} } & {\small{}$a\times b\rightarrow...$ ~or~ $p\triangleright\pi_{1}$
~or~ $p\triangleright\pi_{2}$}\tabularnewline
\hline 
{\small{}7. Create disjunctive value} & {\small{}}\lstinline!Left[A, B](a)! & {\small{}}%
\begin{minipage}[c]{0.19\columnwidth}%
{\small{}\vspace{0.2\baselineskip}
$a^{:A}+\bbnum 0^{:B}$ ~or~~ $\begin{array}{|cc|}
a & \bbnum 0\end{array}$\vspace{0.2\baselineskip}
}%
\end{minipage}\tabularnewline
\hline 
{\small{}8. Use disjunctive value} & {\small{}}%
\begin{minipage}[c]{0.33\columnwidth}%
{\small{}}\lstinline!p match { case Left(a)   => f(a)            case Right(b)  => g(b)          }!%
\end{minipage} & {\small{}}%
\begin{minipage}[c]{0.23\columnwidth}%
{\small{}\vspace{0.2\baselineskip}
$p\,\triangleright\,\begin{array}{|c||c|}
 & C\\
\hline A & a\rightarrow f(a)\\
B & b\rightarrow g(b)
\end{array}$\vspace{0.2\baselineskip}
}%
\end{minipage}\tabularnewline
\hline 
{\small{}9. Recursive call} & {\small{}}\lstinline!def f(x) = { ... f(y) ... }! & {\small{}$f(x)\triangleq...~\overline{f}(y)~...$}\tabularnewline
\hline 
\end{tabular}
\par\end{centering}
\caption{\index{fully parametric!code constructions}The nine code constructions
of fully parametric programs.\label{tab:nine-pure-code-constructions}}
\end{table}

The practical uses of parametricity were showcased in \index{Philip Wadler}P.~Wadler\textsf{'}s
paper\footnote{See \texttt{\href{https://people.mpi-sws.org/~dreyer/tor/papers/wadler.pdf}{https://people.mpi-sws.org/$\sim$dreyer/tor/papers/wadler.pdf}}
and some further explanations in the blog posts \texttt{\href{https://reasonablypolymorphic.com/blog/theorems-for-free/}{https://reasonablypolymorphic.com/blog/theorems-for-free/}}
and \texttt{\href{https://bartoszmilewski.com/2014/09/22/}{https://bartoszmilewski.com/2014/09/22/}}\\
The title of the paper (\textsf{``}Theorems for free\textsf{''}\index{theorems for free})
is misleading. As we will see, the theorems are obtained only at the
cost of restricting all code to the nine FP code constructions, which
excludes most of the code in today\textsf{'}s real-world applications.} based on results derived by J.~Reynolds.\index{John Reynolds}\footnote{See \texttt{\href{https://people.mpi-sws.org/~dreyer/tor/papers/reynolds.pdf}{https://people.mpi-sws.org/$\sim$dreyer/tor/papers/reynolds.pdf}}.
This book does not use those results. } The Reynolds-Wadler approach first replaces functions of type $A\rightarrow B$
by many-to-many relations between values of types $A$ and $B$. Then
the parametricity theorem shows that a certain law holds for relations.
Since we are usually interested in deriving laws as equations rather
than relations, the last step replaces all relations by functions.

To use the relational approach to parametricity, one needs to learn
the techniques for working with relations, which will be unfamiliar
to most programmers.\footnote{Beginner-level tutorials on relations and relational parametricity
are rare. One is in a paper by \index{Ronald Backhouse}R.~Backhouse,
see \texttt{\href{https://www.researchgate.net/publication/262348393_On_a_Relation_on_Functions}{https://www.researchgate.net/publication/262348393\_On\_a\_Relation\_on\_Functions}}} Only a few parametricity properties can be proved without using relations.
However, a practicing programmer will mostly only use the \emph{formulations}
of the resulting laws rather than their proofs. Adapting the parametricity
theorem to the needs of FP practitioners, this Appendix will begin
by summarizing the naturality laws and other properties of fully parametric
code. Then Section~\ref{sec:Parametricity-theorem-for-relations}
will prove the parametricity theorem via the relational approach.%
\begin{comment}
Section~\ref{sec:Commutativity-laws-for-type-constructors} proves
that fully parametric type constructors obey commutativity laws. Section~\ref{sec:Naturality-laws-for-fully-parametric-functions}
shows, without using relations, that fully parametric functions satisfy
dinaturality laws (a generalization of naturality laws to arbitrary
type signatures). An important consequence, proved in Section~\ref{sec:Uniqueness-of-functor-and-contrafunctor},
is that the code implementing the functor and contrafunctor typeclasses
is unique. Finally, Section~\ref{sec:Parametricity-theorem-for-relations}
explains the relational approach to parametricity and proves the relational
parametricity theorem. 

Additional literature:

{[}1{]}: Girard, J.-Y.; Scedrov, A. \& Scott, P. J. Normal Forms and
Cut-Free Proofs as Natural Transformations. Logic From Computer Science,
Mathematical Science Research Institute Publications 21, Springer-Verlag,
1992, 217-241. http://citeseer.ist.psu.edu/viewdoc/summary?doi=10.1.1.41.811

{[}2{]}: Bainbridge, E. S.; Freyd, P. J.; Scedrov, A. \& Scott, P.
J. Functorial polymorphism. Theoretical computer science, Elsevier,
1990, 70, 35-64. https://core.ac.uk/display/82270459

{[}3{]}: De Lataillade, J. Dinatural Terms in System F. Logic in Computer
Science, 24th Annual IEEE Symposium, 267-276, 2009. https://www.irif.fr/\textasciitilde delatail/dinat.pdf

{[}4{]}: Pistone, P. On completeness and parametricity in the realizability
semantics of System F. https://arxiv.org/abs/1802.05143

{[}5{]}: https://libres.uncg.edu/ir/asu/f/Johann\_Patricia\_2014\_A\_Relationally\_Parametric\_Model\_Of\_Dependent\_Type\_Theory..pdf

See discussion here: https://cstheory.stackexchange.com/questions/42256/is-case-analysis-on-normal-forms-of-lambda-terms-sufficient-to-prove-parametrici
\end{comment}

Parametricity properties only hold for fully parametric programs.\index{fully parametric!program}
This restriction prohibits, for instance, any use of \index{GADT}GADTs
(\textsf{``}unfunctors\textsf{''}),\index{unfunctor} subtyping,\index{subtyping}
\lstinline!null! values, mutable values, multithreading, exceptions,
run-time JVM reflection, or calling any external libraries that are
not known to be fully parametric. Code that uses those features of
Scala will not be covered by the parametricity theorem proved in this
Appendix and may fail naturality laws. 

An example of non-fully parametric code is the following function:
\begin{lstlisting}
def headOptBad[A]: List[A] => Option[A] = {
  case Nil                   => None
  case (head: Int) :: tail   => Some((head + 100).asInstanceOf[A])
  case head :: tail          => Some(head)
}
\end{lstlisting}
This code has no side effects, is purely functional and referentially\index{referential transparency}
transparent. But it is not fully parametric because \lstinline!headOptBad[A]!
does not work in the same way for all type parameters \lstinline!A!.
When \lstinline!A! is set to \lstinline!Int!, the behavior is different
than for all other types. This breaks the naturality law:
\begin{lstlisting}
scala> headOptBad[String](  List(1, 2, 3).map(x => s"value = $x")  )
res0: Option[String] = Some(value = 1)

scala> headOptBad[Int]( List(1, 2, 3) ).map(x => s"value = $x")
res1: Option[String] = Some(value = 101)
\end{lstlisting}
So, full parametricity is a stronger restriction on code than referential
transparency.

Here is an example of fully parametric code that uses all nine constructions:
\begin{lstlisting}
def fmap[A, B](f: A => B): List[(A, A)] => List[(B, B)] = {      // 3
   case Nil            => Nil
//   8   1                1,7 
   case head :: tail   => (f (head._1), f (head._2)) :: fmap(f)(tail)
//   8       6             2 4     6  5 2 4     6    7   9
}  // This code uses each of the nine fully parametric constructions.
\end{lstlisting}
For instance, the expression \lstinline!head :: tail! is a tuple
pattern that extracts the two parts of a tuple (construction 6). The
recursive call \lstinline!fmap(f)(tail)! corresponds to construction
9.

\section{Practical use of parametricity properties}

\subsection{Naturality and dinaturality laws\label{sec:Naturality-laws-for-fully-parametric-functions} }

Parametricity properties apply to fully parametric expressions (most
often, functions) with at least one type parameter. Examples of such
functions are the \lstinline!map!, \lstinline!filter!, and \lstinline!fold!
methods for the \lstinline!Option! functor (denoted by $\text{Opt}^{A}\triangleq\bbnum 1+A$),
with the following type signatures:
\begin{align*}
 & \text{fmap}_{\text{Opt}}^{A,B}:\left(A\rightarrow B\right)\rightarrow\text{Opt}^{A}\rightarrow\text{Opt}^{B}\quad,\\
 & \text{filt}_{\text{Opt}}^{A}:\left(A\rightarrow\bbnum 2\right)\rightarrow\text{Opt}^{A}\rightarrow\text{Opt}^{A}\quad,\\
 & \text{fold}_{\text{Opt}}^{A,B}:B\times\left(A\times B\rightarrow B\right)\rightarrow\text{Opt}^{A}\rightarrow B\quad.
\end{align*}
These methods satisfy certain parametricity laws, one law per type
parameter.

When a function\textsf{'}s type signature is that of a \emph{natural transformation}\index{natural transformation}
between functors or between contrafunctors, the parametricity laws
have the form derived in Section~\ref{subsec:Naturality-laws-and-natural-transformations}.
If $t:\forall A.\,G^{A}\rightarrow H^{A}$ where $G^{\bullet}$ and
$H^{\bullet}$ are functors then $t$ obeys the naturality law:
\begin{equation}
\text{for all }A,B,f^{:A\rightarrow B}\quad:\quad t^{A}\bef f^{\uparrow H}=f^{\uparrow G}\bef t^{B}\quad.\label{eq:naturality-law-for-functors}
\end{equation}
If $t:\forall A.\,G^{A}\rightarrow H^{A}$ where $G^{\bullet}$ and
$H^{\bullet}$ are contrafunctors then $t$ obeys the naturality law:
\begin{equation}
\text{for all }A,B,f^{:A\rightarrow B}\quad:\quad f^{\downarrow G}\bef t^{A}=t^{B}\bef f^{\downarrow H}\quad.\label{eq:naturality-law-for-contrafunctors}
\end{equation}

For example, if we fix the type parameter $A$ in the \lstinline!fmap!
method, we obtain a type signature of the form $F^{B}\rightarrow G^{B}$
where $F^{\bullet}$ and $G^{\bullet}$ are functors:
\[
\text{fmap}_{\text{Opt}}^{A,B}:F^{B}\rightarrow G^{B}\quad,\quad\quad F^{B}\triangleq A\rightarrow B\quad,\quad\quad G^{B}\triangleq\text{Opt}^{A}\rightarrow\text{Opt}^{B}\quad.
\]
The corresponding naturality law is:
\[
\text{fmap}_{\text{Opt}}\bef f^{\uparrow G}=f^{\uparrow F}\bef\text{fmap}_{\text{Opt}}\quad.
\]

Fixing the type parameter $B$ in $\text{fold}_{\text{Opt}}^{A,B}\,$
produces a type signature of the form:
\[
\text{fold}_{\text{Opt}}^{A}:H^{A}\rightarrow K^{A}\quad,\quad\quad H^{A}\triangleq B\times\left(A\times B\rightarrow B\right)\quad,\quad\quad K^{A}\triangleq\text{Opt}^{A}\rightarrow B\quad,
\]
where $K$ and $H$ are contrafunctors. The corresponding naturality
law is:
\[
f^{\downarrow H}\bef\text{fold}_{\text{Opt}}=\text{fold}_{\text{Opt}}\bef f^{\downarrow K}\quad.
\]

Fixing the type parameter $A$ in $\text{fold}_{\text{Opt}}^{A,B}$
gives a type signature that is \emph{not} of the form $\forall B.\,P^{B}\rightarrow Q^{B}$
(with any functors or contrafunctors $P$, $Q$) because $B$ is used
in too many places. In general, it is not straightforward to write
naturality laws for functions with such type signatures. 

To make progress, we note that each occurrence of a type parameter
in a type signature must be either in a covariant or in a contravariant
position. So, any fully parametric type signature may be written in
the form $\forall A.\,P^{A,A}$ using some \emph{profunctor}\index{profunctor}
$P^{X,Y}$ (contravariant in $X$ and covariant in $Y$). For instance,
we can write in this way the type signature of \lstinline!fold! with
the parameter $A$ fixed:
\[
\text{fold}_{\text{Opt}}^{A,B}:P^{B,B}\quad,\quad\quad\text{where}\quad P^{X,Y}\triangleq X\times(A\times Y\rightarrow X)\rightarrow\text{Opt}^{A}\rightarrow Y\quad.
\]
The profunctor $P^{X,Y}$ is complicated but the form of the type
signature ($P^{B,B}$) is simpler.

Given a fully parametric expression $p$ with the type signature $p:\forall A.\,P^{A,A}$,
where $P^{X,Y}$ is some profunctor, one can write the \textbf{wedge
law}\index{wedge law!of profunctors}\index{profunctor!wedge law}
of $p$:

\begin{wrapfigure}[9]{l}{0.26\columnwidth}%
\vspace{-1.3\baselineskip}
\[
\xymatrix{\xyScaleY{2.0pc}\xyScaleX{3.0pc}\forall Z.\,P^{Z,Z}\ar[d]\sp(0.4){\text{set }Z=A}\ar[r]\sp(0.55){\text{set }Z=B} & P^{B,B}\ar[d]\sp(0.37){f^{\downarrow P^{\bullet,B}}}\\
P^{A,A}\ar[r]\sp(0.55){f^{\uparrow P^{A,\bullet}}} & P^{A,B}
}
\]

\vspace{-0.4\baselineskip}
\end{wrapfigure}%

~\vspace{-0.8\baselineskip}

\begin{equation}
\text{for all }A,B,f^{:A\rightarrow B}\quad:\quad p^{A}\triangleright f^{\uparrow P^{A,\bullet}}=p^{B}\triangleright f^{\downarrow P^{\bullet,B}}\quad.\label{eq:wedge-law-for-profunctors}
\end{equation}
\vspace{-0.8\baselineskip}

\noindent The diagram is read by starting from a value $p$ of type
$\forall Z.\,P^{Z,Z}$. We set $Z=A$ and $Z=B$ in $p$ and obtain
two values, $p^{A}$ and $p^{B}$, of types $P^{A,A}$ and $P^{B,B}$
respectively. The values $p^{A}$ and $p^{B}$ must then satisfy Eq.~(\ref{eq:wedge-law-for-profunctors}). 

We will prove later in this Appendix that the wedge law~(\ref{eq:wedge-law-for-profunctors})
is a consequence of the parametricity theorem. In turn, the naturality
laws~(\ref{eq:naturality-law-for-functors}) and~(\ref{eq:naturality-law-for-contrafunctors})
are consequences of the wedge law, as will be shown in Statement~\ref{subsec:Statement-naturality-laws-from-wedge-law}
below.

The wedge law is not easy to use in practice because the profunctors
$P$ are often complicated (as the example of \lstinline!fold! shows).
When we need to derive naturality laws for \emph{function types},
it helps to specialize the wedge law to the case $P^{A,A}=G^{A,A}\rightarrow H^{A,A}$,
where $G^{X,Y}$ and $H^{X,Y}$ are some profunctors. The result (also
derived in Statement~\ref{subsec:Statement-naturality-laws-from-wedge-law})
is the \textbf{dinaturality law}\index{dinaturality law}:
\begin{equation}
\text{for all }A,B,f^{:A\rightarrow B}\quad:\quad f^{\downarrow G^{\bullet,A}}\bef p^{A}\bef f^{\uparrow H^{A,\bullet}}=f^{\uparrow G^{B,\bullet}}\bef p^{B}\bef f^{\downarrow H^{\bullet,B}}\quad.\label{eq:dinaturality-law-for-profunctors}
\end{equation}

\begin{wrapfigure}[7]{l}{0.38\columnwidth}%
\vspace{-2.2\baselineskip}
\[
\xymatrix{\xyScaleY{1.5pc}\xyScaleX{2.0pc} & G^{A,A}\ar[r]\sb(0.5){p^{A}} & H^{A,A}\ar[rd]\sb(0.4){f^{\uparrow H^{A,\bullet}}\negthickspace}\\
G^{B,A}\negthickspace\negthickspace\negthickspace\ar[rd]\sp(0.55){~~f^{\uparrow G^{B,\bullet}}}\ar[ru]\sb(0.6){\negthickspace f^{\downarrow G^{\bullet,A}}} &  &  & \negthickspace\negthickspace\negthickspace H^{A,B}\\
 & G^{B,B}\ar[r]\sp(0.5){p^{B}} & H^{B,B}\ar[ru]\sp(0.45){f^{\downarrow H^{\bullet,B}}\negthickspace}
}
\]

\vspace{-1.4\baselineskip}
\end{wrapfigure}%

\noindent The diagram illustrates the dinaturality law as an equation
between functions of type $G^{B,A}\rightarrow H^{A,B}$. To build
up intuition for that law, notice that Eq.~(\ref{eq:dinaturality-law-for-profunctors})
combines the laws~(\ref{eq:naturality-law-for-functors})\textendash (\ref{eq:naturality-law-for-contrafunctors})
in the way required for all types to match. On the other hand, the
laws~(\ref{eq:naturality-law-for-functors}) and~(\ref{eq:naturality-law-for-contrafunctors})
will follow from Eq.~(\ref{eq:dinaturality-law-for-profunctors})
when $G^{A,A}$ and $H^{A,A}$ are both functors or both contrafunctors
with respect to $A$.

Functions $p:\forall A.\,G^{A,A}\rightarrow H^{A,A}$ satisfying Eq.~(\ref{eq:dinaturality-law-for-profunctors})
are called \textbf{dinatural transformations}\index{dinatural transformation}.
The property of dinaturality is weaker than naturality.\emph{ }A \emph{natural}
transformation between profunctors $G$ and $H$ would be a function
$t$ with type signature $\forall(A,B).\,G^{A,B}\rightarrow H^{A,B}$
defined for arbitrary (not necessarily equal) type parameters $A$,
$B$. If we are given a transformation $p^{A}:G^{A,A}\rightarrow H^{A,A}$
then, as a rule, it will be impossible to extend the code of $p^{A}$
to some $t^{A,B}:G^{A,B}\rightarrow H^{A,B}$ that works with arbitrary
type parameters $A$, $B$.

With the formulas~(\ref{eq:naturality-law-for-functors})\textendash (\ref{eq:dinaturality-law-for-profunctors}),
we can write naturality laws more quickly, starting from any given
type signature of the form $\forall A.\,P^{A,A}$. The following examples
show how to derive naturality laws by specializing the general law~(\ref{eq:dinaturality-law-for-profunctors})
to certain profunctors $G$ and $H$.

\subsubsection{Example \label{subsec:Example-derive-naturality-of-filter-from-dinaturality}\ref{subsec:Example-derive-naturality-of-filter-from-dinaturality}
(naturality law of \lstinline!filter!)\index{solved examples}}

To derive the naturality law of \lstinline!filter!, express \lstinline!filter!\textsf{'}s
type signature through profunctors $G$ and $H$ as:
\[
\text{filt}_{F}^{A}:G^{A,A}\rightarrow H^{A,A}\quad,\quad\quad G^{X,Y}\triangleq(X\rightarrow\bbnum 2)\quad,\quad\quad H^{X,Y}\triangleq F^{X}\rightarrow F^{Y}\quad,
\]
and then write the law~(\ref{eq:dinaturality-law-for-profunctors}):
\begin{equation}
f^{\downarrow G^{\bullet,A}}\bef\text{filt}_{F}^{A}\bef f^{\uparrow H^{A,\bullet}}\overset{?}{=}f^{\uparrow G^{B,\bullet}}\bef\text{filt}_{F}^{B}\bef f^{\downarrow H^{\bullet,B}}\quad.\label{eq:filter-law-via-dinatural-transformation-derivation1}
\end{equation}
It remains to substitute the code for the liftings using the specific
types of $H$ and $G$:
\begin{align*}
(f^{:A\rightarrow B})^{\downarrow G^{\bullet,A}}=p^{:B\rightarrow\bbnum 2}\rightarrow f\bef p\quad, & \quad\quad f^{\uparrow G^{B,\bullet}}=\text{id}\quad,\\
(f^{:A\rightarrow B})^{\downarrow H^{\bullet,B}}=q^{:F^{B}\rightarrow F^{B}}\rightarrow f^{\uparrow F}\bef q\quad, & \quad\quad f^{\uparrow H^{A,\bullet}}=q^{:F^{A}\rightarrow F^{A}}\rightarrow q\bef f^{\uparrow F}\quad.
\end{align*}
Then we rewrite Eq.~(\ref{eq:filter-law-via-dinatural-transformation-derivation1})
as:
\[
(p\rightarrow f\bef p)\bef\text{filt}_{F}\bef(q\rightarrow q\bef f^{\uparrow F})\overset{?}{=}\text{id}\bef\text{filt}_{F}\bef(q\rightarrow f^{\uparrow F}\bef q)\quad.
\]
To simplify the form of the naturality law, apply both sides to an
arbitrary $p^{:P^{B,A}}=p^{:B\rightarrow\bbnum 2}$:
\begin{align*}
{\color{greenunder}\text{left-hand side}:}\quad & p\triangleright(p\rightarrow f\bef p)\bef\text{filt}_{F}\bef(q\rightarrow q\bef f^{\uparrow F})\\
{\color{greenunder}\triangleright\text{-notation}:}\quad & \quad=\gunderline{p\triangleright(p}\rightarrow f\bef p)\triangleright\text{filt}_{F}\triangleright(q\rightarrow q\bef f^{\uparrow F})\\
{\color{greenunder}\text{apply functions}:}\quad & \quad=\gunderline{(f\bef p)\triangleright\text{filt}_{F}}\triangleright(q\rightarrow q\bef f^{\uparrow F})=\gunderline{\text{filt}_{F}(f\bef p)\triangleright(q}\rightarrow q\bef f^{\uparrow F})=\text{filt}_{F}(f\bef p)\bef f^{\uparrow F}\quad,\\
{\color{greenunder}\text{right-hand side}:}\quad & p\triangleright\gunderline{\text{id}\bef}\text{filt}_{F}\bef(q\rightarrow f^{\uparrow F}\bef q)=p\triangleright\text{filt}_{F}\triangleright(q\rightarrow f^{\uparrow F}\bef q)\\
 & \quad=\gunderline{\text{filt}_{F}(p)\triangleright(q}\rightarrow f^{\uparrow F}\bef q)=f^{\uparrow F}\bef\text{filt}_{F}(p)\quad.
\end{align*}
We obtained the naturality law~(\ref{eq:naturality-law-of-filter})
of \lstinline!filter!:
\[
\text{filt}_{F}(f\bef p)\bef f^{\uparrow F}=f^{\uparrow F}\bef\text{filt}_{F}(p)\quad.
\]


\subsubsection{Example \label{subsec:Example-derive-naturality-of-fold-from-dinaturality}\ref{subsec:Example-derive-naturality-of-fold-from-dinaturality}
(naturality law of \lstinline!fold!)}

To derive the naturality law of \lstinline!fold! with respect to
the type parameter $B$, we write the type signature of \lstinline!fold!
as $G^{B,B}\rightarrow H^{B,B}$ with some profunctors $G$, $H$:
\[
\text{fold}_{F}^{A,B}:G^{B,B}\rightarrow H^{B,B}\quad\quad\text{where}\quad G^{X,Y}\triangleq Y\times(A\times X\rightarrow Y)\quad\text{ and }\quad H^{X,Y}\triangleq F^{A}\rightarrow Y\quad.
\]
Since the type parameter $A$ is fixed, let us write the law~(\ref{eq:dinaturality-law-for-profunctors})
with an arbitrary function $f^{:B\rightarrow C}$:
\begin{equation}
(f^{:B\rightarrow C})^{\downarrow G^{\bullet,B}}\bef\text{fold}_{F}^{B}\bef f^{\uparrow H^{B,\bullet}}\overset{!}{=}f^{\uparrow G^{C,\bullet}}\bef\text{fold}_{F}^{C}\bef f^{\downarrow H^{\bullet,C}}\quad.\label{eq:fold-naturality-from-profunctor-derivation2}
\end{equation}
The lifting code required for the profunctors $G^{X,Y}\triangleq Y\times\left(A\times X\rightarrow Y\right)$
and $H^{X,Y}\triangleq F^{A}\rightarrow Y$ is:
\begin{align*}
(f^{:B\rightarrow C})^{\downarrow G^{\bullet,B}}=\text{id}^{B}\boxtimes(h^{:A\times C\rightarrow B}\rightarrow a^{:A}\times b^{:B}\rightarrow h(a\times f(b)))\quad, & \quad\quad f^{\uparrow G^{C,\bullet}}=f\boxtimes(h^{:A\times C\rightarrow B}\rightarrow h\bef f)\quad,\\
(f^{:B\rightarrow C})^{\downarrow H^{\bullet,C}}=\text{id}\quad, & \quad\quad f^{\uparrow H^{B,\bullet}}=q^{:F^{A}\rightarrow B}\rightarrow q\bef f\quad.
\end{align*}
Substituting this code into the law~(\ref{eq:fold-naturality-from-profunctor-derivation2})
and applying to an arbitrary $p^{:G^{C,B}}=z^{:B}\times h^{:A\times C\rightarrow B}$,
we get:
\begin{align*}
{\color{greenunder}\text{left-hand side}:}\quad & (z\times h)\triangleright\gunderline{(f^{:B\rightarrow C})^{\downarrow G^{\bullet,B}}}\bef\text{fold}_{F}\bef\gunderline{f^{\uparrow H^{B,\bullet}}}\\
{\color{greenunder}\text{definitions of liftings}:}\quad & \quad=(z\times h)\triangleright(\text{id}\boxtimes(h\rightarrow a\times b\rightarrow h(a\times f(b))))\gunderline{\bef}\text{fold}_{F}\gunderline{\bef}(q\rightarrow q\bef f)\\
{\color{greenunder}\triangleright\text{-notation}:}\quad & \quad=(z\times h)\triangleright(\text{id}\boxtimes(h\rightarrow a\times b\rightarrow h(a\times f(b))))\triangleright\text{fold}_{F}\triangleright(q\rightarrow q\bef f)\\
{\color{greenunder}\text{apply functions}:}\quad & \quad=\text{fold}_{F}(z\times(a\times b\rightarrow h(a\times f(b))))\bef f\quad,\\
{\color{greenunder}\text{right-hand side}:}\quad & (z\times h)\triangleright\gunderline{f^{\uparrow G^{C,\bullet}}}\bef\text{fold}_{F}\bef\gunderline{f^{\downarrow H^{\bullet,C}}}=(z\times h)\triangleright(f\boxtimes(h^{:A\times C\rightarrow B}\rightarrow h\bef f))\bef\text{fold}_{F}\bef\text{id}\\
{\color{greenunder}\text{apply functions}:}\quad & \quad=\text{fold}_{F}(f(z)\times(h\bef f))\quad.
\end{align*}
We obtain the dinaturality law of \lstinline!fold!:
\[
\text{fold}_{F}(f(z)\times(h\bef f))=\text{fold}_{F}(z\times(a\times b\rightarrow h(a\times f(b))))\bef f\quad.
\]


\subsubsection{Statement \label{subsec:Statement-naturality-laws-from-wedge-law}\ref{subsec:Statement-naturality-laws-from-wedge-law}}

The laws~(\ref{eq:naturality-law-for-functors})\textendash (\ref{eq:dinaturality-law-for-profunctors})
are special cases of the wedge law~(\ref{eq:wedge-law-for-profunctors}). 

\subparagraph{Proof}

We assume that the wedge law holds for any values $p$ of type $\forall A.\,P^{A,A}$.

To prove Eq.~(\ref{eq:naturality-law-for-functors}), we define $P^{X,Y}\triangleq G^{X}\rightarrow H^{Y}$.
The two liftings of a function $f^{:A\rightarrow B}$ to $P$ are:
\[
f^{\uparrow P^{A,\bullet}}=p^{:G^{A}\rightarrow H^{A}}\rightarrow p\bef f^{\uparrow H}\quad,\quad\quad f^{\downarrow P^{\bullet,B}}=p^{:G^{B}\rightarrow H^{B}}\rightarrow f^{\uparrow G}\bef p\quad.
\]
So, the wedge law~(\ref{eq:wedge-law-for-profunctors}) gives:
\begin{align*}
{\color{greenunder}\text{left-hand side}:}\quad & p^{A}\triangleright f^{\uparrow P^{A,\bullet}}=p^{A}\bef f^{\uparrow H}\quad,\\
{\color{greenunder}\text{right-hand side}:}\quad & p^{B}\triangleright f^{\downarrow P^{\bullet,B}}=f^{\uparrow G}\bef p^{B}\quad.
\end{align*}
We obtain $p^{A}\bef f^{\uparrow H}=f^{\uparrow G}\bef p^{B}$, which
is the same as Eq.~(\ref{eq:naturality-law-for-functors}).

To prove Eq.~(\ref{eq:naturality-law-for-contrafunctors}), we define
$P^{X,Y}\triangleq G^{Y}\rightarrow H^{X}$. The two liftings of a
function $f^{:A\rightarrow B}$ to $P$ are:
\[
f^{\uparrow P^{A,\bullet}}=p^{:G^{A}\rightarrow H^{A}}\rightarrow f^{\downarrow G}\bef p\quad,\quad\quad f^{\downarrow P^{\bullet,B}}=p^{:G^{B}\rightarrow H^{B}}\rightarrow p\bef f^{\downarrow H}\quad.
\]
So, the wedge law~(\ref{eq:wedge-law-for-profunctors}) gives:
\begin{align*}
{\color{greenunder}\text{left-hand side}:}\quad & p^{A}\triangleright f^{\uparrow P^{A,\bullet}}=f^{\downarrow G}\bef p^{A}\quad,\\
{\color{greenunder}\text{right-hand side}:}\quad & p^{B}\triangleright f^{\downarrow P^{\bullet,B}}=p^{B}\bef f^{\downarrow H}\quad.
\end{align*}
We obtain $f^{\downarrow G}\bef p^{A}=p^{B}\bef f^{\downarrow H}$,
which is the same as Eq.~(\ref{eq:naturality-law-for-contrafunctors}).

To prove Eq.~(\ref{eq:dinaturality-law-for-profunctors}), we define
$P^{X,Y}\triangleq G^{Y,X}\rightarrow H^{X,Y}$. Note that we need
to swap $X$ and $Y$ in $G^{Y,X}$ in order to conform to the required
variance of $P^{X,Y}$ (contravariant in $X$ and covariant in $Y$). 

The two liftings of a function $f^{:A\rightarrow B}$ to $P$ are
expressed as:
\[
f^{\uparrow P^{A,\bullet}}=p^{:G^{A,A}\rightarrow H^{A,A}}\rightarrow f^{\downarrow G^{\bullet,A}}\bef p\bef f^{\uparrow H^{A,\bullet}}\quad,\quad\quad f^{\downarrow P^{\bullet,B}}=p^{:G^{B,B}\rightarrow H^{B,B}}\rightarrow f^{\uparrow G^{B,\bullet}}\bef p\bef f^{\downarrow H^{\bullet,B}}\quad.
\]
So, the wedge law~(\ref{eq:wedge-law-for-profunctors}) gives:
\begin{align*}
{\color{greenunder}\text{left-hand side}:}\quad & p^{A}\triangleright f^{\uparrow P^{A,\bullet}}=f^{\downarrow G^{\bullet,A}}\bef p^{A}\bef f^{\uparrow H^{A,\bullet}}\quad,\\
{\color{greenunder}\text{right-hand side}:}\quad & p^{B}\triangleright f^{\downarrow P^{\bullet,B}}=f^{\uparrow G^{B,\bullet}}\bef p^{B}\bef f^{\downarrow H^{\bullet,B}}\quad.
\end{align*}
We obtain $f^{\downarrow G^{\bullet,A}}\bef p^{A}\bef f^{\uparrow H^{A,\bullet}}=f^{\uparrow G^{B,\bullet}}\bef p^{B}\bef f^{\downarrow H^{\bullet,B}}$,
which is the same as Eq.~(\ref{eq:dinaturality-law-for-profunctors}).
$\square$

\subsection{Uniqueness of functor and contrafunctor liftings\label{sec:Uniqueness-of-functor-and-contrafunctor}}

Naturality and dinaturality laws use function liftings such as $f^{\uparrow G}$
or $f^{\uparrow P^{A,\bullet}}$. How are these liftings defined?
Sections~\ref{subsec:f-Functor-constructions} and~\ref{subsec:f-Contrafunctor-constructions}
derive lawful and fully parametric implementations of the \lstinline!fmap!
and \lstinline!cmap! methods for all functors and contrafunctors
built up from the six type constructions. The naturality laws involved
in the parametricity theorem must use precisely those \textsf{``}standard\textsf{''}
implementations of \lstinline!fmap! and \lstinline!cmap!, because
the proof of the parametricity theorem significantly depends on the
code of those standard implementations.

The structure of a given fully parametric type constructor $F^{A}$
dictates a unique implementation of a lifting $f^{\uparrow P}$ or
$f^{\downarrow P}$. Let us summarize these implementations for the
six type constructions for the case when $F^{A}$ is covariant in
$A$:

\paragraph{Constant type}

If $F^{A}\triangleq Z$ where $Z$ is a fixed type then $f^{\uparrow F}=\text{id}^{:Z\rightarrow Z}$.

\paragraph{Type parameter}

If $F^{A}\triangleq A$ then $f^{\uparrow F}=f$. If $F^{A}\triangleq G^{H^{A}}$
then $f^{\uparrow F}=(f^{\uparrow H})^{\uparrow G}$ if both $G$
and $H$ are functors and $f^{\uparrow F}=(f^{\downarrow H})^{\downarrow G}$
if both $G$ and $H$ are contrafunctors.

\paragraph{Products}

If $F^{A}\triangleq G^{A}\times H^{A}$ then $f^{\uparrow F}=f^{\uparrow G}\boxtimes f^{\uparrow H}$.

\paragraph{Co-products}

If $F^{A}\triangleq G^{A}+H^{A}$ then $f^{\uparrow F}=f^{\uparrow G}\boxplus f^{\uparrow H}$.

\paragraph{Function types}

If $F^{A}\triangleq G^{A}\rightarrow H^{A}$ then $f^{\uparrow F}=p^{:G^{A}\rightarrow H^{A}}\rightarrow f^{\downarrow G}\bef p\bef f^{\uparrow H}$.

\paragraph{Recursive type}

If $F^{A}\triangleq S^{A,F^{A}}$ then $f^{\uparrow F}=f^{\uparrow S^{\bullet,F^{A}}}\bef\big(\overline{f^{\uparrow F}}\big)^{\uparrow S^{B,\bullet}}$.
Here $\overline{f^{\uparrow F}}$ is a recursive call to $f^{\uparrow F}$.

In addition, we will need a seventh type construction: the universally
quantified\index{types!universally quantified} type.

\paragraph{Quantified type}

If $F^{A}\triangleq\forall X.\,P^{X,A}$ where $P^{X,A}$ is covariant
in $A$ then: 
\[
f^{\uparrow F}=\forall Y.\,p^{:\forall X.\,P^{X,A}}\rightarrow p^{Y}\triangleright f^{\uparrow P^{Y,\bullet}}\quad.
\]

It turns out that there no other lawful implementations of \lstinline!fmap!
for these type constructions, if we assume that the naturality laws
hold:

\subsubsection{Statement \label{subsec:Statement-functor-is-unique}\ref{subsec:Statement-functor-is-unique}}

There is only one implementation of a given functor $F$\textsf{'}s \lstinline!fmap!
method that satisfies the identity, composition, and naturality laws.
Here, the naturality laws are formulated with the standard implementations
of the liftings.

\subparagraph{Proof}

Section~\ref{subsec:f-Functor-constructions} derived the standard
implementations of the \lstinline!fmap! method for all functors $F$
built up via the six type constructions. Throughout this book, this
standard lifting code is denoted by $\text{fmap}_{F}(f)$ and equivalently
by $f^{\uparrow F}$. Now suppose that there exists \emph{another}
lawful implementation of \lstinline!fmap! for $F$, denoted by $\text{fmap}_{F}^{\prime}(f)$:
\[
\text{fmap}_{F}^{\prime}:\left(A\rightarrow B\right)\rightarrow F^{A}\rightarrow F^{B}\quad,\quad\quad\text{fmap}_{F}^{\prime}(f^{:A\rightarrow B})=\text{???}^{:F^{A}\rightarrow F^{B}}\quad.
\]
We will now show that $\text{fmap}_{F}^{\prime}=\text{fmap}_{F}$.
Let us fix the type parameter $A$ and apply the naturality law to
$\text{fmap}_{F}^{\prime}$ with respect to $B$. The resulting law
involves an arbitrary $g^{:B\rightarrow C}$:
\[
\text{fmap}_{F}^{\prime}(f^{:A\rightarrow B}\bef g^{:B\rightarrow C})\overset{!}{=}\text{fmap}_{F}^{\prime}(f)\bef g^{\uparrow F}\quad.
\]
In the naturality law, the lifting $g^{\uparrow F}$ means the \emph{standard}
lifting code: $g^{\uparrow F}\triangleq\text{fmap}_{F}(g)$. By assumption,
$\text{fmap}_{F}^{\prime}$ obeys the composition law, so we may write:
\[
\text{fmap}_{F}^{\prime}(f\bef g)=\text{fmap}_{F}^{\prime}(f)\bef\text{fmap}_{F}^{\prime}(g)\overset{!}{=}\text{fmap}_{F}^{\prime}(f)\bef g^{\uparrow F}\quad.
\]
Since $f^{:A\rightarrow B}$ is arbitrary, we can choose $A=B$ and
$f=\text{id}^{:B\rightarrow B}$ to obtain:
\[
\text{fmap}_{F}^{\prime}(\text{id})\bef\text{fmap}_{F}^{\prime}(g)\overset{!}{=}\text{fmap}_{F}^{\prime}(\text{id})\bef g^{\uparrow F}\quad.
\]
The identity law for $\text{fmap}_{F}^{\prime}$ gives $\text{fmap}_{F}^{\prime}(\text{id})=\text{id}$,
so we can simplify the last equation to:
\[
\text{fmap}_{F}^{\prime}(g)\overset{!}{=}g^{\uparrow F}=\text{fmap}_{F}(g)\quad.
\]
This must hold for arbitrary $g^{:B\rightarrow C}$, which proves
that $\text{fmap}_{F}^{\prime}=\text{fmap}_{F}$.

\subsubsection{Statement \label{subsec:Statement-contrafunctor-is-unique}\ref{subsec:Statement-contrafunctor-is-unique}}

Any contrafunctor $H$ has a unique implementation of a lawful \lstinline!cmap!
method. 

\subparagraph{Proof}

We use similar arguments as in the proof of Statement~\ref{subsec:Statement-functor-is-unique}.
For any lawful alternative implementation $\text{cmap}_{H}^{\prime}$,
the naturality law is:
\[
\text{cmap}_{H}^{\prime}(f^{:A\rightarrow B}\bef g^{:B\rightarrow C})\overset{!}{=}(g^{:B\rightarrow C})^{\downarrow H}\bef\text{cmap}_{H}^{\prime}(f)\quad.
\]
By assumption, the identity and composition law hold for $\text{cmap}_{H}^{\prime}$.
Setting $f=\text{id}^{:B\rightarrow B}$, we get:
\[
\text{cmap}_{H}^{\prime}(\text{id}\bef g)=\text{cmap}_{H}^{\prime}(g)\overset{!}{=}g^{\downarrow H}\bef\text{cmap}_{H}^{\prime}(\text{id})=g^{\downarrow H}\quad.
\]
This must hold for arbitrary $g^{:B\rightarrow C}$, which shows that
$\text{cmap}_{H}^{\prime}(g)=g^{\downarrow H}=\text{cmap}_{H}(g)$
as required. $\square$

We just proved that lawful implementations of \lstinline!fmap! and
\lstinline!cmap! are unique when they obey the naturality laws. However,
this does not yet prove that any implementation of \lstinline!fmap!
and \lstinline!cmap! satisfying only the functor laws must also satisfy
naturality laws or be fully parametric (Problem~\ref{subsec:Problem-unique-functor-liftings}).

\subsection{Commutativity laws for bifunctors and profunctors\label{sec:Commutativity-laws-for-type-constructors}}

A special property that holds as a consequence of parametricity is
the bifunctor commutativity law\index{commutativity law!of bifunctors}~(\ref{eq:f-fmap-fmap-bifunctor-commutativity})
introduced in Section~\ref{subsec:Bifunctors}. If a type constructor
$P^{A,B}$ is a functor with respect to $A$ and $B$ separately then
the liftings with respect to the two type parameters will commute:

\begin{wrapfigure}{l}{0.28\columnwidth}%
\vspace{-1.9\baselineskip}
\[
\xymatrix{\xyScaleY{2.5pc}\xyScaleX{3.0pc}P^{A,B}\ar[r]\sp(0.5){f^{\uparrow P^{\bullet,B}}}\ar[d]\sp(0.45){g^{\uparrow P^{A,\bullet}}} & P^{C,B}\ar[d]\sb(0.45){g^{\uparrow P^{C,\bullet}}}\\
P^{A,D}\ar[r]\sp(0.5){f^{\uparrow P^{\bullet,D}}} & P^{C,D}
}
\]

\vspace{-1\baselineskip}
\end{wrapfigure}%

~\vspace{-0.5\baselineskip}
\[
\text{for all }A,B,C,D,f^{:A\rightarrow C},g^{:B\rightarrow D}\quad:\quad f^{\uparrow P^{\bullet,B}}\bef g^{\uparrow P^{C,\bullet}}=g^{\uparrow P^{A,\bullet}}\bef f^{\uparrow P^{\bullet,D}}\quad.
\]

Similar properties hold for type constructors with any number of type
parameters that are either covariant or contravariant with respect
to each of those type parameters.

We will now prove that the commutativity law holds for any type constructors
with two type parameters, assuming that their lifting methods satisfy
\emph{naturality} laws with respect to each type parameter separately.\footnote{B.~Yorgey showed a proof of the commutativity law for bifunctors
based on the Reynolds-Wadler relational parametricity theorem, see
the blog post \texttt{\href{https://byorgey.wordpress.com/2018/03/30/}{https://byorgey.wordpress.com/2018/03/30/}}\index{Brent Yorgey} } As we will show later (without using these commutativity laws), naturality
laws follow from parametricity. So, the commutativity laws hold for
any type constructors with two type parameters as long as the lifting
code is fully parametric.

\subsubsection{Statement \label{subsec:Proof-of-the-profunctor-commutativity-law}\ref{subsec:Proof-of-the-profunctor-commutativity-law}}

Assume a profunctor $P^{A,B}$ contravariant with respect to $A$
and covariant with respect to $B$, such that $\text{cmap}_{P^{\bullet,B}}$
obeys the naturality law with respect to the type parameter $B$.
Then the profunctor \textbf{commutativity law}\index{commutativity law!of profunctors}\index{profunctor!commutativity law}
holds:
\begin{align}
{\color{greenunder}\text{commutativity law of }P:}\quad & \text{cmap}_{P^{\bullet,B}}(f^{:A\rightarrow C})\bef\text{fmap}_{P^{A,\bullet}}(g^{:B\rightarrow D})=\text{fmap}_{P^{C,\bullet}}(g)\bef\text{cmap}_{P^{\bullet,D}}(f)\quad,\nonumber \\
{\color{greenunder}\text{in a shorter notation}:}\quad & f^{\downarrow P^{\bullet,B}}\bef g^{\uparrow P^{A,\bullet}}=g^{\uparrow P^{C,\bullet}}\bef f^{\downarrow P^{\bullet,D}}\quad.\label{eq:profunctor-commutativity-law}
\end{align}

\begin{wrapfigure}{i}{0.37\columnwidth}%
\vspace{-1.7\baselineskip}
\[
\xymatrix{\xyScaleY{2.5pc}\xyScaleX{6.0pc}P^{C,B}\ar[r]\sp(0.55){\text{cmap}_{P^{\bullet,B}}(f^{:A\rightarrow C})~~~}\ar[d]\sb(0.45){\text{fmap}_{P^{C,\bullet}}(g)} & P^{A,B}\ar[d]\sb(0.45){\text{fmap}_{P^{A,\bullet}}(g^{:B\rightarrow D})}\\
P^{C,D}\ar[r]\sb(0.45){~~~~\text{cmap}_{P^{\bullet,D}}(f)} & P^{A,D}
}
\]

\vspace{-1.7\baselineskip}
\end{wrapfigure}%


\subparagraph{Proof}

The type signature of the \lstinline!cmap! method is:
\[
\text{cmap}_{P^{\bullet,B}}:(A\rightarrow C)\rightarrow P^{C,B}\rightarrow P^{A,B}\quad.
\]
If we fix \lstinline!cmap!\textsf{'}s argument as an arbitrary function $f^{:A\rightarrow C}$
and hold the type parameters $A$ and $C$ fixed, the value $\text{cmap}_{P^{\bullet,B}}(f)$
has the type of a natural transformation between two functors covariant
in $B$. By assumption, \lstinline!cmap! obeys the naturality law
with respect to the parameter $B$. So, the following law holds:
\[
\text{for all }g^{:B\rightarrow D}:\quad\text{cmap}_{P^{\bullet,B}}(f)\bef g^{\uparrow P^{A,\bullet}}\overset{!}{=}g^{\uparrow P^{C,\bullet}}\bef\text{cmap}_{P^{\bullet,D}}(f)\quad.
\]
This equation is exactly the same as the commutativity law~(\ref{eq:profunctor-commutativity-law})
that we need to prove. $\square$

Similarly, we can prove the commutativity laws for bifunctors and
bi-contrafunctors.

\subsubsection{Statement \label{subsec:Proofs-of-commutativity-for-bifunctor}\ref{subsec:Proofs-of-commutativity-for-bifunctor}}

Any bifunctor or bi-contrafunctor satisfies its commutativity law
as long as its lifting methods satisfy the naturality laws. A \textbf{bi-contrafunctor}
$P^{A,B}$ is a type constructor contravariant with respect to both
$A$ and $B$. The commutativity law for bi-contrafunctors is written
as:
\begin{align}
{\color{greenunder}\text{commutativity law of }P:}\quad & \text{cmap}_{P^{\bullet,D}}(f^{:A\rightarrow C})\bef\text{cmap}_{P^{A,\bullet}}(g^{:B\rightarrow D})=\text{cmap}_{P^{C,\bullet}}(g)\bef\text{cmap}_{P^{\bullet,B}}(f)\quad.\nonumber \\
{\color{greenunder}\text{shorter notation}:}\quad & f^{\downarrow P^{\bullet,D}}\bef g^{\downarrow P^{A,\bullet}}=g^{\downarrow P^{C,\bullet}}\bef f^{\downarrow P^{\bullet,B}}\quad.\label{eq:bi-contrafunctor-commutativity-law}
\end{align}

\begin{wrapfigure}{l}{0.37\columnwidth}%
\vspace{-1.7\baselineskip}
\[
\xymatrix{\xyScaleY{2.5pc}\xyScaleX{6.0pc}P^{C,D}\ar[r]\sp(0.55){\text{cmap}_{P^{\bullet,D}}(f)~~~}\ar[d]\sb(0.45){\text{cmap}_{P^{C,\bullet}}(g)} & P^{A,D}\ar[d]\sb(0.45){\text{cmap}_{P^{A,\bullet}}(g)}\\
P^{C,B}\ar[r]\sb(0.45){~~~~\text{cmap}_{P^{\bullet,B}}(f)} & P^{A,B}
}
\]

\vspace{-1.7\baselineskip}
\end{wrapfigure}%


\subparagraph{Proof}

The commutativity law of bifunctors is Eq.~(\ref{eq:f-fmap-fmap-bifunctor-commutativity}),
written more briefly as:
\[
\text{fmap}_{F^{\bullet,B}}(f^{:A\rightarrow C})\bef(g^{:B\rightarrow D})^{\uparrow F^{C,\bullet}}=g^{\uparrow F^{A,\bullet}}\bef\text{fmap}_{F^{\bullet,D}}(f)\quad.
\]
This equation is the same as the naturality law of $\text{fmap}_{F^{\bullet,B}}$
with respect to the type parameter $B$, where the function $f^{:A\rightarrow C}$
is fixed but $g^{:B\rightarrow D}$ is arbitrary. That naturality
law holds by assumption.

The commutativity law~(\ref{eq:bi-contrafunctor-commutativity-law})
of bi-contrafunctors is the same as the naturality law of $\text{cmap}_{P^{A,\bullet}}$
with respect to the type parameter $A$. That naturality law holds
by assumption.$\square$

The same techniques and proofs apply to type constructors with more
than two type parameters. Liftings with respect to separate type
parameters always commute.

\section{Relational formulation of parametricity\label{sec:Parametricity-theorem-for-relations}\label{subsec:Relations-between-types}}

Naturality laws are formulated using arbitrary functions $f^{:A\rightarrow B}$
between arbitrary types $A$ and $B$. Typically, a naturality law
is an equation that involves the function $f$ lifted to some functors.
For instance, $f^{\uparrow G}\bef t=t\bef f^{\uparrow H}$ is the
naturality law of a natural transformation $t:\forall A.\,G^{A}\rightarrow H^{A}$,
where $G$ and $H$ are some functors.

To prove that naturality laws hold for any fully parametric function
$t$, we need to use induction in the structure of the code of $t$.
The proof will decompose $t$ into smaller sub-expressions for which
the naturality law should hold by the inductive assumption. Some of
those sub-expressions will have types that are no longer of the form
$\forall A.\,G^{A}\rightarrow H^{A}$. So, we need to generalize naturality
laws to type signatures of the form $\forall A.\,P^{A,A}$, where
$P^{X,Y}$ is a profunctor (contravariant in $X$ and covariant in
$Y$). 

Generalizing the naturality law to profunctors is not easy because
we cannot lift an arbitrary function $f^{:A\rightarrow B}$ to a function
of type $P^{A,A}\rightarrow P^{B,B}$. To resolve this difficulty,
the Reynolds-Wadler approach replaces functions $f^{:A\rightarrow B}$
by arbitrary many-to-many relations between values of types $A$ and
$B$. The type of those relations is denoted by $A\leftrightarrow B$.
It turns out that any relation $r$ of type $A\leftrightarrow B$
\emph{can} be lifted to a relation (denoted by $r^{\updownarrow P}$)
of type $P^{A,A}\leftrightarrow P^{B,B}$. The lifting operation can
be defined for any exponential-polynomial profunctor $P^{\bullet,\bullet}$.
Using that operation, we will prove the \textsf{``}relational parametricity
theorem\textsf{''}: any fully parametric code expression (not necessarily
a function) of type $\forall A.\,P^{A,A}$ satisfies a specially formulated
relational naturality law. That law will then allow us to derive the
wedge law~(\ref{eq:wedge-law-for-profunctors}) and the ordinary
naturality laws.

\subsection{Relations between values of different types}

Programmers are familiar with \textsf{``}relations\textsf{''} as tables in relational
databases. A simple table has two columns with values of some fixed
types, say \lstinline!INT! and \lstinline!FLOAT!. As an example,
consider a table called \lstinline!R! containing this data:
\begin{center}
$R=$ %
\begin{tabular}{|c|c|}
\hline 
\lstinline!SAMPLE_COUNT: INT! & \lstinline!MEAN_VALUE: FLOAT!\tabularnewline
\hline 
\hline 
{\footnotesize{}150} & {\footnotesize{}0.92}\tabularnewline
\hline 
{\footnotesize{}150} & {\footnotesize{}0.95}\tabularnewline
\hline 
{\footnotesize{}180} & {\footnotesize{}0.95}\tabularnewline
\hline 
{\footnotesize{}180} & {\footnotesize{}1.02}\tabularnewline
\hline 
{\footnotesize{}200} & {\footnotesize{}0.95}\tabularnewline
\hline 
\end{tabular}
\par\end{center}

Each row of the table \lstinline!R! has a value of type \lstinline!INT!
and a value of type \lstinline!FLOAT! (no \lstinline!NULL! values
are allowed). The existence of a row $\left(180,0.95\right)$ means
that the values $180$ and $0.95$ are \textsf{``}in the relation\textsf{''} \lstinline!R!.
More generally, two values $a^{:\text{Int}}$ and $b^{:\text{Float}}$
are in the relation \lstinline!R! only if there exists a row $\left(a,b\right)$
in the table \lstinline!R!. All other pairs of values $\left(a,b\right)$
are \emph{not} in the relation \lstinline!R!. 

The table \lstinline!R! contains several values in the first column
that correspond to the same value in the second column, and vice versa.
So, the relation \lstinline!R! is many-to-many and cannot be represented
by a function of type \lstinline!Int => Float! or of type \lstinline!Float => Int!.
Instead, we must view the relation \lstinline!R! as a subset of the
set of \emph{all} possible pairs $(a^{:\text{Int}},b^{:\text{Float}})$.
The table \lstinline!R! lists the pairs that belong to that subset.

Instead of listing the pairs, we may describe a relation by implementing
a function telling us whether two given values are in the relation.
The Scala type signature for such a function could be:
\begin{lstlisting}
def inRelationR(a: Int, b: Float): Boolean
\end{lstlisting}
In the short code notation, this function is written as $r:\text{Int}\times\text{Float}\rightarrow\bbnum 2$.
The value $r(a\times b)$ is \lstinline!true! if and only if the
two given values $a$ and $b$ are in the relation $r$.

For proving the parametricity theorem, we will need relations between
values of arbitrary types. Replacing \lstinline!Int! and \lstinline!Float!
by type parameters $A$ and $B$, we get the following definition:

\subsubsection{Definition \label{subsec:Definition-relation-between-A-B}\ref{subsec:Definition-relation-between-A-B}}

A \textbf{relation} of\index{parametricity theorem!relation between values}\index{relation between values!see \textsf{``}value relation\textsf{''}}\index{value relation}
type $A\leftrightarrow B$ is a function $r:A\times B\rightarrow\bbnum 2$.
The type notation $A\leftrightarrow B$ indicates that relations are
more general than functions of types $A\rightarrow B$ or $B\rightarrow A$.
If values $x$ and $y$ are in a relation $r$, we write $(x,y)\in r$
or with full type annotations: $(x^{:A},y^{:B})\in r^{:A\leftrightarrow B}$.
The condition $(x,y)\in r$ is equivalent to $r(x,y)=\text{true}$.
$\square$

Defined in this sense, relations hold between values of given types.
Certain values $x^{:A}$ and $y^{:B}$ will be in a given relation
$r^{:A\leftrightarrow B}$, while other values of the same types will
not be in the relation $r$. This should not be confused with \index{type relation}\emph{type
relations} described in Section~\ref{subsec:Typeclasses-type-relations},
where certain \emph{types} will be in a given type relation while
other types will not be in that relation. We will not use any type
relations in this Appendix, so we will write simply \textsf{``}a relation
between $A$ and $B$\textsf{''} to mean \textsf{``}a relation between \emph{values}
of types $A$ and $B$\textsf{''}.

A simple example of a relation is the \textbf{identity relation}\index{identity relation},
denoted by $\text{id}^{:A\leftrightarrow A}$. The identity relation
holds only when two values of type $A$ are equal:
\[
(x^{:A},y^{:A})\in\text{id}^{:A\leftrightarrow A}\quad\text{ means }\quad x=y\quad.
\]

To use the parametricity theorem in practice, we will need to convert
functions into relations. A function $f^{:A\rightarrow B}$ generates
the \index{function graph relation}\textbf{function graph} \textbf{relation}
denoted by $\left<f\right>$ and defined as:
\[
(a^{:A},b^{:B})\in\left<f\right>\text{ means }f(a)=b\quad\text{or equivalently}:\quad a\triangleright f=b\quad.
\]
Function graph relations $\left<f\right>$ are \emph{many-to-one}
relations because one or more values of type $A$ are related to a
single value of type $B$.

Given a relation $r^{:A\leftrightarrow B}$, we can swap $x^{:A}$
and $y^{:B}$ in the condition $(x,y)\in r$ and obtain the \index{reverse relation}\textbf{reverse
relation} of type $B\leftrightarrow A$, denoted $\text{rev}\left(r\right)$,
such that $(y,x)\in\text{rev}\left(r\right)$. For a function $f^{:A\rightarrow B}$,
the reverse function graph relation $\text{rev}\left<f\right>$ has
type $B\leftrightarrow A$ and is a one-to-many relation. 

The operation \lstinline!rev! it is its own inverse, and so it gives
an isomorphism between $A\leftrightarrow B$ and $B\leftrightarrow A$:
\[
\text{rev}\left(\text{rev}\left(r\right)\right)=r\quad.
\]
To prove this, substitute the definition of \lstinline!rev! twice:
\[
(x^{:A},y^{:B})\in\text{rev}\,\big(\text{rev}\,(r)\big)\quad\text{is the same as}:\quad(y,x)\in\text{rev}\,(r)\quad\text{and is the same as}:\quad(x,y)\in r\quad.
\]

Here is an example of a relation $r^{:A\leftrightarrow B}$ that is
\emph{not} equivalent to a function graph:
\[
(x^{:A},y^{:B})\in r\text{ means }p(x)=q(y)\text{ where }p^{:A\rightarrow C}\text{ and }q^{:B\rightarrow C}\text{ are some given functions}\quad.
\]
We call this a \textbf{pullback relation}\index{pullback relation|textit}
and denote it by $\text{pull}\,(p,q)$. For some choices of $p$ and
$q$, there will be many different values $x_{1}$, $x_{2}$, $y_{1}$,
$y_{2}$, ..., such that $p(x_{1})=p(x_{2})=q(y_{1})=q(y_{2})=...$
Then the pullback relation will be many-to-many and will not be equivalent
to $\left<f\right>$ or to $\text{rev}\left<f\right>$ for any function
$f$. 

In the derivations below, we will often use relations of type $P^{A,A}\leftrightarrow P^{B,B}$,
where $P^{X,Y}$ is a profunctor (contravariant in $X$ and covariant
in $Y$). For any function $f^{:A\rightarrow B}$, there is a special
pullback relation of type $P^{A,A}\leftrightarrow P^{B,B}$: 

\subsubsection{Definition \label{subsec:Definition-wedge-relation}\ref{subsec:Definition-wedge-relation}}

Given a profunctor\index{profunctor!wedge relation}\index{wedge relation|textit}
$P^{X,Y}$ and a function $f^{:A\rightarrow B}$, two values $x^{:P^{A,A}}$
and $y^{:P^{B,B}}$ are \textbf{in a} $\left(P,f\right)$-\textbf{wedge
relation} if $x\triangleright f^{\uparrow P^{A,\bullet}}=y\triangleright f^{\downarrow P^{\bullet,B}}$.
This condition can be written more concisely as $(x,y)\in\text{pull}\,(f^{\uparrow P},f^{\downarrow P})$.

The wedge relation comes up often in applications of the relational
parametricity theorem. The $\left(P,f\right)$-wedge relation generalizes
the wedge law~(\ref{eq:wedge-law-for-profunctors}) to two arbitrary
values $x$ and $y$. The wedge law of an expression $p:\forall A.\,P^{A,A}$
is equivalent to the requirement that $p^{A}$ and $p^{B}$ are in
the $\left(P,f\right)$-wedge relation for all $f^{:A\rightarrow B}$
and for all types $A$, $B$. 

\subsection{Relational product, co-product, and pair mapper. Relational lifting}

To motivate the formulation of the relational naturality law, we will
begin by rewriting ordinary naturality laws involving functions, such
as Eq.~(\ref{eq:naturality-law-for-functors}), in terms of relations.
A naturality law involves lifting an arbitrary function $f^{:A\rightarrow B}$
to some functors or contrafunctors. Lifting a function $f^{:A\rightarrow B}$
to a functor $G$ yields a function $f^{\uparrow G}$ of type $G^{A}\rightarrow G^{B}$.
Lifting $f$ to a contrafunctor $H$ yields a function $f^{\downarrow H}:H^{B}\rightarrow H^{A}$.
Below we will define a lifting of an arbitrary relation $r^{:A\leftrightarrow B}$
to an arbitrary type constructor $G^{\bullet}$. The lifted relation
will have type $G^{A}\leftrightarrow G^{B}$ and will be denoted by
$r^{\updownarrow G}$.

Now consider a natural transformation $t:\forall A.\,G^{A}\rightarrow H^{A}$
with its naturality law $f^{\uparrow G}\bef t=t\bef f^{\uparrow H}$.
To obtain a relational formulation of that law, we first apply both
sides to an arbitrary value $p^{:G^{A}}$:
\[
p\triangleright f^{\uparrow G}\triangleright t^{B}=p\triangleright t^{A}\triangleright f^{\uparrow H}\quad,\quad\quad\text{or equivalently}:\quad t^{B}(p\triangleright f^{\uparrow G})=t^{A}(p)\triangleright f^{\uparrow H}\quad,
\]
and denote by $q$ the value $q^{:G^{B}}\triangleq p\triangleright f^{\uparrow G}$.
This definition of $q$ is equivalent to the condition $(p,q)\in\langle f^{\uparrow G}\rangle$.
Now we can rewrite the naturality law of $t$ via relations: 
\begin{equation}
\forall f^{:A\rightarrow B},p^{:G^{A}},q^{:G^{B}}\quad:\quad\quad\text{when}\quad(p,q)\in\langle f^{\uparrow G}\rangle\quad\quad\text{then}\quad(t^{A}(p),t^{B}(q))\in\langle f^{\uparrow H}\rangle\quad.\label{eq:naturality-law-of-t-derivation1}
\end{equation}

We will define a concise notation for conditions of this form because
they are used often when working with relations.

\subsubsection{Definition \label{subsec:Definition-pair-mapper-of-relations}\ref{subsec:Definition-pair-mapper-of-relations}}

\index{pair mapper of relations}The \textbf{pair mapper of relations}
$r^{:A\leftrightarrow C}$ and $s^{:B\leftrightarrow D}$ is a relation
$r\varogreaterthan s$ defined by:
\begin{align*}
 & \big(r^{:A\leftrightarrow C}\varogreaterthan s^{:B\leftrightarrow D}\big):\left(A\rightarrow B\right)\leftrightarrow\left(C\rightarrow D\right)\quad,\\
 & (f^{:A\rightarrow B},g^{:C\rightarrow D})\in r\varogreaterthan s\quad\text{means}\quad\quad\forall x^{:A},y^{:C}\quad:\quad\text{if }(x,y)\in r\quad\text{then}\quad(f(x),g(y))\in s\quad.
\end{align*}

\begin{wrapfigure}{i}{0.2\columnwidth}%
\vspace{-2.8\baselineskip}
\[
\xymatrix{\xyScaleY{1.8pc}\xyScaleX{2.2pc}A\ar@{<->}[d]\sp(0.5){r}\ar[rr]\sp(0.5){f} & \ar@{<->}[d(0.9)]\sp(0.55){r\varogreaterthan s} & B\ar@{<->}[d]\sp(0.5){s}\\
C\ar[rr]\sb(0.5){g} &  & D
}
\]

\vspace{-2.5\baselineskip}
\end{wrapfigure}%

\noindent The operation $\ogreaterthan$ associates to the right,
so that $r\ogreaterthan s\ogreaterthan t\triangleq r\ogreaterthan\left(s\ogreaterthan t\right)$.
The type diagram at left illustrates the pair mapper construction.
To read the diagram, we start with two values, $x^{:A}$ at the top
left and $y^{:C}$ at the bottom left, and two relations, $r^{:A\leftrightarrow C}$
and $s^{:B\leftrightarrow D}$. The values $x$ and $y$ are arbitrary
as long as $(x,y)\in r$. The new relation $r\ogreaterthan s$ is
between functions of types $A\rightarrow B$ and $C\rightarrow D$.
Two functions $\left(f,g\right)$ are in the relation $r\varogreaterthan s$
if the diagram commutes, which means that $(f(x),g(y))\in s$. $\square$

Using Definition~\ref{subsec:Definition-pair-mapper-of-relations},
we can now rewrite the naturality law~(\ref{eq:naturality-law-of-t-derivation1})
as:
\[
\text{for any }f^{:A\rightarrow B}\quad:\quad(t^{A},t^{B})\in\langle f^{\uparrow G}\rangle\ogreaterthan\langle f^{\uparrow H}\rangle\quad.
\]

It appears reasonable to expect that the relation $\left<f\right>$
lifted to $G$ is the same relation as $\langle f^{\uparrow G}\rangle$:
\[
\left<f\right>^{\updownarrow G}=\langle f^{\uparrow G}\rangle\quad.
\]
With this property (proved below in Statement~\ref{subsec:Statement-lifting-function-relation-covariant}),
we rewrite Eq.~(\ref{eq:naturality-law-of-t-derivation1}) as:
\[
\text{for any }f^{:A\rightarrow B}:\quad(t^{A},t^{B})\in\left<f\right>^{\updownarrow G}\ogreaterthan\left<f\right>^{\updownarrow H}\quad.
\]
So far, this is just the ordinary naturality law written in terms
of function graph relation $\left<f\right>$. We now replace $\left<f\right>$
by an arbitrary relation $r^{:A\leftrightarrow B}$ and obtain the\index{naturality law!in terms of relations}
\textbf{relational naturality law}:
\begin{equation}
\text{for any relation }r^{:A\leftrightarrow B}:\quad(t^{A},t^{B})\in r^{\updownarrow G}\ogreaterthan r^{\updownarrow H}\quad.\label{eq:naturality-law-of-t-derivation2}
\end{equation}

Note the similarity of the relation $r^{\updownarrow G}\ogreaterthan r^{\updownarrow H}$
and the type $G^{A}\rightarrow H^{A}$ of $t$. Denote the type signature
of $t$ by $P^{A}\triangleq G^{A}\rightarrow H^{A}$. If we \emph{define}
the lifting $r^{\updownarrow P}$ as $r^{\updownarrow G}\ogreaterthan r^{\updownarrow H}$,
the law~(\ref{eq:naturality-law-of-t-derivation2}) becomes:
\begin{equation}
\text{for any relation }r^{:A\leftrightarrow B}:\quad(t^{A},t^{B})\in r^{\updownarrow P}\quad.\label{eq:naturality-law-of-t-derivation3}
\end{equation}

This generalization of the naturality law from functions to relations
now has a concise form: The components $t^{A}$ and $t^{B}$ of a
natural transformation $t$ belong to \emph{any} relation $r$ lifted
to $t$\textsf{'}s type signature (the type constructor $P^{\bullet}$). This
prepares us for the formulation of the parametricity theorem below.
At the same time, this motivates using the pair mapper operation ($\ogreaterthan$)
to define the lifting a relation to a function type constructor such
as $G^{A}\rightarrow H^{A}$.

To build intuition, let us prove a simple property of the pair mapper:

\subsubsection{Statement \label{subsec:Statement-pair-mapper-rev}\ref{subsec:Statement-pair-mapper-rev}}

For any two relations $r^{:A\leftrightarrow C}$ and $s^{:B\leftrightarrow D}$,
we have:
\[
\text{rev}\,(r\varogreaterthan s)=(\text{rev}\,(r))\varogreaterthan\text{rev}\,(s)\quad.
\]


\subparagraph{Proof}

Write the left-hand side in detail. For any functions $f^{:A\rightarrow B}$
and $g^{:C\rightarrow D}$, the relation $(g^{:C\rightarrow D},f^{:A\rightarrow B})\in\text{rev}\,(r\varogreaterthan s)$
means the same as $(f,g)\in r\varogreaterthan s$, or:
\[
\forall x^{:A},y^{:C}:\quad\text{when}\quad(x,y)\in r\quad\text{then}\quad(f(x),g(y))\in s\quad.
\]
The right-hand side: $(g^{:C\rightarrow D},f^{:A\rightarrow B})\in(\text{rev}\,(r))\varogreaterthan\text{rev}\,(s)$
means:
\[
\forall y^{:C},x^{:A}:\quad\text{when}\quad(y,x)\in\text{rev}\,(r)\quad\text{then}\quad(g(y),f(x))\in\text{rev}\,(s)\quad.
\]
The conditions for both sides are equivalent. $\square$ 

What about lifting a relation to product ($G^{A}\times H^{A}$) or
co-product ($G^{A}+H^{A}$) type constructors? Recall that lifting
a function to $G^{A}\times H^{A}$ involves the pair product of functions\index{pair product of functions}
($\boxtimes$):
\[
\big(f^{:A\rightarrow C}\boxtimes g^{:B\rightarrow D}\big):A\times B\rightarrow C\times D\quad,\quad\quad f\boxtimes g\triangleq a\times b\rightarrow f(a)\times g(b)\quad.
\]
Namely, the lifting of a function $k^{:A\rightarrow B}$ to $G^{A}\times H^{A}$
is defined by $k^{\uparrow(G\times H)}\triangleq k^{\uparrow G}\boxtimes k^{\uparrow H}$. 

The pair co-product of functions\index{pair co-product of functions}
is defined in Exercise~\ref{subsec:Exercise-reasoning-1-4-1}(b):
\begin{align*}
 & \big(f^{:A\rightarrow C}\boxplus g^{:B\rightarrow D}\big):A+B\rightarrow C+D\quad,\\
 & (f\boxplus g)(a^{:A}+\bbnum 0^{:B})\triangleq f(a)+\bbnum 0^{:D}\quad,\quad\quad(f\boxplus g)(\bbnum 0+b^{:B})\triangleq\bbnum 0^{:C}+g(b)\quad,\\
 & (k^{:A\rightarrow B})^{\uparrow(G+H)}\triangleq k^{\uparrow G}\boxplus k^{\uparrow H}\quad.
\end{align*}

We will now define the analogous constructions for relations.

\subsubsection{Definition \label{subsec:Definition-pair-product-of-relations}\ref{subsec:Definition-pair-product-of-relations}}

Given two relations $r^{:A\leftrightarrow C}$ and $s^{:B\leftrightarrow D}$,
the \textbf{pair product} \index{pair product of relations} $r\boxtimes s$
is a relation between values of the product types $A\times B$ and
$C\times D$:
\[
\big(r^{:A\leftrightarrow C}\boxtimes s^{:B\leftrightarrow D}\big):A\times B\leftrightarrow C\times D\quad,\quad\quad(a\times b,c\times d)\in r\boxtimes s\quad\text{when}\quad(a,c)\in r\text{ and }(b,d)\in s\quad.
\]

The second construction is the pair co-product of relations\index{pair co-product of relations},
which creates a relation between co-products of types. The definition
resembles that of the \index{pair co-product of functions}pair co-product
of functions:

\subsubsection{Definition \label{subsec:Definition-pair-co-product-of-relations}\ref{subsec:Definition-pair-co-product-of-relations}}

The \textbf{pair co-product} of relations $r^{:A\leftrightarrow C}$
and $s^{:B\leftrightarrow D}$ is a relation $r\boxplus s$ defined
by:
\begin{align*}
 & \big(r^{:A\leftrightarrow C}\boxplus s^{:B\leftrightarrow D}\big):A+B\leftrightarrow C+D\quad,\\
 & \text{either}\quad\quad(a^{:A}+\bbnum 0^{:B},c^{:C}+\bbnum 0^{:D})\in r\boxplus s\quad\text{when}\quad(a,c)\in r\quad,\\
 & \text{or}\quad\quad(\bbnum 0^{:A}+b^{:B},\bbnum 0^{:C}+d^{:D})\in r\boxplus s\quad\text{when}\quad(b,d)\in s\quad.
\end{align*}
Any other combinations of values (such as $a+\bbnum 0$ and $\bbnum 0+d$)
are not in the relation $r\boxplus s$. $\square$

To build intuition, let us prove some properties of the relational
operations $\boxtimes$, $\boxplus$, and $\ogreaterthan$ when applied
to function graph relations.

\subsubsection{Example \label{subsec:Example-pair-product-pair-mapper-relation}\ref{subsec:Example-pair-product-pair-mapper-relation}\index{solved examples}}

For arbitrary given functions $f^{:A\rightarrow B}$ and $g^{:C\rightarrow D}$,
show that:

\textbf{(a)} $\left<f\right>\boxtimes\left<g\right>=\left<f\boxtimes g\right>$,
both relations having type $A\times C\leftrightarrow B\times D$.

\textbf{(b)} $\left<f\right>\boxplus\left<g\right>=\left<f\boxplus g\right>$,
both relations having type $A+C\leftrightarrow B+D$.

\textbf{(c)} $\left<f\right>\ogreaterthan\text{id}^{:C\leftrightarrow C}=\text{rev}\,\langle l^{:B\rightarrow C}\rightarrow f\bef l\rangle$,
both relations having type $\left(A\rightarrow C\right)\leftrightarrow\left(B\rightarrow C\right)$.

\textbf{(d)} $\text{id}^{:A\leftrightarrow A}\ogreaterthan\left<g\right>=\langle k^{:A\rightarrow C}\rightarrow k\bef g\rangle$,
both relations having type $\left(A\rightarrow C\right)\leftrightarrow\left(A\rightarrow D\right)$.

\textbf{(e)} $\left<f\right>\ogreaterthan\left<g\right>=\text{pull}\,(k^{:A\rightarrow C}\rightarrow k\bef g,\;l^{:B\rightarrow C}\rightarrow f\bef l)$,
both relations having type $\left(A\rightarrow C\right)\leftrightarrow\left(B\rightarrow D\right)$.

\textbf{(f)} $\left<f\right>\ogreaterthan\text{rev}\left<g\right>=\text{rev}\langle k^{:B\rightarrow C}\rightarrow f\bef k\bef g\rangle$,
both relations having type $\left(A\rightarrow D\right)\leftrightarrow\left(B\rightarrow C\right)$.

\textbf{(g)} $(\text{rev}\left<f\right>)\ogreaterthan\left<g\right>=\langle k^{:B\rightarrow C}\rightarrow f\bef k\bef g\rangle$,
both relations having type $\left(B\rightarrow C\right)\leftrightarrow\left(A\rightarrow D\right)$.

\subparagraph{Solution}

We will use various relations with arbitrary values $a^{:A}$, $b^{:B}$,
$c^{:C}$, $d^{:D}$.

\textbf{(a)} The following conditions are equivalent:
\begin{align*}
 & (a\times c,b\times d)\in\left<f\right>\boxtimes\left<g\right>\quad\text{means}\quad(a,b)\in\left<f\right>\quad\text{and}\quad(c,d)\in\left<g\right>\quad\text{or}:\quad f(a)=b\text{ and }g(c)=b\quad,\\
 & (a\times c,b\times d)\in\left<f\boxtimes g\right>\quad\text{means}\quad(a\times c)\triangleright(f\boxtimes g)=b\times d\quad\text{or}:\quad f(a)\times g(c)=b\times d\quad.
\end{align*}

\textbf{(b)} The following conditions are equivalent:
\begin{align*}
 & (a+\bbnum 0,b+\bbnum 0)\in\left<f\right>\boxplus\left<g\right>\quad\text{means}\quad(a,b)\in\left<f\right>\quad\text{or equivalently}:\quad a\triangleright f=b\quad,\\
 & (a+\bbnum 0,b+\bbnum 0)\in\left<f\boxplus g\right>\quad\text{means}\quad(a+\bbnum 0)\triangleright(f\boxplus g)=b+\bbnum 0\quad\text{or equivalently}:\quad a\triangleright f=b\quad.
\end{align*}
Similarly we can show that the conditions $(\bbnum 0+c,\bbnum 0+d)\in\left<f\right>\boxplus\left<g\right>$
and $(\bbnum 0+c,\bbnum 0+d)\in\left<f\boxplus g\right>$ are equivalent.
Values from different parts of the disjunctions, such as $(a+\bbnum 0,\bbnum 0+d)$,
are not in the relation $\left<f\right>\boxplus\left<g\right>$ by
definition of $\boxplus$. They are also not in the relation $\left<f\boxplus g\right>$
since the function $f\boxplus g$ preserves the left and right parts
of the disjunctions.

\textbf{(c)} Two functions $k^{:A\rightarrow C}$ and $l^{:B\rightarrow C}$
are in the relation $\left<f\right>\ogreaterthan\text{id}^{:C\leftrightarrow C}$
if:
\[
\text{for all }a^{:A},b^{:B}\quad:\quad\text{when}\quad(a,b)\in\left<f\right>\quad\text{then}\quad(k(a),l(b))\in\text{id}^{:C\leftrightarrow C}\quad.
\]
Simplifying the last condition, we get:
\[
\text{for all }a^{:A}\quad:\quad k(a)=l(f(a))\quad,\quad\text{or equivalently}:\quad k=f\bef l\quad.
\]
So, the relation between $k$ and $l$ may be expressed as $k=\psi(l)$
where the function $\psi$ is defined as:
\[
\psi:\left(B\rightarrow C\right)\rightarrow A\rightarrow C\quad,\quad\quad\psi\triangleq l^{:B\rightarrow C}\rightarrow f\bef l\quad.
\]
This relation is denoted by $\text{rev}\left<\psi\right>$. It follows
that $\left<f\right>\ogreaterthan\text{id}^{:C\leftrightarrow C}=\text{rev}\left<\psi\right>=\text{rev}\langle l^{:B\rightarrow C}\rightarrow f\bef l\rangle$.

\textbf{(d)} Two functions $k^{:A\rightarrow C}$ and $l^{:A\rightarrow D}$
are in the relation $\text{id}^{:A\leftrightarrow A}\ogreaterthan\left<g\right>$
if:
\[
\text{for all }x^{:A},y^{:A}\quad:\quad\text{when}\quad(x,y)\in\text{id}^{:A\leftrightarrow A}\quad\text{then}\quad(k(x),l(y))\in\left<g\right>\quad.
\]
Simplifying the last condition, we get:
\[
\text{for all }x^{:A}\quad:\quad g(k(x))=l(x)\quad,\quad\text{or equivalently}:\quad l=k\bef g\quad.
\]
So, the relation between $k$ and $l$ may be expressed as $\phi(k)=l$
where:
\[
\phi:\left(A\rightarrow C\right)\rightarrow A\rightarrow D\quad,\quad\phi\triangleq k^{:A\rightarrow C}\rightarrow k\bef g\quad.
\]
This relation is denoted by $\left<\phi\right>$. It follows that
we may write $\text{id}^{:A\leftrightarrow A}\ogreaterthan\left<g\right>=\left<\phi\right>=\langle k^{:A\rightarrow C}\rightarrow k\bef g\rangle$.

\textbf{(e)} Two functions $k^{:A\rightarrow C}$ and $l^{:B\rightarrow D}$
are in the relation $\left<f\right>\ogreaterthan\left<g\right>$ if:
\[
\text{for all }a^{:A},b^{:B}\quad:\quad\text{when}\quad(a,b)\in\left<f\right>\quad\text{then}\quad(k(a),l(b))\in\left<g\right>\quad.
\]
Simplifying the last condition, we get:
\[
\text{for all }a^{:A}\quad:\quad g(k(a))=l(f(a))\quad,\quad\text{or equivalently}:\quad f\bef l=k\bef g\quad.
\]
So, the relation between $k$ and $l$ may be expressed as $\phi(k)=\psi(l)$
where: 
\[
\phi:\left(A\rightarrow C\right)\rightarrow A\rightarrow D\quad,\quad\phi\triangleq k^{:A\rightarrow C}\rightarrow k\bef g\quad,\quad\psi:\left(B\rightarrow D\right)\rightarrow A\rightarrow D\quad,\quad\psi\triangleq l^{:B\rightarrow C}\rightarrow f\bef l\quad.
\]
 This is a pullback relation that we denote by $\text{pull}\,(\phi,\psi)=\text{pull}\,\big(k^{:A\rightarrow C}\rightarrow k\bef g,\;l^{:B\rightarrow C}\rightarrow f\bef l\big)$.

\textbf{(f)} Two functions $k^{:B\rightarrow C}$ and $l^{:A\rightarrow D}$
are in the relation $\left<f\right>\ogreaterthan\text{rev}\left<g\right>$
if:
\[
\text{for all }x^{:B},y^{:A}\quad:\quad\text{when}\quad(x,y)\in\left<f\right>\quad\text{then}\quad(k(x),l(y))\in\text{rev}\left<g\right>\quad.
\]
Simplifying the last line, we get:
\[
\text{for all }x^{:B}\quad:\quad g(l(f(x)))=k(x)\quad,\quad\text{or equivalently}:\quad f\bef l\bef g=k\quad.
\]
So, the relation between $k$ and $l$ may be expressed as $\phi(l)=k$
or as $(k,l)\in\text{rev}\left<\phi\right>$, where: 
\[
\phi:\left(B\rightarrow C\right)\rightarrow A\rightarrow D\quad,\quad\phi\triangleq k^{:B\rightarrow C}\rightarrow f\bef k\bef g\quad.
\]
So, we get: $\left<f\right>\ogreaterthan\text{rev}\left<g\right>=\text{rev}\left<\phi\right>=\text{rev}\langle k^{:B\rightarrow C}\rightarrow f\bef k\bef g\rangle$.

\textbf{(g)} Apply the reversing operation to both sides of item \textbf{(f)},
we get:
\[
\text{rev}\,(\left<f\right>\ogreaterthan\text{rev}\left<g\right>)=(\text{rev}\left<f\right>)\ogreaterthan\left<g\right>\overset{!}{=}\langle k^{:B\rightarrow C}\rightarrow f\bef k\bef g\rangle\quad.
\]
Here we have used Statement~\ref{subsec:Statement-pair-mapper-rev}
and the property $\text{rev}\left(\text{rev}\,(r)\right)=r$. $\square$

We now turn to defining the relational lifting $r^{\updownarrow G}$
for an arbitrary type constructor $G^{\bullet}$. It will turn out
that we actually need to define a more general operation: the \emph{simultaneous}
lifting of several relations to a type constructor with several type
parameters. For clarity, we postpone that definition and begin by
lifting a single relation.

\subsubsection{Definition \label{subsec:Definition-relational-lifting}\ref{subsec:Definition-relational-lifting}
(relational lifting)}

Given a relation $r^{:A\leftrightarrow B}$ and a fully parametric
type constructor $G^{\bullet}$, the relational lifting of $r$ to
$G$, denoted by $r^{\updownarrow G}$, is a new relation of type
$G^{A}\leftrightarrow G^{B}$. The relation $r^{\updownarrow G}$
is defined by induction on the structure of $G$ as shown below in
items \textbf{(a)}\textendash \textbf{(g)}. We will use arbitrary
type constructors $K^{A}$, $L^{A}$, $H^{X,A}$, and $S^{A,R}$ that
are assumed to be fully parametric but not necessarily covariant or
contravariant. 

\textbf{(a)} If $G^{A}\triangleq Z$ with a fixed type $Z$ (different
from $A$), we define $r^{\updownarrow G}\triangleq\text{id}^{:Z\leftrightarrow Z}$.

\textbf{(b)} If $G\triangleq\text{Id}$ (that is, $G^{A}\triangleq A$)
then we define $r^{\updownarrow\text{Id}}\triangleq r$. 

\textbf{(c)} If $G^{A}\triangleq K^{A}\times L^{A}$ then we define
$r^{\updownarrow G}\triangleq r^{\updownarrow K}\boxtimes r^{\updownarrow L}$.
The inductive assumptions are that the relational liftings $r^{\updownarrow K}$
and $r^{\updownarrow L}$ are already defined.

\textbf{(d)} If $G^{A}\triangleq K^{A}+L^{A}$ then we define $r^{\updownarrow G}\triangleq r^{\updownarrow K}\boxplus r^{\updownarrow L}$
with the same inductive assumptions.

\textbf{(e)} If $G^{A}\triangleq K^{A}\rightarrow L^{A}$ then we
define $r^{\updownarrow G}\triangleq r^{\updownarrow K}\ogreaterthan r^{\updownarrow L}$
with the same inductive assumptions.

\textbf{(f)} If $G^{A}\triangleq S^{A,G^{A}}$ is defined recursively
via a recursion scheme $S^{\bullet,\bullet}$, we define $r^{\updownarrow G}$
by:
\[
r^{\updownarrow G}\triangleq\big(r,\overline{r^{\uparrow G}}\big)^{\updownarrow S}\quad.
\]
Here the notation $(r,s)^{\updownarrow S}$ means the \emph{simultaneous}
lifting of the two relations $r$, $s$ to the type constructor $S^{\bullet,\bullet}$
(see Definition~\ref{subsec:Definition-simultaneous-relational-lifting}
below). The inductive assumption is that simultaneous liftings to
$S^{\bullet,\bullet}$ are already defined. Also note that we use
$\overline{r^{\updownarrow G}}$ recursively within the definition
of $r^{\updownarrow G}$. This is allowed since we understand $r^{\updownarrow G}$
to be a function (of type $G^{A}\times G^{B}\rightarrow\bbnum 2$,
see Definition~\ref{subsec:Definition-relation-between-A-B}), and
it is permitted to define functions recursively. 

\textbf{(g)} If $G^{A}\triangleq\forall X.\,H^{X,A}$, we define $r^{\updownarrow G}$
of type $(\forall X.\,H^{X,A})\leftrightarrow(\forall Y.\,H^{Y,B})$
by:
\[
(p^{:\forall X.\,H^{X,A}},q^{:\forall X.\,H^{X,B}})\in r^{\updownarrow\forall X.\,H^{X,\bullet}}\quad\text{means}\quad\forall(X,Y).\,\forall s^{X\leftrightarrow Y}.\,(p^{X},q^{Y})\in(s,r)^{\updownarrow H}\quad.
\]
Here $(s,r)^{\updownarrow H}$ denotes the simultaneous lifting of
$s$ and $r$ to $H$ (Definition~\ref{subsec:Definition-simultaneous-relational-lifting}
below). A shorter way of writing the formula above is by formulating
a relation between $p^{X}$ and $q^{Y}$:
\begin{equation}
\forall(X,Y):\quad\big(r^{\updownarrow\forall Z.\,H^{Z,\bullet}}\big)^{X,Y}\triangleq\forall s^{:X\leftrightarrow Y}.\,(s,r)^{\updownarrow H}\quad.\label{eq:relational-lifting-quantified-types-short}
\end{equation}
The inductive assumption is that the simultaneous lifting to $H$
is already defined. $\square$

We can see that Definition~\ref{subsec:Definition-relational-lifting}
translates the type structure of $G^{A}$ into an analogous relational
structure. Constant types ($Z$) are replaced by identity relations
($\text{id}^{:Z\leftrightarrow Z}$). Each occurrence of $A$ in $G^{A}$
is replaced by an occurrence of the relation $r$ being lifted. Products,
co-products, and function arrows are replaced by the relational operations
$\boxtimes$, $\boxplus$, and $\ogreaterthan$. Recursive usage of
types are replaced by the recursive usage of the lifted relations.
Universally quantified types ($\forall X$) are replaced by universally
quantified relations between new arbitrary types. 

So, an expression for the relational lifting $r^{\updownarrow G}$
can be written immediately by looking at the structure of $G$. For
instance, lifting a relation $r^{:A\leftrightarrow B}$ to a type
constructor $G^{A}\triangleq\forall X.\,\left(A\rightarrow X\right)\times\left(X\rightarrow A+Z\right)$
is written as:
\[
\big(r^{\updownarrow G}\big)^{X,Y}=\forall s^{:X\leftrightarrow Y}.\,(r\ogreaterthan s)\boxtimes\big(s\ogreaterthan(r\boxplus\text{id}^{:Z\leftrightarrow Z})\big)\quad.
\]
 We will illustrate this technique in Example~\ref{subsec:Example-relational-lifting}
below.

To get more intuition, let us prove some properties of function graph
relations, $r\triangleq\left<f\right>$. In some (but not in all)
cases, one can express the lifted relation $\left<f\right>^{\updownarrow P}$
through ordinary function liftings $f^{\uparrow P}$ and $f^{\downarrow P}$
and the $\left(P,f\right)$-wedge condition.

\subsubsection{Example \label{subsec:Example-relational-lifting}\ref{subsec:Example-relational-lifting}\index{solved examples}}

Consider the function graph relation $\left<f\right>$ of a given
function $f^{:A\rightarrow B}$. Use Definition~\ref{subsec:Definition-relational-lifting}
to compute the lifting $\left<f\right>^{\updownarrow P}$ for the
following type constructors $P^{\bullet}$:

\textbf{(a)} $P^{A}\triangleq A+A\times A\quad$.

\textbf{(b)} $P^{A}\triangleq(R\rightarrow A)\times(A\rightarrow S)$
where $R$ and $S$ are fixed types.

\textbf{(c)} $P^{A}\triangleq A\rightarrow A\quad$.

\textbf{(d)} $P^{A}\triangleq A\rightarrow\left(A\rightarrow R\right)\rightarrow R$
where $R$ is a fixed type.

\textbf{(e)} $P^{A}\triangleq\left(A\rightarrow A\right)\rightarrow A\quad$.

In each case \textbf{(a)}\textendash \textbf{(e)}, express $P^{A}\triangleq N^{A,A}$
where $N^{X,Y}$ is a suitable profunctor and compare the condition
$(p_{1},p_{2})\in\left<f\right>^{\updownarrow P}$ with the $\left(N,f\right)$-wedge
relation ($p_{1}\triangleright f^{\uparrow N}=p_{2}\triangleright f^{\downarrow N}$). 

\subparagraph{Solution}

In each case, the lifted relation $\left<f\right>^{\updownarrow P}$
has type $P^{A}\leftrightarrow P^{B}$.

\textbf{(a)} At the top level, $P^{A}$ is a disjunction: $P^{A}=\text{Id}^{A}+H^{A}$
where $H^{A}\triangleq A\times A=\text{Id}^{A}\times\text{Id}^{A}$.
The lifting to the identity functor is given by $r^{\updownarrow\text{Id}}\triangleq r$
according to Definition~\ref{subsec:Definition-relational-lifting}(b).
So, the lifting $r^{\updownarrow P}$ can be written as:
\[
r^{\updownarrow P}=r^{\updownarrow\text{Id}}\boxplus r^{\updownarrow H}=r^{\updownarrow\text{Id}}\boxplus\big(r^{\updownarrow\text{Id}}\boxtimes r^{\updownarrow\text{Id}}\big)=r\boxplus\big(r\boxtimes r\big)\quad.
\]
We see that the relational formula $r^{\updownarrow P}=r\boxplus\big(r\boxtimes r\big)$
repeats the type structure $P^{A}=A+A\times A$.

Setting now $r=\left<f\right>$, we find:
\[
\left<f\right>^{\updownarrow P}=\left<f\right>\boxplus\left<f\right>^{\updownarrow H}=\left<f\right>\boxplus\big(\left<f\right>\boxtimes\left<f\right>\big)\quad.
\]
Use Definition~\ref{subsec:Definition-relational-lifting}(d) to
express the condition for some values $(x^{:P^{A}},y^{:P^{B}})$ to
be in the relation $\left<f\right>\boxplus\left<f\right>^{\updownarrow H}$:
\[
(x,y)\in\left<f\right>^{\updownarrow P}\text{ if }x=x_{1}+\bbnum 0,\,y=y_{1}+\bbnum 0,\,(x_{1},y_{1})\in\left<f\right>\text{ or }x=\bbnum 0+x_{2},\,y=\bbnum 0+y_{2},\,(x_{2},y_{2})\in\left<f\right>^{\updownarrow H}\quad.
\]
The condition $(x_{1},y_{1})\in\left<f\right>$ means $f(x_{1})=y_{1}$.
Finally, Definition~\ref{subsec:Definition-relational-lifting}(c)
gives:
\[
(a_{1}^{:A}\times a_{2}^{:A},b_{1}^{:B}\times b_{2}^{:B})\in\left<f\right>\boxtimes\left<f\right>\text{ if }(a_{1},b_{1})\in\left<f\right>\text{ and }(a_{2},b_{2})\in\left<f\right>\quad.
\]
The last condition is simplified to:
\[
(a_{1}^{:A}\times a_{2}^{:A},b_{1}^{:B}\times b_{2}^{:B})\in\left<f\right>^{\updownarrow H}\text{ if }f(a_{1})=b_{1}\text{ and }f(a_{2})=b_{2}\quad.
\]

Putting the pieces together, we obtain the following definition of
the relation $\left<f\right>^{\updownarrow P}$:
\[
(x^{:P^{A}},y^{:P^{B}})\in\left<f\right>^{\updownarrow P}\text{ if }x=x_{1}+\bbnum 0\text{ and }y=f(x_{1})+\bbnum 0,\text{ or }x=\bbnum 0+a_{1}\times a_{2}\text{ and }y=\bbnum 0+f(a_{1})\times f(a_{2})\quad.
\]
We note that this condition is equivalent to applying $f^{\uparrow P}$:
\[
(x^{:P^{A}},y^{:P^{B}})\in\left<f\right>^{\updownarrow P}\text{ if }x\triangleright f^{\uparrow P}=y\quad,\quad\text{ or more concisely}:\quad\left<f\right>^{\updownarrow P}=\langle f^{\uparrow P}\rangle\quad.
\]

Statement~\ref{subsec:Statement-lifting-function-relation-covariant-1}(a)
below will show that $\left<f\right>^{\updownarrow P}=\langle f^{\uparrow P}\rangle$
for all functors $P$.

With $N^{X,Y}\triangleq P^{Y}$, the $\left(N,f\right)$-wedge relation
is equivalent to the equation $x\triangleright f^{\uparrow P}=y$.

\textbf{(b)} At the top level, $P^{A}$ is a product: $P^{A}=G^{A}\times H^{A}$
where $G^{A}\triangleq R\rightarrow A$ and $H^{A}\triangleq A\rightarrow S$.
By Definition~\ref{subsec:Definition-relational-lifting}(c), we
get $r^{\updownarrow P}=r^{\updownarrow G}\boxtimes r^{\updownarrow H}$.
To compute $r^{\updownarrow G}$, we note that $G$ is an exponential
functor construction, $G^{A}=K^{A}\rightarrow\text{Id}^{A}$, that
uses a constant functor $K^{A}\triangleq R$. Lifting to a constant
functor gives $r^{\updownarrow K}=\text{id}^{:R\leftrightarrow R}$
according to Definition~\ref{subsec:Definition-relational-lifting}(a).
So, we find:
\[
r^{\updownarrow G}=\text{id}^{:R\leftrightarrow R}\ogreaterthan r^{\updownarrow\text{Id}}=\text{id}^{:R\leftrightarrow R}\ogreaterthan r\quad.
\]
We treat $r^{\updownarrow H}$ similarly and obtain $r^{\updownarrow H}=r\ogreaterthan\text{id}^{:S\leftrightarrow S}$.
Put the pieces together:
\[
r^{\updownarrow P}=r^{\updownarrow G}\boxtimes r^{\updownarrow H}=(\text{id}^{:R\leftrightarrow R}\ogreaterthan r)\boxtimes(r\ogreaterthan\text{id}^{:S\leftrightarrow S})\quad.
\]
We see that $r^{\updownarrow P}$ repeats the type structure of $P^{A}=(R\rightarrow A)\times(A\rightarrow S)$.
The occurrences of $A$ in $P^{A}$ are replaced by $r$, while constant
types ($R$, $S$) are replaced by the corresponding identity relations
($\text{id}^{:R\leftrightarrow R}$ and $\text{id}^{:S\leftrightarrow S}$).
We will see a similar pattern in the remaining parts of this example.

We now set $r=\left<f\right>$ and transform $\left<f\right>^{\updownarrow P}$
to a more specific formula for a relation of type $P^{A}\leftrightarrow P^{B}$
where $A$, $B$ are arbitrary types. For arbitrary values $g_{1}^{:R\rightarrow A}$,
$g_{2}^{:R\rightarrow B}$, $h_{1}^{:A\rightarrow S}$, $h_{2}^{:B\rightarrow S}$,
we write:
\[
(g_{1}\times h_{1},g_{2}\times h_{2})\in\left<f\right>^{\updownarrow P}\text{ means }(g_{1},g_{2})\in\text{id}^{:R\leftrightarrow R}\ogreaterthan\left<f\right>\text{ and }(h_{1},h_{2})\in\left<f\right>\ogreaterthan\text{id}^{:S\leftrightarrow S}\quad.
\]

The condition $(g_{1},g_{2})\in\text{id}\ogreaterthan\left<f\right>$
is rewritten using Definition~\ref{subsec:Definition-relational-lifting}(e):
\[
(g_{1}^{:R\rightarrow A},g_{2}^{:R\rightarrow B})\in\text{id}\ogreaterthan\left<f\right>\text{ means if }(r_{1}^{:R},r_{2}^{:R})\in\text{id}^{:R\leftrightarrow R}\text{ then }(g_{1}(r_{1}),g_{2}(r_{2}))\in\left<f\right>\quad.
\]
The identity relation between $r_{1}$ and $r_{2}$ holds only if
$r_{1}=r_{2}$. We find:
\[
(g_{1}^{:R\rightarrow A},g_{2}^{:R\rightarrow B})\in\text{id}^{:R\leftrightarrow R}\ogreaterthan\left<f\right>\text{ means }\forall r^{:R}.\,f(g_{1}(r))=g_{2}(r)\quad\text{or equivalently}:\quad g_{1}\bef f=g_{2}\quad.
\]

The condition $(h_{1},h_{2})\in\left<f\right>^{\updownarrow H}$ is
rewritten using the same definitions:
\[
(h_{1}^{:A\rightarrow S},h_{2}^{:B\rightarrow S})\in\left<f\right>\ogreaterthan\text{id}^{:S\leftrightarrow S}\text{ means if }(a^{:A},b^{:B})\in\left<f\right>\text{ then }(h_{1}(a),h_{2}(b))\in\text{id}\quad.
\]
We simplify this to:
\[
(h_{1}^{:A\rightarrow S},h_{2}^{:B\rightarrow S})\in\left<f\right>\ogreaterthan\text{id}^{:S\leftrightarrow S}\text{ means }h_{1}(a)=h_{2}(f(a))\quad\text{or equivalently}:\quad h_{1}=f\bef h_{2}\quad.
\]

Finally, the lifted relation $\left<f\right>^{\updownarrow P}$ is
expressed as:
\[
(g_{1}\times h_{1},g_{2}\times h_{2})\in\left<f\right>^{\updownarrow P}\text{ if }g_{1}\bef f=g_{2}\text{ and }h_{1}=f\bef h_{2}\quad.
\]

Defining the profunctor $N^{X,Y}\triangleq\left(R\rightarrow Y\right)\times\left(X\rightarrow S\right)$,
we find that $g_{1}\times h_{1}$ has type $N^{A,A}$ and $g_{2}\times h_{2}$
has type $N^{B,B}$. Then the condition $(g_{1}\times h_{1},g_{2}\times h_{2})\in\left<f\right>^{\updownarrow P}$
is \emph{equivalent} to the $\left(N,f\right)$-wedge relation:
\[
(g_{1}\times h_{1})\triangleright f^{\uparrow N^{A,\bullet}}=(g_{2}\times h_{2})\triangleright f^{\downarrow N^{\bullet,B}}\quad.
\]

\textbf{(c)} Using Definition~\ref{subsec:Definition-relational-lifting}(b)
and (e), we get:
\begin{align*}
 & r^{\updownarrow P}=r^{\updownarrow\text{Id}}\ogreaterthan r^{\updownarrow\text{Id}}=r\ogreaterthan r\quad,\quad\quad\text{so}\quad\left<f\right>^{\updownarrow P}=\left<f\right>\ogreaterthan\left<f\right>\quad;\\
 & (p_{1}^{:A\rightarrow A},p_{2}^{:B\rightarrow B})\in\left<f\right>^{\updownarrow P}\quad\text{ means }\quad\text{if }(a^{:A},b^{:B})\in\left<f\right>\text{ then }(p_{1}(a),p_{2}(b))\in\left<f\right>\quad.
\end{align*}
The condition can be rewritten as a single equation:
\[
(p_{1}^{:A\rightarrow A},p_{2}^{:B\rightarrow B})\in\left<f\right>^{\updownarrow P}\quad\text{ means }\quad f(p_{1}(a))=p_{2}(f(a))\quad\text{or equivalently}:\quad p_{1}\bef f=f\bef p_{2}\quad.
\]
This is the same as the $\left(N,f\right)$-wedge relation ($p_{1}\triangleright f^{\uparrow N}=p_{2}\triangleright f^{\downarrow N}$)
with the profunctor $N^{X,Y}\triangleq X\rightarrow Y$.

\textbf{(d)} Since $P^{A}=\text{Id}^{A}\rightarrow Q^{A}$ is a function
type (with $Q^{A}\triangleq\left(A\rightarrow R\right)\rightarrow R$),
we use Definition~\ref{subsec:Definition-relational-lifting}(e):
\[
r^{\updownarrow P}=r^{\updownarrow\text{Id}}\ogreaterthan r^{\updownarrow Q}=r\ogreaterthan(r\ogreaterthan\text{id}^{:R\leftrightarrow R})\ogreaterthan\text{id}^{:R\leftrightarrow R}\quad.
\]
Let us now substitute $r=\left<f\right>$ and simplify the conditions:
\[
(p_{1}^{:P^{A}},p_{2}^{:P^{B}})\in\left<f\right>\ogreaterthan\left<f\right>^{\updownarrow Q}\quad\text{ means }\quad\text{if }(a^{:A},b^{:B})\in\left<f\right>\text{ then }(p_{1}(a),p_{2}(b))\in\left<f\right>^{\updownarrow Q}\quad.
\]
The definition of $\left<f\right>^{\updownarrow Q}$ is obtained if
we set $K^{A}\triangleq A\rightarrow R$:
\[
(q_{1}^{:Q^{A}},q_{2}^{:Q^{B}})\in\left<f\right>^{\updownarrow Q}\quad\text{ means }\quad\text{if }(k_{1}^{:A\rightarrow R},k_{2}^{:B\rightarrow R})\in\left<f\right>^{\updownarrow K}\text{ then }(q_{1}(k_{1}),q_{2}(k_{2}))\in\text{id}\quad.
\]
Expanding the definition of $\left<f\right>^{\updownarrow K}$ as
we did in part \textbf{(b)} of this example, we get:
\[
(q_{1},q_{2})\in\left<f\right>^{\updownarrow Q}\quad\text{ means }\quad\text{if }b=f(a)\text{ and }k_{1}(a)=k_{2}(b)\text{ then }q_{1}(k_{1})=q_{2}(k_{2})\quad.
\]
This simplifies to $k_{1}(a)=k_{2}(f(a))$, or $k_{1}=f\bef k_{2}$.
So, we finally express $\left<f\right>^{\updownarrow Q}$ and $\left<f\right>^{\updownarrow P}$
as:
\begin{align*}
 & (q_{1}^{:Q^{A}},q_{2}^{:Q^{B}})\in\left<f\right>^{\updownarrow Q}\quad\text{ means }\quad(k\rightarrow f\bef k)\bef q_{1}=q_{2}\quad,\\
 & (p_{1}^{:P^{A}},p_{2}^{:P^{B}})\in\left<f\right>^{\updownarrow P}\quad\text{ means }\quad a\rightarrow\big((k\rightarrow f\bef k)\bef p_{1}(a)\big)=f\bef p_{2}\quad.
\end{align*}

Defining $N^{X,Y}\triangleq X\rightarrow\left(Y\rightarrow R\right)\rightarrow R$,
we find that the last formula is equivalent to the $\left(N,f\right)$-wedge
relation for $p_{1}$ and $p_{2}$:
\[
p_{1}\triangleright f^{\uparrow N}=p_{1}\bef(q\rightarrow(k\rightarrow f\bef k)\bef q)\quad,\quad\quad p_{2}\triangleright f^{\downarrow N}=f\bef p_{2}\quad.
\]

\textbf{(e)} We may write $P^{A}=Q^{A}\rightarrow\text{Id}^{A}$ with
$Q^{A}\triangleq A\rightarrow A$ and use Definition~\ref{subsec:Definition-relational-lifting}(e):
\begin{align*}
 & r^{\updownarrow P}=r^{\updownarrow Q}\ogreaterthan r^{\updownarrow\text{Id}}=(r\ogreaterthan r)\ogreaterthan r\quad,\quad\quad\left<f\right>^{\updownarrow P}=(\left<f\right>\ogreaterthan\left<f\right>)\ogreaterthan\left<f\right>\quad,\\
 & (p_{1}^{:P^{A}},p_{2}^{:P^{B}})\in(\left<f\right>\ogreaterthan\left<f\right>)\ogreaterthan\left<f\right>\quad\text{means}\quad\text{if }(q_{1}^{:A\rightarrow A},q_{2}^{:B\rightarrow B})\in\left<f\right>\ogreaterthan\left<f\right>\text{ then }(p_{1}(q_{1}),p_{2}(q_{2}))\in\left<f\right>\quad.\\
 & (q_{1}^{:A\rightarrow A},q_{2}^{:B\rightarrow B})\in\left<f\right>\ogreaterthan\left<f\right>\quad\text{means}\quad\text{if }(a^{:A},b^{:B})\in\left<f\right>\text{ then }(q_{1}(a),q_{2}(b))\in\left<f\right>\quad.
\end{align*}
Simplifying these conditions, we get:
\begin{align}
 & (q_{1},q_{2})\in\left<f\right>^{\updownarrow Q}\quad\text{ means }\quad q_{1}\bef f=f\bef q_{2}\quad,\nonumber \\
 & (p_{1}^{:P^{A}},p_{2}^{:P^{B}})\in\left<f\right>^{\updownarrow P}\quad\text{ means }\quad\text{if }q_{1}\bef f=f\bef q_{2}\text{ then }f(p_{1}(q_{1}))=p_{2}(q_{2})\quad.\label{eq:relational-lifting-example-e-derivation1}
\end{align}
The condition for $(p_{1},p_{2})\in\left<f\right>^{\updownarrow P}$
\emph{cannot} be written as a single equation involving $p_{1}$ and
$p_{2}$ because we cannot express $q_{1}$ through $q_{2}$ or $q_{2}$
through $q_{1}$. The relation between $q_{1}$ and $q_{2}$ is many-to-many
and is not equivalent to a function graph.

Let us define $N^{X,Y}\triangleq\left(Y\rightarrow X\right)\rightarrow Y$
and write the $\left(N,f\right)$-wedge relation for $p_{1}^{:N^{A,A}}$
and $p_{2}^{:N^{B,B}}$:
\begin{equation}
k^{:B\rightarrow A}\triangleright(p_{1}\triangleright f^{\uparrow N})=k\triangleright(p_{2}\triangleright f^{\downarrow N})\quad\text{or equivalently}:\quad f(p_{1}(f\bef k))=p_{2}(k\bef f)\quad.\label{eq:N-f-wedge-relation-example-derivation1}
\end{equation}
We notice that if we set $q_{1}\triangleq f\bef k$ and $q_{2}\triangleq k\bef f$
then the precondition $q_{1}\bef f=f\bef q_{2}$ in Eq.~(\ref{eq:relational-lifting-example-e-derivation1})
will be satisfied, and it will follow that $f(p_{1}(q_{1}))=p_{2}(q_{2})$.
So, the $\left(N,f\right)$-wedge relation is a \emph{consequence}
of the relation $(p_{1},p_{2})\in\left<f\right>^{\updownarrow P}$.
However, if we are given some functions $f$, $q_{1}$, and $q_{2}$
such that the precondition $q_{1}\bef f=f\bef q_{2}$ holds, it does
not follow that there exists a function $k$ such that $q_{1}=f\bef k$
and $q_{2}=k\bef f$. A simple counterexample is found when $f$ is
a \emph{constant} function (it ignores its argument and always returns
a fixed value):
\[
f^{:A\rightarrow B}\triangleq\_^{:A}\rightarrow b_{0}\quad,\quad\quad\text{where }b_{0}\text{ is a fixed value of type }B\quad.
\]
Then the precondition $q_{1}\bef f=f\bef q_{2}$ is reduced to the
condition $q_{2}(b_{0})=b_{0}$ with \emph{no} restrictions on $q_{1}$.
But if $q_{1}^{:A\rightarrow A}$ were expressed as $q_{1}=f\bef k$
with some $k^{:B\rightarrow A}$, we would have $q_{1}=\_\rightarrow k(b_{0})$.
So, any function $q_{1}$ expressed as $f\bef k$ must be a constant
function. This is a stronger restriction than $q_{1}\bef f=f\bef q_{2}$
because more pairs $\left(q_{1},q_{2}\right)$ satisfy $q_{1}\bef f=f\bef q_{2}$.
Accordingly, the $\left(N,f\right)$-wedge relation~(\ref{eq:N-f-wedge-relation-example-derivation1})
constrains the functions $\left(p_{1},p_{2}\right)$ weaker and is
satisfied by more pairs $\left(p_{1},p_{2}\right)$ than the relation
$\left<f\right>^{\updownarrow P}$. So, the $\left(N,f\right)$-wedge
relation is\emph{ weaker} than the relation $\left<f\right>^{\updownarrow P}$.
$\square$

Below we will prove (Statement~\ref{subsec:Statement-wedge-law-from-parametricity})
that the $\left(N,f\right)$-wedge relation is always a \emph{consequence}
of the corresponding lifted relation $\left<f\right>^{\updownarrow P}$.
Example~\ref{subsec:Example-relational-lifting} shows that the $\left(N,f\right)$-wedge
relation is \emph{equivalent} to $\left<f\right>^{\updownarrow P}$
only when the type structure of $P$ is sufficiently simple.

Definition~\ref{subsec:Definition-relational-lifting} was motivated
by the ordinary lifting of functions to functors or contrafunctors.
Indeed, as we will prove in Statement~\ref{subsec:Statement-lifting-function-relation-covariant-1}
below, the relational lifting $\langle f\rangle^{\updownarrow P}$
is consistent with the ordinary liftings when $P^{A}$ is a functor
or a contrafunctor. Namely, $\left<f\right>^{\updownarrow P}=\langle f^{\uparrow P}\rangle$
if $P$ is a functor and $\left<f\right>^{\updownarrow P}=\text{rev}\langle f^{\downarrow P}\rangle$
if $P$ is a contrafunctor. 

\subsection{Properties of relational lifting. Simultaneous lifting}

Definition~\ref{subsec:Definition-relational-lifting} requires some
additional work to be fully usable. This section proves the properties
of relational lifting that will be needed later in this Appendix.

We begin by pointing out an ambiguity in applying Definition~\ref{subsec:Definition-relational-lifting}(g)
to $G^{A}\triangleq\forall X.\,H^{X,A}$ when $H^{X,A}$ does \emph{not}
depend on the type parameter $A$; that is, when $H^{X,A}=K^{X}$
with some $K^{\bullet}$. In that case, the type constructor $G$
is a constant functor: $G^{A}=Z\triangleq\forall X.\,K^{X}$. We may
lift a relation $r^{:A\leftrightarrow B}$ to $G^{A}\triangleq Z$
using Definition~\ref{subsec:Definition-relational-lifting}(a) and
obtain $r^{\updownarrow G}=\text{id}^{:Z\leftrightarrow Z}$. To show
that Definition~\ref{subsec:Definition-relational-lifting} is consistent,
we need to prove that Definitions~\ref{subsec:Definition-relational-lifting}(a)
and~(g) define the same lifted relation $r^{\updownarrow G}$. We
will show this is the next statement by using the relational naturality
law~(\ref{eq:relational-naturality-law-simplified}). That law is
the result of the relational parametricity theorem (Section~\ref{subsec:Relational-parametricity-theorem}),
which we will prove without using the next statement. The relational
parametricity theorem turns out to be a necessary requirement for
the consistency of Definition~\ref{subsec:Definition-relational-lifting}.

\subsubsection{Statement \label{subsec:Statement-parametricity-gives-identity-relation}\ref{subsec:Statement-parametricity-gives-identity-relation}}

For a quantified type $Z\triangleq\forall X.\ H^{X}$, the relation
$\text{id}^{:Z\leftrightarrow Z}$ has an equivalent form:
\[
(p^{:\forall X.\,H^{X}},q^{:\forall Y.\,H^{Y}})\in\text{id}\quad\text{ is equivalent to}:\quad\forall(X,Y).\,\forall s^{:X\leftrightarrow Y}.\,(p^{X},q^{Y})\in s^{\updownarrow H}\quad,
\]
or: $\forall s^{:X\leftrightarrow Y}.\,s^{\updownarrow H}=\text{id}$,
as long as $p$ and $q$ are implemented via fully parametric code.

\subparagraph{Proof}

We need to prove the equivalence in both directions.

If some $p$ and $q$ satisfy $(p^{X},q^{Y})\in s^{\updownarrow H}$
where we may choose $s$ arbitrarily, let us choose $s$ to be the
identity relation and force $X$ and $Y$ to be the same type:
\[
(x^{:X},y^{:Y})\in s\text{ only when }X=Y\text{ and }x=y\quad.
\]
Lifting an identity relation will produce an identity relation (Statement~\ref{subsec:Statement-relational-lifting-identity-law}),
so:
\[
\forall(X,Y).\,(p^{X},q^{Y})\in s^{\updownarrow H}\text{ with }s^{\updownarrow H}=\text{id}\quad\text{ means }\quad X=Y\text{ and }p^{X}=q^{X}\quad.
\]
This is the same as the identity relation between $p$ and $q$.

It remains to show that $(p^{X},q^{Y})\in s^{\updownarrow H}$ when
$p=q$. In other words, $(p^{X},p^{Y})\in s^{\updownarrow H}$ for
any $s^{:X\leftrightarrow Y}$. Since $p$ is fully parametric, it
obeys the naturality law~(\ref{eq:relational-naturality-law-simplified}),
which we will prove below without using this statement. We set $Q^{X}\triangleq H^{X}$,
$r\triangleq s$, and $t\triangleq p$ in the naturality law~(\ref{eq:relational-naturality-law-simplified}):
\[
\forall(X,Y).\,\forall s^{:X\leftrightarrow Y}.\,(p^{X},p^{Y})\in s^{\updownarrow H}\quad.
\]
$\square$

Another important construction required for Definition~\ref{subsec:Definition-relational-lifting}
is the simultaneous lifting of two relations (say, $r^{:A\leftrightarrow B}$
and $s^{:X\leftrightarrow Y}$) to a type constructor with two type
parameters (say, $G^{A,X}$ ). The result should be a new relation
of type $G^{A,X}\leftrightarrow G^{B,Y}$. It turns out that the simultaneous
lifting \emph{cannot} be expressed via the liftings $r^{\updownarrow G^{\bullet,X}}$
and $s^{\updownarrow G^{A,\bullet}}$ performed with respect to each
type parameter separately. Instead, we need to define the simultaneous
relational lifting as a special new operation. 

\subsubsection{Definition \label{subsec:Definition-simultaneous-relational-lifting}\ref{subsec:Definition-simultaneous-relational-lifting}
(simultaneous relational lifting)}

For any fully parametric type constructor $G^{\bullet,\bullet}$ with
two type parameters, we define the simultaneous lifting of two relations
$r^{:A\leftrightarrow B}$ and $s^{:X\leftrightarrow Y}$ to $G$
as a new relation denoted by $(r,s)^{\updownarrow G}$ of type $G^{A,X}\leftrightarrow G^{B,Y}$.
We use induction on the structure of $G$:

\textbf{(a)} If $G^{A,X}\triangleq Z$ with a fixed type $Z$, we
define $(r,s)^{\updownarrow G}\triangleq\text{id}^{:Z\leftrightarrow Z}$.

\textbf{(b)} If $G^{A,X}\triangleq A$, we define $(r,s)^{\updownarrow G}\triangleq r$.
If $G^{A,X}\triangleq X$, we define $(r,s)^{\updownarrow G}\triangleq s$.

\textbf{(c)} If $K^{\bullet,\bullet}$ and $L^{\bullet,\bullet}$
are any fully parametric type constructors, we define: 
\begin{align*}
{\color{greenunder}\text{for}\quad G^{A,X}\triangleq K^{A,X}\times L^{A,X}\quad:}\quad & (r,s)^{\updownarrow G}\triangleq(r,s)^{\updownarrow K}\boxtimes(r,s)^{\updownarrow L}\quad;\\
{\color{greenunder}\text{for}\quad G^{A,X}\triangleq K^{A,X}+L^{A,X}\quad:}\quad & (r,s)^{\updownarrow G}\triangleq(r,s)^{\updownarrow K}\boxplus(r,s)^{\updownarrow L}\quad;\\
{\color{greenunder}\text{for}\quad G^{A,X}\triangleq K^{A,X}\rightarrow L^{A,X}\quad:}\quad & (r,s)^{\updownarrow G}\triangleq(r,s)^{\updownarrow K}\ogreaterthan(r,s)^{\updownarrow L}\quad.
\end{align*}
The last relation is between functions of types $K^{A,X}\rightarrow L^{A,X}$
and $K^{B,Y}\rightarrow L^{B,Y}$. 

The inductive assumption is that simultaneous liftings to $K$ and
$L$ are already defined.

\textbf{(d)} If $G^{A,X}\triangleq S^{A,X,G^{A,X}}$ is defined recursively
via a recursion scheme $S^{\bullet,\bullet,\bullet}$, we define:
\[
(r,s)^{\updownarrow G}\triangleq\big(r,s,\overline{(r,s)^{\updownarrow G}}\big)^{\updownarrow S}\quad.
\]
Here we use $\overline{(r,s)^{\updownarrow G}}$ recursively within
the definition of $(r,s)^{\updownarrow G}$. This is allowed since
we understand $(r,s)^{\updownarrow G}$ to be a function, and it is
permitted to define functions recursively. The inductive assumption
is that simultaneous liftings of any \emph{three} relations to $S^{\bullet,\bullet,\bullet}$
are already defined. 

\textbf{(e)} If $G^{A,X}\triangleq\forall Z.\,H^{Z,A,X}$, we define
$(r,s)^{\updownarrow G}$ of type $(\forall U.\,H^{U,A,X})\leftrightarrow(\forall V.\,H^{V,B,Y})$
by:
\[
(p^{:\forall U.\,H^{U,A,X}},q^{:\forall V.\,H^{V,B,Y}})\in(r,s)^{\updownarrow\forall Z.\,H^{Z,\bullet,\bullet}}\quad\text{means}\quad\forall(U,V).\,\forall w^{U\leftrightarrow V}.\,(p^{U},q^{V})\in(w,r,s)^{\updownarrow H^{\bullet,\bullet,\bullet}}\quad.
\]
A shorter way of writing this definition is by formulating a relation
between $p^{U}$ and $q^{V}$ directly:
\begin{equation}
\forall(U,V):\quad\big((r,s)^{\updownarrow\forall Z.\,H^{Z,\bullet,\bullet}}\big)^{U,V}\triangleq\forall w^{:U\leftrightarrow V}.\,(w,r,s)^{\updownarrow H^{\bullet,\bullet,\bullet}}\quad.\label{eq:relational-lifting-quantified-types-short-1}
\end{equation}
The inductive assumption is that simultaneous liftings to $H^{\bullet,\bullet,\bullet}$
are already defined. $\square$

Parts (d) and (e) of Definition~\ref{subsec:Definition-simultaneous-relational-lifting}
use a simultaneous lifting of \emph{three} relations. Comparing Definitions~\ref{subsec:Definition-relational-lifting}
and~\ref{subsec:Definition-simultaneous-relational-lifting}, we
see that a similar inductive definition can be given for simultaneous
liftings of $n$ relations (with $n=1,2,...$) to a type constructor
with $n$ type parameters. We omit the details. This book will only
need simultaneous liftings of two relations.

When working with dinaturality laws (Section~\ref{sec:Naturality-laws-for-fully-parametric-functions}),
we used profunctors $P^{X,Y}$ whose type parameters are set to the
same type (e.g., $G^{A}\triangleq P^{A,A}$). In that case, there
is an ambiguity in lifting a relation $r^{:A\leftrightarrow B}$ to
$G$: First, Definition~\ref{subsec:Definition-relational-lifting}
defines $r^{\updownarrow G}$ as a relation of type $G^{A}\leftrightarrow G^{B}$,
which is the same type as $P^{A,A}\leftrightarrow P^{B,B}$. Second,
we may lift the pair of two relations $\left(r,r\right)$ simultaneously
to $P^{\bullet,\bullet}$ according to Definition~\ref{subsec:Definition-simultaneous-relational-lifting}
and obtain another relation $\left(r,r\right)^{\updownarrow P}$ of
the same type $P^{A,A}\leftrightarrow P^{B,B}$. The next statement
shows that $(r,r)^{\updownarrow P}=r^{\updownarrow G}$.

\subsubsection{Statement \label{subsec:Statement-relational-lifting-consistency-PAA}\ref{subsec:Statement-relational-lifting-consistency-PAA}}

Given any fully parametric type constructor $P^{X,Y}$, define $G^{A}\triangleq P^{A,A}$.
For any relation $r^{:A\leftrightarrow B}$, the liftings $r^{\updownarrow G}$
and $\left(r,r\right)^{\updownarrow P}$ will produce the same relation
of type $P^{A,A}\leftrightarrow P^{B,B}$.

\subparagraph{Proof}

We enumerate all cases of Definition~\ref{subsec:Definition-simultaneous-relational-lifting}
for the type constructor $P^{\bullet,\bullet}$ and the corresponding
cases of Definition~\ref{subsec:Definition-relational-lifting} for
$G^{\bullet}$. In each case, we will show that $(r,r)^{\updownarrow P}=r^{\updownarrow G}$.

If $P^{A,X}\triangleq Z$ with a fixed type $Z$, we have also $G^{A}=Z$.
Then $(r,r)^{\updownarrow P}=\text{id}^{:Z\leftrightarrow Z}$ and
$r^{\updownarrow G}=\text{id}^{:Z\leftrightarrow Z}$.

If $P^{A,X}\triangleq A$ or $P^{A,X}=X$, we have $G^{A}=A$. In
both cases $(r,r)^{\updownarrow P}\triangleq r$ and $r^{\updownarrow G}=r$.

If $P^{\bullet,\bullet}\triangleq K^{\bullet,\bullet}\times L^{\bullet,\bullet}$
then we have $G^{A}=K^{A,A}\times L^{A,A}$. Denote $M^{A}\triangleq K^{A,A}$
and $N^{A}\triangleq L^{A,A}$, so that $G^{A}=M^{A}\times N^{A}$.
The inductive assumptions are $(r,r)^{\updownarrow K}=r^{\updownarrow M}$
and $(r,r)^{\updownarrow L}=r^{\updownarrow N}$. We find:
\[
(r,r)^{\updownarrow P}=(r,r)^{\updownarrow K}\boxtimes(r,r)^{\updownarrow L}=r^{\updownarrow M}\boxtimes r^{\updownarrow N}=r^{\updownarrow(M\times N)}=r^{\updownarrow G}\quad.
\]

If $P^{\bullet,\bullet}\triangleq K^{\bullet,\bullet}+L^{\bullet,\bullet}$
with $G^{A}=M^{A}+N^{A}$ and the same inductive assumptions, we get:
\[
(r,r)^{\updownarrow P}=(r,r)^{\updownarrow K}\boxplus(r,r)^{\updownarrow L}=r^{\updownarrow M}\boxplus r^{\updownarrow N}=r^{\updownarrow(M+N)}=r^{\updownarrow G}\quad.
\]

If $P^{\bullet,\bullet}\triangleq K^{\bullet,\bullet}\rightarrow L^{\bullet,\bullet}$
with $G^{A}=M^{A}\rightarrow N^{A}$ and the same inductive assumptions,
we find: 
\[
(r,r)^{\updownarrow P}=(r,r)^{\updownarrow K}\ogreaterthan(r,r)^{\updownarrow L}=r^{\updownarrow M}\ogreaterthan r^{\updownarrow N}=r^{\updownarrow(M^{\bullet}\rightarrow N^{\bullet})}=r^{\updownarrow G}\quad.
\]

If $P^{A,X}\triangleq S^{A,X,P^{A,X}}$ with a recursion scheme $S^{\bullet,\bullet,\bullet}$,
the type constructor $G^{\bullet}$ is defined by: 
\[
G^{A}\triangleq P^{A,A}=S^{A,A,P^{A,A}}=S^{A,A,G^{A}}\quad.
\]
Denote $Q^{A,X}\triangleq S^{A,A,X}$ and obtain $G^{A}=Q^{A,G^{A}}$.
So, the lifting to $G$ is given by $r^{\updownarrow G}=\big(r,\overline{r^{\updownarrow G}}\big)^{\updownarrow Q}$.
Now write the definition of lifting to $P$ and simplify:
\begin{align*}
 & (r,r)^{\updownarrow P}=\big(r,r,\overline{(r,r)^{\updownarrow P}}\big)^{\updownarrow S}\\
{\color{greenunder}\text{inductive assumption }\overline{(r,r)^{\updownarrow P}}=\overline{r^{\updownarrow G}}:}\quad & =\big(r,r,\overline{r^{\updownarrow G}}\big)^{\updownarrow S}\\
{\color{greenunder}\text{inductive assumption }\big(r,\overline{r^{\updownarrow G}}\big)^{\updownarrow Q}=\big(r,r,\overline{r^{\updownarrow G}}\big)^{\updownarrow S}:}\quad & =\big(r,\overline{r^{\updownarrow G}}\big)^{\updownarrow Q}\\
{\color{greenunder}\text{definition of lifting to }G:}\quad & =r^{\updownarrow G}\quad.
\end{align*}

If $P^{A,X}\triangleq\forall Z.\,S^{Z,A,X}$, the type constructor
$G^{\bullet}$ is $G^{A}=\forall Z.\,S^{Z,A,A}$. Denote $Q^{Z,A}\triangleq S^{Z,A,A}$
and obtain $G^{A}=\forall Z.\,Q^{Z,A}$. The lifting $(r,r)^{\updownarrow P}$
is a relation of type $\forall U.\,S^{U,A,A}\leftrightarrow\forall V.\,S^{V,A,A}$
written as:
\begin{align*}
 & \forall(U,V):\quad\big((r,r)^{\updownarrow\forall Z.\,S^{Z,\bullet,\bullet}}\big)^{U,V}=\forall w^{:U\leftrightarrow V}.\,(w,r,r)^{\updownarrow S^{\bullet,\bullet,\bullet}}\\
{\color{greenunder}\text{inductive assumption }(w,r,r)^{\updownarrow S}=(w,r)^{\updownarrow Q}:}\quad & =\forall w^{:U\leftrightarrow V}.\,(w,r)^{\updownarrow Q}\\
{\color{greenunder}\text{definition of lifting to }G:}\quad & =\big(r^{\updownarrow G}\big)^{U,V}\quad.
\end{align*}


\subsection{Proof of the relational parametricity theorem\label{subsec:Relational-parametricity-theorem}}

Our goal is to prove that any fully parametric function $t:\forall A.\,P^{A}\rightarrow Q^{A}$
obeys the law~(\ref{eq:naturality-law-of-t-derivation2}):
\[
\forall A,B.\,\forall r^{:A\leftrightarrow B}.\,(t^{A},t^{B})\in r^{\updownarrow P}\ogreaterthan r^{\updownarrow Q}\quad.
\]
The proof will need to go by induction on the structure of the code
of the function $t$, which is built from smaller sub-expressions
using the nine code constructions of Table~\ref{tab:nine-pure-code-constructions}.
The inductive assumption is that all sub-expressions already satisfy
the relational naturality law. A problem with this approach is that
some sub-expressions of $t$ may contain free variables or may not
have the type signature of a function. To illustrate this, write the
code of $t$ as $t=z^{:P^{A}}\rightarrow\text{expr}(z)$, or in Scala:
\begin{lstlisting}
def t[A] = { z: P[A] => expr(z) }
\end{lstlisting}
where \textsf{``}\lstinline!expr(z)!\textsf{''} is the function\textsf{'}s body. That function
body \emph{itself} does not necessarily have a type signature of the
form $K^{A}\rightarrow L^{A}$. Also, \textsf{``}\lstinline!expr(z)!\textsf{''} may
contain $z$ as a free variable defined outside the scope of \textsf{``}\lstinline!expr(z)!\textsf{''},
and the type of $z$ may depend on the type parameter $A$. So, we
cannot directly apply the relational naturality law~(\ref{eq:naturality-law-of-t-derivation2})
to the subexpressions, which prevents us from using induction. The
relational naturality law needs to be reformulated to describe function
\emph{bodies}, i.e., arbitrary expressions that may contain externally
defined variables. A suitable formulation of the relational naturality
law is given in the next definition and will be the goal of Statement~\ref{subsec:Statement-main-relational-parametricity-1}.

\subsubsection{Definition \label{subsec:Definition-relational-naturality-law}\ref{subsec:Definition-relational-naturality-law}
(relational naturality law)}

Consider any expression $t:\forall A.\,Q^{A}$ containing a single
free variable\index{free variable} $x^{:P^{A}}$, where $P^{\bullet}$
and $Q^{\bullet}$ are any type constructors. Define the \textbf{binding
function} $t^{\prime}:\forall A.\,P^{A}\rightarrow Q^{A}$ such that
$t=\tilde{t}(x)$ and $\tilde{t}$ has no free variables. (The binding
function describes how the expression $t$ depends on its free variable
$x$.) Then the \textbf{relational naturality law}\index{naturality law!in terms of relations}
of $t$ is written as:
\begin{equation}
\forall(A,B).\,\forall r^{:A\leftrightarrow B}.\,(\tilde{t}^{A},\tilde{t}^{B})\in r^{\updownarrow P}\ogreaterthan r^{\updownarrow Q}\quad.\label{eq:relational-naturality-law-1}
\end{equation}
If $t$ contains several free variables ($x_{1}^{:P_{1}^{A}}$, $x_{2}^{:P_{2}^{A}}$,
etc.), we define the binding function $\tilde{t}$ as a curried function
of all the free variables. For example, with \emph{two} free variables
we will have $t=\tilde{t}(x_{1})(x_{2})$, so that $\tilde{t}^{A}$
will have type $P_{1}^{A}\rightarrow P_{2}^{A}\rightarrow Q^{A}$.
Then the relational naturality law is written as:
\begin{equation}
\forall(A,B).\,\forall r^{:A\leftrightarrow B}.\,(\tilde{t}^{A},\tilde{t}^{B})\in r^{\updownarrow P_{1}}\ogreaterthan r^{\updownarrow P_{2}}\ogreaterthan r^{\updownarrow Q}\quad.\label{eq:relational-naturality-law-two-free-vars}
\end{equation}
The generalization to any number of free variables is straightforward.
$\square$

If $t$ contains no free variables, we may still write $t$ as $\tilde{t}(x)$
where $\tilde{t}$ does not depend on $x$, setting $P^{A}\triangleq\bbnum 1$
for simplicity. In that case, Eq.~(\ref{eq:relational-naturality-law-1})
is simplified to:
\begin{align}
 & \forall(A,B).\,\forall r^{:A\leftrightarrow B}.\,\forall x_{1}^{:\bbnum 1},x_{2}^{:\bbnum 1}.\,\quad\text{if }x_{1}=x_{2}\text{ then }(\tilde{t}^{A}(x_{1}),\tilde{t}^{B}(x_{2}))\in r^{\updownarrow Q}\quad,\nonumber \\
{\color{greenunder}\text{or equivalently}:}\quad & \forall(A,B).\,\forall r^{:A\leftrightarrow B}.\,(t^{A},t^{B})\in r^{\updownarrow Q}\quad.\label{eq:relational-naturality-law-simplified}
\end{align}
Even in that case, to save time, we will keep using Eq.~(\ref{eq:relational-naturality-law-1})
and writing $t=\tilde{t}(x)$.

When an expression contains more than one free variable, we can also
gather all the free variables into a tuple. This creates an equivalent
expression with just one free variable:

\subsubsection{Statement \label{subsec:Statement--relational-naturality-tuple-1}\ref{subsec:Statement--relational-naturality-tuple-1} }

The relational naturality law for an expression $t:\forall A.\,Q^{A}$
containing two free variables $z_{1}^{:K^{A}}$ and $z_{2}^{:L^{A}}$
and the binding function $\tilde{t}(z_{1})(z_{2})=t$ is equivalent
to the relational naturality law for the expression $u$ with \emph{one}
free variable $h$ defined as:
\[
h^{:K^{A}\times L^{A}}\triangleq z_{1}\times z_{2}\quad,\quad\quad u\triangleq\tilde{u}(h)\triangleq\tilde{t}(h\triangleright\pi_{1})(h\triangleright\pi_{2})\quad.
\]


\subparagraph{Proof}

The relational naturality laws for $t$ and $u$ say that, for all
types $A$, $B$, and relations $r^{:A\leftrightarrow B}$:
\[
(\tilde{t}^{A},\tilde{t}^{B})\in r^{\updownarrow K}\ogreaterthan r^{\updownarrow L}\ogreaterthan r^{\updownarrow Q}\quad,\quad\quad(\tilde{u}^{A},\tilde{u}^{B})\in r^{\updownarrow(K\times L)}\ogreaterthan r^{\updownarrow Q}\quad.
\]
We need to show that these two relations are equivalent given the
definition of $\tilde{u}$ via $\tilde{t}$. This is similar to the
equivalence of curried and uncurried function types: $A\rightarrow(B\rightarrow C)\cong A\times B\rightarrow C$.
We write out the definition of the pair mapper operation ($\ogreaterthan$)
and obtain: 
\begin{align*}
(\tilde{t}^{A},\tilde{t}^{B})\in r^{\updownarrow K}\ogreaterthan r^{\updownarrow L}\ogreaterthan r^{\updownarrow Q}\quad\text{means}\quad: & \quad\forall x_{1}^{:K^{A}},x_{2}^{:L^{A}},y_{1}^{:K^{B}},y_{2}^{:L^{B}}:\,\text{ if }(x_{1},y_{1})\in r^{\updownarrow K}\text{ and }(x_{2},y_{2})\in r^{\updownarrow L}\\
 & \quad\quad\text{ then }(\tilde{t}(x_{1})(x_{2}),\tilde{t}(y_{1})(y_{2}))\in r^{\updownarrow Q}\quad;\\
(\tilde{u}^{A},\tilde{u}^{B})\in r^{\updownarrow(K\times L)}\ogreaterthan r^{\updownarrow Q}\quad\text{means}\quad: & \quad\forall h^{:K^{A}\times L^{A}},w^{:K^{B}\times L^{B}}:\,\text{ if }(h,w)\in r^{\updownarrow(K\times L)}\\
 & \quad\quad\text{ then }(\tilde{u}(h),\tilde{u}(w))\in r^{\updownarrow Q}\quad.
\end{align*}
By Definition~\ref{subsec:Definition-relational-lifting}(c) we have
$r^{\updownarrow(K\times L)}=r^{\updownarrow K}\boxtimes r^{\updownarrow L}$,
and so we may write: 
\[
(h,w)\in r^{\updownarrow(K\times L)}\quad\text{ if }\quad h=x_{1}\times x_{2}\quad,\quad w=y_{1}\times y_{2}\quad,\quad(x_{1},x_{2})\in r^{\updownarrow K}\quad,\quad\text{ and }\quad(y_{1},y_{2})\in r^{\updownarrow L}\quad.
\]
When $h=x_{1}\times x_{2}$ and $w=y_{1}\times y_{2}$, we will have
$\tilde{u}(h)=\tilde{t}(x_{1},x_{2})$ and $\tilde{u}(w)=\tilde{t}(y_{1},y_{2})$.
It follows that the relational naturality laws for $t$ and $u$ are
equivalent. $\square$

Due to Statement~\ref{subsec:Statement--relational-naturality-tuple-1},
we are allowed to assume that the expression $t$ always has a \emph{single}
free variable. This simplifies the formulation of the main theorem:

\subsubsection{Statement \label{subsec:Statement-main-relational-parametricity-1}\ref{subsec:Statement-main-relational-parametricity-1}
(relational parametricity theorem)}

Let $H$ and $Q$ be any fully parametric type constructors. Any fully
parametric expression $t:\forall A.\,Q^{A}$ of the form $t=\forall A.\,\tilde{t}^{A}(h)$
containing a single free variable $h^{:H^{A}}$ satisfies the relational
naturality law~(\ref{eq:relational-naturality-law-1}).

\subparagraph{Proof }

By assumption, $t$ is built up from the nine constructions of Table~\ref{tab:nine-pure-code-constructions}.
So, one of these nine constructions is at the top level in the syntax
tree of $t$. For each of those constructions, we will prove that
$t$ satisfies Eq.~(\ref{eq:relational-naturality-law-1}) as long
as all its sub-expressions do. Throughout the proof, all relational
naturality laws will involve an arbitrary relation $r^{:A\leftrightarrow B}$
between arbitrary types $A$, $B$. For brevity, we will not write
the quantifiers $\forall A,B,r^{:A\leftrightarrow B}$ in front of
all formulas.

To shorten the proof further, we note that in every inductive case
the expression $t$ and all its sub-expressions will contain the same
free variable $h$. So, all relational naturality laws will follow
the pattern \textsf{``}for all $h_{1}^{:H^{A}}$ and $h_{2}^{:H^{B}}$ satisfying
the relation $(h_{1},h_{2})\in r^{\updownarrow H}$, some other values
are in some other relation\textsf{''}. Let us choose arbitrary but fixed values
$h_{1}^{:H^{A}}$ and $h_{2}^{:H^{B}}$ satisfying $(h_{1},h_{2})\in r^{\updownarrow H}$,
denote $t_{1}\triangleq\tilde{t}^{A}(h_{1})$ and $t_{2}\triangleq\tilde{t}^{B}(h_{2})$,
and simplify the law~(\ref{eq:relational-naturality-law-1}) to:
\begin{equation}
(t_{1},t_{2})\in r^{\updownarrow Q}\quad\text{where}\quad t_{1}\triangleq\tilde{t}^{A}(h_{1})\text{ and }t_{2}\triangleq\tilde{t}^{B}(h_{2})\quad.\label{eq:relational-naturality-law-with-fixed-h}
\end{equation}
We will now prove this form of the law. Inductive assumptions will
always begin with \textsf{``}for all $h_{1}$ and $h_{2}$ ...\textsf{''}, so we are
allowed to substitute the fixed values $h_{1}$, $h_{2}$ into each
of the inductive assumptions and write those assumptions also in the
form of Eq.~(\ref{eq:relational-naturality-law-with-fixed-h}).

\paragraph{Use unit value}

In this case, $t\triangleq1$ and has the unit type ($Q^{A}\triangleq\bbnum 1$).
Since $t$ contains no free variables, the law~(\ref{eq:relational-naturality-law-with-fixed-h})
becomes:
\[
(1,1)\in r^{\updownarrow Q}\quad.
\]
This holds because by Definition~\ref{subsec:Definition-relational-lifting}(a)
we have $r^{\updownarrow Q}=\text{id}$ regardless of $r$.

The same proof applies for $t\triangleq c$ where $c$ is a value
of a fixed type $C$ (not built from $A$ or $B$).

\paragraph{Use argument}

In this case, $t\triangleq h$ where $h^{:H^{A}}$ is the free variable
(say, the argument of the function whose body is $t$). So, we must
have $Q=H$ and $\tilde{t}=\forall A.\,\text{id}^{:H^{A}\rightarrow H^{A}}$.
Then the law~(\ref{eq:relational-naturality-law-with-fixed-h}) becomes:
\begin{align*}
 & (t_{1},t_{2})\in r^{\updownarrow H}\quad\text{where}\quad t_{1}\triangleq h_{1}\text{ and }t_{2}\triangleq h_{2}\quad,\\
{\color{greenunder}\text{or equivalently}:}\quad & (h_{1},h_{2})\in r^{\updownarrow H}\quad.
\end{align*}
The last condition holds trivially, since $(h_{1},h_{2})\in r^{\updownarrow H}$
is already assumed.

\paragraph{Create function}

In this case, $t\triangleq\forall A.\,p^{:P^{A}}\rightarrow\tilde{g}^{A}(h)(p)$
where the sub-expression $\tilde{g}^{A}(h)(p):G^{A}$ contains two
free variables ($h^{:H^{A}}$ and $p^{:P^{A}}$). Because $\tilde{g}(h)(p)$
is defined with curried arguments, the binding function $\tilde{t}$
satisfies:
\[
\tilde{t}(h)=t=p\rightarrow\tilde{g}(h)(p)=\tilde{g}(h)\quad,\quad\quad\text{or equivalently}:\quad\tilde{t}=\tilde{g}.
\]
The type of $\tilde{g}$ is $H^{A}\rightarrow Q^{A}$, where we set
$Q^{A}\triangleq P^{A}\rightarrow G^{A}$. Definition~\ref{subsec:Definition-relational-lifting}(e)
gives $r^{\updownarrow Q}=r^{\updownarrow P}\ogreaterthan r^{\updownarrow G}$.
So, the inductive assumption for $g$ may be written as:
\[
(g_{1},g_{2})\in r^{\updownarrow P}\ogreaterthan r^{\updownarrow G}\quad\text{where}\quad g_{1}\triangleq\tilde{g}^{A}(h_{1})=t_{1}\text{ and }g_{2}\triangleq\tilde{g}^{B}(h_{2})=t_{2}\quad.
\]
This is equivalent to:
\[
(t_{1},t_{2})\in r^{\updownarrow P}\ogreaterthan r^{\updownarrow G}=r^{\updownarrow Q}\quad,
\]
which is the law~(\ref{eq:relational-naturality-law-with-fixed-h})
we needed to prove.

\paragraph{Use function}

In this case, $t=\tilde{t}(h)\triangleq k(p)$ where $k^{:\forall A.\,P^{A}\rightarrow Q^{A}}$
and $p^{:\forall A.\,P^{A}}$ are some sub-expressions. Both $k$
and $p$ may contain $h^{:H^{A}}$ as a free variable: $k\triangleq\tilde{k}(h)$
and $p\triangleq\tilde{p}(h)$. Denote for brevity $K^{A}\triangleq P^{A}\rightarrow Q^{A}$.
The inductive assumptions say that the laws~(\ref{eq:relational-naturality-law-with-fixed-h})
already hold for both $k$ and $p$:
\begin{align*}
{\color{greenunder}\text{for }k:}\quad & (k_{1},k_{2})\in r^{\updownarrow K}=r^{\updownarrow P}\ogreaterthan r^{\updownarrow Q}\quad\quad\text{where}\quad k_{1}\triangleq\tilde{k}^{A}(h_{1})\text{ and }k_{2}\triangleq\tilde{k}^{B}(h_{2})\quad,\\
{\color{greenunder}\text{for }p:}\quad & (p_{1},p_{2})\in r^{\updownarrow P}\quad\quad\text{where}\quad p_{1}\triangleq\tilde{p}^{A}(h_{1})\text{ and }p_{2}\triangleq\tilde{p}^{B}(h_{2})\quad.
\end{align*}
We need to show that:
\[
(t_{1},t_{2})\in r^{\updownarrow Q}\quad\quad\text{where}\quad t_{1}\triangleq\tilde{t}^{A}(h_{1})=k_{1}(p_{1})\text{ and }t_{2}\triangleq\tilde{t}^{B}(h_{2})=k_{2}(p_{2})\quad.
\]
By definition of $\ogreaterthan$, the inductive assumption $(k_{1},k_{2})\in r^{\updownarrow P}\ogreaterthan r^{\updownarrow Q}$
means:
\[
(k_{1}(p_{1}),k_{2}(p_{2}))\in r^{\updownarrow Q}\quad\text{whenever}\quad(p_{1},p_{2})\in r^{\updownarrow P}\quad.
\]
Since $(p_{1},p_{2})\in r^{\updownarrow P}$ already holds by the
other inductive assumption, we obtain $(t_{1},t_{2})\in r^{\updownarrow Q}$.

\paragraph{Create tuple}

In this case, $t=\tilde{t}(h)\triangleq k^{:K^{A}}\times l^{:L^{A}}$,
where the sub-expressions $k$ and $l$ contain the free variable
$h^{:H^{A}}$ via $k\triangleq\tilde{k}(h)$ and $l\triangleq\tilde{l}(h)$.
We have $Q^{A}\triangleq K^{A}\times L^{A}$, so we use Definition~\ref{subsec:Definition-relational-lifting}(c)
for the lifting $r^{\updownarrow Q}$ to get $r^{\updownarrow Q}=r^{\updownarrow K}\boxtimes r^{\updownarrow L}$.
We need to prove that:
\begin{align*}
 & (t_{1},t_{2})\in r^{\updownarrow Q}\quad\text{or equivalently}:\quad(k_{1}\times l_{1},k_{2}\times l_{2})\in r^{\updownarrow K}\boxtimes r^{\updownarrow L}\quad,\\
 & \quad\text{where}\quad t_{1}\triangleq k_{1}\times l_{1}\triangleq\tilde{k}^{A}(h_{1})\times\tilde{l}^{A}(h_{1})\quad,\quad\quad t_{2}\triangleq k_{2}\times l_{2}\triangleq\tilde{k}^{A}(h_{2})\times\tilde{l}^{A}(h_{2})\quad.
\end{align*}
The inductive assumptions are that the relational naturality law~(\ref{eq:relational-naturality-law-with-fixed-h})
holds for $k$ and $l$:
\[
(k_{1},k_{2})\in r^{\updownarrow K}\quad,\quad\quad(l_{1},l_{2})\in r^{\updownarrow L}\quad.
\]
By definition of $\boxtimes$ for relations, we obtain the required
property: $(k_{1}\times l_{1},k_{2}\times l_{2})\in r^{\updownarrow K}\boxtimes r^{\updownarrow L}$.

\paragraph{Use tuple}

It is sufficient to consider the case $t\triangleq\pi_{1}(g)$ where
$g^{:\forall A.\,Q^{A}\times L^{A}}$ is a sub-expression that contains
the free variable $h^{:H^{A}}$. The proof for $t=\pi_{2}(g)$ is
analogous. 

We need to prove that $t$ satisfies the relational naturality law
if $g$ does. Define $g=\tilde{g}(h)$ and:
\[
g_{1}\triangleq\tilde{g}^{A}(h_{1})\quad,\quad\quad g_{2}\triangleq\tilde{g}^{B}(h_{2})\quad,\quad\quad t_{1}\triangleq\pi_{1}(g_{1})\quad,\quad\quad t_{2}\triangleq\pi_{1}(g_{2})\quad.
\]

The relational naturality law of $g$, which holds by the inductive
assumption, is:
\[
(g_{1},g_{2})\in r^{\updownarrow(Q\times L)}\quad.
\]
By Definition~\ref{subsec:Definition-relational-lifting}(c) for
the lifting $r^{\updownarrow(Q\times L)}$, we have:
\[
(g_{1},g_{2})\in r^{\updownarrow(Q\times L)}\quad\text{ means }\quad(\pi_{1}(g_{1}),\pi_{1}(g_{2}))\in r^{\updownarrow Q}\text{ and }(\pi_{2}(g_{1}),\pi_{2}(g_{2}))\in r^{\updownarrow L}\quad.
\]
The condition $(\pi_{1}(g_{1}),\pi_{1}(g_{2}))\in r^{\updownarrow Q}$
is the same as the relational naturality law of $t$.

\paragraph{Create disjunction}

We consider the case $t\triangleq g+\bbnum 0$ where $Q^{A}\triangleq K^{A}+L^{A}$
and $g^{:\forall A.\,K^{A}}$ is a sub-expression that may contain
the free variable $h^{:H^{A}}$ as $g\triangleq\tilde{g}(h)$. Define
for convenience: 
\[
g_{1}\triangleq\tilde{g}^{A}(h_{1})\quad,\quad\quad g_{2}\triangleq\tilde{g}^{B}(h_{2})\quad,\quad\quad t_{1}=g_{1}+\bbnum 0\quad,\quad\quad t_{2}=g_{2}+\bbnum 0\quad.
\]
The inductive assumption is that $g(h)$ satisfies its relational
naturality law, which is $(g_{1},g_{2})\in r^{\updownarrow K}$. By
Definition~\ref{subsec:Definition-relational-lifting}(d) for the
lifting $r^{\updownarrow(K+L)}$, we have:
\[
(t_{1},t_{2})=(g_{1}+\bbnum 0,g_{2}+\bbnum 0)\in r^{\updownarrow(K+L)}=r^{\updownarrow K}\boxplus r^{\updownarrow L}\text{ when }(g_{1},g_{2})\in r^{\updownarrow K}\quad.
\]
So, the relational naturality law of $t$ holds. The proof for $t\triangleq\bbnum 0+g$
is analogous.

\paragraph{Use disjunction}

In this case, $t\triangleq\forall A.\,\,\begin{array}{|c||c|}
 & G^{A}\\
\hline K^{A} & p\\
L^{A} & q
\end{array}\,\,$ is a pattern-matching function of type $\forall A.\,Q^{A}$ with
$Q^{A}\triangleq K^{A}+L^{A}\rightarrow G^{A}$. The sub-expressions
$p^{:\forall A.\,K^{A}\rightarrow G^{A}}$ and $q^{:\forall A.\,L^{A}\rightarrow G^{A}}$
may contain the free variable $h^{:H^{A}}$. We define:
\[
p\triangleq\tilde{p}(h)\quad,\quad p_{1}\triangleq\tilde{p}^{A}(h_{1})\quad,\quad p_{2}\triangleq\tilde{p}^{B}(h_{2})\quad,\quad\quad q\triangleq\tilde{q}(h)\quad,\quad q_{1}\triangleq\tilde{q}^{A}(h_{1})\quad,\quad q_{2}\triangleq\tilde{q}^{B}(h_{2})\quad.
\]
Then we have:
\[
t_{1}=\,\begin{array}{|c||c|}
 & G^{A}\\
\hline K^{A} & p_{1}\\
L^{A} & q_{1}
\end{array}\quad,\quad\quad t_{2}=\,\begin{array}{|c||c|}
 & G^{B}\\
\hline K^{B} & p_{2}\\
L^{B} & q_{2}
\end{array}\quad.
\]
By the inductive assumption, the relational naturality law already
holds for $p$ and $q$:
\begin{align*}
 & \text{if }(k_{1}^{:K^{A}},k_{2}^{:K^{B}})\in r^{\updownarrow K}\text{ then }\big(p_{1}(k_{1}),p_{2}(k_{2})\big)\in r^{\updownarrow G}\quad,\\
 & \text{if }(l_{1}^{:L^{A}},l_{2}^{:L^{B}})\in r^{\updownarrow L}\text{ then }\big(q_{1}(l_{1}),q_{2}(l_{2})\big)\in r^{\updownarrow G}\quad.
\end{align*}
To derive the specific form of the law for $t$, we use Definition~\ref{subsec:Definition-relational-lifting}(e)
for the lifting $r^{\updownarrow Q}$:
\[
\text{if }(x_{1}^{:K^{A}+L^{A}},x_{2}^{:K^{B}+L^{B}})\in r^{\updownarrow(K+L)}\text{ then }\big(t_{1}(x_{1}),t_{2}(x_{2})\big)\in r^{\updownarrow G}\quad.
\]
By Definition~\ref{subsec:Definition-relational-lifting}(d), the
values $x_{1}$ and $x_{2}$ are in relation $r^{\updownarrow(K+L)}$
only if both $x_{1}$ and $x_{2}$ are in the same part of the disjunction
($K^{\bullet}+L^{\bullet}$). We consider separately the case when
they are in the left part or in the right part.

If both $x_{1}$ and $x_{2}$ in the left part, we can write $x_{1}=k_{1}^{:K^{A}}+\bbnum 0$
and $x_{2}=k_{2}^{:K^{B}}+\bbnum 0$ with some $k_{1}$ and $k_{2}$.
Then the condition $(x_{1},x_{2})\in r^{\updownarrow(K+L)}$ is equivalent
to $(k_{1},k_{2})\in r^{\updownarrow K}$, while the values $t_{1}(x_{1})$
and $t_{2}(x_{2})$ are expressed as:
\[
t_{1}(x_{1})=(k_{1}+\bbnum 0)\triangleright\,\begin{array}{|c||c|}
 & G^{A}\\
\hline K^{A} & p_{1}\\
L^{A} & q_{1}
\end{array}\,=p_{1}(k_{1})\quad,\quad\quad t_{2}(x_{2})=(k_{2}+\bbnum 0)\triangleright\,\begin{array}{|c||c|}
 & G^{B}\\
\hline K^{B} & p_{2}\\
L^{B} & q_{2}
\end{array}\,=p_{2}(k_{2})\quad.
\]
So, the conclusion of the relational naturality law of $t$ holds
due to the relational law of $p$:
\[
\big(t_{1}(x_{1}),t_{2}(x_{2})\big)=\big(p_{1}(k_{1}),p_{2}(k_{2})\big)\in r^{\updownarrow G}\quad.
\]

A similar argument proves the law for the case when both $x_{1}$
and $x_{2}$ in the right part. We write $x_{1}=\bbnum 0+l_{1}^{:L^{A}}$
and $x_{2}=\bbnum 0+l_{2}^{:L^{B}}$ and reduce the relational naturality
law of $t$ to that of $q$.

\paragraph{Recursive call}

In this case, $t\triangleq g$ where $g$ is a recursive call to a
function (defined outside that expression). When proving a law of
a recursively defined function, we may assume that the law holds for
recursive calls to that function. So, the inductive assumption says
that $g$ is some expression for which the relational naturality law
already holds. It then holds for $t$ since $t=g$.

This completes the proof of the relational parametricity theorem.

\subsection{Naturality laws and the wedge law follow from relational parametricity}

A programmer usually needs to derive laws in terms of \emph{functions}
rather than relations. To convert the relational naturality law~(\ref{eq:relational-naturality-law-simplified})
into a law involving functions, we first choose the relation $r$
as $r\triangleq\left<f\right>$ with an arbitrary function $f$. Then
we need to express the lifted relation $r^{\updownarrow Q}$ via an
equation and derive the corresponding equation for $t$. However,
this is not always possible: Example~\ref{subsec:Example-relational-lifting}(e)
shows that for sufficiently complicated type constructors $Q$, the
lifted relation $\left<f\right>^{\updownarrow Q}$ is \emph{not} expressible
as an equation. Nevertheless, for many type signatures found in practice,
this problem does not arise and the relational naturality law is reduced
to the dinaturality law~(\ref{eq:dinaturality-law-for-profunctors})
via the wedge law~(\ref{eq:wedge-law-for-profunctors}). 

For type signatures of natural transformations ($\forall A.\,G^{A}\rightarrow H^{A}$),
the wedge law reduces to the naturality laws~(\ref{eq:naturality-law-for-functors})
or~(\ref{eq:naturality-law-for-contrafunctors}). So, the naturality
laws and the wedge law are shortcuts for using the parametricity theorem
without having to work with a fully general relational law.

For certain more complicated type signatures, one can derive a property
called the \textsf{``}strong\textsf{''} dinaturality. This property is not expressed
as a single equation and is stronger than the wedge law but is still
simpler to use than the full relational law. In this book, strong
dinaturality is needed only to prove Statements~\ref{subsec:relational-property-for-foldFn},
\ref{subsec:Statement-Church-encoding-recursive-type-covariant},
and~\ref{subsec:Statement-strong-dinaturality-property-of-fix},
while in all other places the wedge law is sufficient. This and the
following sections will prove that the wedge law and the strong dinaturality
law follow from the relational naturality law.

Strong dinaturality does \emph{not} hold when the type signature involves
deeply nested functions such as $F^{R}\triangleq\forall A.\,(\left(A\rightarrow R\right)\rightarrow A)\rightarrow A$.
It turns out that $F^{R}\cong R$ (see Exercise~\ref{par:Problem-Peirce-law}).
To prove that, one needs to use the relational naturality law in its
full generality. The wedge law still holds for such functions but
does not provide enough information for proving the type equivalence
$F^{R}\cong R$. 

We will now derive the wedge law from the relational naturality law
of \lstinline!xmap!.\footnote{Our proof is adapted from the blog post by B.~Milewski\index{Bartosz Milewski}
(\texttt{\href{https://bartoszmilewski.com/2017/04/11/}{https://bartoszmilewski.com/2017/04/11/}}).
A different derivation of the dinaturality law from the relational
parametricity theorem was given by J.~Voigtl\"ander\index{Janis@Janis Voigtl\"ander}
in the paper \texttt{\href{https://arxiv.org/pdf/1908.07776.pdf}{https://arxiv.org/pdf/1908.07776.pdf}}.
A proof of the dinaturality law based on finding a syntactic normal
form of fully parametric code (and \emph{without} using relational
approach parametricity) is in the paper \textsf{``}Dinatural terms in System
$F$\textsf{''} by J.~de Lataillade\index{Joachim de Lataillade}: see \texttt{\href{https://www.irif.fr/~delatail/dinat.pdf}{https://www.irif.fr/$\sim$delatail/dinat.pdf}}} 

\subsubsection{Statement \label{subsec:Statement-wedge-law-from-parametricity}\ref{subsec:Statement-wedge-law-from-parametricity}}

\textbf{(a)} For a profunctor $P^{X,Y}$ with fully parametric liftings,
define $T^{A}\triangleq P^{A,A}$. Then:
\[
\text{for all }X,Y,f^{:X\rightarrow Y},x^{:P^{X,X}},y^{:P^{Y,Y}}\quad:\quad\text{if}\quad(x,y)\in\left<f\right>^{\updownarrow T}\quad\text{then}\quad x\triangleright f^{\uparrow P^{X,\bullet}}=y\triangleright f^{\downarrow P^{\bullet,Y}}\quad.
\]
In other words, the $\left(P,f\right)$-wedge relation is a consequence
of the lifted relation $\left<f\right>^{\updownarrow T}$.

\textbf{(b)} Any fully parametric value $t$ of type $\forall A.\,P^{A,A}$
satisfies the wedge law\index{wedge law!of profunctors}~(\ref{eq:wedge-law-for-profunctors}):
\[
\text{for all }X,Y,f^{:X\rightarrow Y}\quad:\quad t^{X}\triangleright f^{\uparrow P^{X,\bullet}}=t^{Y}\triangleright f^{\downarrow P^{\bullet,Y}}\quad.
\]
In other words, $t^{X}$ and $t^{Y}$ are always in the $(P,f)$-wedge
relation\index{wedge relation} for any function $f^{:X\rightarrow Y}$.

\subparagraph{Proof}

\textbf{(a)} The profunctor $P$ has an \lstinline!xmap! method with
the type signature:
\[
\text{xmap}_{P}:\forall A,B,C,D.\,\left(B\rightarrow A\right)\rightarrow\left(C\rightarrow D\right)\rightarrow P^{A,C}\rightarrow P^{B,D}\quad.
\]
This method may be defined by:
\[
\text{xmap}_{P}(f^{:B\rightarrow A})(g^{:C\rightarrow D})=f^{\downarrow P^{\bullet,C}}\bef g^{\uparrow P^{A,\bullet}}\quad.
\]
Here, we do not assume the profunctor commutativity law (Statement~\ref{subsec:Proof-of-the-profunctor-commutativity-law})
but simply choose a specific order of composition, namely $f\bef g$,
while defining \lstinline!xmap!. We can now express the $\left(P,f\right)$-wedge
relation in terms of \lstinline!xmap!:
\[
x\triangleright f^{\uparrow P^{X,\bullet}}=y\triangleright f^{\downarrow P^{\bullet,Y}}\quad\text{is equivalent to}:\quad\text{xmap}_{P}(\text{id})(f)(x)=\text{xmap}_{P}(f)(\text{id})(y)\quad.
\]
The condition we are required to prove can now be expressed via \lstinline!xmap!
like this:
\begin{equation}
\text{if}\quad(x,y)\in\left<f\right>^{\updownarrow T}\quad\text{then}\quad\text{xmap}_{P}(\text{id})(f)(x)=\text{xmap}_{P}(f)(\text{id})(y)\quad.\label{eq:wedge-law-from-parametricity-derivation1}
\end{equation}
The \lstinline!xmap! method is implemented by fully parametric code.
It follows from Statement~\ref{subsec:Statement-main-relational-parametricity-1}
that \lstinline!xmap! satisfies a relational naturality law. Since
\lstinline!xmap! has four type parameters ($A$, $B$, $C$, $D$),
its relational naturality law involves 8 arbitrary types (denote them
by $A$, $B$, $C$, $D$, $A^{\prime}$, $B^{\prime}$, $C^{\prime}$,
$D^{\prime}$) and 4 arbitrary relations (denote them by $a^{:A\leftrightarrow A^{\prime}}$,
$b^{:B\leftrightarrow B^{\prime}}$, $c^{:C\leftrightarrow C^{\prime}}$,
and $d^{:D\leftrightarrow D^{\prime}}$). With these notations, we
write the relational naturality law of \lstinline!xmap! as:
\begin{equation}
\forall(a,b,c,d)\quad:\quad(\text{xmap}_{P}^{A,B,C,D},\text{xmap}_{P}^{A^{\prime},B^{\prime},C^{\prime},D^{\prime}})\in(b\ogreaterthan a)\ogreaterthan(c\ogreaterthan d)\ogreaterthan(a,c)^{\updownarrow P}\ogreaterthan(b,d)^{\updownarrow P}\quad,\label{eq:relational-naturality-law-of-xmap}
\end{equation}
where $(a,c)^{\updownarrow P}$ and $(b,d)^{\updownarrow P}$ are
simultaneous liftings to the profunctor $P$ (see Definition~\ref{subsec:Definition-simultaneous-relational-lifting}).
Writing out the definition of the pair mapper ($\ogreaterthan$),
we express this law as:
\begin{align}
 & \forall\big(a^{:A\leftrightarrow A^{\prime}},b^{:B\leftrightarrow B^{\prime}},c^{:C\leftrightarrow C^{\prime}},d^{:D\leftrightarrow D^{\prime}},g^{:B\rightarrow A},h^{:B^{\prime}\rightarrow A^{\prime}},k^{:C\rightarrow D},l^{:C^{\prime}\rightarrow D^{\prime}},x^{:P^{A,C}},y^{:P^{A^{\prime},C^{\prime}}}\big)\quad:\nonumber \\
 & \quad\text{if}\quad\quad(g,h)\in b\ogreaterthan a\quad,\quad(k,l)\in c\ogreaterthan d\quad,\quad\text{and}\quad(x,y)\in(a,c)^{\updownarrow P}\quad,\nonumber \\
 & \quad\text{then}\quad\quad\big(\text{xmap}_{P}(g)(k)(x),\,\text{xmap}_{P}(h)(l)(y)\big)\in(b,d)^{\updownarrow P}\quad.\label{eq:xmap-relational-law-derivation2}
\end{align}
The relational naturality law~(\ref{eq:xmap-relational-law-derivation2})
of \lstinline!xmap! holds for arbitrary choices of 8 types, 4 relations,
and 4 functions, in addition to the values $x$ and $y$. The plan
of the proof is to substitute certain carefully chosen functions and
relations into Eq.~(\ref{eq:xmap-relational-law-derivation2}) in
order to derive Eq.~(\ref{eq:wedge-law-from-parametricity-derivation1}). 

We note that Eq.~(\ref{eq:wedge-law-from-parametricity-derivation1})
says that two \lstinline!xmap(...)! values must be equal (both values
having type $P^{X,Y}$). This can be obtained from Eq.~(\ref{eq:xmap-relational-law-derivation2})
only if the relation $(b,d)^{\updownarrow P}$ is an identity relation
of type $P^{X,Y}\leftrightarrow P^{X,Y}$. We will prove in Statement~\ref{subsec:Statement-relational-lifting-identity-law}
below that a simultaneous lifting of two identity relations gives
again an identity relation. This suggests choosing $b=\text{id}$
and $d=\text{id}$, which also requires us to set the types as $B=B^{\prime}=X$
and $D=D^{\prime}=Y$.

The relations $a$ and $c$ need to be chosen such that the precondition
$(x,y)\in(a,c)^{\updownarrow P}$ in Eq.~(\ref{eq:xmap-relational-law-derivation2})
reproduces the precondition $(x,y)\in\left<f\right>^{\updownarrow T}$
in Eq.~(\ref{eq:wedge-law-from-parametricity-derivation1}). Statement~\ref{subsec:Statement-relational-lifting-consistency-PAA}
shows that $r^{\updownarrow T}=(r,r)^{\updownarrow P}$ for any relation
$r$. In particular, $\left<f\right>^{\updownarrow T}=(\left<f\right>,\left<f\right>)^{\updownarrow P}$.
So, we need to set $a=c=\left<f\right>$ and the corresponding types
$A=C=X$ and $A^{\prime}=C^{\prime}=Y$.

The functions $g$, $h$, $k$, and $l$ in Eq.~(\ref{eq:xmap-relational-law-derivation2})
are chosen such that:
\[
\text{xmap}_{P}(g)(k)(x)=\text{xmap}_{P}(\text{id})(f)(x)\quad,\quad\quad\text{xmap}_{P}(h)(l)(x)=\text{xmap}_{P}(f)(\text{id})(x)\quad.
\]
This implies $g=\text{id}$, $h=k=f$, and $l=\text{id}$. With these
choices, the preconditions $(g,h)\in b\ogreaterthan a$ and $(k,l)\in c\ogreaterthan d$
will automatically hold in Eq.~(\ref{eq:xmap-relational-law-derivation2}).
So, the conclusion of Eq.~(\ref{eq:xmap-relational-law-derivation2})
also holds, which proves Eq.~(\ref{eq:wedge-law-from-parametricity-derivation1}). 

\textbf{(b)} The relational parametricity theorem (Statement~\ref{subsec:Statement-main-relational-parametricity-1})
gives $(t^{X},t^{Y})\in c^{\updownarrow T}$ for any relation $c^{:X\leftrightarrow Y}$.
We can now choose $c\triangleq\left<f\right>$ and use the result
of part \textbf{(a)} where we set $x\triangleq t^{X}$ and $y\triangleq t^{Y}$.
Then we obtain $t^{X}\triangleright f^{\uparrow P^{X,\bullet}}=t^{Y}\triangleright f^{\downarrow P^{\bullet,Y}}$,
which is the wedge law of $t$. $\square$

While proving Statement~\ref{subsec:Statement-wedge-law-from-parametricity},
we used the \textbf{identity laws}\index{identity laws!of relational lifting}
of the relational lifting:

\subsubsection{Statement \label{subsec:Statement-relational-lifting-identity-law}\ref{subsec:Statement-relational-lifting-identity-law}}

\textbf{(a)} An identity relation can be removed from a simultaneous
lifting. For instance, given any type constructor $H^{\bullet,\bullet}$,
a fixed type $T$, and any relation $r^{:A\leftrightarrow B}$, we
have: 
\[
(r,\text{id}^{:T\leftrightarrow T})^{\updownarrow H}=r^{\updownarrow G}\quad,\quad\quad\text{where we defined}:\quad G^{A}\triangleq H^{A,T}\quad.
\]

\textbf{(b)} Lifting one or more identity relations produces again
an identity relation: 
\[
\text{for any }G^{\bullet},H^{\bullet,\bullet},\text{etc}.:\quad(\text{id}^{:A\leftrightarrow A})^{\updownarrow G}=\text{id}^{:G^{A}\leftrightarrow G^{A}}\quad,\quad\quad(\text{id}^{:A\leftrightarrow A},\text{id}^{:X\leftrightarrow X})^{\updownarrow H}=\text{id}^{:H^{A,X}\leftrightarrow H^{A,X}}\quad,\quad\text{etc}.
\]


\subparagraph{Proof}

\textbf{(a)} Enumerate all cases of Definition~\ref{subsec:Definition-simultaneous-relational-lifting}
for $H^{\bullet,\bullet}$.

If $H^{A,X}\triangleq Z$ with a fixed type $Z$, we have also $G^{A}=Z$.
Then $(r,s)^{\updownarrow H}\triangleq\text{id}^{:Z\leftrightarrow Z}$
and $r^{\updownarrow G}=\text{id}^{:Z\leftrightarrow Z}$ for any
relations $r$, $s$. So, $(r,\text{id})^{\updownarrow H}=r^{\updownarrow G}$.

If $H^{A,X}\triangleq A$, we have $G^{A}=A$. Then $(r,s)^{\updownarrow H}\triangleq r$
and $r^{\updownarrow G}=r$ for any relation $r$, so we obtain $(r,\text{id})^{\updownarrow H}=r^{\updownarrow G}$.

The next three similar cases use the inductive assumptions $(r,\text{id})^{\updownarrow K}=r^{\updownarrow K^{\bullet,T}}$
and $(r,\text{id})^{\updownarrow L}=r^{\updownarrow L^{\bullet,T}}$:
\begin{align*}
{\color{greenunder}\text{if }H^{A,X}\triangleq K^{A,X}\times L^{A,X}:}\quad & (r,\text{id})^{\updownarrow H}=(r,\text{id})^{\updownarrow K}\boxtimes(r,\text{id})^{\updownarrow L}=r^{\updownarrow K^{\bullet,T}}\boxtimes r^{\updownarrow L^{\bullet,T}}=r^{\updownarrow(K^{\bullet,T}\times L^{\bullet,T})}=r^{\updownarrow G}\quad;\\
{\color{greenunder}\text{if }H^{A,X}\triangleq K^{A,X}+L^{A,X}:}\quad & (r,\text{id})^{\updownarrow H}=(r,\text{id})^{\updownarrow K}\boxplus(r,\text{id})^{\updownarrow L}=r^{\updownarrow K^{\bullet,T}}\boxplus r^{\updownarrow L^{\bullet,T}}=r^{\updownarrow(K^{\bullet,T}+L^{\bullet,T})}=r^{\updownarrow G}\quad;\\
{\color{greenunder}\text{if }H^{A,X}\triangleq K^{A,X}\rightarrow L^{A,X}:}\quad & (r,\text{id})^{\updownarrow H}=(r,\text{id})^{\updownarrow K}\ogreaterthan(r,\text{id})^{\updownarrow L}=r^{\updownarrow K^{\bullet,T}}\ogreaterthan r^{\updownarrow L^{\bullet,T}}=r^{\updownarrow(K^{\bullet,T}\rightarrow L^{\bullet,T})}=r^{\updownarrow G}\quad.
\end{align*}

If $H^{A,X}\triangleq S^{A,X,H^{A,X}}$ with a recursion scheme $S^{\bullet,\bullet,\bullet}$,
the type constructor $G^{\bullet}$ is defined by: 
\[
G^{A}\triangleq H^{A,T}=S^{A,T,H^{A,T}}=S^{A,T,G^{A}}\quad.
\]
So, the lifting to $G$ is given by $r^{\updownarrow G}=\big(r,\overline{r^{\updownarrow G}}\big)^{\updownarrow S^{\bullet,T,\bullet}}$.
Write the definition of lifting to $H$ and simplify:
\begin{align*}
 & (r,\text{id})^{\updownarrow H}=\big(r,\text{id},\overline{(r,\text{id})^{\updownarrow H}}\big)^{\updownarrow S}\\
{\color{greenunder}\text{inductive assumption }\overline{(r,\text{id})^{\updownarrow H}}=\overline{r^{\updownarrow G}}:}\quad & =\big(r,\gunderline{\text{id}},\overline{r^{\updownarrow G}}\big)^{\updownarrow S}\\
{\color{greenunder}\text{inductive assumption about lifting id to }S^{\bullet,\bullet,\bullet}:}\quad & =\big(r,\overline{r^{\updownarrow G}}\big)^{\updownarrow S^{\bullet,T,\bullet}}\\
{\color{greenunder}\text{definition of lifting to }G:}\quad & =r^{\updownarrow G}\quad.
\end{align*}

If $H^{A,X}\triangleq\forall Z.\,S^{Z,A,X}$, the type constructor
$G^{\bullet}$ is $G^{A}=\forall Z.\,S^{Z,A,T}$. The lifting $(r,\text{id})^{\updownarrow H}$
is a relation of type $\forall U.\,S^{U,A,X}\leftrightarrow\forall V.\,S^{V,A,X}$
written as:
\begin{align*}
 & \forall(U,V):\quad\big((r,\text{id})^{\updownarrow\forall Z.\,S^{Z,\bullet,\bullet}}\big)^{U,V}=\forall w^{:U\leftrightarrow V}.\,(w,r,\gunderline{\text{id}})^{\updownarrow S^{\bullet,\bullet,\bullet}}\\
{\color{greenunder}\text{omitting id from lifting to }S:}\quad & =\forall w^{:U\leftrightarrow V}.\,(w,r)^{\updownarrow S^{\bullet,\bullet,T}}\\
{\color{greenunder}\text{definition of lifting to }G:}\quad & =(r^{\updownarrow G})^{U,V}\quad.
\end{align*}
Here we used the inductive assumption that identity relations may
be omitted from liftings to $S^{\bullet,\bullet,\bullet}$.

\textbf{(b)} Enumerate all cases of Definition~\ref{subsec:Definition-relational-lifting}
for $G^{\bullet}$ or Definition~\ref{subsec:Definition-simultaneous-relational-lifting}
for $H^{\bullet,\bullet}$. The proofs are similar, so we will only
prove that $(\text{id},\text{id})^{\updownarrow H}=\text{id}$.

If $H^{A,X}\triangleq Z$ with a fixed type $Z$, we have $(r,s)^{\updownarrow H}\triangleq\text{id}^{:Z\leftrightarrow Z}$
for any relations $r$, $s$.

If $H^{A,X}\triangleq A$, we have $(r,s)^{\updownarrow H}\triangleq r$
for any relation $r$, so we obtain $(\text{id},\text{id})^{\updownarrow H}=\text{id}$.
Similarly if $H^{A,X}\triangleq X$ then we obtain $(\text{id},\text{id})^{\updownarrow H}=\text{id}$.

The next three similar cases use the inductive assumptions $(\text{id},\text{id})^{\updownarrow K}=\text{id}$
and $(\text{id},\text{id})^{\updownarrow L}=\text{id}$:
\begin{align*}
{\color{greenunder}\text{if }H^{\bullet,\bullet}\triangleq K^{\bullet,\bullet}\times L^{\bullet,\bullet}:}\quad & (\text{id},\text{id})^{\updownarrow H}=(\text{id},\text{id})^{\updownarrow K}\boxtimes(\text{id},\text{id})^{\updownarrow L}=\text{id}\boxtimes\text{id}\quad;\\
{\color{greenunder}\text{if }H^{\bullet,\bullet}\triangleq K^{\bullet,\bullet}+L^{\bullet,\bullet}:}\quad & (\text{id},\text{id})^{\updownarrow H}=(\text{id},\text{id})^{\updownarrow K}\boxplus(\text{id},\text{id})^{\updownarrow L}=\text{id}\boxplus\text{id}\quad;\\
{\color{greenunder}\text{if }H^{\bullet,\bullet}\triangleq K^{\bullet,\bullet}\rightarrow L^{\bullet,\bullet}:}\quad & (\text{id},\text{id})^{\updownarrow H}=(\text{id},\text{id})^{\updownarrow K}\ogreaterthan(\text{id},\text{id})^{\updownarrow L}=\text{id}\ogreaterthan\text{id}\quad.
\end{align*}
It follows from Example~\ref{subsec:Example-pair-product-pair-mapper-relation}
with $f=\text{id}$ and $g=\text{id}$ that:
\[
\text{id}\boxtimes\text{id}=\text{id}\quad,\quad\quad\text{id}\boxplus\text{id}=\text{id}\quad,\quad\quad\text{and}\quad\quad\text{id}\ogreaterthan\text{id}=\text{id}\quad.
\]
So, in all three cases we obtain: $(\text{id},\text{id})^{\updownarrow H}=\text{id}$.

If $H^{A,X}\triangleq S^{A,X,H^{A,X}}$ with a recursion scheme $S^{\bullet,\bullet,\bullet}$,
we have:
\[
(\text{id},\text{id})^{\updownarrow H}=\big(\text{id},\text{id},\overline{(\text{id},\text{id})^{\updownarrow H}}\big)^{\updownarrow S}=(\text{id},\text{id},\text{id})^{\updownarrow S}\quad,
\]
because by the inductive assumption the identity law holds for the
recursive call: $\overline{(\text{id},\text{id})^{\updownarrow H}}=\text{id}$.
Another inductive assumption is that the identity law holds for the
liftings to $S^{\bullet,\bullet,\bullet}$. So, we get: 
\[
(\text{id},\text{id})^{\updownarrow H}=(\text{id},\text{id},\text{id})^{\updownarrow S}=\text{id}\quad.
\]

If $H^{A,X}\triangleq\forall Z.\,S^{Z,A,X}$, the lifting $(\text{id},\text{id})^{\updownarrow H}$
is a relation of type $\forall U.\,S^{U,A,X}\leftrightarrow\forall V.\,S^{V,A,X}$
written as:
\[
\forall(U,V):\quad\big((\text{id},\text{id})^{\updownarrow\forall Z.\,S^{Z,\bullet,\bullet}}\big)^{U,V}=\forall w^{:U\leftrightarrow V}.\,(w,\text{id},\text{id})^{\updownarrow H^{\bullet,\bullet,\bullet}}\quad.
\]
As we showed in part \textbf{(a)}, any identity relation may be omitted
from the lifting:
\[
\forall w^{:U\leftrightarrow V}.\,(w,\gunderline{\text{id}^{:A\leftrightarrow A}},\gunderline{\text{id}^{:X\leftrightarrow X}})^{\updownarrow H^{\bullet,\bullet,\bullet}}=\forall w^{:U\leftrightarrow V}.\,w^{\updownarrow H^{\bullet,A,X}}\quad.
\]
By Statement~\ref{subsec:Statement-parametricity-gives-identity-relation}
used for $H^{\bullet,A,X}$ as a type constructor with one type parameter,
the relation $\forall w.\,w^{\updownarrow H^{\bullet,A,X}}$ is the
identity relation for the type $\forall Z.\,H^{Z,A,X}$. We find that
$(\text{id},\text{id})^{\updownarrow\forall Z.\,S^{Z,\bullet,\bullet}}=\text{id}$.
$\square$

\begin{comment}
commutativity law does not hold!
\end{comment}


\subsection{Strong dinaturality: definition and general properties\label{subsec:Strong-dinaturality.-General-properties}}

The dinaturality law is a useful shortcut for proofs involving parametricity.
However, certain more complicated cases require a stronger property
that we will now study. As motivation, we begin by looking at the
wedge law~(\ref{eq:wedge-law-for-profunctors}) in more detail.

We have seen in Statement~\ref{subsec:Statement-wedge-law-from-parametricity}
that fully parametric values automatically satisfy the wedge law.
The proof showed that the $\left(P,f\right)$-wedge relation is a
consequence of the relation $(x,y)\in\left<f\right>^{\updownarrow T}$
where $T^{A}\triangleq P^{A,A}$. Note that the type diagrams for
the wedge law~(\ref{eq:wedge-law-for-profunctors}), the dinaturality
law~(\ref{eq:dinaturality-law-for-profunctors}), and the commutativity
law~(\ref{eq:profunctor-commutativity-law}), involve three relations
between values $x^{:P^{A,A}}$ and $y^{:P^{B,B}}$:
\begin{align*}
{\color{greenunder}\text{a }\left(P,f\right)\text{-wedge relation}:}\quad & x\triangleright f^{\uparrow P^{A,\bullet}}=y\triangleright f^{\downarrow P^{\bullet,B}}\quad,\quad\text{or}:\quad(x,y)\in\text{pull}\,(f^{\uparrow P},f^{\downarrow P})\quad,\\
{\color{greenunder}\text{the left part of a dinaturality diagram}:}\quad & \exists z^{:P^{B,A}}\quad\text{such that}\quad x=z\triangleright f^{\downarrow P^{\bullet,A}}\text{ and }y=z\triangleright f^{\uparrow P^{B,\bullet}}\quad,\\
{\color{greenunder}\text{a relation }\left<f\right>\text{ lifted to }T:}\quad & (x,y)\in\left<f\right>^{\updownarrow T}\quad.
\end{align*}
To make the notation shorter, we define the \textbf{pushout} \textbf{relation}\index{pushout relation|textit}
denoted by $\text{push}\,(p,q)$:
\[
(x^{:A},y^{:B})\in\text{push}\,(p,q)\text{ means }\exists z^{:C}\text{ such that }x=p(z)\text{ and }y=q(z)\quad.
\]
Here $A$, $B$, $C$ are any types and $p^{:C\rightarrow A}$ and
$q^{:C\rightarrow A}$ are any given functions. The left fragment
of the dinaturality diagram is then written as $(x,y)\in\text{push}\,(f^{\downarrow P},f^{\uparrow P})$.

The pullback and pushout relations may be used to reformulate the
commutativity law~(\ref{eq:profunctor-commutativity-law}) as: 
\[
\text{for any }x^{:P^{A,A}},y^{:P^{B,B}}\quad:\quad\text{if}\quad(x,y)\in\text{push}\,(f^{\downarrow P},f^{\uparrow P})\quad\text{then}\quad(x,y)\in\text{pull}\,(f^{\uparrow P},f^{\downarrow P})\quad.
\]
We find that the $\left(P,f\right)$-wedge relation, denoted by $\text{pull}\,(f^{\uparrow P},f^{\downarrow P})$
, is a consequence of both the lifted relation $\left<f\right>^{\updownarrow T}$
and of the relation $\text{push}\,(f^{\downarrow P},f^{\uparrow P})$,
which we may call the $\left(P,f\right)$-pushout relation.

The $\left(P,f\right)$-pushout relation is generally stronger than
the lifted relation $\left<f\right>^{\updownarrow T}$:

\subsubsection{Statement \label{subsec:Statement-profunctor-pushout-entails-lifted-f}\ref{subsec:Statement-profunctor-pushout-entails-lifted-f}}

We assume a fully parametric profunctor $P^{\bullet,\bullet}$, arbitrary
types $X$, $Y$, and arbitrary values $x^{:P^{X,X}}$, $y^{:P^{Y,Y}}$,
and $f^{:X\rightarrow Y}$. If $(x,y)\in\text{push}\,(f^{\downarrow P},f^{\uparrow P})$
then $(x,y)\in\left<f\right>^{\updownarrow T}$ where $T^{A}\triangleq P^{A,A}$.

\subparagraph{Proof}

The relation $(x,y)\in\text{push}\,(f^{\downarrow P},f^{\uparrow P})$
means that there exists $z^{:P^{Y,X}}$ such that $x=z\triangleright f^{\downarrow P}$
and $y=z\triangleright f^{\uparrow P}$. Rewrite the expressions for
$x$ and $y$ using the \lstinline!xmap! method of $P$:
\[
x=\text{xmap}_{P}(f)(\text{id})(z)\quad,\quad\quad y=\text{xmap}_{P}(\text{id})(f)(z)\quad.
\]
We need to show that $(x,y)\in\left<f\right>^{\updownarrow T}$, which
we can write in terms of \lstinline!xmap! like this:
\begin{equation}
\forall z^{:P^{Y,X}}:\quad\big(\text{xmap}_{P}(f)(\text{id})(z),\,\text{xmap}_{P}(\text{id})(f)(z)\big)\in\left<f\right>^{\updownarrow T}\quad.\label{eq:pushout-relation-entails-lifted-derivation1}
\end{equation}
This equation is similar to the relational naturality law~(\ref{eq:relational-naturality-law-of-xmap})
of \lstinline!xmap! that was used in the proof of Statement~\ref{subsec:Statement-wedge-law-from-parametricity}:
\[
\forall(a,b,c,d)\quad:\quad(\text{xmap}_{P}^{A,B,C,D},\text{xmap}_{P}^{A^{\prime},B^{\prime},C^{\prime},D^{\prime}})\in(b\ogreaterthan a)\ogreaterthan(c\ogreaterthan d)\ogreaterthan(a,c)^{\updownarrow P}\ogreaterthan(b,d)^{\updownarrow P}\quad.
\]
As we did the proof of that statement, we will now choose specific
types and relations in this law in order to derive Eq.~(\ref{eq:pushout-relation-entails-lifted-derivation1}).
We begin by writing out the definition of $\ogreaterthan$:
\begin{align}
 & \forall\big(a^{:A\leftrightarrow A^{\prime}},b^{:B\leftrightarrow B^{\prime}},c^{:C\leftrightarrow C^{\prime}},d^{:D\leftrightarrow D^{\prime}},g^{:B\rightarrow A},h^{:B^{\prime}\rightarrow A^{\prime}},k^{:C\rightarrow D},l^{:C^{\prime}\rightarrow D^{\prime}},x^{:P^{A,C}},y^{:P^{A^{\prime},C^{\prime}}}\big)\quad:\nonumber \\
 & \quad\text{if}\quad\quad(g,h)\in b\ogreaterthan a\quad,\quad(k,l)\in c\ogreaterthan d\quad,\quad\text{and}\quad(x,y)\in(a,c)^{\updownarrow P}\quad,\nonumber \\
 & \quad\text{then}\quad\quad\big(\text{xmap}_{P}(g)(k)(x),\,\text{xmap}_{P}(h)(l)(y)\big)\in(b,d)^{\updownarrow P}\quad.\label{eq:xmap-relational-law-derivation3}
\end{align}

The conclusion of Eq.~(\ref{eq:pushout-relation-entails-lifted-derivation1})
is that two \lstinline!xmap(...)! values are in the relation $\left<f\right>^{\updownarrow T}$.
We can reproduce that conclusion using Eq.~(\ref{eq:xmap-relational-law-derivation3})
if we set $b=d=\left<f\right>$ since, by Statement~\ref{subsec:Statement-relational-lifting-consistency-PAA},
we will then have $(b,d)^{\updownarrow P}=(\left<f\right>,\left<f\right>)^{\updownarrow P}=\left<f\right>^{\updownarrow T}$.
The corresponding types need to be set as $B=D=X$ and $B^{\prime}=D^{\prime}=Y$.

Next, we need to find the values $g$, $h$, $k$, $l$, $x$, and
$y$ such that the \lstinline!xmap(...)! expressions in Eq.~(\ref{eq:xmap-relational-law-derivation3})
reproduce those in Eq.~(\ref{eq:pushout-relation-entails-lifted-derivation1}):
\[
\text{xmap}_{P}(g)(k)(x)=\text{xmap}_{P}(f)(\text{id})(z)\quad,\quad\quad\text{xmap}_{P}(h)(l)(y)=\text{xmap}_{P}(\text{id})(f)(z)\quad.
\]
It is clear that we need to set $g=l=f$, $h=\text{id}$, $k=\text{id}$,
and $x=y=z$. This also implies that we set the types as $B=C^{\prime}=X$,
$A=D^{\prime}=Y$, $A^{\prime}=B^{\prime}=Y$, and $C=D=X$.

Since we have $x=y$, the precondition $(x,y)\in(a,c)^{\updownarrow P}$
will hold if $(a,c)^{\updownarrow P}$ is an identity relation. To
achieve that, we choose $a=\text{id}$ and $c=\text{id}$, which also
means setting $A=A^{\prime}=Y$ and $C=C^{\prime}=X$. 

Finally, we can check that the preconditions $(g,h)\in b\ogreaterthan a$
and $(k,l)\in c\ogreaterthan d$ in Eq.~(\ref{eq:xmap-relational-law-derivation3})
will then hold automatically. So, we may use the conclusion of Eq.~(\ref{eq:xmap-relational-law-derivation3}),
which proves Eq.~(\ref{eq:pushout-relation-entails-lifted-derivation1}).
$\square$

It turns out that for some (but not all) profunctors $P$, the $\left(P,f\right)$-wedge
relation is \emph{equivalent} to the $\left(P,f\right)$-pushout relation:

\subsubsection{Definition \label{subsec:Definition-profunctor-pushout-property}\ref{subsec:Definition-profunctor-pushout-property}}

A profunctor $P^{X,Y}$ %
\begin{comment}
empty arrow
\end{comment}
{} has the \textbf{pushout property}\index{profunctor!pushout property}\index{pushout property of profunctors}
if the relation $\text{push}\,(f^{\downarrow P},f^{\uparrow P})$
is a

\begin{wrapfigure}{l}{0.415\columnwidth}%
\vspace{-2.3\baselineskip}
\[
\xymatrix{\xyScaleY{2.3pc}\xyScaleX{1.5pc} & P^{A,A}\ar[d]\sb(0.4){f^{\uparrow P^{A,\bullet}}} & \ar@{}[d]\sb(0.4){\displaystyle \Rightarrow} & P^{B,A}\ar[r]\sp(0.55){f^{\downarrow P^{\bullet,A}}}\ar[d]\sb(0.4){f^{\uparrow P^{B,\bullet}}} & P^{A,A}\ar[d]\sb(0.4){f^{\uparrow P^{A,\bullet}}}\\
P^{B,B}\ar[r]\sp(0.6){f^{\downarrow P^{\bullet,B}}} & P^{A,B} &  & P^{B,B}\ar[r]\sp(0.6){f^{\downarrow P^{\bullet,B}}} & P^{A,B}
}
\]

\vspace{-1.7\baselineskip}
\end{wrapfigure}%

\noindent consequence of $\text{pull}\,(f^{\uparrow P},f^{\downarrow P})$.
So, for any $x^{:P^{A,A}}$, $y^{:P^{B,B}}$:
\[
\text{if }(x,y)\in\text{pull}\,(f^{\uparrow P},f^{\downarrow P})\text{ then }(x,y)\in\text{push}\,(f^{\downarrow P},f^{\uparrow P})\quad.
\]

\noindent In other words, for any $f^{:A\rightarrow B}$ and any values
$x^{:P^{A,A}}$ and $y^{:P^{B,B}}$ in the $\left(P,f\right)$-wedge
relation ($x\triangleright f^{\uparrow P^{A,\bullet}}=y\triangleright f^{\downarrow P^{\bullet,B}}$),
we can compute a value $p^{:P^{B,A}}$ such that $x=p\triangleright f^{\downarrow P^{\bullet,A}}$
and $y=p\triangleright f^{\uparrow P^{B,\bullet}}$. The type diagram
means that a wedge starting with arbitrary $x^{:P^{A,A}}$ and $y^{:P^{B,B}}$
can be always completed to a square that represents the profunctor
commutativity law~(\ref{eq:profunctor-commutativity-law}). $\square$

Below we will perform structural analysis for profunctors $P$ with
the pushout property and give some examples. The significance of those
$P$ is that we can obtain a property that is stronger than ordinary
dinaturality:

\subsubsection{Definition \label{subsec:Definition-strong-dinaturality}\ref{subsec:Definition-strong-dinaturality}}

A function $t^{A}:P^{A,A}\rightarrow Q^{A,A}$ is \textbf{strongly
dinatural} \index{strong dinaturality law}if for any $f^{:A\rightarrow B}$,
$x^{:P^{A,A}}$, and $y^{:P^{B,B}}$ the following property holds:\vspace{-0.3\baselineskip}
\begin{equation}
\text{when}\quad x\triangleright f^{\uparrow P^{A,\bullet}}=y\triangleright f^{\downarrow P^{\bullet,B}}\quad\text{ then}\quad x\triangleright t^{A}\triangleright f^{\uparrow Q^{A,\bullet}}=y\triangleright t^{B}\triangleright f^{\downarrow Q^{\bullet,B}}\quad.\label{eq:strong-dinaturality-law}
\end{equation}

\begin{wrapfigure}{i}{0.32\columnwidth}%
\vspace{-2.5\baselineskip}
\[
\xymatrix{\xyScaleY{1.2pc}\xyScaleX{1.0pc}P^{A,A}\ar[rd]\sb(0.35){f^{\uparrow P^{A,\bullet}}\negthickspace\negthickspace}\ar[rr]\sp(0.5){t^{A}} &  & Q^{A,A}\ar[rd]\sp(0.5){f^{\uparrow Q^{A,\bullet}}}\\
 & P^{A,B} & \negthickspace\Rightarrow & Q^{A,B}\\
P^{B,B}\ar[ru]\sp(0.5){f^{\downarrow P^{\bullet,B}}\negthickspace\negthickspace\negthickspace}\ar[rr]\sp(0.5){t^{B}} &  & Q^{B,B}\ar[ru]\sb(0.65){\negthickspace\negthickspace f^{\downarrow Q^{\bullet,B}}}
}
\]
\vspace{-2\baselineskip}
\end{wrapfigure}%

\noindent The strong dinaturality law is an equation (the \textsf{``}conclusion\textsf{''})
that is required to hold only when some values satisfy another equation
(the law\textsf{'}s \textsf{``}precondition\textsf{''}). The type diagram for that law, shown
at left, is a truncated form of the type diagram for Eq.~(\ref{eq:dinaturality-law-for-profunctors}).
The diagram starts with two arbitrary values of types $P^{A,A}$ and
$P^{B,B}$. The implication symbol ($\Rightarrow$) here means that
the left part of the diagram is an assumption used by the right part.

In terms of the wedge relations, strong dinaturality\textsf{'}s \textsf{``}precondition\textsf{''}
is the $\left(P,f\right)$-wedge relation for $(x,y)$, and the \textsf{``}conclusion\textsf{''}
is the $\left(Q,f\right)$-wedge relation for $(t^{A}(x),t^{B}(y))$.
So, the strong dinaturality condition~(\ref{eq:strong-dinaturality-law})
can be expressed as a relation using the pair mapper ($\ogreaterthan$):
\[
(t^{A},t^{B})\in\text{pull}\big(f^{\uparrow P^{A,\bullet}},f^{\downarrow P^{\bullet,B}}\big)\ogreaterthan\text{pull}\big(f^{\uparrow Q^{A,\bullet}},f^{\downarrow Q^{\bullet,B}}\big)\quad.
\]
For comparison, the ordinary dinaturality property of $t$ is written
in a similar way as:
\[
(t^{A},t^{B})\in\text{push}\,(f^{\downarrow P^{\bullet,A}},f^{\uparrow P^{B,\bullet}})\ogreaterthan\text{pull}\big(f^{\uparrow Q^{A,\bullet}},f^{\downarrow Q^{\bullet,B}}\big)\quad.
\]


\subsubsection{Statement \label{subsec:Statement-strong-dinaturality-pushout}\ref{subsec:Statement-strong-dinaturality-pushout}}

If a profunctor $P$ has the pushout property and $Q$ is any profunctor
then any fully parametric function $t:\forall A.\,P^{A,A}\rightarrow Q^{A,A}$
is strongly dinatural.

\subparagraph{Proof}

By Statements~\ref{subsec:Statement-naturality-laws-from-wedge-law}
and~\ref{subsec:Statement-wedge-law-from-parametricity}, the function
$t$ satisfies the ordinary dinaturality law:
\[
\text{if}\quad(x^{:P^{A,A}},y^{:P^{B,B}})\in\text{push}\,(f^{\downarrow P},f^{\uparrow P})\quad\text{then}\quad(t^{A}(x),t^{B}(y))\in\text{pull}\,(f^{\uparrow Q},f^{\downarrow Q})\quad.
\]
The pushout property of $P$ means:
\[
\text{if}\quad(x^{:P^{A,A}},y^{:P^{B,B}})\in\text{pull}\,(f^{\uparrow P},f^{\downarrow P})\quad\text{then}\quad(x,y)\in\text{push}\,(f^{\downarrow P},f^{\uparrow P})\quad.
\]
Chaining the implications, we find:
\[
\text{if}\quad(x^{:P^{A,A}},y^{:P^{B,B}})\in\text{pull}\,(f^{\uparrow P},f^{\downarrow P})\quad\text{then}\quad(t^{A}(x),t^{B}(y))\in\text{pull}\,(f^{\uparrow Q},f^{\downarrow Q})\quad.
\]
This is the strong dinaturality law of $t$. $\square$

The name \textsf{``}strong dinaturality\textsf{''} suggests that this property is
stronger than the ordinary dinaturality. Indeed, strongly dinatural
transformations are always dinatural:

\subsubsection{Statement \label{subsec:Statement-strong-dinaturality-entails-dinaturality}\ref{subsec:Statement-strong-dinaturality-entails-dinaturality}}

\textbf{(a)} Any function $t^{A}:P^{A,A}\rightarrow Q^{A,A}$ satisfying
Eq.~(\ref{eq:strong-dinaturality-law}) will also satisfy Eq.~(\ref{eq:dinaturality-law-for-profunctors}).

\textbf{(b)} A natural transformation $t^{A}:F^{A}\rightarrow G^{A}$
(where $F$ and $G$ are both functors or both contrafunctors) is
strongly dinatural.

\textbf{(c)} A natural transformation $t^{X,Y}:P^{X,Y}\rightarrow Q^{X,Y}$
between profunctors $P$ and $Q$ gives a strongly dinatural transformation
$t^{A,A}:P^{A,A}\rightarrow Q^{A,A}$ if we set $X\triangleq A$ and
$Y\triangleq A$ in $t^{X,Y}$.

\subparagraph{Proof}

\textbf{(a)} The law~(\ref{eq:dinaturality-law-for-profunctors})
is an equality of functions of type $P^{B,A}\rightarrow Q^{A,B}$.
We will now show that those functions will give equal results when
applied to an arbitrary value $p:P^{B,A}$. Choose $x\triangleq p\triangleright f^{\downarrow P^{\bullet,A}}$
and $y\triangleq p\triangleright f^{\uparrow P^{B,\bullet}}$. The
precondition in Eq.~(\ref{eq:strong-dinaturality-law}) is satisfied
with these $x$ and $y$:
\begin{align*}
{\color{greenunder}\text{expect to equal }(y\triangleright f^{\downarrow P^{\bullet,B}}):}\quad & \gunderline x\triangleright f^{\uparrow P^{A,\bullet}}=p\,\gunderline{\triangleright\,f^{\downarrow P^{\bullet,A}}\triangleright f^{\uparrow P^{A,\bullet}}}\\
{\color{greenunder}\text{profunctor commutativity law of }P:}\quad & =p\triangleright f^{\uparrow P^{B,\bullet}}\triangleright f^{\downarrow P^{\bullet,B}}=y\triangleright f^{\downarrow P^{\bullet,B}}\quad.
\end{align*}
So, we can use the law~(\ref{eq:strong-dinaturality-law})\textsf{'}s conclusion
and obtain Eq.~(\ref{eq:dinaturality-law-for-profunctors}) applied
to $p$, completing the proof:
\begin{align*}
 & \gunderline x\triangleright t^{A}\triangleright f^{\uparrow Q^{A,\bullet}}\overset{!}{=}\gunderline y\triangleright t^{B}\triangleright f^{\downarrow Q^{\bullet,B}}\quad,\\
{\color{greenunder}\text{definitions of }x,y:}\quad & p\triangleright f^{\downarrow P^{\bullet,A}}\bef t^{A}\bef f^{\uparrow Q^{A,\bullet}}\overset{!}{=}p\triangleright f^{\uparrow P^{B,\bullet}}\bef t^{B}\bef f^{\downarrow Q^{\bullet,B}}\quad.
\end{align*}

\textbf{(b)} Consider the case where $F$ and $G$ are both functors.
We may view $t:F^{A}\rightarrow G^{A}$ as a function with the type
signature $t:P^{A,A}\rightarrow Q^{A,A}$ if we define the profunctors
$P^{X,Y}\triangleq F^{Y}$ and $Q^{X,Y}\triangleq G^{Y}$. Since $P^{X,Y}$
and $Q^{X,Y}$ ignore the type parameter $X$, we have the liftings
$f^{\uparrow P}=f^{\uparrow F}$, $f^{\downarrow P}=\text{id}$, $f^{\uparrow Q}=f^{\uparrow G}$,
and $f^{\downarrow Q}=\text{id}$. The strong dinaturality law of
$t$ is then written as:
\[
\text{when}\quad x^{:F^{A}}\triangleright f^{\uparrow F}=y^{:F^{B}}\quad\text{ then}\quad x\triangleright t\triangleright f^{\uparrow G}\overset{?}{=}y\triangleright t\quad.
\]
This is equivalent to $x\triangleright t\bef f^{\uparrow G}=x\triangleright f^{\uparrow F}\bef t$,
which holds by the assumed naturality of $t$.

A similar proof works when $F$ and $G$ are both contrafunctors.

We may also derive part \textbf{(b)} from part \textbf{(c)}.

\textbf{(c)} A natural transformation $t^{X,Y}:P^{X,Y}\rightarrow Q^{X,Y}$
satisfies naturality laws separately with respect to the type parameters
$X$ and $Y$. So, for all $f^{:A\rightarrow B}$ we have:
\[
f^{\uparrow P^{X,\bullet}}\bef t^{X,B}=t^{X,A}\bef f^{\uparrow Q^{X,\bullet}}\quad,\quad\quad f^{\downarrow P^{\bullet,Y}}\bef t^{A,Y}=t^{B,Y}\bef f^{\downarrow Q^{\bullet,Y}}\quad.
\]
We need to verify the strong dinaturality law of $t^{A,A}$:
\[
\text{when}\quad x^{:P^{A,A}}\triangleright f^{\uparrow P^{A,\bullet}}=y^{:P^{B,B}}\triangleright f^{\downarrow P^{\bullet,B}}\quad\text{ then}\quad x\triangleright t^{A,A}\bef f^{\uparrow Q^{A,\bullet}}\overset{?}{=}y\triangleright t^{B,B}\bef f^{\downarrow Q^{\bullet,B}}\quad.
\]
Using the precondition of that law and the naturality laws of $t^{X,Y}$,
rewrite the conclusion of that law:
\begin{align*}
 & x\triangleright\gunderline{t^{A,A}\bef f^{\uparrow Q^{A,\bullet}}}=\gunderline{x\triangleright f^{\uparrow P^{A,\bullet}}}\bef t^{A,B}=y\triangleright f^{\downarrow P^{\bullet,B}}\bef t^{A,B}\\
 & \overset{?}{=}y\triangleright\gunderline{t^{B,B}\bef f^{\downarrow Q^{\bullet,B}}}=y\triangleright f^{\downarrow P^{\bullet,B}}\bef t^{A,B}\quad.
\end{align*}
The two sides of the strong dinaturality law are now equal. $\square$

Let us look at some examples of profunctors to see which ones have
the pushout property.

\subsubsection{Example \label{subsec:Example-weak-pullback-property-1}\ref{subsec:Example-weak-pullback-property-1}}

\textbf{(a)} Suppose a profunctor $P^{X,Y}$ does not depend on the
type parameter $X$ (so, $P^{X,Y}\triangleq G^{Y}$ with some functor
$G$). Then $P$ has the pushout property. 

\textbf{(b)} The pushout property holds for profunctors $P^{X,Y}\triangleq H^{X}$,
where $H$ is a contrafunctor.

\subparagraph{Proof}

\textbf{(a)} With the choice $P^{X,Y}\triangleq G^{Y}$, we have the
liftings $f^{\uparrow P}=f^{\uparrow G}$ and $f^{\downarrow P}=f$.
The pushout property of $P^{X,Y}$ says that, for any $f^{:A\rightarrow B}$,
$x^{:G^{A}}$, $y^{:G^{B}}$:
\[
\text{when}\quad x\triangleright f^{\uparrow G}=y\quad\text{ then }\quad\exists p^{:G^{A}}\text{ such that }x=p\text{ and }y=p\triangleright f^{\uparrow G}\quad.
\]
This property is satisfied by choosing $p\triangleq x$.

\textbf{(b)} With the choice $P^{X,Y}\triangleq H^{X}$, we have the
liftings $f^{\uparrow P}=f$ and $f^{\downarrow P}=f^{\downarrow H}$.
The pushout property of $P^{X,Y}$ says that, for any $f^{:A\rightarrow B}$,
$x^{:H^{A}}$, $y^{:H^{B}}$:
\[
\text{when}\quad x=y\triangleright f^{\downarrow H}\quad\text{ then }\quad\exists p^{:H^{A}}\text{ such that }x=p\triangleright f^{\downarrow H}\text{ and }y=p\quad.
\]
This property is satisfied by choosing $p\triangleq y$.

\subsubsection{Example \label{subsec:Example-weak-pullback-property}\ref{subsec:Example-weak-pullback-property}}

The profunctor $P^{X,Y}\triangleq X\rightarrow Y$ does \emph{not}
have the pushout property.

\subparagraph{Proof}

With $P^{X,Y}\triangleq X\rightarrow Y$, we rewrite the $\left(P,f\right)$-wedge
relation ($x\triangleright f^{\uparrow P^{A,\bullet}}=y\triangleright f^{\downarrow P^{\bullet,B}}$)
for arbitrary $f^{:A\rightarrow B}$, $x^{:A\rightarrow A}$, and
$y^{:B\rightarrow B}$ as:
\[
x\bef f=f\bef y\quad.
\]
The pushout property requires us to find a value $p^{:B\rightarrow A}$
such that $x=f\bef p$ and $y=p\bef f$. Choose $f$ as a constant
function ($f\triangleq\_\rightarrow b_{0}$ with a fixed value $b_{0}^{:B}$).
The $\left(P,f\right)$-wedge relation applied to an arbitrary value
$a^{:A}$ gives:
\[
a\triangleright x\bef f=b_{0}\overset{!}{=}a\triangleright f\bef y=y(b_{0})\quad.
\]
So, the wedge relation will hold for any $x^{:A\rightarrow A}$ and
for any $y^{:B\rightarrow B}$ such that $y(b_{0})=b_{0}$. In particular,
$y$ is not necessarily a constant function. But the condition $y=p\bef f=\_\rightarrow b_{0}$
can be satisfied only if $y$ is a constant function. So, there exists
no suitable value $p^{:P^{B,A}}$. $\square$

Structural analysis allows us to discover profunctors that have the
pushout property:

\subsubsection{Statement \label{subsec:Statement-weak-pullback-property}\ref{subsec:Statement-weak-pullback-property}}

A profunctor $P^{X,Y}$ has the pushout property if:

\textbf{(a)} $P^{X,Y}\triangleq F^{Y}$ with some (covariant) functor
$F$.

\textbf{(b)} $P^{X,Y}\triangleq G^{X}$ with some contrafunctor $G$.

\textbf{(c)} $P^{X,Y}\triangleq Z\rightarrow Q^{X,Y}$ with a fixed
type $Z$ and a profunctor $Q$ that has the pushout property.

\textbf{(d)} $P^{X,Y}\triangleq Q^{X,Y}\times R^{X,Y}$ with profunctors
$Q$ and $R$ that both have the pushout property.

\textbf{(e)} $P^{X,Y}\triangleq Q^{X,Y}+R^{X,Y}$ with profunctors
$Q$ and $R$ that both have the pushout property.

\subparagraph{Proof}

Consider some values $f^{:A\rightarrow B}$, $x^{:P^{A,A}}$, and
$y^{:P^{B,B}}$, for which the $\left(P,f\right)$-wedge relation
holds:
\[
x\triangleright f^{\uparrow P^{A,\bullet}}=y\triangleright f^{\downarrow P^{\bullet,B}}\quad.
\]

\textbf{(a)} If $P^{X,Y}=F^{Y}$ with a functor $F$, the wedge relation
is simplified to $x\triangleright f^{\uparrow F}=y$ with $x^{:F^{A}}$
and $y^{:F^{B}}$. We need to find $p^{:F^{A}}$ such that $x=p$
and $y=x\triangleright f^{\uparrow F}$. So, we define $p\triangleq x$
and complete the commuting square.

\textbf{(b)} If $P^{X,Y}=G^{X}$ with a contrafunctor $G$, the wedge
relation is simplified to $x=y\triangleright f^{\downarrow G}$ with
$x^{:G^{A}}$ and $y^{:G^{B}}$. We need to find $p^{:G^{B}}$ such
that $x=p\triangleright f^{\downarrow G}$ and $y=p$. So, we define
$p\triangleq y$ and complete the commuting square.

\textbf{(c)} Write the $\left(P,f\right)$-wedge relation for $x^{:Z\rightarrow Q^{A,A}}$
and $y^{:Z\rightarrow Q^{B,B}}$, applied to an arbitrary $z^{:Z}$:
\[
z\triangleright x\triangleright f^{\uparrow Q^{A,\bullet}}=z\triangleright y\triangleright f^{\downarrow Q^{\bullet,B}}\quad.
\]
Note that this is the same as the $\left(Q,f\right)$-wedge relation
for the values $z\triangleright x$ and $z\triangleright y$. By assumption,
$Q^{X,Y}$ has the pushout property. So, we may apply that property
to $z\triangleright x$ and $z\triangleright y$ and obtain a value
$q^{:Q^{B,A}}$ that satisfies the following two equations:
\begin{equation}
q\triangleright f^{\uparrow Q^{B,\bullet}}=z\triangleright y\quad,\quad\quad q\triangleright f^{\downarrow Q^{\bullet,A}}=z\triangleright x\quad.\label{eq:weak-pullback-c-derivation1}
\end{equation}
We need to find $p^{:Z\rightarrow Q^{B,A}}$ such that $x=p\bef f^{\downarrow Q^{\bullet,A}}$
and $y=p\bef f^{\uparrow Q^{B,\bullet}}$. These equations are the
same as Eq.~(\ref{eq:weak-pullback-c-derivation1}) if we define
$p(z)\triangleq q$. We have found a value $p$ that completes the
square diagram.

\textbf{(d)} Write the wedge relation for $x^{:Q^{A,A}\times R^{A,A}}=x_{1}^{:Q^{A,A}}\times x_{2}^{:R^{A,A}}$
and $y^{:Q^{B,B}\times R^{B,B}}=y_{1}^{:Q^{B,B}}\times y_{2}^{:R^{B,B}}$
as:
\[
(x_{1}\times x_{2})\triangleright f^{\uparrow(Q\times R)}=(x_{1}\triangleright f^{\uparrow Q})\times(x_{2}\triangleright f^{\uparrow R})\overset{!}{=}(y_{1}\times y_{2})\triangleright f^{\downarrow(Q\times R)}=(y_{1}\triangleright f^{\downarrow Q})\times(y_{2}\triangleright f^{\downarrow R})\quad.
\]
This equation is equivalent to the two equations:
\[
x_{1}\triangleright f^{\uparrow Q}\overset{!}{=}y_{1}\triangleright f^{\downarrow Q}\quad\text{ and }\quad x_{2}\triangleright f^{\uparrow R}\overset{!}{=}y_{2}\triangleright f^{\downarrow R}\quad,
\]
which are the $\left(Q,f\right)$- and $\left(R,f\right)$-wedge relations.
By assumption, $Q$ and $R$ have the pushout property. So, we may
compute values $p_{1}^{:Q^{B,A}}$ and $p_{2}^{:R^{B,A}}$ such that:
\[
p_{1}\triangleright f^{\downarrow Q}=x_{1}\quad,\quad\quad p_{1}\triangleright f^{\uparrow Q}=y_{1}\quad,\quad\quad p_{2}\triangleright f^{\downarrow R}=x_{2}\quad,\quad\quad p_{2}\triangleright f^{\uparrow R}=y_{2}\quad.
\]
If we define $p\triangleq p_{1}\times p_{2}$, we complete the square
diagram for $\left(Q\times R,f\right)$-wedge relation because:
\[
p\triangleright f^{\downarrow(Q\times R)}=(p_{1}\triangleright f^{\downarrow Q})\times(p_{2}\triangleright f^{\downarrow R})=x_{1}\times x_{2}\quad,\quad\quad p\triangleright f^{\uparrow(Q\times R)}=(p_{1}\triangleright f^{\uparrow Q})\times(p_{2}\triangleright f^{\uparrow R})=y_{1}\times y_{2}\quad.
\]

\textbf{(e)} Write the wedge relation for $x^{:Q^{A,A}+R^{A,A}}$
and $y^{:Q^{B,B}+R^{B,B}}$ as:
\[
x\triangleright\,\begin{array}{|c||cc|}
 & Q^{A,B} & R^{A,B}\\
\hline Q^{A,A} & f^{\uparrow Q^{A,\bullet}} & \bbnum 0\\
R^{A,A} & \bbnum 0 & f^{\uparrow R^{A,\bullet}}
\end{array}\,=y\triangleright\,\begin{array}{|c||cc|}
 & Q^{A,B} & R^{A,B}\\
\hline Q^{B,B} & f^{\downarrow Q^{\bullet,B}} & \bbnum 0\\
R^{B,B} & \bbnum 0 & f^{\downarrow R^{\bullet,B}}
\end{array}\quad.
\]
This equation can be satisfied only if both sides are either of type
$Q^{A,B}+\bbnum 0$ or of type $\bbnum 0+R^{A,B}$. Since the lifting
matrices are diagonal, this can happen only if $x$ has type $Q^{A,A}+\bbnum 0$
and $y$ has type $Q^{B,B}+\bbnum 0$, of if $x$ has type $\bbnum 0+R^{A,A}$
and $y$ has type $\bbnum 0+R^{B,B}$. In the first case, the property
becomes equivalent to that of the profunctor $Q$; in the second case,
to that of the profunctor $R$.

To see this in detail, consider two cases: $x\triangleq x_{1}^{:Q^{A,A}}+\bbnum 0$
and $x\triangleq\bbnum 0+x_{2}^{:R^{A,A}}$. In the first case, $x\triangleright f^{\uparrow P}=(x_{1}\triangleright f^{\uparrow Q})+\bbnum 0$.
This can be equal to $y\triangleright f^{\downarrow P}$ only if $y\triangleright f^{\downarrow P}$
is in the left part of the disjunction type $Q^{A,B}+R^{A,B}$. This
happens only when $y=y_{1}^{:Q^{B,B}}+\bbnum 0$ with some $y_{1}$.
So, the $\left(P,f\right)$-wedge relation implies $x_{1}\triangleright f^{\uparrow Q}=y_{1}\triangleright f^{\downarrow Q}$.
This is the $\left(Q,f\right)$-wedge relation for $(x_{1},y_{1})$.
By assumption, $Q$ has the pushout property. So, we can compute some
$p_{1}^{:Q^{B,A}}$ such that $x_{1}=p_{1}\triangleright f^{\downarrow Q}$
and $y_{1}=p_{1}\triangleright f^{\uparrow Q}$. If we now define
$p\triangleq p_{1}+\bbnum 0$, we will have $p\triangleright f^{\downarrow P}=x$
and $p\triangleright f^{\uparrow P}=y$, so the pushout property of
$P$ holds.

The case $x\triangleq\bbnum 0+x_{2}^{:R^{A,A}}$ is proved similarly
by using the pushout property of $R$. %
\begin{comment}
\textbf{(f)} Rewrite the wedge relation for $x^{:S^{A,A,P^{A,A}}}$
and $y^{:S^{B,B,P^{B,B}}}$ by using the explicit liftings to $S$:
\[
x\triangleright f^{\uparrow S^{A,\bullet,P^{A,A}}}\bef\big(\overline{f^{\uparrow P^{A,\bullet}}}\big)^{\uparrow S^{A,B,\bullet}}=y\triangleright f^{\downarrow S^{\bullet,B,P^{B,B}}}\bef\big(\overline{f^{\downarrow P^{\bullet,B}}}\big)^{\uparrow S^{A,B,\bullet}}\quad.
\]
Due to the commutativity law of $S$, we may exchange the order of
compositions here:
\[
f^{\uparrow S^{A,\bullet,P^{A,B}}}\big(x\triangleright(f^{\uparrow P^{A,\bullet}})^{\uparrow S^{A,A,\bullet}}\big)=y\triangleright\big(\overline{f^{\downarrow P^{\bullet,B}}}\big)^{\uparrow S^{B,B,\bullet}}\triangleright f^{\downarrow S^{\bullet,B,P^{A,B}}}\quad.
\]
Now we can use the pushout property of $S$ to obtain some $z:S^{B,A,P^{A,B}}$
such that
\[
x\triangleright\big(\overline{f^{\uparrow P^{A,\bullet}}}\big)^{\uparrow S^{A,A,\bullet}}=z\triangleright f^{\downarrow S^{\bullet,A,P^{A,B}}}\text{ and }y\triangleright\big(\overline{f^{\downarrow P^{\bullet,B}}}\big)^{\uparrow S^{B,B,\bullet}}=z\triangleright f^{\uparrow S^{B,\bullet,P^{A,B}}}\quad.
\]
We need to produce a value $t:S^{B,A,P^{B,A}}$ such that 
\[
x=t\triangleright f^{\downarrow S^{\bullet,A,P^{B,A}}}\bef\big(\overline{f^{\downarrow P^{\bullet,A}}}\big)^{\uparrow S^{A,A,\bullet}}\text{ and }y=t\triangleright f^{\uparrow S^{B,\bullet,P^{B,A}}}\bef\big(\overline{f^{\uparrow P^{B,\bullet}}}\big)^{\uparrow S^{B,B,\bullet}}\quad.
\]
\end{comment}
$\square$

The list of constructions in Statement~\ref{subsec:Statement-weak-pullback-property}
does not include the recursive type construction. It remains an open
question whether, say, a suitably limited form of the recursive type
construction would produce new profunctors having the pushout property.

However, profunctors $P$ with the pushout property are not the only
ones that produce strongly dinatural transformations. This is because
the pushout property is too restrictive. It is sufficient if the lifted
relation $\left<f\right>^{\updownarrow P}$ is a \emph{consequence}
of the $\left(P,f\right)$-wedge relation. We call this the \textsf{``}post-wedge\textsf{''}
property\index{profunctor!post-wedge property} of the profunctor
$P$.

Here and below, we will frequently need to use liftings $r^{\updownarrow T}$
to a type constructor $T^{\bullet}$ defined as $T^{A}\triangleq P^{A,A}$
with some profunctor $P^{X,Y}$. Statement~\ref{subsec:Statement-relational-lifting-consistency-PAA}
shows that $r^{\updownarrow T}=(r,r)^{\updownarrow P}$. So, we will
write the simultaneous lifting $(r,r)^{\updownarrow P}$ simply as
$r^{\updownarrow P}$ when this does not cause confusion.

\subsubsection{Definition \label{subsec:Definition-pre-post-wedge-property}\ref{subsec:Definition-pre-post-wedge-property}}

A profunctor $P$ has the \textbf{post-wedge property} if for any
$f^{:A\rightarrow B}$, $x^{:P^{A,A}}$, and $y^{:P^{B,B}}$:
\[
\text{if }\quad x\triangleright f^{\uparrow P}=x\triangleright f^{\downarrow P}\quad\text{ then }\quad(x,y)\in\left<f\right>^{\updownarrow P}\quad.
\]
In other words, the lifted relation $\left<f\right>^{\updownarrow P}$
always follows from the $\left(P,f\right)$-wedge relation. $\square$

By Statement~\ref{subsec:Statement-wedge-law-from-parametricity}(a),
the $\left(P,f\right)$-wedge relation is always a consequence of
the lifted relation $\left<f\right>^{\updownarrow P}$. So, the post-wedge
property of $P$ means that $\left<f\right>^{\updownarrow P}$ is
\emph{equivalent} to the $\left(P,f\right)$-wedge relation.

\subsubsection{Statement \label{subsec:Statement-post-wedge-entails-strong-dinaturality}\ref{subsec:Statement-post-wedge-entails-strong-dinaturality}}

If $P$ is a profunctor with the post-wedge property and $Q$ is any
profunctor then any fully parametric function $t:\forall A.\,P^{A,A}\rightarrow Q^{A,A}$
is strongly dinatural.

\subparagraph{Proof}

The post-wedge property of $P$ means:
\[
\text{if}\quad(x^{:P^{A,A}},y^{:P^{B,B}})\in\text{pull}\,(f^{\uparrow P},f^{\downarrow P})\quad\text{then}\quad(x,y)\in\left<f\right>^{\updownarrow P}\quad.
\]
Since $t$ is fully parametric, it satisfies the relational naturality
law:
\[
\text{if}\quad(x,y)\in\left<f\right>^{\updownarrow P}\quad\text{then}\quad(t^{A}(x),t^{B}(y))\in\left<f\right>^{\updownarrow Q}\quad.
\]
By Statement~\ref{subsec:Statement-wedge-law-from-parametricity}(a),
the wedge law follows from the relational naturality law of $t$:
\[
\text{if}\quad(t^{A}(x),t^{B}(y))\in\left<f\right>^{\updownarrow Q}\quad\text{then}\quad(t^{A}(x),t^{B}(y))\in\text{pull}\,(f^{\uparrow Q},f^{\downarrow Q})\quad.
\]
Chaining the implications, we find:
\[
\text{if}\quad(x^{:P^{A,A}},y^{:P^{B,B}})\in\text{pull}\,(f^{\uparrow P},f^{\downarrow P})\quad\text{then}\quad(t^{A}(x),t^{B}(y))\in\text{pull}\,(f^{\uparrow Q},f^{\downarrow Q})\quad.
\]
This is the strong dinaturality law of $t$. $\square$

We will now do structural analysis to describe the profunctors with
the post-wedge property.\footnote{The following derivations are based on the talk slides: \texttt{\href{https://www.ioc.ee/~tarmo/tday-voore/vene-slides.pdf}{https://www.ioc.ee/$\sim$tarmo/tday-voore/vene-slides.pdf}}} 

\subsubsection{Statement \label{subsec:Statement-post-wedge}\ref{subsec:Statement-post-wedge}}

A profunctor $P$ will have the post-wedge property if:

\textbf{(a)} The type expression $P^{X,Y}$ does not depend either
on $X$ or on $Y$. That is, either $P^{X,Y}\triangleq Q^{Y}$ where
$Q$ is a functor or $P^{X,Y}\triangleq R^{X}$ where $R$ is a contrafunctor. 

\textbf{(b)} We have $P^{X,Y}\triangleq K^{X,Y}\times L^{X,Y}$, where
the profunctors $K$ and $L$ have the post-wedge property.

\textbf{(c)} We have $P^{X,Y}\triangleq K^{X,Y}+L^{X,Y}$, where the
profunctors $K$ and $L$ have the post-wedge property.

\textbf{(d)} We have $P^{X,Y}\triangleq K^{Y,X}\rightarrow L^{X,Y}$,
where the profunctor $K$ has the pushout property and the profunctor
$L$ has the post-wedge property.%
\begin{comment}
\textbf{(e)} We have a recursive type $P^{X,Y}\triangleq S^{X,Y,P^{X,Y}}$,
where $S^{X,Y,R}$ is contravariant in $X$ and covariant in $Y$
and $R$, and has the post-wedge property when viewed as a profunctor
with respect to $X$ and $Y$. Does the post-wedge really hold for
$P$?
\end{comment}


\subparagraph{Proof}

In each case, assuming the $\left(P,f\right)$-wedge relation for
$(x,y)$, we will show that $(x,y)\in\left<f\right>^{\updownarrow P}$. 

\textbf{(a)} If $P^{X,Y}$ does not depend on one of its type parameters
then we have either $P^{X,Y}\triangleq Q^{Y}$ or $P^{X,Y}\triangleq R^{X}$.
We know from Statement~\ref{subsec:Statement-weak-pullback-property}(a,
b) that $P$ will then have the pushout property. So, if any values
$x$ and $y$ are in the $\left(P,f\right)$-wedge relation then $(x,y)\in\text{push}\,(f^{\downarrow P},f^{\uparrow P})$.
By Statement~\ref{subsec:Statement-profunctor-pushout-entails-lifted-f},
we will also have $(x,y)\in\left<f\right>^{\uparrow P}$.

\textbf{(b)} We need to show that $P$ has the post-wedge property:
\[
\text{if }\quad(k_{1}^{:K^{A,A}}\times l_{1}^{:L^{A,A}})\triangleright f^{\uparrow P}=(k_{2}^{:K^{B,B}}\times l_{2}^{:L^{B,B}})\triangleright f^{\downarrow P}\quad\text{ then }\quad(k_{1}\times l_{1},k_{2}\times l_{2})\in\left<f\right>^{\updownarrow P}\quad.
\]
Using the definitions of the liftings $^{\updownarrow P}$, $^{\uparrow P}$,
and $^{\downarrow P}$, we rewrite the above condition as:
\[
\text{if }\quad k_{1}\triangleright f^{\uparrow K}=k_{2}\triangleright f^{\downarrow K}\text{ and }l_{1}\triangleright f^{\uparrow L}=l_{2}\triangleright f^{\downarrow L}\quad\text{ then }\quad(k_{1},k_{2})\in\left<f\right>^{\updownarrow K}\text{ and }(l_{1},l_{2})\in\left<f\right>^{\updownarrow L}\quad.
\]
This is the same as the conjunction of the post-wedge properties of
$K$ and $L$. 

\textbf{(c)} We need to show that $P$ has the post-wedge property:
\[
\text{if }\quad p_{1}^{:K^{A,A}+L^{A,A}}\triangleright f^{\uparrow P}=p_{2}^{:K^{B,B}+L^{B,B}}\triangleright f^{\downarrow P}\quad\text{ then }\quad(p_{1},p_{2})\in\left<f\right>^{\updownarrow P}\quad.
\]
The liftings $f^{\uparrow P}$ and $f^{\downarrow P}$ are defined
via the standard pattern-matching code for disjunctive types:
\[
f^{\uparrow P^{A,\bullet}}\triangleq\,\begin{array}{|c||cc|}
 & K^{A,B} & L^{A,B}\\
\hline K^{A,A} & f^{\uparrow K^{A,\bullet}} & \bbnum 0\\
L^{A,A} & \bbnum 0 & f^{\uparrow L^{A,\bullet}}
\end{array}\quad,\quad\quad f^{\downarrow P^{\bullet,B}}\triangleq\,\begin{array}{|c||cc|}
 & K^{A,B} & L^{A,B}\\
\hline K^{B,B} & f^{\downarrow K^{\bullet,B}} & \bbnum 0\\
L^{B,B} & \bbnum 0 & f^{\downarrow L^{\bullet,B}}
\end{array}\quad.
\]
Since both code matrices are diagonal, the two parts of the disjunctive
type $K+L$ do not mix. It follows that $p_{1}\triangleright f^{\uparrow P}=p_{2}\triangleright f^{\downarrow P}$
only if both $p_{1}$ and $p_{2}$ are in the same part of the disjunction
$K+L$. If both $p_{1}$ and $p_{2}$ are in the left part of the
disjunction, we have $p_{1}\triangleq k_{1}^{:K^{A,A}}+\bbnum 0$
and $p_{2}\triangleq k_{2}^{:K^{B,B}}+\bbnum 0$, and the $\left(P,f\right)$-wedge
relation for $p_{1}$ and $p_{2}$ reduces to the $\left(K,f\right)$-wedge
relation for $k_{1}$ and $k_{2}$:
\[
k_{1}\triangleright f^{\uparrow K}=k_{2}\triangleright f^{\downarrow K}\quad.
\]
By Definition~\ref{subsec:Definition-relational-lifting}(d), the
values $p_{1}$ and $p_{2}$ will be in the relation $\left<f\right>^{\updownarrow P}$
only if $(k_{1},k_{2})\in\left<f\right>^{\updownarrow K}$. Then the
post-wedge property of $P$ becomes:
\[
\text{if }\quad k_{1}\triangleright f^{\uparrow K}=k_{2}\triangleright f^{\downarrow K}\quad\text{ then }\quad(k_{1},k_{2})\in\left<f\right>^{\updownarrow K}\quad.
\]
This holds because it is just the post-wedge property of $K$. Similarly,
we prove that if both $p_{1}$ and $p_{2}$ are in the right part
of the disjunction then the post-wedge property of $P$ is reduced
to the post-wedge property of $L$.

\textbf{(d)} Express the $\left(P,f\right)$-wedge relation and the
relation $\left<f\right>^{\updownarrow P}$ for $(x,y)$ through liftings
to $K$ and $L$:
\begin{align}
 & x^{:K^{A,A}\rightarrow L^{A,A}}\triangleright f^{\uparrow P^{A,\bullet}}=y^{:K^{B,B}\rightarrow L^{B,B}}\triangleright f^{\downarrow P^{\bullet,B}}\quad\text{or equivalently}:\quad f^{\downarrow K^{\bullet,A}}\bef x\bef f^{\uparrow L^{A,\bullet}}=f^{\uparrow K^{B,\bullet}}\bef y\bef f^{\downarrow L^{\bullet,B}}\quad,\label{eq:p-f-wedge-relation-x-y-derivation1}\\
 & (x^{:K^{A,A}\rightarrow L^{A,A}},y^{:K^{B,B}\rightarrow L^{B,B}})\in\left<f\right>^{\updownarrow P}\text{ means if }(k_{1}^{:K^{A,A}},k_{2}^{:K^{B,B}})\in\left<f\right>^{\updownarrow K}\text{ then }(x(k_{1}),y(k_{2}))\in\left<f\right>^{\updownarrow L}\quad.\nonumber 
\end{align}
Fix any $k_{1}^{:K^{A,A}}$ and $k_{2}^{:K^{B,B}}$ such that $(k_{1},k_{2})\in\left<f\right>^{\updownarrow K}$.
We need to prove that $(x(k_{1}),y(k_{2}))\in\left<f\right>^{\updownarrow L}$
assuming Eq.~(\ref{eq:p-f-wedge-relation-x-y-derivation1}).

The pushout property of $K$ gives:
\[
\text{if }\quad k_{1}^{:K^{A,A}}\triangleright f^{\uparrow K}=k_{2}^{:K^{B,B}}\triangleright f^{\downarrow K}\quad\text{ then }\quad\exists k_{0}^{:K^{B,A}}\text{ such that }k_{1}=k_{0}\triangleright f^{\downarrow K^{\bullet,A}}\text{ and }k_{2}=k_{0}\triangleright f^{\uparrow K^{B,\bullet}}\quad.
\]
So, there exists a suitable value $k_{0}^{:K^{B,A}}$.%
\begin{comment}
By Statement~\ref{subsec:Statement-wedge-law-from-parametricity}(a),
the values $k_{1}$, $k_{2}$ are in the $\left(P,f\right)$-wedge
relation: $k_{1}\triangleright f^{\uparrow K}=k_{2}\triangleright f^{\downarrow K}$.
\end{comment}
{} Apply both sides of Eq.~(\ref{eq:p-f-wedge-relation-x-y-derivation1})
to that $k_{0}$:
\[
k_{0}\triangleright f^{\downarrow K^{\bullet,A}}\bef x\bef f^{\uparrow L^{A,\bullet}}=k_{0}\triangleright f^{\uparrow K^{B,\bullet}}\bef y\bef f^{\downarrow L^{\bullet,B}}\quad\text{or equivalently}:\quad k_{1}\triangleright x\triangleright f^{\uparrow L}=k_{2}\triangleright y\triangleright f^{\downarrow L}\quad.
\]
 It is also given that $L$ has the post-wedge property:
\[
\text{if }\quad l_{1}^{:L^{A,A}}\triangleright f^{\uparrow L}=l_{2}^{:L^{B,B}}\triangleright f^{\downarrow L}\quad\text{ then }\quad(l_{1},l_{2})\in\left<f\right>^{\updownarrow L}\quad.
\]
We now set $l_{1}\triangleq k_{1}\triangleright x$ and $l_{2}\triangleq k_{2}\triangleright y$
to obtain $(x(k_{1}),y(k_{2}))\in\left<f\right>^{\updownarrow L}$
as required. $\square$

As a consequence of Statement~\ref{subsec:Statement-post-wedge},
we can quickly prove (without structural analysis) that a function
graph relation lifted to functors or contrafunctors will yield another
function graph relation:

\subsubsection{Statement \label{subsec:Statement-lifting-function-relation-covariant-1}\ref{subsec:Statement-lifting-function-relation-covariant-1}}

Lifting a function $f^{:A\rightarrow B}$ to a fully parametric functor
or a contrafunctor $G$ agrees with lifting the function graph relation
$\left<f\right>$ to $G$. In detail: \textbf{(a)} $\left<f\right>^{\updownarrow G}=\langle f^{\uparrow G}\rangle$
if $G$ is a functor. \textbf{(b)} $\left<f\right>^{\updownarrow G}=\text{rev}\langle f^{\downarrow G}\rangle$
if $G$ is a contrafunctor.

\subparagraph{Proof}

Whether $G$ is a functor or a contrafunctor, Statement~\ref{subsec:Statement-post-wedge}(a)
shows that it has the post-wedge property. So, the relation $(\left<f\right>,\left<f\right>)^{\updownarrow P}$
is \emph{equivalent} to the $\left(P,f\right)$-wedge relation:
\[
(x,y)\in(\left<f\right>,\left<f\right>)^{\updownarrow P}\quad\text{is equivalent to}:\quad x\triangleright f^{\uparrow P}=y\triangleright f^{\downarrow P}\quad.
\]

\textbf{(a)} If $G$ is a functor, define $P^{X,Y}\triangleq G^{Y}$.
Simplifying $(\left<f\right>,\left<f\right>)^{\updownarrow P}=\left<f\right>^{\updownarrow G}$
and using the liftings $f^{\uparrow P}=f^{\uparrow G}$ and $f^{\downarrow P}=\text{id}$,
we get:
\[
(x,y)\in\left<f\right>^{\updownarrow G}\quad\text{is equivalent to}:\quad x\triangleright f^{\uparrow G}=y\quad\text{or equivalently}:\quad(x,y)\in\langle f^{\uparrow G}\rangle\quad.
\]
So, we find $\left<f\right>^{\updownarrow G}=\langle f^{\uparrow G}\rangle$
as required.

\textbf{(b)} If $G$ is a contrafunctor, define $P^{X,Y}\triangleq G^{X}$.
Simplifying $(\left<f\right>,\left<f\right>)^{\updownarrow P}=\left<f\right>^{\updownarrow G}$
and using the liftings $f^{\uparrow P}=\text{id}$ and $f^{\downarrow P}=f^{\downarrow G}$,
we get:
\[
(x,y)\in\left<f\right>^{\updownarrow G}\quad\text{is equivalent to}:\quad x=y\triangleright f^{\downarrow G}\quad\text{or equivalently}:\quad(x,y)\in\text{rev}\langle f^{\downarrow G}\rangle\quad.
\]
 So, we find $\left<f\right>^{\updownarrow G}=\text{rev}\langle f^{\downarrow G}\rangle$
as required. $\square$

Strong dinaturality holds only for type signatures $\forall A.\,K^{A,A}\rightarrow L^{A,A}$
where the profunctor $K$ has a certain structure. Nevertheless, a
broad range of practically encountered functions have type signatures
of that form. Here are some examples.

\subsubsection{Example \label{subsec:Example-strong-dinaturality-for-some-type-signatures}\ref{subsec:Example-strong-dinaturality-for-some-type-signatures}\index{solved examples}}

Let $F$ and $G$ be either a functor or a contrafunctor, independently
of each other, and let $L^{X,Y}$ be any profunctor. Use Statement~\ref{subsec:Statement-post-wedge-entails-strong-dinaturality}
to show that strong dinaturality holds for all fully parametric functions
with the following type signatures:

\textbf{(a)} $\forall A.\,F^{A}\rightarrow L^{A,A}\quad.$\textbf{$\quad$(b)}
$\forall A.\,(F^{A}\rightarrow G^{A})\rightarrow L^{A,A}\quad.$

\subparagraph{Solution}

\textbf{(a)} We can represent the type signature as $\forall A.\,K^{A,A}\rightarrow L^{A,A}$
with some profunctor $K^{X,Y}$. Then we will need to consider two
cases where $F$ is either covariant or contravariant. If $F$ is
covariant, we define $K^{X,Y}\triangleq F^{Y}$. If $F$ is contravariant,
we define $K^{X,Y}\triangleq F^{X}$. In every case, $K$ depends
on only one of its type parameters. By Statement~\ref{subsec:Statement-post-wedge}(a),
$K$ will have the post-wedge property. Strong dinaturality will then
follow from Statement~\ref{subsec:Statement-post-wedge-entails-strong-dinaturality}.

\textbf{(b)} The type signature $\forall A.\,(F^{A}\rightarrow G^{A})\rightarrow L^{A,A}$
needs to be expressed as $\forall A.\,K^{A,A}\rightarrow L^{A,A}$
with some profunctor $K$. We define $K^{X,Y}\triangleq M^{Y,X}\rightarrow N^{X,Y}$
with suitably chosen profunctors $M$ and $N$. Reasoning as before,
we find that the profunctors $M$ and $N$ will depend only on one
of their type parameters. So, they will have the post-wedge property.
In addition, Example~\ref{subsec:Example-weak-pullback-property-1}(a)
shows that $M$ has the pushout property. The post-wedge property
of $K$ is then established via Statement~\ref{subsec:Statement-post-wedge}(d).
So, the profunctor $K$ satisfies the conditions of Statement~\ref{subsec:Statement-post-wedge-entails-strong-dinaturality}.
$\square$

Example~\ref{subsec:Example-strong-dinaturality-for-some-type-signatures}(b)
is used in Statement~\ref{subsec:Statement-Church-encoding-recursive-type-covariant}
with the Church encoding of recursive types.

The following example is another illustration of using the laws of
strong dinaturality for deriving specific properties more directly
than via the general laws of relational parametricity.

\subsubsection{Example \label{subsec:Example-strong-dinaturality-show-void}\ref{subsec:Example-strong-dinaturality-show-void}}

\textbf{(a)} For any (covariant) functor $F$, show that the type
of fully parametric functions with type signature $\forall A.\,\left(A\rightarrow A\right)\rightarrow F^{A}$
is equivalent to the type $F^{\bbnum 0}$.

\textbf{(b)} For any contrafunctor $H$, show that the type of fully
parametric functions with type signature $\forall A.\,\left(A\rightarrow A\right)\rightarrow H^{A}$
is equivalent to the type $H^{\bbnum 1}$.

\subparagraph{Solution}

\textbf{(a)} We need to define a one-to-one correspondence between
all values of type $F^{\bbnum 0}$ and all fully parametric functions
of type $\forall A.\,\left(A\rightarrow A\right)\rightarrow F^{A}$.

Note that the Yoneda lemma does not apply to type signatures of that
form. The Yoneda identity:
\[
\forall A.\,\left(T\rightarrow A\right)\rightarrow F^{A}\cong F^{T}
\]
may be used only if the type $T$ is independent of the bound type
parameter $A$.

However, we can use that Yoneda identity with $T\triangleq\bbnum 0$
to obtain the type equivalence $F^{\bbnum 0}\cong\forall A.\,F^{A}$:
\[
\forall A.\,F^{A}\cong\forall A.\,\bbnum 1\rightarrow F^{A}\cong\forall A.\,(\bbnum 0\rightarrow A)\rightarrow F^{A}\cong F^{\bbnum 0}\quad.
\]

By Example~\ref{subsec:Example-strong-dinaturality-for-some-type-signatures}(c),
any fully parametric function $k$ of type $\forall A.\,\left(A\rightarrow A\right)\rightarrow F^{A}$
obeys the strong dinaturality law. That law gives for any types $A$,
$B$ and for any $f^{:A\rightarrow B}$, $p^{:A\rightarrow A}$, and
$q^{:B\rightarrow B}$:
\begin{equation}
\text{if}\quad p\bef f=f\bef q\quad\text{then}\quad k(p)\triangleright f^{\uparrow F}=k(q)\quad.\label{eq:strong-dinaturality-example-void-type}
\end{equation}
Choose $A\triangleq\bbnum 0$, $p\triangleq\text{id}^{:\bbnum 0\rightarrow\bbnum 0}$,
and $f\triangleq\text{absurd}^{:\bbnum 0\rightarrow B}$ (the function
\lstinline!absurd! is defined in Example~\ref{subsec:ch-Example-type-identity-0-to-A}).
Then the precondition of the law~(\ref{eq:strong-dinaturality-example-void-type})
is satisfied for arbitrary $q^{:B\rightarrow B}$. Indeed, both sides
of the precondition ($p\bef f=f\bef q$) are functions of type $\bbnum 0\rightarrow B$,
and there exists only one distinct function of that type (Example~\ref{subsec:ch-Example-type-identity-0-to-A}).
So, the conclusion of the law~(\ref{eq:strong-dinaturality-example-void-type})
holds for any $q$. It means that the value $k(q)$ must be independent
of $q$. In other words, any fully parametric function $k$ of type
$\forall A.\,\left(A\rightarrow A\right)\rightarrow F^{A}$ must ignore
its argument of type $A\rightarrow A$. By Example~\ref{subsec:ch-Example-type-identity-6-1},
functions of type $\left(A\rightarrow A\right)\rightarrow F^{A}$
that ignore their argument are equivalent to values of type $F^{A}$.
So, we obtain the type equivalence:
\[
\forall A.\,\left(A\rightarrow A\right)\rightarrow F^{A}\cong\forall A.\,(\_^{:A\rightarrow A}\rightarrow F^{A})\cong\forall A.\,F^{A}\cong F^{\bbnum 0}.
\]

\textbf{(b)} Any fully parametric function $k$ of type $\forall A.\,\left(A\rightarrow A\right)\rightarrow H^{A}$
obeys the strong dinaturality law:
\begin{equation}
\forall f^{:A\rightarrow B},p^{:A\rightarrow A},q^{:B\rightarrow B}:\quad\text{if}\quad p\bef f=f\bef q\quad\text{then}\quad k(p)=k(q)\triangleright f^{\downarrow H}\quad.\label{eq:strong-dinaturality-example-contrafunctor}
\end{equation}
Choose $B\triangleq\bbnum 1$, $q\triangleq\text{id}^{:\bbnum 1\rightarrow\bbnum 1}$,
and $f\triangleq\_^{:A}\rightarrow1$. Then the precondition of the
law~(\ref{eq:strong-dinaturality-example-contrafunctor}) is satisfied
for arbitrary $p^{:A\rightarrow A}$ (both sides of the precondition
are functions of type $\_^{:A}\rightarrow\bbnum 1$, and there exists
only one distinct function of that type). So, the conclusion of the
law~(\ref{eq:strong-dinaturality-example-contrafunctor}) holds for
any $p$. It means that $k(p)$ must be independent of $p$. In other
words, any fully parametric function $k$ of type $\forall A.\,\left(A\rightarrow A\right)\rightarrow H^{A}$
must ignore its argument of type $A\rightarrow A$. By Example~\ref{subsec:ch-Example-type-identity-6-1},
functions of type $\left(A\rightarrow A\right)\rightarrow H^{A}$
that ignore their argument are equivalent to values of type $H^{A}$.
So, we obtain the type equivalence:
\[
\forall A.\,\left(A\rightarrow A\right)\rightarrow H^{A}\cong\forall A.\,(\_^{:A\rightarrow A}\rightarrow H^{A})\cong\forall A.\,H^{A}\cong H^{\bbnum 1}.
\]
The last step is obtained via the contravariant Yoneda identity (Statement~\ref{subsec:Statement-contravariant-yoneda-identity-for-types}):
\[
\forall A.\,H^{A}\cong\forall A.\,\bbnum 1\rightarrow H^{A}\cong\forall A.\,(A\rightarrow\bbnum 1)\rightarrow H^{A}\cong H^{\bbnum 1}\quad.
\]


\subsection{Strong dinaturality of \texttt{foldFn}}

This section will show that the function \lstinline!foldFn! used
in Statement~\ref{subsec:Statement-foldleft-foldmap-equivalence}
is strongly dinatural. The type signature of \lstinline!foldFn!,
\[
\text{foldFn}_{L}:\forall B.\,L^{B\rightarrow B}\rightarrow B\rightarrow B\quad,
\]
is not immediately covered by Statement~\ref{subsec:Statement-post-wedge-entails-strong-dinaturality}
because it contains the composition of an arbitrary functor $L$ and
a profunctor. To handle \lstinline!foldFn!, we will first need to
prove some additional properties.

\subsubsection{Statement \label{subsec:Statement-functor-post-pre-wedge}\ref{subsec:Statement-functor-post-pre-wedge}}

If $L^{\bullet}$ is a polynomial functor and $K^{\bullet,\bullet}$
is a profunctor with the post-wedge property then the profunctor $P^{X,Y}\triangleq L^{K^{X,Y}}$
also has the post-wedge property.

\subparagraph{Proof}

Assume any two values $x^{:L^{K^{A,A}}}$ and $y^{:L^{K^{B,B}}}$
that are in the $\left(P,f\right)$-wedge relation:
\[
(x,y)\in\text{pull}\,(f^{\uparrow K^{A,\bullet}\uparrow L},f^{\downarrow K^{\bullet,B}\uparrow L})\quad\text{or equivalently}:\quad x\triangleright f^{\uparrow K^{A,\bullet}\uparrow L}=y\triangleright f^{\downarrow K^{\bullet,B}\uparrow L}\quad.
\]
We need to show that $\left(x,y\right)\in\left<f\right>^{\updownarrow P}$.
As shown in Statement~\ref{subsec:Statement-functor-composition-relational-lifting}
below, lifting a relation to $P$ means first lifting to $K$ and
then to $L$, so $\left<f\right>^{\updownarrow P}=\left<f\right>^{\updownarrow K\updownarrow L}$.
Since $K$ has the post-wedge property, the $\left(K,f\right)$-wedge
relation is equivalent to the relation $\left<f\right>^{\updownarrow K}$:
\[
(k_{1},k_{2})\in\left<f\right>^{\updownarrow K}\quad\text{is equivalent to}:\quad(k_{1},k_{2})\in\text{pull}\,(f^{\uparrow K^{A,\bullet}},f^{\downarrow K^{B,\bullet}})\quad.
\]
By Statement~\ref{subsec:Statement-pullback-lifted-to-functor} proved
below, lifting a pullback relation to $L$ gives again a pullback
relation:
\[
\left(x,y\right)\in\left<f\right>^{\updownarrow P}=\big(\text{pull}\,(f^{\uparrow K^{A,\bullet}},f^{\downarrow K^{B,\bullet}})\big)^{\updownarrow L}\quad\text{means}\quad(x,y)\in\text{pull}\,(f^{\uparrow K^{A,\bullet}\uparrow L},f^{\downarrow K^{\bullet,B}\uparrow L})\quad.
\]
So, $\left(x,y\right)$ are in the relation $\left<f\right>^{\updownarrow P}$.

\subsubsection{Statement \label{subsec:Statement-functor-composition-relational-lifting}\ref{subsec:Statement-functor-composition-relational-lifting}}

For any type constructors $G^{\bullet}$ and $H^{\bullet}$, define
$F\triangleq G\circ H$, equivalently denoted as $F^{A}\triangleq G^{H^{A}}$.
Then the lifting of any relation $r^{:A\leftrightarrow B}$ to $F^{\bullet}$
can be expressed as $r^{\updownarrow F}=(r^{\updownarrow H})^{\updownarrow G}$,
which we may write more concisely as $r^{\updownarrow F}=r^{\updownarrow H\updownarrow G}$.

\subparagraph{Proof}

For each case of Definition~\ref{subsec:Definition-relational-lifting}
for the type constructor $G^{\bullet}$, we show that $r^{\updownarrow F}=(r^{\updownarrow H})^{\updownarrow G}$.

\paragraph{Constant type}

With $G^{A}\triangleq Z$ where $Z$ is a fixed type, we have $F^{A}=Z$,
so we write:
\[
r^{\updownarrow F}=\text{id}^{:Z\leftrightarrow Z}\quad\text{and}\quad(r^{\updownarrow H})^{\updownarrow G}=\text{id}^{:Z\leftrightarrow Z}\quad.
\]


\paragraph{Type parameter}

With $G^{A}\triangleq A$, we have $F^{A}=H^{A}$, so we write:
\[
r^{\updownarrow F}=r^{\updownarrow H}\quad\text{and}\quad(r^{\updownarrow H})^{\updownarrow G}=(r^{\updownarrow H})^{\updownarrow\text{Id}}=r^{\updownarrow H}\quad.
\]


\paragraph{Products}

With $G^{A}\triangleq K^{A}\times L^{A}$, we have $F^{A}=K^{H^{A}}\times L^{H^{A}}$.
Assuming that Statement~\ref{subsec:Statement-functor-composition-relational-lifting}
already holds for $K\circ H$ and $L\circ H$, we write:
\begin{align*}
 & r^{\updownarrow F}=r^{\updownarrow(K\circ H)}\boxtimes r^{\updownarrow(L\circ H)}=r^{\updownarrow H\updownarrow K}\boxtimes r^{\updownarrow H\updownarrow L}=(r^{\updownarrow H})^{\updownarrow K}\boxtimes(r^{\updownarrow H})^{\updownarrow L}\quad,\\
 & (r^{\updownarrow H})^{\updownarrow G}=(r^{\updownarrow H})^{\updownarrow(K\times L)}=(r^{\updownarrow H})^{\updownarrow K}\boxtimes(r^{\updownarrow H})^{\updownarrow L}\quad.
\end{align*}
The two relations are now equal.

\paragraph{Co-products and function types}

The proofs are similar to that for products if we replace the operation
$\boxtimes$ by $\boxplus$ or by $\ogreaterthan$ everywhere as appropriate.

\paragraph{Recursive types}

With $G^{A}\triangleq S^{A,G^{A}}$, we have $F^{A}=S^{H^{A},G^{H^{A}}}=S^{H^{A},F^{A}}$.
Denoting $Q^{A,B}\triangleq S^{H^{A},B}$, we can write $F^{A}=Q^{A,F^{A}}$.
One inductive assumption is that Statement~\ref{subsec:Statement-functor-composition-relational-lifting}
already holds separately with respect to each type parameter of $S^{\bullet,\bullet}$
and, in particular, a simultaneous lifting to $Q$ satisfies:
\[
(r,s)^{\updownarrow P}=(r,s)^{\updownarrow S^{H^{\bullet},\bullet}}=(r^{\updownarrow H},s)^{\updownarrow S^{\bullet,\bullet}}\quad.
\]
Then we can finish the derivation:
\begin{align*}
{\color{greenunder}\text{expect to equal }(r^{\updownarrow H})^{\updownarrow G}:}\quad & r^{\updownarrow F}=(r,\overline{r^{\updownarrow F}})^{\updownarrow P}=(r^{\updownarrow H},\overline{r^{\updownarrow F}})^{\updownarrow S}\\
{\color{greenunder}\text{inductive assumption }\overline{r^{\updownarrow F}}=\overline{r^{\updownarrow H\updownarrow G}}:}\quad & =(r^{\updownarrow H},\overline{r^{\updownarrow H\updownarrow G}})^{\updownarrow S}=(r^{\updownarrow H})^{\updownarrow G}\quad.
\end{align*}


\paragraph{Quantified types}

With $G^{A}\triangleq\forall X.\,P^{X,A}$, we have $F^{A}=\forall X.\,P^{X,H^{A}}$.
We may assume that Statement~\ref{subsec:Statement-functor-composition-relational-lifting}
already holds separately for liftings with respect to each type parameter
of $P^{\bullet,\bullet}$. In particular, if we denote $Q^{X,B}\triangleq P^{X,H^{B}}$
then:
\[
(s,r)^{\updownarrow Q}=(s,r)^{\updownarrow P^{\bullet,H^{\bullet}}}=(s,r^{\updownarrow H})^{\updownarrow P}\quad.
\]
For any types $A$, $B$, $X$, $Y$ and for any values $p^{:F^{A}}$
and $q^{:F^{B}}$, we write:
\begin{align*}
 & (p,q)\in r^{\updownarrow F}\quad\text{means}\quad\forall s^{:X\leftrightarrow Y}\,:\,(p^{X},q^{Y})\in(s,r)^{\updownarrow P^{\bullet,H^{\bullet}}}=(s,r^{\updownarrow H})^{\updownarrow P}\quad,\\
 & (p,q)\in(r^{\updownarrow H})^{\updownarrow G}\quad\text{means}\quad\forall s^{:X\leftrightarrow Y}\,:\,(p^{X},q^{Y})\in(s,r^{\updownarrow H})^{\updownarrow P^{\bullet,\bullet}}\quad.
\end{align*}
The two relations are now equal. 

\subsubsection{Statement \label{subsec:Statement-pullback-lifted-to-functor}\ref{subsec:Statement-pullback-lifted-to-functor}}

The lifting of a pullback relation to any polynomial functor $F$
is equivalent to a pullback relation with lifted functions:
\[
\text{for all }f^{:A\rightarrow C},g^{:B\rightarrow C}\quad:\quad\big(\text{pull}\,(f,g)\big)^{\updownarrow F}=\text{pull}\,(f^{\uparrow F},g^{\uparrow F})\quad.
\]


\subparagraph{Proof}

\begin{comment}
Most likely this is wrong for non-polynomial (although covariant)
functors $F$
\end{comment}
We enumerate the type constructions that $F$ is built from. Denote
$r^{:A\leftrightarrow B}\triangleq\text{pull}\,(f,g)$.

\paragraph{Constant types}

If $F^{A}\triangleq Z$ where $Z$ is a fixed type then $r^{\updownarrow F}=\text{id}^{:Z\leftrightarrow Z}$.
We write:
\[
\text{pull}\,(f^{\uparrow F},g^{\uparrow F})=\text{pull}\,(\text{id}^{:Z\leftrightarrow Z},\text{id}^{:Z\leftrightarrow Z})=\text{id}^{:Z\leftrightarrow Z}=r^{\updownarrow F}\quad.
\]


\paragraph{Type parameter}

If $F^{A}\triangleq A$ then $r^{\updownarrow F}=r$, $f^{\uparrow F}=f$,
and $g^{\uparrow F}=g$. We write:
\[
\text{pull}\,(f^{\uparrow F},g^{\uparrow F})=\text{pull}\,(f,g)=r=r^{\updownarrow F}\quad.
\]


\paragraph{Products}

If $F^{A}\triangleq K^{A}\times L^{A}$ then $r^{\updownarrow F}=r^{\updownarrow K}\boxtimes r^{\updownarrow L}$,
$f^{\uparrow F}=f^{\uparrow K}\boxtimes f^{\uparrow L}$, and $g^{\uparrow F}=g^{\uparrow K}\boxtimes g^{\uparrow L}$.
For any values $k_{1}^{:K^{A}}$, $l_{1}^{:L^{A}}$, $k_{2}^{:K^{B}}$,
and $l_{2}^{:L^{B}}$, we write out the relation $r^{\updownarrow F}$:
\[
(k_{1}\times l_{1},k_{2}\times l_{2})\in r^{\updownarrow F}=r^{\updownarrow K}\boxtimes r^{\updownarrow L}\quad\text{means}\quad(k_{1},k_{2})\in r^{\updownarrow K}\text{ and }(l_{1},l_{2})\in r^{\updownarrow L}\quad.
\]
The inductive assumptions for $K$ and $L$ allow us to rewrite the
last conditions as:
\[
(k_{1},k_{2})\in\text{pull}\,(f^{\uparrow K},g^{\uparrow K})\quad\text{ and }\quad(l_{1},l_{2})\in\text{pull}\,(f^{\uparrow L},g^{\uparrow L})\quad.
\]
Turning now to the pullback relation, $\text{pull}\,(f^{\uparrow F},g^{\uparrow F})$,
we write:
\begin{align*}
 & (k_{1}^{:K^{A}}\times l_{1}^{:L^{A}},k_{2}^{:K^{B}}\times l_{2}^{:L^{B}})\in\text{pull}\,(f^{\uparrow F},g^{\uparrow F})=\text{pull}\,(f^{\uparrow K}\boxtimes f^{\uparrow L},g^{\uparrow K}\boxtimes g^{\uparrow L})\\
 & \quad\text{means}\quad(k_{1}\times l_{1})\triangleright(f^{\uparrow K}\boxtimes f^{\uparrow L})=(k_{2}\times l_{2})\triangleright(g^{\uparrow K}\boxtimes g^{\uparrow L})\quad,\\
{\color{greenunder}\text{or equivalently}:}\quad & k_{1}\triangleright f^{\uparrow K}=k_{2}\triangleright g^{\uparrow K}\quad\text{ and }\quad l_{1}\triangleright f^{\uparrow L}=l_{2}\triangleright g^{\uparrow L}\quad.
\end{align*}
The last conditions are now the same as for the relation $r^{\updownarrow F}$.

\paragraph{Co-products}

If $F^{A}\triangleq K^{A}+L^{A}$ then $r^{\updownarrow F}=r^{\updownarrow K}\boxplus r^{\updownarrow L}$,
$f^{\uparrow F}=f^{\uparrow K}\boxplus f^{\uparrow L}$, and $g^{\uparrow F}=g^{\uparrow K}\boxplus g^{\uparrow L}$.
For any values $k_{1}^{:K^{A}}$, $l_{1}^{:L^{A}}$, $k_{2}^{:K^{B}}$,
and $l_{2}^{:L^{B}}$, we write out the relation $r^{\updownarrow F}$:
\begin{align*}
{\color{greenunder}\text{either}:}\quad & (k_{1}+\bbnum 0,k_{2}+\bbnum 0)\in r^{\updownarrow K}\boxplus r^{\updownarrow L}\quad\text{when}\quad(k_{1},k_{2})\in r^{\updownarrow K}\quad,\\
{\color{greenunder}\text{or}:}\quad & (\bbnum 0+l_{1},\bbnum 0+l_{2})\in r^{\updownarrow K}\boxplus r^{\updownarrow L}\quad\text{when}\quad(l_{1},l_{2})\in r^{\updownarrow L}\quad.
\end{align*}
The inductive assumptions for $K$ and $L$ allow us to rewrite the
last conditions as:
\begin{align*}
 & (k_{1},k_{2})\in r^{\updownarrow K}=\text{pull}\,(f^{\uparrow K},g^{\uparrow K})\quad\text{or equivalently}:\quad k_{1}\triangleright f^{\uparrow K}=k_{2}\triangleright g^{\uparrow K}\quad,\\
 & (l_{1},l_{2})\in r^{\updownarrow L}=\text{pull}\,(f^{\uparrow L},g^{\uparrow L})\quad\text{or equivalently}:\quad l_{1}\triangleright f^{\uparrow L}=l_{2}\triangleright g^{\uparrow L}\quad.
\end{align*}
Turning now to the pullback relation, $\text{pull}\,(f^{\uparrow F},g^{\uparrow F})$,
we write:
\begin{align}
 & (p_{1}^{:K^{A}+L^{A}},p_{2}^{:K^{B}+L^{B}})\in\text{pull}\,(f^{\uparrow F},g^{\uparrow F})=\text{pull}\,(f^{\uparrow K}\boxplus f^{\uparrow L},g^{\uparrow K}\boxplus g^{\uparrow L})\nonumber \\
 & \quad\text{means}\quad p_{1}\triangleright(f^{\uparrow K}\boxplus f^{\uparrow L})=p_{2}\triangleright(g^{\uparrow K}\boxplus g^{\uparrow L})\quad.\label{eq:p1-p2-condition-derivation1}
\end{align}
The pair co-product (such as $f^{\uparrow K}\boxplus f^{\uparrow L}$)
preserves the left and right parts of the disjunctive type, so the
condition~(\ref{eq:p1-p2-condition-derivation1}) is satisfied only
when $p_{1}$ and $p_{2}$ are both in the same part of the disjunction:
\begin{align*}
 & p_{1}\triangleright(f^{\uparrow K}\boxplus f^{\uparrow L})=p_{2}\triangleright(g^{\uparrow K}\boxplus g^{\uparrow L})\quad\text{means}\quad:\\
{\color{greenunder}\text{either}:}\quad & p_{1}=k_{1}+\bbnum 0\quad,\quad p_{2}=k_{2}+\bbnum 0\quad,\quad k_{1}\triangleright f^{\uparrow K}=k_{2}\triangleright g^{\uparrow K}\quad;\\
{\color{greenunder}\text{or}:}\quad & p_{1}=\bbnum 0+l_{1}\quad,\quad p_{2}=\bbnum 0+l_{2}\quad,\quad l_{1}\triangleright f^{\uparrow L}=l_{2}\triangleright g^{\uparrow L}\quad.
\end{align*}
The last conditions are now the same as for the relation $r^{\updownarrow F}$.

\paragraph{Function types}

We may not use this construction since $F$ is assumed to be a polynomial
functor.

\paragraph{Recursive types}

If $F^{A}\triangleq S^{A,F^{A}}$ where $S^{\bullet,\bullet}$ is
a polynomial bifunctor then:
\[
r^{\updownarrow F}=\big(r,\overline{r^{\updownarrow F}}\big)^{\updownarrow S}\quad,\quad\quad f^{\uparrow F}=f^{\uparrow S^{\bullet,F^{A}}}\bef\big(\overline{f^{\uparrow F}}\big)^{\uparrow S^{C,\bullet}}\quad,\quad\quad g^{\uparrow F}=g^{\uparrow S^{\bullet,F^{B}}}\bef\big(\overline{g^{\uparrow F}}\big)^{\uparrow S^{C,\bullet}}\quad.
\]
The inductive assumptions are that Statement~\ref{subsec:Statement-pullback-lifted-to-functor}
already holds for the recursively used lifting $\overline{r^{\updownarrow F}}$
and for simultaneous liftings of \emph{two} pullback relations to
$S^{\bullet,\bullet}$. Write those assumptions as:
\begin{align*}
{\color{greenunder}\text{for }\overline{r^{\updownarrow F}}:}\quad & \overline{r^{\updownarrow F}}\overset{!}{=}\text{pull}\,\big(\overline{f^{\uparrow F}},\overline{g^{\uparrow F}}\big)\quad,\\
{\color{greenunder}\text{for }S:}\quad & \big(\text{pull}\,(f^{:A\rightarrow C},g^{:B\rightarrow C}),\,\,\text{pull}\,(h^{:X\rightarrow Z},k^{:Y\rightarrow Z})\big)^{\updownarrow S}\overset{!}{=}\text{pull}\,(f^{\uparrow S^{\bullet,X}}\bef h^{\uparrow S^{C,\bullet}},\,\,g^{\uparrow S^{\bullet,Y}}\bef k^{\uparrow S^{C,\bullet}})\quad,
\end{align*}
where the last pullback relation involves functions of types $S^{A,X}\rightarrow S^{C,Z}$
and $S^{B,Y}\rightarrow S^{C,Z}$. Then:
\begin{align*}
{\color{greenunder}\text{expect to equal }\text{pull}\,(f^{\uparrow F},g^{\uparrow F}):}\quad & r^{\updownarrow F}=\big(r,\overline{r^{\updownarrow F}}\big)^{\updownarrow S}=\big(\text{pull}\,(f,g),\,\,\text{pull}\,\big(\overline{f^{\uparrow F}},\overline{g^{\uparrow F}}\big)\big)^{\updownarrow S}\\
{\color{greenunder}\text{assumption for }S:}\quad & =\text{pull}\,\big(f^{\uparrow S^{\bullet,F^{A}}}\bef\big(\overline{f^{\uparrow F}}\big)^{\uparrow S^{C,\bullet}}\,\,,g^{\uparrow S^{\bullet,F^{B}}}\bef\big(\overline{g^{\uparrow F}}\big)^{\uparrow S^{C,\bullet}}\big)\\
{\color{greenunder}\text{definitions of }(...)^{\uparrow F}:}\quad & =\text{pull}\,(f^{\uparrow F},g^{\uparrow F})\quad.
\end{align*}
$\square$

The strong dinaturality property of \lstinline!foldFn! can be proved
by using Statement~\ref{subsec:Statement-functor-post-pre-wedge}:

\subsubsection{Example \label{subsec:Example-strong-dinaturality-proof-of-foldFn-law}\ref{subsec:Example-strong-dinaturality-proof-of-foldFn-law}\index{solved examples}}

Show that any fully parametric function $f:\forall A.\,L^{A\rightarrow A}\rightarrow A\rightarrow A$
is strongly dinatural when $L$ is any polynomial functor.

\subparagraph{Solution}

Define the profunctor $K^{X,Y}\triangleq X\rightarrow Y$ so that
the type signature of $f$ is written as:
\[
\forall A.\,L^{A\rightarrow A}\rightarrow A\rightarrow A=\forall A.\,L^{K^{A,A}}\rightarrow K^{A,A}\quad.
\]
By Statement~\ref{subsec:Statement-post-wedge}(a) and (d), the profunctor
$K$ has the post-wedge property. Statement~\ref{subsec:Statement-functor-post-pre-wedge}
then shows that the profunctor $L^{K^{X,Y}}$ also has the post-wedge
property. By Statement~\ref{subsec:Statement-post-wedge-entails-strong-dinaturality},
$f$ satisfies the strong dinaturality law. 

\subsection{Non-disjunctive type constructors}

As an advanced example of using the relational parametricity theorem,
we will now show that certain type constructors $P$ permit us to
simplify the type $\forall A.\,P^{A}\rightarrow Q^{A}+R^{A}$ into
the disjunctive type $(\forall A.\,P^{A}\rightarrow Q^{A})+(\forall A.\,P^{A}\rightarrow R^{A})$.

Generally, the type expression $P\rightarrow Q+R$ cannot be simplified.
In particular, it is not equivalent to $(P\rightarrow Q)+(P\rightarrow R)$.
However, the situation becomes different once we assume full parametricity
and a universally quantified type. As a first example, consider a
fully parametric function $k$ of type:
\[
k:\forall A.\,A\rightarrow\bbnum 1+A\quad.
\]
The function $k$ must construct a value of type $\bbnum 1+A$ given
a value of type $A$. So, the code of $k$ must decide to return either
the first or the second part of the disjunctive type $\bbnum 1+A$.
That decision \emph{cannot} depend on the value of the input argument
(of type $A$) because nothing is known about the type $A$. In particular,
the code of $k$ cannot perform any pattern-matching on the input
argument. It follows that the decision about the disjunctive return
value must be hard-coded: for some $k$ the value $k(x)$ must be
$1+\bbnum 0$ for all $x$, and for other functions $k$ the result
must be $\bbnum 0+x$ for all $x$. So, the fully parametric type
$\forall A.\,A\rightarrow\bbnum 1+A$ splits into a disjunction of
$\forall A.\,A\rightarrow\bbnum 1$ and $\forall A.\,A\rightarrow A$.
Each of those types has only one possible implementation:
\begin{lstlisting}
def k1[A]: A => Option[A] = None
def k2[A]: A => Option[A] = Some(_)
\end{lstlisting}
Indeed, we may use Yoneda identities to show that the types $\forall A.\,A\rightarrow\bbnum 1$
and $\forall A.\,A\rightarrow A$ are both equivalent to $\bbnum 1$:
\begin{align*}
 & \forall A.\,A\rightarrow\bbnum 1\cong\forall A.\,(\bbnum 1\rightarrow A)\rightarrow\bbnum 1\cong\bbnum 1\quad,\\
 & \forall A.\,A\rightarrow A\cong\forall A.\,(\bbnum 1\rightarrow A)\rightarrow A\cong\bbnum 1\quad.
\end{align*}
In this way, we have been able to simplify the initial type:
\[
\forall A.\,A\rightarrow\bbnum 1+A\cong(\forall A.\,A\rightarrow\bbnum 1)+(\forall A.\,A\rightarrow A)\cong\bbnum 1+\bbnum 1\cong\bbnum 2\quad.
\]

So far, our arguments are not fully rigorous: we relied upon the intuition
that the input type ($A$) carries no \textsf{``}disjunctive information\textsf{''}
and cannot be used to make decisions about which disjunctive value
to return. To make this argument rigorous, we will prove that certain
type constructors $P$ have the property of being \textsf{``}non-disjunctive\textsf{''}.
When $P$ is non-disjunctive, input values of type $P^{A}$ cannot
be used in pattern-matching. Fully parametric functions of type $\forall A.\,P^{A}\rightarrow Q^{A}+R^{A}$
must hard-code the decision about the returned disjunctive value.

\subsubsection{Definition \label{subsec:Definition-non-disjunctive-type-constructors}\ref{subsec:Definition-non-disjunctive-type-constructors}}

A type constructor $P$ is called \index{non-disjunctive type constructor}\textbf{non-disjunctive}\index{type constructor!non-disjunctive}
if for any type constructors $Q$ and $R$ the following type equivalence
holds (assuming that all functions are fully parametric):
\begin{equation}
\forall A.\,P^{A}\rightarrow Q^{A}+R^{A}\cong(\forall A.\,P^{A}\rightarrow Q^{A})+(\forall A.\,P^{A}\rightarrow R^{A})\quad.\label{eq:non-disjunctive-type-equivalence}
\end{equation}

The main technique for proving the properties of non-disjunctive type
constructors is to use the relational parametricity theorem with relations
that always hold and relations that never hold. For convenience, we
give the following definitions:

\subsubsection{Definition \label{subsec:Definition-always-true-relation}\ref{subsec:Definition-always-true-relation}}

For any two types $A$ and $B$, we denote by $\Omega^{:A\leftrightarrow B}$
the \textbf{always-true} relation\index{always-true relation}:
\[
(x^{:A},y^{:B})\in\Omega^{:A\leftrightarrow B}\quad\text{for any }x^{:A}\text{ and }y^{:B}\quad.
\]
The \textbf{always-false} relation,\index{always-false relation}
denoted by $\emptyset^{:A\leftrightarrow B}$, never holds for any
values $(x^{:A},y^{:B})$.

\subsubsection{Definition \label{subsec:Definition-lifting-to-true-and-to-false-property}\ref{subsec:Definition-lifting-to-true-and-to-false-property}}

A type constructor $P$ is \index{type constructor!lifting-to-true property}\textsf{``}\textbf{lifting-to-true}\textsf{''}
if for any two types $A$ and $B$ such that $P^{A}\neq\bbnum 0$
and $P^{B}\neq0$ there exists a relation $r^{:A\leftrightarrow B}$
such that $r^{\updownarrow P}=\Omega^{:P^{A}\leftrightarrow P^{B}}$.

A type constructor $P$ is \index{type constructor!lifting-to-false property}\textsf{``}\textbf{lifting-to-false}\textsf{''}
if for any two types $A$ and $B$ there exists a relation $r^{:A\leftrightarrow B}$
such that $r^{\updownarrow P}=\emptyset^{:P^{A}\leftrightarrow P^{B}}$.
$\square$

The key observation is that lifting a relation to a disjunctive type
constructor does \emph{not} give an always-true relation. Instead,
it gives a relation that holds only for values belonging to the same
part of the disjunctive type. So, if a lifted relation is always-true,
the type constructor contains no disjunctions. The following statement
makes this argument precise.

\subsubsection{Statement \label{subsec:Statement-undisjunctive-type-constructors-relational}\ref{subsec:Statement-undisjunctive-type-constructors-relational}}

If a type constructor $P$ is lifting-to-true then $P$ is non-disjunctive.

\subparagraph{Proof}

By assumption, there exists a relation $r^{:A\leftrightarrow B}$
such that $r^{\updownarrow P}=\Omega^{:P^{A}\leftrightarrow P^{B}}$.
Then the relational parametricity law for the type $\forall A.\,P^{A}\rightarrow Q^{A}+R^{A}$
says that any fully parametric functions $p:P^{A}\rightarrow Q^{A}+R^{A}$
and $q:P^{B}\rightarrow Q^{B}+R^{B}$ satisfy:
\[
\forall x^{:P^{A}},y^{:P^{B}}:\quad\text{if}\quad(x,y)\in r^{\updownarrow P}=\Omega\quad\text{then}\quad(p(x),q(y))\in r^{\updownarrow Q^{\bullet}+R^{\bullet}}=r^{\updownarrow Q}\boxplus r^{\updownarrow R}\quad.
\]
Since $(x,y)\in\Omega$ is always true, we find:
\begin{equation}
\forall x^{:P^{A}},y^{:P^{B}}:\quad(p(x),q(y))\in r^{\updownarrow Q}\boxplus r^{\updownarrow R}\quad.\label{eq:relational-law-example-all-x-y}
\end{equation}
By definition of the relational co-product $r^{\updownarrow Q}\boxplus r^{\updownarrow R}$,
the values $\left(p(x),q(y)\right)$ must be in the same part of the
disjunctive type $Q^{\bullet}+R^{\bullet}$. By assumption, $P^{A}\neq\bbnum 0$
and and $P^{B}\neq0$, and there exists a pair $(x_{0}^{:P^{A}},y_{0}^{:P^{B}})$
such that $\left(p(x_{0}),q(y_{0})\right)\in r^{\updownarrow Q}\boxplus r^{\updownarrow R}$.
It follows that the values $p(x_{0})$ and $q(y_{0})$ must belong
to the same part of the type $Q^{\bullet}+R^{\bullet}$. Suppose they
belong to the part $Q^{\bullet}+\bbnum 0$, so that $p(x_{0})=s_{0}^{:Q^{A}}+\bbnum 0^{:R^{A}}$
and $q(y_{0})=t_{0}^{:Q^{B}}+\bbnum 0^{:R^{B}}$ with some values
$s_{0}$ and $t_{0}$. The relational law~(\ref{eq:relational-law-example-all-x-y})
holds for all $x$ and $y$, so: 
\begin{align*}
 & \forall x^{:P^{A}}:\quad(p(x),q(y_{0}))=(p(x),t_{0}+\bbnum 0)\in r^{\updownarrow Q}\boxplus r^{\updownarrow R}\quad,\\
 & \forall y^{:P^{B}}:\quad(p(x_{0}),q(y))=(s_{0}+\bbnum 0,q(y))\in r^{\updownarrow Q}\boxplus r^{\updownarrow R}\quad.
\end{align*}
Then the values $p(x)$ and $q(y)$ must be in the first part of the
disjunctive type for \emph{all} $x$ and $y$. So, the functions $p$
and $q$ are equivalent to some functions of types $P^{A}\rightarrow Q^{A}+\bbnum 0^{:R^{A}}$
and $P^{B}\rightarrow Q^{B}+\bbnum 0^{:R^{A}}$.

The other possibility is that both values $p(x_{0})$ and $q(y_{0})$
belong to the second part ($\bbnum 0+R^{\bullet}$) of the type $Q^{\bullet}+R^{\bullet}$.
A similar argument shows that $p(x)$ and $q(y)$ are of type $\bbnum 0+R^{\bullet}$
for all $x$ and $y$. So, the functions $p$ and $q$ are equivalent
to functions of types $P^{A}\rightarrow\bbnum 0^{:Q^{A}}+R^{A}$ and
$P^{B}\rightarrow\bbnum 0^{:Q^{B}}+R^{B}$.

We have found that the relational parametricity law forces each fully
parametric function $p:P^{A}\rightarrow Q^{A}+R^{A}$ to belong either
to the subtype $P^{A}\rightarrow Q^{A}+\bbnum 0^{:R^{A}}$ or to the
subtype $P^{A}\rightarrow\bbnum 0^{:Q^{A}}+R^{A}$. In this way, we
have proved the type equivalence~(\ref{eq:non-disjunctive-type-equivalence}).
$\square$

We will now use structural analysis to discover which type constructors
are non-disjunctive.

\subsubsection{Statement \label{subsec:Statement-undisjunctive-type-constructors-structural}\ref{subsec:Statement-undisjunctive-type-constructors-structural}}

A type constructor $P$ is non-disjunctive when:

\textbf{(a)} $P^{A}\triangleq K^{A}\times L^{A}$ where both $K$
and $L$ are non-disjunctive.

A type constructor $P$ is lifting-to-true when:

\textbf{(b)} $P^{A}\triangleq\bbnum 1$.

\textbf{(c)} $P^{A}\triangleq A$.

\textbf{(d)} $P^{A}\triangleq K^{A}\rightarrow L^{A}$ where $K$
is arbitrary and $L$ is lifting-to-true.

\textbf{(e)} $P^{A}\triangleq K^{A}\rightarrow L^{A}$ where $K$
is lifting-to-false and $L$ is arbitrary.

A type constructor $P$ is lifting-to-false when:

\textbf{(f)} $P^{A}\triangleq A$. (Compare with item \textbf{(b)}
to see that a type constructor can be lifting-to-true and lifting-to-false
at the same time.)

\textbf{(g)} $P^{A}\triangleq K^{A}\times L^{A}$ where $K$ is lifting-to-false
and $L$ is arbitrary.

\textbf{(h)} $P^{A}\triangleq K^{A}\rightarrow L^{A}$ where $K$
is \emph{not} lifting-to-false and $L$ is lifting-to-false.

\subparagraph{Proof}

\textbf{(a)} Use currying on the type $P^{A}$ and the assumption
that both $K$ and $L$ are non-disjunctive:
\begin{align*}
 & P^{A}\rightarrow Q^{A}+R^{A}=\gunderline{K^{A}\times L^{A}}\rightarrow Q^{A}+R^{A}\cong K^{A}\rightarrow(\gunderline{L^{A}\rightarrow Q^{A}+R^{A}})\\
 & \cong\gunderline{K^{A}\rightarrow\,}(L^{A}\rightarrow Q^{A})+(L^{A}\rightarrow R^{A})\\
 & \cong(\gunderline{K^{A}\rightarrow L^{A}}\rightarrow Q^{A})+(\gunderline{K^{A}\rightarrow L^{A}}\rightarrow R^{A})\\
 & \cong(\gunderline{K^{A}\times L^{A}}\rightarrow Q^{A})+(\gunderline{K^{A}\times L^{A}}\rightarrow R^{A})=(P^{A}\rightarrow Q^{A})+(P^{A}\rightarrow R^{A})\quad.
\end{align*}

\textbf{(b)} The unit type ($\bbnum 1$) has only one distinct value.
So, any relation $r^{:A\leftrightarrow B}$ lifted to $P^{A}\triangleq\bbnum 1$
will give an identity relation of type $\bbnum 1\leftrightarrow\bbnum 1$.
That relation is the same as the always-true relation $\Omega^{:\bbnum 1\leftrightarrow\bbnum 1}$.

\textbf{(c)} Lift an always-true relation $\Omega^{:A\leftrightarrow B}$
to $P^{A}\triangleq A$ and obtain again $\Omega^{:A\leftrightarrow B}$.

\textbf{(d)} Choose $r^{:A\leftrightarrow B}$ such that $r^{\updownarrow L}=\Omega^{:L^{A}\leftrightarrow L^{B}}$
and lift that $r$ to $P$. The result is:
\[
r^{\updownarrow P}=r^{\updownarrow K}\ogreaterthan r^{\updownarrow L}=r^{\updownarrow K}\ogreaterthan\Omega\quad.
\]
The condition $(x,y)\in r^{\updownarrow P}$ becomes:
\[
(x^{:P^{A}},y^{:P^{B}})\in r^{\updownarrow P}\quad\text{means}\quad\text{if}\quad(a,b)\in r^{\updownarrow K}\quad\text{then}\quad(x(a),y(b))\in\Omega\quad.
\]
The condition $(x(a),y(b))\in\Omega$ is always true by definition
of $\Omega$. The logical formula \textsf{``}if ... then true\textsf{''} is always
true. So, we have proved that $(x,y)\in r^{\updownarrow P}$ is always
true.

\textbf{(e)} Choose $r^{:A\leftrightarrow B}$ such that $r^{\updownarrow K}=\emptyset^{:K^{A}\leftrightarrow K^{B}}$
and lift that $r$ to $P$. The result is:
\[
r^{\updownarrow P}=r^{\updownarrow K}\ogreaterthan r^{\updownarrow L}=\emptyset\ogreaterthan r^{\updownarrow L}\quad.
\]
The condition $(x,y)\in r^{\updownarrow P}$ becomes:
\[
(x^{:P^{A}},y^{:P^{B}})\in r^{\updownarrow P}\quad\text{means}\quad\text{if}\quad(a,b)\in\emptyset\quad\text{then}\quad(x(a),y(b))\in r^{\updownarrow L}\quad.
\]
The condition $(a,b)\in\emptyset$ is always false. The logical formula
\textsf{``}if false then ...\textsf{''} is always true. So, we have proved that $(x,y)\in r^{\updownarrow P}$
is always true.

\textbf{(f)} Lift an always-false relation $\emptyset^{:A\leftrightarrow B}$
to $P^{A}\triangleq A$ and obtain again $\emptyset^{:A\leftrightarrow B}$.

\textbf{(g)} Choose $r^{:A\leftrightarrow B}$ such that $r^{\updownarrow K}=\emptyset^{:K^{A}\leftrightarrow K^{B}}$
and lift that $r$ to $P$. The result is:
\[
r^{\updownarrow P}=r^{\updownarrow K}\boxtimes r^{\updownarrow L}=\emptyset\boxtimes r^{\updownarrow L}\quad.
\]
The condition $(x,y)\in r^{\updownarrow P}$ becomes:
\[
(x^{:P^{A}},y^{:P^{B}})=(k_{1}^{:K^{A}}\times l_{1}^{:L^{A}},k_{2}^{:K^{B}}\times l_{2}^{:L^{B}})\in r^{\updownarrow P}\quad\text{means}\quad(k_{1},k_{2})\in\emptyset\quad\text{and}\quad(l_{1},l_{2})\in r^{\updownarrow L}\quad.
\]
The condition $(k_{1},k_{2})\in\emptyset$ is always false, so $(x,y)\in r^{\updownarrow P}$
is also always false. 

\textbf{(h)} By assumption, $K$ is not lifting-to-false, which means
that there exist types $A$, $B$ such that \emph{any} relation $r^{:A\leftrightarrow B}$
is lifted to a relation $r^{\updownarrow P}$ that is \emph{not} always-false.
Choose $r^{:A\leftrightarrow B}$ such that $r^{\updownarrow L}=\emptyset^{:L^{A}\leftrightarrow L^{B}}$
and lift that $r$ to $P$. The result is:
\[
r^{\updownarrow P}=r^{\updownarrow K}\ogreaterthan r^{\updownarrow L}=r^{\updownarrow K}\ogreaterthan\emptyset\quad.
\]
We will now show that $r^{\updownarrow P}=\emptyset$ by finding a
contradiction when we assume that $(x,y)\in r^{\updownarrow P}$ for
some $x^{:P^{A}}$ and $y^{:P^{B}}$. If we assume that such $x$
and $y$ exist then we write out the condition $(x,y)\in r^{\updownarrow P}$:
\[
(x,y)\in r^{\updownarrow P}\quad\text{means}\quad\text{if}\quad(a,b)\in r^{\updownarrow K}\quad\text{then}\quad(x(a),y(b))\in\emptyset\quad.
\]
Since $r^{\updownarrow K}$ is \emph{not} always-false, there exist
some values $a^{:K^{A}}$ and $b^{:K^{B}}$ such that $(a,b)\in r^{\updownarrow K}$.
Using these values, we obtain $(x(a),y(b))\in\emptyset$, which is
impossible. This contradiction shows that there do not exist any $x$
and $y$ such that $(x,y)\in r^{\updownarrow P}$. So, $r^{\updownarrow P}=\emptyset$
as required. $\square$

The following examples will help us build up intuition about non-disjunctive
types.

\subsubsection{Example \label{subsec:Example-undisjunctive-type-constructors}\ref{subsec:Example-undisjunctive-type-constructors}\index{solved examples}}

\textbf{(a)} Show that $P^{A}\triangleq A\rightarrow A$ is lifting-to-true
but \emph{not} lifting-to-false.

\textbf{(b)} Show that $P^{A}\triangleq\bbnum 1+A\rightarrow A$ is
both lifting-to-true and lifting-to-false.

\textbf{(c)} Show that $P^{A}\triangleq\left(A\rightarrow A\right)\rightarrow A\rightarrow A$
is lifting-to-true.

\subparagraph{Solution}

\textbf{(a)} We know from Statement~\ref{subsec:Statement-undisjunctive-type-constructors-structural}(c)
that $L^{A}\triangleq A$ is lifting-to-true. It follows from Statement~\ref{subsec:Statement-undisjunctive-type-constructors-structural}(d)
with $K^{A}\triangleq L^{A}\triangleq A$ that $P$ is also lifting-to-true.

To show that $P$ is not lifting-to-false, it is sufficient to demonstrate
that for any relation $r^{:A\leftrightarrow B}$ there exist $x^{:P^{A}}$
and $y^{:P^{B}}$ such that $(x,y)\in r^{\updownarrow P}$. Write
out the condition $(x,y)\in r^{\updownarrow P}$:
\[
(x,y)\in r^{\updownarrow P}\quad\text{means}\quad\text{if}\quad(a^{:A},b^{:B})\in r\quad\text{then}\quad(x(a),y(b))\in r\quad.
\]
If we choose $x\triangleq\text{id}^{:A\rightarrow A}$ and $y\triangleq\text{id}^{:B\rightarrow B}$,
we will have $(x(a),y(b))=(a,b)$. Then the condition becomes: if
$(a,b)\in r$ then $(a,b)\in r$. This is true for any $r$. So, $(x,y)\in r^{\updownarrow P}$
holds.

\textbf{(b)} We know from Statement~\ref{subsec:Statement-undisjunctive-type-constructors-structural}(c,f)
that $L^{A}\triangleq A$ is both lifting-to-true and lifting-to-false.
By Statement~\ref{subsec:Statement-undisjunctive-type-constructors-structural}(d)
with $K^{A}\triangleq\bbnum 1+A$ and $L^{A}\triangleq A$, we find
that $K^{A}\rightarrow L^{A}$ is lifting-to-true. To show that $K^{A}\rightarrow L^{A}$
is lifting-to-false, we can use Statement~\ref{subsec:Statement-undisjunctive-type-constructors-structural}(h)
if we prove that $K^{A}$ is \emph{not} lifting-to-false. Lifting
any relation $r^{:A\leftrightarrow B}$ to $K$, we find:
\[
r^{\updownarrow K}=\text{id}^{:\bbnum 1\leftrightarrow\bbnum 1}\boxplus r\quad.
\]
The pair of values $x^{:K^{A}}\triangleq1+\bbnum 0^{:A}$ and $y^{:K^{B}}\triangleq1+\bbnum 0^{:B}$
are always in the left part of the relation $\text{id}\boxplus r$.
In this way, we have shown that $r^{\updownarrow K}$ is not always-false. 

By Statement~\ref{subsec:Statement-undisjunctive-type-constructors-relational},
the type constructor $P$ is non-disjunctive. Note that the type expression
$\bbnum 1+A\rightarrow A$ contains a disjunctive type but can be
rewritten equivalently as $P^{A}\cong(\bbnum 1\rightarrow A)\times(A\rightarrow A)$,
which no longer contains a disjunctive type.

\textbf{(c)} We have shown in item (a) that $L^{A}\triangleq A\rightarrow A$
is lifting-to-true. It remains to use Statement~\ref{subsec:Statement-undisjunctive-type-constructors-structural}(d)
with $K^{A}\triangleq A\rightarrow A$ to conclude that $P^{A}=K^{A}\rightarrow L^{A}$
is lifting-to-true.

\subsection{Further directions}

Several aspects of relational parametricity are beyond the scope of
this book.
\begin{itemize}
\item Binary relations support some additional operations, such as union,
intersection, and composition (an operation similar to \lstinline!JOIN!
in relational databases). The composition of two relations $r:A\leftrightarrow B$
and $s:B\leftrightarrow C$ is a new relation of type $A\leftrightarrow C$,
denoted by $r\circ s$ and defined as $(x^{:A},z^{:C})\in r\circ s$
when $\exists y^{:B}:(x,y)\in r$ and $(y,z)\in s$. However, this
book proves the practically useful parametricity properties without
using the composition or other relational operations. The simultaneous
lifting construction (Definition~\ref{subsec:Definition-simultaneous-relational-lifting})
cannot be defined via the relational composition, in particular because
$(r\circ s)^{\updownarrow F}\neq r^{\updownarrow F}\circ s^{\updownarrow F}$
for some type constructors $F$ and because the relation $(r\circ s)\ogreaterthan(u\circ v)$
follows from but is \emph{not} equivalent to $(r\ogreaterthan u)\circ(s\ogreaterthan v)$. 
\item Parametricity can be extended to types and code constructions beyond
those covered in this book. The scope of material in this Appendix
is intentionally limited to theory that has already proven its practical
importance for programmers. For instance, we did not cover parametricity
for functions with typeclass constraints. The corresponding theory
is complicated\footnote{The theory of parametricity applied to functions with typeclass constraints
is developed by J.~Voigtl\"ander\index{Janis@Janis Voigtl\"ander}
in the paper \texttt{\href{https://www.janis-voigtlaender.eu/papers/FreeTheoremsInvolvingTypeConstructorClasses.pdf}{https://www.janis-voigtlaender.eu/papers/FreeTheoremsInvolvingTypeConstructorClasses.pdf}}} and yet delivers few practically useful results. For example, that
theory does not seem to solve Problem~\ref{par:Problem-identity-natural-monad-morphism},
which considers functions with the following type signature:
\begin{lstlisting}
def epsilon[M[_]: Monad, A]: M[A] => M[A]
\end{lstlisting}
If \lstinline!epsilon! is a monad morphism that works in the same
way for all monads \lstinline!M! and all types \lstinline!A!, can
we prove that \lstinline!epsilon! is an identity function? The analogous
question involving functors can be answered via parametricity. It
follows from the Yoneda lemma for functors (Statement~\ref{subsec:Statement-covariant-yoneda-identity-for-type-constructors})
that any fully parametric function with the type signature $\forall F^{\bullet}.\,\forall A.\,F^{A}\rightarrow F^{A}$,
where $F$ is a functor, or in Scala:
\begin{lstlisting}
def epsilon[F[_]: Functor, A]: F[A] => F[A]
\end{lstlisting}
must be an identity function.
\end{itemize}

\chapter{Solutions of some exercises}

\addsec{Chapter \ref{chap:1-Values,-types,-expressions,}}

\subsubsection*{Exercise \ref{subsec:ch1-aggr-Exercise-1}}

~
\begin{lstlisting}
def at(n: Double, maxN: Int) = (0 to maxN)
  .map { k => 1.0*(1 - k % 2 * 2) / (2 * k + 1) / math.pow(n, 2 * k + 1) }
  .sum
def p(n: Int) = 16 * at(5, n) - 4 * at(239, n)

scala> p(12)
res0: Double = 3.141592653589794
\end{lstlisting}


\subsubsection*{Exercise \ref{subsec:ch1-aggr-Example-6}}

~

\begin{lstlisting}
def euler_series(n: Int): Double = (1 to n).map(k => 1.0 / k / k).sum

scala> euler_series(100000)
res0: Double = 1.6449240668982423

scala> val pi = math.Pi
pi: Double = 3.141592653589793

scala> pi * pi / 6
res1: Double = 1.6449340668482264 
\end{lstlisting}


\subsubsection*{Exercise \ref{subsec:ch1-aggr-Exercise-2}}

~
\begin{lstlisting}
def isPrime(n: Int) = (2 to n - 1).takeWhile(k => k * k <= n).forall(k => n % k != 0)
def ep(n: Int): Double = (2 to n)
  .filter(isPrime)
  .map  { k => 1.0 / (1.0 - 1.0 / k / k / k /k) }
  .product
val pi = 3.1415926535897932

scala> ep(100); pi*pi*pi*pi/90
res0_0: Double = 1.0823231553280295
res0_1: Double = 1.082323233711138 
\end{lstlisting}


\addsec{Chapter \ref{chap:2-Mathematical-induction}}

\subsubsection*{Exercise \ref{tuples-Exercise-10}}

~
\begin{lstlisting}
numsLists.map(_.sortBy(- _).take(3))
\end{lstlisting}


\subsubsection*{Exercise \ref{tuples-Exercise-11}}

~
\begin{lstlisting}
a.flatMap(x => b.map(y => (x, y)))
\end{lstlisting}


\subsubsection*{Exercise \ref{tuples-Exercise-12}}

~
\begin{lstlisting}
def payments[Person, Amount](data: Seq[Map[Person, Amount]]): Map[Person, Seq[Amount]] =
  data.flatMap(_.toSeq).groupBy(_._1).mapValues(_.map(_._2))
\end{lstlisting}


\subsubsection*{Exercise \ref{subsec:Exercise-2.2-foldleft-5}}

~
\begin{lstlisting}
def batching[A](xs: Seq[A], size: Int): Seq[Seq[A]] = {  
  type Acc = (Seq[Seq[A]], Seq[A], Int)
  val init: Acc = ((Seq(), Seq(), 0))
  val (result, rem, _) = xs.foldLeft(init){ case ((seq, rem, len), x) =>
    val newLen = len + 1
    if (newLen > size) (seq ++ Seq(rem), Seq(x), 1)
    else (seq, rem ++ Seq(x), newLen)
  }
  result ++ Seq(rem)
}
\end{lstlisting}


\subsubsection*{Exercise \ref{subsec:Exercise-2.2-foldleft-5-1}}

~
\begin{lstlisting}
def weightBatching[A](xs: Seq[A], maxW: Double)(w: A => Double): Seq[Seq[A]] = {  
  type Acc = (Seq[Seq[A]], Seq[A], Double)
  val init: Acc = ((Seq(), Seq(), 0.0))
  val (result, rem, _) = xs.foldLeft(init) { case ((seq, rem, weight), x) =>
    val wx = w(x)
    if (wx > maxW) (seq ++ Seq(rem, Seq(x)), Seq(), 0.0)
    else {
      val newWeight = weight + wx
      if (newWeight > maxW) (seq ++ Seq(rem), Seq(x), wx)
      else (seq, rem ++ Seq(x), newWeight)
    }
  }
  result ++ Seq(rem)
}
\end{lstlisting}


\subsubsection*{Exercise \ref{subsec:Exercise-2.2-foldleft-6}}

~
\begin{lstlisting}
def groupBy[A, K](xs: Seq[A])(by: A => K): Map[K, Seq[A]] = {  
  val init: Map[K, Seq[A]] = Map()
  xs.foldLeft(init) { (res, x) =>
    val key = by(x)
    val seq = res.getOrElse(key, Seq()) ++ Seq(x)
    res.updated(key, seq)
  }
}
\end{lstlisting}


\subsubsection*{Exercise \ref{subsec:ch2Exercise-seq-3}}

~
\begin{lstlisting}
def digitsOf(n: Int): Seq[Int] = Stream.iterate(n)(_ / 10).takeWhile(_ != 0).map(_ % 10).toList
def cubeDigits(n: Int): Int = digitsOf(n).map(x => x*x*x).sum
def cubes(n: Int): Stream[Int] = Stream.iterate(n)(cubeDigits)

def stopRepeats[T](str: Stream[T]): Stream[T] = {
  val halfSpeed = str.flatMap(x => Seq(x, x))
  val result = halfSpeed.zip(str).drop(1).takeWhile{ case (h, s) => h != s }.map(_._2)
  if (result.isEmpty) str.take(1) else str
}
def cubesReach1(n: Int): Boolean = stopRepeats(cubes(n)).contains(1)
\end{lstlisting}


\subsubsection*{Exercise \ref{subsec:ch2Exercise-seq-4}}

~
\begin{lstlisting}
def prod3(a: Set[Int], b: Set[Int], c: Set[Int]): Set[Set[Int]] =
  a.flatMap(x => b.flatMap(y => c.map(z => Set(x, y, z))))
\end{lstlisting}


\subsubsection*{Exercise \ref{subsec:ch2Exercise-seq-5}}

~
\begin{lstlisting}
def prodSet(sets: Set[Set[Int]]): Set[Set[Int]] =
    sets.foldLeft(Set[Set[Int]](Set())) {
      // Combine each of results so far with each element in current set
      case (accumSet: Set[Set[Int]], currSet: Set[Int]) =>
        for {
          s <- accumSet
          i <- currSet
        } yield s + i
      }
\end{lstlisting}


\subsubsection*{Exercise \ref{subsec:ch2Exercise-seq-4-1}}

~
\begin{lstlisting}
@tailrec def pairs(goal: Int, xs: Array[Int])(
  res: Set[(Int, Int)] = Set(), left: Int = 0, right: Int = xs.length): Set[(Int, Int)] =
    if (left == right) res else {
      val sum = xs(left) + xs(right - 1)
      val (newLeft, newRight, newRes) = if (sum == goal) 
                  (left + 1, right, res + ((xs(left), xs(right - 1)))) 
                else if (sum < goal)
                  (left + 1, right, res)
                else (left, right - 1, res)
      pairs(goal, xs)(newRes, newLeft, newRight)
}
\end{lstlisting}


\subsubsection*{Exercise \ref{subsec:ch2Exercise-seq-6}}

~
\begin{lstlisting}
def revSentence(sentence: String): String = sentence.split(" ").reverse.mkString(" ")
\end{lstlisting}


\subsubsection*{Exercise \ref{subsec:ch2revdigits-Exercise-seq-7}}

~
\begin{lstlisting}
def digitsOf(n: Int): Seq[Int] = Stream.iterate(n)(_ / 10).takeWhile(_ != 0).map(_ % 10).toList
def revDigits(n: Int): Int = digitsOf(n).foldLeft(0){case (acc, d) => acc * 10 + d }
def isPalindrome(n: Int): Boolean = revDigits(n) == n
\end{lstlisting}


\subsubsection*{Exercise \ref{subsec:ch2Exercise-seq-8}}

~
\begin{lstlisting}
def findPalindrome(n: Int): Int = 
  Stream.iterate(n) { x => x + revDigits(x) } .filter(isPalindrome).head
\end{lstlisting}


\subsubsection*{Exercise \ref{subsec:ch2Exercise-seq-9-1}}

~
\begin{lstlisting}
def unfold2[A,B](init: A)(next: A => Option[(A,B)]): Stream[B] = next(init) match {
   case None           => Stream()
   case Some((a, b))   => Stream.cons(b, unfold2(a)(next))
}
\end{lstlisting}


\addsec{Chapter \ref{chap:Disjunctive-types}}

\subsubsection*{Exercise \ref{subsec:Disjunctive-Exercise-non-empty-list-1}}

~
\begin{lstlisting}
def toList[A](nel: NEL[A]): List[A] = nel match {
  case Last(x)         => List(x)
  case More(x, tail)   => x :: toList(tail)
} // Not tail-recursive.
def toList[A](nel: NEL[A]): List[A] = foldLeft(nel)(Nil:List[A]) {
  (prev, x) =>  x :: prev
}.reverse // Tail-recursive, but performs two traversals.
\end{lstlisting}


\subsubsection*{Exercise \ref{subsec:Disjunctive-Exercise-non-empty-list-2}}

~
\begin{lstlisting}
def concat[A](xs: NEL[A], ys: NEL[A]): NEL[A] =
  foldLeft(reverse(xs))(ys)((p, x) => More(x, p))
\end{lstlisting}


\subsubsection*{Exercise \ref{subsec:Exercise-disjunctive-EvenList}}

~
\begin{lstlisting}
sealed trait EvenList[A]
final case class Lempty[A]() extends EvenList[A]
final case class Lpair[A](x: A, y: A, tail: EvenList[A]) extends EvenList

def fmap[A, B](f: A => B): EvenList[A] => EvenList[B] = {
  case Lempty()            => Lempty[B]()
  case Lpair(x, y, tail)   => Lpair[B](f(x), f(y), fmap(f)(tail))
}
\end{lstlisting}


\addsec{Chapter \ref{chap:Higher-order-functions}}

\subsubsection*{Exercise \ref{subsec:Exercise-hof-simple-8}}

~
\begin{lstlisting}
@tailrec def convergeN[X](p: X => Boolean)(x:X)(m:Int)(f: X => X): Option[X] =  {
              if (m <= 0) None
              else if (p(x)) Some(x) else converge(p)(f(x))(m-1)(f)             }
// Defining it as def convergeN[X]: (X => Boolean) => X => Int => (X => X) => Option[X] = ???
// will break tail recursion!
\end{lstlisting}


\subsubsection*{Exercise \ref{subsec:Exercise-hof-simple-7-1}}

~
\begin{lstlisting}
def recover[E, A]: Option[Either[E, A]] => (E => A) => Option[A] = {
  case None             => _ => None
  case Some(Right(a))   => _ => Some(a)
  case Some(Left(e))    => f => Some(f(e))
}
\end{lstlisting}


\subsubsection*{Exercise \ref{subsec:Exercise-hof-composition-1}}

\textbf{(a)} Choose $f^{:A\rightarrow A}\triangleq(\_\rightarrow z)$
and compute $(f\bef h)(x)=h(z)\overset{!}{=}h(x)$, for any $x$.
So, $h(x)$ equals a fixed value $h(z)$. It follows that $h(x)$
does not depend on $x$, i.e., $h$ is a constant function.

\textbf{(b)} Choose $f^{:A\rightarrow B}\triangleq\_\rightarrow b$
and compute $(f\bef h)(x)=h(b)\overset{!}{=}(g\bef f)(x)=f(g(x))=b$.
It follows that $h(b)=b$ for any $b^{:B}$, so $h$ is an identity
function. Substitute that into the law and get $f=g\bef f$ for any
function $f$. Substitute $f\triangleq\text{id}$ into that and derive
$\text{id}=g\bef\text{id}=g$. So, $g$ is the identity function.

\addsec{Chapter \ref{chap:5-Curry-Howard}}

\subsubsection*{Exercise \ref{subsec:Exercise-type-notation-2}}

The type expression is: $A\times\text{Int}+A\times\text{Char}+A\times\text{Float}$.

\addsec{Chapter \ref{chap:Reasoning-about-code}}

\subsubsection*{Exercise \ref{subsec:Exercise-reasoning-1-4-1}}

It is assumed that $\phi:\forall A.\,F^{A}\rightarrow G^{A}$ satisfies
its naturality law: for any $f^{:A\rightarrow B}$,
\[
f^{\uparrow F}\bef\phi=\phi\bef f^{\uparrow G}\quad.
\]

\textbf{(a)} To verify the naturality law of $\phi^{\uparrow K}$:
for any $f^{:A\rightarrow B}$,
\begin{align*}
 & f^{\uparrow F\uparrow K}\bef\phi^{\uparrow K}\overset{?}{=}\phi^{\uparrow K}\bef f^{\uparrow G\uparrow K}\quad,\\
{\color{greenunder}\text{composition under }^{\uparrow K}:}\quad & (f^{\uparrow F}\bef\phi)^{\uparrow K}\overset{?}{=}(\phi\bef f^{\uparrow G})^{\uparrow K}\quad.
\end{align*}
The last equation holds due to the naturality law of $\phi$.

\textbf{(b)} The naturality law of the pair product ($\phi\boxtimes\psi$)
says that for any $p^{:A\rightarrow B}$,
\[
p^{\uparrow(F\times K)}\bef(\phi\boxtimes\psi)\overset{?}{=}(\phi\boxtimes\psi)\bef p^{\uparrow(G\times L)}\quad.
\]
Begin with the left-hand side of this equation:
\begin{align*}
{\color{greenunder}\text{left-hand side}:}\quad & p^{\uparrow(F\times K)}\bef(\phi\boxtimes\psi)\\
{\color{greenunder}\text{definition of }^{\uparrow(F\times K)}:}\quad & =\big(f^{:F^{A}}\times k^{:K^{A}}\rightarrow(f\triangleright p^{\uparrow F})\times(k\triangleright p^{\uparrow K})\big)\bef(\phi\boxtimes\psi)\\
{\color{greenunder}\text{definition of }\phi\boxtimes\psi:}\quad & =f\times k\rightarrow\phi(f\triangleright p^{\uparrow F})\times\psi(k\triangleright p^{\uparrow K})=f\times k\rightarrow(f\triangleright p^{\uparrow F}\bef\phi)\times(k\triangleright p^{\uparrow K}\bef\psi)\quad.
\end{align*}
To rewrite the right-hand side, introduce the function argument into
$\phi\boxtimes\psi$:
\begin{align*}
{\color{greenunder}\text{right-hand side}:}\quad & (\phi\boxtimes\psi)\bef p^{\uparrow(G\times L)}=\big(f^{:F^{A}}\times k^{:K^{A}}\rightarrow(f\triangleright\phi)\times(k\triangleright\psi)\big)\bef p^{\uparrow(G\times L)}\\
{\color{greenunder}\text{definition of }^{\uparrow(G\times L)}:}\quad & =f\times k\rightarrow(f\triangleright\gunderline{\phi\triangleright p^{\uparrow G}})\times(k\triangleright\gunderline{\psi\triangleright p^{\uparrow L}})=f\times k\rightarrow(f\triangleright\phi\bef p^{\uparrow G})\times(k\triangleright\psi\bef p^{\uparrow L})\quad.
\end{align*}
The remaining differences between the two sides disappear due to the
naturality laws of $\phi$ and $\psi$:
\[
p^{\uparrow F}\bef\phi=\phi\bef p^{\uparrow G}\quad,\quad\quad p^{\uparrow K}\bef\psi=\psi\bef p^{\uparrow L}\quad.
\]

The naturality law of the pair co-product ($\phi\boxplus\psi$) says
that for any $p^{:A\rightarrow B}$,
\[
p^{\uparrow(F+K)}\bef(\phi\boxplus\psi)\overset{?}{=}(\phi\boxplus\psi)\bef p^{\uparrow(G+L)}\quad.
\]
Begin with the left-hand side of this equation:
\begin{align*}
{\color{greenunder}\text{left-hand side}:}\quad & p^{\uparrow(F+K)}\bef(\phi\boxplus\psi)\\
{\color{greenunder}\text{definitions of }^{\uparrow(F+K)}\text{ and of }\phi\boxplus\psi:}\quad & =\,\begin{array}{|c||cc|}
 & F^{B} & K^{B}\\
\hline F^{A} & p^{\uparrow F} & \bbnum 0\\
K^{A} & \bbnum 0 & p^{\uparrow K}
\end{array}\,\bef\,\begin{array}{|c||cc|}
 & G^{B} & L^{B}\\
\hline F^{B} & \phi & \bbnum 0\\
K^{B} & \bbnum 0 & \psi
\end{array}\\
{\color{greenunder}\text{matrix composition}:}\quad & =\,\,\begin{array}{|c||cc|}
 & G^{B} & L^{B}\\
\hline F^{A} & p^{\uparrow F}\bef\phi & \bbnum 0\\
K^{A} & \bbnum 0 & p^{\uparrow K}\bef\psi
\end{array}\quad.
\end{align*}
The right-hand side is rewritten in a similar way:
\begin{align*}
{\color{greenunder}\text{right-hand side}:}\quad & (\phi\boxplus\psi)\bef p^{\uparrow(G+L)}\\
 & =\,\begin{array}{|c||cc|}
 & G^{A} & L^{A}\\
\hline F^{A} & \phi & \bbnum 0\\
K^{A} & \bbnum 0 & \psi
\end{array}\,\bef\,\begin{array}{|c||cc|}
 & G^{B} & L^{B}\\
\hline G^{A} & p^{\uparrow G} & \bbnum 0\\
L^{A} & \bbnum 0 & p^{\uparrow L}
\end{array}\,=\,\begin{array}{|c||cc|}
 & G^{B} & L^{B}\\
\hline F^{A} & \phi\bef p^{\uparrow G} & \bbnum 0\\
K^{A} & \bbnum 0 & \psi\bef p^{\uparrow L}
\end{array}\quad.
\end{align*}
The remaining differences between the two sides disappear due to the
naturality laws of $\phi$ and $\psi$. 

\addsec{Chapter \ref{chap:Typeclasses-and-functions}}

\subsubsection*{Exercise \ref{subsec:tc-Exercise-3}}

\textbf{(a)}
\begin{lstlisting}
def monoidFunc[A: Monoid, R] = Monoid[R => A](
  { (x, y) => r => x(r) |+| y(r) }, _ => implicitly[Monoid[A]].empty
)
\end{lstlisting}

In the code notation:
\[
f^{:R\rightarrow A}\oplus g^{:R\rightarrow A}\triangleq a\rightarrow f(a)\oplus_{A}g(a)\quad,\quad\quad e\triangleq(\_\rightarrow e_{A})\quad.
\]
Proof of monoid laws:
\begin{align*}
 & a\triangleright\left(\left(f\oplus g\right)\oplus h\right)=\left(a\triangleright(f\oplus g)\right)\oplus_{A}h(a)=f(a)\oplus_{A}g(a)\oplus_{A}h(a)\quad.\\
 & a\triangleright\left(f\oplus\left(g\oplus h\right)\right)=f(a)\oplus_{A}\left(a\triangleright(g\oplus h)\right)=f(a)\oplus_{A}g(a)\oplus_{A}h(a)\quad.\\
 & a\triangleright\left(e\oplus f\right)=e(a)\oplus_{A}f(a)=e_{A}\oplus_{A}f(a)=f(a)=a\triangleright f\quad.\\
 & a\triangleright(f\oplus e)=f(a)\oplus_{A}e(a)=f(a)\oplus_{A}e_{A}=f(a)=a\triangleright f\quad.
\end{align*}


\subsubsection*{Exercise \ref{subsec:tc-Exercise-9-1} }

\textbf{(a)} A counterexample is the functor $F^{A}\triangleq R\rightarrow A$,
where $R$ is a fixed type.

\textbf{(b)} The function $C^{A}\times C^{B}\rightarrow C^{A+B}$
cannot be implemented for $C^{A}\triangleq\left(A\rightarrow P\right)+\left(A\rightarrow Q\right)$.
This more complicated contrafunctor $C$ is necessary because the
simpler contrafunctor $C^{A}\triangleq A\rightarrow P$ does not provide
a counterexample.

\subsubsection*{Exercise \ref{subsec:tc-Exercise-9-1-1-1}}

Define the method $\text{ex}_{F}$ as:
\[
\text{ex}_{F}\triangleq x^{:F^{A}}\rightarrow x\triangleright(a^{:A}\rightarrow a\times1)^{\uparrow F}\triangleright q^{A,\bbnum 1}\triangleright\pi_{1}\quad\text{ or equivalently: }\quad\text{ex}_{F}\triangleq(a^{:A}\rightarrow a\times1)^{\uparrow F}\bef q^{A,\bbnum 1}\bef\pi_{1}\quad.
\]
To show that the naturality law ($f^{\uparrow F}\bef\text{ex}_{F}=\text{ex}_{F}\bef f$)
holds, use the identity $(f\boxtimes g)\bef\pi_{1}=\pi_{1}\bef f$:
\begin{align*}
 & f^{\uparrow F}\bef\text{ex}_{F}=\gunderline{f^{\uparrow F}\bef(a^{:A}\rightarrow a\times1)^{\uparrow F}}\bef q^{A,\bbnum 1}\bef\pi_{1}=(a^{:A}\rightarrow a\times1)^{\uparrow F}\bef\gunderline{(f\boxtimes\text{id})^{\uparrow F}\bef q^{A,\bbnum 1}}\bef\pi_{1}\\
 & =(a^{:A}\rightarrow a\times1)^{\uparrow F}\bef q^{A,\bbnum 1}\bef\gunderline{(f\boxtimes\text{id}^{\uparrow F})\bef\pi_{1}}=\gunderline{(a^{:A}\rightarrow a\times1)^{\uparrow F}\bef q^{A,\bbnum 1}\bef\pi_{1}}\bef f=\text{ex}_{F}\bef f\quad.
\end{align*}

Given a method $\text{ex}_{F}$, define $q$ as:
\[
q^{A,B}\triangleq f^{:F^{A\times B}}\rightarrow(f\triangleright\pi_{1}^{\uparrow F}\triangleright\text{ex}_{F}^{A})\times(f\triangleright\pi_{2}^{\uparrow F})\quad\text{ or equivalently: }\quad q^{A,B}\triangleq\Delta\bef(\pi_{1}^{\uparrow F}\boxtimes\pi_{2}^{\uparrow F})\bef(\text{ex}_{F}^{A}\boxtimes\text{id})\quad.
\]
Show that the required laws hold for $q$. Identity law: 
\[
f\triangleright q^{\bbnum 1,B}=(f\triangleright\pi_{1}^{\uparrow F}\triangleright\text{ex}_{F}^{\bbnum 1})\times(f\triangleright\pi_{2}^{\uparrow F})=1\times(f\triangleright\pi_{2}^{\uparrow F})
\]
because $\text{ex}_{F}^{\bbnum 1}$ produces a value of type $\bbnum 1$,
which can only be $1$.

To prove the naturality law:
\begin{align*}
(f^{:A\rightarrow C}\boxtimes g^{:B\rightarrow D})^{\uparrow F}\bef\Delta\bef(\pi_{1}^{\uparrow F}\boxtimes\pi_{2}^{\uparrow F})\bef(\text{ex}_{F}^{A}\boxtimes\text{id}) & =\Delta\bef(\pi_{1}^{\uparrow F}\boxtimes\pi_{2}^{\uparrow F})\bef(\text{ex}_{F}^{A}\boxtimes\text{id})\bef f\boxtimes(g^{\uparrow F})\\
\Delta\bef((f\boxtimes g)^{\uparrow F}\boxtimes(f\boxtimes g)^{\uparrow F})\bef(\pi_{1}^{\uparrow F}\boxtimes\pi_{2}^{\uparrow F})\bef(\text{ex}_{F}^{A}\boxtimes\text{id}) & =\Delta\bef(\pi_{1}^{\uparrow F}\bef\text{ex}_{F}\bef f)\boxtimes(\pi_{2}^{\uparrow F}\bef g^{\uparrow F})\\
\Delta\bef(\pi_{1}^{\uparrow F}\bef\,\gunderline{f^{\uparrow F}\bef\text{ex}_{F}})\boxtimes(\pi_{2}^{\uparrow F}\bef g^{\uparrow F}) & =\Delta\bef(\pi_{1}^{\uparrow F}\bef\,\gunderline{\text{ex}_{F}\bef f})\boxtimes(\pi_{2}^{\uparrow F}\bef g^{\uparrow F})
\end{align*}
To prove the associativity law, write the left-hand side as:
\begin{align*}
 & f^{:F^{A\times B\times C}}\triangleright q^{A,B\times C}\bef(\text{id}^{A}\boxtimes q^{B,C})=(f\triangleright\pi_{1}^{\uparrow F}\triangleright\text{ex}_{F}^{A})\times(f\triangleright(a\times b\times c\rightarrow b\times c)^{\uparrow F}\triangleright q^{B,C})\\
 & =(f\triangleright\pi_{1}^{\uparrow F}\triangleright\text{ex}_{F}^{A})\times(f\triangleright(a\times b\times c\rightarrow b\times c)^{\uparrow F}\triangleright\pi_{1}^{\uparrow F}\triangleright\text{ex}_{F}^{B})\times(f\triangleright(a\times b\times c\rightarrow b\times c)^{\uparrow F}\triangleright\pi_{2}^{\uparrow F})\\
 & =\left(f\triangleright\text{ex}_{F}\triangleright(a\times b\times c\rightarrow a)\right)\times\left(f\triangleright\text{ex}_{F}\triangleright(a\times b\times c\rightarrow b)\right)\times(f\triangleright(a\times b\times c\rightarrow c)^{\uparrow F})\quad.
\end{align*}
The right-hand side is then simplified to the same expression:
\begin{align*}
 & f^{:F^{A\times B\times C}}\triangleright q^{A\times B,C}=(f\triangleright\text{ex}_{F}\triangleright(a\times b\times c\rightarrow a\times b))\times(f\triangleright(a\times b\times c\rightarrow c)^{\uparrow F})\\
 & =\left(f\triangleright\text{ex}_{F}\triangleright(a\times b\times c\rightarrow a)\right)\times\left(f\triangleright\text{ex}_{F}\triangleright(a\times b\times c\rightarrow b)\right)\times(f\triangleright(a\times b\times c\rightarrow c)^{\uparrow F})\quad.
\end{align*}


\addsec{Chapter \ref{chap:Filterable-functors}}

\subsubsection*{Exercise \ref{subsec:filt-exercise-derive-liftOpt-equivalence-1}}

Starting from \lstinline!liftOpt!\textsf{'}s law, derive the naturality law:
\[
\text{liftOpt}\left(g\right)\bef\text{liftOpt}\,(f\bef\text{pu}_{\text{Opt}})=\text{liftOpt}(f\bef\text{pu}_{\text{Opt}}\diamond_{_{\text{Opt}}}g)=\text{liftOpt}(f\bef g)\quad.
\]
Now use the naturality-identity law and get $\text{liftOpt}\,(f\bef\text{pu}_{\text{Opt}})=f^{\downarrow C}$.
The result is the naturality law $\text{liftOpt}\left(f\bef g\right)=\text{liftOpt}\left(g\right)\bef f^{\downarrow C}$.

\subsubsection*{Exercise \ref{subsec:Exercise-filterable-laws-4}}

A counterexample is the functor $F^{A}\triangleq\bbnum 1+A\times\left(Z\rightarrow A\right)$.
For this functor, one can implement \lstinline!deflate!\textsf{'}s type signature,
but the code cannot obey the identity law because it must always return
$1+\bbnum 0$.

\addsec{Chapter \ref{chap:Semimonads-and-monads}}

\subsubsection*{Exercise \ref{subsec:Exercise-1-monads-3-1}}

\index{monads!List monad with empty sub-lists@\texttt{List} monad with empty sub-lists}The
given non-standard implementation of \lstinline!flatten! will return
the same results as the standard \lstinline!flatten! method of the
\lstinline!List! type constructor, except if one of the nested sub-lists
is empty. In that case, the \lstinline!flatten! function returns
an empty list (unlike \lstinline!List!\textsf{'}s standard \lstinline!flatten!
method). The \lstinline!pure! method remains unchanged. The code
is:
\begin{lstlisting}[mathescape=true]
def pure[A](x: A): List[A] = List(x) // $\color{dkgreen}\textrm{pu}_L$
def flatten[A](p: List[List[A]]): List[A] = if (p.exists(_.isEmpty)) Nil else p.flatten  // $\color{dkgreen}\textrm{ftn}_L$
\end{lstlisting}

To verify the monad laws, we use the known fact that the standard
\lstinline!List! monad obeys the laws. So, we only need to check
the laws in the cases when the new \lstinline!flatten! function is
applied to a value of type \lstinline!List[List[A]]! having an empty
nested sub-list. That case cannot arise in the identity laws since
neither $\text{pu}_{L}(x^{:A})$ nor $\text{pu}_{L}^{\uparrow L}(x^{:\text{List}^{A}})$
ever returns a value with a nested empty sub-list:
\begin{lstlisting}
pure(List(a, b, c)) == List(List(a, b, c))
List(a, b, c).map(pure) == List(List(a), List(b), List(c))
\end{lstlisting}
Applying \lstinline!flatten! to these values gives the initial list
\lstinline!List(a, b, c)!. So, both identity laws hold.

It remains to check the associativity law, which is an equality between
functions $\text{ftn}_{L}\bef\text{ftn}_{L}$ and $\text{ftn}_{L}^{\uparrow L}\bef\text{ftn}_{L}$
of type \lstinline!List[List[List[A]]] => List[A]!. A value $p$
of type \lstinline!List[List[List[A]]]! could contain a nested empty
list at the first and/or the second nesting depth, for instance:
\begin{lstlisting}
val p1: List[List[List[Int]]] = List(List[List[Int]](), List(List(123), List(456)))
val p2: List[List[List[Int]]] = List(List(List[Int](), List(123)), List(List(456)))
\end{lstlisting}
Whenever $p$ contains an empty sub-list at the first nesting depth,
we will have $p\triangleright\text{ftn}_{L}=\text{Nil}$ because $\text{ftn}_{L}$
explicitly checks for the existence of an empty sub-list. So, $p\triangleright\text{ftn}_{L}\bef\text{ftn}_{L}=\text{Nil}$.
On the other hand, $p\triangleright\text{ftn}_{L}^{\uparrow L}$ will
be again a list containing an empty sub-list, for example:
\begin{lstlisting}
scala> p1.map(flatten)
res0: List[List[Int]] = List(List(), List(123, 456))
\end{lstlisting}
We will then have $p\triangleright\text{ftn}_{L}^{\uparrow L}\triangleright\text{ftn}_{L}=\text{Nil}$,
and the law holds. If $p$ contains an empty sub-list at the \emph{second}
nesting depth, $p\triangleright\text{ftn}_{L}$ will contain an empty
sub-list at the first nesting depth, for example:
\begin{lstlisting}
scala> flatten(p2)
res1: List[List[Int]] = List(List(), List(123), List(456)) 
\end{lstlisting}
So $p\triangleright\text{ftn}_{L}\bef\text{ftn}_{L}=\text{Nil}$.
On the other hand, $p\triangleright\text{ftn}_{L}^{\uparrow L}$ will
also be a list with an empty sub-list at the first nesting depth,
for example:
\begin{lstlisting}
scala> p2.map(flatten)
res2: List[List[Int]] = List(List(), List(456))
\end{lstlisting}
Applying $\text{ftn}_{L}$ to the last result, we will get an empty
list. Thus, $p\triangleright\text{ftn}_{L}\triangleright\text{ftn}_{L}=\text{Nil}$,
and the law again holds. So, we have shown that the associativity
law holds for the non-standard \lstinline!List! monad.

\subsubsection*{Exercise \ref{subsec:Exercise-flatten-concat-distributive-law}}

The values $p$ and $q$ must have type $\text{List}^{\text{List}^{A}}$.
There are two possibilities: $p$ is an empty list ($p=1+\bbnum 0$),
and $p=\bbnum 0+h\times t$. If $p$ is empty, so is $p\triangleright\text{ftn}$
and the law holds. In the other case, we have (due to the code of
$\pplus$) that:
\[
\left(\bbnum 0+h\times t\right)\pplus q=\bbnum 0+h\times\left(t\pplus q\right)\quad,
\]
and so:
\begin{align*}
 & \left(p\pplus q\right)\triangleright\text{ftn}=\left(\bbnum 0+h\times\left(t\pplus q\right)\right)\triangleright\text{ftn}=h\pplus\gunderline{\left(t\pplus q\right)\triangleright\overline{\text{ftn}}}\\
{\color{greenunder}\text{inductive assumption}:}\quad & =\gunderline{h\pplus(t\triangleright\overline{\text{ftn}})}\pplus(q\triangleright\overline{\text{ftn}})\\
{\color{greenunder}\text{code of }\text{ftn}:}\quad & =\left(\bbnum 0+h\times t\right)\triangleright\overline{\text{ftn}}\pplus(q\triangleright\overline{\text{ftn}})=\left(p\triangleright\text{ftn}\right)\pplus\left(q\triangleright\text{ftn}\right)\quad.
\end{align*}


\subsubsection*{Exercise \ref{subsec:Exercise-monad-of-monoid-is-monoid}}

Define the empty element ($e_{M}$) of the monoid $M^{W}$ as:
\[
e_{M}\triangleq\text{pu}_{M}(e_{W})\quad.
\]
The binary operation $\oplus_{M}$ of the monoid $M^{W}$ may be implemented
through $\oplus_{W}$ as:
\[
p\oplus_{M}q\triangleq p\triangleright\text{flm}_{M}\big(u^{:W}\rightarrow q\triangleright(v^{:W}\rightarrow u\oplus_{W}v)^{\uparrow M}\big)\quad.
\]
To check the left identity law of $M^{W}$:
\begin{align*}
{\color{greenunder}\text{expect to equal }p:}\quad & e_{M}\oplus_{M}p=e_{W}\triangleright\gunderline{\text{pu}_{M}\triangleright\text{flm}_{M}}\big(u^{:W}\rightarrow p\triangleright(v^{:W}\rightarrow u\oplus_{W}v)^{\uparrow M}\big)\\
 & =e_{W}\triangleright\big(u^{:W}\rightarrow p\triangleright(v^{:W}\rightarrow u\oplus_{W}v)^{\uparrow M}\big)=p\triangleright(v\rightarrow\gunderline{e_{W}\oplus_{W}v})^{\uparrow M}\\
{\color{greenunder}\text{left identity law of }W:}\quad & =p\triangleright(v\rightarrow v)^{\uparrow M}=p\triangleright\text{id}^{\uparrow M}=p\quad.
\end{align*}
To check the right identity law of $M^{W}$:
\begin{align*}
{\color{greenunder}\text{expect to equal }p:}\quad & p\oplus_{M}e_{M}=p\triangleright\text{flm}_{M}\big(u^{:W}\rightarrow e_{W}\triangleright\gunderline{\text{pu}_{M}\triangleright(v^{:W}\rightarrow u\oplus_{W}v)^{\uparrow M}}\big)\\
{\color{greenunder}\text{naturality of }\text{pu}_{M}:}\quad & =p\triangleright\text{flm}_{M}\big(u^{:W}\rightarrow\gunderline{e_{W}\triangleright(v^{:W}}\rightarrow u\oplus_{W}v)\triangleright\text{pu}_{M}\big)\\
{\color{greenunder}\text{apply function}:}\quad & =p\triangleright\text{flm}_{M}\big(u^{:W}\rightarrow(\gunderline{u\oplus_{W}e_{W}})\triangleright\text{pu}_{M}\big)\\
{\color{greenunder}\text{right identity law of }W:}\quad & =p\triangleright\text{flm}_{M}\big(u^{:W}\rightarrow u\triangleright\text{pu}_{M}\big)=p\triangleright\text{id}=p\quad.
\end{align*}
To check the associativity law of $M^{W}$, we use the associativity
law of $\text{flm}_{M}$:
\begin{align*}
{\color{greenunder}\text{left-hand side}:}\quad & (p\oplus_{M}q)\oplus_{M}r\\
 & =p\triangleright\gunderline{\text{flm}_{M}}\big(u\rightarrow q\triangleright(v\rightarrow u\oplus_{W}v)^{\uparrow M}\big)\,\gunderline{\triangleright\text{flm}_{M}}\big(t\rightarrow r\triangleright(w\rightarrow t\oplus_{W}w)^{\uparrow M}\big)\\
{\color{greenunder}\text{associativity of }\text{flm}_{M}:}\quad & =p\triangleright\text{flm}_{M}\big(\gunderline (u\rightarrow q\triangleright(v\rightarrow u\oplus_{W}v)^{\uparrow M}\gunderline{)\bef}\,\text{flm}_{M}(t\rightarrow r\triangleright(w\rightarrow t\oplus_{W}w)^{\uparrow M})\big)\\
{\color{greenunder}\text{compute composition}:}\quad & =p\triangleright\text{flm}_{M}\big(u\rightarrow q\triangleright(v\rightarrow u\oplus_{W}v\gunderline{)^{\uparrow M}\triangleright\text{flm}_{M}}(t\rightarrow r\triangleright(w\rightarrow t\oplus_{W}w)^{\uparrow M})\big)\\
{\color{greenunder}\text{naturality of }\text{flm}_{M}:}\quad & =p\triangleright\text{flm}_{M}\big(u\rightarrow q\triangleright\text{flm}_{M}(v\rightarrow\gunderline{u\oplus_{W}v})\,\gunderline{\bef(t}\rightarrow r\triangleright(w\rightarrow\gunderline t\oplus_{W}w)^{\uparrow M}))\big)\\
{\color{greenunder}\text{compute composition}:}\quad & =p\triangleright\text{flm}_{M}\big(u\rightarrow q\triangleright\text{flm}_{M}(v\rightarrow r\triangleright(w\rightarrow(u\oplus_{W}v)\oplus_{W}w)^{\uparrow M})\big)\quad.
\end{align*}
Now write the right-hand side of the law:
\begin{align*}
{\color{greenunder}\text{right-hand side}:}\quad & p\oplus_{M}(q\oplus_{M}r)=p\triangleright\text{flm}_{M}\big(u\rightarrow(q\oplus r)\triangleright(t\rightarrow u\oplus_{W}t)^{\uparrow M}\big)\\
{\color{greenunder}\text{substitute }q\oplus r:}\quad & =p\triangleright\text{flm}_{M}\big(u\rightarrow q\triangleright\text{flm}_{M}(v\rightarrow r\triangleright(w\rightarrow v\oplus_{W}w)^{\uparrow M})\,\gunderline{\triangleright\,(}t\rightarrow u\oplus_{W}t\gunderline{)^{\uparrow M}}\big)\\
{\color{greenunder}\text{naturality of }\text{flm}_{M}:}\quad & =p\triangleright\text{flm}_{M}\big(u\rightarrow q\triangleright\text{flm}_{M}(v\rightarrow r\triangleright(w\rightarrow u\oplus_{W}w\gunderline{)^{\uparrow M}\bef}\,(t\rightarrow u\oplus_{W}t\gunderline{)^{\uparrow M}})\big)\\
{\color{greenunder}\text{composition under }^{\uparrow M}:}\quad & =p\triangleright\text{flm}_{M}\big(u\rightarrow q\triangleright\text{flm}_{M}(v\rightarrow r\triangleright(w\rightarrow u\oplus_{W}(v\oplus_{W}w))^{\uparrow M})\big)\quad.
\end{align*}
 The difference between the two sides now vanishes due to the assumed
associativity law of $W$:
\[
(u\oplus_{W}v)\oplus_{W}w\overset{!}{=}u\oplus_{W}(v\oplus_{W}w)\quad.
\]


\subsubsection*{Exercise \ref{subsec:Exercise-1-monads-9-1}}

\textbf{(b)} The code is converted into monad methods like this:
\begin{align*}
 & r_{1}=p\triangleright\text{flm}_{M}(x\rightarrow q\triangleright(y\rightarrow f(x,y))^{\uparrow M})\quad,\\
 & r_{2}=q\triangleright\text{flm}_{M}(y\rightarrow p\triangleright(x\rightarrow f(x,y))^{\uparrow M})\quad.
\end{align*}
For a commutative monad $M$, we have $r_{1}=r_{2}$. The monoid operation
$\oplus_{M}$ is defined by:
\[
p^{:M^{\bbnum 1}}\oplus_{M}q^{:M^{\bbnum 1}}\triangleq p\triangleright\text{flm}_{M}(1\rightarrow q)\quad.
\]
Commutativity of $\oplus_{M}$ means that:
\[
p\triangleright\text{flm}_{M}(1\rightarrow q)=p\oplus_{M}q\overset{?}{=}q\oplus_{M}p=q\triangleright\text{flm}_{M}(1\rightarrow p)\quad.
\]
Use $f(x,y)\triangleq1$ in the definitions of $r_{1}$ and $r_{2}$;
then the above equation is equivalent to $r_{1}=r_{2}$.

\subsubsection*{Exercise \ref{subsec:Exercise-1-monads-12}}

The second definition, $\text{pu}_{L}\triangleq a\rightarrow\bbnum 0+\text{pu}_{F}(a)$,
fails the right identity law:
\begin{align*}
 & \text{pu}_{L}^{\uparrow L}\bef\text{ftn}_{L}=\,\begin{array}{|c||ccc|}
 & A & F^{A} & F^{L^{A}}\\
\hline A & \bbnum 0 & \text{pu}_{F} & \bbnum 0\\
F^{A} & \bbnum 0 & \bbnum 0 & \text{pu}_{L}^{\uparrow F}
\end{array}\,\bef\,\begin{array}{|c||cc|}
 & A & F^{A}\\
\hline A & \text{id} & \bbnum 0\\
F^{A} & \bbnum 0 & \text{id}\\
F^{L^{A}} & \bbnum 0 & \gamma^{\uparrow F}\bef\text{ftn}_{F}
\end{array}\,=\,\begin{array}{|c||cc|}
 & A & F^{A}\\
\hline A & \bbnum 0 & \text{pu}_{F}\\
F^{A} & \bbnum 0 & ...
\end{array}\,\neq\text{id}\quad.
\end{align*}
This matrix cannot be equal to the identity function because it has
a missing diagonal element.

\subsubsection*{Exercise \ref{subsec:Exercise-monad-composition-mm}}

If $M$ is a semimonad, we have the Kleisli composition $\diamond_{_{M}}$
that satisfies the associativity law. Define $\diamond_{_{L}}$ by:
\[
f^{:A\rightarrow M^{M^{B}}}\diamond_{_{L}}g^{:B\rightarrow M^{M^{C}}}\triangleq f\bef\text{ftn}_{M}\diamond_{_{M}}g\quad.
\]
Here, the expression of the form $f\bef h\diamond_{_{M}}g$ does not
need parentheses (see Statement~\ref{subsec:Statement-equivalence-kleisli-composition-and-flatMap}).

It is inconvenient to mix the Kleisli composition and the \lstinline!flatten!
method, so we express \lstinline!flatten! as:
\[
\text{ftn}_{M}^{:M^{M^{A}}\rightarrow M^{A}}=\text{flm}_{M}(\text{id}^{:M^{A}\rightarrow M^{A}})=\text{id}^{:M^{M^{A}}\rightarrow M^{M^{A}}}\diamond_{_{M}}\text{id}^{:M^{A}\rightarrow M^{A}}\quad.
\]
For brevity, we will omit type annotations from now on. So, we can
express the Kleisli composition $\diamond_{_{L}}$ through $\diamond_{_{M}}$
by:
\[
f\diamond_{_{L}}g\triangleq f\bef\gunderline{\left(\text{id}\diamond_{_{M}}\text{id}\right)\diamond_{_{M}}g}=f\bef\left(\text{id}\diamond_{_{M}}\left(\text{id}\diamond_{_{M}}g\right)\right)=\left(f\bef\text{id}\right)\diamond_{_{M}}\left(\text{id}\diamond_{_{M}}g\right)=f\diamond_{_{M}}\text{id}\diamond_{_{M}}g\quad.
\]

Associativity of $\diamond_{_{L}}$ then follows from associativity
of $\diamond_{_{M}}$ as:
\begin{align*}
\left(f\diamond_{_{L}}g\right)\diamond_{_{L}}h & =\left(f\diamond_{_{M}}\text{id}\diamond_{_{M}}g\right)\diamond_{_{M}}\text{id}\diamond_{_{M}}h=f\diamond_{_{M}}\text{id}\diamond_{_{M}}g\diamond_{_{M}}\text{id}\diamond_{_{M}}h\quad,\\
f\diamond_{_{L}}\left(g\diamond_{_{L}}h\right) & =f\diamond_{_{M}}\text{id}\diamond_{_{M}}\left(g\diamond_{_{M}}\text{id}\diamond_{_{M}}h\right)=f\diamond_{_{M}}\text{id}\diamond_{_{M}}g\diamond_{_{M}}\text{id}\diamond_{_{M}}h\quad.
\end{align*}

This definition of $\diamond_{_{L}}$ corresponds to a definition
of $\text{ftn}_{L}$ that flattens the \emph{first} three layers of
$M$ in $M\circ M\circ M\circ M$:
\[
\text{ftn}_{L}\triangleq\text{ftn}_{M}\bef\text{ftn}_{M}\quad.
\]

An alternative definition will flatten the \emph{last} three layers:
\[
\text{ftn}_{L}\triangleq(\text{ftn}_{M}\bef\text{ftn}_{M})^{\uparrow M}\quad.
\]
Both definitions satisfy the associativity law and so define a semimonad
$L$,
\[
\text{ftn}_{L}^{\uparrow L}\bef\text{ftn}_{L}=\text{ftn}_{L}\bef\text{ftn}_{L}\quad.
\]
Let us verify that directly. For the first definition:
\begin{align*}
 & \gunderline{(\text{ftn}_{M}\bef\text{ftn}_{M})^{\uparrow M\uparrow M}\bef\text{ftn}_{M}}\bef\text{ftn}_{M}=\text{ftn}_{M}\bef\big(\gunderline{\text{ftn}_{M}\bef\text{ftn}_{M}}\big)^{\uparrow M}\bef\text{ftn}_{M}\\
{\color{greenunder}\text{associativity of }M:}\quad & =\gunderline{\text{ftn}_{M}\bef\text{ftn}_{M}^{\uparrow M\uparrow M}}\bef\text{ftn}_{M}^{\uparrow M}\bef\text{ftn}_{M}=\gunderline{\text{ftn}_{M}^{\uparrow M}\bef\text{ftn}_{M}}\bef\gunderline{\text{ftn}_{M}^{\uparrow M}\bef\text{ftn}_{M}}\\
{\color{greenunder}\text{associativity of }M:}\quad & =\text{ftn}_{M}\bef\text{ftn}_{M}\bef\text{ftn}_{M}\bef\text{ftn}_{M}\quad.
\end{align*}
For the second definition, we just apply $^{\uparrow M}$ to the preceding
derivation.

A definition of \lstinline!flatten! that flattens separately the
first two and the last two layers of $M$ ($\text{ftn}_{L}\triangleq\text{ftn}_{M}\bef\text{ftn}_{M}^{\uparrow M}$)
will fail the associativity law when $M$ is chosen in a suitable
way.

If $M$ is a full monad, we have $\text{pu}_{M}$ that satisfies the
identity laws. Define $\text{pu}_{L}$ by:
\[
\text{pu}_{L}\triangleq\text{pu}_{M}\bef\text{pu}_{M}\quad.
\]
There is no other way of defining $\text{pu}_{L}$ since we cannot
obtain a value of type $M^{A}$ other than via $\text{pu}_{M}$.

With any of the two possible definitions of $\text{ftn}_{L}$, at
least one of the identity laws for $\diamond_{_{L}}$ fails.

For the definition of $\text{ftn}_{L}$ that flattens the first three
layers, the right identity law will fail:
\[
\text{pu}_{L}^{\uparrow L}\bef\text{ftn}_{L}=\gunderline{(\text{pu}_{M}\bef\text{pu}_{M})^{\uparrow M\uparrow M}\bef\text{ftn}_{M}}\bef\text{ftn}_{M}=\text{ftn}_{M}\bef\text{pu}_{M}^{\uparrow M}\bef\gunderline{\text{pu}_{M}^{\uparrow M}\bef\text{ftn}_{M}}=\text{ftn}_{M}\bef\text{pu}_{M}^{\uparrow M}\quad.
\]
In general, this function is not equal to an identity function, because
\lstinline!flatten! merges the two monadic layers of $M$ and in
that way loses information about a value of type $M^{M^{A}}$.

For the definition of $\text{ftn}_{L}$ that flattens the last three
layers, the left identity law will fail:
\[
\text{pu}_{L}\bef\text{ftn}_{L}=\text{pu}_{M}\bef\gunderline{\text{pu}_{M}\bef(\text{ftn}_{M}}\bef\text{ftn}_{M})^{\uparrow M}=\gunderline{\text{pu}_{M}\bef\text{ftn}_{M}}\bef\text{pu}_{M}\bef\text{ftn}_{M}^{\uparrow M}=\text{pu}_{M}\bef\text{ftn}_{M}^{\uparrow M}=\text{ftn}_{M}\bef\text{pu}_{M}\quad.
\]
In general, this function is not equal to an identity function.

\addsec{Chapter \ref{chap:8-Applicative-functors,-contrafunctors}}

\subsubsection*{Exercise \ref{subsec:Exercise-simplify-law-omit-lifted-function}}

We may choose $f=\text{id}$ and derive $u=v$ from the given law.
Conversely, if $u=v$ then $u\bef f^{\uparrow F}=v\bef f^{\uparrow F}$
for any function $f$.

\subsubsection*{Exercise \ref{subsec:Exercise-applicative-II-4-1}}

To verify the law, write:
\begin{align*}
 & \text{ap}\,(r)(\text{pu}_{L}(a))=\text{zip}\big(r\times\text{pu}_{L}(a)\big)\triangleright\text{eval}^{\uparrow L}\\
{\color{greenunder}\text{right identity law of }\text{zip}:}\quad & =r\triangleright\left(f\rightarrow f\times a\right)^{\uparrow L}\bef\text{eval}^{\uparrow L}\\
{\color{greenunder}\text{composition under }^{\uparrow L}:}\quad & =r\triangleright(f\rightarrow f(a))\quad.
\end{align*}


\subsubsection*{Exercise \ref{subsec:Exercise-function-type-construction-not-applicative}}

\textbf{(a)} We cannot implement \lstinline!zip! as a fully parametric
function having this type:
\[
\text{zip}_{F}(p^{:(A\rightarrow P)\rightarrow Q}\times q^{:(B\rightarrow P)\rightarrow Q})\triangleq h^{:A\times B\rightarrow P}\rightarrow\text{???}^{:Q}\quad.
\]
\begin{lstlisting}
import io.chymyst.ch._
type S[A] = (A => P) => Q

scala> def zip[A, B](p: S[A], q: S[B]): S[(A, B)] = implement
type ((A => P) => Q) => ((B => P) => Q) => (Tuple2[A,B] => P) => Q cannot be implemented
\end{lstlisting}

The reason it cannot be implemented is that the only way of getting
a value of type $Q$ is to call the given functions $p$ or $q$.
But we cannot call $p$ or $q$ since we cannot supply their arguments:
we have a function of type $A\times B\rightarrow P$, and we cannot
produce a function of type $A\rightarrow P$ or $B\rightarrow P$
out of it.

The only solution is for \lstinline!zip! to ignore its arguments
and always return the empty value $e_{Q}$. However, that implementation
loses information and would fail the identity laws.

\textbf{(b)} We \emph{can} implement the type signature of \lstinline!zip!,
but only in a trivial way:
\[
\text{zip}_{F}(p^{:(A\rightarrow P)\rightarrow\bbnum 1+A}\times q^{:(B\rightarrow P)\rightarrow\bbnum 1+B})\triangleq h^{:A\times B\rightarrow P}\rightarrow\text{???}^{:\bbnum 1+A\times B}\quad.
\]
The functions $p$ and $q$ cannot be called since we cannot supply
their arguments, just as in part \textbf{(a)}. The only solution is
that \lstinline!zip! should ignore its arguments and always return
$1+\bbnum 0^{:A\times B}$. However, that implementation loses information
and would fail the identity laws.

\subsubsection*{Exercise \ref{subsec:Exercise-additional-law-of-ap}}

We define $\text{ex}_{F}(h\times g)\triangleq\text{ex}_{H}(h)$ and
obtain:
\begin{align*}
 & \text{ex}_{F}(\text{zip}_{F}(h_{1}\times g_{1}\times h_{2}\times g_{2}))=\text{ex}_{F}(\text{zip}_{H}(h_{1}\times h_{2})\times\text{zip}_{G}(g_{1}\times g_{2}))=\text{ex}_{H}(\text{zip}_{H}(h_{1}\times h_{2}))\quad,\\
 & \text{ex}_{F}(h_{1}\times g_{1})\times\text{ex}_{F}(h_{2}\times g_{2})=\text{ex}_{H}(h_{1})\times\text{ex}_{H}(h_{2})\quad.
\end{align*}
The two sides are equal due to the compatibility law of $\text{zip}_{H}$
and $\text{ex}_{H}$.

\subsubsection*{Exercise \ref{subsec:Exercise-profunctor-example}}

The first and the third occurrences of $A$ in $Q^{A}$ are contravariant
while the others are covariant. So, we define a profunctor $P^{X,Y}$
and get:
\[
P^{X,Y}\triangleq\left(X\rightarrow\text{Int}\right)\times Y\times\left(X\rightarrow Y\right)\quad,\quad\quad Q^{A}=P^{A,A}\quad.
\]


\subsubsection*{Exercise \ref{subsec:Exercise-applicative-II-11}(b)}

Hint: Use $P^{A}\triangleq A\rightarrow Z$ as a counterexample.

\addsec{Chapter \ref{chap:9-Traversable-functors-and}}

\subsubsection*{Exercise \ref{subsec:Exercise-traversables-7-1}(c)}

We use the definitions of \lstinline!NEL[A]! and of the function
\lstinline!foldLeft! shown in Example~\ref{subsec:Disjunctive-Example-non-empty-list-foldLeft}.
\begin{lstlisting}
def foldMap[A, M: Monoid](f: A => M)(t: NEL[A]): M =
  foldLeft(t)(Monoid[M].empty)((m, a) => f(a) |+| m)
\end{lstlisting}


\subsubsection*{Exercise \ref{subsec:Exercise-traversables-1}}

Begin by expressing \lstinline!consume! via \lstinline!sequence!
and back:
\begin{align}
 & \text{consume}_{L}(f^{:L^{A}\rightarrow B})(p^{:L^{F^{A}}})=p\triangleright\text{seq}_{L}\triangleright f^{\uparrow F}\quad,\quad\text{or equivalently}:\quad\text{consume}_{L}(f)=\text{seq}_{L}\bef f^{\uparrow F}\quad,\label{eq:consume-via-seq}\\
 & \text{seq}_{L}=\text{consume}_{L}(\text{id}^{:L^{A}\rightarrow L^{A}})\quad.\label{eq:seq-via-consume}
\end{align}
The pattern is similar to that in the equivalence of \lstinline!sequence!
and \lstinline!traverse!. We could use the Yoneda identity directly,
as we did in the proof of Statement~\ref{subsec:Statement-tr-equivalent-to-ftr}.
Nevertheless, let us write out the derivation in detail.

We need to derive the equivalence between \lstinline!consume! and
\lstinline!sequence! in both directions. To figure out the necessary
naturality law, we begin with the direction that restores \lstinline!consume!
from \lstinline!sequence! because \lstinline!consume! is the more
complicated function (having two type parameters).

\textbf{(a)} Given a function \lstinline!consume!, we define \lstinline!sequence!
via Eq.~(\ref{eq:seq-via-consume}) and then define a new function
\lstinline!consume!$^{\prime}$ via Eq.~(\ref{eq:consume-via-seq}).
Then we need to show that \lstinline!consume!$^{\prime}$ equals
\lstinline!consume!. For an arbitrary $f^{:L^{A}\rightarrow B}$,
we write:
\begin{align*}
{\color{greenunder}\text{expect to equal }\text{consume}_{L}(f):}\quad & \text{consume}_{L}^{\prime}(f)=\text{seq}_{L}\bef f^{\uparrow F}=\text{consume}_{L}(\text{id})\bef f^{\uparrow F}=???
\end{align*}
If we know nothing about \lstinline!consume!, we cannot conclude
that $\text{consume}_{L}(f)\overset{?}{=}\text{consume}_{L}(\text{id})\bef f^{\uparrow F}$
because the function $f$ is arbitrary and the value $\text{consume}_{L}(f)$
does not need to be related in any way to $\text{consume}_{L}(\text{id})$.
We must use a naturality law involving a lifted function applied after
\lstinline!consume!:
\begin{equation}
\text{consume}_{L}(f^{:L^{A}\rightarrow B})\bef(g^{:B\rightarrow C})^{\uparrow F}=\text{consume}_{L}(f\bef g)\quad.\label{eq:consume-naturality-law}
\end{equation}
If this law holds, we can derive the required equation:
\[
\text{consume}_{L}^{\prime}(f)=\text{consume}_{L}(\text{id})\bef f^{\uparrow F}=\text{consume}_{L}(\text{id}\bef f)=\text{consume}_{L}(f)\quad.
\]

\textbf{(b)} Given a function \lstinline!sequence!, we first define
\lstinline!consume! via Eq.~(\ref{eq:consume-via-seq}) and then
define a new function \lstinline!sequence!$^{\prime}$ via Eq.~(\ref{eq:seq-via-consume}).
We then show that \lstinline!sequence!$^{\prime}$ equals \lstinline!sequence!:
\[
\text{seq}_{L}^{\prime}=\text{consume}_{L}(\text{id})=\text{seq}_{L}\bef\gunderline{\text{id}^{\uparrow F}}=\text{seq}\quad.
\]

If a function \lstinline!consume! is defined via \lstinline!sequence!,
the naturality law~(\ref{eq:consume-naturality-law}) will hold automatically:
\[
\text{consume}_{L}(f)\bef g^{\uparrow F}=\text{seq}_{L}\bef\gunderline{f^{\uparrow F}\bef g^{\uparrow F}}=\text{seq}_{L}\bef(f\bef g)^{\uparrow F}=\text{consume}_{L}\left(f\bef g\right)\quad.
\]


\subsubsection*{Exercise \ref{subsec:Exercise-traversables-laws-1}}

The naturality law of \lstinline!ftr! with respect to the type parameter
$B$ is:
\begin{equation}
\text{ftr}\,(f^{:A\rightarrow G^{B}})\bef(g^{:B\rightarrow C})^{\uparrow H}=\text{ftr}\,(f\bef g^{\uparrow G})\quad.\label{eq:ftr-right-naturality-law}
\end{equation}
The type signature of \lstinline!tr! has only one type parameter,
so \lstinline!tr! has only one naturality law:
\begin{equation}
(f^{:A\rightarrow B})^{\uparrow G\uparrow F}\bef\text{tr}=\text{tr}\bef f^{\uparrow H}\quad.\label{eq:tr-naturality-law}
\end{equation}

1) We show that the naturality law~(\ref{eq:tr-naturality-law})
of \lstinline!tr! follows from the two naturality laws of \lstinline!ftr!.
Express \lstinline!tr! as $\text{ftr}\left(\text{id}\right)$ and
substitute into the two sides of \lstinline!tr!\textsf{'}s naturality law:
\begin{align*}
{\color{greenunder}\text{left-hand side}:}\quad & f^{\uparrow G\uparrow F}\bef\text{tr}=f^{\uparrow G\uparrow F}\bef\text{ftr}\left(\text{id}\right)\\
{\color{greenunder}\text{naturality law~(\ref{eq:ftr-left-naturality-law})}:}\quad & \quad=\text{ftr}\,(f^{\uparrow G}\bef\text{id})=\text{ftr}\,(f^{\uparrow G})\quad,\\
{\color{greenunder}\text{right-hand side}:}\quad & \text{tr}\bef f^{\uparrow H}=\text{ftr}\left(\text{id}\right)\bef f^{\uparrow H}\\
{\color{greenunder}\text{naturality law~(\ref{eq:ftr-right-naturality-law})}:}\quad & \quad=\text{ftr}\,(\text{id}\bef f^{\uparrow G})=\text{ftr}\,(f^{\uparrow G})\quad.
\end{align*}
The two sides of the law are now equal.

2) We show that the second naturality law of \lstinline!ftr! follows
from the naturality law of \lstinline!tr!. Express \lstinline!ftr!
through \lstinline!tr! as $\text{ftr}\,(f)=f^{\uparrow F}\bef\text{tr}$,
and substitute into the two sides of \lstinline!ftr!\textsf{'}s naturality
law~(\ref{eq:ftr-right-naturality-law}): 
\begin{align*}
{\color{greenunder}\text{left-hand side}:}\quad & \text{ftr}\,(f\big)\bef g^{\uparrow H}=f^{\uparrow F}\bef\gunderline{\text{tr}\bef g^{\uparrow H}}\\
{\color{greenunder}\text{naturality law~(\ref{eq:tr-naturality-law}) of }\text{tr}:}\quad & =f^{\uparrow F}\bef g^{\uparrow G\uparrow F}\bef\text{tr}=(f\bef g^{\uparrow G})^{\uparrow F}\bef\text{tr}\quad,\\
{\color{greenunder}\text{right-hand side}:}\quad & \text{ftr}\,(f\bef g^{\uparrow G})=(f\bef g^{\uparrow G})^{\uparrow F}\bef\text{tr}\quad.
\end{align*}
The two sides of the law are now equal.

\subsubsection*{Exercise \ref{subsec:Exercise-traversables-5}}

To verify the identity law~(\ref{eq:identity-law-of-sequence}),
set $F=\text{Id}$ in the definition~(\ref{eq:def-sequence-for-functor-composition})
of $\text{seq}_{L}$:
\begin{align*}
{\color{greenunder}\text{expect to equal }\text{id}:}\quad & \text{seq}_{L}^{\text{Id},A}=(\text{seq}_{N}^{\text{Id},A})^{\uparrow M}\bef\text{seq}_{M}^{\text{Id},N^{A}}\\
{\color{greenunder}\text{identity laws of }\text{seq}_{M}\text{ and }\text{seq}_{N}:}\quad & =\text{id}^{\uparrow M}\bef\text{id}=\text{id}\quad.
\end{align*}

To verify the composition law~(\ref{eq:composition-law-of-sequence}),
begin with the left-hand side:
\begin{align*}
{\color{greenunder}\text{expect to equal }\text{seq}_{L}^{F\circ G,A}:}\quad & \text{seq}_{L}^{F,G^{A}}\bef(\text{seq}_{L}^{G,A})^{\uparrow F}\\
 & \quad=(\text{seq}_{N}^{F,G^{A}})^{\uparrow M}\bef\gunderline{\text{seq}_{M}^{F,N^{G^{A}}}\bef\big((\text{seq}_{N}^{G,A})^{\uparrow M}}\bef\text{seq}_{M}^{G,N^{A}}\big)^{\uparrow F}\\
{\color{greenunder}\text{naturality law of }\text{seq}_{M}:}\quad & \quad=(\text{seq}_{N}^{F,G^{A}})^{\uparrow M}\bef(\text{seq}_{N}^{G,A})^{\uparrow F\uparrow M}\bef\text{seq}_{M}^{F,G^{N^{A}}}\bef(\text{seq}_{M}^{G,N^{A}})^{\uparrow F}\\
{\color{greenunder}\text{composition laws of }\text{seq}_{M}\text{ and }\text{seq}_{N}:}\quad & \quad=(\text{seq}_{N}^{F\circ G,A})^{\uparrow M}\bef\text{seq}_{M}^{F\circ G,N^{A}}=\text{seq}_{L}^{F\circ G,A}\quad.
\end{align*}


\subsubsection*{Exercise \ref{subsec:Exercise-traversables-10}}

The empty elements of the monoids $M^{R}$ and $M^{S}$ are, by definition,
$\text{pu}_{M}(e_{R})$ and $\text{pu}_{M}(e_{S})$ respectively.
The binary operations of $M^{R}$ and $M^{S}$ are:
\begin{align*}
 & p\underset{M^{R}}{\oplus}q\triangleq p\triangleright\text{flm}_{M}\big(u^{:R}\rightarrow q\triangleright(v^{:R}\rightarrow u\oplus_{R}v)^{\uparrow M}\big)\quad,\\
 & p\underset{M^{S}}{\oplus}q\triangleq p\triangleright\text{flm}_{M}\big(u^{:S}\rightarrow q\triangleright(v^{:S}\rightarrow u\oplus_{S}v)^{\uparrow M}\big)\quad.
\end{align*}
To verify the properties of the monoid morphism, begin with the identity
law:
\begin{align*}
{\color{greenunder}\text{expect to equal }e_{M^{S}}:}\quad & \gunderline{e_{M^{R}}}\triangleright\phi^{\uparrow M}=e_{R}\triangleright\gunderline{\text{pu}_{M}\triangleright\phi^{\uparrow M}}=\gunderline{e_{R}\triangleright\phi}\triangleright\text{pu}_{M}=e_{S}\triangleright\text{pu}_{M}=e_{M^{S}}\quad.
\end{align*}
Now check the composition law. For any $p^{:M^{R}}$ and $q^{:M^{R}}$:
\begin{align*}
{\color{greenunder}\text{expect to equal }(p\underset{M^{R}}{\oplus}q)\triangleright\phi^{\uparrow M}:}\quad & (p\triangleright\phi^{\uparrow M})\underset{M^{S}}{\oplus}(q\triangleright\phi^{\uparrow M})\\
 & =(p\triangleright\gunderline{\phi^{\uparrow M})\triangleright\text{flm}_{M}}\big(u\rightarrow q\triangleright\phi^{\uparrow M}\triangleright(v\rightarrow u\oplus_{S}v)^{\uparrow M}\big)\\
{\color{greenunder}\text{naturality of }\text{flm}_{M}:}\quad & =p\triangleright\text{flm}_{M}\big(u\rightarrow q\triangleright(\gunderline{\phi\bef(}v\rightarrow\phi(u)\oplus_{S}v))^{\uparrow M}\big)\\
{\color{greenunder}\text{compute composition}:}\quad & =p\triangleright\text{flm}_{M}\big(u\rightarrow q\triangleright(v\rightarrow\gunderline{\phi(u)\oplus_{S}\phi(v)}))^{\uparrow M}\big)\\
{\color{greenunder}\text{composition law of }\phi:}\quad & =p\triangleright\text{flm}_{M}\big(u\rightarrow q\triangleright(v\rightarrow\phi(u\oplus_{R}v))^{\uparrow M}\big)\\
{\color{greenunder}\text{move }\phi^{\uparrow M}\text{ outside}:}\quad & =p\triangleright\text{flm}_{M}\big(u\rightarrow q\triangleright(v\rightarrow u\oplus_{R}v)^{\uparrow M}\bef\phi^{\uparrow M}\big)\\
{\color{greenunder}\text{naturality of }\text{flm}_{M}:}\quad & =\gunderline{p\triangleright\text{flm}_{M}\big(u\rightarrow q\triangleright(v\rightarrow u\oplus_{R}v)^{\uparrow M}\big)}\bef\phi^{\uparrow M}\\
 & =(p\underset{M^{R}}{\oplus}q)\triangleright\phi^{\uparrow M}\quad.
\end{align*}


\subsubsection*{Exercise \ref{subsec:Exercise-traversables-10-1}}

The empty elements of the monoids $M^{R}$ and $N^{R}$ are, by definition,
are $\text{pu}_{M}(e_{R})$ and $\text{pu}_{N}(e_{R})$ respectively.
The binary operations of $M^{R}$ and $N^{R}$ are:
\begin{align*}
 & p\underset{M^{R}}{\oplus}q\triangleq p\triangleright\text{flm}_{M}\big(u^{:R}\rightarrow q\triangleright(v^{:R}\rightarrow u\oplus_{R}v)^{\uparrow M}\big)\quad,\\
 & p\underset{N^{R}}{\oplus}q\triangleq p\triangleright\text{flm}_{N}\big(u^{:R}\rightarrow q\triangleright(v^{:R}\rightarrow u\oplus_{R}v)^{\uparrow N}\big)\quad.
\end{align*}
To verify the properties of the monoid morphism, begin with the identity
law:
\begin{align*}
{\color{greenunder}\text{expect to equal }e_{N^{R}}:}\quad & \gunderline{e_{M^{R}}}\triangleright\phi=\gunderline{e_{R}\triangleright\text{pu}_{M}}\triangleright\phi=e_{R}\triangleright\text{pu}_{N}=e_{N^{R}}\quad.
\end{align*}
Now check the composition law. For any $p^{:M^{R}}$ and $q^{:M^{R}}$:
\begin{align*}
{\color{greenunder}\text{expect to equal }(p\underset{M^{R}}{\oplus}q)\triangleright\phi:}\quad & (p\triangleright\phi)\underset{N^{R}}{\oplus}(q\triangleright\phi)=(p\triangleright\phi)\triangleright\text{flm}_{N}\big(u\rightarrow q\triangleright\gunderline{\phi\triangleright(v\rightarrow u\oplus_{R}v)^{\uparrow N}}\big)\\
{\color{greenunder}\text{naturality of }\phi:}\quad & =p\triangleright\gunderline{\phi\bef\big(u\rightarrow q\triangleright(v\rightarrow u\oplus_{R}v)^{\uparrow M}\bef\phi\big)^{\uparrow N}}\bef\text{ftn}_{N}\\
{\color{greenunder}\text{naturality of }\phi:}\quad & =p\triangleright\big(u\rightarrow q\triangleright(v\rightarrow u\oplus_{R}v)^{\uparrow M}\big)^{\uparrow M}\bef\gunderline{\phi\bef\phi^{\uparrow N}\bef\text{ftn}_{N}}\\
{\color{greenunder}\text{composition law of }\phi:}\quad & =p\triangleright\big(u\rightarrow q\triangleright(v\rightarrow u\oplus_{R}v)^{\uparrow M}\gunderline{\big)^{\uparrow M}\bef\text{ftn}_{M}}\bef\phi\\
{\color{greenunder}\text{definition of }\text{ftn}_{M}:}\quad & =p\triangleright\text{flm}_{M}\big(u\rightarrow q\triangleright(v\rightarrow u\oplus_{R}v)^{\uparrow M}\big)\bef\phi=(p\underset{M^{R}}{\oplus}q)\triangleright\phi\quad.
\end{align*}


\addsec{Chapter \ref{chap:monad-transformers}}

\subsubsection*{Exercise \ref{subsec:Exercise-monad-transformer-extra-layer}}

The identity law of monad morphisms is:
\[
\text{pu}_{\text{Id}}\bef\phi=\text{pu}_{M}\quad.
\]
Since $\text{pu}_{\text{Id}}=\text{id}$, we get $\phi=\text{pu}_{M}$.
So, there can be no other monad morphisms $\text{Id}\leadsto M$.

\subsubsection*{Exercise \ref{subsec:Exercise-monad-transformer-extra-layer-5}}

The identity law of monad morphisms says that $\text{pu}_{\bbnum 1}\bef\phi=\text{pu}_{M}$.
However, $\text{pu}_{\bbnum 1}=\_^{:A}\rightarrow1$ and ignores its
argument. So, $\text{pu}_{M}$ is also a function that ignores its
argument. By the monad $M$\textsf{'}s left identity law, $\text{pu}_{M}\bef\text{ftn}_{M}=\text{id}^{:M^{A}\rightarrow M^{A}}$.
So, the identity function $\text{id}:M^{A}\rightarrow M^{A}$ ignores
its argument. It always returns the same value of type $M^{A}$, say
$m_{0}$. It follows that the type $M^{A}$ has only one distinct
value, namely $m_{0}$. So $M^{A}\cong\bbnum 1$. This argument holds
for each type $A$. 

\subsubsection*{Exercise \ref{subsec:Exercise-monad-transformer-extra-layer-3}}

\textbf{(a)} For any monad morphism $\phi:M\leadsto N$, the monadic
naturality law must hold:
\[
\text{dbl}\bef\phi\overset{?}{=}\phi\bef\text{dbl}\quad.
\]
The left-hand side applied to an arbitrary value $m^{:M^{A}}$ is:
\begin{align*}
 & m\triangleright\text{dbl}\bef\phi=m\triangleright(\_\rightarrow m)^{\uparrow M}\bef\gunderline{\text{ftn}_{M}\bef\phi}\\
{\color{greenunder}\text{monad morphism law of }\phi:}\quad & =m\triangleright\gunderline{(\_\rightarrow m)^{\uparrow M}\bef\phi^{\uparrow M}}\bef\phi\bef\text{ftn}_{M}\\
{\color{greenunder}\text{compute composition}:}\quad & =m\triangleright\gunderline{(\_\rightarrow m\triangleright\phi)^{\uparrow M}\bef\phi}\bef\text{ftn}_{M}\\
{\color{greenunder}\text{naturality of }\phi:}\quad & =m\triangleright\phi\bef(\_\rightarrow m\triangleright\phi)^{\uparrow M}\bef\text{ftn}_{M}\quad.
\end{align*}
The right-hand side applied to $m$ is:
\begin{align*}
 & m\triangleright\phi\bef\text{dbl}=m\triangleright\phi\triangleright\text{dbl}=m\triangleright\phi\triangleright(\_\rightarrow m\triangleright\phi)^{\uparrow M}\bef\text{ftn}_{M}\\
 & =m\triangleright\phi\bef(\_\rightarrow m\triangleright\phi)^{\uparrow M}\bef\text{ftn}_{M}\quad.
\end{align*}
Both sides are now equal.

\textbf{(b)} The identity law holds:
\[
a\triangleright\text{pu}_{M}\bef\text{dbl}=a\triangleright\text{pu}_{M}\triangleright\text{flm}_{M}(\_\rightarrow a\triangleright\text{pu}_{M})=a\triangleright(\_\rightarrow a\triangleright\text{pu}_{M})=a\triangleright\text{pu}_{M}\quad.
\]
The composition law:
\[
\text{dbl}^{\uparrow M}\bef\text{dbl}\bef\text{ftn}_{M}\overset{?}{=}\text{ftn}_{M}\bef\text{dbl}\quad.
\]
Simplify the left-hand side, applying to an arbitrary value $p:M^{M^{A}}$.
\begin{align*}
{\color{greenunder}\text{left-hand side}:}\quad & p\triangleright\text{dbl}^{\uparrow M}\bef\text{dbl}\bef\text{ftn}_{M}=p\triangleright\text{dbl}^{\uparrow M}\triangleright(\_\rightarrow p\triangleright\text{dbl}^{\uparrow M})^{\uparrow M}\bef\gunderline{\text{ftn}_{M}\bef\text{ftn}_{M}}\\
{\color{greenunder}\text{associativity law of }M:}\quad & =p\triangleright\text{dbl}^{\uparrow M}\bef(\_\rightarrow p\triangleright\text{dbl}^{\uparrow M})^{\uparrow M}\bef\text{ftn}_{M}^{\uparrow M}\bef\text{ftn}_{M}\\
 & =p\triangleright\big(\text{dbl}\bef(\_\rightarrow p\triangleright\text{dbl}^{\uparrow M})\bef\text{ftn})\big)^{\uparrow M}\bef\text{ftn}_{M}=p\triangleright\big(\_\rightarrow p\triangleright\text{dbl}^{\uparrow M}\bef\text{ftn}_{M})\big)^{\uparrow M}\bef\text{ftn}_{M}\\
 & =p\triangleright\text{flm}_{M}\big(\_\rightarrow p\triangleright\text{flm}_{M}(\text{dbl})\big)\quad.
\end{align*}
The right-hand side:
\begin{align*}
{\color{greenunder}\text{right-hand side}:}\quad & p\triangleright\text{ftn}_{M}\bef\text{dbl}=p\triangleright\gunderline{\text{ftn}_{M}\bef(\_\rightarrow p\triangleright\text{ftn}_{M})^{\uparrow M}}\bef\text{ftn}_{M}\\
{\color{greenunder}\text{naturality of }\text{ftn}_{M}:}\quad & =p\triangleright(\_\rightarrow p\triangleright\text{ftn}_{M})^{\uparrow M\uparrow M}\bef\gunderline{\text{ftn}_{M}\bef\text{ftn}_{M}}\\
{\color{greenunder}\text{associativity law of }M:}\quad & =p\triangleright(\_\rightarrow p\triangleright\text{ftn}_{M})^{\uparrow M\uparrow M}\bef\text{ftn}_{M}^{\uparrow M}\bef\text{ftn}_{M}\\
 & =p\triangleright\big((\_\rightarrow p\triangleright\text{ftn}_{M})^{\uparrow M}\bef\text{ftn}_{M}\big)^{\uparrow M}\bef\text{ftn}_{M}\\
 & =p\triangleright\text{flm}_{M}\big(\text{flm}_{M}(\_\rightarrow p\triangleright\text{ftn}_{M})\big)\quad.
\end{align*}
The simplification is stuck \textemdash{} we cannot make the two sides
coincide; no law can be applied to simplify further.

Choosing specific values $p=\text{pu}_{M}(m)$ or $p=\text{pu}_{M}^{\uparrow M}(m)$
does not help to obtain a counter-example: the laws hold for those
specific choices.

To obtain a counter-example, consider a specific non-commutative monad,
such as \lstinline!List!. For the \lstinline!List! monad, the \lstinline!double!
function repeats the entire list as many times as elements in the
list:
\begin{lstlisting}
scala> double(List(1, 2))
res0: List[Int] = List(1, 2, 1, 2)

scala> val ll = List(List(1, 2), List(3))
ll: List[List[Int]] = List(List(1, 2), List(3))

scala> double(ll.flatten)
res1: List[Int] = List(1, 2, 3, 1, 2, 3, 1, 2, 3)

scala> double(ll.map(double))
res2: List[List[Int]] = List(List(1, 2, 1, 2), List(3), List(1, 2, 1, 2), List(3)) 

scala> double(ll.map(double)).flatten
res3: List[Int] = List(1, 2, 1, 2, 3, 1, 2, 1, 2, 3)
\end{lstlisting}
The composition law does not hold.

\subsubsection*{Exercise \ref{subsec:Statement-search-and-selector-monads}}

The identity law does not hold:
\begin{align*}
{\color{greenunder}\text{expect to equal }a\triangleright\text{pu}_{\text{Search}}:}\quad & a\triangleright\text{pu}_{\text{Sel}}\bef\text{finder}=(\_^{:A\rightarrow\bbnum 1+P}\rightarrow a)\triangleright\text{finder}\\
 & =p\rightarrow p(a)\triangleright\,\begin{array}{|c||cc|}
 & \bbnum 1 & A\\
\hline \bbnum 1 & \text{id} & \bbnum 0\\
P & \bbnum 0 & a
\end{array}\,=p\rightarrow a\triangleright p\bef(\_\rightarrow a)^{\uparrow\text{Opt}}\neq p\rightarrow\bbnum 0+a\quad.
\end{align*}
The result should have been $\text{pu}_{\text{Search}}(a)=\_\rightarrow\text{pu}_{\text{Opt}}(a)=\_\rightarrow\bbnum 0+a$.

\subsubsection*{Exercise \ref{subsec:Exercise-selector-and-continuation-monads}}

To verify the identity law:
\begin{align*}
{\color{greenunder}\text{expect to equal }a\triangleright\text{pu}_{\text{Cont}}:}\quad & a\triangleright\text{pu}_{\text{Sel}}\bef\text{scc}=f\rightarrow f(f\triangleright(\gunderline{a\triangleright\text{pu}_{\text{Sel}}}))\\
{\color{greenunder}\text{use definition of }\text{pu}_{\text{Sel}}:}\quad & =f\rightarrow f(f\triangleright(\_\rightarrow a))=f\rightarrow f(a)=a\triangleright\text{pu}_{\text{Cont}}\quad.
\end{align*}
To verify the composition law, we begin with the flipped Kleisli formulation
of the two monads. The flipped Kleisli functions have types $\left(B\rightarrow P\right)\rightarrow A\rightarrow P$
(for \lstinline!Cont!) and $\left(B\rightarrow P\right)\rightarrow A\rightarrow B$
(for \lstinline!Sel!).
\begin{align*}
 & f^{:\left(B\rightarrow P\right)\rightarrow A\rightarrow P}\tilde{\diamond}_{_{\text{Cont}}}g^{:\left(C\rightarrow P\right)\rightarrow B\rightarrow P}\triangleq h^{:C\rightarrow P}\rightarrow h\triangleright g\triangleright f=g\bef f\quad,\\
 & f^{:\left(B\rightarrow P\right)\rightarrow A\rightarrow B}\tilde{\diamond}_{_{\text{Sel}}}g^{:\left(C\rightarrow P\right)\rightarrow B\rightarrow C}\triangleq h^{:C\rightarrow P}\rightarrow f(g(h)\bef h)\bef g(h)\quad.
\end{align*}
We need to check that \lstinline!Sel!\textsf{'}s Kleisli composition ($\tilde{\diamond}_{_{\text{Sel}}}$)
is mapped to \lstinline!Cont!\textsf{'}s Kleisli composition ($\tilde{\diamond}_{_{\text{Cont}}}$).
First, we need to modify \lstinline!scc! so that it works on the
flipped Kleisli functions:
\[
\tilde{\text{scc}}^{A,B}:\left(\left(B\rightarrow P\right)\rightarrow A\rightarrow B\right)\rightarrow\left(B\rightarrow P\right)\rightarrow A\rightarrow P\quad,\quad\tilde{\text{scc}}=c^{:\left(\left(B\rightarrow P\right)\rightarrow A\rightarrow B\right)}\rightarrow k^{:B\rightarrow P}\rightarrow c(k)\bef k\quad.
\]

Note that the implementation of \lstinline!scc! is uniquely determined
by its type signature, and so is the implementation of $\tilde{\text{scc}}$.
So, we can spare ourselves the effort of translating $\text{scc}$
into $\tilde{\text{scc}}$.

Now it remains to show that for any $f^{:\left(B\rightarrow P\right)\rightarrow A\rightarrow B}$
and $g^{:\left(C\rightarrow P\right)\rightarrow B\rightarrow C}$
the law holds:
\[
(f\triangleright\tilde{\text{scc}}^{A,B})\tilde{\diamond}_{_{\text{Cont}}}(g\triangleright\tilde{\text{scc}}^{B,C})=(f\tilde{\diamond}_{_{\text{Sel}}}g)\triangleright\tilde{\text{scc}}^{A,C}\quad.
\]
Rewrite the two sides of the law separately:
\begin{align*}
{\color{greenunder}\text{left-hand side}:}\quad & (f\triangleright\tilde{\text{scc}})\tilde{\diamond}_{_{\text{Cont}}}(g\triangleright\tilde{\text{scc}})=(g\triangleright\tilde{\text{scc}})\bef(f\triangleright\tilde{\text{scc}})=(k\rightarrow g(k)\bef k)\bef(h\rightarrow f(h)\bef h)\\
 & \quad=k\rightarrow f(g(k)\bef k)\bef g(k)\bef k\quad,\\
{\color{greenunder}\text{right-hand side}:}\quad & (f\tilde{\diamond}_{_{\text{Sel}}}g)\triangleright\tilde{\text{scc}}=\big(h\rightarrow f(g(h)\bef h)\bef g(h)\big)\triangleright\tilde{\text{scc}}=k\rightarrow\big(h\rightarrow f(g(h)\bef h)\bef g(h)\big)(k)\bef k\\
 & \quad=k\rightarrow f(g(k)\bef k)\bef g(k)\bef k\quad.
\end{align*}
Both sides are now equal.

\subsubsection*{Exercise \ref{par:Exercise-mt-3-1}}

The operations $\text{pu}_{N}$ and $\text{ftn}_{N}$ are defined
by:
\[
\text{pu}_{N}\triangleq\,\begin{array}{|c||cc|}
 & A & M^{A}\\
\hline A & \text{id} & \bbnum 0
\end{array}\quad,\quad\quad\text{ftn}_{N}\triangleq\,\begin{array}{|c||cc|}
 & A & M^{A}\\
\hline A & \text{id} & \bbnum 0\\
M^{A} & \bbnum 0 & \text{id}\\
M^{N^{A}} & \bbnum 0 & \gamma_{M}^{\uparrow M}\bef\text{ftn}_{M}
\end{array}\quad.
\]

To verify the identity law:
\begin{align*}
{\color{greenunder}\text{expect to equal }\text{pu}_{M}:}\quad & \text{pu}_{N}\bef\gamma_{M}=\,\begin{array}{|c||cc|}
 & A & M^{A}\\
\hline A & \text{id} & \bbnum 0
\end{array}\,\bef\,\begin{array}{|c||c|}
 & M^{A}\\
\hline A & \text{pu}_{M}\\
M^{A} & \text{id}
\end{array}\,=\text{pu}_{M}\quad.
\end{align*}
To verify the composition law, transform separately the two sides
of the law:
\begin{align*}
{\color{greenunder}\text{left-hand side}:}\quad & \gamma_{M}\bef\gamma_{M}^{\uparrow M}\bef\text{ftn}_{M}=\,\begin{array}{|c||c|}
 & M^{N^{A}}\\
\hline N^{A} & \text{pu}_{M}\\
M^{N^{A}} & \text{id}
\end{array}\,\bef\gamma_{M}^{\uparrow M}\bef\text{ftn}_{M}=\,\begin{array}{|c||c|}
 & M^{A}\\
\hline N^{A} & \gunderline{\text{pu}_{M}\bef\gamma_{M}^{\uparrow M}}\bef\text{ftn}_{M}\\
M^{N^{A}} & \gamma_{M}^{\uparrow M}\bef\text{ftn}_{M}
\end{array}\\
 & \quad=\,\begin{array}{|c||c|}
 & M^{A}\\
\hline N^{A} & \gamma_{M}\bef\gunderline{\text{pu}_{M}\bef\text{ftn}_{M}}\\
M^{N^{A}} & \gamma_{M}^{\uparrow M}\bef\text{ftn}_{M}
\end{array}\,=\,\begin{array}{|c||c|}
 & M^{A}\\
\hline A & \text{pu}_{M}\\
M^{A} & \text{id}\\
M^{N^{A}} & \gamma_{M}^{\uparrow M}\bef\text{ftn}_{M}
\end{array}\quad,\\
{\color{greenunder}\text{right-hand side}:}\quad & \text{ftn}_{N}\bef\gamma_{M}=\,\begin{array}{|c||cc|}
 & A & M^{A}\\
\hline A & \text{id} & \bbnum 0\\
M^{A} & \bbnum 0 & \text{id}\\
M^{N^{A}} & \bbnum 0 & \gamma_{M}^{\uparrow M}\bef\text{ftn}_{M}
\end{array}\,\bef\,\begin{array}{|c||c|}
 & M^{A}\\
\hline A & \text{pu}_{M}\\
M^{A} & \text{id}
\end{array}\,=\,\begin{array}{|c||c|}
 & M^{A}\\
\hline A & \text{pu}_{M}\\
M^{A} & \text{id}\\
M^{N^{A}} & \gamma_{M}^{\uparrow M}\bef\text{ftn}_{M}
\end{array}\quad.
\end{align*}
The two sides are now equal.

\subsubsection*{Exercise \ref{par:Exercise-mt-3}}

The operations $\text{ftn}_{L}$ and $\text{ftn}_{N}$ are defined
as usual for a free pointed monad:
\begin{align*}
 & \text{ftn}_{L}\triangleq\,\begin{array}{|c||c|}
 & L^{A}\\
\hline L^{A} & \text{id}\\
K^{L^{A}} & k\rightarrow\bbnum 0+k\triangleright\gamma_{K}^{\uparrow K}\bef\text{ftn}_{K}
\end{array}\,=\,\begin{array}{|c||cc|}
 & A & K^{A}\\
\hline A & \text{id} & \bbnum 0\\
K^{A} & \bbnum 0 & \text{id}\\
K^{L^{A}} & \bbnum 0 & \gamma_{K}^{\uparrow K}\bef\text{ftn}_{K}
\end{array}\quad,\\
 & \text{ftn}_{N}\triangleq\,\begin{array}{|c||c|}
 & N^{A}\\
\hline N^{A} & \text{id}\\
M^{N^{A}} & m\rightarrow\bbnum 0+m\triangleright\gamma_{M}^{\uparrow M}\bef\text{ftn}_{M}
\end{array}\,=\,\begin{array}{|c||cc|}
 & A & M^{A}\\
\hline A & \text{id} & \bbnum 0\\
M^{A} & \bbnum 0 & \text{id}\\
M^{N^{A}} & \bbnum 0 & \gamma_{M}^{\uparrow M}\bef\text{ftn}_{M}
\end{array}\quad.
\end{align*}

We define $\psi$ by:
\[
\psi=\,\begin{array}{|c||cc|}
 & A & M^{A}\\
\hline A & \text{id} & \bbnum 0\\
K^{A} & \bbnum 0 & \phi
\end{array}\quad.
\]

To verify the identity law for $\psi$:
\[
\text{pu}_{L}\bef\psi=\,\begin{array}{|c||cc|}
 & A & K^{A}\\
\hline A & \text{id} & \bbnum 0
\end{array}\,\bef\,\begin{array}{|c||cc|}
 & A & M^{A}\\
\hline A & \text{id} & \bbnum 0\\
K^{A} & \bbnum 0 & \phi
\end{array}\,=\,\begin{array}{|c||cc|}
 & A & M^{A}\\
\hline A & \text{id} & \bbnum 0
\end{array}\,=\text{pu}_{N}\quad.
\]

To verify the composition law for $\psi$, write both sides of the
law separately:
\begin{align*}
\psi^{\uparrow L}\bef\psi\bef\text{ftn}_{N} & =\,\begin{array}{|c||ccc|}
 & A & M^{A} & K^{N^{A}}\\
\hline A & \text{id} & \bbnum 0 & \bbnum 0\\
K^{A} & \bbnum 0 & \phi & \bbnum 0\\
K^{L^{A}} & \bbnum 0 & \bbnum 0 & \psi^{\uparrow K}
\end{array}\,\bef\,\begin{array}{|c||ccc|}
 & A & M^{A} & M^{N^{A}}\\
\hline A & \text{id} & \bbnum 0 & \bbnum 0\\
M^{A} & \bbnum 0 & \text{id} & \bbnum 0\\
K^{N^{A}} & \bbnum 0 & \bbnum 0 & \phi
\end{array}\,\bef\,\begin{array}{|c||cc|}
 & A & M^{A}\\
\hline A & \text{id} & \bbnum 0\\
M^{A} & \bbnum 0 & \text{id}\\
M^{N^{A}} & \bbnum 0 & \gamma_{M}^{\uparrow M}\bef\text{ftn}_{M}
\end{array}\\
 & =\,\,\begin{array}{|c||cc|}
 & A & M^{A}\\
\hline A & \text{id} & \bbnum 0\\
K^{A} & \bbnum 0 & \phi\\
K^{L^{A}} & \bbnum 0 & \psi^{\uparrow K}\bef\phi\bef\gamma_{M}^{\uparrow M}\bef\text{ftn}_{M}
\end{array}\quad,\\
\text{ftn}_{L}\bef\psi & =\,\begin{array}{|c||cc|}
 & A & K^{A}\\
\hline A & \text{id} & \bbnum 0\\
K^{A} & \bbnum 0 & \text{id}\\
K^{L^{A}} & \bbnum 0 & \gamma_{K}^{\uparrow K}\bef\text{ftn}_{K}
\end{array}\,\bef\,\begin{array}{|c||cc|}
 & A & M^{A}\\
\hline A & \text{id} & \bbnum 0\\
K^{A} & \bbnum 0 & \phi
\end{array}\,=\,\begin{array}{|c||cc|}
 & A & M^{A}\\
\hline A & \text{id} & \bbnum 0\\
K^{A} & \bbnum 0 & \phi\\
K^{L^{A}} & \bbnum 0 & \gamma_{K}^{\uparrow K}\bef\text{ftn}_{K}\bef\phi
\end{array}\quad.
\end{align*}
It remains to show that:
\[
\psi^{\uparrow K}\bef\phi\bef\gamma_{M}^{\uparrow M}\bef\text{ftn}_{M}\overset{?}{=}\gamma_{K}^{\uparrow K}\bef\text{ftn}_{K}\bef\phi\quad.
\]
The monad morphism law for $\phi$ gives:
\[
\phi\bef\phi^{\uparrow M}\bef\text{ftn}_{M}=\text{ftn}_{K}\bef\phi\quad.
\]
It remains to show that:
\begin{align*}
 & \gunderline{\psi^{\uparrow K}\bef\phi}\bef\gamma_{M}^{\uparrow M}\overset{?}{=}\gunderline{\gamma_{K}^{\uparrow K}\bef\phi}\bef\phi^{\uparrow M}\quad.\\
 & \phi\bef\psi^{\uparrow M}\bef\gamma_{M}^{\uparrow M}=\phi\bef(\psi\bef\gamma_{M})^{\uparrow M}\overset{?}{=}\phi\bef\gamma_{K}^{\uparrow M}\bef\phi^{\uparrow M}=\phi\bef(\gamma_{K}\bef\phi)^{\uparrow M}\quad.
\end{align*}
It remains to show that $\psi\bef\gamma_{M}=\gamma_{K}\bef\phi$.
For that, we use the identity law of $\phi$:
\begin{align*}
 & \psi\bef\gamma_{M}=\,\begin{array}{|c||cc|}
 & A & M^{A}\\
\hline A & \text{id} & \bbnum 0\\
K^{A} & \bbnum 0 & \phi
\end{array}\,\bef\,\begin{array}{|c||c|}
 & M^{A}\\
\hline A & \text{pu}_{M}\\
M^{A} & \text{id}
\end{array}\,=\,\begin{array}{|c||c|}
 & M^{A}\\
\hline A & \text{pu}_{M}\\
K^{A} & \phi
\end{array}\quad,\\
 & \gamma_{K}\bef\phi=\,\begin{array}{|c||c|}
 & K^{A}\\
\hline A & \text{pu}_{K}\\
K^{A} & \text{id}
\end{array}\,\bef\phi=\,\begin{array}{|c||c|}
 & M^{A}\\
\hline A & \gunderline{\text{pu}_{K}\bef\phi}\\
K^{A} & \phi
\end{array}\,=\,\begin{array}{|c||c|}
 & M^{A}\\
\hline A & \text{pu}_{M}\\
K^{A} & \phi
\end{array}\quad.
\end{align*}


\subsubsection*{Exercise \ref{par:Exercise-mt-3-3}}

\textbf{(a)} Verify the naturality law of $\psi$, assuming the naturality
laws of $\phi$ and $\chi$:
\begin{align*}
 & (f^{:A\rightarrow B})^{\uparrow(K+L)}\bef\psi=\,\begin{array}{|c||cc|}
 & K^{B} & L^{B}\\
\hline K^{A} & f^{\uparrow K} & \bbnum 0\\
L^{A} & \bbnum 0 & f^{\uparrow L}
\end{array}\,\bef\,\begin{array}{|c||c|}
 & M^{B}\\
\hline K^{B} & \phi\\
L^{B} & \chi
\end{array}\,=\,\begin{array}{|c||c|}
 & M^{B}\\
\hline K^{A} & \gunderline{f^{\uparrow K}\bef\phi}\\
L^{A} & \gunderline{f^{\uparrow L}\bef\chi}
\end{array}\,=\,\begin{array}{|c||c|}
 & M^{B}\\
\hline K^{A} & \phi\bef f^{\uparrow M}\\
L^{A} & \chi\bef f^{\uparrow M}
\end{array}\quad,\\
 & \psi\bef f^{\uparrow M}=\,\begin{array}{|c||c|}
 & M^{A}\\
\hline K^{A} & \phi\\
L^{A} & \chi
\end{array}\,\bef f^{\uparrow M}=\,\begin{array}{|c||c|}
 & M^{B}\\
\hline K^{A} & \phi\bef f^{\uparrow M}\\
L^{A} & \chi\bef f^{\uparrow M}
\end{array}\quad.
\end{align*}

\textbf{(b)} Given the definitions of $\phi$ and $\chi$, we can
write $\psi$ as:
\[
\psi=\,\begin{array}{|c||c|}
 & M^{A}\\
\hline K^{A} & \phi\\
L^{A} & \chi
\end{array}\quad.
\]
Assuming the naturality law of $\psi$, verify the naturality law
of $\phi$ by applying to an arbitrary value $k^{:K^{A}}$ and expressing
$f^{\uparrow K}$ through $f^{\uparrow(K+L)}$: 
\begin{align*}
{\color{greenunder}\text{expect to equal }k\triangleright\phi\bef f^{\uparrow M}:}\quad & k\triangleright f^{\uparrow K}\bef\phi=k\triangleright f^{\uparrow K}\triangleright(k\rightarrow k+\bbnum 0)\triangleright\psi=\big((k\triangleright f^{\uparrow K})+\bbnum 0\big)\triangleright\psi\\
 & =(k+\bbnum 0)\triangleright\,\begin{array}{|c||cc|}
 & K^{B} & L^{B}\\
\hline K^{A} & f^{\uparrow K} & \bbnum 0\\
L^{A} & \bbnum 0 & f^{\uparrow L}
\end{array}\,\bef\psi=(k+\bbnum 0)\triangleright\gunderline{f^{\uparrow(K+L)}\bef\psi}\\
{\color{greenunder}\text{naturality law of }\psi:}\quad & =(\gunderline{k+\bbnum 0})\triangleright\psi\bef f^{\uparrow M}=k\triangleright\gunderline{(x\rightarrow x+\bbnum 0)\bef\psi}\bef f^{\uparrow M}=k\triangleright\phi\bef f^{\uparrow M}\quad.
\end{align*}

The proof of the naturality law of $\chi$ is analogous.

\subsubsection*{Exercise \ref{par:Exercise-mt-3-2-1}}

\textbf{(a)} Choose the monads $K^{A}\triangleq R\rightarrow A$ and
$M^{A}\triangleq S\rightarrow A$, where $R$, $S$ are fixed (but
arbitrary) types. Define $P^{A}\triangleq A+T_{K}^{M,A}$ and show
that there exist no monad morphisms $\phi:M^{A}\rightarrow P^{A}$.
If such $\phi$ exists, it must satisfy the identity law, $\text{pu}_{M}\bef\phi=\text{pu}_{P}$.
The type signature of $\phi$ is:
\[
\phi:(R\rightarrow A)\rightarrow A+(S\rightarrow R\rightarrow A)\quad.
\]
The only fully parametric implementation is:
\[
\phi:f^{:R\rightarrow A}\rightarrow\bbnum 0+(\_^{:S}\rightarrow f)\quad,
\]
because we cannot produce values of type $A+\bbnum 0$ since we cannot
create values of the unknown type $R$. However, this implementation
of $\phi$ does not satisfy the identity law since $\text{pu}_{P}$
must return values of type $A+\bbnum 0$.

\textbf{(b)} Given $\theta_{K}:K^{A}\rightarrow M^{A}$, we define:
\[
\theta_{L}:L^{A}\rightarrow M^{A}\quad,\quad\quad\theta_{L}\triangleq\,\begin{array}{|c||c|}
 & M^{A}\\
\hline A & \text{pu}_{M}\\
K^{A} & \theta_{K}
\end{array}\quad.
\]
The proof becomes shorter if we express $\theta_{L}$ as:
\[
\theta_{L}=\gamma_{K}\bef\theta_{K}\quad,\quad\quad\gamma_{K}\triangleq\,\begin{array}{|c||c|}
 & K^{A}\\
\hline A & \text{pu}_{K}\\
K^{A} & \text{id}
\end{array}\quad.
\]
We already know that $\gamma_{K}:L\leadsto K$ is a monad morphism
(Exercise~\ref{par:Exercise-mt-3-1}). So, $\theta_{L}$ is a composition
of monad morphisms.

\textbf{(c)} Choose $K^{A}\triangleq\bbnum 1$ (the unit monad) and
note that $L^{A}=A+\bbnum 1$ is the standard \lstinline!Option!
monad (there are no other lawful monad implementations for the type
constructor $L^{A}\triangleq A+\bbnum 1$). Choose $M\triangleq L$,
so that a monad morphism $\theta_{L}\triangleq\text{id}^{:L\leadsto M}$
exists. If it were possible to define a monad morphism $K\leadsto M$,
we would have a monad morphism $\bbnum 1\rightarrow\bbnum 1+A$, but
this is impossible: the only natural transformation of type $\bbnum 1\rightarrow\bbnum 1+A$
is $1\rightarrow1+\bbnum 0$, which does not satisfy the identity
law of monad morphisms, $\text{pu}_{K}\bef\theta_{K}=\text{pu}_{M}$,
since it never returns any values of type $\bbnum 0+A$. (Generally,
the existence of a monad morphism $\bbnum 1\leadsto M$ means that
$M=\bbnum 1$.)

\textbf{(d)} By Exercise~\ref{par:Exercise-mt-3-3}, we have a natural
transformation $\theta_{K}:K^{A}\rightarrow A$ defined by:
\[
\theta_{K}\triangleq(k\rightarrow\bbnum 0+k)\bef\theta_{L}\quad.
\]
The given runner $\theta_{L}$ is then expressed through $\theta_{K}$
as:
\[
\theta_{L}=\,\begin{array}{|c||c|}
 & A\\
\hline A & \text{id}\\
K^{A} & \theta_{K}
\end{array}\quad,
\]
because the natural transformation in the upper row of the matrix
has the type signature $A\rightarrow A$ and so must be an identity
function.

It remains to verify the monad morphism laws of $\theta_{K}:K\leadsto\text{Id}$.
The identity law is:
\[
\text{pu}_{K}\bef\theta_{K}\overset{?}{=}\text{pu}_{\text{Id}}=\text{id}\quad.
\]
 Since the function $p\triangleq\text{pu}_{K}\bef\theta_{K}$ has
type signature $A\rightarrow A$ and is a natural transformation (as
a composition of two natural transformations), that function must
satisfy $p\bef f=f\bef p$ for any function $f^{:A\rightarrow B}$,
and so $p$ must be equal to the identity function.

The composition law of $\theta_{K}$ is an equation for functions
of type $K^{K^{A}}\rightarrow A$:
\[
\text{ftn}_{K}\bef\theta_{K}\overset{?}{=}\theta_{K}^{\uparrow K}\bef\theta_{K}\bef\text{ftn}_{\text{Id}}=\theta_{K}\bef\theta_{K}\quad.
\]
Applied to an arbitrary value $k:K^{K^{A}}$, this law becomes:
\[
k\triangleright\text{ftn}_{K}\bef\theta_{K}\overset{?}{=}k\triangleright\theta_{K}^{\uparrow K}\bef\theta_{K}\quad.
\]
Since $\theta_{K}$ is defined via $\theta_{L}$, we need to use the
composition law of $\theta_{L}$:
\[
\text{ftn}_{L}\bef\theta_{L}\overset{!}{=}\theta_{L}^{\uparrow L}\bef\theta_{L}\bef\text{ftn}_{\text{Id}}=\theta_{L}^{\uparrow K}\bef\theta_{L}\quad.
\]
Rewrite this law as:
\begin{align*}
 & \text{ftn}_{L}\bef\theta_{L}=\,\begin{array}{|c||cc|}
 & A & K^{A}\\
\hline A & \text{id} & \bbnum 0\\
K^{A} & \bbnum 0 & \text{id}\\
K^{L^{A}} & \bbnum 0 & \gamma_{K}^{\uparrow K}\bef\text{ftn}_{K}
\end{array}\,\bef\,\begin{array}{|c||c|}
 & A\\
\hline A & \text{id}\\
K^{A} & \theta_{K}
\end{array}\,=\,\begin{array}{|c||c|}
 & A\\
\hline A & \text{id}\\
K^{A} & \theta_{K}\\
K^{L^{A}} & \gamma_{K}^{\uparrow K}\bef\text{ftn}_{K}\bef\theta_{K}
\end{array}\quad,\\
 & \theta_{L}^{\uparrow L}\bef\theta_{L}=\,\begin{array}{|c||cc|}
 & A & K^{A}\\
\hline A & \text{id} & \bbnum 0\\
K^{A} & \bbnum 0 & \text{id}\\
K^{L^{A}} & \bbnum 0 & \theta_{L}^{\uparrow K}
\end{array}\,\bef\,\begin{array}{|c||c|}
 & A\\
\hline A & \text{id}\\
K^{A} & \theta_{K}
\end{array}\,=\,\begin{array}{|c||c|}
 & A\\
\hline A & \text{id}\\
K^{A} & \theta_{K}\\
K^{L^{A}} & \theta_{L}^{\uparrow K}\bef\theta_{K}
\end{array}\quad.
\end{align*}
The third rows of the matrices give the equation for functions of
type $K^{L^{A}}\rightarrow A$:
\[
\gamma_{K}^{\uparrow K}\bef\text{ftn}_{K}\bef\theta_{K}\overset{!}{=}\theta_{L}^{\uparrow K}\bef\theta_{K}\quad.
\]
Apply both sides of this equation to an arbitrary value $p$ of type
$K^{\bbnum 0+K^{A}}$, defined via an arbitrary value $k^{:K^{K^{A}}}$
as $p\triangleq k\triangleright(x\rightarrow\bbnum 0+x)^{\uparrow K}$:
\begin{align*}
 & p\triangleright\gamma_{K}^{\uparrow K}\bef\text{ftn}_{K}\bef\theta_{K}\overset{!}{=}p\triangleright\theta_{L}^{\uparrow K}\bef\theta_{K}\quad,\\
{\color{greenunder}\text{or equivalently}:}\quad & k\triangleright(x\rightarrow\bbnum 0+x)^{\uparrow K}\bef\gamma_{K}^{\uparrow K}\bef\text{ftn}_{K}\bef\theta_{K}\overset{!}{=}k\triangleright(x\rightarrow\bbnum 0+x)^{\uparrow K}\bef\theta_{L}^{\uparrow K}\bef\theta_{K}\quad.
\end{align*}
We compute some sub-expressions separately:
\begin{align*}
 & (x\rightarrow\bbnum 0+x)\bef\gamma_{K}=(x\rightarrow\bbnum 0+x)\bef\,\begin{array}{|c||c|}
 & K^{A}\\
\hline A & \text{pu}_{K}\\
K^{A} & \text{id}
\end{array}\,=x\rightarrow x=\text{id}\quad,\\
 & (x\rightarrow\bbnum 0+x)\bef\theta_{L}=\theta_{K}\quad.
\end{align*}
After these simplifications, the composition law of $\theta_{L}$
gives:
\[
k\triangleright\text{ftn}_{K}\bef\theta_{K}\overset{!}{=}k\triangleright\theta_{K}^{\uparrow K}\bef\theta_{K}\quad.
\]
This is exactly the composition law of $\theta_{K}$ that we still
needed to prove.

\subsubsection*{Exercise \ref{par:Exercise-mt-3-2}}

Consider the monad morphism $\varepsilon:K\leadsto\bbnum 1$ defined
by $\varepsilon\triangleq\_\rightarrow1$. (This is a monad morphism
because all its laws reduce to the equation $1=1$.) By Exercise~\ref{par:Exercise-mt-3},
we can implement a corresponding monad morphism between free pointed
monads $\text{Id}+K\leadsto\text{Id}+\bbnum 1$. Define $\delta$
as that monad morphism. The monad $\text{Id}+\bbnum 1$ (the free
pointed monad on $\bbnum 1$) is the standard \lstinline!Option!
monad. So, we have implemented a monad morphism $\delta:A+K^{A}\rightarrow\text{Opt}^{A}$.

\subsubsection*{Exercise \ref{subsec:Exercise-effectful-list-not-monad}}

\textbf{(a)} Consider the requirement of mapping $L^{L^{A}}\rightarrow L^{A}$
via a \lstinline!flatten! function:
\[
\text{ftn}_{L}:\bbnum 1+L^{A}\times M^{L^{L^{A}}}\rightarrow\bbnum 1+A\times M^{L^{A}}\quad,\quad\quad\text{ftn}_{L}=\text{???}
\]
The result value of this function cannot always be $1$, or else it
will fail the monad laws. This function must sometimes return a pair
of type $A\times M^{L^{A}}$. Let us rewrite the type signature of
\lstinline!flatten! as
\[
\text{ftn}_{L}:\bbnum 1+(\bbnum 1+A\times M^{L^{A}})\times M^{L^{L^{A}}}\rightarrow\bbnum 1+A\times M^{L^{A}}\quad.
\]
Consider input values of the form:
\[
x\triangleq\bbnum 0^{:\bbnum 1}+(1+\bbnum 0^{:A\times M^{L^{A}}})\times m^{:M^{L^{L^{A}}}}\quad.
\]
the result of evaluating $\text{ftn}_{L}(x)$ must be $1+\bbnum 0$:
a fully parametric function cannot extract values of type $A$ from
a value of type $M^{A}$ with an unknown monad $M$. Note that the
value $(1+\bbnum 0^{:A\times M^{L^{A}}})$ represents an empty list;
let us denote that value by $\text{Nil}^{:L^{A}}$. So, we must have:
\[
\text{ftn}_{L}(\bbnum 0+\text{Nil}\times(...))=\text{Nil}\quad.
\]
Since the only way to extract list values is to run the $M$-effects,
the implementation of $\text{ftn}_{L}$ must proceed recursively,
extracting list elements left to right. Now consider $M=\text{Id}$
(so we can simply omit the functor layers of $M$) and the nested
list value:
\[
z\triangleq\left[p,q,\text{Nil},r,s,...\right]\quad,
\]
 where $p$, $q$, $r$, $s$ are some sub-lists of type $L^{A}$
with some type $A$. Then the value $z$ has the form:
\[
z=\bbnum 0+p\times(\bbnum 0+q\times(\bbnum 0+\text{Nil}\times(\bbnum 0+r\times(\bbnum 0+s\times(1+\bbnum 0)))))\quad.
\]
As we must have $\text{ftn}_{L}(\bbnum 0+\text{Nil}\times t)=\text{Nil}$
for any $t^{:M^{L^{L^{A}}}}$, the recursive evaluation of $\text{ftn}_{L}(z)$
will give:
\[
\text{ftn}_{L}(z)=...\text{ftn}_{L}(\bbnum 0+\text{Nil}\times(\bbnum 0+r\times(\bbnum 0+s\times(1+\bbnum 0))))=...\text{Nil}\quad.
\]
So, the result may be some function of $p$ and $q$ (which could
be $p\pplus q$) but cannot depend on $r$ and $s$. We have shown
that $\text{ftn}_{L}$ must ignore all sub-lists that follow an empty
sub-list. A possible behavior of $\text{ftn}_{L}$ is:
\[
\text{ftn}_{L}(\left[\left[1,2\right],\left[3,4\right],\left[\right],\left[5,6,7,8\right]\right])=\left[1,2,3,4\right]\quad,
\]
instead of the full flattened list $\left[1,2,3,4,5,6,7,8\right]$.
Regardless of how we implement $\text{ftn}_{L}$ (and whether it satisfies
the monad laws), the result of evaluating $\text{ftn}_{L}(z)$ cannot
be the full concatenation $p\pplus q\pplus r\pplus s$ because the
computation must ignore the sub-lists $r$ and $s$.

\textbf{(b)} The code for the non-standard \lstinline!flatten! function
is:
\begin{lstlisting}[mathescape=true]
def flatten[A](p: List[List[A]]): List[A] = p.takeWhile(_.nonEmpty).flatten  // $\color{dkgreen}\textrm{ftn}_L$
\end{lstlisting}
The new \lstinline!flatten! function gives  the same results as \lstinline!List!\textsf{'}s
standard \lstinline!flatten! method, except if one of the nested
sub-lists is empty. Then the \lstinline!flatten! function truncates
the result after the first empty sub-list.

It turns out that this code fails the monad\textsf{'}s associativity law of
\lstinline!flatten!. That law is an equality of functions $\text{ftn}_{L}\bef\text{ftn}_{L}$
and $\text{ftn}_{L}^{\uparrow L}\bef\text{ftn}_{L}$ of type \lstinline!List[List[List[A]]] => List[A]!.
The failure is found when a value $p$ of type \lstinline!List[List[List[A]]]!
contains a nested empty list at the \emph{second} nesting depth, following
some non-empty lists. Here is an example that triggers the failure
of the law:
\begin{lstlisting}
val p: List[List[List[Int]]] = List(List(List(1, 2, 3)), List(List(4), Nil), List(List(5, 6)))
\end{lstlisting}
Here the nested list \lstinline!List(List(4), Nil)! contains an empty
list (\lstinline!Nil!) after a non-empty list. Applying both sides
of the law to \lstinline!p!, we find that the two sides of the law
are not equal:
\begin{lstlisting}
scala> flatten(flatten(p))
res0: List[List[Int]] = List(1, 2, 3, 4)

scala> flatten(p.map(flatten))
res1: List[List[Int]] = List(1, 2, 3, 4, 5, 6)
\end{lstlisting}

\textbf{(c)} Try implementing the method $\text{flift}:M^{A}\rightarrow L^{A}$.
The function \lstinline!flift! must produce a value of type $L^{A}\cong\bbnum 1+A\times M^{L^{A}}$.
Since $M$ is an arbitrary monad, we cannot extract a value of type
$A$ out of $M^{A}$ while keeping the code fully parametric. So,
we can implement \lstinline!flift! only by defining $\text{flift}\triangleq\_^{:M^{A}}\rightarrow\text{Nil}^{:L^{A}}$.
However, that implementation loses information and fails the identity
law:
\[
\text{pu}_{M}\bef\text{flift}=(\_\rightarrow\text{Nil})\neq\text{pu}_{L}\quad.
\]


\subsubsection*{Exercise \ref{subsec:Exercise-combined-codensity-monad}}

\textbf{(a)} Denote for brevity $\text{Cod}_{F}^{M,A}\triangleq C^{A}$.
The naturality law for functions $c^{:C^{A}}$ says that for any $k^{:A\rightarrow F^{X}}$
and $q^{:X\rightarrow Y}$, we have:
\[
(k^{:A\rightarrow F^{X}}\bef q^{\uparrow F})\triangleright c^{Y}=k\triangleright c^{X}\bef q^{\uparrow M\uparrow F}\quad.
\]

The flipped Kleisli method $\tilde{\text{pu}}_{C}$ is defined by:
\[
\tilde{\text{pu}}_{C}:\forall X.\,(A\rightarrow F^{X})\rightarrow A\rightarrow F^{M^{X}}\quad,\quad\quad\tilde{\text{pu}}_{C}\triangleq\forall X.\,k^{:A\rightarrow F^{X}}\rightarrow k\bef\text{pu}_{M}^{\uparrow F}\quad.
\]

To verify the left identity law:
\begin{align*}
{\color{greenunder}\text{expect to equal }g:}\quad & \tilde{\text{pu}}_{C}\tilde{\diamond}\,g^{:\forall Y.\,(B\rightarrow F^{Y})\rightarrow A\rightarrow F^{M^{Y}}}=\forall Z.\,k^{:B\rightarrow F^{Z}}\rightarrow\big(\gunderline{k\triangleright g^{Z}\triangleright\tilde{\text{pu}}_{C}^{M^{Z}}}\big)\bef\text{ftn}_{M}^{\uparrow F}\\
 & =\forall Z.\,k^{:C\rightarrow F^{Z}}\rightarrow\big(g^{Z}(k)\bef\gunderline{\text{pu}_{M}^{\uparrow F}\big)\bef\text{ftn}_{M}^{\uparrow F}}\\
{\color{greenunder}\text{left identity law of }M:}\quad & =\forall Z.\,k^{:C\rightarrow F^{Z}}\rightarrow g^{Z}(k)=g\quad.
\end{align*}

To verify the right identity law:
\begin{align*}
{\color{greenunder}\text{expect to equal }f:}\quad & f^{:\forall X.\,(B\rightarrow F^{X})\rightarrow A\rightarrow F^{M^{X}}}\tilde{\diamond}\,\tilde{\text{pu}}_{C}=\forall Z.\,k^{:B\rightarrow F^{Z}}\rightarrow\big(\gunderline{k\triangleright\tilde{\text{pu}}_{C}^{Z}}\triangleright f^{M^{Z}}\big)\bef\text{ftn}_{M}^{\uparrow F}\\
 & =\forall Z.\,k^{:B\rightarrow F^{Z}}\rightarrow\big(\gunderline{(k\bef\text{pu}_{M}^{\uparrow F})\triangleright f^{M^{Z}}}\big)\bef\text{ftn}_{M}^{\uparrow F}\\
{\color{greenunder}\text{naturality law of }f:}\quad & =\forall Z.\,k^{:B\rightarrow F^{Z}}\rightarrow\big(k\triangleright f^{Z}\bef\gunderline{\text{pu}_{M}^{\uparrow M\uparrow F}\big)\bef\text{ftn}_{M}^{\uparrow F}}\\
{\color{greenunder}\text{right identity law of }M:}\quad & =\forall Z.\,k^{:B\rightarrow F^{Z}}\rightarrow k\triangleright f^{Z}=f\quad.
\end{align*}

To verify the associativity law, write its two sides separately; omit
all types for brevity:
\begin{align*}
 & (f\,\tilde{\diamond}\,g)\,\tilde{\diamond}\,h=l\rightarrow\big(l\triangleright h\triangleright(\gunderline{f\,\tilde{\diamond}\,g})\big)\bef\text{ftn}_{M}^{\uparrow F}=l\rightarrow\big(\gunderline{l\triangleright h\triangleright(k}\rightarrow(k\triangleright g\triangleright f)\bef\text{ftn}_{M}^{\uparrow F})\big)\bef\text{ftn}_{M}^{\uparrow F}\\
 & \quad=l\rightarrow\big((l\triangleright h\triangleright g\triangleright f)\bef\text{ftn}_{M}^{\uparrow F}\big)\bef\text{ftn}_{M}^{\uparrow F}=l\rightarrow l\triangleright h\bef g\bef f\bef\text{ftn}_{M}^{\uparrow F}\bef\text{ftn}_{M}^{\uparrow F}\quad,\\
 & f\,\tilde{\diamond}\,(g\,\tilde{\diamond}\,h)=l\rightarrow\big(l\triangleright(g\,\tilde{\diamond}\,h)\triangleright f\big)\bef\text{ftn}_{M}^{\uparrow F}=l\rightarrow\big(l\triangleright(k\rightarrow(k\triangleright h\triangleright g)\bef\text{ftn}_{M}^{\uparrow F})\triangleright f\big)\bef\text{ftn}_{M}^{\uparrow F}\\
 & \quad=l\rightarrow\big(\big((l\triangleright h\triangleright g)\bef\gunderline{\text{ftn}_{M}^{\uparrow F}\big)\triangleright f}\big)\bef\text{ftn}_{M}^{\uparrow F}=l\rightarrow l\triangleright h\triangleright g\triangleright f\bef\text{ftn}_{M}^{\uparrow M\uparrow F}\bef\text{ftn}_{M}^{\uparrow F}\quad.
\end{align*}
In the last line, we have used the naturality law of $f$. The remaining
difference between the two sides is:
\[
\text{ftn}_{M}^{\uparrow F}\bef\text{ftn}_{M}^{\uparrow F}\overset{?}{=}\text{ftn}_{M}^{\uparrow M\uparrow F}\bef\text{ftn}_{M}^{\uparrow F}\quad,
\]
which follows from the associativity law of $\text{ftn}_{M}$.

\textbf{(b)} The \lstinline!flatMap! method must have the type signature:
\[
\text{flm}_{L}:\big((A\rightarrow X)\rightarrow M^{X}\big)\rightarrow(A\rightarrow(B\rightarrow X)\rightarrow M^{X})\rightarrow(B\rightarrow X)\rightarrow M^{X}\quad.
\]
Choose $M^{A}\triangleq\bbnum 1+A$; now we need to implement the
type signature:
\begin{align*}
 & \text{flm}_{L}:\big((A\rightarrow X)\rightarrow\bbnum 1+X\big)\rightarrow\left(A\rightarrow(B\rightarrow X)\rightarrow\bbnum 1+X\right)\rightarrow(B\rightarrow X)\rightarrow\bbnum 1+X\quad,\\
 & \text{flm}_{L}\triangleq p^{:(A\rightarrow X)\rightarrow\bbnum 1+X}\rightarrow q^{:\left(A\rightarrow(B\rightarrow X)\rightarrow\bbnum 1+X\right)}\rightarrow r^{:B\rightarrow X}\rightarrow\text{???}^{:\bbnum 1+X}\quad.
\end{align*}
Can this function ever return a value of type $\bbnum 0+X$? When
we try filling out the typed hole $\text{???}^{:\bbnum 1+X}$, we
cannot apply the function $r$ since we have no available values of
type $B$. We could substitute $r$ into the second curried argument
of $q$, obtaining a function of type $A\rightarrow\bbnum 1+X$. But
we have no available values of type $A$. We also cannot apply the
function $p$ since its argument is of type $A\rightarrow X$, but
we only have $A\rightarrow\bbnum 1+X$, which is not guaranteed to
return nonempty values. So, the only way of implementing \lstinline!flatMap!
via fully parametric code is to return the constant value $1+\bbnum 0^{:X}$.
This would lose information and violate an identity law of monads.

\chapter{A humorous disclaimer}

\index{jokes}\emph{The following text is quoted in part from an anonymous
online source (\textsf{``}Project Guten Tag\textsf{''}) dating back at least to 1997.
The original text is no longer available on the Internet.}

\medskip{}

\noun{Warranto Limitensis; Disclamatantus Damagensis}

Solus exceptus \textsf{``}Rectum Replacator Refundiens\textsf{''} describitus ecci,
\begin{enumerate}
\item Projectus (etque nunquam partum quis hic etext remitibus cum \noun{Project
Guten Tag}$^{\text{TM}}$ identificator) disclamabat omni liabilitus
tuus damagensis, pecuniensisque, includibantus pecunia legalitus,
et 
\item \noun{Remedia Negligentitia Non Habet Tuus, Warrantus Destructi\-bus
Contractus Nullibus Ni Liabilitus Sumus, Inclutatibus Non Limitatus
Destructio Directibus, Consequentius, Punitio, O Incidentus, Non Sunt
Si Nos Notificat Vobis}. 
\end{enumerate}
Sit discubriatus defectus en etextum sic entram diariam noventam recibidio,
pecuniam tuum refundatorium receptorus posset, sic scribatis vendor.
Sit veniabat medium physicalis, vobis idem reternat et replacator
possit copius. Sit venitabat electronicabilis, sic viri datus chansus
segundibus. 

\noun{Hic Etext Venid \textsf{``}Como-asi\textsf{''}. Nihil Warranti Nunquam Classum,
Expressito Ni Implicato, Le Macchen Como Si Etexto Bene Sit O Il Medio
Bene Sit, Inclutat Et Non Limitat Warranti Mercatensis, Appropriatensis
Purposem. }

Statuen varias non permitatent disclamabaris ni warranti implicatoren
ni exclusioni limitatio damagaren consequentialis, ecco lo qua disclamatori
exclusato\-rique non vobis applicant, et potat optia alia legali.

\twocolumn

\chapter{GNU Free Documentation License\label{sec:GFDL} }

{\footnotesize{}Version 1.2, November 2002}{\footnotesize\par}

{\tiny{}Copyright (c) 2000,2001,2002 Free Software Foundation, Inc.
59 Temple Place, Suite 330, Boston, MA 02111-1307, USA}{\tiny\par}

{\tiny{}Everyone is permitted to copy and distribute verbatim copies
of this license document, but changing it is not allowed.}{\tiny\par}

{\tiny{}\setcounter{subsection}{-1}}{\tiny\par}

\subsection*{{\tiny{}Preamble}}

{\tiny{}The purpose of this License is to make a manual, textbook,
or other functional and useful document free in the sense of freedom:
to assure everyone the effective freedom to copy and redistribute
it, with or without modifying it, either commercially or noncommercially.
Secondarily, this License preserves for the author and publisher a
way to get credit for their work, while not being considered responsible
for modifications made by others.}{\tiny\par}

{\tiny{}This License is a kind of \textquotedblleft copyleft\textsf{''}, which
means that derivative works of the document must themselves be free
in the same sense. It complements the GNU General Public License,
which is a copyleft license designed for free software.}{\tiny\par}

{\tiny{}We have designed this License in order to use it for manuals
for free software, because free software needs free documentation:
a free program should come with manuals providing the same freedoms
that the software does. But this License is not limited to software
manuals; it can be used for any textual work, regardless of subject
matter or whether it is published as a printed book. We recommend
this License principally for works whose purpose is instruction or
reference.}{\tiny\par}

\subsection{Applicability and definitions\label{subsec:1Applicability-and-definitions}}

{\tiny{}This License applies to any manual or other work, in any medium,
that contains a notice placed by the copyright holder saying it can
be distributed under the terms of this License. Such a notice grants
a world-wide, royalty-free license, unlimited in duration, to use
that work under the conditions stated herein. The \textquotedblleft Document\textsf{''},
below, refers to any such manual or work. Any member of the public
is a licensee, and is addressed as \textquotedblleft you\textsf{''}. You accept
the license if you copy, modify or distribute the work in a way requiring
permission under copyright law.}{\tiny\par}

{\tiny{}A \textquotedblleft Modified Version\textsf{''} of the Document means
any work containing the Document or a portion of it, either copied
verbatim, or with modifications and/or translated into another language.}{\tiny\par}

{\tiny{}A \textquotedblleft Secondary Section\textsf{''} is a named appendix
or a front-matter section of the Document that deals exclusively with
the relationship of the publishers or authors of the Document to the
Document\textsf{'}s overall subject (or to related matters) and contains nothing
that could fall directly within that overall subject. (Thus, if the
Document is in part a textbook of mathematics, a Secondary Section
may not explain any mathematics.) The relationship could be a matter
of historical connection with the subject or with related matters,
or of legal, commercial, philosophical, ethical or political position
regarding them.}{\tiny\par}

{\tiny{}The \textquotedblleft Invariant Sections\textsf{''} are certain Secondary
Sections whose titles are designated, as being those of Invariant
Sections, in the notice that says that the Document is released under
this License. If a section does not fit the above definition of Secondary
then it is not allowed to be designated as Invariant. The Document
may contain zero Invariant Sections. If the Document does not identify
any Invariant Sections then there are none.}{\tiny\par}

{\tiny{}The \textquotedblleft Cover Texts\textsf{''} are certain short passages
of text that are listed, as Front-Cover Texts or Back-Cover Texts,
in the notice that says that the Document is released under this License.
A Front-Cover Text may be at most 5 words, and a Back-Cover Text may
be at most 25 words.}{\tiny\par}

{\tiny{}A \textquotedblleft Transparent\textsf{''} copy of the Document means
a machine-readable copy, represented in a format whose specification
is available to the general public, that is suitable for revising
the document straightforwardly with generic text editors or (for images
composed of pixels) generic paint programs or (for drawings) some
widely available drawing editor, and that is suitable for input to
text formatters or for automatic translation to a variety of formats
suitable for input to text formatters. A copy made in an otherwise
Transparent file format whose markup, or absence of markup, has been
arranged to thwart or discourage subsequent modification by readers
is not Transparent. An image format is not Transparent if used for
any substantial amount of text. A copy that is not \textquotedblleft Transparent\textsf{''}
is called \textquotedblleft Opaque\textsf{''}.}{\tiny\par}

{\tiny{}Examples of suitable formats for Transparent copies include
plain ASCII without markup, Texinfo input format, \LaTeX{} input format,
SGML or XML using a publicly available DTD, and standard-conforming
simple HTML, PostScript or PDF designed for human modification. Examples
of transparent image formats include PNG, XCF and JPG. Opaque formats
include proprietary formats that can be read and edited only by proprietary
word processors, SGML or XML for which the DTD and/or processing tools
are not generally available, and the machine-generated HTML, PostScript
or PDF produced by some word processors for output purposes only.}{\tiny\par}

{\tiny{}The \textsf{``}Title Page\textsf{''} means, for a printed book, the title
page itself, plus such following pages as are needed to hold, legibly,
the material this License requires to appear in the title page. For
works in formats which do not have any title page as such, \textquotedblleft Title
Page\textquotedblright{} means the text near the most prominent appearance
of the work\textsf{'}s title, preceding the beginning of the body of the text.}{\tiny\par}

{\tiny{}A section \textsf{``}Entitled XYZ\textsf{''} means a named subunit of the
Document whose title either is precisely XYZ or contains XYZ in parentheses
following text that translates XYZ in another language. (Here XYZ
stands for a specific section name mentioned below, such as \textquotedblleft Acknowledgements\textquotedblright ,
\textquotedblleft Dedications\textquotedblright , \textquotedblleft Endorsements\textquotedblright ,
or \textquotedblleft History\textquotedblright .) To \textquotedblleft Preserve
the Title\textquotedblright{} of such a section when you modify the
Document means that it remains a section \textquotedblleft Entitled
XYZ\textquotedblright{} according to this definition.}{\tiny\par}

{\tiny{}The Document may include Warranty Disclaimers next to the
notice which states that this License applies to the Document. These
Warranty Disclaimers are considered to be included by reference in
this License, but only as regards disclaiming warranties: any other
implication that these Warranty Disclaimers may have is void and has
no effect on the meaning of this License.}{\tiny\par}

\subsection{Verbatim copying\label{subsec:2Verbatim-copying}}

{\tiny{}You may copy and distribute the Document in any medium, either
commercially or noncommercially, provided that this License, the copyright
notices, and the license notice saying this License applies to the
Document are reproduced in all copies, and that you add no other conditions
whatsoever to those of this License. You may not use technical measures
to obstruct or control the reading or further copying of the copies
you make or distribute. However, you may accept compensation in exchange
for copies. If you distribute a large enough number of copies you
must also follow the conditions in section~\ref{subsec:3Copying-in-quantity}.}{\tiny\par}

{\tiny{}You may also lend copies, under the same conditions stated
above, and you may publicly display copies.}{\tiny\par}

\subsection{Copying in quantity\label{subsec:3Copying-in-quantity}}

{\tiny{}If you publish printed copies (or copies in media that commonly
have printed covers) of the Document, numbering more than 100, and
the Document\textsf{'}s license notice requires Cover Texts, you must enclose
the copies in covers that carry, clearly and legibly, all these Cover
Texts: Front-Cover Texts on the front cover, and Back-Cover Texts
on the back cover. Both covers must also clearly and legibly identify
you as the publisher of these copies. The front cover must present
the full title with all words of the title equally prominent and visible.
You may add other material on the covers in addition. Copying with
changes limited to the covers, as long as they preserve the title
of the Document and satisfy these conditions, can be treated as verbatim
copying in other respects.}{\tiny\par}

{\tiny{}If the required texts for either cover are too voluminous
to fit legibly, you should put the first ones listed (as many as fit
reasonably) on the actual cover, and continue the rest onto adjacent
pages.}{\tiny\par}

{\tiny{}If you publish or distribute Opaque copies of the Document
numbering more than 100, you must either include a machine-readable
Transparent copy along with each Opaque copy, or state in or with
each Opaque copy a computer-network location from which the general
network-using public has access to download using public-standard
network protocols a complete Transparent copy of the Document, free
of added material. If you use the latter option, you must take reasonably
prudent steps, when you begin distribution of Opaque copies in quantity,
to ensure that this Transparent copy will remain thus accessible at
the stated location until at least one year after the last time you
distribute an Opaque copy (directly or through your agents or retailers)
of that edition to the public.}{\tiny\par}

{\tiny{}It is requested, but not required, that you contact the authors
of the Document well before redistributing any large number of copies,
to give them a chance to provide you with an updated version of the
Document.}{\tiny\par}

\subsection{Modifications\label{subsec:4Modifications}}

{\tiny{}You may copy and distribute a Modified Version of the Document
under the conditions of sections~\ref{subsec:2Verbatim-copying}
and \ref{subsec:3Copying-in-quantity} above, provided that you release
the Modified Version under precisely this License, with the Modified
Version filling the role of the Document, thus licensing distribution
and modification of the Modified Version to whoever possesses a copy
of it. In addition, you must do these things in the Modified Version:}{\tiny\par}

{\tiny{}A. Use in the Title Page (and on the covers, if any) a title
distinct from that of the Document, and from those of previous versions
(which should, if there were any, be listed in the History section
of the Document). You may use the same title as a previous version
if the original publisher of that version gives permission.}{\tiny\par}

{\tiny{}B. List on the Title Page, as authors, one or more persons
or entities responsible for authorship of the modifications in the
Modified Version, together with at least five of the principal authors
of the Document (all of its principal authors, if it has fewer than
five), unless they release you from this requirement.}{\tiny\par}

{\tiny{}C. State on the Title page the name of the publisher of the
Modified Version, as the publisher.}{\tiny\par}

{\tiny{}D. Preserve all the copyright notices of the Document.}{\tiny\par}

{\tiny{}E. Add an appropriate copyright notice for your modifications
adjacent to the other copyright notices.}{\tiny\par}

{\tiny{}F. Include, immediately after the copyright notices, a license
notice giving the public permission to use the Modified Version under
the terms of this License, in the form shown in the Addendum below.}{\tiny\par}

{\tiny{}G. Preserve in that license notice the full lists of Invariant
Sections and required Cover Texts given in the Document\textsf{'}s license
notice.}{\tiny\par}

{\tiny{}H. Include an unaltered copy of this License.}{\tiny\par}

{\tiny{}I. Preserve the section Entitled \textsf{``}History\textsf{''}, Preserve its
Title, and add to it an item stating at least the title, year, new
authors, and publisher of the Modified Version as given on the Title
Page. If there is no section Entitled \textquotedblleft History\textquotedblright{}
in the Document, create one stating the title, year, authors, and
publisher of the Document as given on its Title Page, then add an
item describing the Modified Version as stated in the previous sentence.}{\tiny\par}

{\tiny{}J. Preserve the network location, if any, given in the Document
for public access to a Transparent copy of the Document, and likewise
the network locations given in the Document for previous versions
it was based on. These may be placed in the \textsf{``}History\textsf{''} section.
You may omit a network location for a work that was published at least
four years before the Document itself, or if the original publisher
of the version it refers to gives permission.}{\tiny\par}

{\tiny{}K. For any section Entitled \textsf{``}Acknowledgements\textsf{''} or \textsf{``}Dedications\textsf{''},
Preserve the Title of the section, and preserve in the section all
the substance and tone of each of the contributor acknowledgements
and/or dedications given therein.}{\tiny\par}

{\tiny{}L. Preserve all the Invariant Sections of the Document, unaltered
in their text and in their titles. Section numbers or the equivalent
are not considered part of the section titles.}{\tiny\par}

{\tiny{}M. Delete any section Entitled \textsf{``}Endorsements\textsf{''}. Such a
section may not be included in the Modified Version.}{\tiny\par}

{\tiny{}N. Do not retitle any existing section to be Entitled \textsf{``}Endorsements\textsf{''}
or to conflict in title with any Invariant Section.}{\tiny\par}

{\tiny{}O. Preserve any Warranty Disclaimers.}{\tiny\par}

{\tiny{}If the Modified Version includes new front-matter sections
or appendices that qualify as Secondary Sections and contain no material
copied from the Document, you may at your option designate some or
all of these sections as invariant. To do this, add their titles to
the list of Invariant Sections in the Modified Version\textsf{'}s license notice.
These titles must be distinct from any other section titles.}{\tiny\par}

{\tiny{}You may add a section Entitled \textsf{``}Endorsements\textsf{''}, provided
it contains nothing but endorsements of your Modified Version by various
parties \textemdash{} for example, statements of peer review or that
the text has been approved by an organization as the authoritative
definition of a standard.}{\tiny\par}

{\tiny{}You may add a passage of up to five words as a Front-Cover
Text, and a passage of up to 25 words as a Back-Cover Text, to the
end of the list of Cover Texts in the Modified Version. Only one passage
of Front-Cover Text and one of Back-Cover Text may be added by (or
through arrangements made by) any one entity. If the Document already
includes a cover text for the same cover, previously added by you
or by arrangement made by the same entity you are acting on behalf
of, you may not add another; but you may replace the old one, on explicit
permission from the previous publisher that added the old one.}{\tiny\par}

{\tiny{}The author(s) and publisher(s) of the Document do not by this
License give permission to use their names for publicity for or to
assert or imply endorsement of any Modified Version.}{\tiny\par}

\subsection*{{\tiny{}Combining documents}}

{\tiny{}You may combine the Document with other documents released
under this License, under the terms defined in section 4 above for
modified versions, provided that you include in the combination all
of the Invariant Sections of all of the original documents, unmodified,
and list them all as Invariant Sections of your combined work in its
license notice, and that you preserve all their Warranty Disclaimers.}{\tiny\par}

{\tiny{}The combined work need only contain one copy of this License,
and multiple identical Invariant Sections may be replaced with a single
copy. If there are multiple Invariant Sections with the same name
but different contents, make the title of each such section unique
by adding at the end of it, in parentheses, the name of the original
author or publisher of that section if known, or else a unique number.
Make the same adjustment to the section titles in the list of Invariant
Sections in the license notice of the combined work.}{\tiny\par}

{\tiny{}In the combination, you must combine any sections Entitled
\textquotedblleft History\textquotedblright{} in the various original
documents, forming one section Entitled \textquotedblleft History\textquotedblright ;
likewise combine any sections Entitled \textquotedblleft Acknowledgements\textquotedblright ,
and any sections Entitled \textquotedblleft Dedications\textquotedblright .
You must delete all sections Entitled \textquotedblleft Endorsements.\textquotedblright{}}{\tiny\par}

\subsection*{{\tiny{}Collections of documents}}

{\tiny{}You may make a collection consisting of the Document and other
documents released under this License, and replace the individual
copies of this License in the various documents with a single copy
that is included in the collection, provided that you follow the rules
of this License for verbatim copying of each of the documents in all
other respects.}{\tiny\par}

{\tiny{}You may extract a single document from such a collection,
and distribute it individually under this License, provided you insert
a copy of this License into the extracted document, and follow this
License in all other respects regarding verbatim copying of that document.}{\tiny\par}

\subsection*{{\tiny{}Aggregation with independent works}}

{\tiny{}A compilation of the Document or its derivatives with other
separate and independent documents or works, in or on a volume of
a storage or distribution medium, is called an \textquotedblleft aggregate\textquotedblright{}
if the copyright resulting from the compilation is not used to limit
the legal rights of the compilation\textsf{'}s users beyond what the individual
works permit. When the Document is included an aggregate, this License
does not apply to the other works in the aggregate which are not themselves
derivative works of the Document.}{\tiny\par}

{\tiny{}If the Cover Text requirement of section~\ref{subsec:3Copying-in-quantity}
is applicable to these copies of the Document, then if the Document
is less than one half of the entire aggregate, the Document\textsf{'}s Cover
Texts may be placed on covers that bracket the Document within the
aggregate, or the electronic equivalent of covers if the Document
is in electronic form. Otherwise they must appear on printed covers
that bracket the whole aggregate.}{\tiny\par}

\subsection*{{\tiny{}Translation}}

{\tiny{}Translation is considered a kind of modification, so you may
distribute translations of the Document under the terms of section~\ref{subsec:4Modifications}.
Replacing Invariant Sections with translations requires special permission
from their copyright holders, but you may include translations of
some or all Invariant Sections in addition to the original versions
of these Invariant Sections. You may include a translation of this
License, and all the license notices in the Document, and any Warranty
Disclaimers, provided that you also include the original English version
of this License and the original versions of those notices and disclaimers.
In case of a disagreement between the translation and the original
version of this License or a notice or disclaimer, the original version
will prevail.}{\tiny\par}

{\tiny{}If a section in the Document is Entitled \textquotedblleft Acknowledgements\textquotedblright ,
\textquotedblleft Dedications\textquotedblright , or \textquotedblleft History\textquotedblright ,
the requirement (section~\ref{subsec:4Modifications}) to Preserve
its Title (section~\ref{subsec:1Applicability-and-definitions})
will typically require changing the actual title.}{\tiny\par}

\subsection*{{\tiny{}Termination}}

{\tiny{}You may not copy, modify, sublicense, or distribute the Document
except as expressly provided for under this License. Any other attempt
to copy, modify, sublicense or distribute the Document is void, and
will automatically terminate your rights under this License. However,
parties who have received copies, or rights, from you under this License
will not have their licenses terminated so long as such parties remain
in full compliance.}{\tiny\par}

\subsection*{{\tiny{}Future revisions of this license}}

{\tiny{}The Free Software Foundation may publish new, revised versions
of the GNU Free Documentation License from time to time. Such new
versions will be similar in spirit to the present version, but may
differ in detail to address new problems or concerns. See \url{http://www.gnu.org/copyleft/}.}{\tiny\par}

{\tiny{}Each version of the License is given a distinguishing version
number. If the Document specifies that a particular numbered version
of this License \textquotedblleft or any later version\textquotedblright{}
applies to it, you have the option of following the terms and conditions
either of that specified version or of any later version that has
been published (not as a draft) by the Free Software Foundation. If
the Document does not specify a version number of this License, you
may choose any version ever published (not as a draft) by the Free
Software Foundation.}{\tiny\par}

\subsection*{\noun{\tiny{}Addendum}{\tiny{}: How to use this License for your
documents}}

{\tiny{}To use this License in a document you have written, include
a copy of the License in the document and put the following copyright
and license notices just after the title page:}{\tiny\par}

{\tiny{}Copyright (c) <year> <your name>. Permission is granted to
copy, distribute and/or modify this document under the terms of the
GNU Free Documentation License, Version 1.2 or any later version published
by the Free Software Foundation; with no Invariant Sections, no Front-Cover
Texts, and no Back-Cover Texts. A copy of the license is included
in the section entitled \textquotedblleft GNU Free Documentation License\textquotedblright .}{\tiny\par}

{\tiny{}If you have Invariant Sections, Front-Cover Texts and Back-Cover
Texts, replace the \textquotedblleft with...Texts.\textquotedblright{}
line with this:}{\tiny\par}

{\tiny{}with the Invariant Sections being <list their titles>, with
the Front-Cover Texts being <list>, and with the Back-Cover Texts
being <list>.}{\tiny\par}

{\tiny{}If you have Invariant Sections without Cover Texts, or some
other combination of the three, merge those two alternatives to suit
the situation.}{\tiny\par}

{\tiny{}If your document contains nontrivial examples of program code,
we recommend releasing these examples in parallel under your choice
of free software license, such as the GNU General Public License,
to permit their use in free software.}{\tiny\par}

\subsection*{{\tiny{}Copyright }}

{\tiny{}Copyright (c) 2000, 2001, 2002 Free Software Foundation, Inc.
59 Temple Place, Suite 330, Boston, MA 02111-1307, USA}{\tiny\par}

{\tiny{}Everyone is permitted to copy and distribute verbatim copies
of this license document, but changing it is not allowed.}{\tiny\par}


\listoftables

\listoffigures

\printindex{}

\input{sofp-back-cover-page}
\end{document}
